\section{№5 Гомоморфизм абелевых групп}

\begin{reminder}
  \[\phi : G \ra H \quad \im \phi \cong G / \ke \phi\]
\end{reminder}

\begin{example}
  А --- абелева группа конечно порожденная
  \[A = \langle g_1, \dots , g_2 \rangle \quad A = \{g_1^{x_1} \ra g_n^{x_n}; x \in \ZZ\}\]
\end{example}

\begin{proposition}
  \[\phi : \ZZ^n \ra A \quad \phi: (x_1, \dots, x_n) \raone g_1^{x_1}, \dots g_n^{x_n}\]
\end{proposition}

\begin{theorem}
  \[A \cong \ZZ^n / \ke \phi\]
  где $\ZZ^n = \{(x_1, \dots, x_n) \mid x_i \in \ZZ\}, \quad (x + y)_i = x_i + y_i$
\end{theorem}

\begin{theorem}
  $G < \ZZ^n$, то $F \cong \ZZ^m, \quad 0 \leq m \leq n$
\end{theorem}

\begin{example}
  % \[u_1 = (u_1, u_2, \dots u_n)\] -- сбоку
  \[
  \begin{matrix}

  \begin{matrix}
    u_1 \\
    u_2 \\
    \vdots \\
    u_m
  \end{matrix}
  &
  \begin{pmatrix}
    u_{11} & u_{12} & \dots & u_{1n} \\
    u_{21} & u_{22} & \dots & u_{2n} \\
    \vdots \\
    u_{m1} & u_{m2} & \dots & u_{mn} \\
  \end{pmatrix}
  = M
  \end{matrix}
  \]
\end{example}

\begin{example}
  A = $\langle a, b \rangle \  ab^5 = a^5 b = e$
  \[M = 
  \begin{pmatrix}
    1 & 5 \\
    5 & 1
  \end{pmatrix}
  \]
  Тогда, $\ZZ^2 / R(M) \cong C_{24}$
\end{example}

\begin{proposition}
  \[M = 
  \begin{pmatrix}
    d_1 & 0 & 0 & \dots & 0 \\
    0 & d_2 & 0 & \dots & 0 \\
    \vdots \\
    0 & 0 & d_m & \dots & 0 \\
  \end{pmatrix}
  \]

\end{proposition}
Тогда, $R(M) = (d_1 y_1, d_2 y_2, \dots d_m y_m, 0 \dots 0), \quad y_i \in \ZZ$.
А именно, $\ZZ^n / R(M) \cong Z_{d_1} \times Z_{d_2} \times \dots \times Z_{d_i} \times Z^{n - m} = G$
\begin{example}
  Тогда рассмотрим гомоморфизм:
  \[\phi : \ZZ^n \ra G \quad \ke \phi = R(M)\]
  \[(x_1, \dots, x_n) \raone (x_1 \mod d_1, \dots, x_i \mod d_i, x_{m+i}, \dots, x_n)\]

\end{example}

\begin{theorem}
  Конечная порожденная абелева группа изморофна прямому произведению изоморфизмов (?\footnote{Здесь и далее знаки вопроса в скобках означают, что я не уверен в каком-либо утверждении на все сто, поэтому его стоит проверить.})
\end{theorem}

\begin{corollary}
  Элементарные проебразования строк и столбцов 
  \begin{itemize}
    \item перестановка строк (или столбцов)
    \item умножение строки на $-1$ 
    \item сложение строки и целового кратного дургой строки (аналогично со столбцом)
    \item отбрасывание первой строки
  \end{itemize}
\end{corollary}

\begin{lemma}
  Элементарные проебразования сохраняют факторгруппу: $\ZZ^n / R(M)$
\end{lemma}

\begin{example}
  \[
  \begin{pmatrix}
    1 & 5 \\
    5 & 1
  \end{pmatrix}
  \sim
  \begin{pmatrix}
    1 & 5 \\
    0 & -24
  \end{pmatrix}
  \sim
  \begin{pmatrix}
    1 & 0 \\
    0 & +24
  \end{pmatrix}
  \]
  Тогда: $A \cong \ZZ_1 \times \ZZ_{24} \cong \ZZ_{24}$
\end{example}

\begin{theorem}[Нормальная форма Смита]
  \[
  M \sim
  \begin{pmatrix}
    d_1 & 0 & 0 \\
    \vdots & \ddots \\
    0 & d_n & 0

  \end{pmatrix}
  \quad
  d_i \mid \d{i + 1}
  \]
\end{theorem}

\begin{example}
  Рассмотрим группу $Z_8 \times Z_{16}: (Тут сделать две строчки) \quad a = \langle 2, 0 \rangle \quad \ord a = 4$

  $\langle a  \rangle \cong \langle b \rangle, \quad a = \langle 4, 4 \rangle \quad \ord a = 4$ 

  $Z_8 \times Z_{16} / \langle a \rangle \not \cong Z_8 \times Z_{16} / \langle b \rangle$

  % \[
  %   M_1 = 
  %   \begin{pmatrix}
  %     8 & 0  \\
  %     0 & 16 \\
  %     2 & 0
  %   \end{pmatrix}
    
  %   \quad 
  %   M_2 = 
  %   \begin{pmatrix}
  %   8 & 0  \\
  %   0 & 16 \\
  %   4 & 4
  %   \end{pmatrix}

  % \]

  % \[

  \[
  M_1 = 
  \begin{pmatrix}
    8 & 0 \\
    0 & 16 \\
    2 & 0
  \end{pmatrix}
  \quad
  M_2 = 
  \begin{pmatrix}
    8 & 0 \\
    0 & 16 \\
    4 & 4
  \end{pmatrix}
  \]

  \[\ZZ^2 / R(M_1) \cong G_1 \cong Z_2 \times Z_{16} \quad \ZZ^2/R(M) \cong G_2 \cong Z_4 \times Z_8\]

\end{example}

\subsection{Гомоморфизм не абелевых групп}

\begin{example}[Не абелва группа]
  \[\phi: S_m \ra Z_2\]
  Где $s(\pi)$ = \# циклов в $\pi$.
  \[\phi: \pi \raone (n + s(\pi)) \mod 2\]

  То есть четные перестановки $A_n = \ke \phi$
\end{example}

\begin{reminder}
  $(ij)$ --- транспозиция.
\end{reminder}

\begin{lemma}
  Каждая перестановка равна произведению транспозиций.
\end{lemma}

\begin{example}
  \[(a_1, \dots, a_l) = (a_1, a_l) (a_1, a_{l-1}, \dots, (a_1, a_2))\]
  \[a_i \raone \ra a_i \ra a_1 \ra a_{i+1} \ra \dots \ra a_{i +1}\]
\end{example}

\begin{lemma}
  Умножение на транпосзицию изменяет коливчество циклов на $\pm$ 1.
\end{lemma}

\begin{proposition}
  Элемент называется сорпяженном к $g \in G$, если выполнено:
  \[x ^ g \defeq g ^{-1} x g\]
  \[x \raone x^g\], то есть изоморфизм $G \ra G$ (автоморфизм)
  
\end{proposition}

\begin{proof}
  ~
  \begin{itemize}
    \item Биективность: $y = g ^ {-1} x g, \quad x = g(g ^ {-1} x g) g^{-1} = x$
    \item Сохранение операции: $g ^{-1} (xy) g = \underbrace{(g ^{-1} xg) (g^{-1} y g)}_{l}$
  \end{itemize}
\end{proof}

\begin{proposition}
  \[x \sim y \Lra \exists g \in G \ y = g^{-1} x g\]
\end{proposition}

\begin{proof}
  ~
  \begin{itemize}
    \item $x \sim x \Ra x = e ^{-1} x e$
    \item $x \sim y \Ra y \sim x$
    \item $x \sim y, y \sim z \Ra x \sim z$ 
  \end{itemize}
  То есть, образуется класс эквивалентности.
\end{proof}

\subsection{Критерий сопряжения перестановок}

\begin{example}
  \[\sigma^{\prime} = \pi^{\prime} \sigma \pi\]
  Цикловой тип $l(\pi) = (l_1 \geq l_2 \geq \dots \geq l_5)$
  
  $S_3: \pi = (12) \dots$
  % тут я очень много не успел написать
\end{example}
\begin{center}
  \texttt{тут началась какая-то жесть, я потерялся...}
\end{center}
\begin{theorem}
  $\sigma^{\prime} \sim \sigma \Lra l(\sigma^{\prime}) = l(\sigma)$
\end{theorem}

\begin{proof}
  \[l(\sigma^{\prime}) = l(\sigma)\]
  \[\sigma = (a_1^1, \dots, a_{l_1}^1)(a_1^{2}, \dots a_{l_2}^2)\]
  \[\pi: \sigma^{\prime}  = (b_1^{1}, \dots , b_{l_1}^1)(b_1^2, \dots, b_{l_2}^2 )\]

  \[\pi: \raone a_j^i \xmapsto{\sigma} a_{j + 1}^i 
  \xmapsto{\pi} b_{j+1}^{i}\]
\end{proof}

