\section{№11 Теорема о максимальном идеале и евклидовы кольца}

\begin{reminder}
  $\FF_p \subset F, \quad |F| < \inf, \quad p = \charf F = \langle 1 \rangle _{+}$
\end{reminder}

\begin{reminder}
  $\phi: \FF_p[x] \ra F, \quad \im \phi = F, \quad \ke \phi = I$
\end{reminder}

\begin{reminder}
  $F \cong \FF_p[x]/I$ --- Все идеалы главные
\end{reminder}

\begin{theorem}[Теорема о максимальном идеале]
  $R/I$ --- полный $\Lra$ $I$ --- максимальный по включению.
  \[I = (a) = \{ qa : q \in R \}\]
\end{theorem}

\begin{proof}
  \begin{gather}
    (a) \subseteq (b), \quad q', q'' \in \RR^{*} \text{обратимые} \\
    a = qb \Lra b \mid a, \quad \RR^{*} \text{--- группа по умножению} \\
    a \neq 0, \quad (a) = (b) \\
    a = q'b, \quad a = q' q'' q \\
    b = q''a, \quad a(1 - q'q'') = 0 \\
    a \neq 0, \quad 1 = q' q''
  \end{gather}
\end{proof}

\begin{definition}
  $a \sim b \Lra a = \epsilon b, \  \epsilon \in \RR^{*}$
\end{definition}

\begin{corollary}
  $(a)$ --- максимальный $\Lra$ $a$ --- неразложим, $a \notin \RR^{*}$. То есть:
  \[a = bc \Ra b \in \RR^{*} \lor c \in \RR^{*}\]
\end{corollary}

\begin{proposition}~ \\
  $a$ --- неразложим $\Ra$ простые в $\ZZ$ \\
  $a$ --- неразложим $\Ra$ неприводим в $\FF[x]$
\end{proposition}

\begin{example}
  $a = \epsilon (\epsilon ^{-1} a)$
\end{example}

\subsection{Основная теорема арифметики для евклидовых колец}

\begin{proposition}
  Пусть $\RR$ --- евклидово кольцо. Тогда $\forall a \neq 0, \quad a = \epsilon p_1 \dots p_n = \delta q_1 \dots q_s$. Тогда отсюда следует:
  \begin{enumerate}
    \item Существует $a = q_1 \dots q_t, \quad q_i \text{ --- неразличимые}$
    \item Единственность
    \[s = n, \quad \exists \pi: [n] \ra [n]\]
    \[q_i = \epsilon_i p_{\pi(j)}\]
  \end{enumerate}
\end{proposition}

\begin{example}
  $15 = (-3) \cdot (-5) = 5 \cdot 3$
\end{example}

\begin{lemma}[1]
  $a = bc, \quad b, c \notin \RR^{*} \cup \{0\}$
\end{lemma}

\begin{proof}
  \begin{gather}
    N(a) > N(b), \quad N(a) > N(c) \\
    b = qa + r  = qbc + r \\
    b(1 - qc) = r \Ra r \neq 0 \\
    N(b) \leq N(r) < N(a)
  \end{gather}
\end{proof}

\begin{proof}[Доказательство существования]  
  База индукции: $n_0 = \min N(x), \quad x \in \RR \setminus 0$.
  \[N(a) = n_0 \Ra a \in \RR^{*}\]
  \[1 = qa + r \Ra r = 0\]
  \[r \neq 0 \Ra N(r) < N(a)\]
  Шаг индукции. Пусть доказано для $N(a) < m \Ra N(a) = m$.
  \begin{enumerate}
    \item $a$ --- неразложим
    \item $a = bc, \quad b, c \notin \RR^{*}$
    \[N(b) < N(a)\]
    \[N(c) < N(a)\]

    \[b = \epsilon p_1 \dots p_n, \quad c = \delta q_1 \dots q_s\]
    \[a = (\epsilon \delta) p_1 \dots p_n q_1 \dots q_s\]
  \end{enumerate}
\end{proof}

\begin{lemma}[2]
  $a$ --- неразложим: 
\end{lemma}

\begin{proof}
  \begin{gather}
    a = xy \Ra a \mid x \lor a \mid y \\
    R / (a) \text{ --- поле} \\
    [0] = [a] = [x] \cdot [y] \Ra [x] = [0] \Ra x = q'a \lor [y] = [0] \Ra y = q''a
  \end{gather}
\end{proof}

\begin{proof}[Доказательство единственности]
  \begin{gather}
    a = \epsilon p_1 \dots p_n = \delta q_1 \dots q_s \\
    R/(a), \quad [0] = [\delta] [q_1] \dots [q_s], \quad \delta \neq 0 \\
    \exists i: \quad  q_i = u \cdot p_1
  \end{gather}
\end{proof}

\subsection{Наибольший общий делитель в евклидовом кольце}

\begin{proposition}
  \[(d) = (a, b) = \{ xa + yb: \quad x,y \in R\}\]
  Но этот идеал является главным, то есть:
  \[d \defev \gcdru(a, b)\]
\end{proposition}

\begin{corollary}
  Наибольший общий делитель является линейной комбинацией: 
  \[ d = \tilde a a + \tilde b b\]
\end{corollary}

\begin{corollary}
  $b, c$ --- взаимно простые $\defev$ $(b, c) = (1)$
\end{corollary}

\subsection{Китайская теорема об остатках для евклидовых колец}

\begin{definition}
  $a = bc, \quad (b, c) = (1)$. Тогда: 
  \[R/(a) \cong R/(b) \oplus  R/(c)\]
\end{definition}

\begin{definition}
  Покоординатная операция.
  \[R_1 \oplus  R_2 = \{ (r_1, r_2): \quad r_1 \in R_1, \ r_2 \in R_2 \}\]
\end{definition}

\begin{proof}
  \begin{note}[Спойлер на доказательство]
    \begin{gather}
      \ke \phi = (bc) = (a) \\
      \im \phi = R/ (b) \oplus R/(c) \\
      \text{ТМС: } \ R/\ke \phi \cong \im \phi
    \end{gather}
  \end{note}

  \begin{gather}
    \phi: R \ra R/(a) \oplus R/(c) \\
    r \raone ({[r]}_b, {[r]}_c)
  \end{gather}

  Сюръективность:
  \[v + q_2 c = w = u + q_1 b\]
  \[u - v = -q_1 b + q_2 c\]
  \[1 = \tilde b b + \tilde c c \Ra\]
  \[q_1 = -(u - v) \tilde b\]
  \[q_2 = (u - v) \tilde c\]

  Пусть $k \in \ke \phi$. Тогда $k \equiv 0 \mod b$ и $k \equiv 0 \mod c$. То есть $k \in (bc)$.
  \[k = q_1 b = q_2 c\]
  \[q_1 (b \tilde b) = q_2 c \tilde b\]
  \[q_1 (1 - \tilde c c) = q_2 c \tilde b\] 
  \[q_1 = c(q_1 \tilde c + q_2 \tilde b)\]
\end{proof}

\subsection{Алгоритм Евклида}

\begin{note}
  Кольцо в котором работает алгоритм Евклида --- евклидово кольцо.
\end{note}

\begin{proposition}
  $(a, b) = (a, b + qa)$
\end{proposition}

\begin{proof}
  $b = (b + qa) - qa$
\end{proof}

\begin{example}[Линейное диафантово уравнение]
  $xa + yb = d$
\end{example}

\begin{solution}
  \begin{gather}
    a_i, q_i, x_i, y_i \\
    a_0 = 1 \cdot a = 0 \cdot b \quad x_0 = 1, y_0 = 0 \\
    a_1 = 0 \cdot a + 1 \cdot b \quad x_1 = 0, y_1 = 1 \\
  \end{gather}
  То есть делим с остатком:
  \begin{gather}
    a_i = q_{i + 1} a_{i + 1} + a_{i + 2} \\
    a_{i + 2} = q_i - q_{i + 1}a_{i + 1} \\
    x_{i + 2} = x_i - q_{i + 1} x_{i + 1} \\
    y_{i + 2} = y_i - q_{i + 1} y_{i + 1}
  \end{gather}

  \[a_{t + 1} = 0\]
  \[a_{t - 1} = q_t a_t\]
  \[a_t = \gcdru(a, b)\]

  То есть в итоге мы имеем:
  \[(a, b) = (a_0, a_1) = (a_1, a_2) = \dots = (q_t a_t, a_t)\]
\end{solution}

\begin{proposition}
  Инвариант:
  \[\forall i: \quad a_i = x_i a + y_i b\]
\end{proposition}

\begin{proof}
  \begin{gather}
    a_i = x_i a + y_i b \\
    a_{i + 1} = x_{i + 1}a + y_{i + 1} b \\\
  \end{gather}
  \begin{multline}
    a _{i + 2} = a_i - q_{i + 1} a_{i + 1} = x_i a + y_i b - q_{i + 1}(x_{i + 1} a + y_{i + 1}b) =\\
    = (x_i - q_{i + 1}x_{i + 1})a + (y_i - q_{i + 1}y_{i + 1})b = x_{i +  2}a + y_{i + 2}b
  \end{multline}
\end{proof}