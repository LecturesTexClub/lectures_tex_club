\section{№8 Кольца}

\subsection{Лемма Бернсайда}

\begin{lemma}
  Число орбит равно средней мощности стабилизатора элементов группы $G$.
  \[|X/G| = \frac{1}{G} \sum_{g \in G} |\stab g|\]
\end{lemma}

\begin{definition}
  Кольцо~--- множество с \underline{двумя} операциями: $(R, \cdot, +)$, такое что:
  \begin{enumerate}
    \item Является абелевой группой по сложению (0~--- нейтральный).
    \item Является ассоциативной по умножению (1~--- нейтральный).
    \item Двусторонняя дистрибутивность: 
    \begin{gather}
      a \cd (b + c) = a\cd + a \cd c \\
      (b + c) \cd a = b \cd a + c \cd a
    \end{gather}
  \end{enumerate}
\end{definition}

\begin{example}
  $\ZZ$, $\QQ$, $\RR$, $\Cm$.


  $2 \ZZ$~--- кольцо без $1$.

  % $M_n(\RR)$
  $\ZZ / n \ZZ$~--- вычеты по модулю $n$.
\end{example}

\begin{proposition}
  $a(b - c) = ab - ac, \quad (b-c)a = ba - ca$
\end{proposition}

\begin{proof}
  \begin{gather}
    a(b - c) + ac = a(b - c + c) = ab \\
    \Ra a(b - c) = ab - ac
  \end{gather}
\end{proof}

\begin{proposition}
  $0 \cd a = a \cd 0 = 0$.
\end{proposition}

\begin{proof}
  \[0 \cd a = (x - x) \cd a = xa - xa = 0\]
\end{proof}

\begin{proposition}
  Если в $\RR$ есть $1$, то $(-1) \cd (-1) = 1$.
\end{proposition}

\begin{proof}
  \[0 = (1 + (- 1)) \cd (-1) = 1 \cd (-1) + (-1) (-1) = (-1) + (-1) (-1)\]
  \[\Ra 1 = (-1) \cd (-1)\]
\end{proof}

\begin{note}
  Если по умножению группа, то:
  \[ 0 = 0^{-1} \cd 0 = 1 \Ra 0 = 1\]
  , то есть получаем, что: 
  \[\forall a: \ a = a \cd 1 = a \cd 0 = 0 \Ra |R| = 1\]  
\end{note}
В итоге, в кольце нельзя делить на ноль, тогда и только тогда, когда кольцо состоит из одного элемента (нуля). 

\begin{definition}
  Поле~--- множество вида: $(F, \cd, +)$, такое что:
  \begin{itemize}
    \item[$F1$]: Коммутативное кольцо с $1$ ($a \cd b = b \cd a$).
    \item[$F2$]: $0 \neq 1$.
    \item[$F3$]: $F^{*} = F \setminus  \{ 0 \}$~--- группа по умножению
  \end{itemize}
\end{definition}

\begin{example}
  $\FF = \ZZ / p \ZZ$~--- $p$ простое.
\end{example}

\begin{proposition}
  Если $n$~--- состовное, то $\ZZ / n \ZZ$~--- не поле.
\end{proposition}

\begin{proof}
  \begin{gather}
    [a]_n \cd [b]_n = [0]_n \\
    [a^{-1}]_n [a]_n [b]_n = [b]_n = [0]_n \\
    a^{-1} \cd a b = 0 \Ra b = 0
  \end{gather}
\end{proof}

\begin{definition}
  Делитель нуля:
  \[a \neq 0, \ b \neq 0, \ ab = 0 \]
\end{definition}

\begin{corollary}
  В кольцах есть делителя нуля, в полях нет.
\end{corollary}

\begin{example}
  Простые поля~--- $\FF_p, \ \QQ$.
\end{example}

\begin{lemma}
  В каждом поле есть простое подполе.
\end{lemma}

\begin{proof}
  Пусть $\{ n \cd 1 \} \subset F$~--- циклическая группа по сложению с образующей 1. 
  \[\{n, 1\} < \infty \Ra \deg F = p\]
  где $p$~--- простое. Тогда, кратные 1 - поле, то есть $\FF_p$.

  (Если $\{n \cd 1\} = + \infty$, то $\FF_p = \QQ$).
\end{proof}