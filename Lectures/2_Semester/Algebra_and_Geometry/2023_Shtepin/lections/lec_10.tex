\subsubsection{Закон инерции для квадратичной формы. Теоремы Якоби и Сильвестра}

\begin{definition} 
    Рассмотрим квадратичную форму $q(x)$. Будем называть $q(x)$ положительно (отрицательно) 
    определенной, если для всех $x \neq 0$ верно $q(x) > 0$ ($q(x) < 0$). В случае нестрогих неравенств 
    будем называть $q(x)$ положительно (отрицательно) полуопределенной.
\end{definition}

\begin{note}
    В случае если существуют $x_1$ и $x_2$ такие, что $q(x_1) > 0$ и $q(x_2) < 0$, $q(x)$ не определена.
\end{note}

\begin{agreement}
    До конца раздела будем считать что $V$ -- поле над пространством действительных чисел.
\end{agreement}

\begin{proposition}~
    \label{pr10.1}
    \begin{enumerate}
        \item Функция $q(x)$ положительно определена тогда и только тогда когда приводится к каноническому
        виду с матрицей $E$.
        \item Функция $q(x)$ положительно полуопределена тогда и только тогда когда приводится к 
        каноническому виду с матрицей, не имеющей $-1$ на главной диагонали.
    \end{enumerate}
\end{proposition}

\begin{proof}~
    \begin{enumerate}
        \item \begin{enumerate}
            \item Необходимость. 
            
            Пусть $q(x)$ положительно определена. Рассмотрим канонический базис $e$.
            В этом базисе $i$-й элемент матрицы $q$ равен $a_{ii} = q(e_i) > 0$. 
            
            В силу того, что в каноническом 
            базисе матрица может иметь только значения $\pm 1$ и $0$ на главной диагонали, получаем $a_{ii} = 1$.
            Таким образом матрица формы $q$ является единичной.
            \item Достаточность. 
            
            Пусть $q$ приводится к каноническому виду с $E$. Тогда в каноническом базисе: 
            $$q(x) = \xi_1^2 + \, \dots \,+ \xi_n^2, \, \text{ где } \, n = \dim V.$$ 
            Это значит, что для всех $x \neq 0$ верно $q(x) > 0$, так как в каноническом базисе 
            $x$ представляется в виде $x = (\xi_1 \, \xi_2 \, \dots \, \xi_n)^T$. Таким образом 
            $q$ положительно определена.
        \end{enumerate} 

        \item \begin{enumerate}
            \item Необходимость.
            
            Пусть $q(x)$ положительно полуопределена. Тогда в каноническом базисе $i$-й элемент 
            главной диагонали матрицы $q$ равен $a_{ii} = q(x_i) \geq 0$, 
            откуда $a_{ii} \in \{0, 1\}$.

            \item Достаточность.
            
            Пусть в каноническом базисе $a_{ii} \in \{ 0, 1\}$. Тогда $q$ в нем имеет вид:
            $$q(x) = \xi_1^2 + \, \dots \,+ \xi_p^2, \, \text{ где } \, p < \dim V.$$
            Таким образом для всех $x$ верно $q(x) \geq 0$, что значит, что $q$ положительно
            полуопределена.
        \end{enumerate}
    \end{enumerate}
\end{proof}

\begin{exercise}
    Переформулировать утверждение \ref{pr10.1} для отрицательных определенных и полуопределенных функций.
\end{exercise}

\begin{definition}
    Пусть $e$ -- канонический базис. Представим $q(x)$ как: 
    $$q(x) = \xi_1^2 + \, \dots \,+ \xi_p^2 - \xi^2_{p+1} - \, \dots \,- \xi^2_{p+q}.$$ 
    Числа $p$ и $q$ называются индексами инерции относительно канонического базиса $e$.
\end{definition}

\begin{note}
    В любом базисе верно $p + q = \rk V$. Далее будет показано, что значения $p$ и $q$ не зависят 
    от выбора базиса.
\end{note}

\begin{theorem}[Закон инерции]
    Пусть $q \in Q(V)$, $e$ -- канонический базис в $V$, $p$ и $q$ -- положительный и отрицательный 
    индексы инерции относительно базиса $e$. Тогда верно следующее: 
    \begin{enumerate}
        \item $p = \max\{\dim U \, \vert \, U \leq V : q \vert_{U} \text{ -- положительно определена}\}$,
        \item $q = \max\{\dim U \, \vert \, U \leq V : q \vert_{U} \text{ -- отрицательно определена}\}$,
        \item Индексы $p$ и $q$ не зависят от выбора базиса в $V$.
    \end{enumerate}
\end{theorem}

\begin{proof}~
    \begin{enumerate}
        \item Пусть $e = (e_1, e_2, \, \dots \,e_n)$. Рассмотрим следующие подпространства V:
        \begin{align*}
            U_0 = \langle e_1, e_2, \, \dots \,e_p \rangle && W_0 = \langle e_{p+1}, e_{p+2}, \, \dots \,e_n\rangle. 
        \end{align*}
        Их размерности равны $\dim U_0 = p$ и $\dim W_0 = n-p$ соответственно.

        Пусть $m = \max \{ \dim U \vert U \leq V: q\vert_U \text{ -- положительно определена}\}$.

        По построению $U_0$ верно что $q \vert_{U_0}$ положительно определена, а значит $m \geq p$. Пусть $m > p$.
        Тогда по построению $m$ существует $U_1  \leq V$ такое, что $q \vert_{U_1}$ положительно определена
        и $\dim U_1 = m$.
        При этом по формуле Грассмана: 
        $$\dim (U \cap W) = \dim U + \dim W_0  - \dim (U + W_0) = m + n - p - \dim (U + W_0) \geq 
        m + n - p - n > 0.$$
        Тогда $\exists z \in U_1 \cup W_0$. Однако по построению этих подпространств получим:
        \begin{align*}
            z \in U_1 \, & \Rightarrow q(z) > 0, \\
            z \in W_0 & \Rightarrow q(z) \leq 0. \\
        \end{align*}
        Таким образом предположение $m > p$ приводит к противоречию из-за невозможности существования 
        нетривиального пересечения $U_1$ и $W_0$. Это значит, что $m = p$. 

        \item Доказательство аналогично первому пункту.
        \item Истинность утверждения вытекает из первых двух пунктов, так как размерность подпространств 
        не зависит от выбора базисов в них.
    \end{enumerate}
\end{proof}

\begin{definition}
    Пусть квадратичная билинейная форма $q$ представляется как: \begin{gather*}
        q \leftrightarrow \begin{pmatrix}
        a_{11} & a_{12} & \dots  & a_{1n}   \\
        a_{21} & a_{22} & \dots  & a_{2n}   \\
        \vdots & \vdots & \ddots & \vdots   \\
        a_{n1} & a_{n2} & \dots  & a_{nn}
        \end{pmatrix}
    \end{gather*}
    Главным минором $\Delta_i$ называется определитель левой верхней подматрицы размера $i \times i$:
    \begin{gather*}
        \Delta_i = \begin{pmatrix}
            a_{11} & a_{12} & \dots  & a_{1i}   \\
            a_{21} & a_{22} & \dots  & a_{2i}   \\
            \vdots & \vdots & \ddots & \vdots   \\
            a_{i1} & a_{i2} & \dots  & a_{ii}
        \end{pmatrix}
    \end{gather*} 
\end{definition}

\begin{theorem}[Як\'{о}би]
    Пусть $q(x)$ -- квадратичная функция в линейном пространстве над $\R$, $A$ -- её матрица относительно 
    некоторого базиса $e$ в $V$ и пусть $\forall i \: = 1, \dots n$ верно $\Delta_i \neq 0$. Тогда 
    существует базис $e'$ в $V$ такой что в нем $q(x)$ принимает вид: 
    \begin{gather*}
        q(x) = \frac{\Delta_0}{\Delta_1} \xi_1^2 + \frac{\Delta_1}{\Delta_2} \xi_2^2 + \, \dots \,+ 
        \frac{\Delta_{n-1}}{\Delta_n} \xi_n^2, \: \text{где } \, \Delta_0 = 1. 
    \end{gather*}
    Более того, $e'$ можно выбрать так, что матрица перехода $S = S_{e \to e'}$ 
    является верхнетреугольной. 
\end{theorem}

\begin{proof}
    Индукция по $n$ -- размерности пространства $V$:
    \begin{enumerate}
        \item База $n = 1$:  
        
        В пространстве размерности $1$ форма принимает вид $q(x) = a_{11} x_1^2$. 

        Тогда можно осуществить переход $e_1 \to e'_1 = \frac{1}{a_{11}} e_1$. Для нового базисного вектора: 
        $$q(e'_1) = f(\frac{e_1}{a_{11}}, \frac{e_1}{a_{11}}) = \frac{1}{a_{11}^2} a_{11} = \frac{1}{a_{11}}.$$
        Тогда в новом базисе $q(x) = a_{11} \xi_1 = \frac{1}{\Delta_1} \xi_1^2$, что и требовалось.

        \item Пусть теорема справедлива для любого $V$ для которого верно $\dim V < n$. 
        
        Рассмотрим пространство $V$ размерности $n$, 
        и его подпространство $U = \langle e_1, e_2, \, \dots \,e_{n-1} \rangle$. 

        По предположению индукции существует базис $e' = \langle e'_1, e'_2, \dots e'_{n-1} \rangle$ в $U$
        такой что $q$ имеет вид:
        $$q(x)\vert_U = \frac{\Delta_0}{\Delta_1} \xi_1^2 + \frac{\Delta_1}{\Delta_2} \xi_2^2 + \dots + 
        \frac{\Delta_{n-2}}{\Delta_{n-1}} \xi_{n-1}^2,$$ и матрица перехода от него к нашему базису 
        имеет верхнетреугольный вид:
        \begin{gather*}
            S_{e \to e'} = \begin{pmatrix}
            S_{11} & S_{12} & \dots  & S_{1, n-1} \\
            0      & S_{22} & \dots  & S_{2, n-1} \\
            \vdots & \vdots & \ddots & \vdots     \\
            0      & 0      & \dots  & S_{n-1, n-1}
            \end{pmatrix}
        \end{gather*}
        При этом форма $q(x)\vert_{U}$ невырождена, так как $\Delta_{n-1} \neq 0$. 

        Тогда по теореме о невырожденном подпространстве $V = U \oplus U^{\perp}$, 
        где ортогональное дополнение $U^{\perp}$ используется в смысле $f$ ассоциированного с $q$, 
        $\dim U^{\perp} = 1$. 
        
        Заметим, что в $U^{\perp}$ есть ненулевой вектор $e$, для которого верно $f(e_n, e) \neq 0$. 
        
        В противном случае для любого вектора $e \in U^{\perp}$ верно $f(e_n, e) = 0$, что значит, 
        что все вектора $e \in U^{\perp}$ перпендикулярны $e_n$. При этом $e \perp U = \langle e_1, \dots e_n\rangle$, 
        откуда $e \in \ker f$. 
        
        Это противоречит тому, что $\dim (\ker f) = \dim V - \rk f = 0$, а значит необходимый нам вектор существует.
        
        Положим $f(e, e_n) = c \neq 0$, 
        тогда $f(e_n, \frac{e}{c}) = 1$. Пусть $e'_n = \frac{e}{c} \in U^{\perp}$, откуда $f(e_n, e'_n)  = 1.$

        Покажем, что $e' = \langle e'_1, \dots e'_{n-1}, e'_n\rangle$ -- искомый базис. 

        Рассмотрим матрицу перехода $S = S_{e \to e'}$. Заметим, что $S_{ni} = 0$ для всех $i < n$ 
        в силу того, что $e'_i \in U$, а значит при переходе к новому базису вектор $e_n$ не повлияет 
        на него. Таким образом матрица $S_{e \to e'}$ диагональна.

        Осталось показать, что в новом базисе форма $q$ имеет необходимый вид. Благодаря предположению 
        индукции мы имеем:
        $$q(x)\vert_U = \frac{\Delta_0}{\Delta_1} \xi_1^2 + \frac{\Delta_1}{\Delta_2} \xi_2^2 + \dots + 
        \frac{\Delta_{n-2}}{\Delta_{n-1}} \xi_{n-1}^2.$$ 

        Таким образом необходимо только показать, что коэффициент при $\xi_n$ равен 
        $\frac{\Delta_{n-1}}{\Delta_n}$. Заметим, что этот коэффициент равен $q(e'_n)$.

        Вектор $e'_n$ выражается через коэффициенты матрицы перехода и векторы начального базиса: 
        $$e'_n = S_{1n}e_1 + \dots S_{nn}e_n.$$ Тогда:
        \begin{equation*}
            \begin{cases}
                f(e_1, e'_n) = 0,         \\
                f(e_2, e'_n) = 0,         \\
                \dots                     \\
                f(e_{n-1}, e'_n) = 0,     \\
                f(e_n, e'_n) = 1.
            \end{cases}
        \end{equation*}

        Первые $n-1$ значений равны $0$ в силу того, что $e'_n \in U^{\perp}$, $e_i \in U$.

        Тогда $q(e'_n)$ можно выразить следующим образом: \begin{gather*}
            q(e'_n) = f(e'_n, e'_n) = f(S_{1n} e_1 + \, \dots \, + S_{n-1, n} e_{n-1} + S_{nn}e_n, e_n') = \\
            = S_{1n} \cdot f(e_1, e'_1) + \, \dots \, + S_{nn} \cdot f(e_n, e'_n) = S_{nn}. 
        \end{gather*}  
        
        Выразим $S_{nn}$ из системы выше:

        \begin{equation*}
            \begin{cases}
                f(e_1, e'_n) = f(e_1, S_{1n}e_1 + \dots S_{nn}e_n) = 0,         \\
                f(e_2, e'_n) = f(e_2, S_{1n}e_1 + \dots S_{nn}e_n) = 0,         \\
                \dots                                                           \\
                f(e_{n-1}, e'_n) = f(e_{n-1}, S_{1n}e_1 + \dots S_{nn}e_n) = 0, \\
                f(e_n, e'_n) = f(e_1, S_{1n}e_1 + \dots S_{nn}e_n) = 1.
            \end{cases}
        \end{equation*}

        В силу невырожденности $q$ матрица перехода невырождена, а значит и система уравнений невырождена,
        так как её матрица в точности является матрицей оператора $q$ в базисе $e$.
        
        Тогда по теореме Крамера для неё существует единственное решение и $S_{nn} = \frac{\Delta_{n-1}}{\Delta_n}$.

        Таким образом мы получили диагональную матрицу $S_{e \to e'}$ и необходимое нам представление 
        $q$ в базисе $e'$ для пространства размерности $V$, что завершает доказательство по индукции.
    \end{enumerate} 
\end{proof}

\begin{note}[О модификациях]~
    Можно построить матрицу $S$, являющуюся верхнетреугольной и имеющей главную диагональ из единиц. 
    Такая матрица называется унипотентной и $q(x)$ в таком случае принимает вид: 
    \begin{gather*}
        q(x) = \frac{\Delta_0}{\Delta_1} \xi_1^2 + \dots + \frac{\Delta_n}{\Delta_{n-1}} \xi_n^2.
    \end{gather*}
\end{note}

\begin{corollary}
    Пусть $q(x)$ -- квадратичная форма с ненулевыми главными минорами. Тогда отрицательный индекс 
    инертности $q$ равен числу перемен знаков в последовательности главных миноров. 
\end{corollary}

\begin{lemma}
    \label{pr10.3}
    Пусть $B \in M_n(\R)$ -- квадратная матрица над полем вещественных чисел. Тогда $B$ положительно 
    определена тогда и только тогда, когда существует невырожденная $A \in M_n(\R)$ такая, что 
    $B = A^T A$.
\end{lemma}

\begin{proof}~
    \begin{enumerate}
        \item Необходимость.
        
        Пусть $B$ положительно определена. Тогда она является матрицей некоторой 
        квадратичной функции $q$, что значит, что существует матрица $S = S_{e \to e'}$ такая, что 
        $S^T B S = E$. 
        
        Домножим выражение на $(S^T)^{-1}$ слева и на $S^{-1}$ справа и получим $B = (S^T)^{-1} S^{-1}$.

        Тогда искомая $A$ существует и равна $A = S^{-1}$. 
        \item Достаточность.
        
        Пусть $B = A^T A$, $\det A \neq 0$? тогда $B$. Возьмем матрицу перехода между базисами $S = A^{-1}$. 

        В новом базисе $B' = S^TBS = (A^{-1})^T A^T A A^{-1} = E$, откуда $B$ положительно определена по 
        утверждению \ref{pr10.1}.
    \end{enumerate}
\end{proof}

\begin{theorem}[Критерий Сильвестра]
    Пусть $q(x) = Q(V)$. Тогда верно следующее:
    \begin{enumerate}
        \item Форма $q(x)$ положительно определена тогда и только тогда когда для всех $i$ главный минор 
        положителен: $\Delta_i > 0$.
        \item Форма $q(x)$ отрицательно определена тогда и только тогда когда знаки главных миноров чередуются:
        $\sgn(\Delta_i) = (-1)^i$.
    \end{enumerate}
\end{theorem}

\begin{proof}~
    \begin{enumerate}
        \item \begin{enumerate}
            \item Необходимость.
            
            Пусть $B$ -- матрица квадратичной функции $q(x)$ и $q$ положительно определена.
            Тогда по утверждению \ref{pr10.3} верно $B = A^T A$, $\det A \neq 0$. В таком случае: 
            \begin{gather*}
                |B| = |A^T| \cdot |A| = |A|^2 > 0.
            \end{gather*}
            \item Достаточность.
            
            Пусть $\Delta_1 > 0, \dots \Delta_n > 0$. Тогда: 
            \begin{gather*}
                q(x) = \frac{\Delta_0}{\Delta_1} \xi_1^2 + \frac{\Delta_1}{\Delta_2} \xi_2^2 + \dots + 
                \frac{\Delta_{n-1}}{\Delta_n} \xi_n^2,
            \end{gather*}
            что значит, что $q(x)$ положительно определена так как при $x \neq 0$ верно $q(x) > 0$.
        \end{enumerate}
        \item Заметим, что если $q(x)$ положительно определена, то $-q(x)$ отрицательно определена. Пусть 
        $q(x)$ определена отрицательно, тогда $-q(x)$ определена положительно. Выпишем её матрицу:
        \begin{gather*}
            \begin{pmatrix}
                -a_{11} & -a_{12} & \dots  & -a_{2n} \\
                -a_{21} & -a_{22} & \dots  & -a_{2n} \\
                \vdots  & \vdots  & \ddots & \vdots  \\
                -a_{n1} & -a_{n2} & \dots  & -a_{nn}
            \end{pmatrix}
        \end{gather*}

        Тогда $\Delta_1 = -a_{11} > 0$, откуда $a_{11} < 0$. 
        
        Продолжим вычислять миноры:
        $\Delta_2 = \begin{vmatrix}
        -a_{11} & -a_{12}  \\
        -a_{21} & -a_{22}  
        \end{vmatrix} = \begin{vmatrix}
            a_{11} & a_{12}  \\
            a_{21} & a_{22}  
        \end{vmatrix} > 0$. 
        
        Вычисляя аналогично миноры большего размера получим, 
        что знак меняется на каждом шаге, что значит, что $\sgn(\Delta_i) = (-1)^i$.
    \end{enumerate}
\end{proof}

\subsection{Канонический вид кососимметричных билинейных функций}

\begin{definition}
    Базис $e = \langle e_1, \dots e_n \rangle$ называется симплектическим для билинейной формы $f(x, y)$, 
    если для $S = 1,\dots n$ верно:
    \begin{gather*}
        f(e_{2S - 1}, e_{2S}) = 1 \Rightarrow f(e_{2S}, e_{2S-1}) = -1,
    \end{gather*} а для остальных значений $i, j$ верно $f(e_i, e_j) = 0$. 
    Матрица в таком случае имеет следующий вид: 
    \[A_f = \left(\begin{array}{@{}cccc@{}}
		\cline{1-1}
		\multicolumn{1}{|c|}{A_1} & 0 & \dots & 0\\
		\cline{1-2}
		0 & \multicolumn{1}{|c|}{A_2} & \dots & 0\\
		\cline{2-2}
		\vdots & \vdots & \ddots & \vdots\\
		\cline{4-4}
		0 & 0 & \dots & \multicolumn{1}{|c|}{A_m}\\
		\cline{4-4}
	\end{array}\right),\]
	
	где для всех $i$ матрица $A_i$ нулевая или имеет вид $A_i = 
    \begin{pmatrix}
        0  &1 \\
		-1 &0
    \end{pmatrix}$.
\end{definition}

\begin{theorem}[О каноническом виде кососимметричной билинейной функции]~

    Если $f(x, y)$ -- кососимметричная билинейная функция в $V$, то в $V$ существует симплектический базис.
\end{theorem}

\begin{proof}~
    Докажем по индукции по размерности пространства $V$. 
    \begin{enumerate}
        \item Если $f(x, y) = 0$ для всех $x, y$, то $S = 0$ -- очевидно.
        \item Если $f \neq 0$, то найдутся векторы $e_1, e_2$ такие, что $f(e_1, e_2) = c \neq 0$. 
        
        Рассмотрим тогда векторы $e_1' = e_1$, $e_2' = \frac{e_2}{c}$, для которых верно $f(e_1', e_2') = 1$.
        
        Тогда в $V$ существует невырожденное подпространство $U = \langle e'_1, e'_2 \rangle$,
        в котором матрица будет иметь вид $A_{f \vert_{U}} = \begin{pmatrix}
            0  &1 \\
            -1 &0
        \end{pmatrix}$.
        
        По теореме о невырожденном пространстве $V = U \oplus U^{\perp}$. Таким образом 
        если $\dim V = 2$, то искомый базис получен. Иначе по предположению
        индукции искомый базис найдется для $U^{\perp}$, а значит при объединении с $e'_1$ и $e'_2$ 
        получим базис для $V$.
    \end{enumerate}
\end{proof}

\begin{corollary}
    Кососимметричная невырожденная билинейная функция существует только в пространстве чётной размерности.
\end{corollary}

\begin{exercise}
    Доказать следствие, применяя только свойства опредлителей.
\end{exercise}

\begin{idea}
    При транспонировании матрицы $A$ значение определителя не меняется. Тогда если $A = - A^T$, то 
    $|A| = 0$.
\end{idea}

\section{Эрмитовы полуторалинейные функции и формы}
\begin{definition}
    Если рассматривать $V$ над $\Cm$, то в $V$ не бывает положительных функций в привычном нам виде.
    Для сравнения функции с $0$ на комплексных значениях будем считать, что если  $q(x) > 0$, то 
    $q(ix) = f(ix, ix) = -f(x,x) = -q(x) < 0$.
\end{definition}

\begin{definition}
    Полуторалинейными функциями будем называть такие $f^ V \times V \to \Cm$, для которых верны: 
    \begin{enumerate}
        \item Аддитивность по первому аргументу: $f(x_1 + x_2, y) = f(x_1, y) + f(x_2, y)$,
        \item Однородность по первому аргументу: $f(\lambda x, y) = \lambda f(x, y)$ для всех $\lambda \in \Cm$,
        \item Аддитивность по второму аргументу: $f(x, y_1 + y_2) = f(x, y_1) + f(x, y_2)$.
        \item $f(x, \lambda y) = \overline{\lambda} f(x, y)$.
    \end{enumerate}
\end{definition}

\begin{note}
    Первое и второе утверждения вместе называются линейностью по первому аргументу.
    Третье и четвертое утверждения вместе называются антилинейностью по второму аргументу.
\end{note}

\begin{proposition}
    Пусть $f$ -- полуторалинейная функция на $V$, $e$ -- базис в $V$, и векторы $x, y \in V$ имеют 
    координаты $x \leftrightarrow (x_1, x_2, \dots x_n)^T$, $y \leftrightarrow (y_1, y_2, \dots y_n)$.
    Тогда: \begin{gather*}
        f(x, y) = \sum_{i=1}^{n} \sum_{j=1}^{n} a_{ij} x_i \overline{y_j} = x^T A \overline{y}.
    \end{gather*} 
\end{proposition}

\begin{definition}
    Полученное выше выражение называется полуторалинейной формой от $x$, $y$.
\end{definition}

\begin{proposition}
    Пусть $f$ -- полуторалинейная функция в $V$, $e$, $f$ -- базисы в $V$, $S$ -- матрица перехода 
    $S = S_{e \to f}$ и функция $f$ представляется в базисах $V$ матрицами
    $f \underset{e}{\leftrightarrow} A$, $f  \underset{f}{\leftrightarrow} B$, то $B = S^T A \overline{S}$.
\end{proposition}

\begin{proof}
    В базисе $e$ функция $f$ выражается как $f(x, y) = x^T A \overline{y}$. При переходе к базису 
    $f$ получим $x = S x'$, $y =  S y'$. Тогда:
    $$f(x, y) = (Sx')^T A \overline{(Sy')} = (x')^T S^TA \overline{S} \overline{y'} = (x')^TB\overline{y'},$$
    откуда $B = S^TA\overline{S}$.
\end{proof}

\begin{definition}
    Полуторалинейная функция $f(x, y)$ называется эрмитовой или (эрмитово-симметричной) если для всех 
    $x, y \in V$ верно $f(x, y) = \overline{f(y, x)}$. Матрица называется эрмитово-симметричной если 
    $A = \overline{A^T}$. 
\end{definition}

\begin{note}
    Комплексное сопряжение $\overline{A}$ к матрице $A$ стоит воспринимать как замену всех её элементов 
    на комплексно-сопряженные к ним.
\end{note}

\begin{proposition}
    Полуторалинейная функция $f$ эрмитова тогда и только тогда, когда в произвольном базисе $e$ её 
    матрица эрмитова.
\end{proposition}

\begin{proof}~
    \begin{enumerate}
        \item Необходимость. Пусть $f$ эрмитова. Тогда верно: \begin{gather*}
            a_{ij} = f(e_i, e_j) = \overline{f(e_i, e_j)} = \overline{a_{ij}}.
        \end{gather*} Отсюда следует $A = \overline{A^T}$.
        \item Достаточность. Пусть $A = \overline{A^T}$, откуда $A^T = \overline{A}$.
        Тогда: 
        \begin{gather*}
            f(x, y) = (x^T A \overline{y}) = (x^T A \overline{y})^T = \overline{y^T} A^T x = 
            \overline{y^T} A^T \overline{\overline{x}} = \overline{y^T A \overline{x}} = \overline{f(y, x)}.
        \end{gather*}
    \end{enumerate}
\end{proof}

\begin{definition}
    Пусть $f(x, y)$ -- эрмитова полуторалинейная функция. Будем говорить, что векторы $x, y \in V$ 
    ортогональны $x \vert y$, если $f(x, y) = f(y, x) = 0$. \\ 
    Определим ортогональное дополнение как   
    $U^{\perp} = \{y \in V \vert \forall x \in U \hookrightarrow f(x, y) = 0\}$.
\end{definition}

\begin{theorem}
    Пусть $f$ -- эрмитова полуторалинейная функция  и $f \vert_{U}$ -- сужение $f$ на $U$.
    Тогда $f \vert_{U}$ невырождена тогда и только тогда, когда $V = U \oplus U^{\perp}$.
\end{theorem}

\begin{definition}
    Пусть $\Delta = \{(x, x) \vert x \in V\}$ -- диагональ декартового квадрата. Тогда функция $q: V \to \Cm$ 
    назвается эрмитовой квадратичной функцией $q(x) = f(x, x) = f \vert_{\Delta}$, где $f$ -- эрмитова 
    симметричная функция.
\end{definition}

\begin{note}
    Если $f(x, x) = \overline{f(x, x)}$, то $q(x)$ -- эрмитова квадратичная функция.
\end{note}

\begin{exercise}
    Доказать изоморфизм $Q(V)_{\R}$ и $H(V)_{\R}$ -- пространства эрмитовых симметричных функций.
\end{exercise}

\begin{idea}
    Пусть $q(x)$ -- эрмитова квадратичная функция. Тогда: \begin{gather*}
        q(x + y) = f(x + y, x + y) = f(x, x) + \overline{f(y, x)} + f(x, y) + f(y, y) = \\ 
        = f(x, x) + f(y, x) + f(x, y) + f(y, y) = q(x) + q(y) + 2 \re(f(x, y)).
    \end{gather*}
    Отсюда получаем: \begin{gather*}
        \re(f(x, y)) = \frac{1}{2} (q(x+y) - q(x) - q(y))
    \end{gather*}
    Аналогично для аргументов с мнимой частью можно получить:
    \begin{gather*}
        q(x + iy) = f(x + iy, x + iy) = f(x, x) + \overline{f(iy, x)} + f(x, iy) + f(iy, iy) = \\ 
        =  f(x, x) + f(iy, x) + f(x, iy) + f(iy, iy) = q(x) + q(iy) + 2 \re(-if(x, y)) = \\ 
        = q(x) + q(y) + 2 \im(f(x, y)).
    \end{gather*}
\end{idea}

\begin{theorem}[О существовании канонического базиса]~

    Пусть $q$ -- эрмитова квадратичная функция (или соответствующая ей эрмитова симметричная функция $f$).
    Тогда в $V$ существует базис $e$, в котором матрица $q(f)$ диагональна, причем на главной диагонали 
    стоят числа $\pm 1$ и $0$.
\end{theorem}

\begin{idea}
    Пусть $q \neq 0$. Тогда в $V$ существует такой ненулевой вектор $e_1$, что $q(e_1) \neq 0$.
    Без ограничения общности можно перейти к $q(e_1) = \pm 1$. \\ Тогда можно рассмотреть пространство 
    $U = \langle e_1 \rangle$ и ортогональное дополнение к нему, образующие прямую сумму.
\end{idea}