% 01.02.23 Оля

\section{Алгебра многочленов}
\subsection{Алгебра многочленов над коммутативным кольцом с единицей}

\begin{reminder}
    Алгебра -- алгебраическая структура, с одной стороны являющаяся кольцом со сложением и умножением, 
    а с другой стороны линейным пространством со сложением и умножением на скаляр.
\end{reminder}

\begin{definition}
    Последовательность $(a_0, a_1, a_2,\dots), a_i \in R$ называют финитной если 
    $\exists N : \forall n>N \hookrightarrow a_n = 0$, т.е. если начиная с некоторого номера $N$ все 
    значения $a_n$ равны нулю.
\end{definition}

\begin{definition}
    Пусть R -- коммутативное кольцо с единицей. Многочлен над R -- финитная последовательность элементов 
    A = $(a_0, a_1, a_2,\dots), a_i \in R$. Дополнительно будем использовать обозначение $(A)_i = a_i$.
\end{definition}

\begin{definition}
$R[x]$ -- множество многочленов над кольцом R.
\end{definition}

\begin{definition}
Пусть $A, B \in R[x]$, тогда верны следующие свойства:
\begin{enumerate}
    \item $(A + B)_n = (A)_n + (B)_n$,
    \item $(A \cdot B)_n = \displaystyle\sum_{i = 0}^{n}(A)_i \cdot (B)_{n-i}$,
    \item $\lambda \in R \; (\lambda A)_n = \lambda \cdot (A)_n$.
\end{enumerate}
\end{definition}

\begin{proposition}
    Множество всех многочленов R[x] является коммутативным кольцом с единицей.
\end{proposition}

\begin{note} 
    $1 = (1, 0, 0,\dots)$ -- нейтральный по умножению многочлен. 
\end{note}
\begin{proof}    
    Докажем непосредственной проверкой:

    $$(1 \cdot A)_n = \sum_{i = 0}^{n} 1\cdot(A)_n = (A)_n$$ 
    
    Таким образом, $1 \cdot A = A$. Умножение на единицу справа можно проверить аналогично.
\end{proof}

\begin{definition}
    Введем обозначения: $x = (0, 1, 0, 0, \dots)$, $x^2 = (0, 0, 1, 0, \dots)$ и так далее, чтобы многочлен 
    с единицей на $i$-й позиции записывался как $x^{i}$ (нумерация коэффициентов с нуля). Тогда произвольный многочлен 
    $A = (a_0, a_1, a_2, \dots)$ можно записать как $A = a_0 \cdot 1 + a_1 \cdot x + a_2 \cdot x^2 + \dots$
\end{definition}

\begin{definition}
    Пусть $P = (a_0, a_1, a_2, \dots)$ -- многочлен. Последний отличный от нуля коэффициент называется 
    старшим коэффициентом многочлена. Номер старшего коэффициента называется степенью многочлена и 
    обозначается как $\deg P$.
\end{definition}

\begin{note}
Будем считать, что степень нулевого многочлена и только нулевого многочлена не определена.
\end{note}

\begin{reminder}
    Делителями нуля называются такие числа $a$ и $b$, что $a \neq 0$ и $b \neq 0$ но $a \cdot b = 0$.
\end{reminder}

\begin{definition}
    Коммутативное кольцо с единицей называется областью целостности или целостным кольцом если оно 
    не имеет делителей нуля.
\end{definition}

\begin{example}
    $\mathbb{Z}_n$ -- область целостности тогда и только тогда, когда $n$ -- простое.
\end{example}

\begin{proposition} 
    \label{pr1.2}
    В области целостности выполняется правило сокращения:
    $$ab = ac, a \neq 0 \Rightarrow b = c.$$
\end{proposition}

\begin{proof}
    $a(b-c) = 0$, $a \neq 0 \Rightarrow b-c = 0$.
\end{proof}

\begin{proposition}
    \label{pr1.3}
    Пусть R -- коммутативное кольцо с единицей, $A, B \in R[x]$, тогда:
    \begin{enumerate}
        \item $\deg(A+B) \leq max(\deg(A), \deg(B))$,
        \item $\deg(A \cdot B) \leq \deg(A) + \deg(B)$,
        \item Если вдобавок R -- область целостности, то $\deg(AB) = \deg(A) + \deg(B)$.
    \end{enumerate}
\end{proposition}

\begin{proof}~
    \begin{enumerate}
        \item Обозначим $\deg A = a$, $\deg B = b$. Пусть $n > max(a, b)$, тогда:
        $$(A+B)_n = (A)_n + (B)_n = 0 + 0 = 0,$$ а значит $\forall n > max(a, b) \Rightarrow (A+B)_n = 0$. 
        Тогда номер последнего ненулевого элемента не превосходит $max(a, b)$, а значит 
        $deg(A+B) \leqslant max(a, b)$

        \item Пусть $n > a + b$, покажем что $(AB)_n = 0$:
        
        $$(AB)_n = \sum_{i = 0}^{a}(A_i)(B_{n-i}) + \sum_{i = a+1}^{n}(A_i)(B_{n-i}) = 0 + 0 = 0$$

        В первой сумме $B_{n-i} = 0$ во всех слагаемых так как $n > a + b$, а значит $n - i > b$ для
        всех $i$ от $0$ до $a$. Во второй сумма во всех слагаемых $A_i = 0$ так как $i > a$ на всем
        диапазоне суммирования. Таким образом обе суммы равны нулю, а значит $(AB)_n = 0$.

        \item Положим $n = a + b$, тогда:
        
        $$(AB)_{n} = \sum_{i=0}^{a-1} (A)_i(B)_{n-i} + (A)_a(B)_b + \sum_{i=a+1}^{a+b} (A)_i(B)_{n-i}$$

        Аналогично предыдущему пункту первое и третье слагаемое будут нулевыми. 
        При этом $(A)_i \neq 0$ и $(B)_{n-i} = (B)_b \neq 0$,  и в силу целостности, 
        $(A)_i(B)_{n-i} \neq 0$, то есть $(AB)_{n} \neq 0$. \\ Для больших,
        чем $n$, номеров сумма будет нулевой, что было показано в предыдущем пункте, 
        а значит $\deg AB = a$.
    \end{enumerate}
\end{proof}

\begin{corollary}
    Если R -- область целостности, то R[x] -- область целостности.
\end{corollary}

\begin{proof}
    Мы уже знаем, что $R[x]$ -- коммутативное кольцо с единицей. Покажем, что в нём нет делителей нуля.
    Пусть $A \neq 0, B \neq 0, A, B \in R$. Согласно пункту 3 доказанного выше утверждения и в силу 
    того, что $\deg(A) \geq 0$ и $\deg(B) \geq 0$, верно $\deg(AB) = \deg(A) + \deg(b) \geq 0$, 
    а значит $AB \neq 0$.
\end{proof}

\begin{definition}
    Пусть R, S -- кольца с единицей. Отображение $\phi : R \to S$ называется гомоморфизмом колец с единицей если 
    выполняется:
    \begin{enumerate}
        \item $\forall r_1, r_2 \in R \;\; \phi(r_1 + r_2) = \phi(r_1) + \phi(r_2)$,
        \item $\phi(r_1 \cdot r_2) = \phi(r_1) \cdot \phi(r_2)$,
        \item $\phi(1) = 1$.
    \end{enumerate}
\end{definition}

\begin{note}
    Условие перехода единицы в единицу является существенным и выполняется не всегда. Рассмотрим,
    например, отображение $\phi : R \to S$, где $R = M_n(R)$, $S = M_m(R)$, переводящее матрицу А 
    размером $n \times n$ в левый верхний угол матрицы размером $m \times m$ (где $m > n$), 
    заполняя остальное нулями:
    \[\phi(A) = \left(\begin{array}{@{}c|c@{}}
		A & 0\\
		\hline
		0 & 0
	\end{array}\right)\]
    Заметим, что единичная матрица $E_n \in R$ не переходит в единичную матрицу $E_m \in S$, несмотря
    на то, что условия 1 и 2 выполнены, а значит отображение $\phi$ не является гомоморфизмом колец 
    с единицей.
\end{note}

\begin{proposition}[об универсальности свойства кольца многочленов]~

    Пусть А -- кольцо с единицей (не обязательно коммутативное), такое что $R \subseteq A$, и пусть 
    $\forall a\in A \; \forall r \in R$ верно $ar = ra$. Зафиксируем $a \in A$. Тогда существует 
    единственный гомоморфизм $\phi_a : R[x] \to A$, такой что $\forall r \in R \rightarrow \phi_a(r) = r$ и 
    $\phi_a(x) = a$.
\end{proposition}

\begin{proof}~
    \begin{enumerate}
        \item Докажем единственность полагая существование:

        $$P = \sum_{i = 0} p_ix^i \Rightarrow \phi_a(P) = \sum_{i = 0} \phi(p_i)\cdot\phi(x^i) = \sum_{i = 0}p_i \cdot a^i$$
        
        В силу свойств гомоморфизма колец с единицей образ для каждого многочлена задается однозначно.

        \item Покажем, что отображение $\phi_a : R[x] \to A$ является гомоморфизмом колец:

        Выберем два многочлена $P = \displaystyle\sum_{i = 0} p_i x^i$ и 
        $Q = \displaystyle\sum_{i = 0} q_i x^i$. Их произведение выражается как: 
        $$P \cdot Q = \displaystyle\sum_{i=0}\displaystyle\sum_{j = 0}(p_i q_j)x^{i+j}.$$ 
        Тогда верна следующая цепочка равенств:
        \begin{align}
            \phi_a(P \cdot Q) = \phi_a(\sum_{n = 0}\sum_{i + j = n} A_i B_j) = \sum_{n = 0}\sum_{i + j = n} \phi_a(A_i B_j) = \\ = \sum_{n = 0}\sum_{i + j = n} p_i a^i q_j a^j = (\sum_{i = 0}p_i a^i)(\sum_{j = 0} q_j a^j) = \phi_a(P) \cdot \phi_a(Q)
        \end{align}
        
        Таким образом мы доказали линейность по умножению. Проверка линейности по сложению 
        предоставляется в качестве упражнения. Истинность $\phi_a(1) = 1$ будем считать очевидной.
    \end{enumerate}
\end{proof}

\begin{definition}
    В условиях предыдущего утверждения положим $R = A$, $p \in R[x]$.
    Значением многочлена $P$ на элементе $a$ кольца $A$ называется $P(a) = \phi_a(P)$.
\end{definition}

\begin{note}
    Определение выше согласуется с привычным: пусть многочлен $P = \displaystyle\sum_{i} p_i x^i$.
    
    Тогда его значение на элементе $a$: $P(a) = \displaystyle\sum_{i} p_i a^i$.
\end{note}

\subsubsection{Применение конструкции значения многочлена}

\begin{algorithm}
    Пусть $F$ -- поле, $F[x]$ -- кольцо многочленов над $F$, V -- линейное пространство над $F$, 
    $A = \mathcal{L}(V)$ -- кольцо линейных операторов над V. Произведем вложение 
    $F \subseteq \mathcal{L}(V)$ следующей инъекцией: 
    $$\alpha \in F \mapsto \alpha \cdot E \in \mathcal{L},$$ где Е -- единичная матрица.
    
    Возьмем $\psi \in \mathcal{L}(V)$. Рассмотрим $\phi_\psi: F[x] \to \mathcal{L}$ в условиях 
    предыдущего утверждения. 
    
    Пусть $P = \displaystyle\sum_{i}p_i x^i$, 
    тогда $\phi_\psi(P) = P(\psi) = \displaystyle\sum_{i} p_i \psi^i$.
\end{algorithm}

\begin{note}
    Кольцо от двух переменных $R[x_1, x_2] = R[x_1][x_2]$ -- кольцо многочленов над $R[x_1]$.
\end{note}

\begin{note}
    $R[[x]]$ -- кольцо формальных степенных рядов.
\end{note}

\subsection{Многочлены над полем F}
\subsubsection{Алгоритм деления с остатком}

\begin{theorem}
    Пусть $A, B \in F[x]$, F -- поле, $B \neq 0$. Тогда:
    \begin{enumerate}
        \item Cуществуют $Q, R \in F[x]$ т.ч. $A = QB + R$, где $R = 0$ или $\deg R < \deg B$.
        \item Многочлены R и Q определены однозначно.
    \end{enumerate}
\end{theorem}

\begin{proof}~
    \begin{enumerate}
        \item Индукция по $\deg A$:
    
        Пусть $A = 0$ или $\deg A \leq B$, тогда очевидно $A = 0 \cdot B +A$.

        Пусть теперь $\deg A \geq \deg B$, и они равны a и b соответственно. Тогда старшие 
        члены равны $HT(A) = \alpha x^a$ и $HT(B) = \beta x^b$. Подберем моном M такой что 
        $HT(A) = M \cdot HT(B)$, например $M = \frac{\alpha}{\beta} \cdot x^{a - b}$.

        Введем обозначение $A' = A - MB$, $\deg A' < \deg A$ по построению $M$. По предположению 
        $A' = Q'B + R'$, где $R' = 0$ или $\deg R' < \deg B$. Тогда искомое разложение: 
        $$A = A' + MB = Q'B + MB + R' = (Q' + M)B + R'.$$

        \item Предположим существуют два разложения $A = Q_1 B + R_1 = Q_2 B + R_2$, многочлены 
        удовлетворяют условиям. Тогда $(Q_1 - Q_2)B = R_2 - R_1$. 
        
        Предположим $Q_1 \neq Q_2$, тогда $\deg((Q_1 - Q_2)B) > \deg(B)$.
        При этом $\deg (R_2 - R_1) < \deg B$ так как $\deg R_1 < \deg B$ и $\deg R_2 < \deg B$, 
        а значит мы пришли к противоречию и $Q_1 = Q_2$. В таком 
        случае так же верно и $R_1 = R_2$. 
    \end{enumerate}
\end{proof}

\begin{note}
    Для R, являющегося коммутативным кольцом с единицей, но не являющегося полем, доказанное выше 
    может быть неверно в общем случае, так как обратный к $\beta$ элемент $\beta^{-1}$ не обязан существовать 
    и построение $M$ не будет корректным.
\end{note}

\begin{note}
    Доказательство существования аналогично делению в столбик, где A' является промежуточным частным.
\end{note}

\begin{definition}
    Пусть $f(x) = a_0 x^n + a_{1} x^{n-1} + \dots + a_n, a_i \in F[x]$, $c \in F$. 
    Тогда значением многочлена $f(x)$ в элементе поля $F$ называется
    $f(c) = a_0 c^n + a_{1} c^{n-1} + \dots + a_n$.
\end{definition}

\begin{definition}
    Элемент поля $F$ является корнем многочлена $f$, если $f(c) = 0$.
\end{definition}

\begin{proposition}
    Значение F(c) в точке $c \in F$ равно остатку от деления $F$ на $(x-c)$.
\end{proposition}

\begin{proof}
    Рассмотрим многочлен: $$f(x) = q(x)(x-c) + r(x).$$ 
    При этом $r(x) = 0$ или $\deg r < \deg(x-c) = 1$, а значит $r(x)$ -- константа поля $F$. 
\\
    При подстановке точки $c$ получаем значение $f(c) = r$. Тогда верно $f(x) = q(x)(x-c) + r$ и при $x = c$ 
    верно $f(c) = r$.
\end{proof}

\begin{theorem}[Безу]
    \label{th1.2}
    Число $c$ является корнем $f(x) \Leftrightarrow f(x) \vdots (x-c)$.
\end{theorem}

\begin{proof}
    Пусть $c$ -- корень $f(x)$, тогда $f(c) = 0$, а значит $(x-c) \, \vert \, f(x)$. В обратную сторону, 
    если $(x-c) \, \vert \, f(x)$, то $f(x) = q(x)(x-c)$, то есть $f(c) = 0$ и $c$ -- корень.
\end{proof}

\begin{note}
    Для упрощенного деления многочлена на (x-c) не обязательно делить в столбик, можно использовать 
    схему Горнера.
\end{note}

\begin{theorem}[Схема Горнера]
    Пусть задан многочлен $f(x) \in F[x]$: 
    \begin{gather*}
        f(x) = a_0 x^n + a_{1} x^{n-1} + \dots + a_n, a_i \in F,
    \end{gather*}
    и необходимо разделить его на $(x-c)$, то есть представить в следующем виде:
    \begin{gather*}
        f(x) = q(x)(x-c) + r.
    \end{gather*}
    Для этого получить коэффициенты многочлена $q(x)$:
    \begin{gather*}
        q(x) = b_0 x^{n-1} + b_1 x^{n-2} + \dots + b_{n-1}.
    \end{gather*}
    Для этого запишем коэффициенты в таблицу: в верхней строчке коэффициенты $a_i$, под ними 
    соответствующие $b_i$. Тогда $b_i \cdot c + a_{i+1} = b_{i+1}$.
\end{theorem}

\begin{proof}
    Приравниваем коэффициенты при х в одинаковых степенях в получившемся произведении и в $f(x)$ и 
    получаем искомое соотношение:
    $$(b_0 x^{n-1} + b_1 x^{n-2} + \dots + b_{n-1})(x-c) = b_0 x^n + (b_1 - c \cdot b_0)x^{n-1} + 
    \dots - b_{n-1}c.$$
\end{proof}

\begin{problem}
    Обобщить схему Горнера на трехчлен $x^2 + bx + c$. 
\end{problem}

\begin{definition}
    $A$ делится на $B$, если существует такой многочлен $Q$ что $A = QB$. Пишут $A \vdots B$ или 
    $B \vert A$.
\end{definition}

\begin{definition}
    Многочлены $A$ и $B$ называются ассоциироваными если $B \vert A$ и $A \vert B$, то есть когда верны 
    представления $A = Q_1 B$, $B = Q_2 A$. При этом:
    \begin{eqnarray*}
        \deg A = \deg Q_1 + \deg B \geq \deg B, \\
        \deg B = \deg Q_2 + \deg B \geq \deg A,
    \end{eqnarray*}
    откуда $\deg A = \deg B$, $\deg Q_1 = \deg Q_2 = 0$.
\end{definition}

\begin{definition}
    Пусть $f(x)$ и $g(x) \in F[x]$ -- ненулевые одновременно многочлены. Многочлен $d(x) \in F[x]$ 
    называется наибольшим общим делителем (НОД, gcd) если:
    \begin{enumerate}
        \item $d \vert f$, $d \vert g$.
        \item если $d'$ -- общий делитель $f$ и $g$, то $d' \vert d$.
    \end{enumerate}

    Иначе говоря, НОД многочленов $f$ и $g$ -- такой общий делитель, который делится на любой общий делитель.
    НОД определен с точностью до ассоциированности.
\end{definition}

\begin{theorem}[о существовании НОД]
    \label{th1.4}
    Пусть $f, g \in F[x]$ и $f, g$ ненулевые одновременно. Тогда существует 
    $d(x) = \gd(f, g) \in F[x]$ и, более того, существуют 
    $u(x), v(x) \in F(x)$, такие что $u(x)f(x) + v(x)g(x) = d(x)$.
\end{theorem}

\begin{proof}
    Пусть без ограничения общности $f(x) = 0, g(x) \neq 0$, то есть нулевой ровно один из многочленов. 
    Тогда верно представление $d(x) = g(x), d = 0\cdot f + 1\cdot g$.

    Пусть теперь оба многочлена ненулевые. Тогда можно выполнить цепочку делений многочленов, где 
    на каждом новом шаге делимым и делителем будут становиться делитель и частное предыдущего деления
    соответственно. Таким образом для каждой пары НОД будет сохраняться, так как если делитель кратен 
    некоторому многочлену, то делимое и частное будут кратны ему одновременно. Первые несколько шагов:
    \begin{align*}
        f(x)   & = q_1(x)g(x) + r_1(x), \\
        g(x)   & = q_2(x)r_1(x) + r_2(x), \\
        r_1(x) & = q_3(x)r_2(x) + r_3(x).
    \end{align*}
    Продолжая действовать так дойдем до последних двух шагов, после которых остаток будет равен нулю.
    При делении степень остатка меньше степени делителя, а значит, в силу конечности номеров старших 
    членов начальных многочленов, в некоторый момент процесс действительно остановится:
    \begin{align*}
        r_{n-2}(x) & = q_n(x)r_{n-1}(x) + r_n(x), \\
        r_{n-1}(x) & = q_{n+1}(x)r_{n}(x).
    \end{align*}
    Получается, что $\gd(f, g)$ = $r_n$ -- последний ненулевой остаток. Проверим:
    \begin{enumerate}
        \item $r_n \vert r_{n-1}$, $r_n \vert r_{n-2}, \dots$. 
        Продолжая подниматься наверх получаем $r_n \vert f$, $r_n \vert g$
        \item Теперь будем спускаться вниз, пусть $d' \vert f$, $d' \vert g$. 
        Таким образом мы дойдем до $d' \vert d$.
    \end{enumerate}
\end{proof}

\begin{corollary}
    $\gd(f, g) = \gd(g_1, r_1) = \gd(r_1, r_2) = \dots = \gd(r_{n-1}, r_n)$.
\end{corollary}

\begin{proof}
    Покажем, что все остатки $r_1, r_2, \dots, r_n$ являются линейными комбинациями многочленов $f$ и $g$:
    $$r_1 = f - q_1g$$
    $$r_2 = g - q_2r_1 = -q_2f + (1 + q_1q_2)g$$

    Спускаясь вниз и подставляя выражения предыдущих остатков в последующие получим все разложения.
    Положим $r_{n-2} = u''f + v''g$ и $r_{n-1} = u'f + v'g$. Тогда:
    $$d = r_n = r_{n-2} - q_n - r_{n-1} = f(u'' - u'q_n) + g(v'' - v'g_n).$$

    Таким образом все остатки можно выразить через $f$ и $g$.
\end{proof}

\begin{exercise}
    Докажите, что используя неоднозначность выбора коэффициентов у $u(x)$ и $v(x)$ можно добиться, 
    чтобы $\deg u < \deg g$, $\deg v < \deg f$.
\end{exercise}

\subsubsection{Неприводимые многочлены}

\begin{definition}
    Многочлен $P \in F[x]$ степени больше нуля называется неприводимым над полем $F$, если из $P = AB$ 
    следует $\deg A = 0$ или $\deg B = 0$.
    Иначе говоря, многочлен называется неприводимым над полем F, если его нельзя разложить в 
    произведение двух многочленов более низких степеней из этого же кольца $F[x]$.
\end{definition}

\begin{note}
    Важно над каким полем многочлен является неприводимым, например многочлен $x^2 + 1$ является 
    приводимым над полем комплексных чисел $\mathbb{C}$, но неприводимым над полем действительных 
    чисел $\mathbb{R}$.
\end{note}

\begin{note}
    Для многочлена справедлива основная теорема арифметики при разложении на неприводимые множители. 
\end{note}
