% 19.04.23

\begin{reminder}
    Функция $q(x)$ называется эрмитовой квадратичной функцией, если она получена сужением на $\Delta \subseteq V \times V$
    эрмитовой полуторалинейной функции $q(x) = f(x, y) \vert_{\Delta} = f(x, x)$. \\
    Таким образом $f(x, y) = \overline{f(y, x)}$, откуда $q(x) = f(x, x) = \overline{f(x, x)} \in \R$.
\end{reminder}

\begin{proposition}
    Пусть $e$ - канонический базис в пространстве $V$, $q(x)$ - квадратичная эрмитова функция. 
    Тогда:
    \begin{enumerate}
        \item $q(x)$ положительно определена тогда и только Тогда, когда её матрица в каноническом 
        базисе единичная
        \item $q(x)$ положительно полуопределена тогда и только тогда, когда её матрица в 
        каноническом базисе не имеет -1 на главной диагонали.
    \end{enumerate}
\end{proposition}

\begin{proof}~
    \begin{enumerate}
        \item \begin{enumerate}
            \item Необходимость - как в вещественном случае.
            \item Достаточность: $$q(x) = x^T A \overline{x} = x^T \overline{x} = \sum_{i=1}^{n} x_i \overline{x_i} 
            = \sum_{i=1}^{n} |x_i| = 0.$$
        \end{enumerate}
        \item \begin{enumerate}
            \item Необходимость - как в вещественном случае.
            \item Достаточность. Матрица принимает следующий вид:
            \begin{gather*}
                A = \begin{pmatrix}
                E_r    & 0      \\
                0      & 0      \\
                \end{pmatrix}
            \end{gather*}
            Тогда $r = \rk r$, $q(x) = x^T A \overline{x} = \sum_{i=1}^{r} |x_i|^2 \geq 0$ для всех $x \in V$.
        \end{enumerate}
    \end{enumerate}
\end{proof}

\begin{proposition}[Закон инерции для квадратичных эрмитовых функций]~ 

    Пусть $e$ - произвольный канонический базис для $q(x)$ и пусть $p, q$ - положительный и отрицательный
    индексы инерции относительно $e$. Тогда:
    \begin{enumerate}
        \item $p = \max \{\dim U | U \leq V: q \vert_{U} \text{ - положительно опеределена}\}$.
        \item $q = \max \{\dim U | U \leq V: q \vert_{U} \text{ - отрицательно опеределена}\}$.
        \item $p$ и $q$ не зависят от выбора канонического базиса.
    \end{enumerate}
\end{proposition}

\begin{proposition}[Аналог критерия Сильвестра]~ 

    Пусть $q(x) \in H(V)$ - эрмитова квадратичная функция, $A$ - её матрица в произвольном базисе, где 
    выполняется условие эрмитовости $\overline{A^T} = A$. Тогда:
    \begin{enumerate}
        \item $q(x)$ положительно определена тогда и только тогда, когда $\Delta_1 > 0$, $\Delta_2 > 0$, $\dots$, 
        $\Delta_n > 0$.
        \item $q(x)$ отрицательно определена тогда и только тогда, когда $\Delta_1 > 0$, $\Delta_2 > 0$, $\dots$, 
        $\sgn (\Delta_n) = (-1)^n$.
    \end{enumerate}
\end{proposition}

\subsection{Евклидовы и Эрмитовы пространства}

\begin{definition}
    Линейное пространство над полем действительных чисел называется Евклидовым, если 
    на нем определена положительно определенная билинейная симметрическая функция $f(x, y)$. По 
    определению $f(x, y)$ называется скалярным произведением и обозначается $(x, y)$.
    Скалярное произведение можно определить при помощи следующих аксиом:
    \begin{enumerate}
        \item $(x, y) = (y, x)$ - симметричность,
        \item $(x_1 + x_2, y) = (x_1, y) + (x_2, y)$, 
        \item $\forall \lambda \in \R \hookrightarrow (\lambda x, y) = \lambda(x, y)$,
        \item $(x, x) \geq 0$, при этом $(x, x) = 0 \Leftrightarrow x = 0$ - положительная определенность.
    \end{enumerate}
    Второе и третье свойства вместе называются линейностью по первому аргументу.
\end{definition}

\begin{example}~
    \begin{enumerate}
        \item Пространство $V_3$ со скалярным произведением $(x, y) = |x| |y| \cos\angle (x, y)$,
        \item Пространство $\R^n$ со скалярным произведением $(x, y) = x^T E y = x^T y = \displaystyle\sum_{x_i}^{y_i}$.
        \item Пространство $M_n(\R)$ со скалярным произведением $(X, Y) = \tr(X^T, Y)$,
        \item Пространство $C_{[a, b]}$ со скалярным произведением $ (f, g) = \int_{a}^{b} f(x)g(x) dx$.
    \end{enumerate} 
\end{example}

\begin{definition}
    Длиной вектора $x$ называется $|x| = \sqrt{(x,x)}$.
\end{definition}

\begin{definition}
    Если $x, y$ - два ненулевых вектора, то можно ввести угол между ними: 
    $$\phi = \angle (x, y) = \arccos \frac{(x, y)}{\sqrt{(x, x)(y, y)}}.$$
\end{definition}

\begin{note}
    В понятие скалярного произведения включена возможность введения длины вектора и угла между 
    векторами.
\end{note}

\begin{definition}
    Пусть $V$ - линейное пространство над $\Cm$. $V$ называется эрмитовым, если на $V$ определена 
    положительно определенная эрмитова полуторалинейная функция $f(x, y)$. Аналогично с евклидовыми 
    пространствами $f(x, y)$ называется скалярным произведением и обозначается $(x, y)$.
\end{definition}

\begin{note}
    В отличие от Евклидовых прространств, на эрмитовых пространствах скалярное произведение может 
    принимать комлексные значениях.
\end{note}

\begin{note}
    Скалярное произведение на Эрмитовых пространствах можно определить при помощи следующих аксиом:
    \begin{enumerate}
        \item $(x, y) = \overline{y, x}$ - эрмитова симметричность.
        \item $(x_1 + x_2, y) = (x_1, y) + (x_2, y)$, 
        \item $\forall \lambda \in \Cm \hookrightarrow (\lambda x, y) = \lambda(x, y)$
        \item Для всех $x \in V$ верно $(x, x) \in \R$, причем $(x, x) = 0 \Leftrightarrow x = 0$.
    \end{enumerate}
\end{note}

\begin{example}~
    \begin{enumerate}
        \item Эрмитово пространство $\Cm^n$ со скалярным произведением 
        $(x, y) = x^T \cdot E \cdot \overline{y} = \displaystyle\sum_{i=1}^{n} x_i \overline{y_i}$.
        \item Эрмитово пространство $M_n(\Cm)$ со скалярным произведением $(X, Y) = \tr (X^T \overline{Y})$.
        \item Пространство $C_{[a, b]}^{\Cm}$ со скалярным произведением $(f, g) = \int_{a}^{b} f(x) \overline{g(x)} dx$.
    \end{enumerate}
\end{example}

\begin{note}
    Многие утверждения этого раздела верны одновременно и для евклидовых и для эрмитовых пространств,
    поэтому в случае, когда это так, будем говорить "пространство со скалярным произведением"\,, подразумевая 
    любое из них.
\end{note}

\begin{definition}
    Матрицей Грама системы $a_1, a_2, \dots a_k$ называется матрица: 
    \begin{gather*}
        \Gamma(a_1, \dots a_n) = \begin{pmatrix}
        (a_1, a_1)      & (a_1, a_2)      & \dots  & (a_1, a_k)        \\
        (a_2, a_1)      & (a_2, a_2)      & \dots  & (a_2, a_k)        \\
        \vdots & \vdots & \ddots & \vdots   \\
        (a_k, a_1)      & (a_k, a_2)      & \dots  & (a_k, a_k)
        \end{pmatrix}
    \end{gather*}
\end{definition}

\begin{theorem}~
    \label{th11.1}
    \begin{enumerate}
        \item
        Пусть $e_1, e_2, \dots e_n$ - базис в $V$, $\Gamma = \Gamma(e)$. Тогда $\forall x, y \in V$ верно 
        $(x, y) = x^T \cdot \Gamma \cdot \overline{y}$
        \item Пусть $a_1, a_2, \dots a_k$ - произвольная система векторов в $V$. Тогда $|\Gamma(a_1, \dots a_n)| \geq 0$, 
        причем равенство достигается тогда и только тогда, когда система линейно зависима.
    \end{enumerate}
\end{theorem}

\begin{proof}~
    \begin{enumerate}
        \item $f(x, y) = x^T \cdot A \cdot \overline{y} =  x^T \cdot \Gamma \cdot \overline{y}$,
        где $a_{ij} = f(e_i, e_j) = (e_i, e_j) = (\Gamma)_{ij}$.
        \item Пусть система линейно независима. Тогда $U = \langle a_1, a_2, \dots a_k \rangle$, 
        $f \vert_{U}$ - положительно определена, а значит по критерию Сильвестра $|\Gamma(a_1, \dots a_n)| > 0$.

        Пусть теперь система линейно зависима и без ограничения общности: 
        $$a_k = \lambda_1 a_1 + \dots + \lambda_{k-1} a_{k-1}.$$
        Заметим, что элемент нижней строки матрицы Грама в таком случае равен: 
        $$(a_k, a_i) = (\lambda_1 a_1 + \dots + \lambda_{k-1} a_{k-1}, a_i) = \lambda_1 (a_1, a_i) + 
        \lambda_2 (a_2, a_i) + \, \dots \, + \lambda_{k-1} (a_{k-1}, a_i).$$
        Таким образом строки матрицы Грама линейно зависимы, так как последнюю строку можно представить 
        в виде суммы всех предыдущих с коэффициентами $\lambda_1$, $\lambda_2$, $\dots$, $\lambda_{k-1}$.

        Тогда матрица Грама вырождена, а значит её определитель равен нулю.
    \end{enumerate}
\end{proof}

\begin{corollary}
    Пусть $\Gamma = \Gamma(a_1, a_2, \dots a_n)$, $n = \dim V$. Тогда $\Gamma$ положительно определена 
    тогда и только тогда, когда $a_1, a_2, \dots a_n$ - базис в $V$.
\end{corollary}

\begin{proof}~
    \begin{enumerate}
        \item Необходимость. 
        
        Пусть $|\Gamma(a_1, a_2, \dots a_n)| > 0$. Тогда по теореме \ref{th11.1} система 
        $a_1, a_2, \dots a_n$ линейно независима, а значит является базисом в $V$.
        \item Достаточность. 
        
        Пусть система $a_1, a_2, \dots a_n$ является базисом. Тогда она 
        линейно независима, а значит определитель матрицы Грама $|\Gamma(a_1, a_2, \dots a_n)| > 0$. Тогда по 
        критерию Сильвестра $\Gamma$ положительно определена.
    \end{enumerate}
\end{proof}

\begin{theorem}[Неравенство Коши-Буняковского]
    Пусть $V$ - пространство со скалярным произведением, и пусть $x, y \in V$. Тогда: 
    \begin{gather*}
        |(x, y)|^2 \leq (x, x) \cdot (y, y)
    \end{gather*}
\end{theorem}

\begin{proof}~
    \begin{enumerate}
        \item Пусть $x$ или $y$ - нулевой вектор, тогда $0 = 0$.
        \item Пусть $x$ и $y$ ненулевые и коллинеарны, то есть $y = \lambda x$. Тогда: 
        \begin{gather*}
            |(x, \lambda x)|^2 = |\lambda|^2 |(x, x)|^2 = \lambda \overline{\lambda} |(x, x)|^2 = 
            (x, x) (y, y)
        \end{gather*}
        \item Пусть $x$ и $y$ ненулевые и неколлинеарны. Тогда система из $x$ и $y$ линейно независима, а значит 
        по теореме \ref{th11.1}:
        \begin{gather*}
            0 < |\Gamma(x, y)| = (x, x)(y, y) - (x, y)(y, x) = (x, x)(y, y) - |(x, y)|^2.
        \end{gather*}
    \end{enumerate}
\end{proof}

\begin{corollary}
    В неравенстве Коши-Буняковского равенство достигается тогда и только тогда, когда система один из 
    векторов нулевой или они коллинеарны.
\end{corollary}

\begin{corollary}[Корректность определения угла]~ \\
    Пусть $V$ - евклидово пространство, $x, y$ - ненулевые векторы. Тогда по КБ 
    $|(x, y)| \leq \sqrt{(x, x)(y, y)}$, а значит аргумент $\arccos \frac{(x, y)}{\sqrt{(x, x)(y, y)}}$
    не превосходит 1.
\end{corollary}

\begin{corollary}[Неравенство треугольника]
    Для всех $x, y \in V$ верно:
    \begin{gather*}
        |x + y| \leq |x| + |y|
    \end{gather*}
\end{corollary}

\begin{proof}
    Докажем непосредственной проверкой:
    \begin{gather*}
        | x + y |^2 = (x+y, x+y) = (x, x) + (x, y) + (y, x) + (y, y) = (x, x) + 2 \re (x, y) + (y, y).
    \end{gather*}
    При этом $\re (x, y) \leq |(x, y)| \leq |x| \cdot |y|$, а значит:
    \begin{gather*}
        |x + y|^2 \leq |x|^2 + 2 |x| \cdot |y| + |y|^2 = (|x| + |y|)^2.
    \end{gather*}
    Величины $|x|$ и $|y|$ неотрицательны, а значит $|x+y| \leq |x| + |y|$.
\end{proof}

\subsubsection{Ортогональность в пространстве со скалярным произведением}

\begin{definition}
    $x \perp y$ если $(x, y) = 0$.
\end{definition}

\begin{definition}
    Система векторов $x_1, x_2, \dots x_k$ называется ортогональной тогда и только тогда, когда 
    $(x_i, x_j) = 0$ для всех $i \neq j$.
\end{definition}

\begin{definition}
    Система векторов $x_1, x_2, \dots x_k$ называется ортонормированной тогда и только тогда, когда 
    она ортогональна и нормирована. Нормированность означает, что $(x_i, x_i) = 1$ для всех $i$.
\end{definition}

\begin{definition}
    Система подпространств $U_1$, $U_2, \dots$, $U_k$ называется ортогональной тогда и только тогда,
    когда для любой системы векторов $u_1 \in U_1$, $u_2 \in U_2$, $\dots u_k \in U_k$ верно, 
    что она ортогональна.
\end{definition}

\begin{proposition}
    Пусть $U_1$, $U_2, \dots$, $U_k$ - ортогональная система подпространств. Тогда 
    \begin{gather*}
        U_1 + U_2 + \dots U_k = U_1 \oplus U_2 \oplus \dots \oplus U_k.
    \end{gather*}
\end{proposition}

\begin{proof}
    Покажем, что пересечение подпространств $U_i \cap (U_1 + \dots + U_k) = \{0\}$. 

    Действительно, пусть $x \in U_i$ и $x \in U_1 + \, \dots \, + U_{i-1} + U_{i+1} \, \dots \, + U_k$. Тогда:
    \begin{gather*}
        (x, x) = (x_i, x_1 + \, \dots \, + x_{i-1} + x_{i+1} + \, \dots \, + x_k) = 0.
    \end{gather*}

    Таким образом $x = 0$, а значит $U_i \cap (U_1 + \dots + U_k) = \{0\}$.
\end{proof}

\begin{corollary}
    \label{col11.1}
    Если $U_1$, $U_2, \dots$, $U_k$ - ортогональная система подпространств, то эти подпространства 
    линейно независимы.
\end{corollary}

\begin{corollary}
    Если векторы $x_1, \, \dots , \, x_k$ ненулевые и образуют ортогональную систему, то они линейно-независимы.
\end{corollary}

\begin{proof}
    Пусть $U_1 = \langle x_1 \rangle$, $U_2 = \langle x_2 \rangle$, $\dots$, $U_k = \langle x_k \rangle$.
    Тогда если система $x_1, \, \dots , \, x_k$ линейно зависима, то соответствующие пространства 
    $U_1$, $U_2, \dots$, $U_k$ линейно зависимы, что противоречит следствию \ref{col11.1}.
\end{proof}

\begin{proposition}
    Пусть $V$ - пространство со скалярным произведением. Тогда в нём существует ортонормированный базис.
\end{proposition}

\begin{proof}
    Пусть $f(x, y) = (x, y)$, тогда для неё существует канонический базис, в котором $f$ имеет 
    матрицу $E$. $f(e_i, e_j) = (e_i, e_j) = \delta_{ij}$, откуда этот базис - ортонормированный.
\end{proof}

\begin{corollary}[Выражение координат векторов и скалярного произведения в ортонормированном базисе]
    Пусть $V$ - пространство со скалярным произведением, $e$ - ортонормированный базис. Пусть 
    $x \leftrightarrow \alpha$, $y \leftrightarrow \beta$. Тогда:
    \begin{enumerate}
        \item $(x, y) = \alpha^T \cdot \overline{\beta}$,
        \item $\alpha_i = (x, e_i)$.
    \end{enumerate}
\end{corollary}

\begin{proof}~
    \begin{enumerate}
        \item $f(x, y) = \alpha^T \cdot A \cdot \overline{\beta} = \alpha^T \cdot E \cdot \overline{\beta} 
        = \alpha^T \cdot \overline{\beta}$.
        \item $(x, e_i) = (\alpha_1, \,\dots ,\,) \cdot (0, \, \dots , \, 1, \, \dots , \, 0)^T = 
        \alpha_i$, 
        
        где единица в столбце координат вектора $e_i$ стоит на $i$-й позиции.
    \end{enumerate}
\end{proof}

\subsubsection{Задача об ортогональной проекции}

\begin{problem}
    Пусть $V$ - пространство со скалярным произведением, $U$ - подпространство $V$. Обозначим 
    размерность $V$ за $n$, размерность $U$ за $k$. Тогда сужение на $U$
    невырожденной функции $f(x, y)$, являющейся скалярным произведением в $V$, так же будет являться
    скалярным произведением и в $U$. \\
    Пространство $V$ будет представляться как $V = U + U^{\perp}$. \\
    Дан базис в $U$, вектор $x \in V$. Требуется представить вектор $x$ в виде суммы его проекций 
    $\tilde{x} = \pr_U x$ и $\stackrel{o}{x} = \ort_U x$ на $U$ и $U^{\perp}$ соответственно.
\end{problem}

\begin{algorithm}~
    \begin{enumerate}
        \item Зафиксируем в $U$ ортонормированный базис $e_1, \dots e_k$, достроив его до базиса 
        $e_1, \dots e_n$ в $V$. Тогда:
        \begin{gather*}
            x = \sum_{i=1}^{k} \alpha_i e_i + \sum_{i=k+1}^{n}\alpha_i e_i.
        \end{gather*}
        При этом $\displaystyle\sum_{i=1}^{k} \alpha_i e_i \in U$, 
        $\displaystyle \sum_{i=k+1}^{n}\alpha_i e_i \in U^{\perp}$. Тогда 
        $\tilde{x} = \displaystyle\sum_{i=1}^{k}\alpha_i e_i$, откуда $\stackrel{o}{x} = x - \tilde{x}$.
        \item Зафиксируем в $U$ ортогональный базис $e_1, e_2, \dots e_k$, достроив его до 
        ортогонального базиса $e_1, \dots e_n$ в $V$. Тогда рассмотрим базис $e'$ такой, что
        $e'_i = \frac{e_i}{|e_i|}$, очевидно являющийся ортонормированным. Тогда 
        \begin{gather*}
            \tilde{x} = \displaystyle\sum_{i=1}^{k} (x, e'_i)e'_i = \displaystyle\sum_{i=1}^{k} (x, e'_i) 
            \frac{e_i}{|e_i|} = \displaystyle\sum_{i=1}^{k} \frac{(x, e_i)}{(e_i, e_i)} e_i = 
            \pr_e x = \frac{(x, e)}{(e, e)} e.
        \end{gather*}
        \item Зафиксируем произвольный базис $e_1, e_2, \dots e_k$ в $U$, достроив его до базиса
        $e_1, \dots e_n$ в $V$.
        Тогда необходимые нам векторы выражаются как $\tilde{x} = \displaystyle\sum_{i=1}^{k} \lambda_i e_i$ и $\stackrel{o}{x} = x - \tilde{x} = 
        x - \displaystyle\sum_{i=1}^{k} \lambda_i e_i \perp e_1, \, \dots , \, e_k$.
        
        Чтобы получить коэффициенты $\lambda_i$ запишем следующую систему:
        \begin{eqnarray*}
            \begin{cases*}
                (x - \displaystyle\sum_{i=1}^{k} \lambda_i e_i, e_1) = 0,
                \\
                (x - \displaystyle\sum_{i=1}^{k} \lambda_i e_i, e_2) = 0,
                \\
                \dots
                \\
                (x - \displaystyle\sum_{i=1}^{k} \lambda_i e_i, e_k) = 0.
            \end{cases*} \Leftrightarrow \begin{cases*}
                (e_1, e_1) \lambda_1 + (e_2, e_1) \lambda_2 + \, ... \, + (e_k, e_1) = (x, e_1),
                \\
                (e_1, e_2) \lambda_1 + (e_2, e_2) \lambda_2 + \, ... \, + (e_k, e_2) = (x, e_2),
                \\
                \dots
                \\
                (e_1, e_k) \lambda_1 + (e_2, e_k) \lambda_2 + \, ... \, + (e_k, e_k) = (x, e_k).
            \end{cases*}
        \end{eqnarray*}
        Матрица системы является сужением $\Gamma$ на $U$, а значит $|\Gamma \vert_{U} (e_1, \, \dots, \, e_k)| > 0$.
        Таким образом по теореме Крамера система имеет единственное решение.
    \end{enumerate}
\end{algorithm}

\begin{note}
    Есть и другой способ решения данной задачи, основанный на процедуре ортогонализации Грама-Шмидта.
\end{note}

\begin{definition}
    Процедурой ортогонализации называется любой алгоритм, позволяющий по произвольному базису 
    в пространстве $V$ построить ортогональный базис.
\end{definition}

\begin{theorem}[Грама-Шмидта]
    Пусть $e$ - произвольный базис в пространстве $V$ со скалярным произведением. Тогда существует 
    базис $f$ в пространстве $V$, такой что:
    \begin{enumerate}
        \item Для всех $k = 1, \dots n$ верно $\langle e_1, \, \dots, \, e_k \rangle = \langle f_1, \dots f_k \rangle$
        \item Матрица перехода $S = S_{e \to f}$ унипотентна.
    \end{enumerate}
\end{theorem}

\begin{proof}
    Докажем по индукции по $n = \dim V$.
    \begin{enumerate}
        \item База $n = 1$ - $f_1 = e_1$, $S = (1)$.
        \item Предположим, для подпространства из $k-1$ вектора построено. Покажем, что для подпространства 
        образованного $e_1, \dots e_k$ можно достроить построенный базис. Выберем 
        $f_k = \ort e_k \langle f_1, \dots f_{k-1} \rangle$. Тогда:
        \begin{enumerate}
            \item По построению $f_k$ получаем $f_k \perp \langle f_1, \dots f_{k-1} \rangle$, 
            откуда $f_1, \dots f_k$ - ортогональная система.
            \item Пусть $f_k = 0$. Тогда $e_k \in \langle f_1, \dots f_{k_1} \rangle = \langle e_1, \dots e_{k_1} \rangle$, 
            а значит базис $e$ - линейно зависимый, что приводит к противоречию. Таким образом $f_k \neq 0$.
            \item Покажем, что $\langle f_1, \dots f_{k} \rangle = \langle e_1, \dots e_{k} \rangle$.
            Действительно, вектор $f_k$ выражается как: 
            $$f_k = e_k - \tilde{e_k} \in \langle f_1, \, \dots ,\, f_{k-1}\rangle 
            = e_k - \displaystyle\sum_{i=1}^{k-1} \lambda_i f_i.$$ 
            Тогда $e_k = f_k + \displaystyle\sum_{i=1}^{k-1} \lambda_i f_i \in \langle f_1, \dots f_{k} \rangle$.
            
            При этом по предположению индукции $\langle f_1, \dots f_{k-1} \rangle = 
            \langle e_1, \dots e_{k-1} \rangle$. \\ Таким образом 
            $\langle f_1, \dots f_{k} \rangle \subseteq 
            \langle e_1, \dots e_{k} \rangle$. Покажем включение в обратную сторону.

            Как было показано выше, $f_k = e_k - \tilde{e_k}$, что значит, что $f_k \in \langle e_1, 
            \, \dots, \, e_k \rangle$. Так же по предположению индукции $\langle f_1, \dots f_{k-1} \rangle = 
            \langle e_1, \dots e_{k-1} \rangle$, откуда $\langle e_1, \dots e_{k} \rangle \subseteq 
            \langle f_1, \dots f_{k} \rangle$.
        \end{enumerate}
        Таким образом построенный базис будет иметь вид $f_1 = e_1$, $f_k = e_k - \tilde{e_k} = 
        e_k - \displaystyle\sum_{i=1}^{k-1} \frac{(e_k, f_i)}{(f_i, f_i)}f_i$. Осталось показать, что 
        матрица перехода является унипотентной. Используя выражения $f_k$ через $e_1, \dots e_k$ запишем 
        матрицу $S_{e \to f}$:
        \begin{gather*}
            S_{e \to f} = \begin{pmatrix}
            1      & -\alpha & \dots  & -\beta_1   \\
            0      & 1       & \dots  & -\beta_2   \\
            \vdots & \vdots  & \ddots & \vdots     \\
            0      & 0       & \dots  & 1
            \end{pmatrix}
        \end{gather*}
        Где коэффициент $\alpha$ равен $\alpha = \frac{(e_2, f_1)}{(f_1, f_1)}$, коэффициенты $\beta$
        равны $\beta_i = \frac{(e_k, f_i)}{(f_i, f_i)}$.
    \end{enumerate}
\end{proof}

\begin{note}
    Процедура ортогонализации Грама-Шмидта дает нулевой вектор на некотором шаге тогда и только тогда, 
    когда система $e_1, \dots e_n$ линейно зависима, что позволяет использовать её для проверки линейной
    независимости системы.
\end{note}

\begin{exercise}
    Пусть по базису $(a_1, \dots a_k)$ при помощи процедуры Грама-Шмидта был построен базис $(f_1, \dots f_k)$.
    Доказать, что $|\Gamma(f_1, \dots f_k)| = |\Gamma(a_1, \dots a_k)|$.
\end{exercise}

\begin{corollary}
    $0 \leq |\Gamma(a_1, \dots a_k)| \leq |a_1|^2 \cdot |a_2|^2 \cdot \dots |a_k|^2$.
\end{corollary}

\begin{corollary}
    Если $U \leq V$, $e_1, \dots e_k$  - ортогональный базис в $U$, то существует ортогональный 
    базис $e_1, \dots e_k, e_{k+1}, \dots e_n$ в $V$.
\end{corollary}

\begin{idea}
    Дополним $e_1, \dots e_k$ до какого-нибудь базиса в $V$, после чего ортогонализируем его. Очевидно,
    метод Грама-Шмидта не испортит первые $k$ векторов.
\end{idea}

\subsubsection{Ортогональные и унитарные матрицы}

\begin{definition}
    Матрица $A \in M_n(\R)$ называется ортогональной, если $A^T A = E$, откуда так же $A A^T = E$.
\end{definition}

\begin{example}
    Матрица поворота $A = \begin{pmatrix}
    \cos \phi & - \sin \phi    \\
    \sin \phi &   \cos \phi    \\
    \end{pmatrix}$
\end{example}

\begin{definition}
    Матрица $A \in M_n(\Cm)$ называется унитарной, если $\overline{A^T} A = E = A \overline{A^T}$.
\end{definition}
 
\begin{proposition}
    Пусть $V$ - пространство со скалярным произведением, $e$ - ортонормированный базис в $V$, 
    $f$ - произвольный базис в $V$. Тогда матрица перехода $S = S_{e \to f}$ является 
    ортогональной тогда и только тогда, когда $f$ - ортонормированный базис.
\end{proposition}

\begin{proof}
    Пусть $f(x, y)$ имеет матрицу $\Gamma$. Тогда так как $e$ - ортонормированный базис, его 
    матрица Грама является единичной $\Gamma(e) = E$.

    Тогда $\Gamma(f) = S^T \cdot \Gamma(e) \cdot \overline{S} = S^T \cdot \overline{S}$. Верна следующая цепочка эквивалентностей:

    $f$ - ортонормированный базис $\Leftrightarrow$ $\Gamma(f) = E$ $\Leftrightarrow$ $S^T \cdot \overline{S} = E$ 
    $\Leftrightarrow \overline{S^T} \cdot S = E$.
\end{proof}

\begin{corollary}~
    \begin{enumerate}
        \item
        Множество всех ортогональных матриц в $GL_n(\R)$ является подгруппой и называется ортогональной
        подгруппой, используется обозначение $O_n(\R)$.
        \item Множество всех унитарных матриц в $GL_n(\R)$ является подгруппой и называется 
        унитарной подгруппой, используется обозначение $U_n(\R)$.
    \end{enumerate}
\end{corollary}

\begin{proof}~

    Пусть $A, B$ - унитарные матрицы, $A = S_{e \to f}$, $B = S_{f \to g}$.
    Тогда $A \cdot B = S_{e \to g}$, $A^{-1} = S_{f \to e}$. Таким образом $A\cdot B$ и $A^{-1}$ 
    так же являются унитарными, а значит $U_n(\R)$ - группа.
\end{proof}