% 22.03.23 Оля

\subsection{Приложения Жордановой нормальной формы}

\subsubsection{Вычисление минимального многочлена}

\begin{reminder}
    Пусть $\phi: V \to V$. Многочлен $P \in F[x]$ называется аннулирующим для $\phi$, 
    если $P(\phi) = 0$.
\end{reminder}

\begin{reminder}
    Аннулирующий многочлен минимальной степени называется минимальным многочленом $\mu(\phi)$.
\end{reminder}

\begin{reminder}
    Если $P$ -- аннулирующий для $\phi$, $\mu$ -- минимальный многочлен для $\phi$, то $\mu \vert P$.
\end{reminder}

\begin{proposition}
    \label{pr7.1}
    Пусть пространство $V$ раскладывается в прямую сумму инвариантных относительно оператора $\phi$ 
    подпространтсв: $V = V_1 \oplus V_2 \oplus \dots \oplus V_k$. Введем обозначение 
    $\psi_i = \phi \vert_{V_i}$. Тогда $\mu_{\psi_i} \vert \mu_{\phi}$.
\end{proposition}

\begin{proof}
    По определению $\mu_{\phi} (\phi) = 0$, 
    тогда $\mu_{\phi} (\phi) |_{V_i} = \mu_{\phi} (\psi_i) = 0$. Значит $\mu_{\phi}$ -- 
    аннулирующий для $\phi$ и $\mu_{\psi_i} \vert \mu_{\phi}$ для всех $i$ по теореме \ref{th4.5}. 
\end{proof}

\begin{corollary}
    В условиях утверждения \ref*{pr7.1} для аннулирующего многочлена $\mu_{\phi}$ оператора $\phi$ 
    верно $\mu_{\phi} = \lf(\mu_{\psi_1}, \mu_{\psi_2}, \dots, \mu_{\psi_k})$.
\end{corollary}


\begin{proof}
    Многочлен $\mu$ минимальной степени, удовлетворяющий условию $\forall i \; \mu_{\psi_i} \vert \mu$ 
    является наименьшим общим кратным многочленов $\mu_{\psi_1}, \mu_{\psi_2}, \dots, \mu_{\psi_k}$ 
    по определению.

    Покажем, что $\mu(\phi) = 0$: так как для всех $i$ верно, что $\mu(\psi_i) \vert \mu$ и 
    $\mu_{\psi_i} (\psi_i) = 0$, то так же верно $\mu(\phi \vert_{V_i}) = \mu(\psi_i) = 0$ для всех $i$. 
    По условию $V = V_1 \oplus V_2 \oplus \dots \oplus V_k$, а значит $\phi$ полностью определяется 
    своими сужениями $\phi \vert_{V_i}$, аннулирующимися многочленом $\mu$. 
    Отсюда следует, что и $\phi$ аннулируется многочленом $\mu$.

    Так как $\mu$ -- аннулятор $\phi$, по теореме $\ref{th4.5}$ $\mu_{\phi} \vert \mu$. При этом 
    $\mu_{\phi}$ является общим кратным многочленов $\mu_{\psi_1}, \mu_{\psi_2}, \dots, \mu_{\psi_k}$,
    а значит кратен их НОК, то есть $\mu \vert \mu_{\phi}$. Отсюда следует, что $\mu \sim \mu_{\phi}$,
    а значит $\mu_{\phi} = \lf(\mu_{\psi_1}, \mu_{\psi_2}, \dots, \mu_{\psi_k})$.
\end{proof}

\begin{proposition}
    Пусть матрица отображения имеет вид Жордановой клетки: $J = J_k(\lambda)$. 
    Тогда его минимальный многочлен имеет вид $\mu_j(x) = (x - \lambda)^k$.
\end{proposition}

\begin{proof}
    Так как матрица отображения имеет вид Жордановой клетки, 
    его характеристический многочлен $\chi_J(x)$ представляется как:
    $$\chi_J(x) = (\lambda - x)^k = (-1)^k (x - \lambda)^k \sim (x - \lambda)^k.$$
    По теореме \ref{th4.4} Гамильтона-Кэли $\mu_J \vert \chi_J$, значит $\mu_J (x) = (x - \lambda)^t$, $t \leq k$.
    В данном случае Жорданова диаграмма преобразования имеет вид столбца, а значит 
    $(J - \lambda \epsilon)^t f_{1k} \neq 0$. Если $t < k$, то $(J - \lambda \epsilon)^t \neq 0$, 
    что приводит к противоречию с определением минимального многочлена $\mu_J(J) = 0$.  
    Таким образом $t$ не может быть меньше $k$, а значит $t = k$. 
\end{proof}

\begin{proposition}
    Пусть $\phi: V \to V$ -- линейно-факторизуемый оператор. Тогда минимальный многочлен 
    представляется в виде произведения: 
    \begin{eqnarray*}
        \mu(\phi) = \prod_{i = 1}^{k} (x - \lambda_i)^{l_i},
    \end{eqnarray*}
    где $\lambda_1, \lambda_2, \dots, \lambda_k$ -- попарно-различные собственные значения, 
    $l_1, l_2, \dots, l_k$ -- максимальные порядки клеток, относящихся к соответствующим 
    собственным значениям.
\end{proposition}

\begin{proof}
    Благодаря линейной факторизуемости $V$ раскладывается в прямую сумму корневых 
    подпространств: $V = V^{\lambda_1} \oplus V^{\lambda_2} \oplus \dots V^{\lambda_k}$.
    При этом каждое из них так же можно предстваить в виде суммы: 
    $$V^{\lambda_i} = \displaystyle\sum_{j = 1}^{geom(\lambda_i)} V_{ij},$$
    где все $V_{ij}$ соответствуют оператору $(\phi - \lambda_i \epsilon)$. 
    Положим $\dim V_{ij} = m_{ij}$. 
    Пусть в циклическом базисе $V_{ij}$ верно $\phi \leftrightarrow J_{m_{ij}} (\lambda_i)$,
    тогда по утверждению \ref{pr7.1}: 
    $$\mu_i = \lf((x - \lambda_i)^{\alpha_1}, \dots, 
    (x - \lambda_i)^{\alpha_k}) = (x - \lambda_i)^{\max(\alpha_s)} =  (x - \lambda_i)^{l_1},$$ 
    где $l_i = \max(\alpha_1, \dots, \alpha_k)$.

    По следствию из утверждения \ref{pr7.1}: $$\mu_{\phi} = \lf((x - \lambda_1)^{l_1}, \dots, 
    (x - \lambda_k)^{l_k}) =  \prod_{i = 1}^{k} (x - \lambda_i)^{l_i}.$$
\end{proof}

\begin{corollary}
    Пусть $\phi: V \to V$ -- линейно-факторизуемый оператор. Тогда $\phi$ диагонализируема тогда и 
    только тогда, когда $\mu_{\phi}(x)$ не имеет кратных корней.
\end{corollary}

\begin{proof}~
    \begin{enumerate}
        \item Необходимость. Пусть $\phi$ -- диагонализируем. Тогда $\forall i \: l_i = 1$, а значит
        $\mu_{\phi} = \prod_{i=1}^{k} (x - \lambda_i)$. Тогда все корни простые 
        (имеют кратность равную 1).
        \item Достаточность. Пусть $\mu_{\phi} = \prod_{i=1}^{k} (x - \lambda_i)$. 
        Тогда для всех $i$ верно $l_i = 1$. Таким образом нет Жордановых клеток порядка больше 1, 
        а значит оператор $\phi$ диагонализируем.\\
        Если бы было хотя бы две клетки, отвечающие одному значению, в НОК было бы два члена 
        соответствующих этому значению, и степень $\mu_i$ была бы меньше, чем степень многочлена.
    \end{enumerate}
\end{proof}

\subsubsection{Вычисление многочлена от линейного оператора}

\begin{agreement}
    До конца раздела будем считать, что оператор $\phi$ линейно факториуем над $F$.
\end{agreement}

\begin{proposition}
    Для дифференцируемого $k-1$ раз многочлена верно: $$f^{(k-1)}(x) = C_n^{k-1} (k-1)! x^{n-k+1}.$$
\end{proposition}

\begin{proof}
    $$f^{(k-1)}(x) = n(n-1)\dots(n-k+2) x^{n - k + 1} = \frac{n!}{(n-k+1)! (k+1)!} (k+1)! x^{n-k+1} 
    = C_n^{k-1} (k-1)! x^{n-k+1}.$$
\end{proof}

\begin{algorithm}[Вычисление многочлена от линейного оператора]~

Рассмотрим вычисление многочлена от оператора $\phi$ на примере многочлена $f(x) = x^k$. 
В случае, если мы научимся возводить матрицу преобразования $A$ в любую степень, более сложные 
многочлены можно будет представить в виде суммы необходимых нам степеней матрицы $A$.

Для матрицы, приведенной к Жордановой нормальной форме, возведение в степень $k$ поризводится 
следующим образом:

\begin{eqnarray*}
    A^{k} = \begin{pmatrix}
        A_1    & 0      & \dots  & 0     \\
        0      & A_2    & \dots  & 0     \\
        \vdots & \vdots & \ddots & \vdots\\
        0      & 0      & \dots  & A_n
    \end{pmatrix}^k = \begin{pmatrix}
        A_1^k  & 0      & \dots  & 0     \\
        0      & A_2^k  & \dots  & 0     \\
        \vdots & \vdots & \ddots & \vdots\\
        0      & 0      & \dots  & A_n^k
    \end{pmatrix}
\end{eqnarray*}

Таким образом для возведения матрицы в степень достаточно научиться возводить в степень 
Жордановы клетки, из которых состоит наша матрица.

    Рассмотрим Жорданову клетку $J = J_k(\lambda)$. Её можно представить в следующем виде:
    \begin{eqnarray*}
        J_k(\lambda) = \begin{pmatrix}
            \lambda & 1       & 0       & \dots  & 0       & 0          \\
            0       & \lambda & 1       & \dots  & 0       & 0          \\
            0       & 0       & \lambda & \dots  & 0       & 0          \\
            \vdots  & \vdots  & \vdots  & \ddots & \vdots  & \vdots     \\
            0       & 0       & 0       & \dots  & \lambda & 1          \\
            0       & 0       & 0       & \dots  & 0       & \lambda
        \end{pmatrix} = \lambda E + J_k(0) = \begin{pmatrix}
            \lambda & 0       & 0       & \dots  & 0       & 0          \\
            0       & \lambda & 0       & \dots  & 0       & 0          \\
            0       & 0       & \lambda & \dots  & 0       & 0          \\
            \vdots  & \vdots  & \vdots  & \ddots & \vdots  & \vdots     \\
            0       & 0       & 0       & \dots  & \lambda & 0          \\
            0       & 0       & 0       & \dots  & 0       & \lambda
        \end{pmatrix} + \begin{pmatrix}
            0       & 1       & 0       & \dots  & 0       & 0          \\
            0       & 0       & 1       & \dots  & 0       & 0          \\
            0       & 0       & 0       & \dots  & 0       & 0          \\
            \vdots  & \vdots  & \vdots  & \ddots & \vdots  & \vdots     \\
            0       & 0       & 0       & \dots  & 0       & 1          \\
            0       & 0       & 0       & \dots  & 0       & 0
        \end{pmatrix}
    \end{eqnarray*}
    Тогда возведение $J$ в степень можно представить как: 
    $$J_k(\lambda)^n = (\lambda E + J_k(0))^n = \displaystyle\sum_{s = 0}^{n} C_n^s 
    (\lambda E)^{n-s} J_k(0)^s.$$
    Возведение $\lambda E$ в $(n-s)$-ю степень очевидно дает $\lambda^{n-s} E$. Рассмотрим 
    возведение матрицы $J_k(0)$ в $s$-ю степень. Заметим, что с каждым следующим умножением на 
    $J_k(0)$ диагональ из единиц смещается вверх на одну строку, и $J_k(0)^{k-1} \neq 0$, 
    $J_k(0)^k = 0$:
    \begin{eqnarray*}
        J_k(0) = \begin{pmatrix}
            0       & 1       & 0       & 0       & \dots  & 0       & 0       & 0          \\
            0       & 0       & 1       & 0       & \dots  & 0       & 0       & 0          \\
            0       & 0       & 0       & 1       & \dots  & 0       & 0       & 0          \\
            0       & 0       & 0       & 0       & \dots  & 0       & 0       & 0          \\
            \vdots  & \vdots  & \vdots  & \vdots  & \ddots & \vdots  & \vdots  & \vdots     \\
            0       & 0       & 0       & 0       & \dots  & 0       & 1       & 0          \\
            0       & 0       & 0       & 0       & \dots  & 0       & 0       & 1          \\
            0       & 0       & 0       & 0       & \dots  & 0       & 0       & 0
        \end{pmatrix}, \; \;  J_k(0)^2 = \begin{pmatrix}
            0       & 0       & 1       & 0       & \dots  & 0       & 0       & 0          \\
            0       & 0       & 0       & 1       & \dots  & 0       & 0       & 0          \\
            0       & 0       & 0       & 0       & \dots  & 0       & 0       & 0          \\
            0       & 0       & 0       & 0       & \dots  & 0       & 0       & 0          \\
            \vdots  & \vdots  & \vdots  & \vdots  & \ddots & \vdots  & \vdots  & \vdots     \\
            0       & 0       & 0       & 0       & \dots  & 0       & 0       & 1          \\
            0       & 0       & 0       & 0       & \dots  & 0       & 0       & 0          \\
            0       & 0       & 0       & 0       & \dots  & 0       & 0       & 0
        \end{pmatrix}, \\ \\ J_k(0)^{k-1} = \begin{pmatrix}
            0       & 0       & 0       & 0       & \dots  & 0       & 0       & 1          \\
            0       & 0       & 0       & 0       & \dots  & 0       & 0       & 0          \\
            0       & 0       & 0       & 0       & \dots  & 0       & 0       & 0          \\
            0       & 0       & 0       & 0       & \dots  & 0       & 0       & 0          \\
            \vdots  & \vdots  & \vdots  & \vdots  & \ddots & \vdots  & \vdots  & \vdots     \\
            0       & 0       & 0       & 0       & \dots  & 0       & 0       & 0          \\
            0       & 0       & 0       & 0       & \dots  & 0       & 0       & 0          \\
            0       & 0       & 0       & 0       & \dots  & 0       & 0       & 0
        \end{pmatrix}, \; \;  J_k(0)^k = \begin{pmatrix}
            0       & 0       & 0       & 0       & \dots  & 0       & 0       & 0          \\
            0       & 0       & 0       & 0       & \dots  & 0       & 0       & 0          \\
            0       & 0       & 0       & 0       & \dots  & 0       & 0       & 0          \\
            0       & 0       & 0       & 0       & \dots  & 0       & 0       & 0          \\
            \vdots  & \vdots  & \vdots  & \vdots  & \ddots & \vdots  & \vdots  & \vdots     \\
            0       & 0       & 0       & 0       & \dots  & 0       & 0       & 0          \\
            0       & 0       & 0       & 0       & \dots  & 0       & 0       & 0          \\
            0       & 0       & 0       & 0       & \dots  & 0       & 0       & 0
        \end{pmatrix}.
    \end{eqnarray*}

    Для $s \geq k$ получим $J_k(0)^s = 0$. Таким образом все слагаемые с $s \geq k$ в сумме выше 
    нулевые и достаточно вычислять её до значения $s = k-1$. Подставляя получившиеся матрицы 
    в формулу получаем следующую матрицу:
    \begin{gather*}
        J_k(\lambda)^n = \displaystyle\sum_{s = 0}^{k-1} C_n^s 
        (\lambda E)^{n-s} J_k(0)^s = \\ = \begin{pmatrix}
            \lambda^n & C_n^1 \lambda^{n-1}  & C_n^2 \lambda^{n-2}  & \dots  & C_n^{k-1} \lambda^{n-k+1} \\
            0         & \lambda^n            & C_n^1 \lambda^{n-1}  & \dots  & C_n^{k-2} \lambda^{n-k+2} \\
            0         & 0                    & \lambda^n            & \dots  & C_n^{k-1} \lambda^{n-k+3} \\
            \vdots    & \vdots               & \vdots               & \ddots & \vdots                    \\
            0         & 0                    & 0                    & \dots  & \lambda^n
        \end{pmatrix} = \begin{pmatrix}
            f(\lambda)  & \frac{f'(\lambda)}{1!} & \frac{f''(\lambda)}{2!}  & \dots  & \frac{f^{(k-1)}(\lambda)}{(k-1)!} \\
            0           & f(\lambda)             & \frac{f'(\lambda)}{1!}   & \dots  & \frac{f^{(k-2)}(\lambda)}{(k-2)!} \\
            0           & 0                      & f(\lambda)               & \dots  & \frac{f^{(k-3)}(\lambda)}{(k-3)!} \\
            \vdots      & \vdots                 & \vdots                   & \ddots & \vdots                            \\
            0           & 0                      & 0                        & \dots  & f(\lambda)
        \end{pmatrix}
    \end{gather*}

    Пусть S -- матрица перехода от начального базиса к жордановому. Тогда $J = S^{-1} A S$, 
    $A = S J S^{-1}$. Возведение в степень матрицы $A$ в начальном базисе будет осуществляться 
    следующим образом: $$A^n = (S J S^{-1}) (S J S^{-1} \dots (S J S^{-1})) = S J^n S^{-1}.$$
    
    Матрица для более сложного многочлена является суммой матриц соответствующих слагаемых.
\end{algorithm}

\subsubsection{Примечание Жордановой нормальной формы к вычислению аналитической функции от линейных операторов}

\begin{definition}
    Функция называется аналитической, если она представляется сходящимся степенным рядом.
\end{definition}

\begin{agreement}
    До конца раздела $\phi: V \to V$, $V$ -- вещественное или комплексное.
\end{agreement}

\begin{definition}
    Функция $||\cdot||: V \to \R$ называется нормой если 
    \begin{enumerate}
        \item $||x|| > 0$, если $x \neq 0$,
        \item $|| \lambda x|| = |\lambda| \cdot ||x||$,
        \item $||x + y|| \leq ||x|| + ||y||$.
    \end{enumerate}
\end{definition}

\begin{example}~
    \begin{enumerate}
        \item $x$ -- координатный столбец, 
            $||x|| = \underset{i}{\max} |x_i|$ -- максимум из модулей координат.
        \item $x$ -- координатный столбец, $||x|| = \sqrt{\displaystyle\sum_{i=1}^{n}x_i^2}$ 
            -- Евклидова норма.
        \item $x$ -- координатный столбец, $||x|| = \displaystyle\sum_{i=1}^{n}|x_i|$ 
            -- Манхеттенская норма.
    \end{enumerate}
\end{example}

\begin{definition}
    Последовательность векторов $\{x^m\}$ сходится по норме к $x_0$, 
    если $||x^m - x_0|| \to 0$ при $m \to +\infty$.
\end{definition}

\begin{definition}
    Ряд $\displaystyle\sum_{m=1}^{+\infty} x^m$ называется сходящимся, если он сходится по норме 
    $S^n = \displaystyle\sum_{m=1}^{n}x^m$.
\end{definition}

\begin{definition}
    Ряд $\displaystyle\sum_{m=1}^{+\infty} x^m$ называется абсолютно сходящимся, если сходится ряд 
    $\displaystyle\sum_{m=1}^{+\infty} ||x^m||$.
\end{definition}

\begin{proposition}
    Если ряд $\displaystyle\sum_{m=1}^{+\infty} a_m x^m$ сходится абсолютно, то он сходится, 
    и для сумм верно:
    $$||\displaystyle\sum_{m=1}^{+\infty} x^m|| \leq \displaystyle\sum_{m=1}^{+\infty} ||x^m||.$$
\end{proposition}

\begin{proposition}
    Если ряд $\displaystyle\sum_{m=1}^{+\infty} a_m x^m$ сходится абсолютно и $\phi: \N \to \N$, то 
    ряд $\displaystyle\sum_{m=1}^{+\infty} a_{\phi(m)} x^{\phi(m)}$ сходится и для этих двух рядов 
    верно: $$||\displaystyle\sum_{m=1}^{+\infty} a_{\phi(m)} x^{\phi(m)}|| = 
    ||\displaystyle\sum_{m=1}^{+\infty} a_m x^m||$$
\end{proposition}

\begin{definition}
    \label{def6.8}
    Пусть $\phi: V \to V$, $V$ конечномерно над $\R$ или $\Cm$. Тогда:
    $$||\phi|| \overset{\text{def}}{=} \underset{x \neq 0}{\max} \frac{||\phi(x)||}{||x||} = 
    \underset{||x|| = 1}{\max} \frac{||\phi(x)||}{||x||} = \underset{||x|| = 1}{\max} ||\phi(x)||.$$
\end{definition}

\begin{note}
    Если $\lambda$ -- собственное значение оператора $\phi$, то $||\phi|| \geq \lambda$.
\end{note}

\begin{proposition}[о свойствах нормы оператора]~
    \begin{enumerate}
        \item Определение \ref{def6.8} опеределяет норму в $\mathcal{L}(V)$.
        \item $||\phi(x)|| \leq ||\phi|| \cdot ||x||$.
        \item $||\phi \cdot \psi|| \leq ||\phi|| \cdot ||\psi||$.
    \end{enumerate}
\end{proposition}

\begin{proof}~
    \begin{enumerate}
        \item Докажем неравенство треугольника для нормы:
        \begin{multline*}
            ||\phi + \psi|| \overset{\text{def}}{=} \underset{x \neq 0}{\max} 
            \frac{||(\phi + \psi)(x)||}{||x||} \leq \underset{x \neq 0}{\max} 
            \frac{||\phi(x)|| + ||\psi(x)||}{||x||} \leq \\ \leq \underset{x \neq 0}{\max} 
            \frac{||\phi(x)||}{||x||} + \underset{x \neq 0}{\max} 
            \frac{||\psi(x)||}{||x||} = ||\phi|| + ||\psi||
        \end{multline*} 

        \item Докажем непосредственной проверкой:
        \begin{eqnarray*}
            ||\phi(x)|| = \frac{||\phi(x)||}{||x||} ||x|| \leq  \underset{x \neq 0}{\max} 
            \frac{||\phi(x)||}{||x||} ||x|| = ||\phi|| \cdot ||x||
        \end{eqnarray*}
        \item Докажем непосредственной проверкой:
        \begin{multline*}
            ||\phi \cdot \psi|| = \underset{x \neq 0}{\max} \frac{||\phi \cdot \psi(x)||}{||x||} = 
            \underset{\psi(x) \neq 0}{\max} \frac{||\phi(x)||}{||x||} =
            \underset{\psi(x) \neq 0}{\max} \frac{||\phi \cdot \psi(x)||}{||\psi(x)||} \cdot 
            \frac{||\psi(x)||}{||x||} \leq \\ \leq \underset{\psi(x) \neq 0}{\max} 
            \frac{||\phi \cdot \psi(x)||}{||\psi(x)||} \cdot \underset{\psi(x) \neq 0}{\max} 
            \frac{||\psi(x)||}{||x||} \leq ||\phi|| \cdot ||\psi||
        \end{multline*}
    \end{enumerate}
\end{proof}

\begin{theorem}
    Пусть ряд $f(t) = \displaystyle\sum_{m=1}^{+\infty} a_m t^m$ сходится при $|t| < R$.
    Тогда ряд $\displaystyle\sum_{m=1}^{+\infty} a_m \phi^m$ сходится абсолютно для любого оператора 
    $\phi: ||\phi|| = R_0 < R$. Более того, $f(\phi) = \displaystyle\sum_{m=1}^{+\infty} a_m \phi^m$ 
    -- задает линейный оператор в $V$.
\end{theorem}

\begin{proof}
    $\forall x \in V$ докажем, что ряд $\displaystyle\sum_{m=1}^{+\infty} a_m \phi^m(x)$ 
    сходится абсолютно:
    \begin{multline*}
        \displaystyle\sum_{m} |a_m| \cdot ||\phi^m(x)|| \leq \displaystyle\sum_{m} |a_m| \cdot 
        ||\phi^m|| \cdot ||x|| \leq \\ \leq ||x|| \displaystyle\sum_{m} |a_m| \cdot ||\phi^m|| = 
        ||x|| \displaystyle\sum_{m} |a_m| R_0^m \text{\: -- \, сходится при } R_0 < R.   
    \end{multline*}
    Ряд $f(t) = \displaystyle\sum_{m} a_m t^m$ сходится при $|t| < R$, 
    а значит $\displaystyle\sum_{m} |a_m| |t|^m$ сходится при $|t| < R$ по теореме Абеля.
\end{proof}

\begin{note}
    \begin{gather*}
        exp(\phi) = \epsilon + \frac{\phi}{1!} + \dots + \frac{\phi^n}{n!} + \dots, \, R = +\infty \\
        sin(\phi) = \phi - \frac{\phi^3}{3!} + \dots + (-1)^n \frac{\phi^{2n+1}}{(2n+1)!} + \dots, 
        \, R = +\infty \\
        cos(\phi) = \epsilon - \frac{\phi^2}{2!} + \frac{\phi^4}{4!} - \dots + 
        (-1)^n \frac{\phi^{2n}}{(2n)!} + \dots, \, R = +\infty
    \end{gather*}
\end{note}

\section{Линейные рекурренты}

\begin{definition}
    Будем рассматривать последовательности $(a_0, a_1, \dots)$, $a_i \in F$. Множество всех таких 
    последовательностей будем обозначать $F^{\infty}$.
\end{definition}

\begin{definition}
    \label{def7.1}
    Зафиксируем многочлен $p(x) \in F[x]$ степени $S$, 
    $p(x) = x^s + p_{s-1} x^{s-1} + \dots + p_1 x + p_0$.
    Линейным рекуррентным соотношением с характеристическим многочленом $p(x)$ 
    назывется последовательность $a_n$ такая что $\forall n \in \N \cup \{0\}$ верно:
    \begin{eqnarray*}
        a_{n+s} + p_{s-1} a_{n + s - 1} + \dots + p_1 a_{n+1} + p_0 a_n = 0, \: p_0 \neq 0
    \end{eqnarray*}
    Рекуррентное соотношение выражает $a_{n + s}$ через $s$ предыдущих членов.
    $V_p$ - множество всех последовательностей, удовлетворяющих рекуррентному соттношению выше.
\end{definition}

\begin{proposition}
    $V_p$ - линейное пространство над $F$ и $\dim V_p = s$.
\end{proposition}

\begin{proof}
    Если $\{a_n\}$ и $\{ b_n \}$ удовлетворяют условию определения \ref{def7.1}, 
    то и $\{a_n + b_n\}$ удовлетворяют этому условию.
    Базис в $V_p$:
    \begin{gather*}
        e_0 = (\underbrace{1,\, 0,\, 0,\, \dots,\, 0,}_{S}\, -p_0,\, \dots) \\
        e_1 = (\underbrace{0,\, 1,\, 0,\, \dots,\, 0,}_{S}\, -p_1, \dots) \\
        \dots \\
        e_{s-1} = (\underbrace{0,\, 0,\, 0,\, \dots,\, 1,}_{S}\, -p_{s-1}, \dots)
    \end{gather*}
\end{proof}

\begin{proposition}
    \label{pr7.2}
    Рассмотрим оператор $\phi: F^{\infty} \to F^{\infty}$, такой что 
    $\phi(a_0, a_1, \dots, a_n, \dots) = (a_1, a_2, \dots, a_{n-1}, \dots)$. 
    Тогда $V_p = \ker p(\phi)$.
\end{proposition}

\begin{proof}
    По определению ядра отображения последовательность $\{b_n\}$ лежит в $\ker p(\phi)$ тогда 
    и только тогда, когда верно $p(\phi) (b_n) = (0) \in F^{+\infty}$. При этом имеет место 
    следующая равносильность:
    \begin{eqnarray*}
        (\phi^s + p_{s-1} \phi^{s-1} + \dots p_1 \phi + p_0 \epsilon) (b_n) = (0) \Leftrightarrow
        b_{n+s} + p_{s-1} b_{n+s-1} + \dots p_1 b_{n+1} + p_0 b_n = (0)
    \end{eqnarray*}
    Второе равенство эквивалентно тому, что $\{b_n\}$ лежит в $V_p$, а значит верно вложение $V_p$ и 
    $\ker p(\phi)$ друг в друга в обе стороны.
\end{proof}

\begin{note}
    Оператор $\phi$ называется оператором левого сдвига. $V_p$ инварианто относительно $\phi$.
\end{note}

\begin{corollary}
    Пусть $\psi_p = \phi \vert_{V_p}$. Тогда $p(\psi_p) = 0$.
\end{corollary}

\begin{proof}
    $p(\phi) \vert_{V_p} = 0$ так как $V_p = \ker p(\phi)$.
\end{proof}

\begin{proposition}
    $\mu_{\psi_p} (x) = p(x)$.
\end{proposition}

\begin{proof}
    Пусть $a_n \in V_p$, тогда $p(\phi) (a_n) = (0)$. По следствию из утверждения \ref{pr7.2} 
    для сужения $\psi_p = p(\phi) \vert_{V_p}$ так же верно $\psi_p (a_n) = (0)$, а
    значит $p(\psi_p) (a_n) = 0$. Таким образом $p$ - аннулирующий многочлен для $\psi_p$ и по 
    теореме \ref{th4.5} $\mu \vert p$, где $\mu = \mu_{\psi_p}$.
    По определению минимального многочлена $\mu(\psi_p) = 0$, тогда и $\mu(\phi) \vert_{V_p} = 0$.

    Отсюда следует, что $V_p$ вложено в $\ker \mu(\phi) = V_{\mu}$ (равенство верно по утверждению 
    \ref{pr7.2}). Из вложенности $V_p \subseteq V_{\mu}$ и кратности $\mu \vert p$ получаем 
    равенство степеней многочленов $\deg p = \deg \mu$, откуда следует их ассоциированность.
\end{proof}

\begin{definition}
    Пусть $p(x) = x^s + p_{s-1} x^{s-1} + \dots + p_1 x + p_0 \in F[x]$, $p_0 \neq 0$.\\
    Сопутствующей матрицей для многочлена $p(x)$ называется матрица размера $s \times s$ вида:
    \begin{gather*}
        \begin{pmatrix}
        0      & 1      & 0      & \dots  & 0        & 0          \\
        0      & 0      & 1      & \dots  & 0        & 0          \\
        0      & 0      & 0      & \dots  & 0        & 0          \\
        \vdots & \vdots & \vdots & \ddots & \vdots   & \vdots     \\
        0      & 0      & 0      & \dots  & 0        & 1          \\
        -p_0   & -p_1   & -p_2   & \dots  & -p_{s-2} & -p_{s-1}
        \end{pmatrix}
    \end{gather*}
\end{definition}

\begin{proposition}
    \label{pr7.4}
    Пусть $\psi_p = \phi_p \vert_{V_p}$. В базисе $(e_0, e_1, \dots, e_{s-1})$ из стандартных 
    последовательностей оператор $\psi_p$ имеет в точности сопутвующую матрицу $A_p$.
\end{proposition}

\begin{proof}
    \begin{flalign*}
        &\psi_p(e_0) = (0, 0, \dots, 0, 0,\, -p_0, \dots) = -p_0 e_{s-1} \\
        &\psi_p(e_1) = (1, 0, \dots, 1, 0,\, -p_1, \dots) = e_0 - p_1 e_{s-1} \\
        &\dots \\
        &\psi_p(e_i) = (0, 0, \dots 1, 0, \dots, 0,\, -p_i, \dots) = e_{i-1} - p_i e_{s-1}
    \end{flalign*}
    При этом для $e_i$ единица стоит на $i-1$ позиции, $-p_i$ всегда стоит на $s$-й позиции.
\end{proof}

\begin{proposition}
    $\chi_{\psi_p} (x) = \chi_{A_p}(x) = (-1)^s p(x)$.
\end{proposition}

\begin{proof}
    Из утверждения \ref{pr7.4} следует $\chi_{\phi_p}(x) = \chi_{A_p}(x) = (-1)^s p(x)$.
    Докажем наше утверждение по индукции:
    \begin{enumerate}
        \item База $s = 2$:
        \begin{gather*}
            \begin{vmatrix}
                -x   & 1    \\
                -p_0 & x - p_1
            \end{vmatrix} = x^2 + p_1 x + p_0 \text{ -- верно.}
        \end{gather*} 
        \item Пусть для матрицы размера $s$ минор размера $(s-1) \times (s-1)$ в правом нижнем углу 
        имеет определитель равный: $$M_{2\dots s}^{2\dots s} = (-1)^{s-1}(x^{s-1} + p_{s-1}x^{s-2} + 
        \dots p_2 x + p_1).$$
        Покажем, что переход индукции верен. Для этого вычислим определитель матрицы $M_{A_p}$, 
        разложив его по верхнему столбцу:

        \begin{gather*}
            \chi_{A_p} = \begin{vmatrix}
                -x     & 1      & 0      & \dots  & 0      & 0          \\
                0      & -x     & 1      & \dots  & 0      & 0          \\
                0      & 0      & -x     & \dots  & 0      & 0          \\
                \vdots & \vdots & \vdots & \ddots & \vdots & \vdots\\
                0      & 0      & 0      & \dots  & -x     & 1          \\
                0      & 0      & 0      & \dots  & 0      & -x
            \end{vmatrix} = \\ = -x (-1)^{s-1} (x^{s-1} + p_{s-1}x^{s-2} + \dots p_2 x + p_1)
            + (-p_s)(-1)^{s-1} = (-1)^s p_s(x).
        \end{gather*}
        \end{enumerate}
    \end{proof}