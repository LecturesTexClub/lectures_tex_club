% 29.03.23

\begin{theorem}[Основная теорема о линейных рекуррентах]
    Пусть $V_p$ - пространство линейных рекуррент, относящихся к $p(x)$ и пусть $p(x)$ 
    раскладывается на линейные множители: $p(x) = \displaystyle\prod_{i=1}^{k}(x -\lambda_i)^{l_i}$,
    $\lambda_i \in F$ - попарно различные.
    Тогда для любой $\{a_n\}_{n=0}^{\infty} \in V_p$ справедливо представление:
    \begin{gather*}
        a_n = \sum_{i=1}^{k}\sum_{s=1}^{l_i} c_{is} \cdot C_n^{s - 1} \lambda_i^{n + 1 - s}
    \end{gather*}
\end{theorem}

\begin{proof}
    Ранее было доказано, что $\mu_{\psi_p} ~ \chi_{\psi_p}$. Теперь наша цель разложить пространство 
    в прямую сумму корневых подпространств и найти для каждого циклическое подпространство 
    $\langle b_1^{(i)}, \dots b_{l_i}^{(i)} \rangle$ такое, что:
    \begin{gather*}
        (\phi - \lambda_i \epsilon)b_1^{(i)} = 0, \\
        (\phi - \lambda_i \epsilon) b_{s}^{(i)} = b_{s-1}^{(i)}.
    \end{gather*}
    Заметим, что если $b_s^{(i)}$ построены, то они дают Жорданов базис в $V_p$. 
    При этом $l_1 + l_2 +\dots l_k = s = \dim V_p$. Получаем:
    \begin{gather*}
        \prod_{i=1}^{k}(\phi - \lambda_i \epsilon)^{l_i} (b_s^{(i)}) = 0 \Leftrightarrow p(\phi) (b_s^{(i)}) = 0 \Leftrightarrow b_s^{(i)} \in V_p
    \end{gather*}

    Для упрощения вычислений отбросим индекс $i$, считая, что мы всё время работаем с одним и тем же собственным значением.

    \begin{gather*}
        b_1 = (1, \lambda, \lambda^2, \dots, \lambda^n, \dots) \\
        (\phi - \lambda \epsilon)b_1 = (\lambda, \lambda^2, \dots, \lambda^{n + 1}, \dots) - (\lambda, \lambda^2, \dots, \lambda^{n + 1}, \dots) = 0
    \end{gather*}

    Таким образом мы доказали, что $b_1 = \{\lambda^n\}_{n=0}^{\infty}$ - собственный вектор.
    Пусть вектор высоты $s-1$ построен. Тогда $b_{s-1} = f_{s-1}(n) \lambda^n, b_s = f_s(n) \lambda^n$. Заметим, что: $$f_s(n+1) \lambda^{n+1} - f_s(n) \lambda^{n+1} - f_{s-1}(n) \lambda^{n} \vdots \lambda^{n + 1}.$$ Разделим на $\lambda^{n + 1}$: $$f_s(n+1) - f_s(n) = \frac{f_{s-1}(n)}{\lambda}.$$
    При $\lambda = 1$ решением этого уравнения является $f_s(n) = C_n^{s-1}$, что можно доказать 
    самостоятельно в качестве упражнения (на самом деле это следует из формулы $C_n^{s - 1} + C_n^s = C_{n + 1}^s$). В общем случае будем искать решение в виде квазимногочлена:
    $f_s(n) = C_n^{s-1} \cdot \lambda^{\alpha(s)}$. Подставим это решение в полученное выше уравнение:
    $$C_{n+1}^{s-1} \lambda^{\alpha(s)} + C_n^{s-1} \lambda^{\alpha(s)} 
    = C_n^{s-2} \lambda^{\alpha(s-1) - 1}.$$ В силу того, что $C_{n+1}^{s-1} = C_n^{s-1} + C_n^{s-2}$,
    получаем $\alpha(s) = \alpha(s-1) - 1$. В силу того, что при $s = 1$ мы должны получить собственный 
    вектор $b_1$, полученный ранее, верно $f_1(n) = 1$, а значит $\alpha(1) = 0$. 
    Тогда $\alpha(2) = \alpha(1) - 1 = -1$, и $\alpha(s) = 1 - s$. 
    Отсюда следует, что $f_s(n) = C_{n}^{s-1} \lambda^{1-s}$, а значит $b_s = C_{n}^{s-1} \lambda^{n+1-s}$.
    Таким образом, мы получили Жорданов базис, отвечающий $\lambda$: $b_1, b_2, \dots, b_l$.
\end{proof}

\begin{example}[Числа Фибоначчи]
    Элементы последовательности задаются соотношением: $$x_{n+2} = x_{n+1} + x_n,$$ а значит, характеристический многочлен выглядит следующим образом: $p(x) = x^2 - x - 1$. Дискриминант $D = 5$,
    размерность пространтсва равна $\dim V_p = 2$, корни равны $x_1 = \frac{1 + \sqrt{5}}{2}$, $x_2 = \frac{1 - \sqrt{5}}{2}$.
    Тогда сама последовательность имеет вид $a_n = c_1(\frac{1 + \sqrt{5}}{2})^n + c_2(\frac{1 - \sqrt{5}}{2})^n$. Константы можно найти из начальных условий, например 
    для $a_0 = 0$, $a_1 = 1$ получается формула Бинэ: $$a_n = \frac{1}{\sqrt{5}}(\frac{1 + \sqrt{5}}{2})^n - \frac{1}{\sqrt{5}}(\frac{1 - \sqrt{5}}{2})^n.$$
    При этом $\gd(a_n, a_m) = a_{\gd(n, m)}, \\ F_z \in \Cm$: $F_{z + 1} = F_{z + 1} + F_z$, $F_z |_{\N} = {a_n}$.
\end{example}

\subsection{Приложение линейных рекуррент к расширению полей}

\begin{theorem}
    Пусть дан многочлен $p(x) \in F[x]$ и $p$ неприводим над полем $F$. Тогда существует расширение 
    $K$ поля $F$ в котором $p$ имеет хотя бы один корень.
\end{theorem}

\begin{proof}
    Пусть $\deg p = s$, $A_p$ - сопутствующая матрица для многочлена $p$. В качестве искомого поля 
    будем рассматривать кольцо $K = F[A_p]$ с единицей, то есть кольцо многочленов от матрицы $A_p$. \\
    Сначала проверим, что оно является расширением поля $F$ и само является полем.\\
    Пусть $\alpha \in F$. Тогда можно сопоставить число $\alpha \in F$ элементу 
    $\alpha E \in K = F[A_p]$. Таким образом $F \subset K$.

    Если $f \in F[A_p]$ - не константа, то к $f$ есть обратный многочлен в этом же поле. \\
    $f = \alpha \in F = const$, тогда $\alpha^{-1} = f^{-1}$, Утверждается, что $f$ не кратно $p$. Докажем от противного: пусть кратно, тогда $f(A_p) = p(A_p) \cdot q(A_p) = 0$, откуда $\gd(f, p) = 1$. 

    По свойству наименьшего общего делителя для многочленов:
    \begin{gather*}
        \exists u(x), v(x) \in F[x]: u(x)f(x) + v(x)p(x) = 1, \\
        u(A_p) \cdot f(A_p) + v(A_p) \cdot p(A_p) = E.
    \end{gather*}
     Из того, что $p(A_p) = 0$ получаем $f^{-1}(A_p) = u(A_p)$ и приходим к противоречию. \\
    Таким образом, $F[A]$ - поле. У многочлена $p(x)$ есть корень $A_p \in F[A_p]$ так как $p(A_p) = 0$.
\end{proof}

\begin{example}
    Над $\R$ многочлен $p(x) = x^2 + x + 1$ не имеет корней. Матрица $A_p$ имеет вид:
    \begin{gather*}
        \begin{pmatrix}
        0      & 1      \\
        -1     & 0      
        \end{pmatrix}
    \end{gather*}
    Множество решений в терминах матрицы при этом имеет следующий вид (в данном случае получаем, что $y$ -- коэффициент перед мнимой частью, а $x$ -- перед действительной):
    $$\R[A_p] = \{x E - y A_p \vert x, y \in \R\} = \{\begin{pmatrix}
        x     & -y      \\
        y     & x      
        \end{pmatrix}\}.$$
    Покажем, что $A_p$ соответсвует $-i$ в введенных обозначениях: \\
    $$A_p^2 = \begin{pmatrix}
        0      & 1      \\
        -1     & 0      
        \end{pmatrix} \cdot \begin{pmatrix}
        0      & 1      \\
        -1     & 0      
        \end{pmatrix} = \begin{pmatrix}
        -1      & 0      \\
        0     & -1      
        \end{pmatrix},$$
        что соответствует $-1 \in \R$. При этом верно:
        \begin{align*}
            A_p^3 = A_p \cdot A_p^2 = -A_p, && A_p^4 = (-A_p) \cdot A_p = 1 \in \R.
        \end{align*}
        Таким образом,
        \begin{gather*}
        i \longleftrightarrow \begin{pmatrix}
        0      & -1      \\
        1     & 0      
        \end{pmatrix} = R(\frac{\pi}{2}).
        \end{gather*}
\end{example}

\begin{corollary}
    Пусть $p(x) \in F[x]$ - произвольный многочлен. Тогда существует расширение $\tilde{F}$ поля $F$, в котором $p(x)$ раскладывается на линейные множители.
\end{corollary}

\begin{proof}
    Дедукция по количеству неприводимых множителей, на которые раскладывается многочлен $p$:
    \begin{enumerate}
        \item База: пусть $p$ раскладывается в произведение $s$ линейных множителей: $p(x) = \displaystyle\prod_{i = 1}^s (x - \lambda_i)$ -- все доказано.
        \item Предположим, что для многочлена $p$, у которого  разложении на неприводимые множители больше чем $t$ множителей, разложение существует. Докажем, что тогда можно разложить и в случае, если есть ровно $t$ неприводимых множителей:

        Пусть $p_i \vert p$ и $p_i$ неприводим, $\deg p_i \geq 2$. Расширим $F$ до поля $F_1$ такого, что $p_i$ имеет корень $\alpha$ в $F_1$. \\
        $p_i(x) = (x - \alpha) q_i(x)$, в $F_1$ имеет больше $t$ неприводимых множителей. Теперь к $p(x)$ применимо предположение дедукции.
    \end{enumerate}
\end{proof}

\begin{definition}
    Минимальное поле, в котором многочлен $p$ раскладывается на линейные множители, называется полем разложения многочлена.
\end{definition}

\begin{note}
    Поле разложения многочлена единственно с точностью до изоморфизма, однако это утверждение в курсе доказано не будет.
\end{note}

\begin{corollary}
    Теорема \ref{th4.4} Гамильтона-Кэли справедлива над любым полем.
\end{corollary}

\begin{proof}
    Пусть $A \in M_n(F)$, по теореме Гамильтона-Кэли $\chi_A (A) = 0$. \\
    $\tilde{F} \supset F: \tilde{F}$ -- поле разложения $\chi_A(x)$. Тогда по теореме Гамильтона-Кэли $\chi_A(A) = 0$ в $F$.
\end{proof}

\begin{theorem}
    Если $p$ - простое число, то найдется поле $F_{p^n}$, состоящее из $p^n$ элементов, где $n$ - любое натуральное число.
\end{theorem}

\begin{idea}
    Пусть такое поле существует. Тогда $F_{p^n}^* = F_{p^n} \setminus \{0\}$ - группа по умножению. 
    Тогда по теореме Лагранжа $x \in F_{p^n}$, $x^{p^n - 1} = 1$, $p^n - 1 = \ord x$. Рассмотрим $f(x) = x^{p^n} - x$(коэффициенты из $F_p$) -- корни, обратные $F_{p^n}$.
\end{idea}

\begin{proof}
    Пусть $f(x) \in F_p[x]$, $f(x) = x^{p^n} - x$. Пусть $\tilde{F}$ -- поле разложения $f$. Тогда 
    в $\tilde{F}$ у $f$ имеется $p^n$ корней. Рассмотрим $F_{p^n}$ -- множество корней $f$. Проверим, 
    что оно является полем:
    \begin{enumerate}
        \item $a \in F_{p^n} \Leftrightarrow a^{p^n} = a$.
        \item $a, b \in F$, тогда $(a \cdot b)^{p^n} = a^{p^n} \cdot b^{p^n} = a \cdot b$ -- замкнутость относительно умножения.
        \item Если $a \neq 0$, то $a^{-1} \in F_{p^n}$ т.к. $(a^{-1})^{p^n} = (a^{p^n})^{-1} = a^{-1}$ -- существование обратного элемента.
        \item По построению $F_p \subset \tilde{F}$, откуда $char \tilde{F} = p$. 
    \end{enumerate}
    В любом поле характеристики $p$ справедлива формула Бинома Ньютона: $(a+b)^p = a^p + b^p$, $a, b \in F_p$. \\
    \begin{gather*}
    (a + b)^p = a^p + C_p^1 a^{p - 1} b + C_p^2 a^{p - 1} b^2 + \dots + b^p = a^p + b^p, \\
    C_p^k = \frac{p!}{k!(p - k)!}, \\ 
    C_p^k \equiv 0 (mod p).
    \end{gather*}
    Докажем, что $(a+b)^{p^n} = a^{p^n} + b^{p^n}$ если $a, b \in F$. Индукция по $n$:
    \begin{enumerate}
        \item $n=1$: как было доказано выше, $(a+b)^p = a^p + b^p$
        \item Предположим, что для $n-1$ верно. Пусть $a$, $b \in F_{p^n}$. Тогда:
        $$(a+b)^{p^n} = ((a+b)^p)^{p^{n-1}} = (a^p + b^p)^{p^{n-1}} = a^{p^n} + b^{p^n}.$$
    \end{enumerate}
\end{proof}

\section{Билинейные операторы}

\begin{definition}
    Пусть $V$ - линейное пространство над $F$. Функция $f: V \times V \to F$ называется билинейной, если выполняются следующие условия:
    \begin{enumerate}
        \item Аддитивность по первому аргументу $f(x_1 + x_2, y) = f(x_1, y) + f(x-2, y)$.
        \item Линейность по первому аргументу $f(\lambda x, y) = \lambda f(x, y)$.
        \item Аддитивность по втроому аргументу.
        \item Линейность по второму аргументу.
    \end{enumerate}
\end{definition}

\begin{example}~
    \begin{enumerate}
        \item $V_3$, скалярное произведение - билинейный оператор: $(x, y) = |x| \cdot |y| \cos(\phi)$
        \item Пусть $f$, $g$ - линейные функции на $V$, тогда $h(x, y) = f(x) \cdot g(y)$ - билинейная функция $h: V \times V \to V$.
        \item $V = M_{n \times m}(F)$ и $f(X, Y) = tr(X^T A Y)$, где $A \in M_n(F)$. \\
        Покажем аддитивность и линейность: $$f(X_1 + X_2, Y) = tr((X_1 + X_2)^T A Y) = tr(X_1^T A Y) + tr(X_2^T A Y) = f(X_1, Y) + f(X_2, Y),$$
        $$f(\lambda X, Y) = tr((\lambda X)^T A Y) = \lambda f(X, Y).$$
        Важный частный случай - при $m=1$: $$f(X, Y) = X^T \cdot A \cdot Y = (x_1, x_2, \dots x_n) 
        \cdot A \cdot (y_1, \dots y_n)^T = \displaystyle\sum_{i=1}^{n} \displaystyle\sum_{j=1}^{m} a_{ij} x_i y_j \in F.$$
    \end{enumerate}
\end{example}

\begin{note}
    $f(0, y) = 0$ так как $f(0, y) = f(0 \cdot x, y) = 0 \cdot f(x, y) = 0$.
\end{note}

\begin{definition}
    Если $x, y \in F^n$, то выражение $\displaystyle\sum_{i=1}^{n} \displaystyle\sum_{j=1}^{m} a_{ij} x_i y_j$ называется билинейной формой от координатных столбцов $x$ и $y$. Билинейная форма сама является билинейной функцией: $F^n \times F^n \to F$.
\end{definition}

\begin{proposition}
    \label{pr8.1}
    Если $f(x, y)$ -- билинейная функция $V \times V \to F$, то она может быть записана в виде билинейной формы от координат $x$ и $y$ при добавлении коэффициентов $a_{ij} = f(e_i, e_j)$ - значения функции $f$ на базисных векторах. 
\end{proposition}

\begin{proof}
    Пусть $f(x, y)$ - билинейная функция, $e = (e_1, \dots e_n)$ - базис в $V$. Запишем разложения векторов $x$ и $y$ по базису:
    \begin{align*}
        x = \sum_{i=1}^{n} x_i e_i && y = \sum_{j=1}^{n} y_j e_j
    \end{align*}
    Тогда верно следующее:
    \begin{gather*}
        f(x, y) = f(\sum_{i=1}^{n} x_i e_i, \sum_{j=1}^{n} y_j e_j) = \sum_{i=1}^{n} \sum_{j=1}^{n} x_i y_j f(e_i, e_j) = \sum_{i=1}^{n} \sum_{j=1}^{n} a_{ij} x_i y_j
    \end{gather*}
\end{proof}

\begin{proposition}
    \label{pr8.2}
    Пусть $f(x, y)$ -- билинейная функция в $V$. $e$, $e'$ -- базисы в $V$. $A$, $A'$ -- 
    матрицы билинейной формы $f$ в этих базисах. Тогда $A' = S^T A S$, где $S$ -- матрица перехода 
    между $e$ и $e'$. 
\end{proposition}

\begin{proof}
    Пусть $x$ и $y$ имеют в $e$ координаты $\alpha$ и $\beta$ соответственно.\\ Было доказано, что $\alpha = S \alpha'$, $\alpha` = s^{-1} \alpha$, $\beta = S \beta'$. Тогда:
    $$f(x, y) = x^T A y = \alpha^T A \beta = (S \alpha')^T A (S \beta) = (\alpha')^T S^T A S \beta' = (\alpha')^T A' \beta'.$$ Из последнего равенства сразу следует, что $S^T A S = A'$.
\end{proof}

\begin{note}
    В отличие от билинейной формы, матрица линейного оператора $A' = S^{-1} A S$.
\end{note}

\begin{proposition}[Инварианты матрицы $A$ билинейной формы]
    $rk A$ не зависит от выбора базиса, где $A$ -- матрица билинейной функции.
\end{proposition}

\begin{proof}
        $\rk (A') = \rk (S^T A S) \leq rk A$. При этом $A = (S^T)^{-1} A' S{-1}$, откуда $\rk A \leq \rk A'$, а значит, ранги равны.
\end{proof}

\begin{corollary}
    Определители $A$ и $A'$ над полем вещественных чисел всегда имеют одинаковый знак: $\det A' = \det(S^T A S) = (\det S)^2 det A$, где $(\det S)^2 > 0$.
\end{corollary}
