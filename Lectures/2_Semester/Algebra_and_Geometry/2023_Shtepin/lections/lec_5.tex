% 01.03.23 Оля

\begin{reminder}
    Вспомним теорему с предыдущей лекции: пусть $\phi \in \mathcal{L}(V)$, $f$ --
    аннулирующий многочлен для $\phi$, раскладывающийся на произведение двух взаимно-простых 
    многочленов: $f = f_1 \cdot f_2$, $\gd (f_1, f_2) = 1$. 
    Тогда $V$ раскладывается в прямую сумму $V = V_1 \oplus V_2$, 
    где $V_i = \ker f_i(\phi)$ -- инвариантные подпространства.
\end{reminder}

\begin{corollary}
    Пусть $\phi \in \mathcal{L}(V)$, $f$ -- аннулирующий многочлен для $\phi$, такой что $f$ 
    раскладывается в произведение $f = f_1 \cdot f_2 \dots f_n$ попарно взаимно-простых многочленов.
    Тогда $V$ раскладывется в прямую сумму $V = V_1 \oplus V_2 \oplus \dots V_n$, 
    где $V_i = \ker f_i(\phi)$ -- инвариантные подпространства.
\end{corollary}

\begin{proof}
    Индукция по $n$:
    \begin{enumerate}
        \item База: случай $n = 2$ доказан в теореме \ref{th4.6}
        \item Рассмотрим случай $n$ сомножителей. Тогда f можно разложить следующим образом: 
        $$f = (f_1 \cdot f_2 \cdot ... \cdot f_{n-1}) \cdot f_n,$$ при этом $(f_1 \dots f_{n-1})$ 
        и $f_n$ взаимно просты.
        Тогда по теореме \ref{th4.6}:
        $$V = \ker (f_1(\phi) \cdot f_2(\phi) \cdot ... \cdot f_{n-1}(\phi)) \oplus V_n.$$ 
        При этом многочлен $f_1 f_2 \dots f_n$ -- аннулирующий многочлен сужения $\phi_{v'}$, 
        а значит по предположению индукции:
        $$V' = \ker (f_1(\phi) \vert_{V'}) \oplus \dots \oplus \ker (f_n(\phi) \vert_{V'}).$$
        Осталось проверить только что $\forall i \; \ker f_i(\phi) \vert_{V'} = \ker f_i(\phi) = V_i$.
        
        Вложение вправо очевидно: пусть $x \in V'$ и $f_i(\phi) \vert_{V'} x = 0$, тогда верно и 
        $f_i(\phi) x = 0$.

        В обратную сторону, пусть $x \in \ker f_i(\phi)$ то есть $\ker f_i(\phi) x = 0$. Тогда полное
        произведение операторов $f_1(\phi) f_2(\phi) ... f_{n-1}(\phi) = 0$, а значит $x \in V'$.
        Таким образом $x \in \ker f_i(\phi) \vert_{V'}$.
    \end{enumerate}
\end{proof}

\section{Корневые подпространства}

\begin{definition}
    $\phi : V \to V$, $V$ -- линейное пространство над полем $F$, $\lambda \in f$.
    Вектор $x \in V$ называют корневым для $\phi$ отвечающим $\lambda \in F$, если 
    $\exists k \in \N \; : (\phi - \lambda \epsilon)^k x = 0$.
\end{definition}

\begin{definition}
    Число $k$ -- высота корневого вектора $x$, отвечающего $\lambda$ из $F$, если $k$ -- наименьшее 
    число такое что $(\phi - \lambda \epsilon)^k x = 0$. 
    Будем считать, что нулевой вектор имеет высоту 0.
\end{definition}

\begin{example}
    Пусть $x$ -- собственный вектор для $\phi$, соответствующий собственному значению $\lambda$. 
    Тогда $\phi(x) = \lambda x$, а значит $x$ -- корневой вектор высоты 1 для оператора $\phi$ 
    так как $(\phi - \lambda \epsilon) x = 0$.
    Следствия верны и в обратную сторону, а значит собственные векторы являются корневыми векторами 
    высоты 1.
\end{example}

\begin{example}
    Пусть $\phi = \frac{d}{dx}$, $V = \R_n[x] = \{ c \in \R_n[x] \, \vert \deg f \leq n\}$. $V$ -- 
    корневое пространство для $\phi$ с собственным значением $\lambda = 0$. 
    Тогда $\phi^{n+1}(V) = \frac{d^{n+1}}{dx^{n+1}} = 0$. 
    Максимальную высоту при этом имеет $f(x) = x^n$. 
\end{example}


\begin{proposition}
    Пусть $V^{\lambda}$ -- множество всех корневых векторов для $\phi$ относящихся к $\lambda$.
    Тогда $V^{\lambda}$ -- подпространство в $V$.
\end{proposition}

\begin{proof}
    Пусть $x, y \in V^{\lambda}$, $s$ и $t$ -- высоты $x$ и $y$ соответственно. 
    Положим $M = max(s, t)$. Тогда верно следующее:
    \begin{eqnarray}
        (\phi - \lambda \epsilon)^M(x+y) = (\phi - \lambda \epsilon)^M x + 
        (\phi - \lambda \epsilon)^M y = 0 + 0 = 0.
    \end{eqnarray}
    Таким образом замкнутость относительно сложения выполняется. Замкнутость относительно умножения 
    на скаляры предлагается проверить самостоятельно.
\end{proof}

\begin{definition}
    Построенное подпространство $ V^{\lambda}$ называется корневым для оператора $\phi$ 
    относящегося к $\lambda$.
\end{definition}

\begin{proposition}
    Корневое подпространство $ V^{\lambda}$ отлично от нулевого тогда и только тогда когда 
    $\lambda$ -- собственное значение оператора $\phi$.
\end{proposition}

\begin{proof}~

    Достаточность. Пусть $\lambda$ -- собственное значение оператора $\phi$, тогда 
    $V_{\lambda} \neq \{ 0\}$. Однако $V_{\lambda} \subseteq V^{\lambda}$, а значит 
    $V^{\lambda}$ -- ненулевое.

    Неоходимость. Пусть корневое подпространство $ V^{\lambda}$ отлично от нуля. Пусть есть вектор 
    $x$ высоты $k$ такой что $x \in V^{\lambda}$. Тогда $y = (\phi - \lambda \epsilon)^{k-1} x \neq 0$,
    а $(\phi - \lambda \epsilon)y = (\phi - \lambda \epsilon)^{k} x = 0$. 
    Таким образом $\phi(y) = \lambda y$, а значит $x$ -- собственный вектор $\phi$, $\lambda$ -- 
    собственное значение.
\end{proof}

\begin{agreement}
    Начиная с этого момента корневыми подпространствами будем называть только те $V^{\lambda}$, 
    для которых $\lambda$ -- собственное значения оператора $\phi$.
\end{agreement}

\begin{reminder}
    Подпространство $W$ называется дополнительным к $V^{\lambda}$, 
    если их пересечение состоит только из нуля.
\end{reminder}

\begin{reminder}
    Оператор $\phi$ действует невырожденным образом когда $\ker(\phi - \lambda \epsilon) = 0$.
\end{reminder}

\begin{theorem}[о свойствах корневых подпространств]~
    \label{th5.1}

    Пусть $V^{\lambda}$ -- корневое подпространство для $\phi$ отвечающее $\lambda$. Тогда
    \begin{enumerate}
        \item $V^{\lambda}$ инвариантно относительно $\phi$.
        \item Подпространство $V^{\lambda}$ имеет единственное собственное значение $\lambda$.
        \item Если $W$ -- тоже инвариантное относительно $\phi$ подпространство, при этом являющееся 
        дополнительным к $V^{\lambda}$, то на $W$ оператор 
        $\phi - \lambda \epsilon$ действует невырожденным образом.
    \end{enumerate} 
\end{theorem}

\begin{proof}~
    \begin{enumerate}
        \item Пусть m -- максимальная высота векторов $x \in V^{\lambda}$, в силу конечномерности 
              $V^{\lambda}$ такая существует и является конечным числом.
              Тогда $V^{\lambda} = \ker (\phi - \lambda \epsilon)^m$. 
              
              Операторы $\phi$ и $\epsilon$
              коммутируют с $\phi$, а значит и оператор $(\phi - \lambda \epsilon)^m$ коммутирует с 
              $\phi$. Таким образом, можно записать 
              $(\phi - \lambda \epsilon)^m \phi = \phi (\phi - \lambda \epsilon)^m$. 
              Тогда по теореме \ref{th3.2} получаем, что $\ker (\phi - \lambda \epsilon)^m$ 
              инвариантно относительно $\phi$. 
        \item Докажем от противного, пусть в $V^{\lambda}$ найдется ненулевой собственный вектор $x$ 
              с собственным значением $\mu \neq \lambda$, то есть $\phi(x) = \mu x$. Применим к этому 
              вектору оператор $(\phi - \lambda \epsilon)$:  
              $$(\phi - \lambda \epsilon) x = \phi(x) - (\lambda \epsilon)(x) = (\mu - \lambda) x.$$ 
              Тогда при многократном применении 
              получим $(\phi - \lambda \epsilon)^m x = (\mu - \lambda)^m x = 0$,
              так как $x \in V^{\lambda}$ и должен аннулироваться. Тогда $\mu - \lambda = 0$, 
              что дает противоречие.
        \item По условию $V$ представляется как $V = V^{\lambda} \oplus W$. При этом подпространства 
              $V^{\lambda}$ и $W$ инвариантны относительно $\phi$, а значит, согласно утверждению 
              \ref{prop4.2}, они так же инвариантны относительно $(\phi - \lambda \epsilon)$. 
              Нам нужно доказать, что $\phi - \lambda \epsilon$ невырожден на $W$, то есть что 
              $\ker (\phi - \lambda \epsilon) \vert_{W} = \{0\}$.
              
              Докажем от противного, пусть $\exists x \neq 0$ такое что 
              $x \in \ker (\phi - \lambda \epsilon) \vert_{W}$. 
              Отсюда следует, что вектор $x$ лежит в пространстве $W$, так как лежит в ядре сужения
              оператора на это подпространство.

              Однако $(\phi - \lambda \epsilon) x = 0$, а значит х -- собственный для $\phi$ 
              с собственным значением $\lambda$. Тогда вектор $x$ так же лежит и в пространстве 
              $V^{\lambda}$, что приводит к 
              противоречию с тем, что по условию $V^{\lambda} \cap W = \{0\}$.
    \end{enumerate}
\end{proof}

\begin{corollary}
    Корневое подпространство $V^{\lambda}$ - максимальное по включению инвариантное подпространство, 
    на котором $\phi$ имеет единственное собственное значение $\lambda$.
\end{corollary}

\begin{theorem}[о разложении пространства V в прямую сумму корневых]~
    \label{th5.2}

    Пусть $\phi \in \mathcal{L}$, $\phi$ -- линейно факторизуем над $F$ 
    (характеристический многочлен раскладывается в произведение линейных множителей над F).
    Тогда пространство $V$ есть прямая сумма корневых подпространств: 
    $V = V^{\lambda_1} \oplus V^{\lambda_2} \oplus \, \dots \, \oplus V^{\lambda_k}$, где все $\lambda$ попарно различны.
\end{theorem}

\begin{proof}
    По условию $\phi$ линейно факторизуем, а значит
    $\chi_{\phi}(t) = \displaystyle\prod_{i= 1}^{k} (\lambda_i - t)^{m_i}$, где 
    $\lambda_i$ -- собственные значения. Многочлены $(\lambda_i - t)^{m_i}$ попарно взаимно просты 
    из попарной различности $\lambda_i$, поэтому по следствию из теоремы \ref{th4.6} можно заключить:
    $$V = \ker (\phi - \lambda_1 \epsilon)^{m_1} \oplus \ker (\phi - \lambda_2 \epsilon)^{m_2} 
    \oplus \dots \oplus \ker (\phi - \lambda_k \epsilon)^{m_k}$$
    При этом $\ker (\phi - \lambda_i \epsilon)^{m_i} \subseteq V^{\lambda_i}$ для всех $i$,
    а значит вектор $x \in V$ представим в виде суммы 
    $x = x_1 + \dots + x_k$, где $x_i \in V^{\lambda_i}$.
    Отсюда очевидно, что пространство $V$ является суммой подпространств: 
    $$V = V^{\lambda_1} + V^{\lambda_2} + \dots V^{\lambda_k}.$$ 
    
    Осталось доказать что $V^{\lambda_i} \subseteq \ker(\phi - \lambda_i \epsilon)^{m_i}$, 
    в таком случае сумма будет прямой. Докажем от противного, пусть существует индекс $i$ такой, 
    что $\ker (\phi - \lambda_i \epsilon) \leq V^{\lambda_i}$. Тогда найдется вектор 
    $x \in V^{\lambda_i}$ такой, что он не лежит в ядре. Обозначим высоту $x$ за $M > m_i$, тогда:
    $$\chi_{\phi}(\phi) x = \left(\displaystyle\prod_{j \neq i} (\phi - \lambda_j \epsilon)^{m_j}\right) \cdot 
    (\phi - \lambda_i \epsilon)^{m_i} x = \displaystyle\prod_{j \neq i} 
    ((\phi - \lambda_j \epsilon)^{m_j})x' \neq 0.$$ 
    Если найдется такой $j$ что $(\phi - \lambda_j \epsilon)x = 0$, то у $x'$ есть собственное значение
    $\lambda_j$, что приводит к противоречию с пунктом 2 теоремы \ref{th5.1}. 
    В противном случае возникает противоречие с 
    $\chi_{\phi}(\phi) = 0$ по теореме \ref{th4.4} (Гамильтона-Кэли). 

    Таким образом, $V^{\lambda_i} = \ker(\phi - \lambda_i \epsilon)^{m_i}$, а значит $V$ представляется 
    в виде прямой суммы $V^{\lambda_i}$:

    $$V = V^{\lambda_1} \oplus V^{\lambda_2} \oplus \, \dots \, \oplus V^{\lambda_k}.$$
\end{proof}

\begin{remarkfrom}
    В записи лекции Вадима Владимировича за 2021 год приводится ещё один вариант доказательства. 
    После получения $V = V^{\lambda_1} + V^{\lambda_2} + \dots V^{\lambda_k}$ покажем, что подпространства 
    $V^{\lambda_i}$ линейно независимы.

    Для того, чтобы показать линейную независимость подпространств покажем, что равенство нулю суммы 
    $x_1 + x_2 + \, \dots \, + x_k = 0$, где $x_i \in V^{\lambda_i}$, равносильно тому, что все $x_i = 0$.

    Рассмотрим оператор $\psi = \displaystyle\prod_{j = 2}^{k} (\phi - \lambda_j \epsilon)^{m_j}$ и применим его 
    к левой и правой частям равенства. Все $x_j$ при $j \geq 2$ аннулируются этим оператором так как 
    в произведении присутствуют $(\phi - \lambda_j \epsilon)^{m_j}$, аннулирующие соответствующие $x_j$.

    Правая часть после применения оператора так же остается нулевой, откуда $\psi x_1 = 0$.
    При этом операторы $(\phi - \lambda_j \epsilon)^{m_j}$ невырождены на $V^{\lambda_1}$,  а значит 
    и оператор $\psi$ невырожден на $V^{\lambda_1}$. Таким образом произведение $\psi x_1$ может быть 
    нулевым только если $x_1 = 0$. Проводя аналогичные рассуждения для всех $x_i$ получим, что 
    все эти векторы обязаны быть нулевыми, а значит подпространства $V^{\lambda_i}$ линейно независимы.

    По теореме о характеризации прямой суммы $V = V^{\lambda_1} \oplus V^{\lambda_2} \oplus \, \dots \, \oplus V^{\lambda_k}.$
\end{remarkfrom}

\begin{corollary}
    В условиях теоремы 2 $\dim V^{\lambda_i} = m_i = alg(\lambda_i)$.
\end{corollary}

\begin{proof}
    Пусть $\dim V^{\lambda_i} = n_i$. Выберем базис согласованный с разложением в прямую сумму. 
    Тогда матрица $A_{\phi}$ имеет диагональный вид:
    \[A_{\phi} = \begin{pmatrix}
		A_1    & 0      & \dots  & 0\\
		0      & A_2    & \dots  & 0\\
		\vdots & \vdots & \ddots & \vdots\\
		0      & 0      & \dots  & A_n
	\end{pmatrix}\]
    где $A_i \in M_{n_i}(F)$. 
    Тогда $\chi_{\phi} (t)$ раскладывется по условию фактороизуемости: 
    $\chi_{\phi} (t) = \prod \chi_{\phi \vert_{V^{\lambda_i}}} (t)$.
    Тогда $\chi_{\phi \vert_{V^{\lambda_i}}} (t) = (\lambda_i - t)^{n_i}$, при этом
    $\chi_{\phi \vert_{V^{\lambda_i}}} (t) \vert \chi_{\phi} (t)$, а значит степень 
    не превосходит алгебраическую кратность $\lambda_i$, то есть $n_i \leq m_i$.
    $$\sum n_i = \displaystyle\sum_{i = 1}^{k} \dim V = n, \; 
    \displaystyle\sum_{i = 1}^{k} m_i = \deg \chi = n,$$ 
    а значит равенство точное: $n_i = m_i$.
\end{proof}

\subsection{Нильпотентные операторы}  
\begin{definition}
    Оператор $\phi: V \to V$ называется нильпотентным если $\exists k \in \N :\: \phi^k = 0$.
\end{definition}

\begin{note}
    Нильпотентные операторы не стоит путать с непотентными и унипотентными. Слово происходит от 
    nil -- ноль и potention -- возведение в степень.
\end{note}

\begin{definition}
    Наименьшее натуральное число $k$ такое что $\phi^k = 0$, $\phi^{k-1} \neq 0$ называют 
    индексом нильпотентности или (ст\'{у}пенью нильпотентности) относительно $\phi$. 
\end{definition}

\begin{example}
    $V^{\lambda}$ -- корневое для оператора $\phi$, $\exists k: (\phi - \lambda \epsilon)^k 
    \vert_{V^{\lambda}} = 0$, а значит на $V^{\lambda}$ оператор $(\phi - \lambda \epsilon)$ 
    -- нильпотентный.
\end{example}

\begin{example}
    На $\R_n[x] \phi = \frac{d}{dx}$ имеет ступень нильпотентности $n+1$.
\end{example}

\begin{definition}
    Пусть $\phi$ -- нильпотентный и $x \in V$ -- вектор, имеющий высоту $k$. 
    Рассмотрим $U = \langle x, \phi(x), \dots \phi^{k-1}(x)\rangle$.
    Построенное инвариантное подпространство $U$ называется циклическим подпространством, 
    порожденным вектором $x$.
\end{definition}

\begin{note}
    Очевидно что циклическое подпространство порожденное вектором $x$ является наименьшим 
    $\phi$--инвариантным линейным подпространством порожденным $x$.
\end{note}

\begin{proposition}
    Векторы $x, \phi(x), \dots \phi^{k-1}(x)$ образуют базис циклического подпространства 
    образованного $x$ если высота вектора $x$ относительно нильпотентного оператора $\phi$ равна $k$.
\end{proposition}

\begin{proof}
    Достаточно доказать линейную независимость этих векторов. Пусть есть некоторая нетривиальная 
    линейная комбинация $\displaystyle\sum_{s = 0}^{k-1} \alpha_i \phi^{s}(x) = 0$ 
    и пусть $\alpha_l$ -- лидер строки, тогда все предыдущие слагаемые можно выбросить из суммы 
    в силу того, что они нулевые. Применим $\phi^{k-l-1}$ к оставшимся в сумме слагаемым: 
    $\alpha_l \phi^{k-1}(x) + \alpha_{l+1} \phi^{k}(x) + \dots = 0$.
    Тогда $\alpha_l = 0$, так как все последующие слагаемые равны нулю в силу того, что высота 
    порождающего подпространство вектора равна $k$. Противоречие получено.
\end{proof}

\begin{note}
    Найдем матрицу нильпотентного оператора $\phi \vert_{V}$ в следующем базисе: 
    $e_1 = \phi^{k-1}(x)$, $e_2 = \phi^{k-2}(x)$, $\dots$, $\phi^0(x) = x$.
    Применим к ним оператор $\phi$: $\phi(e_1) = 0$, $\phi(e_2) = e_1$, $\dots$, $\phi(e_k) = e_{k-1}$ 
    Тогда матрица преобразования $A$ имеет вид жордановой клетки $J_k(0)$:
    \[A_{\phi}^{e} = \begin{pmatrix}
		0      & 1      & 0      & \dots  & 0      & 0\\
		0      & 0      & 1      & \dots  & 0      & 0\\
        0      & 0      & 0      & \dots  & 0      & 0\\
		\vdots & \vdots & \vdots & \ddots & \vdots & \vdots\\
        0      & 0      & 0      & \dots  & 0      & 1\\
		0      & 0      & 0      & \dots  & 0      & 0
	\end{pmatrix}\]
    Это верно в силу того, что $i$-й столбец матрицы является координатами $\phi(e_i)$, равного, 
    как было сказано ранее, $e_{i-1}$ для $i > 0$ и 0 для $i - 0$. Так же $\rk A = k-1$.
\end{note}

\begin{definition}
    Построенный базис в циклическом подпространстве называется циклическим базисом.
\end{definition}

\begin{theorem}[о нильпотентном операторе]~
    \label{th5.3}

    Пусть $\phi: V \to V$ -- нильпотентный оператор индекса нильпотентности $k$, 
    (то есть $\phi^k = 0$, $\phi^{k-1} \neq 0$), $x$ -- ненулевой вектор $x \in V$ 
    высоты $k$ (то есть $\phi^k(x) = 0$, $\phi^{k-1}(x) \neq 0$),
    $U = \langle x, \phi(x), \dots \phi^{k-1}(x) \rangle$ -- циклическое подпространство, инвариантное $\phi$.
    Тогда найдется $\phi$-инвариантное пространство $W$ дополнительное к $U$ такое что $V = U \oplus W$.
\end{theorem}

\begin{idea}
    Найти $\phi$-инвариантное подпространство $W$ такое что:
    \begin{equation}
         \begin{cases}
            U \cap W = {0},\\
            U + W = V.
         \end{cases}
        \end{equation}
    Для этого покажем что существуют подпространства $W$ инвариантные относительно $\phi$ 
    и удовлетворяющие первому условию и среди всех таких выберем максимальное по размерности.
    $U + W < V \Rightarrow \exists W' = W + \langle y\rangle$, $y \notin W$ такое что $W'$ 
    $\phi$-инвариантно и удовлетворяет первому условию.
\end{idea}

\begin{proof}~

    \begin{enumerate}
        \item Пусть $W$ -- максимальное $\phi$-инвариантное подпространство в $V$, 
        такое что $U \cap W = {0}$. Предположим что $U + W < V$. 
        Тогда найдется ненулевой $a \in V$, такой что $ a \notin U + W$.  Пусть $l$ -- наименьшее 
        значение для которого $z = \phi^{l-1}(a) \notin W + U$, $\phi^l(a) \in W + U$. 
        Такое очевидно найдется так как $a \notin W + U$ и $\phi^{k}(a) = 0 \in W + U$.
        Таким образом в этом пункте мы нашли вектор $z \notin U+W$, такой что $\phi(z) \in U + W$.

        \item Пусть $\phi(z) = \displaystyle\sum_{s = 0}^{k-1} \alpha_s \phi^s(x) + w$, 
        при этом $\phi^s(x) \in U$, $w \in W$. Тогда:
        $$\phi^{k}(z) = \alpha_0 \phi^{k-1}(x) + 0 + \dots + 0 + \phi^{k-1}(w) = 0$$ 
        Тогда $\alpha_0 \phi^{k-1}(x) + \phi^{k-1}(w) = 0$. 
        В силу линейной независимости линейных подпространств 
        $\alpha_o \phi^{k-1}(x) = 0$, $\phi^{k-1}(w) = 0$. 
        При этом в силу того, что $\phi^{k-1}(x) \neq 0$, получаем $\alpha_0 = 0$.

        \item Введем вектор $y = z - \displaystyle\sum_{s = 1}^{k-1} \alpha_s \phi^{s-1}(x) \notin U + W$ 
        (так как $z \notin U+W$, а сумма принадлежит $U$).
        Введем пространство $W' = W + \langle y \rangle$, $\dim W' = \dim W + 1$. 
        Покажем что вновь построенное подпространство так же инвариантно $\phi$:
        $$\phi(y) = \phi(z) - \displaystyle\sum_{s=1}^{k-1} \alpha_s \phi^s (x) = \phi(z) - 
        \displaystyle\sum_{s=0}^{k-1} \alpha_s \phi^s (x) = w \in W.$$

        \item Покажем теперь что $W'$ удовлетворяет условию $U \cap W' = {0}$. 
        Пусть $0 \neq u \in U \cap W'$, $u \notin U \cap W = \{ 0 \}$. 
        Тогда $u$ представим в виде $u = \widetilde{w} + \lambda y$, $\lambda \neq 0$. Отсюда 
        $y = \frac{1}{\lambda} u - \frac{1}{\lambda} \widetilde{w} \in U + W$. 
        Значит $U \cap W \neq \{0\}$ -- противоречие.
    \end{enumerate}
\end{proof}

\begin{theorem}[о разложении в прямую сумму циклических подпространств для нильпотентного оператора]
    Пусть $\phi: V \to V$, зафиксируем индекс нильпотентности $k$. Тогда существует разложение $V$ 
    в прямую сумму инвариантных циклических подпространств $V = V_1 \oplus V_2 \oplus \dots V_s$. 
    При этом количество слагаемых $s = \dim (\ker \phi) = \dim V_0 = geom(0)$.
\end{theorem}

\begin{proof}
    Индукция по $n = \dim V$. 
    \begin{enumerate}
        \item База: $n = 1 \Rightarrow \phi = 0$, $alg(0) = geom(0) = 1$.
        \item Предположение индукции: для пространства $V$ размерности менее $n$ утверждение выполняется. 
        Пусть теперь $\dim V = n$. 
        Тогда существует $x$ высоты $k$ такой что $\phi^k(x) = 0$, $\phi^{k-1}(x) \neq 0$. 
        Пусть $U = \langle x, \phi(x), \dots, \phi^{k-1}(x) \rangle$ -- $\phi$-инвариантное подпространство.
        По теореме \ref{th5.3} существует $\phi$-инвариантное подпространство $W$, такое что  
        $V = U \oplus W$, $\dim W \leq n-1$. Тогда $W$ раскладывается в прямую сумму 
        $\phi$-инвариантных циклических подпространств.
    \end{enumerate}
\end{proof}

\begin{problem}
    Рассмотрим оператор лапласа $\Delta = \frac{\delta^2}{\delta x^2} + \frac{\delta^2}{\delta y^2}$,
    действующий в пространстве $V = \R_2[x, y] = \langle 1, x, y, x^2, xy, y^2 \rangle$, $\dim V = 6$. Требуется:
    \begin{enumerate}
        \item Доказать $\Delta$-нильпотентность,
        \item Найти матрицу $\Delta$ в данном базисе,
        \item Разложить $V$ в прямую сумму циклических подпространств и выбрать циклический базис 
              в каждом из них,
        \item Проверить, что в получившемся базисе $\Delta$ имеет Жорданову нормальную форму,
        \item Доказать, что в пространстве однородных многочленов степени $k$ в $R[x,y]$ есть два 
              собственных вектора с собственным значением 0.
    \end{enumerate}
\end{problem}

\begin{reminder}
    Многочлен степени $k$ называется однородным если верно: 
    $$\forall \lambda \in F \hookrightarrow f(\lambda x_1, \lambda x_2, \dots \lambda x_k) 
    = \lambda^k f(x_1, x_2, \dots x_k).$$
\end{reminder}
