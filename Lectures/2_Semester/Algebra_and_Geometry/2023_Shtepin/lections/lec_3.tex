% 15.02.23 Оля

\subsection{Кратные корни многочленов. Продолжение.}

\begin{definition}
    Рассмотрим $F[x]$ -- кольцо многочленов над полем $F$.
    Формальной производной многочлена $x^n$ называется $\frac{d}{dx} x^n = n \cdot x^{n-1}$, 
    так же используется обозначение $(x^n)'$. Распространим $\frac{d}{dx}$ на остальные векторы 
    $F[x]$ по линейности. Тогда дифференцирование является линейным оператором: 
    $\frac{d}{dx}: F[x] \to F[x]$.
\end{definition}

\begin{proposition}
    Формальная производная $\frac{d}{dx}$ удовлетворяет правилу Лейбница: 
    $$(f \cdot g)' = f' \cdot g + f \cdot g'.$$
\end{proposition}

\begin{proof}
    Обе части являются линейными по многочленам $f$ и $g$, поэтому достаточно доказать правило 
    для базисных векторов. Рассмотрим $f = x^m$ и $g = x^l$ -- базисные векторы в кольце $F[x]$. 
    Тогда: $$(x^m \cdot x^l)' = (x^{m+l})' = (m+n) \cdot x^{m+l-1}.$$
    Так же можно продифференцировать $f$ и $g$ по отдельности: 
    \begin{align*}
        (x^m)' = m \cdot x^{m-1} && (x^l)' = l \cdot x^{l-1} 
    \end{align*}
    Отсюда очевидно, что равенство действительно выполняется:
    \begin{gather*}
        f' \cdot g + f \cdot g' = m \cdot x^{m-1} \cdot x^{l} + x^{m} \cdot l \cdot x^{l-1} = 
        (m+l) \cdot x^{m+l-1} = (f+g)'
    \end{gather*}
\end{proof}

\begin{corollary}
    Произведение нескольких многочленов и возведение многочлена в степень так же удовлетворяют 
    привычным правилам:
    \begin{enumerate}
        \item $(f_1 f_2 \dots f_n)' = f_1' f_2 \dots f_n + f_1 f_2' \dots f_n + \dots + f_1 f_2 \dots f_n'$,
        \item $(f^n)' = n \cdot f^{n-1} f'$.
    \end{enumerate}
\end{corollary}

\begin{proof}
    \begin{enumerate}
        \item Индукция по $n$:

        База: для $n = 2$ доказано в утверждении 1.

        Переход: докажем, что из истинности утверждения для 2 и для $n-1$ многочленов следует его 
        истинность для $n$:
        \begin{align*}
            (f_1 f_2 \dots f_n)' = ((f_1 f_2 \dots f_{n-1}) \cdot f_n)' = (f_1 f_2 \dots f_{n-1})' 
            \cdot f_n + (f_1 f_2 \dots f_{n-1}) \cdot f_n' = \\ =
            (f_1' f_2 \dots f_{n-1} + f_1 f_2' \dots f_{n-1} + \dots + f_1 f_2 \dots f_{n-1}') 
            \cdot f_n + f_1 f_2 \dots f_n \cdot f_n' = \\ = f_1' f_2 \dots f_n + f_1 f_2' \dots f_n 
            + \dots + f_1 f_2 \dots f_n'.
        \end{align*}

        \item Докажем используя результат, полученный в предыдущем пункте:
        \begin{align*}
            (f^n)' = (f \cdot f \cdot \dots f)' = f' \cdot f \dots f + f \cdot f' \dots f 
            + \dots f \cdot f \dots f' = n \cdot f^{n-1} f'.
        \end{align*}
    \end{enumerate}
\end{proof}

\begin{definition}
    Число $c \in F$ называется корнем многочлена $f \in F[x]$ кратности 
    $R$ если $f$ представим в виде $f(x) = q(x) (x-c)^R$, где $q(c) \neq 0$.
\end{definition}

\begin{theorem}[о кратности корня x]
    Пусть $F$ -- поле, $f \in F[x]$, $c \in F$ -- корень многочлена $f$. Тогда верно следующее:
    \begin{enumerate}
        \item c -- кратный корень f $\Leftrightarrow$ $f(c) = 0$ и $f'(c) = 0$.
        \item с -- корень кратности $R$ $\Rightarrow$ $f(c) = 0$, $f'(c) = 0$, $\dots$, 
            $f^{(R-1)}(c) = 0$.
        \item В условиях предыдущего пункта при выполнении дополнительного условия на характеристику
            поля $char F = 0$ или $char F > R$, верно так же $f^{(R)}(c) \neq 0$.
    \end{enumerate}
\end{theorem}

\begin{proof}~
    \begin{enumerate}
        \item \begin{enumerate}
            \item Необходимость.
            
            По условию $f(x) = q(x) (x-c)$. Продифференцируем $f$:
            $$f'(x) = q'(x) (x-c) + q(x).$$ 
            Тогда $f'(c) = q(c)$. При этом многочлен $q(x)$ кратен $(x-c)$ в силу того, что $c$ - кратный корень $f$. 
            Таким образом вся производная $f'$ кратна $(x-c)$. 
            
            \item Достаточность.
            
            Пусть $f(c) = f'(c) = 0$, тогда $q(c) = 0$, а значит $q(x)$ кратен $(x-c)$.
            
        \end{enumerate}

        \item Пусть $c$ -- корень кратности $R$. Тогда многочлен $f$ представим в виде $f = q(x) (x-c)^R$, где $q(c) \neq 0$.
        Возьмем производную от $f$:
        $$f'(x) = q'(x) (x-c)^R + R \cdot q(x) (x-c)^{R-1}.$$ 
        Продолжим брать производные. Тогда для 
        k-производной кратность корня $c$ не меньше $R-k$.

        \item Кратность корня для производной в точности равна $R-k$, а значит $f^{(R-1)}$ имеет $c$ 
        в качестве простого корня (кратности 1). 
        
        Предположим противное. 
        Пусть $(f^{(R-1)})' (c) = 0$. Тогда $c$ -- кратный корень $f^{(R-1)}$, что приводит к противоречию.

        Таким образом $f^{R}(c) \neq 0$. В обратную сторону, пусть 
        $f(c) = f'(c) = \dots = f^{(R-1)} (c) = 0$. Если $c$ -- корень кратности 1, то $f'(c) \neq 0$, 
        противоречие. Если $c$ -- корень кратности 2, то $f^{(2)} (c) \neq 0$. 
        
        Аналогично если $c$ -- 
        корень кратности $R-1$, то $f^{(R-1)} (c) \neq 0$. Таким образом $c$ -- корень кратности не 
        менее чем $R$. При этом если он имеет кратность большую, чем $R$, то $f^{(R)} = 0$, чего не 
        может быть, а значит $R$ -- корень кратности ровно $R$.
    \end{enumerate}
\end{proof}

\section{Инвариантные подпространства}

\begin{reminder}
    Пусть $V$ -- линейное пространство над полем $F$. Линейным оператором называется отображение 
    $\phi: V \to V$, такое что выполняются следующие аксиомы:
    \begin{enumerate}
        \item $\phi (x+y) = \phi(x) + \phi (y)$
        \item $\phi(\lambda x) = \lambda \cdot \phi(x)$
    \end{enumerate}

\end{reminder}

\begin{reminder}
    Пусть $e = (e_1, e_2, \dots, e_n)$ -- базис в $V$. Тогда равенство 
    $\phi(e_i) = \displaystyle\sum_{k = 1}^{n} a_{kj} e_k$ можно принять за определение матрицы 
    оператора. Иначе говоря в j-ом столбце матрицы оператора стоят координаты вектора $\phi(e_{j})$ 
    относительно начального базиса. В силу произвольности выбора вектора $e_j$ верно: 
    $\forall j \; \phi(e_j) = e \cdot A_{*j} \Leftrightarrow \phi(e) = e \cdot A$.
\end{reminder}

\begin{example}
    Пусть $x \in V$, $x \xleftrightarrow[e]{} \alpha$, $x = e \alpha$.
    Чтобы получить координаты образа вектора $x$ под действием оператора $\phi$ нужно матрицу 
    оператора $A_{\phi}$ умножить на вектор $x$:
    $$\phi(x) = \phi(e) \alpha = e A \alpha \Rightarrow \phi(x) \xleftrightarrow[e]{} A \alpha.$$
\end{example}

\begin{reminder}
    Изменение матрицы оператора при переходе от одного базиса к другому осуществляется следующим образом:

    Пусть $e$ и $f$ -- два базиса в $V$. Пусть матрицы $\phi$ в этих базисах равны $A$ и $B$ 
    соответственно, тогда $\phi(e) = eA$, $\phi(f) = fB$.
    Матрицу перехода от $e$ к $f$ назовем $S$: $f = e \cdot S$, $|S| \neq 0$.
    Тогда: 
    $$\phi(f) = \phi(e) \cdot S = e A \cdot S = f \cdot S^{-1} A S.$$ 
    Таким образом, $B = S^{-1} A S$.
\end{reminder}

\begin{note}
    Матрицы $A$ и $S^{-1} A S$ называются подобными. Матрица любого линейного оператора 
    определена с точностью до подобия.
\end{note}

\begin{reminder}
    Пусть $V$ -- линейное пространство, $\dim V = n$. Множество всех линейных 
    операторов на $V$ обозначается как $\mathcal{L}(V)$ и является линейным пространством, 
    $\dim \mathcal{L}(V) = n^2$. \\
    Произведение (композиция) линейных операторов определяется как 
    $(\phi \psi)(x) \stackrel{def}{=} \phi(\psi(x))$, где $\phi, \psi \in \mathcal{L}(V)$. 
    Оно обладает свойством ассоциативности из-за ассоциативности перемножения матриц, а значит 
    $\mathcal{L}(V)$ является алгеброй.
\end{reminder}

\begin{definition}
    Пусть $V$ -- линейное пространство, $\phi: V \to V$. Подпространство $U \leq V$ называется 
    инвариантным если для всех $x \in U$ выполняется $\phi(x) \in U$. Другими словами, действие 
    оператора $\phi$ на вектор из $U$ не выводит его за пределы $U$, а значит 
    $\phi(U) \subset U \Leftrightarrow \phi(U) \leq U$.
\end{definition}

\begin{example}
    Рассмотрим следующие примеры инвариантных подпространств:
    \begin{enumerate}
        \item $O: x \to 0 \; \forall x \in V$. Тогда $U \leq V \Leftrightarrow O(U) = \{0\} \leq U$, 
            а значит любое подпространство $V$ является инвариантным относительно $O$.
        \item Тождественное отображение $id(x) = x$. Тогда $id(U) = U$, 
            а значит любое подпространство инвариантно.
        \item Рассмотрим $V_3$ и проекцию на подпространство $(e_1, e_2)$. Тогда инвариантными будут 
            являться все $V_3$, нулевое подпространство, а так же линейная оболочка $(e_1, e_2)$, 
            любая прямая в этой плоскости, линейная оболочка $e_3$ и линейная оболочка $e_3$
            и некоторого вектора из плоскости $(e_1, e_2)$.
    \end{enumerate}
\end{example}

\begin{proposition}
    Пусть $\phi: V \to V$ -- линейный оператор, $U$ -- инвариантное подпространство. Тогда в базисе, 
    согласованном с $V$ оператор $\phi$ имеет матрицу с левым нижним углом нулей:
    \[\phi(A) = \left(\begin{array}{@{}c|c@{}}
		A & B\\
		\hline
		0 & C
	\end{array}\right)\]
    Здесь $A \in M_k(F)$, $k = \dim U$.
\end{proposition}

\begin{proof}
    $U$ инвариантно относительно $\phi$, а значит $\phi(e_1), \phi(e_2), \dots \phi(e_k) \in U$.
    Тогда для базисного вектора из $U$ ненулевыми могут быть только первые $k$ элементов 
    соответствующего ему столбца.
\end{proof}

\begin{note}
    Блок нулей в левом нижнем углу означает, что $\phi(e_1), \phi(e_2), \dots \phi(e_k) \in U$, 
    а значит подпространство $U$ является инвариантным относительно $\phi$.
\end{note}

\begin{note}
    Чтобы блок нулей был и выше и ниже главной диагонали, необходимо и достаточно, чтобы 
    пространство раскладывалось в прямую сумму двух подпространств, 
    являющихся инвариантными относительно $\phi$.
\end{note}

\begin{theorem}
    Пусть $\phi: V \to V$, $U_1$ и $U_2$ -- инвариантные подпространства. 
    Тогда $U_1 \cap U_2$ и $U_1 \oplus U_2$ так же являются инвариантными относительно $\phi$.
\end{theorem}

\begin{proof}~
    \begin{enumerate}
        \item $\phi(U1 \cap U2) \subset \phi(U1) \cap \phi(U2) \subset U1 \cap U2$. 
        \item $\phi(U1 \oplus U2) = \phi(U1) \oplus \phi(U2) \subset U1 \oplus U2$.
    \end{enumerate}
\end{proof}

\begin{proposition}
    Для линейного оператора $\phi: V \to V$ его ядро $\ker \phi$ и образ $\im \phi$ являются 
    инвариантными подпространствами.
\end{proposition}

\begin{proof}~
    \begin{enumerate}
        \item $\phi(ker (\phi)) = \{0\} \in ker \phi$.
        \item Пусть $y \in \im \phi$. Тогда $\phi(y) \in \im \phi$, так как $y \in V$.
    \end{enumerate}
\end{proof}

\begin{theorem}[о коммутирующих линейных операторах]
    \label{th3.2}
    Пусть $\phi, \psi \in \mathcal{L}(V)$ и верно $\phi \psi = \psi \phi$. Тогда подпространства 
    $\ker \phi$, $\ker \psi$, $\im \psi$, $\im \phi$ являются инвариантными 
    относительно обоих операторов.
\end{theorem}

\begin{proof}
    Докажем, что $\ker \phi$ и $\im \phi$ инвариантны относительно $\psi$, доказательство для 
    инвариантности $\ker \psi$ и $\im \psi$ относительно $\phi$ симметрично.
    \begin{enumerate}
        \item Пусть $x \in \ker \phi$. Тогда $\phi(\psi(x)) = \psi(\phi(x)) = \psi(0) = 0$, 
        а значит $\psi(x) \in \ker \phi$.
        \item Пусть $y \in \im \phi$. Тогда $\psi(y) = \psi(\phi(x)) = \phi(\psi(x)) \in \im \phi$.
    \end{enumerate}
\end{proof}

\begin{note}
    Рассмотрим многочлен $P \in F[x]$. По определению $P(\phi) \in F[x]$. 
    Тогда по доказанной выше теореме \ref{th3.2} подпространства $\ker P(\phi)$ и $\im P(\phi)$ 
    инвариантны относительно $\phi$, так как $P(x) x = x P(x)$, 
    а значит $P(\phi) \cdot \phi = \phi \cdot P(\phi)$.
\end{note}

\begin{note}
    Пусть $U$ инвариантно относительно $\phi$, $\psi \in \mathcal{L}(V)$. 
    Тогда $U$ инвариантно так же относительно оператора $\alpha \phi + \beta \psi$, где 
    $\alpha, \beta \in F$ и операторов $\phi \psi$, $\psi \phi$. Так же если $P(x, y) \in F[x, y]$, 
    то $U$ инвариантно относительно $P(\phi, \psi)$.
\end{note}

\subsection{Собственные значения и векторы}

\begin{definition}
    Пусть $\phi: V \to V$. Ненулевой вектор 
    $x \in V: \phi(x) = \lambda x$ называется собственным вектором оператора $\phi$, 
    отвечающим собственному значению $\lambda$.
\end{definition}

\begin{definition}
    Число $\lambda \in F$ называется собственным значением оператора $\phi$, если существует собственный 
    вектор $x \in V$, такой что $\phi(x) = \lambda x$, то есть если некоторый $x$ отвечает $\lambda$.
\end{definition}

\begin{note}
    Пусть $x$ -- собственный вектор, отвечающий собственному значению $\lambda$. Тогда верно 
    $\phi(x) = \lambda x$, а значит $\phi^2(x) = \lambda^2 x$, $\dots$, $\phi^n(x) = \lambda^n x$.
\end{note}

\begin{corollary}
    Пусть $P \in F[x]$. Тогда $(P(\phi))(x) = P(\lambda) \cdot x$. В частности 
    если $P$ аннулирует $\phi$, то есть $P(\phi) = 0$, 
    то каждое собственное значение $\phi$ является корнем $P$.
\end{corollary}

\begin{definition}
    Пусть $\lambda$ -- собственное значение оператора $\phi: V \to V$. Собственным подпространством 
    оператора $\phi$, отвечающим $\lambda$ называется подпространство 
    $V_{\lambda} = \ker (\phi - \lambda \epsilon) \leq V$.
\end{definition}

\begin{proposition}
    $V_{\lambda} \neq \{0\} \Leftrightarrow \lambda$ -- собственное значение оператора $\phi$.
\end{proposition}

\begin{proof}~
    \begin{enumerate}
        \item Необходимость. Пусть $V_\lambda \neq \{0\}$, $\exists x \neq 0: \; (\phi - \lambda E)(x) = 0$. 
        Тогда $\phi(x) = \lambda x$, а значит $x$ -- собственный вектор, 
        $\lambda$ -- собственное значение.
        \item Достаточность. Пусть $\lambda$ -- собственное значение оператора $\phi$. 
        Тогда $\exists x \neq 0: \: \phi(x) = \lambda(x)$, где $\lambda$ -- собственное значение.
        Значит $(\phi - \lambda E)x = 0$, откуда $x \in \ker(\phi - \lambda E)$ и $V_\lambda \neq \{0\}.$
    \end{enumerate}
\end{proof}

\begin{note}
    Начиная с этого момента будем называть собственными подпространствами только такие $V_{\lambda}$, 
    которые отличны от нуля, что равносильно тому, что $\lambda$ -- собственное значение. 
    Действительно, доопределение собственных подпространств для несобственных $\lambda$ 
    не представляет интереса, так как они будут нулевыми.
\end{note}

\begin{reminder}
    Подпространства $U_1, U_2, \dots U_n$ называются линейно независимыми, если из равенства 
    $x_1 + x_2 + \dots + x_n = \bar{0}$, где $x_i \in U_i$, следует, что $x_1 = x_2 = \dots = x_n = 0$.
\end{reminder}

\begin{theorem}[О линейной независимости собственных подпространств, отвечающих попарно различным собственным значениям]
    \label{o_lnz}
    Пусть $\phi: V \to V$, $\lambda_1, \lambda_2, \dots \lambda_n$ -- различные собственные значения. 
    Тогда $V_{\lambda_1}, ..., V_{\lambda_n}$ линейно независимы.
\end{theorem}

\begin{proof}
    От противного. Пусть существует такой набор
    $x_1 \in V_{\lambda_1}, x_2 \in V_{\lambda_2}, \dots, 
    x_n \in V_{\lambda_n}$, что хотя бы один вектор ненулевой, но $x_1 + x_2 + \dots + x_n = \bar{0}$. 

    Назовем все такие наборы опровергающими, а мощностью набора будем считать количество ненулевых 
    векторов. Из всех подходящих наборов выберем один наименьшей мощности. Пусть 
    указанный выше набор без ограничения общности -- искомый. Перенумеруем множества и $x_i$ так,
    чтобы ненулевыми были первые $j$ векторов. Тогда $x_1 + x_2 + \dots + x_j = 0$, и все
    $x_i \neq 0$ в силу перенумерации. Применим к сумме оператор $\phi$:
    $$\lambda_1 x_1 + \lambda_2 x_2 + ... + \lambda_j x_j = 0.$$ 
    Умножим изначальную сумму на $-\lambda_1$ и сложим с получившейся: 
    \begin{align*}
        (\lambda_1 x_1 + \lambda_2 x_2 + \, \dots \, + \lambda_j x_j) + (-\lambda_1 x_1 - \lambda_2 x_2 
        - \, \dots \, - \lambda_j x_j) = \\ = (\lambda_2 - \lambda_1) x_2 + \, \dots \, + (\lambda_j - 
        \lambda_1) x_j = 0.
    \end{align*}
    Таким образом мы получили опровергающий набор меньшей мощности, что приводит к противоречию.
\end{proof}

\subsubsection{Нахождение собственных подпространств и векторов}  

\begin{algorithm}[Нахождения собственных векторов]~

    Пусть в $V$ фиксирован базис $e = (e_1, e_2, \dots, e_n)$, оператор $\phi$ имеет матрицу $A$.
    Тогда собственным будет являться такой вектор $x$, что $Ax = \lambda x$:
    \begin{equation*}
        \left(
            \begin{array}{cccc}
            a_{11} & a_{12} & \ldots & a_{1n}\\
            a_{21} & a_{22} & \ldots & a_{2n}\\
            \vdots & \vdots & \ddots & \vdots\\
            a_{n1} & a_{n2} & \ldots & a_{nn}
            \end{array}
        \right) \left(
            \begin{array}{c}
            x_{1}\\
            x_{2}\\
            \vdots\\
            x_{n}
            \end{array}
        \right) = \left(
            \begin{array}{c}
            \lambda x_{1}\\
            \lambda x_{2}\\
            \vdots\\
            \lambda x_{n}
            \end{array} 
        \right)
    \end{equation*}
    Данное равенство равносильно следующему:
    \begin{equation*}
        \left(
            \begin{array}{cccc}
            a_{11} - \lambda & a_{12}           & \ldots & a_{1n}\\
            a_{21}           & a_{22} - \lambda & \ldots & a_{2n}\\
            \vdots           & \vdots           & \ddots & \vdots\\
            a_{n1}           & a_{n2}           & \ldots & a_{nn} - \lambda
            \end{array}
        \right) \left(
            \begin{array}{c}
            x_{1}\\
            x_{2}\\
            \vdots\\
            x_{n}
            \end{array}
        \right) = \left(
            \begin{array}{c}
            0\\
            0\\
            \vdots\\
            0
            \end{array} 
        \right)
    \end{equation*}
    По теореме Крамера наличие у данной системы ненулевого решения равносильно ненулевому определителю
    $\det (A - \lambda E) = 0$. Такой определитель называется характеристическим многочленом 
    оператора $\phi$ относительно базиса $e$:
    $$\chi(\lambda) = (-1)^n \lambda^n + (-1)^{n-1} tr A \lambda^{n-1} + \dots + det A.$$
    Таким образом, собственными значениями будут $\lambda$, являющиеся корнями характеристического 
    многочлена. Собственными будут векторы, являющиеся решениями систем с подставленными 
    собственными значениями $\lambda$.
\end{algorithm}

\begin{theorem}
    Верны следующие свойства характеристического многочлена:
    \begin{enumerate}
        \item Корни $\chi(\lambda)$ принадлежащие полю $F$ и только они являются собственными 
        значениями $\phi$.
        \item Многочлен $\chi(\lambda)$ не зависит от выбора базиса.
    \end{enumerate} 
\end{theorem}

\begin{proof}~
    \begin{enumerate}
        \item Пусть $\lambda_0$  – корень $\chi_{\phi}(\lambda)$, тогда $|A - \lambda_0 E| = 0$. 
        Это значит, что система $A - \lambda_0 E$, имеет ненулевое решение при $x_0 \neq 0$.
        Тогда $\phi(x_0) = \lambda_0$, а значит $\lambda_0$ - собственное значение $\phi$.
        В силу равносильных переходов, обратное утверждение тоже верно.

        \item Наряду с $e$ выберем базис $f$, обозначим за $S$ матрицу перехода между ними: 
        $S = S_{e \to f}$. Тогда $\phi \xleftrightarrow[e]{} A$, $\phi \xleftrightarrow[f]{} B$, 
        $B = S^{-1}AS$. Верна следующая цепочка равенств:
        \begin{align*}
            \chi_b(\lambda) = |B &- \lambda E| = |S^{-1}AS - \lambda E| 
            = |S^{-1}AS - S^{-1} \lambda E S| = \\ &= |S^{-1}(A - \lambda E)S| 
            = |S^{-1}| \cdot |A - \lambda E| \cdot |S| = |A - \lambda E| = \chi_a(\lambda).
        \end{align*}
        Таким образом, характеристический многочлен одинаков для всех базисов.
    \end{enumerate}
\end{proof}

\begin{definition}
    В дальнейшем будем использовать следующее определение: характеристическим многочленом оператора 
    $\phi: V \to V$ называется определитель $|A - \lambda E| = \chi_A(\lambda)$,
    где $A$ -- матрица $\phi$ в произвольном базисе.
\end{definition}

\begin{corollary}
    От выбора базиса не зависят так же коэффициенты характеристического многочлена, в частности 
    $det A$ и $tr A$, поэтому часто пишут $det \phi$ и $tr \phi$ соответственно.
\end{corollary}


\begin{algorithm}[Нахождения собственных подпространств]~

    Выпишем $\chi_{\phi}(\lambda)$ и найдем его корни принадлежащие $F$.
    Пусть $\lambda_1, \lambda_2,  \dots \lambda_k$ -- полученные собственные значения.
    Для каждого собственного значения $\lambda_j$ находим $\ker B_j = \ker (A - \lambda E)$ -- 
    подпространство таких векторов, для которых $Ax = \lambda x$. Оно очевидно совпадает с $V_j$.
\end{algorithm}

\begin{corollary}
    Если $V$ -- линейное пространство над $\Cm$, $\dim V \geq 1$, то всякий линейный оператор $\phi: V \to V$ имеет в $V$ имеет хотя бы один собственный вектор.
\end{corollary}

\begin{proof}
    Пусть $\chi_{\phi}(t)$ -- характеристический многочлен. По теореме \ref{ota} многочлен $\chi_{\phi}(t)$ 
    имеет хотя бы один корень $\lambda_0$. Следовательно, $\lambda_0$ -- 
    собственное значение для оператора $\phi$ и, значит, существует собственный вектор с собственным 
    значением $\lambda_0$.
\end{proof}

\begin{corollary}
    Если $V$ - линейное пространство над $\R$ и $\dim V = 2k + 1, k \geq 0, k \in \Z$, то всякий линейный оператор $\phi: V \to V$ имеет хотя бы один собственный вектор. 
\end{corollary}

\begin{proof}
    Аналогично предыдущему следствию в таком пространстве существует хотя бы один корень $\lambda_0 \in \R$.
\end{proof}
