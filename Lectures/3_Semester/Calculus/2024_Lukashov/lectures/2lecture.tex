\section{Кратные интегралы}

\begin{designation}
	Когда функция задаётся записью $f \colon E \to \R$, то, если не сказано явно иного, считается, что она определена на $E$.
\end{designation}

\begin{note}
	Мы будем довольно часто опускать факты, которые неявным образом подразумеваются (в силу определения, например). Так
	\begin{itemize}
		\item Если функция $f \colon E \to \R$, $E \subseteq \R^n$ измерима на $E$, то измеримость $E$ подразумевается по определению измеримой функции.
		
		\item Список может дополняться
	\end{itemize}
\end{note}

\subsection{Определение кратного интеграла}

\begin{designation}
	В этом параграфе, если не обговорено обратного, обозначение $E$ используется для подмножества $\R^n$ \textbf{конечной меры}.
\end{designation}

\begin{definition}
	\textit{Разбиением $P$ множества $E$} называется его произвольное представление в виде конечного числа непересекающихся измеримых по Лебегу (Жордану) множеств. Обозначение такое:
	\[
		P \colon E = \bscup_{i = 1}^N E_i
	\]
\end{definition}

\begin{anote}
	Когда мы не обращаемся к конкретному виду разбиения, я буду обозначать разбиение $P$ над $E$ как $P(E)$.
	
	Дополнительно будет полезно знать, что разбиение $P$ можно воспринимать как $P = \{E_i\}$
\end{anote}

\begin{designation}
	Если $f$ определена на $E$, то инфинумы и супремумы функции $f$ на разбиении $P(E)$ будут обозначаться так:
	\begin{align*}
		&{M_k := \sup_{x \in E_k} f(x)}
		\\
		&{m_k := \inf_{x \in E_k} f(x)}
	\end{align*}
	где $E_k$ - это какой-то элемент разбиения $P$.
	
	Аналогично $M$ и $m$ - это супремум и инфинум функции по всему $E$.
\end{designation}

\subsubsection*{Определение интеграла Лебега (Римана) в случае ограниченной функции}

\begin{note}
	В рамках этой части, если не оговорено обратного, $f$ --- ограниченная и определенная на $E$ функция. Местами это оговорено явно, чтобы картина была более ясной.
\end{note}

\begin{definition}
	Пусть $P \colon E = \bscup_{i = 1}^N E_i$. Тогда \textit{нижней суммой Дарбу (Дарбу-Лебега)} мы будем называть следующую величину:
	\[
		L(P, f) := \sum_{k = 1}^N m_k \cdot \mu(E_k)
	\]
	Аналогично есть \textit{нижняя сумма Дарбу-Римана}:
	\[
		L_R(P, f) := \sum_{k = 1}^N m_k \cdot \jm(E_k)
	\]
\end{definition}

\begin{definition}
	Пусть $P \colon E = \bscup_{i = 1}^N E_i$. Тогда \textit{верхней суммой Дарбу (Дарбу-Лебега)} мы будем называть следующую величину:
	\[
	U(P, f) := \sum_{k = 1}^N M_k \cdot \mu(E_k)
	\]
	Аналогично есть \textit{верхняя сумма Дарбу-Римана}:
	\[
	U_R(P, f) := \sum_{k = 1}^N M_k \cdot \jm(E_k)
	\]
\end{definition}

\begin{note}
	Чтобы не дублировать один и тот же текст, далее мы будем сжимать определения для Лебега и Римана. Чтобы получить из определения Лебега определение Римана, надо вместо соответствующих частей исходного текста читать то, что в скобках. С формулами аналогично мерам: $U_\comR, L_\comR$ и так далее.
\end{note}

\begin{definition}
	\textit{Верхним интегралом Лебега (Римана) функции $f$ по множеству $E$} называется инфинум по всем разбиениям верхних сумм Дарбу-Лебега (-Римана):
	\[
		\ole{I}_\comR := \inf_{P(E)} U_\comR(P, f)
	\]
\end{definition}

\begin{definition}
	\textit{Нижним интегралом Лебега (Римана) функции $f$ по множеству $E$} называется супремум по всем разбиениям нижних сумм Дарбу-Лебега (-Римана):
	\[
		\ule{I}_\comR := \sup_{P(E)} L_\comR(P, f)
	\]
\end{definition}

\begin{definition}
	Если нижний и верхний интегралы Лебега (Римана) $f$ по $E$ существуют и они равны друг другу, то их общее значение называется \textit{интегралом Лебега (Римана) функции $f$ по множеству $E$}. Обозначается так:
`	\[
		\ole{I} = \ule{I} = \int_E f(x) d\mu(x); \quad \ole{I}_R = \ule{I}_R = \int_E f(x)dx
	\]
\end{definition}

\begin{definition}
	Пусть есть разбиения $P_1 \colon E = \bscup_{i = 1}^{N_1} E_{i, 1}$ и $P_2 \colon E = \bscup_{i = 1}^{N_2} E_{i, 2}$. Тогда \textit{измельчением} $P_1$ и $P_2$ мы назовём следующее разбиение:
	\[
		P_1 \cup P_2 \colon E = \bscup_{i = 1}^{N_1} \bscup_{j = 1}^{N_2} (E_{i, 1} \cap E_{j, 2})
	\]
\end{definition}

\begin{proposition}
	С обозначениями из определения измельчения имеют смысл такие неравенства:
	\[
		L_\comR(P_1, f) \le L_\comR(P_1 \cup P_2, f) \le U_\comR(P_1 \cup P_2, f) \le U_\comR(P_1, f)
	\]
	На концах, естественно, можно поставить и $P_2$ вместо $P_1$.
\end{proposition}

\begin{proof}
	Начнём с доказательства крайних неравенств. Покажем его для нижних сумм, ибо для верхних абсолютно аналогично.
	
	Распишем нижние суммы по определению:
	\begin{align*}
		&{L_\comR(P_1, f) = \sum_{i = 1}^{N_1} \ps{\inf_{x \in E_{i, 1}} f(x)} \cdot \jlm(E_{i, 1})}
		\\
		&{L_\comR(P_1 \cup P_2, f) = \sum_{i = 1}^{N_1} \sum_{j = 1}^{N_2} \ps{\inf_{x \in E_{i, 1} \cap E_{j, 2}} f(x)} \cdot \jlm(E_{i, 1} \cap E_{j, 2})}
	\end{align*}
	При этом $\inf_{x \in E_{i, 1}} f(x) \le \inf_{x \in E_{i, 1} \cap E_{j, 2}} f(x)$. Отсюда получаем искомое неравенство:
	\begin{multline*}
		L_\comR(P_1 \cup P_2, f) \ge \sum_{i = 1}^{N_1} \sum_{j = 1}^{N_2} \ps{\inf_{x \in E_{i, 1}} f(x)} \cdot \jlm(E_{i, 1} \cap E_{j, 2}) =
		\\
		\sum_{i = 1}^{N_1} \ps{\inf_{x \in E_{i, 1}} f(x)} \sum_{j = 1}^{N_2} \jlm(E_{i, 1} \cap E_{j, 2}) = \sum_{i = 1}^{N_1} \ps{\inf_{x \in E_{i, 1}} f(x)} \jlm(E_{i, 1}) = L_\comR(P_1, f)
	\end{multline*}
	Неравенство $L_\comR(P_1 \cup P_2, f) \le U_\comR(P_1 \cup P_2, f)$ полагается тривиальным.
\end{proof}

\begin{proposition} \label{commonEquivLebeg}
	$f$ интегрируема по Лебегу (Риману) по $E$ тогда и только тогда, когда верно следующее условие:
	\[
		\forall \eps > 0\ \exists P(E) \such U_\comR(P, f) - L_\comR(P, f) < \eps
	\]
\end{proposition}

\begin{proof}
	Проведём доказательство в случае интегрируемости по Лебегу.
	\begin{itemize}
		\item $\Ra$ Коль скоро $f$ интегрируема по Лебегу, то определены равные верхний и нижний интеграл Лебега. Общее значение обозначим $I$, тогда по свойствам инфинума и супремума можно записать следующее:
		\begin{align*}
			&{\forall \eps > 0\ \exists P_1(E) \colon U(P_1, f) < I + \frac{\eps}{2}}
			\\
			&{\forall \eps > 0\ \exists P_2(E) \colon L(P_2, f) > I - \frac{\eps}{2}}
		\end{align*}
		Зафиксируем $\eps > 0$, тогда найдутся $P_1$ и $P_2$. Положим $P = P_1 \cup P_2$ и покажем, что это искомое разбиение:
		\[
			U(P, f) - L(P, f) \le U(P_1, f) - L(P_2, f) < I + \frac{\eps}{2} - I + \frac{\eps}{2} = \eps
		\]
		
		\item $\La$ Коль скоро $f$ ограничена на измеримом множестве конечной меры, суммы Дарбу всегда будут конечными числами, то есть определены верхний и нижний интегралы Лебега. При этом:
		\[
			\forall P(E)\ L(P, f) \le \ule{I} \le \ole{I} \le U(P, f)
		\]
		Стало быть, при помощи условия про разность интегралов можно сказать следующее:
		\[
			\forall \eps > 0\ \exists P(E) \such \ole{I} - \ule{I} \le U(P, f) - L(P, f) < \eps
		\]
	\end{itemize}
\end{proof}

\begin{proposition}
	$f \colon [a; b] \to \R$ интегрируема по Риману на $[a; b]$ в старом смысле тогда и только тогда, когда она интегрируема по Риману по $[a; b]$ в новом смысле.
\end{proposition}

\begin{proof}
	\textcolor{red}{Когда-нибудь обязательно.}
\end{proof}

\begin{proposition}
	Если функция $f \colon E \to \R$ интегрируема по Риману по $E$, то она интегрируема по Лебегу по $E$. Более того, интегралы Римана и Лебега совпадают.
\end{proposition}

\begin{proof}
	\textcolor{red}{Доказательство написано после критерия интегрируемости по Риману.}
\end{proof}

\begin{definition}
	Пусть $f \colon E \to \R$ --- измеримая на $E$ функция. \textit{Разбиением Лебега}, соответствующим разбиению $Q = \{m = y_0 < \ldots < y_q = M\}$, называется разбиение $P_Q(E)$ следующего вида:
	\[
		P_Q \colon E = \ps{\bscup_{i = 1}^{q - 1} \underbrace{\{x \in E \colon f(x) \in \lsi{y_{i - 1}; y_i}\}}_{E_i}} \sqcup \underbrace{\{x \in E \colon f(x) \in [y_{q - 1}; y_q]\}}_{E_q}
	\]
\end{definition}

\begin{note}
	Разбиение Лебега является ключевым в теории интеграла Лебега. По нему также понятно, зачем мы требуем измеримость функции: $E_i = f^{-1}(\lsi{y_{i - 1}; y_i})$ должно быть подмножеством $E$, причём измеримым по Лебегу, чтобы разбиение Лебега соответствовало понятию разбиения.
\end{note}

\begin{proposition} \label{uminl-le-dqe}
	Для разбиения Лебега множества $E$ и измеримой на $E$ функции $f$ верно следующее неравенство:
	\[
		U(P_Q, f) - L(P_Q, f) \le \Delta Q \mu(E)
	\]
	где $\Delta(Q)$ --- диаметр разбиения $Q$.
\end{proposition}

\begin{anote}
	Лишний раз отмечу, что $f$ не только измерима на $E$, но и ограничена и определена на $E$ в силу того замечания, что написано в конце страницы 2.
\end{anote}

\begin{proof}
	Распишем левую часть по определению:
	\[
		U(P_Q, f) - L(P_Q, f) = \sum_{i = 1}^q (M_i - m_i)\mu(E_i) \le \sum_{i = 1}^q \Delta Q\mu(E_k) = \Delta Q \mu(E)
	\]
\end{proof}

\begin{definition}
	\textit{Интегральной суммой Лебега (Римана)} называется величина $S(P, f, \{t_k\})$, где $P \colon E = \bscup_{k = 1}^N E_k$, $t_k \in E_k$:
	\[
		S(P, f, \{t_k\}) := \sum_{k = 1}^N f(t_k) \jlm(E_k)
	\]
\end{definition}

\begin{theorem} (Основная теорема об интеграле Лебега для ограниченных и измеримых функций)
	Если $f \colon E \to \R$ и верны условия:
	\begin{enumerate}
		\item $E \subset \R^n$ --- измеримое по Лебегу множество конечной меры
		
		\item $f$ ограничена
		
		\item $f$ измерима на $E$
	\end{enumerate}
	Тогда $f$ интегрируема по Лебегу по множеству $E$, причём
	\[
		\int_E f(x)d\mu(x) = \lim_{\Delta Q \to 0} S(P_Q, f, \{t_k\})
	\]
	где $P_Q$ --- это разбиение Лебега. Равенство в кванторной форме имеет вид:
	\[
		\forall \eps > 0\ \exists \delta > 0 \such \forall Q, \Delta(Q) < \delta;\ \ \forall \{t_k\}, t_k \in E_k \quad \mo{\int_E f(x)d\mu(x) - S(P_Q, f, \{t_k\})} < \eps
	\]
\end{theorem}

\begin{reminder}
	Если поменять в определении предела последнее неравенство на нестрогое, то получится эквивалентное определение.
\end{reminder}

\begin{proof}
	Если $\mu(E) = 0$, то все верхние и нижние суммы равны нулю, и утверждение теоремы очевидно. Далее будем считать $\mu(E) > 0$.
	\begin{enumerate}
		\item Существование интеграла. Для этого воспользуемся утверждением $\ref{uminl-le-dqe}$. Для любого $\eps > 0$ нам достаточно взять разбиение $Q$, что $\Delta Q < \delta := \eps / \mu(E)$, ибо тогда
		\[
			U(P_Q, f) - L(P_Q, f) \le \Delta Q \mu(E) = \eps
		\]
		Тогда, по утверждению \ref{commonEquivLebeg} $f$ интегрируема по Лебегу.
		
		\item Сходимость предела к интегралу. Заметим такое соотношение:
		\[
			\forall P(E), \{t_k\} \quad L(P, f) \le S(P, f, \{t_k\}) \le U(P, f)
		\]
		Стало быть, предел сходится к общему значению верхнего и нижнего интегралов в силу зажатости между верхней и нижней суммами.
	\end{enumerate}
\end{proof}

\subsubsection*{Определение интеграла Лебега в случае неотрицательной неограниченной измеримой функции}

\begin{designation}
	$\sxR = \R \cup \{\pm\infty\}$, $\sxR_{\ge 0} = \R_{\ge 0} \cup \{+\infty\}$
\end{designation}

\begin{note}
	В рамках этой части, если не оговорено обратного, $f \colon E \to \sxR_{\ge 0} \cup \{+\infty\}$ --- неотрицательная, неограниченная и измеримая функция, определенная в каждой точке $E$ и способная принимать значение $+\infty$.
\end{note}

\begin{designation}
	Когда мы говорим о таких функциях, то без дополнительной оговорённости в разбиении Лебега всегда будет $E_0 := \{x \in E \colon f(x) = +\infty\}$.
\end{designation}

\begin{note}
	\textbf{Если не оговорено обратного, $\mu(E_0) = 0$. Это правило распространяется на всю главу как минимум.}
\end{note}

\begin{note}
	Чтобы ввести интеграл здесь, нам потребуются уже счётные разбиения (как обычное, так и лебегово) и соответствующие суммы:
	\begin{itemize}
		\item Счётное простое разбиение имеет вид $P \colon E = \bscup_{i = 1}^\infty E_i$
		
		\item Счётное разбиение Лебега, соответствующее, естественно, счётному разбиению $Q \colon m = y_0 < \ldots < \ldots$, будет выглядеть как $P \colon E = E_0 \sqcup \ps{\bscup_{i = 1}^\infty \underbrace{\{x \in E \colon f(x) \in \lsi{y_{i - 1}; y_i}\}}_{E_i}}$
		
		\item Нижняя и верхняя суммы Дарбу-Лебега имеют соответственно вид
		\[
			L(P, f) = \sum_{i = 0}^\infty m_i \mu(E_i); \quad U(P, f) = \sum_{i = 0}^\infty M_i \mu(E_i)
		\]
		
		\item Интегральная сумма $S(P, f, \{t_k\})$ принимает вид
		\[
			S(P, f, \{t_k\}) = \sum_{k = 0}^\infty f(t_k) \mu(E_k)
		\]
	\end{itemize}
\end{note}

\begin{note}
	Чтобы суммы имели смысл, мы требуем следующие равенства:
	\begin{align*}
		&{0 \cdot (+\infty) = 0}
		\\
		&{\forall a \in \R\ a + (\pm\infty) = \pm\infty}
	\end{align*}
	Тогда все утверждения (без теорем) остаются верными для таких расширений предыдущих определений. Оставшиеся определения естественным образом подгоняются на текущие.
\end{note}

\begin{anote}
	Возникает справедливый вопрос: <<А почему, если дополнить $\R$ положительной бесконечностью и таким соглашением на умножение, это останется непротиворечивой и корректной системой?>> Как я выяснил, ответ на этот вопрос лежит в разделе математики, носящим название \textit{нестандартного анализа}. Утверждается, что с наложенными нами условиями, можно достроить (попутно добавив какие-то ещё элементы, в том числе \textit{подобия} бесконечности) до корректной арифметики. Так как мы не будем использовать деление и умножение произвольного числа на бесконечность, то с остальной частью корректной системы мы не познакомимся.
	
	Лично мне такое определение интеграла Лебега не нравится, но можно считать, что таким образом мы делаем его эквивалентным с определением интеграла Лебега из предмета <<Основы Вероятности и Теории Мер>>.
\end{anote}

\begin{theorem} (Основная теорема для неотрицательных, неограниченных и измеримых функций)
	Если дана функция $f \colon E \to \sxR_{\ge 0}$ и верны условия:
	\begin{enumerate}
		\item $E \subset \R^n$ --- измеримое по Лебегу множество конечной меры
		
		\item $f$ --- неограниченная функция
		
		\item $f$ --- измеримая на $E$ функция
	\end{enumerate}
	Тогда $f$ интегрируема по Лебегу на $E$, причём интеграл равен пределу интегральных сумм:
	\[
		I = \int_E f(x)d\mu(x) = \lim_{\Delta Q \to 0} S(P_Q, f, \{t_k\})
	\]
\end{theorem}

\begin{proof}~
	\begin{enumerate}
		\item Существование интеграла. Работает ровно то же самое, что написано в основной теореме для ограниченных функций.
		
		\item Сходимость предела к интегралу. Если $I \neq +\infty$ всё тривиально, а в бесконечном $\ule{I} = \sup_{P(E)} L(P, f) = +\infty$. Так как следующее неравенство всё ещё в силе:
		\[
			L(P, f) \le S(P, f, \{t_k\}) \le U(P, f)
		\]
		то предел интегральных сумм тоже будет в бесконечности и соответственно равен интегралу.
	\end{enumerate}
\end{proof}

\begin{definition}
	Если интеграл $\int_E f(x)d\mu(x)$ конечен, то $f$ называется \textit{суммируемой на $E$}.
\end{definition}

\begin{anote}
	Не забываем учитывать все остальные свойства, которые мы наложили на $f$ и $E$ в этой части.
\end{anote}

\begin{proposition}
	$f$ суммируется на $E$ тогда и только тогда, когда верно условие:
	\[
		\forall \eps > 0\ \exists P(E) \such U(P, f) - L(P, f) < \eps
	\]
\end{proposition}

\begin{note}
	Аналогичное определение и утверждение верны и в случае, когда $f \colon E \to \R$ --- ограниченная, измеримая на $E$ функция.
\end{note}

\subsubsection*{Определение интеграла Лебега для произвольных измеримых функций}

\begin{note}
	В данной части $f \colon E \to \R$ --- произвольная функция, измеримая на $E$.
\end{note}

\begin{definition}
	Пусть $f \colon E \to \R$ --- произвольная функция. Тогда, мы положим по определению за $f^+$ и $f^-$ следующие функции:
	\begin{align*}
		&{f^+(x) := \max(f(x), 0)}
		\\
		&{f^-(x) := -\min(f(x), 0) = \max(-f(x), 0)}
	\end{align*}
\end{definition}

\begin{note}
	Понятно, что $f^+$ и $f^-$ --- неотрицательные функции.
\end{note}

\begin{proposition}
	Если дана функция $f \colon E \to \R$, то $f = f^+ - f^-$
\end{proposition}

\begin{proof}
	Тривиально.
\end{proof}

\begin{proposition}
	Если $f \colon E \to \R$ --- измеримая функция, то $f^+$ и $f^-$ --- тоже
\end{proposition}

\begin{proof}
	Действительно, функции $\max$ и $\min$ являются непрерывными функциями $\R^2 \to \R$, поэтому их композиции с $f$ будет измеримыми, как и разность.
\end{proof}

\begin{definition}
	Если есть произвольная функция $f \colon E \to \R$, причём выполнены следующие условия:
	\begin{enumerate}
		\item $E \subset \R^n$ --- измеримое по Лебегу множество конечной меры
		
		\item $f$ --- измеримая на $E$
	\end{enumerate}
	Тогда $f$ называется \textit{интегрируемой по Лебегу по $E$}, если хотя бы одна из функций $f^+, f^-$ суммируется на $E$. Значением интеграла по определению считается следующее:
	\[
		\int_E f(x)d\mu(x) := \int_E f^+(x)d\mu(x) - \int_E f^-(x)d\mu(x)
	\]
\end{definition}

\begin{definition}
	Если при условиях последнего определения и $f^+$, и $f^-$ оказались одновременно суммируемыми на $E$, то говорят, что \textit{$f$ суммируется на $E$}.
\end{definition}

\begin{example} (Интеграл Лебега функции Дирихле)
	Напомним, что из себя представляет $D(x)$:
	\[
		D(x) := \System{
			&{1,\ x \in \Q}
			\\
			&{0,\ x \in \R \bs \Q}
		}
	\]
	Найдём интеграл от этой функции на отрезке $[0; 1]$ (она точно интегрируема по основной теореме для ограниченных функций): $m = 0$, $M = 1$. Для любого лебегова разбиения $P_Q$ точек со значениями из $(0; 1)$ нет, если же берётся часть $E_1 = \{x \in E \colon f(x) \in \lsi{0; y_1}\}$, у любой точки из $E_1$ значение 0, а для $E_q = \{x \in E \colon f(x) \in [y_{q - 1}; 1]\}$ значение в любой точке будет 1, но $\mu(E_q) = \mu(\Q \cap [0; 1]) = 0$. Следовательно
	\[
		\int_{[0; 1]} D(x)d\mu(x) = \lim_{\Delta Q \to 0} S(P_Q, f, \{t_k\}) = 0
	\]
\end{example}
