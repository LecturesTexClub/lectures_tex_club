\begin{corollary}
	Большинство интегральных соотношений, похожих на теорему Стокса-Пуанкаре, являются просто её следствием:
	\begin{enumerate}
		\item $m = 2$, $n = 3$. Тогда формула из теоремы Стокса-Пуанкаре называется просто \textit{формулой Стокса}:
		\begin{multline*}
			\int_M
			\ps{\pd{R}{y} - \pd{Q}{z}} dy \wedge dz +
			\ps{\pd{P}{z} - \pd{R}{x}} dz \wedge dx +
			\ps{\pd{Q}{x} - \pd{P}{y}} dx \wedge dy = \\
			= \int_{\vdelta M} P dx + Q dy + R dz 
		\end{multline*}
		\item $m = 3 = n$. В этом особом случае мы говорим о 2-формах. Обозначим одну из таких $\Omega$ и пусть она имеет следующий вид:
		\[
			\Omega = P dy \wedge dz + Q dz \wedge dx + R dx \wedge dy
		\]
		Тогда дифференциал может быть лаконично записан так:
		\[
			d\Omega = \ps{\pd{P}{x} + \pd{Q}{y} + \pd{R}{z}}dx \wedge dy \wedge dz
		\]
		Если подставить эти выражения в формулу Стокса-Пуанкаре, то получим \textit{формулу Гаусса-Остроградского в терминах дифференциальных форм}:
		\[
			\int_M \ps{\pd{P}{x} + \pd{Q}{y} + \pd{R}{z}} dx \wedge dy \wedge dz = \int_{\vdelta M} P dy \wedge dz + Q dz \wedge dx + R dx \wedge dy
		\]
		
		\item $m = 2 = n$. В этом случае у нас всего лишь 1-формы. Пусть она имеет такой вид:
		\[
			\Omega = Pdx + Qdy \Lora d\Omega = \ps{\pd{Q}{x} - \pd{P}{y}} dx \wedge dy
		\]
		Если подставить эти выражения в формулу Стокса-Пуанкаре, то получится \textit{формула Грина в терминах дифференциальных форм}:
		\[
			\int_{\vdelta M} Pdx + Qdy = \int_M \ps{\pd{Q}{x} - \pd{P}{y}} dx \wedge dy
		\]
	\end{enumerate}
\end{corollary}

\begin{note}
	На время забудем о всяких многообразиях, а вернёмся к простому евклидову пространству $\R^m$ со стандартным скалярным произведением $\tbr{\cdot, \cdot}$ и ортонормированным базисом $e_0$.
	
	Если я хочу выяснить объём призмы $\Pi$, натянутой на векторы $\{\vv{H}_i\}_{i = 1}^m$, то это записывается следующим образом:
	\[
		\mu(\Pi) = \vol(\vv{H}_1, \ldots, \vv{H}_m) = |\det(H_i^j)|
	\]
	где $\vv{H}_i = H_i^j \vv{e}_0^j$ (используем соглашение Эйнштейна). С другой стороны, в курсе алгебры мы говорили о матрице Грама для системы векторов, заданной следующим образом:
	\[
		\Gamma(\vv{H}_1, \ldots, \vv{H}_m) = \Matrix{
			&\ntbr{\vv{H}_1, \vv{H}_1}& &\cdots& &\ntbr{\vv{H}_1, \vv{H}_m}
			\\
			&\quad\;\ \vdots& &\ddots& &\quad\;\ \vdots
			\\
			&\ntbr{\vv{H}_m, \vv{H}_1}& &\cdots& &\ntbr{\vv{H}_m, \vv{H}_m}
		}
	\]
	В силу того, что мы используем стандартное скалярное произведение, эта матрица является произведением матрицы столбцов системы векторов на саму себя транспонированную. Иначе говоря, имеет место равенство:
	\[
		(\det(H_i^j))^2 = \det\Gamma(\vv{H}_1, \ldots, \vv{H}_m)
	\]
	Следовательно, искомый объём призмы можно записать ещё и так:
	\[
		\mu(\Pi) = \vol{\vv{H}_1, \ldots, \vv{H}_m} = |\det(H_i^j)| = \sqrt{\det\Gamma(\vv{H}_1, \ldots, \vv{H}_m)}
	\]
	Удобство матрицы Грама состоит в том, что она позволяет находить объём призмы не только в смысле всеобъемлющего пространства, но и объём в подпространстве, порождённом произвольным набором векторов. В нашем случае это будет касательное пространство в каждой точке многообразия.
\end{note}

\begin{definition}
	\textit{Формой ориентированного объёма на клетке} $M$ называется такая $m$-форма $V$, что
	\[
		\forall \{\vv{G}_i\}_{i = 1}^m \subset T(x)\ \ V(x)(\vv{G}_1, 	\ldots, \vv{G}_m) = \pm \vol(\vv{G}_1, \ldots, \vv{G}_m)
	\]
	причём, если система $(\vv{G}_1, \ldots, \vv{G}_m)$ образует базис в $T(x)$, то знак выбирается в соответствии с его ориентацией.
\end{definition}

\begin{proposition}
	Если $M \subseteq \R^n$ --- клетка со стандартным скалярным произведением в $\R^n$, $\phi \colon \R^m \to \R^n$ - положительная параметризация $M$, то имеет место формула:
	\[
		V(x) = \sqrt{\det\tbr{\pd{\phi}{u_i}(u), \pd{\phi}{u_j}(u)}} \cdot \psi^*(du^1 \wedge \ldots \wedge du^m)
	\]
\end{proposition}

\begin{anote}
	Отмечу явный вид вектора в определителе:
	\[
		\pd{\phi}{u_i}(u) = \ps{\pd{\phi_1}{u_i}(u), \ldots, \pd{\phi_n}{u_i}(u)}^T
	\]
	Из того, что $u = \psi(x)$ следует, что весь квадратный корень является просто функцией, зависящей только от $x$.
\end{anote}

\begin{proof}
	Заметим, что при любом фиксированном $x$ линейное пространство $m$-форм над $T(x)$ одномерно. Это значит, что форма ориентированного объёма может быть выражена через базисную форму и какой-то коэффициент. Например, так:
	\[
		V(x) = \alpha(x) \cdot \psi^*(du^1 \wedge \ldots \wedge du^m)
	\]
	Посмотрим на $\{\vv{H}_i\}_{i = 1}^m$ --- ортонормированный базис в пространстве $\R^m$, двойственный к $\{du^i\}_{i = 1}^m$. Тогда у нас есть репер $\{\phi'(u)\vv{H}_i\}_{i = 1}^m$, чей конкретный набор векторов в $x = \phi(u)$ образует положительный базис (в силу определения положительной параметризации). С одной стороны (по определению):
	\[
		V(x)(\phi'(u)\vv{H}_1, \ldots, \phi'(u)\vv{H}_m) = \sqrt{\det\ntbr{\phi'(u)\vv{H}_i, \phi'(u)\vv{H}_j}}
	\]
	где отдельно взятый вектор $\phi'(u)\vv{H}_i$ будет просто вектором $\pd{\phi}{u_i}(u)$. С другой стороны, можем подставить эти вектора в найденный вид формы $V$:
	\begin{multline*}
		V(x)(\phi'(u)\vv{H}_1, \ldots, \phi'(u)\vv{H}_m) = \alpha(x) \cdot \psi^*(du^1 \wedge \ldots \wedge du^m)(\phi'(u)\vv{H}_1, \ldots, \phi'(u)\vv{H}_m) =
		\\
		\alpha(x) \cdot (du^1 \wedge \ldots \wedge du^m)(\vv{H}_1, \ldots, \vv{H}_m)
	\end{multline*}
	Стало быть, так как $\{\vv{H}_i\}_{i = 1}^m$ был ортонормированным базисом, двойственным к $\{du^i\}_{i = 1}^m$, то значение формы $du^1 \wedge \ldots \wedge du^m$ на нём будет просто единицей. Таким образом, мы нашли коэффициент $\alpha(x)$:
	\[
		\alpha(x) = \sqrt{\det\tbr{\pd{\phi}{u_i}(u), \pd{\phi}{u_j}(u)}}
	\]
\end{proof}

\begin{note}
	Для отрицательной параметризации нужно поставить знак минус перед радикалом.
\end{note}

\begin{definition}
	Пусть $E$ --- измеримое по Лебегу подмножество $K$, $\phi \colon K \to M$ --- положительная параметризация, $G = \phi(E)$. Тогда \textit{мерой риманова объёма $G$ с параметризацией $\phi$} называется
	\[
		\mu_M(G) := \int_{E} \sqrt{\det\tbr{\pd{\phi}{u_i}(u), \pd{\phi}{u_j}(u)}} d\mu(u)
	\]
\end{definition}

\begin{note}
	$\sigma$-аддитивность меры риманова объёма сразу следует из того же свойства интеграла Лебега и того факта, что $\phi$ является биекцией.
\end{note}

\begin{definition}
	\textit{интегралом (1-го рода) по риманову объёму} на клетке $M$ от функции $f$, заданной на $M$, называется (если правая часть определена)
	\[
		\int_M f(x) d\mu_M(x) := \int_K f(\phi(u)) \sqrt{\det\tbr{\pd{\phi}{u_i}(u), \pd{\phi}{u_j}(u)}} d\mu(u)
	\]
\end{definition}

\begin{proposition}
	Интеграл по риманову объёму не зависит от параметризации.
\end{proposition}

\begin{proof}
	Положим
	\begin{align*}
		&{g(u) := \Gamma\ps{\pd{\phi}{u_1}(u), \hdots,  \pd{\phi}{u_m}(u)}}\\
		&{\hat{g}(v) := \Gamma\ps{\pd{\hat\phi}{v_1}(v), \hdots,  \pd{\hat\phi}{v_m}(v)}}
	\end{align*}
	Мы хотим установить следующее равенство:
	\[
		\int_K f(\phi(u)) \sqrt{\det{g(u)}} d\mu(u) = \int_K f(\hat\phi(v)) \sqrt{\det{\hat g(v)}} d\mu(v)
	\]
	Сделаем замену переменных в правом интеграле:
	\begin{multline*}
		\int_K f(\hat\phi(v)) \sqrt{\det{\hat g(v)}} d\mu(v) = \int_K f(\hat\phi(\pi(u))) \sqrt{\det{\hat g(\pi(u))}} |\det{\pi'(u)}| d\mu(u) =\\
		= \int_K f(\phi(u)) \sqrt{\det{\hat g(\pi(u))}} |\det{\pi'(u)}| d\mu(u)
	\end{multline*}
	Приравнивая полученное к интегралу в исходной параметризации, получаем, что достаточно показать следующее:
	\[
		\det{g(u)} = \det{\hat g(\pi(u))} \cdot (\det{\pi'(u)})^2
	\]
	Теперь вспомним из курса алгебры, как меняется матрица Грама при переходе к новому базису:
	\[
		\Gamma\ps{G_1, \hdots, G_m} = S^T \Gamma\ps{\hat G_1, \hdots, \hat G_m} S
	\]
	где $\ps{x_1, \hdots, x_m} = \ps{\hat{x_1}, \hdots, \hat{x_m}}S$.
	Взяв определитель от обеих частей, получим
	\[
		\det{\Gamma\ps{G_1, \hdots, G_m}} = \det{\Gamma\ps{\hat G_1, \hdots, \hat G_m}} \cdot (\det{S})^2
	\]
	Это очень похоже на наше выражение. Действительно, если подставить $G = (G_1, \ldots, G_m) = \phi'(u)$, $\hat G = (\hat G_1, \ldots, \hat G_m) = \hat\phi'(\pi(u))$, то
	$G$ и $G'$ связаны формулой $G = \hat G \cdot \pi'(u)$, то есть матрицей перехода $S$ является $\pi'(u)$, что и устанавливает требуемое равенство.
\end{proof}

\begin{definition}
	Если $m = 1$, $n = 3$, то $M$ становится одномерной клеткой $M = \{\phi(u) \in \R^n \colon u \in [0; 1]\}$. \textit{Криволинейным интегралом 1-го рода} называется следующий интеграл:
	\[
		\int_M f(x, y, z)ds := \int_M f(x, y, z)d\mu_M
	\]
\end{definition}

\begin{note}
	Продолжим существовать в рамках предыдущего определения. Рассмотрим положительную параметризацию $\phi(u) = (x(u), y(u), z(u))^T$. Выясним явную формулу в рамках этой параметризации для криволинейного интеграла первого рода. Определитель вырождается в следующее:
	\[
		\det\tbr{\frac{d\phi}{du}(u), \frac{d\phi}{du}(u)} = (x'(u))^2 + (y'(u))^2 + (z'(u))^2
	\]
	Получается вот такая позитивная формула:
	\[
		\int_M f(x, y, z)ds = \int_0^1 f(x(u), y(u), z(u))\sqrt{(x'(u))^2 + (y'(u))^2 + (z'(u))^2}d\mu(u)
	\]
	Причём, если $f = 1$, то значение интеграла является \textit{длиной соответствующей кривой}.
\end{note}

\begin{definition}
	Если $m = 2$, $n = 3$, то $M$ становится двумерной клеткой $M$. \textit{Поверхностным интегралом 1-го рода} называется следующий интеграл:
	\[
		\iint_M f(x, y, z)dS := \int_M f(x, y, z)d\mu_M
	\]
\end{definition}

\begin{note}
	Если $m = 2$, $n = 3$, то $M$ --- двумерная клетка. Положительную параметризацию $\phi$ запишем так:
	\[
		\phi(u, v) = (x(u, v), y(u, v), z(u, v))^T
	\]
	Тогда $g(u, v)$ будет функцией матрицы, от которой мы хотим посчитать определитель:
	\[
		g(u, v) = \Matrix{
			&\tbr{\pd{\phi}{u}(u, v), \pd{\phi}{u}(u, v)}& &\tbr{\pd{\phi}{u}(u, v), \pd{\phi}{v}(u, v)}
			\\
			&\tbr{\pd{\phi}{v}(u, v), \pd{\phi}{u}(u, v)}& &\tbr{\pd{\phi}{v}(u, v), \pd{\phi}{v}(u, v)}
		}
	\]
	Карл Гаусс ввёл следующие обозначения для частей определителя этой формулы (скобки $(u, v)$ опускаются):
	\begin{align*}
		&{E = \tbr{\pd{\phi}{u}, \pd{\phi}{u}} = \ps{\pd{x}{u}}^2 + \ps{\pd{y}{u}}^2 + \ps{\pd{z}{u}}^2}
		\\
		&{G = \tbr{\pd{\phi}{v}, \pd{\phi}{v}} = \ps{\pd{x}{v}}^2 + \ps{\pd{y}{v}}^2 + \ps{\pd{z}{v}}^2}
		\\
		&{F = \tbr{\pd{\phi}{u}, \pd{\phi}{v}} = \pd{x}{u} \cdot \pd{x}{v} + \pd{y}{u} \cdot \pd{y}{v} + \pd{z}{u} \cdot \pd{z}{v}}
	\end{align*}
	Тогда $\det g(u, v) = EG - F^2$, а соответствующий поверхностный интеграл имеет такой вид:
	\[
		\iint_M f(x, y, z)dS = \iint_K f(x(u, v), y(u, v), z(u, v))\sqrt{EG - F^2}d\mu(u, v)
	\]
	Причём, если $f = 1$, то значение интеграла является \textit{площадью соответствующей поверхности}.
\end{note}

\begin{note}
	Начиная отсюда $m = 1$.
\end{note}

\begin{reminder}
	Кривые изучались в конце первого семестра. Все связанные определения можно найти в соответствующем конспекте.
\end{reminder}

\begin{proposition}
	Пусть $M$ --- одномерная ориентированная клетка, $V$ --- соответствующая форма ориентированного объёма, $(\vv{i}, \vv{j}, \vv{k})$ --- орты в $\R^3$ и $\vv{\tau}$ --- вектор положительной единичной касательной к $M$. Тогда имеют место следующие формулы:
	\begin{align*}
		&{dx|_M = \cos(\vv{\tau}, \vv{i})V}
		\\
		&{dy|_M = \cos(\vv{\tau}, \vv{j})V}
		\\
		&{dz|_M = \cos(\vv{\tau}, \vv{k})V}
	\end{align*}
\end{proposition}

\begin{proof}
	Проверим только первое равенство, ибо остальные доказываются аналогично. Для определённости зафиксируем $t_0 \in K$ и обозначим $\vv{x_0} := \phi(t_0)$. Распишем $dx|_M$ через перенос формы:
	\[
		dx|_M(\vv{x_0}) = \psi^* \ps{\pd{\phi^1}{t}(t_0)dt} = \pd{\phi^1}{t}(t_0) \cdot \psi^*dt
	\]
	При положительной параметризации вектор единичной касательной выражается как нормированный положительный репер. Подойдёт такой:
	\[
		\vv\tau(\vv{x_0}) = \frac{\phi'(t_0)}{\md{\phi'(t_0)}}
	\]
	Так как $\vv{r}$ и $\vv{i}$ имеют единичную длину, косинус угла между ними равен их скалярному произведению, а это просто соответствующая координата вектора $\vv{\tau}$, т.е.
	\[
		\cos(\vv{\tau}, \vv{i}) = \frac{(\phi^1)'(t_0)}{\md{\phi'(t_0)}}
	\]
	Ну а коэффициент в форме ориентированного объёма в случае одномерной клетки вырождается в длину того же касательного вектора:
	\[
		V(\vv{x_0}) = \sqrt{\tbr{\phi'(t_0), \phi'(t_0)}} \cdot \psi^*dt = \md{\phi'(t_0)} \cdot \psi^*(dt)
	\]
	Итого имеем
	\[
		\cos(\vv{\tau}, \vv{i})V(\vv{x_0}) = \frac{\pd{\phi^1}{t}(t_0)}{\md{\phi'(t_0)}} \md{\phi'(t_0)} \psi^*(dt) = \pd{\phi^1}{t}(t_0) \psi^*(dt) = dx|_M(\vv{x_0})
	\]
\end{proof}

\begin{definition}
	Пусть $\vv{A}(r) = (P(r), Q(r), R(r))^T$ --- векторное поле, заданное на одномерной клетке $M$. Тогда \textit{криволинейным интегралом 2-го рода} называется следующий интеграл:
	\[
		\int_M \vv{A}^\# = \int_M Pdx + Qdy + Rdz
	\]
	где $dx = dx|_M$, $dy = dy|_M$ и $dz = dz|_M$.
\end{definition}

\begin{corollary} (из последнего утверждения)
	Если нужно посчитать криволинейный интеграл второго рода для векторного поля, то
	\[
		\int_M Pdx + Qdy + Rdz = \pm \int_M (P\cos(\vv{\tau}, \vv{i}) + Q\cos(\vv{\tau}, \vv{j}) + R\cos(\vv{\tau}, \vv{k}))V = \pm \int_M \ntbr{\vv{A}, \vv{\tau}}ds
	\]
	Причём здесь, в отличие от криволинейного интеграла 1-го рода, знак зависит от параметризации.
\end{corollary}

\begin{note}
	Последнее выражение очень важно в физике, ибо является \textit{работой силы $\vv{A}$ по кривой $M$}.
\end{note}

\begin{definition}
	Если $\vv{A}(r)$ --- векторное поле, а $M$ --- замкнутая кривая, то криволинейный интеграл второго рода
	\[
		\int_M \ntbr{\vv{A}, \vv{\tau}}ds
	\]
	называется \textit{циркуляцией $A$ вдоль $M$}.
\end{definition}

\begin{note}
	Далее мы возвращаемся к ситуации $m = 2$, $n = 3$. При наличии параметризации $\phi$ мы используем следующие обозначения для координат:
	\[
		\phi(u, v) = (x, y, z)
	\]
\end{note}

\begin{definition}
	Пусть $(e_1(r), e_2(r))$ --- положительный базис касательного пространства $T(r)$. \textit{Положительной нормалью к поверхности в точке} $r \in M$ называется единичный вектор $n$, обладающий двумя свойствами:
	\begin{enumerate}
		\item $n \bot T(r)$
		
		\item $(n, e_1(r), e_2(r))$ --- положительный базис в $\R^3$
	\end{enumerate}
\end{definition}

\begin{proposition}
	Пусть $M$ --- ориентированная двумерная клетка, $V$ --- соответствующая форма ориентированного объёма, $(\vv{i}, \vv{j}, \vv{k})$ --- орты в $\R^3$ (ортонормированный базис) и $\vv{n}$ --- вектор положительной нормали. Тогда в каждой точке $r \in M$ верны следующие формулы:
	\begin{align*}
		&{dx \wedge dy|_M = \cos(\vv{n}, \vv{k})V}
		\\
		&{dy \wedge dz|_M = \cos(\vv{n}, \vv{i})V}
		\\
		&{dz \wedge dx|_M = \cos(\vv{n}, \vv{j})V}
	\end{align*}
\end{proposition}

\begin{proof}
	Проверим лишь первое равенство, ибо остальные делаются аналогично. Пусть $\phi$ --- положительная параметризация. С одной стороны:
	\begin{multline*}
		dx \wedge dy|_M = \psi^* \ps{\ps{\pd{\phi^1}{u}du + \pd{\phi^1}{v}dv} \wedge \ps{\pd{\phi^2}{u}du + \pd{\phi^2}{v}dv}} = \\
		= \ps{\pd{\phi^1}{u} \cdot \pd{\phi^2}{v} - \pd{\phi^1}{v} \cdot \pd{\phi^2}{u}} \psi^* (du \wedge dv)
	\end{multline*}
	С другой стороны, посмотрим на форму ориентированного объёма двумерной клетки:
	\[
		V(x) = \sqrt{\det\tbr{\pd{\phi}{u_i}(u), 	\pd{\phi}{u_j}(u)}} \cdot \psi^*(du \wedge dv)
	\]
	Как уже было сказано, коэффициент в данной форме есть просто объём призмы, натянутой на некоторые вектора. А поскольку в случае двумерной клетки этими векторами являются $\pd{\phi}{u}$ и $\pd{\phi}{v}$, то объём построенного на них параллелограмма также по определению равен модулю их векторного произведения. А значит, проверка сводится к следующему:
	\[
		\pd{\phi^1}{u} \cdot \pd{\phi^2}{v} - \pd{\phi^1}{v} \cdot \pd{\phi^2}{u} = \cos(\vv{n}, \vv{k}) \md{\sbr{\pd{\phi}{u}, \pd{\phi}{v}}} \tag{$\ast$}
	\]
	В силу положительной параметризации, вектора $\pd{\phi}{u}(u_0, v_0)$ и $\pd{\phi}{v}(u_0, v_0)$ образуют положительный базис касательного пространства $T(x_0, y_0, z_0)$. Тогда положительную нормаль $\vv{n}$ в этой точке можно записать известным соотношением:
	\[
		\vv{n} = \frac{\sbr{\pd{\phi}{u}, 	\pd{\phi}{v}}}{\md{\sbr{\pd{\phi}{u}, \pd{\phi}{v}}}}
	\]
	Далее, мы можем выразить косинус через скалярное произведение:
	\[
		\cos(\vv{n}, \vv{k}) = \frac{\ntbr{\vv{n}, \vv{k}}}{\md{\vv{n}} \cdot |\vv{k}|} = \ntbr{\vv{n}, \vv{k}} = \frac{\ntbr{\sbr{\pd{\phi}{u}, \pd{\phi}{v}}, \vv{k}}}{\md{\sbr{\pd{\phi}{u}, \pd{\phi}{v}}}}
	\]
	Явно векторное произведение, как мы знаем, можно записать таким мнемоническим определителем:
	\[
	\sbr{\pd{\phi}{u}, \pd{\phi}{v}} = \Det{
		&\vv{i}& &\vv{j}& &\vv{k}
		\\
		&\pd{\phi^1}{u}& &\pd{\phi^2}{u}& &\pd{\phi^3}{u}
		\\
		&\pd{\phi^1}{v}& &\pd{\phi^2}{v}& &\pd{\phi^3}{v}
	}
	\]
	После скалярного умножения на $\vv{k}$ от него останется лишь соответствующая координата, т.е.
	\[
		\cos(\vv{n}, \vv{k}) = \frac{\pd{\phi^1}{u} \pd{\phi^2}{v} - \pd{\phi^1}{v} \pd{\phi^2}{u}}{\md{\sbr{\pd{\phi}{u}, \pd{\phi}{v}}}}
	\]
	Подставляя это выражение в $(*)$, получаем требуемое.
\end{proof}

\begin{definition}
	Пусть $\vv{A} = (P, Q, R)^T$ --- векторное поле на двумерной клетке $M$. \textit{Поверхностным интегралом 2-го рода} называется следующий интеграл:
	\[
	\iint_M Pdydz + Qdzdx + Rdxdy := \int_M *\vv{A}^\# = \int_M Pdy \wedge dz|_M + Q dz \wedge dx|_M + Rdx \wedge dy|_M
	\]
\end{definition}

\begin{corollary} (из последнего утверждения)
	Если нужно посчитать поверхностный интеграл второго рода, то это можно сделать так:
	\begin{multline*}
	\iint_M Pdydz + Qdzdx + Rdxdy =
	\\
	\iint_M (P\cos(\vv{n}, \vv{i}) + Q\cos(\vv{n}, \vv{j}) + R\cos(\vv{n}, \vv{k}))V = \iint_M \ntbr{\vv{A}, \vv{n}}dS
	\end{multline*}
\end{corollary}

\begin{definition}
	Если $\vv{A}$ --- векторное поле, а $M$ --- двумерная клетка, то поверхностный интеграл второго рода
	\[
	\iint_M \ntbr{\vv{A}, \vv{n}}dS
	\]
	называется \textit{потоком поля $\vv{A}$ через $M$}
\end{definition}

\begin{anote}
	Когда говорят об интегрировании на замкнутых объектах (замкнутая кривая, замкнутая поверхность), то используют значок $\oint$ и соответствующие с двумя/тремя интегралами.
\end{anote}

\begin{reminder}
	Нормаль, о которой мы говорили в теореме Стокса-Пуанкаре для куба, является \textit{внешней}. В более общем случае, это просто единичная нормаль, выходящая \textit{из} какого-то трехмерного объекта.
\end{reminder}

\begin{note}
	Отсюда $m = n = 3$.
\end{note}

\begin{lemma}
	Если $M$ --- трёхмерная клетка, причём граница $\vdelta M$ ориентирована по правилу выходящего вектора, то положительная нормаль клетки совпадает с внешней нормалью.
\end{lemma}

\textcolor{red}{Тут снова должна быть картинка, в этот раз с кубок и какой-то желешкой.}

\textcolor{red}{Дальнейшие рассуждения некорректны.}

\begin{proof}
	Пусть $\phi$ --- положительная параметризация. В силу определения положительной параметризации, набор векторов $\{\pd{\phi}{u_i}\}_{i = 1}^3$ образует положительный базис в касательном пространстве $T_M$ в каждой точке многообразия $M$. Несложно понять, что границей клетки должны быть образы граней $K_\alpha^j$ стандартного куба $K$, которые притом сами являются двумерными многообразиями $M_\alpha^j = \phi(K_\alpha^j)$. Не умаляя общности, изучим клетку $M_1^2 = \phi(K_1^2)$. Её касательные пространства $T_{M_1^2}(x)$ имеют индуцированный положительный базис $\ps{\pd{\phi}{u_1}(u), \pd{\phi}{u_3}(u)}$. Зафиксируем точку $x_0 = \phi(u_0) \in M_1^2$ и последовательно докажем следующие факты:
	\begin{enumerate}
		\item Вектор $\grad \psi_2(x_0)$ ортогонален соответствующему базису в $T_{M_1^2}$. Для начала поймём, чем является скалярное произведение градиента с любым вектором $\phi'(u_0)\vv{H} \in T_{M_1^2}$. Вспомним уже возникавшее равенство:
		\[
		\psi'(x_0)(\phi'(u_0)\vv{H}) = \psi'(\phi(u_0))(\phi'(u_0)\vv{H}) = \vv{H}
		\]
		Скалярное произведение с градиентом от $\psi_2$ в точке $x$ в точности соответствует взятию первой координаты от равенства выше. Таким образом:
		\[
		(\grad \psi_2(x_0), \phi'(u_0)\vv{H}) = H^2
		\]
		где $H^2$ является соответствующей координатой вектора $\vv{H}$. Если теперь мы хотим взять произведение с базисными векторами, то $\vv{H}$ будет обладать столбцами $(1, 0, 0)^T$ и $(0, 0, 1)^T$ соответственно, в обоих случаях получим ноль.
		
		\item Вектор $\grad \psi_2(x_0)$ направлен вовне клетки $M$. В силу диффеоморфизма, необходимо и достаточно показать, что прообраз любого достаточно малого вектора, отложенного в точке на клетке $M_1^2$ в сторону от $M$, будет иметь положительное скалярное произведение с градиентом. Для этого можно рассмотреть приращение $\psi_2$:
		\[
		\psi_2(x) - \psi_2(x_0) = \tbr{\grad \psi_2(x_0), x - x_0} + o(x - x_0)
		\]
		Если $x - x_0$ соответствует описанному выше вектору, то $\psi_2(x) > 1$, а тогда по равенству скалярное произведение будет больше нуля, то есть градиент действительно смотрит вовне $M$.
	\end{enumerate}
	Таким образом, единичный вектор $n$, заданный формулой ниже, является внешней нормалью для $M$ на части границы, соответствующей $M_1^2$:
	\[
	n = \frac{\grad \psi_2(x)}{|\grad \psi_2(x)|}
	\]
	Если мы теперь докажем, что тройка $(n, \pd{\phi}{u_1}(u_0), \pd{\phi}{u_3}(u_0))$ является положительным базисом в $T_M$, то $n$ будет положительной нормалью клетки. Для этого выясним, как выражается основной базис (репер в точке) через этот:
	\begin{align*}
		&{\pd{\phi}{u_1}(u_0) = 0 \cdot n + 1 \cdot \pd{\phi}{u_1}(u_0) + 0 \cdot \pd{\phi}{u_3}(u_0)}
		\\
		&{\pd{\phi}{u_2}(u_0) = \frac{(\grad \psi_2(x_0), \pd{\phi}{u_2}(u_0))}{|\grad \psi_2(x_0)|^2} \cdot \grad \psi_2(x_0)} + C_1 \cdot \pd{\phi}{u_1}(u_0) + C_3 \cdot \pd{\phi}{u_3}(u_0)
		\\
		&{\pd{\phi}{u_3}(u_0) = 0 \cdot n + 0 \cdot \pd{\phi}{u_1}(u_0) + 1 \cdot \pd{\phi}{u_3}(u_0)}
	\end{align*}
	Без долгих расписываний, утверждаю, что соответствующий определитель матрицы перехода будет положителен:
	\[
	\begin{vmatrix}
		0 & \frac{1}{|\grad \psi_2(x_0)|^2} & 0
		\\
		1 & C_1 & 0
		\\
		0 & C_3 & 1
	\end{vmatrix} = \frac{1}{|\grad \psi_2(x_0)|^2} > 0
	\]
\end{proof}

\begin{theorem} (Гаусса-Остроградского для клетки)
	Если $\vv{A}$ --- гладкое векторное поле, заданное на трёхмерной клетке $M$, то имеет место равенство:
	\[
	\iint_{\vdelta M} \ntbr{\vv{A}, \vv{n}}dS = \iiint_M \Div \vv{A} d\mu(x)
	\]
	где $\vv{n}$ --- внешняя нормаль к границе $\vdelta M$.
\end{theorem}

\begin{note} (Пояснение к термину <<поток>>)
	Рассмотрим постоянное векторное поле $\vv{A}$ и ориентированный параллелограмм, натянутый на вектора $(\vv{\xi}_1, \vv{\xi_2})$. В таком случае поток $\vv{A}$ через параллелограмм равен $(\vv{A}, \vv{\xi}_1, \vv{\xi}_2)$.
\end{note}

\begin{proof}
	Всё доказательство сводится к применению теоремы Стокса-Пуанкаре для клетки, которую мы уже доказали. Пусть $\phi$ --- положительная параметризация. Достаточно заметить, что
	\[
	\ps{\vv{A}(x), \pd{\phi}{u_1}(u), \pd{\phi}{u_2}(u)} = \ps{\vv{A}, \sbr{\pd{\phi}{u_1}, \pd{\phi}{u_2}}} = \ntbr{\vv{A}, \vv{n}} \md{\sbr{\pd{\phi}{u_1}, \pd{\phi}{u_2}}} = \ntbr{\vv{A}, \vv{n}} \sqrt{EG - F^2}
	\]
	Стало быть:
	\[
	\iint_{\vdelta M} \ntbr{\vv{A}, \vv{n}}dS = \iint_K \ntbr{\vv{A}, \vv{n}}\sqrt{EG - F^2} d\mu(u)
	\]
\end{proof}

\begin{corollary} (Геометрическое определение дивергенции)
	Если $\vv{A}$ --- гладкое векторное поле в окрестности точки $\vv{r}_0$, то имеет место формула:
	\[
	\Div \vv{A}(\vv{r}_0) = \lim_{\eps \to +0} \frac{1}{V_{\text{шара}}(\eps)} \iint_{\vdelta B_\eps(\vv{r}_0)} \ntbr{\vv{A}, \vv{n}}dS
	\]
	где $V_{\text{шара}}(\eps) = \frac{4}{3}\pi\eps^3$ --- объём шара радиуса $\eps$, а $B_\eps(\vv{r_0})$ --- замкнутый шар радиуса $\eps$ с центром в точке $\vv{r}_0$.
\end{corollary}

\begin{lemma}
	Положительная ориентация границы двумерной клетки $M$ совпадает с ориентацией с помощью положительной касательной.
\end{lemma}

\begin{proof}
	\textcolor{red}{Пока что под вопросом. В курсе не вводилось понятия ориентации границы двумерной клетки.}
\end{proof}

\begin{theorem} (Формула Стокса)
	Пусть $M$ --- двумерная клетка в $\R^3$, $\vv{A}$ --- гладкое векторное поле на $M$. Тогда циркуляция поля $\vv{A}$ вдоль границы $\vdelta M$, ориентированной по правилу правого обхвата, равна потоку ротора $\vv{A}$ через $M$, то есть
	\[
	\int_{\vdelta M} \ntbr{\vv{A}, \vv{\tau}}ds = \iint_M \ntbr{\rot \vv{A}, \vv{n}}dS
	\]
\end{theorem}

\begin{proof}
	\textcolor{red}{Ожидается}
\end{proof}

\begin{corollary} (Геометрическое определение ротора)
	Пусть $\vv{A}$ --- гладкое векторное поле в окрестности точки $\vv{r}_0$. Тогда для любого единичного вектора $\vv{l} \in \R^3$ верно, что
	\[
	\ntbr{\rot \vv{A}, \vv{l}} = \lim_{\eps \to +0} \frac{1}{\pi \eps^2} \int_{\vdelta S_\eps(\vv{r}_0)} \ntbr{\vv{A}, \vv{\tau}}ds
	\]
	где $S_\eps(\vv{r}_0)$ --- круг с центром в точке $\vv{x}_0$ радиуса $\eps$, ортогональный к вектору $\vv{l}$.
\end{corollary}

\begin{proof}
	У круга в любой точке будет одинаковая положительная нормаль $\vv{l}$, поэтому
	\[
	\int_{\vdelta S_\eps(\vv{r}_0)} \ntbr{\vv{A}, \vv{\tau}}ds = \iint_{S_\eps(\vv{r}_0)} \ntbr{\rot \vv{A}, \vv{l}}dS
	\]
	Остаётся заметить, что $\ntbr{\rot \vv{A}, \vv{l}}$ является непрерывной функцией, а потому интеграл тривиально оценивается снизу и сверху:
	\[
	\min_{x \in S_\eps(\vv{r}_0)} \ntbr{\rot \vv{A}, \vv{l}} \cdot \pi\eps^2 \le \iint_{S_\eps(\vv{r}_0)} \ntbr{\rot \vv{A}, \vv{l}}dS \le \max_{x \in S_\eps(\vv{r}_0)} \ntbr{\rot \vv{A}, \vv{l}} \cdot \pi\eps^2
	\]
	Устремление $\eps \to 0$ даёт нужный результат.
\end{proof}

\begin{note}
	Следствие также показывает, что ротор инвариантен относительно замены координат.
\end{note}

\textcolor{red}{Сюда можно картиночку с кругом}
