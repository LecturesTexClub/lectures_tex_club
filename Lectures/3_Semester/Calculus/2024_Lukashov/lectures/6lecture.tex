\begin{anote}
	В следующей теореме мы пользуемся тем, что функция $f$ непрерывно дифференцируема и её якобиан не обращается в ноль. По теореме о локальной обратимости отображения из второго семестра эти два свойства влекут существование обратной дифференцируемой функции, а значит, лолькано $f$ осуществляет диффеоморфизм. Но и обратное тоже верно, поскольку
	\[
		\det{f'} \cdot \det{(f^{-1})'} = \det{(f \circ f^{-1})'} = \det{(id)'} = 1
	\]
\end{anote}

\begin{definition}
	Диффеоморфизм $f \colon U \to \R^n, \ U \subset \R^n$ называется \textit{простым}, если
	\begin{enumerate}
		\item
		$0 \in U$ и $f(0) = 0$
		\item 
		$\exists i \colon \ \forall x, \ \forall j \neq i \ f_j(x) = x_j$, где нижний индекс означает $j$-ю координату.
	\end{enumerate}
\end{definition}

\begin{theorem} (О разложении)
	Пусть $f \colon G \to \R^n$ --- диффеоморфизм области $G \subset \R^n$, $0 \in G$, $f(0) = 0$. Тогда в некоторой окрестности нуля $\exists$ простые диффеоморфизмы $g^{[1]}, \hdots, g^{[N]}$ и линейные отображения $B_1, \hdots, B_N$, такие что каждое из них либо тождественное, либо меняет местами пару координат, и в некоторой (возможно, меньшей) окрестности нуля
	\[
		f = g^{[N]} \circ B_N \circ \hdots \circ g^{[1]} \circ B_1
	\]
\end{theorem}

\begin{proof}
	Построим требуемое разложение по индукции вместе со вспомогательными функциями $f^{[m]}$, обладающими таким свойством: $\forall i < m \ f^{[m]}(x)_i = x_i$. \\
	База: $f^{[1]} := f$. У $g^{[i]}$ и $B_i$ индекс будет на 1 меньше. \\
	Переход: пусть построены все $f^{[i]}$ вплоть до $m$, а также $g^{[i]}$ и $B_i$ до $m - 1$.
	Посмотрим на матрицу Якоби $f^{[m]}$:
	\[
		(f^{[m]})' =
		\left(\begin{array}{@{}c|c@{}}
		E_{m-1} & 0\\
		\hline
		* & A\end{array}\right)
	\]
	Поскольку определитель не ноль, первая строка матрицы A должна содержать ненулевой элемент --- пусть он стоит в $j$-м столбце относительно всей матрицы $(f^{[m]})'$. (Тут важно заметить, что этот элемент не зануляется в некоторой окрестности, поскольку все частные производные непрерывны, и это даёт нам право взять один $j$ для всех $x$).	Здесь в игру вступает отображение $B_m$, которое будет менять координаты $m$ и $j$, $j \geq m$. $g^{[m]}$, в свою очередь, определим так: $\forall i \neq m \ g^{[m]}(x)_i = x_i$ и $g^{[m]}(x)_m = (f^{[m]} \circ B_m)_m$. Поскольку $g^{[m]}$ не меняет никакие координаты, кроме $m$, а приращение $m$-й координаты $g^{[m]}_m$ соответствует приращению $j$-й координаты $f^{[m]}$, якобиан $g^{[m]}$ не обращается в ноль:
	\[
		(g^{[m]})' =
		\left(\begin{array}{@{}c|c|c@{}}
			E_{m-1} & 0 & 0\\
			\hline
			* & \pd{f^{[m]}}{x_j} & *\\
			\hline
			0 & 0 & E_{n-m}
		\end{array}\right)
	\]
	Стало быть, по теореме о локальной обратимости отображения в некоторой окрестности (каждый раз сужаемой) существует обратная функция $g^{-1}$, через которую и определим следующую $f$:
	\[
		f^{[m+1]} := f^{[m]} \circ B_m \circ (g^{[m]})^{-1}
	\]
	Поймём, что делают координатные функции $f_i^{[m+1]}$.
	\begin{itemize}
		\item 
		Если $i < m$, то все три координатные функции действуют как проектор соответствующей координаты ($f^{[m]}$ и $B_m$ по определению, $(g^{[m]})^{-1}$ как обратная к $g^{[m]}$), а значит, и сама $f_i^{[m+1]}$ удовлетворяет заявленному свойству.
		\item 
		Если $i = m$, то $(f^{[m]} \circ B_m \circ (g^{[m]})^{-1})_m = (g^{[m]} \circ (g^{[m]})^{-1})_m$ = $id_m$, что завершает индукционный переход.
	\end{itemize}
	Что нам даёт эта громоздкая конструкция? Мы хотим красиво разложить $f$, которая также $f^{[1]}$. При этом $f^{[m]}$ можно выразить через $f^{[m+1]}$:
	\[
		f^{[m]} = f^{[m+1]} \circ g^{[m]} \circ B_m
	\]
	поскольку обратная к $B_m$ --- она сама. С учётом этого уже раскроем $f$:
	\[
		f = f^{[1]} = f^{[2]} \circ g^{[1]} \circ B_1 = f^{[3]} \circ g^{[2]} \circ B_2 \circ g^{[1]} \circ B_1 = f^{[N+1]} \circ g^{[N]} \circ B_N \circ \hdots \circ g^{[1]} \circ B_1
	\]
	На $f^{[N+1]}$ построение обрывается, т.к. это просто тождественная функция. Нужно ещё проверить, что все $g^{m}(0) = 0$, но это тоже нетрудно увидеть из их индуктивного построения.
	Таким образом, требуемое разложение получено.
\end{proof}

\begin{theorem} (Замена переменных в кратном интеграле)
	Пусть $f$ суммируема на ограниченном измеримом множестве $V \subset \R^n$, $\phi$ --- диффеоморфизм области $\Omega_U \supset \ole{U}$ на $\Omega_V \supset \ole{V}$, $\phi(U) = V$. Тогда
	\[
		\int_V f(v) d\mu(v) = \int_U f(\phi(u)) \md{\det{\phi'(u)}} d\mu(u)
	\]
	В частности, если положить $f = \chi_A$ для любого измеримого $A \subset V$, то
	\[
		\mu(A) = \int_{\phi^{-1}(A)} \md{\det{\phi'(u)}} d\mu(u)
	\]
\end{theorem}

\begin{proof}
	Сначала заметим, что если теорема доказана для отображений $\phi$ и $\psi$, таких что $\phi(U) = V$, $\psi(W) = U$, то она верна и для $\phi \circ \psi$, ибо в таком случае
	\begin{multline*}
		\int_V f(v) d\mu(v) = \int_{U} f(\phi(u)) \md{\det{\phi'(u)}} d\mu(u) = \\
		= \int_{W} f(\phi(\psi(w))) \md{\det{\phi'(\psi(w))}} \md{\det{\psi'(w)}} d\mu(w) = \int_{W} f(\phi(\psi(w))) \md{\det{(\phi \circ \psi)'(w))}} d\mu(w)
	\end{multline*}
	С учётом этого наш план таков: для каждой точки $a \in U$ рассмотрим функцию $\hat\phi(x) = \phi(x + a) - \phi(a)$, которая определена в некоторой окрестности нуля, причём $\hat\phi(0) = 0$. Тогда $\phi(x) = \hat\phi(x - a) + \phi(a)$ --- композиция сдвигов и некоторого диффеоморфизма, который раскладывается по предыдущей теореме. Тогда, чтобы доказать локальную версию теоремы для любого диффеоморфизма, достаточно сделать это для простых диффеоморфизмов, для перестановки двух координат и для сдвигов (очевидно, последние два являются диффеоморфизмами). Потом придумаем, как распространить замену на всё $U$. Кроме того, как и в одномерной версии теоремы, мы акцентируемся лишь на частном случае с $\chi_A$, поскольку он обобщается на произвольную измеримую функцию ровно так же: разбиваем на $f^+$ и $f^-$ и применяем сначала линейность интеграла, затем теорему Леви и снова линейность. \\
	Итак, самый содержательный случай --- $g$ простой диффеоморфизм. Для простоты обозначений будем считать, что $g$ меняет первую координату. Разобьём $\R^n$ на $\R \times \R^{n-1}$, где $R^{n-1}$ обозначим за $X$, и любую точку $v \in V$ представим как $v = (y, x), \ y \in \R, x \in X$. При этом также $v = g(\tilde y, x) = (g_1(\tilde y, x), x)$, где при каждом фиксированном $x$ $g_1(\tilde y, x)$ --- функция из $\R$ в $\R$. Обозначим её $\tau_x(\tilde y)$. Дальше просто проследует цепочка равенств и пояснения к каждому переходу.
	\begin{multline*}
		\mu(A) = \int_X \mu_Y(A_Y(x)) d\mu(x) = \int_X \ps{ \int_{\tau_x^{-1}(A_Y(x))} \md{\tau_x'(\tilde y)} d\mu(\tilde y)} d\mu(x) = \\
		= \int_X \ps{ \int_{\tau_x^{-1}(A_Y(x))} \md{\det{g'(u)}} d\mu(\tilde y)} d\mu(x) = \int_{g^{-1}(A)} \md{\det{g'(u)}} d\mu(u)
	\end{multline*}
	\begin{enumerate}
		\item
		Применили теорему Фубини, т.к. A измеримо.
		\item
		Из вида матрицы $g'$ следует, что $\tau_x$ --- диффеоморфизм, поскольку элемент на пересечении первой строки и первого столбца не ноль и равен $\pd{g}{\tilde y}$, а это и есть $\tau_x'$. Также по условию $V$ ограничено, а значит, существует отрезок $I \colon A_Y(x) \subset I \subset \Omega_V$, на котором к подынтегральной мере сечения применима теорема о замене одномерной переменной --- она и написана.
		\item
		Равенство якобианов $r_x$ и $g$ тоже очевидно из того, как раскладывается последний.
		\item 
		Здесь снова теорема Фубини. Нужно только прояснить, что
		\[
			x \in X, \ \tilde y \in \tau_x^{-1}(A_Y(x)) \Lra x \in X, \ y \in (A_Y(x) \Lra (y, x) \in A \Lra (\tilde y, x) \in g^{-1}(A)
		\]
	\end{enumerate}
	С простыми диффеоморфизмами разобрались. У перестановок координат и сдвигов якобиан равен единице, поэтому проверка сводится к равенству $\mu(A)$ = $\mu(\phi^{-1}(A))$. Здесь принимается геометрически очевидным, что мера Лебега инварианта относительно данных типов преобразований (во всяком случае, подобные свойства изучались в курсе ОВиТМа). А значит, с учётом всех вышесказанных замечаний локальная версия теоремы доказана. \\
	Мы научились в каждой точке $a \in \ole{V}$ находить свою окрестность $U(a, \eps) \cap V$, в которой теорема выполняется. В то же время $\ole{V}$ --- компактное множество, и тогда $\exists \ U(a_1, \eps_n), \hdots,$\\$ U(a_N, \eps_N)$, такие что $\cup_{i=1}^N U(a_i, \eps_i) \supset \ole{V}$. Однако непосредственно свести это к свойствам интеграла не выйдет, поэтому вводится набор функций $\{\zeta_i\}_{i=1}^N$, называемых \textit{разбиением единицы} и обладающих следующими свойствами:
	\begin{enumerate}
		\item
		$\zeta_i\ \in C^\infty$, т.е. бесконечно дифференцируемые на $M$ \ \textcolor{red}{зачем?}
		\item
		$\forall x \in V \ \zeta_i(x) \geq 0$, причём $\zeta_i(x) > 0 \Lra x \in U(a_i, \eps_i)$
		\item
		$\forall x \in V \ \sum_{i=1}^N \zeta_i(x) = 1$
	\end{enumerate} 
	И тут же предъявляется пример такого разбиения:
	\begin{align*}
		&{\eta_s(v):= \System{
			&{e^{-\frac{1}{(\eps_s^2 - (v - a_s)^2)^2}}}, \ \md{v - a_s} < \eps_s \\
				&{0, \ \md{v - a_s} \geq \eps_s}
		}}\\
		&{\zeta_i = \frac{\eta_i}{\sum_{s=1}^N \eta_s}}
	\end{align*}
	При $\md{v - a_s} < \eps_s$ $\eta_s$ бесконечно дифференцируемы как композиция $e^{-x^{-2}} \in C^\infty$ на $\R^n \setminus 0$ и многочлена $\eps_s^2 - (v - a_s)^2$. Для $\md{v - a_s} > \eps_s$ тоже понятно, а на стыке все производные внутри и снаружи стремятся к нулю --- значит, вся $\eta_s \in C^\infty$. Далее, мы имеем право делить на $\sum_{s=1}^N \eta_s$, ибо рассматриваем разбиение только на $V$ и $\forall v \in M \ \exists s : |v - a_s| < \eps \Ra \eta_s(v) > 0$. Дальше понятно, что все свойства разбиения выполняются для $\zeta^i$.\\
	Коль скоро разбиение единицы получено, глобальная замена переменных доказывается просто:
	\begin{multline*}
		\int_V f(v) d\mu(v) = \int_V \sum_{s=1}^N \zeta_s(v) f(v) d\mu(v) = \sum_{s=1}^N \int_V \zeta_s(v) f(v) d\mu(v) = \\
		= \sum_{s=1}^N \int_{V \cap U(a_s, \eps_s)} \zeta_s(v) f(v) d\mu(v) = \sum_{s=1}^N \int_{\phi^{-1}(V \cap U(a_s, \eps_s))} \zeta_s(\phi(u)) f(\phi(u)) \md{\det{\phi'(u)}} d\mu(u) = \\
		= \sum_{s=1}^N \int_{\phi^{-1}(V)} \zeta_s(\phi(u)) f(\phi(u)) \md{\det{\phi'(u)}} d\mu(u) = \\
		= \int_U \sum_{s=1}^N \zeta_s(\phi(u)) f(\phi(u)) \md{\det{\phi'(u)}} d\mu(u) = \int_U f(\phi(u)) \md{\det{\phi'(u)}} d\mu(u)
	\end{multline*}
\end{proof}