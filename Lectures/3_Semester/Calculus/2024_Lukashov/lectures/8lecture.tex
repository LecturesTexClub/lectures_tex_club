\begin{example} (Строгое неравенство)
	Рассмотрим канторову лестницу. 
	\textcolor{red}{Можно посмотреть в конспекте ОВиТМа}
\end{example}

\begin{corollary}
	Если $f$ --- функция ограниченной вариации на $[a; b]$, то она дифференцируема почти всюду на $[a; b]$, причём $f'$ суммируема на том же отрезке.
\end{corollary}

\begin{lemma} \label{int_lemma}
	Если $f$ суммируема на $[a; b]$, то $F(x)$, определенная как
	\[
		F(x) = F(a) + \int_{[0; x]} f(t)d\mu(t)
	\]
	является непрерывной функцией ограниченной вариации на $[a; b]$.
\end{lemma}

\begin{proof}~
	\begin{itemize}
		\item Непрерывность следует из абсолютной непрерывности интеграла Лебега:
		\[
			\forall \eps > 0\ \exists \delta > 0 \such \forall e \subset [a; b],\ \mu(e) < \delta \quad \md{\int_e f(t)d\mu(t)} < \eps
		\]
		Чтобы получить непрерывность, заметим, что $F(x + t) - F(x) = \int_{[x; x + t] f(t)d\mu(t)}$ и при $t < \delta$ получится, что надо рассмотреть $e = [x; x + t],\ \mu(e) < \delta$.
		
		\item Чтобы показать ограниченность вариации, рассмотрим произвольное разбиение отрезка $P \colon a = x_0 < x_1 < \ldots < x_n = b$. Тогда вариация по определию распишется так:
		\begin{multline*}
			V(P, F) = \sum_{k = 1}^n |F(x_k) - F(x_{k - 1})| = \sum_{k = 1}^n \md{\int_{[x_{k - 1}; x_k]} f(t)d\mu(t)} \le \sum_{k = 1}^n \int_{[x_{k - 1}; x_k]} |f(t)|d\mu(t) =
			\\
			\int_{[a; b]} |f(t)|d\mu(t)
		\end{multline*}
		Последняя величина не зависит от рассматриваемого разбиения.
	\end{itemize}
\end{proof}

\begin{lemma}
	Если $f$ суммируема на $[a; b]$ и $\int_{[a; x]} f(t)d\mu(t) = 0$ для любого $x \in [a; b]$, то $f = 0$ почти всюду на $[a; b]$.
\end{lemma}

\begin{proof}
	Достаточно рассмотреть случай неотрицательной $f$, ибо остальные сводятся при помощи $f^+$ и $f^-$. Обозначим за $E$ множество, на котором функция принимает положительные значения. Тогда, если $\mu(E) = 0$, то всё доказано, а иначе $\mu(E) > 0$ (и это число определено в силу суммируемости). Существует замкнутое множество $F \subset E$ такое, что $\mu(F) > 0$. Положим $G := (a; b) \bs F$ --- открытое множество и воспользуемся аддитивностью интеграла Лебега:
	\[
		0 = \int_{[a; b]} f(t)d\mu(t) = \underbrace{\int_F f(t)d\mu(t)}_{> 0} + \underbrace{\int_G f(t)d\mu(t)}_{\neq 0}
	\]
	Коль скоро $G$ является открытым множеством прямой, оно представимо в виде не более чем счётного числа интервалов $G = \bscup_{i = 1}^\infty (a_i; b_i)$. Благодаря $\sigma$-аддитивности интеграла Лебега, имеет место следующий факт:
	\[
		0 \neq \int_G f(t)d\mu(t) = \sum_{i = 1}^\infty \int_{(a_i; b_i)} f(t)d\mu(t) \Lora \exists i \in \N \colon 0 \neq \int_{(a_i; b_i)} f(t)d\mu(t)
	\]
	Остаётся прийти к противоречию, показав равенство:
	\[
		0 \neq \int_{(a_i; b_i)} f(t)d\mu(t) = \int_{(a_i; b_i]} f(t)d\mu(t) = \int_{[a; b_i]} f(t)d\mu(t) - \int_{[a; a_i]} f(t)d\mu(t) = 0
	\]
\end{proof}

\begin{theorem}
	Если $f$ суммируема на $[a; b]$, то для $F(x) = F(a) + \int_{[a; x]}f(t)d\mu(t)$ справедливо 2 вещи:
	\begin{enumerate}
		\item $\int_{[a; b]} F'(x)d\mu(x) = F(b) - F(a)$
		
		\item $F'(x) = f(x)$ почти всюду на $[a; b]$
	\end{enumerate}
\end{theorem}

\begin{proof}
	Разберём отдельные случаи:
	\begin{enumerate}
		\item $f$ --- ограниченная функция на $[a; b]$. Обозначим константу ограничения за $M$: $|f(x)| \le M$. По лемме \ref{int_lemma} и следствию из последней теоремы, производная $F'$ существует почти всюду на $[a; b]$.
		\textcolor{red}{Дописать}
		
		\item $f$ --- неотрицательная неограниченная функция
		\textcolor{red}{Дописать}
		
		\item $f$ --- произвольная функция. Тогда ссылаемся на $f = f^+ - f^-$ и всё доказано.
	\end{enumerate}
\end{proof}

\begin{definition}
	Если $f$ --- суммируемая на $[a; b]$ функция, то \textit{неопределённым интегралом Лебега} называется функция $F(x) = F(a) + \int_{[a; x]} f(t)d\mu(t)$.
\end{definition}

\begin{definition}
	Функция $f$ называется \textit{абсолютно непрерывной} на $[a; b]$, если верно утверждение:
	\[
		\forall \eps > 0\ \exists \delta > 0 \such \forall \{[x_i; y_i]\}_{i = 1}^n,\ \ps{\bscup_{i = 1}^n (x_i; y_i) \subset [a; b],\ \sum_{i = 1}^n |y_i - x_i| < \delta} \quad \sum_{i = 1}^n |f(y_i) - f(x_i)| < \eps
	\]
	Обозначается как $f \in AC[a; b]$
\end{definition}

\begin{note}
	Абсолютная непрерывность является обобщением равномерной. Действительно, равномерная непрерывность является случаем абсолютной с $n = 1$.
\end{note}

\begin{lemma}
	Если $f \in AC[a; b]$, то она является функцией ограниченной вариации на $[a; b]$.
\end{lemma}

\begin{proof}
	Положим $\eps = 1$. По абсолютной непрерывности
	\[
		\exists \delta > 0 \such \forall \{[x_i; y_i]\}_{i = 1}^n,\ \ps{\bscup_{i = 1}^n (x_i; y_i) \subset [a; b],\ \sum_{i = 1}^n |y_i - x_i| < \delta} \quad \sum_{i = 1}^n |f(y_i) - f(x_i)| < 1
	\]
	\textcolor{red}{Дописать}
\end{proof}

\begin{corollary}
	Если $f \in AC[a; b]$, то она дифференцируема почти всюду.
\end{corollary}

\begin{lemma}
	Если $f \in AC[a; b]$, причём $f' = 0$ почти всюду на $[a; b]$, то $f$ постоянна на $[a; b]$.
\end{lemma}

\begin{proof}
	Рассмотрим произвольный интервал $(a; c) \subset [a; b]$. Рассмотрим $E \subset (a; c)$ --- множество точек, на которых производная равна нулю. Тогда из условия $\mu(E) = c - a$.
	
	\textcolor{red}{Дописать}
\end{proof}

\begin{corollary}
	Канторова лестница не является абсолютно непрерывной.
\end{corollary}

\begin{theorem}
	Пусть $F$ определена на $[a; b]$. Тогда $F$ абсолютно непрерывна на $[a; b]$ тогда и только тогда, когда $F$ является неопределенным интегралом Лебега на $[a; b]$.
\end{theorem}

\begin{proof}
	Проведём доказательство в 2 стороны:
	\begin{itemize}
		\item $\La$
		
		\item $\Ra$
	\end{itemize}
\end{proof}

\begin{note}
	Из всей построенной теории следует, что произвольную функцию $f$ на отрезке $[a; b]$ можно представить как $f = g + f_{ac} + f_0$, где
	\begin{itemize}
		\item $g$ является ступенчатой функцией
		
		\item $f_{ac}$ обладает абсолютной непрерывностью на $[a; b]$
		
		\item $f_0$ является произвольной функцией, почти всюду совпадающая с нулём.
	\end{itemize}
	Такое разложение функции называется \textit{разложением Лебега}.
\end{note}

\begin{proposition}
	Если $F, G \in AC[a; b]$, то $F \cdot G \in AC[a; b]$
\end{proposition}

\begin{proof}
	Раз функции непрерывны абсолютно, то они непрерывны просто, причём на отрезке. По теореме Больцано-Коши получаем, что они ограничены.
	\textcolor{red}{Дописать}
\end{proof}

\begin{corollary} (Формула интегрирования по частям в интеграле Лебега)
	Если $F, G \in AC[a; b]$, то имеет место формула интегрирования по частям:
	\[
		\int_{[a; b]} F(x)G'(x)d\mu(x) = F(b)G(b) - F(a)G(a) - \int_{[a; b]} F'(x)G(x)d\mu(x)
	\]
\end{corollary}

\begin{proof}
	Коль скоро $F, G \in AC[a; b]$, то $F \cdot G \in AC[a; b]$ по уже доказанному утверждению. Стало быть, $FG$ является неопределенным интегралом Лебега и верна формула:
	\[
		\int_{[a; b]} (FG)'d\mu(x) = F(b)G(b) - F(a)G(a) = \int_{[a; b]} F(x)G'(x)d\mu(x) + \int_{[a; b]} F'(x)G(x)d\mu(x)
	\]
\end{proof}
