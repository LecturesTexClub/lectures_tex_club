\subsection{Гиперграфы}

\begin{definition}
	\textit{Гиперграфом} $H = (V, E)$ называется пара из множеств $V$ --- вершины и $E \subseteq 2^V$ --- гиперрёбра. Если не оговорено явно обратного, то мы требуем, что у любого гиперребра как минимум 2 вершины, ему принадлежащие: $\forall A \in E\ |A| \ge 2$.
\end{definition}

\begin{definition}
	Гиперграф $H = (V, E)$ называется $r$-однородным, если мощность любого его ребра равна $r$:
	\[
		\forall A \in E\ \ |A| = r
	\]
\end{definition}

\begin{definition}
	Величиной $f(n, r, s)$ в теории гиперграфов обозначается экстремальная величина, равная максимальному $k \in \N$, для которого существует $r$-однородный гиперграф с $k$ рёбрами на $n$ вершинах такой, что $\forall A, B \in E\ \ |A \cap B| \ge s$
\end{definition}

\begin{note}
	$f(n, r, s)$ отражает максимальное количество рёбер в графе на $n$ вершинах, где они плотно пересекаются (по не менее чем $s$ вершинам).
\end{note}

\begin{definition}
	Величиной $h(n, r, s)$ в теории гиперграфов обозначается экстремальная величина, равная максимальному $k \in \N$, для которого существует $r$-однородный гиперграф с $k$ рёбрами на $n$ вершинах такой, что $\forall A, B \in E\ \ |A \cap B| \le s$
\end{definition}

\begin{note}
	$h(n, r, s)$ отражает максимальное количество рёбер в гиперграфе на $n$ вершинах, где они разреженно пересекаются (по не более чем $s$ вершинам).
\end{note}

\begin{note}
	Гиперграфы, которые рассматривает $h(n, r, s)$, уже встречались ранее в курсе ОКТЧ. Это задача о кодах, исправляющих ошибки. У нас снова сообщение из $n$ битов с ровно $r$ числом единиц. Требование $|A \cap B| \le s$ эквивалентно $(\vec{x}_A, \vec{x}_B) \le s$. Это же значит, что у нас не более $s$ одинаковых позиций, на которых стоят единицы, то есть расстояние между словами $\ge 2(r - s)$. Это значит, что мы можем позволить делать $< r - s$ искажений на каждое сообщение, чтобы мы всё ещё могли отличить одно от другого и восстановить их.
\end{note}

\begin{definition}
	Величиной $m(n, r, s)$ в теории гиперграфов обозначается экстремальная величина, равная максимальному $k \in \N$, для которого существует $r$-однородный гиперграф с $k$ рёбрами на $n$ вершинах такой, что $\forall A, B \in E\ \ |A \cap B| \neq s$
\end{definition}

\begin{note}
	$m(n, r, s)$ отражает максимальное количество рёбер в гиперграфе на $n$ вершинах, где просто запрещено пересечение по конкретному числу $s$.
\end{note}

\begin{note}
	В конце ОКТЧ у нас был один интересный граф для теоремы Эрдеша-Хватала: $V = \{A \subseteq \{1, \ldots, n\} \colon |A| = 3\}$, $E = \{(A, B) \colon |A \cap B| = 1\}$. Его естественно можно обобщить до общей конструкции. Также мы доказали следующий факт про число независимости этого графа:
	\[
		\alpha(G(n, 3, 1)) = \System{
			&{n,\ n \equiv 0 \pmod 4}
			\\
			&{n - 1,\ n \equiv 1 \pmod 4}
			\\
			&{n - 2, \text{ иначе}}
		}
	\]
\end{note}

\begin{definition}
	Графом $G(n, r, s)$ называется граф $(V, E)$, где
	\begin{itemize}
		\item $V = \{A \subseteq \{1, \ldots, n\} \colon |A| = r\}$
		
		\item $E = \{(A, B) \colon |A \cap B| = s\}$
	\end{itemize}
\end{definition}

\begin{proposition}
	Имеет место равенство: \(m(n, r, s) = \alpha(G(n, r, s))\)
\end{proposition}

\begin{proof}
	По определению
\end{proof}

\begin{theorem}
	Имеет место оценка сверху: \(h(n, r, s) \le C_n^{s + 1} / C_r^{s + 1}\)
\end{theorem}

\begin{proof}
	Рассмотрим любой гиперграф, удовлетворяющий начальным условиям $h(n, r, s)$. Пусть $A_1, \ldots, A_k$ --- это его гиперрёбра. Тогда $|A_i| = r$, $|A_i \cap A_j| \le s$. Рассмотрим все возможные $(s + 1)$-элементные подмножества вершин гиперграфа (то есть подмножества $\{1, \ldots, n\}$). С одной стороны, есть $C_r^{s + 1}$ таких подмножеств, которые лежат внутри каждого из $A_i$. Обозначим соответствующий набор подмножеств за $\cA_i$, то есть $|\cA_i| = C_r^{s + 1}$. Тогда из-за условия очевидно, что $\cA_i \cap \cA_j = \emptyset$. С другой стороны, всего $(s + 1)$-элементных подмножеств будет $C_n^{s + 1}$, а значит верна оценка:
	\[
		|\cA_1| + \ldots + |\cA_k| = k \cdot C_r^{s + 1} \le C_n^{s + 1} 
	\]
\end{proof}

\subsubsection*{Небольшая историческая справка}

\begin{itemize}
	\item В 1980г. математиком по фамилии Rödl было доказано, что если $r(n) = const, s(n) = const$, то имеет место неравенство \(h(n, r, s) \ge (1 + o(1)) \frac{C_n^{s + 1}}{C_r^{s + 1}}\)
	
	\item Около 2015г. математик по фамилии Keevash доказал в условиях соответствующей делимости, что $h(n, r, s) \ge \frac{C_n^{s + 1}}{C_r^{s + 1}}$. Более того, с некоторого $n$ это неравенство становится равенством.
\end{itemize}

\begin{theorem} (Эрдёш, Ко, Радо)
	Если $r \le \floor{n / 2}$, то $f(n, r, 1) = C_{n - 1}^{r - 1}$.
\end{theorem}
