\subsection{Кнезеровские графы}

\begin{definition}
	\textit{Кнезеровским графом} называется $KG_{n, r} := G(n, r, 0) = (V, E)$, то есть:
	\begin{itemize}
		\item $V = C_{\range{n}}^r$
		
		\item $\forall A, B \in V\ \ (A, B) \in E \Lra A \cap B = \emptyset$
	\end{itemize}
\end{definition}

\begin{proposition}
	$\alpha(KG_{n, r}) = C_{n-1}^{r-1}$, если $2r \le n$
\end{proposition}

\begin{proof}
	Следует из определений
\end{proof}

\begin{proposition}
	Простыми рассуждениями можно утверждать, что $\ceil{n / r} \le$ \\ $\chi(KG_{n, r}) \le n - 2r + 2$.
\end{proposition}

\begin{proof}~
	\begin{itemize}
		\item Для оценки снизу мы просто можем воспользоваться связью хроматического числа с числом независимости или кликовым числом графа:
		\[
			\chi(KG_{n, r}) \ge \frac{C_n^r}{\alpha(KG_{n, r})} = \frac{C_n^r}{C_{n - 1}^{r - 1}} = \frac{n}{r} \Ra \chi(KG_{n, r}) \ge \ceil{\frac{n}{r}}
		\]
		Кликовое число даст ту же самую оценку с точностью до округления.
		
		\item Покрасим все вершины, у которых минимальный элемент равен $i < n - 2r + 2$, в соответствующий цвет. Тогда для покрашенных вершин мы получили корректную раскраску, а оставшихся чисел всего $2r - 1$. Это значит, что любые вершины, чьи множества полностью лежат в оставшихся числах, обязательно пересекаются хотя бы по 1 элементу, а потому мы объявляем их всех торжественно покрашенными в последний цвет.
	\end{itemize}
\end{proof}

\begin{example}
	Есть 2 примера, реализующих наши оценки на хроматическое число:
	\begin{enumerate}
		\item Случай, когда $r = n / 2$. Тогда $KG_{n, r}$ --- это просто паросочетание. Подстановка говорит, что он красится в 2 цвета, а это действительно так.
		
		\item Случай, когда $r = 1$. Это вообще просто полный граф на $n$ вершинах.
		
		\item $KG_{5, 3}$ --- это так называемый граф Петерсена
	\end{enumerate}
\end{example}

\begin{hypothesis}(Кнезера)
\[
\chi(KG_{n,r}) = n - 2r + 2, n \ge 2r
\]
Гипотеза оказалась верной и была доказана Ласло Ловасом.
\end{hypothesis}

\begin{theorem}(1930-1932, Борсук, Улан, Люстерник, Шкирельман)
$S^{n-1} \subset \mathbb{R}^n$ - сфера. $S^{n-1} = A_1 \cup \dots \cup A_n$, $\forall i$ $A_i$ замкнуто. Тогда $\exists i$ $\exists \vv{x} \in A_i: -\vv{x} \in A_i$.

\end{theorem}

\begin{proof}
Докажем для плоскости. Пусть $\vv{x} \in A_1$. Пусть $-\vv{x} \in A_2$, иначе теорема доказана. Будем двигаться по дуге от $\vv{x}$ до $- \vv{x}$. Так как $A_1$ замкнуто, можем найти на ней последнюю точку $\vv{y} \in A_1$. Если $\vv{y}$ не принадлежит $A_2$, то в силу замкнутости $\vv{y}$ не принадлежит $A_2$ вместе с некоторой своей окрестностью. Значит, $\vv{y}$  вместе с этой окрестностью принадлежит $A_1$, но тогда это не последняя точка. Противоречие.
\end{proof}


\begin{proof}(теоремы Ловаса)


Предположим противное: $\chi(KG_{n,r}) \le n-2r+1 = d$. Обозначим через $A_1, \dots, A_{C_n^2}$ вершины $KG_{n,r}$, а через $\chi_1, \dots, \chi_d$ - цвета, в которые нам удалось их покрасить. Если $A_i \cap A_j = \emptyset$, то $\chi(A_i) \neq \chi(A_j).$\\
Перенесём наш граф на сферу $S^d\subset \mathbb{R}^{d+1}$. Поместим на неё точки $\vv{x_1}, \dots, \vv{x_n}$ так, чтобы на каждом "экваторе" было не более $d$ точек. Экватором называем пересечение $d$-мерной сферы с $d$-мерной плоскостью, проходящей через $\vv{0}$. Как это можно сделать? Берём первые $d$ точек как угодно, а следующую ставим так, чтобы она не попадала на экватор, содержащий данные $d$ точек и начало координат. Потом располагаем точку так, чтобы избежать всех экваторов, которые образуют уже поставленные точки и так далее. \\
Обозначим через $H(\vv{x})$ открытую сферу с эпицентром в точке $\vv{x}$. Это часть полусферы, которая содержит $\vv{x}$ без экватора, образованного гиперплоскостью, перпендикулярной $OX$. \\
Построим множества $B_1, \dots, B_{d}$:

\[
B_i = \{ \vv{x} \in S^{d} \such \ \text{в} \ H(\vv{x}) \cap \{ \vv{x_1}, \dots, \vv{x_n}  \} \ \text{есть хотя бы одна вершина} \ A_j \in V(KG_{n,r}) \such \chi(A_j) = \chi_i  \}
\]

Не факт, что $B_1, \dots, B_{d}$ покрывают всю сферу:

\[
B_{d+1} = \{ \vv{x} \in S \such |H(\vv{x}) \cap \{\vv{x_1}, \dots, \vv{x_n} \}| \le r - 1 \} 
\]

Теперь сфера покрывается $B_1, \dots, B_{d+1}$ полностью. \\

Заметим, что $B_1, \dots, B_d$ открыты. Неформально говоря, если для некоторой точки $\vv{x}$ в её полусфере нашлась точка графа цвета $\chi_i$, то если мы немного подвинем $\vv{x}$, то немного подвинется его полусфера, но это можно сделать так, чтобы вершина графа всё равно там осталась, ведь полусфера открыта, а значит, ни одна точка вершины не лежит на границе. $B_{d+1}$ при этом замкнуто как дополнение объединения открытых. 

Воспользуемся теперь теоремой 1.24:

Рассмотрим два случая:

\begin{itemize}
    \item $\exists i \in \{1, \dots, d\} \such B_i \ \text{содержит} \ \vv{x} \ \text{и} \ \vv{-x}.$  Но тогда на противоположных сторонах сферы есть два $r$-элементных множества одного цвета. Но эти множества не пересекаются, а значит, не могут одновременно иметь цвет $\chi_i$. Противоречие.
    \item $\vv{x}, \vv{-x} \in B_{d+1}.$ Значит, в верхней полусфере не больше, чем $r-1$ точка, в нижней тоже не больше чем $r-1$ точка. Но всего точек $n$, значит, все остальные лежат на экваторе, а их там хотя бы $n -2(r-1) = n-2r+2 = d + 1$. Но мы распологали точки так, чтобы на экваторе их было не более $d$. Противоречие.
\end{itemize}
\end{proof}

