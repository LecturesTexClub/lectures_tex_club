\subsection{Жадные алгоритмы для оценки характеристических чисел случайных графов}

\begin{note}
	Оценку на хроматическое число графа, которую можно получить данным алгоритмом, обозначим за $\chi_g(G)$ (g - greedy).
\end{note}

\begin{theorem}
	Если ввести $(\Omega, F, P)$ --- вероятностное пространство случайных графов ($|\Omega| = 2^{C_n^2}$, $F = 2^\Omega$, $P$ --- равномерная вероятность), то для описанных жадных алгоритмов и их оценок имеют место следующие утверждения:
	\begin{itemize}
		\item \(\forall \eps > 0\ \ P\ps{G \colon \frac{\alpha(G)}{\alpha_g(G)} \le 2 + \eps} \xrightarrow[n \to \infty]{} 1\)
	\end{itemize}
\end{theorem}

\begin{note}
	Для жадных алгоритмов $\eps > 0$ из вероятности убрать нельзя. Однако, существуют и такие детерминированные алгоритмы, для которых это сделать возможно.
\end{note}

\begin{proof}
	Мы уже знаем, что $\lim_{n \to \infty} P(\alpha(G) < 2\log_2 n) = 1$. Если мы как-то докажем, что
	\[
		\forall \eps > 0\ \ \lim_{n \to \infty} P(\alpha_g(G) \ge (1 - \eps)\log_2 n) = 1
	\]
	То этого будет достаточно. Действительно, обратим внимание на 2 факта:
	\begin{enumerate}
		\item Если $A_n \subseteq B_n$ и $\lim_{n \to \infty} P(A_n) = 1$, то $\lim_{n \to \infty} P(B_n) = 1$ --- просто из свойства $P(A_n) \le P(B_n) \le 1$
		
		\item Что из себя представляет пересечение свойств $\alpha(G) < 2\log_2 n$ и $\alpha_g(G) \ge (1 - \eps)\log_2 n$?
		\[
			\frac{\alpha(G)}{\alpha_g(G)} \le \frac{2\log_2 n}{(1 - \eps)\log_2 n} = \frac{2}{1 - \eps} = 2 + \eps'
		\]
		Понятно, что из предела к 1 при $\eps$ мы вполне законно говорим и о пределе для $\eps'$.
	\end{enumerate}
	Чтобы доказать достаточно неравенство, рассмотрим его дополнение:
	\[
		\forall \eps > 0\ \ \lim_{n \to \infty} P(\underbrace{\alpha_g(G) < (1 - \eps)\log_2 n}_{\mathcal{A}}) = 0
	\]
	Теперь покажем некоторое событие $\mathcal{B} \subseteq \mathcal{A}$, потому что из $\mathcal{A} \Ra \mathcal{B}$. Это позволит оценить вероятность $\mathcal{A}$ сверху:
	\[
		\mathcal{B} = \System{
			&{\exists a_1, \ldots, a_m \in \N_0 \colon \forall i\ a_i < (1 - \eps)\log_2 n}
			\\
			&{\exists C_1, \ldots, C_m \subseteq V \colon \forall i\ |C_i| = a_i}
			\\
			&{\forall i \neq j\ \ C_i \cap C_j = \emptyset}
			\\
			&{\forall x \notin (C_1 \cup \ldots \cup C_m)\ \forall i\ \exists y \in C_i \colon (x, y_i) \in E}
		}
	\]
	Теперь объясним, что такое $\mathcal{B}$ и каково $m$. Что утверждает $\mathcal{A}$ в нашем алгоритме? А то, что он не смог найти одноцветное множество размера $\ge (1 - \eps)\log_2 n$. Значит, $\chi_g(G)$ лежит в полуинтервале $(n / (1 - \eps)\log_2 n; n]$. Положим $m = \floor{n / \big(2(1 - \eps)\log_2 n\big)}$ и выберем произвольные $m$ одноцветных множеств, найденных алгоритмом. Тогда, они удовлетворяют всем условиям выше, и, более того, верно следующее:
	\[
		\sum_{i = 1}^m |C_i| = \sum_{i = 1}^m a_i < m (1 - \eps)\log_2 n \le \frac{n}{2}
	\]
	Теперь, начнём восстанавливать вероятность $P(\mathcal{B})$ по цепочке снизу-вверх:
	\begin{itemize}
		\item Пусть $x \notin (C_1 \cup \ldots \cup C_m)$ и $i \in \range{m}$ уже зафиксированы, и мы хотим найти вероятность оставшегося:
		\[
			P(\exists y \in C_i \colon (x, y_i) \in E) = 1 - P(\forall y \in C_i\ (x, y_i) \notin E) = 1 - \frac{1}{2^{a_i}}
		\]
		
		\item Теперь позволим $i$ быть произвольным. Так как $\forall i \neq j\ C_i \cap C_j = \emptyset$, то выбор рёбер с разными множествами вообще никак не влияет друг на друга (иначе говоря, события предыдущего пункта независимы):
		\[
			P(\forall i\ \exists y \in C_i \colon (x, y_i) \in E) = \prod_{i = 1}^m \ps{1 - \frac{1}{2^{a_i}}} < \ps{1 - \frac{1}{2^{(1 - \eps)\log_2 n}}}^m = \ps{1 - \frac{1}{n^{1 - \eps}}}^m
		\]
		
		\item Теперь мы добавляем варьирование $x$. Вероятность тогда тоже будет считаться через произведение, ибо снова имеется независимость событий:
		\begin{multline*}
			P(\forall x \notin (C_1 \cup \ldots \cup C_m)\ \forall i\ \exists y \in C_i \colon (x, y_i) \in E) \le \ps{1 - \frac{1}{n^{1 - \eps}}}^{m \cdot \frac{n}{2}} = e^{\frac{mn}{2} \cdot \ln \ps{1 - \frac{1}{n^{1 - \eps}}}} \le
			\\
			\exp\ps{-\frac{mn}{2} \cdot \frac{1}{n^{1 - \eps}}} = \exp\ps{-\frac{mn^\eps}{2}}
		\end{multline*}
		Для достаточно больших $n$ мы можем сделать оценку на $m$:
		\[
			m = \floor{\frac{n}{2(1 - \eps)\log_2 n}} \ge \frac{n}{2\log_2 n}
		\]
		Из этого следует последняя оценка на вероятность с $x$:
		\[
		P(\forall x \notin (C_1 \cup \ldots \cup C_m)\ \forall i\ \exists y \in C_i \colon (x, y_i) \in E) \le \exp\ps{-\frac{n^{1 + \eps}}{4\log_2 n}}
		\]
		
		\item Теперь высвободим строчку с $\exists C_1, \ldots, C_m \subseteq V \ldots$ Вероятность будем считать через дробление на отдельно взятые $C_i$ (то есть для данного графа $G$ мы перебираем все возможные $C_i$ во вложенных суммах). Понятно, что вероятности всех <<плохих>> наборов подмножеств занулятся (ибо будет ложное утверждение в событии) и останутся только те, что подходят условиям на $C_i$. Итого:
		\begin{multline*}
			P(\exists C_1, \ldots, C_m \subseteq V \ldots) < \exp\ps{-\frac{n^{1 + \eps}}{4\log_2 n}} \sum_{C_1, \ldots, C_m \subseteq V \colon \over {\forall i\ |C_i| = a_i \over \forall i \neq j\ C_i \cap C_j = \emptyset}} 1 =
			\\
			\exp\ps{-\frac{n^{1 + \eps}}{4\log_2 n}} \cdot C_n^{a_1} \cdot C_{n - a_1}^{a_2} \cdot \ldots \cdot  C_{n - a_1 - \ldots - a_{m - 1}}^{a_m} <
			\\
			\exp\ps{-\frac{n^{1 + \eps}}{4\log_2 n}} \cdot n^{a_1 + \ldots + a_m} \le \exp\ps{-\frac{n^{1 + \eps}}{4\log_2 n} + \frac{n \ln n}{2}}
		\end{multline*}
		
		\item Ну и наконец-то мы приступаем к оценке $P(\mathcal{B})$:
		\begin{multline*}
			P(\mathcal{B}) \le \sum_{a_1 = 1}^{(1 - \eps)\log_2 n} \cdots \sum_{a_m = 1}^{(1 - \eps)\ln n} \exp\ps{-\frac{n^{1 + \eps}}{4\log_2 n} + \frac{n \ln n}{2}} <
			\\
			(\log_2 n)^m \cdot \exp\ps{-\frac{n^{1 + \eps}}{4\log_2 n} + \frac{n \ln n}{2}} \le \exp\ps{\frac{n\ln(\log_2 n)}{2(1 - \eps)\log_2 n}-\frac{n^{1 + \eps}}{4\log_2 n} + \frac{n \ln n}{2}} \xrightarrow[n \to \infty]{} 0
		\end{multline*}
	\end{itemize}
\end{proof}

\begin{hypothesis}
	Не существует полиномиального алгоритма $A$, для которого имеет место предел:
	\[
		P\ps{\frac{\alpha(G)}{\alpha_A(G)} < 2} \xrightarrow[n \to \infty]{} 1
	\]
\end{hypothesis}

\begin{theorem} (Куч\'{е}ра, без доказательства)
	$\forall \eps > 0\ \forall \delta > 0\ \exists \{G_n\}_{n = 1}^\infty \such |V_n| = n$ верно, что
	\[
		\frac{\# \text{перестановок множества вершин, при которых } \frac{\alpha(G_n)}{\alpha_g(G_n) \ge n^{1 - \eps}}}{n!} > 1 - \delta
	\]
\end{theorem}

