\subsection{Теоремы Франкла и Уилсона}

\begin{theorem} (1981г., Франкл, Уилсон)
	Пусть $r - s = p^\alpha$, где $p$ - простое, $\alpha \in \N$. Также потребуем, что $r - 2p^\alpha < 0$. Тогда верна следующая оценка сверху:
	\[
		m(n, r, s) \le \sum_{k = 0}^{p^\alpha - 1} C_n^k
	\]
\end{theorem}

\begin{note}
	Если $r(n), s(n) = const$, то имеет место асимтотическое утверждение:
	\[
		m(n, r, s) \sim \sum_{k = 0}^{p^\alpha - 1} C_n^k \sim \frac{n^{r - s - 1}}{(r - s - 1)!}
	\]
	Это работает, коль скоро фиксация $r, s$ даёт нам фиксированное значение $p^\alpha$ для каждого $n$. Стало быть, сумму коэффициентов можно оценить последним.
\end{note}

\begin{proof}
	Доказывается только случай $\alpha = 1$, он по сути является обобщением того, что уже доказали для $m(n, 5, 2)$. Возьмём произвольный гиперграф $H = (\{1, \ldots, n\}, E)$, удовлетворяющий условиям $m(n, r, s)$. Обозначим число гиперрёбер за $t$, каждое ребро за $A_i$ и перейдём к уже известным векторам $\vv{x}_i$. В общем случае, многочлен $P_i \in \Z_p[\vv{y}]$ имеет вид:
	\[
		P_i(\vv{y}) = \prod_{j = 0 \atop {j \neq s}}^{p - 1} ((\vv{x}_i, \vv{y}) - j)
	\]
	Понятно, что $\deg P_i \le p - 1$. Мы живём и смотрим на многочлены в таком пространстве, что  них подставляются только $\vv{y} \in \{0, 1\}^n$. Тогда $P'_i$ --- это просто раскрытый многочлен $P_i$, где у каждого одночлена срезали степени переменных до не более 1й. Тогда каков базис пространства $\dim \trb{P'_i}$? Он состоит из всех одночленов вида $1$, $y_1$, $y_1y_2$, $\ldots$, $y_1\ldots y_{p - 1}$. Иначе говоря, верно равенство:
	\[
		\dim \trb{P'_i} = C_n^0 + C_n^1 + \ldots + C_n^{p - 1} = \sum_{k = 0}^{p - 1} C_n^k
	\]
	Осталось стандартным образом показать, что многочлены $P'_i$ независимы. Запишем произвольную линейную комбинацию:
	\[
		c_1P'_1 + \ldots + c_tP'_t = 0,\ c_i \in \Z_p
	\]
	Снова рассмотрим, например, подстановку $\vv{y} = \vv{x}_1$. Тогда:
	\begin{enumerate}
		\item $P'_1(\vv{x_1}) = \prod_{j = 0 \atop {j \neq s}}^{p - 1} (r - j)$, при этом $r - s = p \Ra s = r - p$ и $r - 2p < 0$. Говоря иначе, наши ограничения так хитро подобраны, что в произведении не может получиться числа, кратного $p$. Стало быть, $P'_1(\vv{x}_1) \neq 0$
		
		\item $\forall i > 1\ P'_i(\vv{x}_1) = \prod_{j = 0 \atop {j \neq s}}^{p - 1} ((\vv{x}_i, \vv{x}_1) - j)$. Проверим, что $(\vv{x}_i, \vv{x}_1) \centernot\equiv s \pmod p$ (тогда мы точно занулимся по модулю). Мы знаем, что в просто целых числах $(\vv{x}_i, \vv{x}_1) < r$. Более того, по требованию $m(n, r, s)$ у нас всегда $(\vv{x}_i, \vv{x}_1) \neq s = r - p$, а $r - 2p$ уже меньше нуля. Стало быть, $P'_i(\vv{x}_1) \equiv 0 \pmod p$
	\end{enumerate}
	Снова пришли к тому, что $c_1 = 0$, но то же самое верно и для остальных векторов.
\end{proof}

\begin{theorem} (Улучшенная теорема Франкла-Уилсона)
	Пусть $r - s = p^\alpha$, где $p \in \bP$, $\alpha \in \N$. Тогда верна следующая оценка сверху:
	\[
		m(n, r, s) \le \sum_{k = 0}^{p^\alpha - 1} C_n^k
	\]
\end{theorem}

\begin{proof}
	Разница состоит в том, что мы отказались от требования $r - 2p < 0$. Так как этот случай доказан, то здесь нужно разобрать $r - 2p \ge 0$. \textcolor{red}{А это произойдёт уже после сессии, это вопрос на отл9.}
\end{proof}