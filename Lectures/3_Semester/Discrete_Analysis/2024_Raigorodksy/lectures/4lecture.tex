\subsection{Биномиальная модель случайного графа. Теорема Эрдеша.}


\begin{theorem} 
Пусть в $G(n, p)$, где $p = p(n)$, причём $np(n) \to \infty, n \to \infty$. $\xi(G)$ - случайная величина, равная количеству треугольников в графе $G$. Тогда\\
\begin{enumerate}
    \item если $np \to 0, p \to \infty$, то $P(\xi = 0) \to 1.$
    \item если $np \to \infty$, то $P(\xi = 0) \to 0$
\end{enumerate}
\end{theorem}

\begin{proof}
	1. Воспользуемся классическим приёмом теории вероятностей: разобьём нашу случайную величину на сумму индикаторов:
	\[
		\xi = \xi_1 + \dots + \xi_{C_n^3}
	\]
	где $\xi_i$ --- это индикатор того, что $i$-я тройка оказалась подграфом $G$. Тогда
	\[
		E\xi = \sum_{i = 1}^{C_n^3} E \xi_i = C_n^3 \cdot p^3 \sim \frac{(np)^3}{6} \xrightarrow[n \to \infty]{} 0
	\]
	По неравенству Маркова отсюда моментально следует, что и $P(\xi \ge 1) \to 0$ при $n \to \infty$.\\
    2. Пусть $\xi(G) = $ число треугольников в $G$. Тогда $P(\xi \ge 1) = 1 - P(\xi \le 0) = 1 - P(\xi = 0)$.  Провернём такой цыганский фокус:
	\[
		P(\xi \ge 1) = 1 - P(\xi \le 0) = 1 - P(-\xi \ge 0) = 1 - P(E \xi - \xi \ge E \xi)
	\]
	Теперь, мы можем в 2 действия получить оценку неравенством Чебышёва:
	\[
		P(E \xi - \xi \ge E \xi) \le P(|E \xi - \xi| \ge E \xi) \le \frac{DX}{(E \xi)^2} \Longrightarrow P(\xi \ge 1) \ge 1 - \frac{DX}{(E \xi)^2}
	\]
	При этом $DX = E \xi^2 - (E \xi)^2$, иначе говоря
	\[
		P(\xi \ge 1) \ge 2 - \frac{E \xi^2}{(E \xi)^2}
	\]
	Распишем матожидание квадрата:
	\begin{multline*}
		E \xi^2 = E (\xi_1 + \dots + \xi_{C_n^3})^2 = E({\xi_1^2 + \dots + \xi_{C_n^3}^2 + \sum_{i \neq j} \xi_i \xi_j}) =
		\\
		\underbrace{E(\xi_1 + \ldots + \xi_{C_n^3})}_{E \xi} + \sum_{i \neq j} E(\xi_i \cdot \xi_j)
	\end{multline*}
	Чтобы посчитать сумму, нужно понять, что есть 3 возможных ситуации:
	\begin{enumerate}
		\item Тройки $i, j$ пересекаются по двум элементам. Для суммы таких $i, j$ получится выражение $(C_n^3 \cdot C_3^2 \cdot C_{n - 3}^1) \cdot p^5$
		
		\item Тройки $i, j$ пересекаются по одному элементу. Получаем $(C_n^3 \cdot C_3^1 \cdot C_{n - 3}^2) \cdot p^6$
		
		\item Тройки $i, j$ не пересекаются. Тогда $(C_n^3 \cdot C_{n - 3}^3) \cdot p^6$
	\end{enumerate}
	Теперь подставим всё в дробь:
	\begin{multline*}
		\frac{E \xi^2}{(E \xi)^2} = \frac{E \xi + 3p^5 C_n^3 C_{n - 3}^1 + 3p^6 C_n^3 C_{n - 3}^2 + p^6 C_n^3 C_{n - 3}^3}{(C_n^3)^2 p^6} =
		\\
		\frac{1}{E \xi} + \frac{3C_{n - 3}^1}{pC_n^3} + \frac{3C_{n - 3}^2}{C_n^3} + \frac{C_{n - 3}^3}{C_n^3} \xrightarrow[n \to \infty]{} 1
	\end{multline*}
	Этого нам и достаточно.
\end{proof}

\begin{theorem} (без доказательства)
	Пусть в $G(n, p)$, где $p = p(n)$, причём $np \to c > 0, n \to \infty$. Если $\xi(G)$ --- это число треугольников в графе, то
	\[
		P(\xi = 0) \xrightarrow[n \to \infty]{} e^{-\frac{c^3}6}
	\]
\end{theorem}

\begin{definition}
	\textit{Обхватом} (на англ. \textit{girth}) графа $G$ называется величина $g(G)$, равная длине минимального цикла в этом графе.
\end{definition}

\begin{proposition}
	Чтобы лучше прочувствовать определение, заметим такое интересное свойство:
	\[
		g(G) > 3 \Lra \text{в $G$ нет треугольников}
	\]
\end{proposition}

\begin{proof}
	Действительно, $g(G) > 3$ означает, что в графе не существует цикла длины 3, коим является треугольник.
\end{proof}

\begin{theorem} (1957, Эрдеш)
	Имеет место следующее утверждение:
	\[
		\forall k, l \in \mathbb{N}\ \exists G = (V, E) \mid \chi(G) > k,\ g(G) > l
	\]
\end{theorem}

\begin{proof}
	$G(n, p)$, $p = p(n) = n^{\Theta - 1}$, $\Theta = \frac{1}{2l}$.
	
	$X_l := X_l(G)$ --- количество простых циклов длины $\le l$ в $G$. Посмотрим матожидание:
	\[
		\E X_l = \sum_{r = 3}^l C_n^r \frac{(r - 1)!}{2} p^r < \sum_{r = 3}^l \frac{n^r}{r!} \cdot r! \cdot p^r = \sum_{r = 3}^l (np)^r < l \cdot n^{\Theta l} = l \cdot n^{1/2}
	\]
	По неравенству Маркова:
	\[
		P\ps{X_l \ge \frac{n}{2}} \le \frac{\E X_l}{n / 2} < \frac{l\sqrt{n}}{n / 2} \xrightarrow[n \to \infty]{} 0
	\]
	Отсюда возникает следствие:
	\[
		\exists n_1 \mid \forall n \ge n_1\ P\ps{X_l < \frac{n}{2}} > \frac{1}{2}
	\]
	\textcolor{red}{Тут был замысел, около 16й минуты 7й лекции}
	
	$Y_x := Y_x(G)$ --- количество независимых множеств размера $x$ в $G$. Положим $x = \lceil{\frac{3\ln n}{p}}\rceil \sim \frac{3\ln n}{p}$. Оценим матожидание:
	\[
		E Y_x = C_n^x (1 - p)^{C_x^2} \le n^x \cdot e^{-pC_x^2} = \exp({x\ln n - (1 + o(1))p \frac{x^2}{2}})
	\]
	Посмотрим асимптотику штуки в скобочках:
	\[
		x\ps{\ln n - (1 + o(1))\frac{px}{2}} \sim x\ps{\ln n - (1 + o(1))\frac{3\ln n}{2}} \xrightarrow[n \to \infty]{} -\infty
	\]
	Снова применим неравенство Маркова:
	\[
		P(\alpha(G) \ge x) = P(Y_x \ge 1) \le \frac{\E Y_x}{1} \to 0
	\]
	Стало быть, $\exists n_2 \mid \forall n \ge n_2\ P(\alpha(G) < x) > 1 / 2$. Теперь можно рассмотреть $n > \max \{n_1, n_2\}$, а потому
	\[
		\exists G \mid X_l(G) < \frac{n}{2} \wedge \alpha(G) < x
	\]
	Получим индуцированный граф $G'$ из $G$ путём удаления вершин и инциндентных им рёбер. Добьёмся того, что $X_l(G') = 0$, то есть $g(G') > l$. Чтобы убрать все простые циклы (по свойству $G$), нам потребуется удалить не более $n / 2$ вершин. Иначе говоря, $|V(G')| > n - n / 2 = n / 2$. При этом, мы не могли улучшить число независимости \textcolor{red}{почему?}. Отсюда
	\[
		\chi(G') > \frac{n}{2x} \sim \frac{np}{2 \cdot 3\ln n} = \frac{n^\Theta}{6\ln n} \to +\infty
	\]
	По итогу $\exists n_3 \mid \forall n \ge n_3\ \chi(G') > k$.
\end{proof}
