\begin{proof}
	Рассмотрим произвольный гиперграф $H = (V, E)$, удовлетворяющий условиям $f(n, r, 1)$. Пусть $\cF = \{F_1, \ldots, F_t\}$ --- множество из произвольных $t$ рёбер этого графа. Дополнительно введём $\cA = \{A_1, \ldots, A_n\}$ --- рёбра гиперграфа, которые получаются сдвигами $A_1 = \{1, \ldots, r\}$ по кругу ($A_2 = \{2, \ldots, r, r + 1\}$, $A_n = \{n, 1, \ldots, r - 1\}$)
	
	\textcolor{red}{Тут можно сделать красивый рисуночек, а пока можно посмотреть картинку на 15й лекции, 30:20}
	
	\begin{lemma}
		Верно неравенство $|\cA \cap \cF| \le r$
	\end{lemma}

	\begin{proof}
		Произведём разбор случаев:
		\begin{enumerate}
			\item $\cA \cap \cF = \emptyset$ --- сразу всё очевидно
			
			\item $|\cA \cap \cF| \ge 1$ Тогда, без ограничения общности, можно считать $A_1 \in \cF$ (иначе можно их просто перенумеровать). Какие рёбра из $\cA$ могут теоретически принадлежать $\cF$? Мы запишем их симметричными парами: $(A_2, A_{n - r + 2}), (A_3, A_{n - r + 3}), \ldots, (A_r, A_{n - r + r})$. В чём смысл объединять эти рёбра в пары? А в том, что они не могут одновременно находится в $\cF$, иначе просто не будет пересечения между ними. Стало быть, мы можем взять не более $r - 1$ множества, что вместе с $A_1$ даёт требуемую оценку.
		\end{enumerate}
	\end{proof}
	Рассмотреть множество $\cA$ мы могли не только на фиксированной последовательности вершин, но и на любой их перестановке. Для $\sigma \in S_n$ мы введём $\cA_\sigma$, которое как раз означает $\cA$, но уже на другой перестановке. Понятно, что лемма работает и для $\cA_\sigma$: $|\cA_\sigma \cap \cF| \le r$.
	
	Вся теорема сводится к подсчёту двумя способами следующей суммы:
	\[
		\sum_{i = 1}^t \sum_{\sigma \in S_n} \chi(F_i, \cA_\sigma) = \sum_{\sigma \in S_n} \sum_{i = 1}^t \chi(F_i, \cA_\sigma)
	\]
	где $\chi(F_i, \cA_\sigma)$ --- это индикатор вхождения $F_i$ в $\cA_\sigma$.
	\begin{itemize}
		\item Посмотрим на правую часть. Сумма внутри --- это ничто иное как $|\cF \cap \cA_\sigma|$, которое $\le r$. Так как всего перестановок $n!$, то мы можем через правую часть равенства оценить левую:
		\[
			\sum_{i = 1}^t \sum_{\sigma \in S_n} \chi(F_i, \cA_\sigma) = \sum_{\sigma \in S_n} \sum_{i = 1}^t \chi(F_i, \cA_\sigma) \le n! \cdot r
		\]
		
		\item Осталось разобраться с левой частью. Внутренняя сумма буквально равна числу таких перестановок, что $F_i$ лежит где-то на круге. Оно может начинаться на любой из $n$ позиций для вершин круга, при этом допускается свободно переставлять как все вершины вне $F_i$, так и все вершины отдельно внутри $F_i$. Итого $n \cdot r! \cdot (n - r)!$ перестановок. В результате получаем оценку на $t$:
		\[
			t \cdot n \cdot r! \cdot (n - r)! \le n! \cdot r \Ra t \le C_{n - 1}^{r - 1}
		\]
		Этого факта в совокупности с предыдущим утверждением достаточно для обоснования теоремы.
	\end{itemize}
\end{proof}