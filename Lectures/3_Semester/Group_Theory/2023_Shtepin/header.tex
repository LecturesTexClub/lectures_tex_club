\documentclass[a4paper,11pt]{article}

%%% Работа с русским языком
\usepackage{cmap}					% поиск в PDF
\usepackage{mathtext} 				% русские буквы в формулах
\usepackage[T2A]{fontenc}			% кодировка
\usepackage[utf8]{inputenc}			% кодировка исходного текста
\usepackage[english,russian]{babel}	% локализация и переносы
\usepackage{indentfirst}            % красная строка в первом абзаце
\frenchspacing                      % равные пробелы между словами и предложениями

%%% Дополнительная работа с математикой
\usepackage{amsmath,amsfonts,amssymb,amsthm,mathtools} % пакеты AMS
\usepackage{icomma}                                    % "Умная" запятая

%%% Свои символы и команды
\usepackage{centernot} % центрированное зачеркивание символа
\usepackage{stmaryrd}  % некоторые спецсимволы
\usepackage{dsfont}
\usepackage{amsthm}

\renewcommand{\epsilon}{\ensuremath{\varepsilon}}
\renewcommand{\phi}{\ensuremath{\varphi}}
\renewcommand{\kappa}{\ensuremath{\varkappa}}
\renewcommand{\le}{\ensuremath{\leqslant}}
\renewcommand{\leq}{\ensuremath{\leqslant}}
\renewcommand{\ge}{\ensuremath{\geqslant}}
\renewcommand{\geq}{\ensuremath{\geqslant}}
\renewcommand{\emptyset}{\ensuremath{\varnothing}}

\DeclareMathOperator*{\Mid}{\scalebox{1.1}{$\mid$}}

\DeclareMathOperator{\sgn}{sgn}
\DeclareMathOperator{\gd}{\text{НОД}}
\DeclareMathOperator{\lf}{\text{НОК}}
\DeclareMathOperator{\rk}{rk}
\DeclareMathOperator{\pr}{pr}
\DeclareMathOperator{\ort}{ort}
\DeclareMathOperator{\im}{Im}
\DeclareMathOperator{\ke}{Ker}
\DeclareMathOperator{\re}{Re}
\DeclareMathOperator{\cha}{char}
\DeclareMathOperator{\ord}{ord}
\DeclareMathOperator{\tr}{tr}
\DeclareMathOperator{\md}{mod}
\DeclareMathOperator{\Aut}{Aut}
\DeclareMathOperator{\Inn}{Inn}
\DeclareMathOperator{\Out}{Out}
\DeclareMathOperator{\Sub}{Sub}
\DeclareMathOperator{\End}{End}
\DeclareMathOperator{\GL}{GL}
\DeclareMathOperator{\SL}{SL}
\DeclareMathOperator{\diag}{diag}
\DeclareMathOperator{\inter}{\text{Inter}}
\DeclareMathOperator{\shift}{\text{Shift}}
\DeclareMathOperator{\stationar}{\text{St}}

\newcommand{\divby}{
	\mathrel{\vbox{\baselineskip.65ex\lineskiplimit0pt\hbox{.}\hbox{.}\hbox{.}}}
}
\newcommand{\notdivby}{\centernot\divby}
\newcommand{\N}{\mathbb{N}}
\newcommand{\Z}{\mathbb{Z}}
\newcommand{\Q}{\mathbb{Q}}
\newcommand{\R}{\mathbb{R}}
\newcommand{\Cm}{\mathbb{C}}
\newcommand{\F}{\mathbb{F}}
\newcommand{\id}{\mathrm{id}}
\newcommand{\imp}[2]{(#1\,\,$\ra$\,\,#2)\,\,}
\newcommand{\nset}[1]{\{1, \dotsc, #1\}}
\newcommand{\Chi}{\scalebox{1.1}{\raisebox{\depth}{$\chi$}}}
\newcommand{\FF}{\scalebox{0.95}{$\mathcal F$}}
\newcommand{\FFF}{\scalebox{0.55}{$\mathcal F$}}
\newcommand{\GG}{\scalebox{0.95}{$\mathcal G$}}
\newcommand{\GGG}{\scalebox{0.55}{$\mathcal G$}}

\renewcommand\labelitemi{$\triangleright$}

\let\bs\backslash
\let\vect\overline
\let\normal\trianglelefteqslant
\let\lra\Leftrightarrow
\let\ra\Rightarrow
\let\la\Leftarrow
\let\gl\langle
\let\gr\rangle
\let\sd\leftthreetimes
\let\emb\hookrightarrow
\let\mc\mathcal
\let\mf\mathfrak

%%% Перенос знаков в формулах (по Львовскому)
\newcommand*{\hm}[1]{#1\nobreak\discretionary{}{\hbox{$\mathsurround=0pt #1$}}{}}

%%% Работа с картинками
\usepackage{graphicx}    % Для вставки рисунков
\setlength\fboxsep{3pt}  % Отступ рамки \fbox{} от рисунка
\setlength\fboxrule{1pt} % Толщина линий рамки \fbox{}
\usepackage{wrapfig}     % Обтекание рисунков текстом

%%% Работа с таблицами
\usepackage{array,tabularx,tabulary,booktabs} % Дополнительная работа с таблицами
\usepackage{longtable}                        % Длинные таблицы
\usepackage{multirow}                         % Слияние строк в таблице

%%% Теоремы
\theoremstyle{plain}
\newtheorem{theorem}{Теорема}[section]
\newtheorem{lemma}{Лемма}[section]
\newtheorem{proposition}{Утверждение}[section]
\newtheorem*{exercise}{Упражнение}
\newtheorem*{problem}{Задача}

\theoremstyle{definition}
\newtheorem{definition}{Определение}[section]
\newtheorem*{corollary}{Следствие}
\newtheorem*{note}{Замечание}
\newtheorem*{reminder}{Напоминание}
\newtheorem*{example}{Пример}
\newtheorem*{remarkfrom}{Примечание авторов}
\newtheorem*{agreement}{Соглашение}
\newtheorem*{idea}{Идея доказательства}
\newtheorem*{algorithm}{Алгоритм}

\theoremstyle{remark}
\newtheorem*{solution}{Решение}

%%% Оформление страницы
\usepackage{extsizes}     % Возможность сделать 14-й шрифт
\usepackage{geometry}     % Простой способ задавать поля
\usepackage{setspace}     % Интерлиньяж
\usepackage{enumitem}     % Настройка окружений itemize и enumerate
\setlist{leftmargin=25pt} % Отступы в itemize и enumerate

\geometry{top=25mm}    % Поля сверху страницы
\geometry{bottom=30mm} % Поля снизу страницы
\geometry{left=20mm}   % Поля слева страницы
\geometry{right=20mm}  % Поля справа страницы

\setlength\parindent{15pt}                % Устанавливает длину красной строки 15pt
\linespread{1.3}                          % Коэффициент межстрочного интервала
%\setlength{\abovedisplayskip}{3pt}       % Отступы от выключных формул
%\setlength{\belowdisplayskip}{3pt}       % Отступы от выключных формул
%\setlength{\abovedisplayshortskip}{3pt}  % Отступы от выключных формул
%\setlength{\abovedisplayshortskip}{3pt}  % Отступы от выключных формул
%\setlength{\parskip}{0.5em}              % Вертикальный интервал между абзацами
%\setcounter{secnumdepth}{0}              % Отключение нумерации разделов
%\setcounter{section}{-1}                 % Нумерация секций с нуля
\usepackage{multicol}			          % Для текста в нескольких колонках
\usepackage{soulutf8}                     % Модификаторы начертания

%%% Содержание
\usepackage{tocloft}
\tocloftpagestyle{main}
%\setlength{\cftsecnumwidth}{2.3em}
%\renewcommand{\cftsecdotsep}{1}
%\renewcommand{\cftsecpresnum}{\hfill}
%\renewcommand{\cftsecaftersnum}{\quad}

%%% Шаблонная информация для титульного листа
\newcommand{\CourseName}{Теория групп}
\newcommand{\FullCourseNameFirstPart}{\so{ТЕОРИЯ ГРУПП}}
\newcommand{\SemesterNumber}{III}
\newcommand{\LecturerInitials}{Штепин Вадим Владимирович}
\newcommand{\CourseDate}{осень 2023}
\newcommand{\AuthorInitials}{Терехова Ольга}
\newcommand{\AuthorInitialssecond}{Парфенова Анна}
\newcommand{\VKLink}{https://vk.com/sqiportinhsa}
\newcommand{\VKLinksecond}{https://vk.com/id265834202}
\newcommand{\GithubLink}{https://github.com/MIPT-Group/Lectures_Tex_Club}

%%% Колонтитулы
\usepackage{titleps}
\newpagestyle{main}{
	\setheadrule{0.4pt}
	\sethead{\CourseName}{}{\hyperlink{intro}{\;Назад к содержанию}}
	\setfootrule{0.4pt}                       
	\setfoot{ФПМИ МФТИ, \CourseDate}{}{\thepage} 
}
\pagestyle{main}  

%%% Нумерация уравнений
\makeatletter
\def\eqref{\@ifstar\@eqref\@@eqref}
\def\@eqref#1{\textup{\tagform@{\ref*{#1}}}}
\def\@@eqref#1{\textup{\tagform@{\ref{#1}}}}
\makeatother                      % \eqref* без гиперссылки
\numberwithin{equation}{section}  % Нумерация вида (номер_секции).(номер_уравнения)
\mathtoolsset{showonlyrefs=true} % Номера только у формул с \eqref{} в тексте.

%%% Гиперссылки
\usepackage{hyperref}
\usepackage[usenames,dvipsnames,svgnames,table,rgb]{xcolor}
\hypersetup{
	unicode=true,            % русские буквы в раздела PDF
	colorlinks=true,       	 % Цветные ссылки вместо ссылок в рамках
	linkcolor=black!15!blue, % Внутренние ссылки
	citecolor=green,         % Ссылки на библиографию
	filecolor=magenta,       % Ссылки на файлы
	urlcolor=NavyBlue,       % Ссылки на URL
}

%%% Графика
\usepackage{tikz}        % Графический пакет tikz
\usepackage{tikz-cd}     % Коммутативные диаграммы
\usepackage{tkz-euclide} % Геометрия
\usepackage{stackengine} % Многострочные тексты в картинках