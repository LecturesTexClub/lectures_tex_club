% 14.09.23 Костя и Аня

\subsection{Нормальные подгруппы}

\begin{reminder}
    Пусть $H \leq G, x \in G$, тогда $xH = \{xh | h \in H\}$ -- левый смежный класс. Свойства левых смежных классов:

\begin{enumerate}
    \item Каждый левый смежный класс порождается любым из своих элементов: $$y \in xH \implies xH = yH$$
    \item Всякие два левых смежных класса либо не пересекаются, либо совпадают: $$xH \cap yH = 0 \text{ или } 
    \exists z \in xH, z \in yH: zH = xH = yH \implies xH = yH$$
    \item Группа $G$ представима в виде дизъюнктного объединения левых смежных классов по $H$: $$G = 
    \bigsqcup_{i \in I} x_iH.$$ 
    Действительно, справедливо представление $G = \bigcup_{x \in G} xH$. Далее из каждого семейства 
    совпадающих левых смежных классов выбросим все, оставив по одному представителю.
\end{enumerate}
\end{reminder}

\begin{note}
    Аналогичные свойства верны и  для правых смежных классов.
\end{note}

\begin{theorem}[Критерий принадлежности двух элементов группы $G$: $x, y$ одному и тому же левому смежному классу]
    Следующие условия эквивалентны тому, что $x$ и $y$ принадлежат одному и тому же левому смежному классу:
    \begin{enumerate}
        \item $x^{-1}y \in H$
        \item $y^{-1}x \in H$
        \item $xH = yH$
        \item $x \in yH$
        \item $y \in xH$
        \item $xH \cap yH \neq \emptyset$
    \end{enumerate}
\end{theorem}

\begin{definition}
    Пусть $G$ конечно, тогда $G/H$ -- множество левых смежных классов, а $H \backslash G$ -- 
    множество правых смежных классов
\end{definition}

\begin{proposition}
    Верно $|G/H| = \frac{|G|}{|H|} = |H \backslash G|$.
\end{proposition}

\begin{proof}
    $(xH)^{-1} = H^{-1}x^{-1} = Hx^{-1}$
\end{proof}

\begin{definition}
    $|G/H| = |H \backslash G| = (G : H)$ -- индекс подгруппы $H$ в $G$
\end{definition}

\begin{exercise}
    Покажите, что если $G$ - конечная, то верна мультипликативность индекса: $$\forall K, H: K 
    \leq H \leq G \hookrightarrow (G : K) = (G : H) \cdot (H : K)$$
\end{exercise}

\begin{note}
    Из истинности $aH = bH $ не следует $ Ha = Hb$
\end{note}

\begin{example}
    Рассмотрим группу $S = S_3$ и подгруппу $H = \langle(1 2)\rangle = \{e, (1 2)\}$. Положим $a = 
    (1 3)$ и $b = (1 2 3)$ и покажем,
    что $aH = bH$, но $Ha \neq Hb$:
    \begin{enumerate}
        \item $(1 3) \cdot (1 2) = (1 2 3), (1 2 3) \in (1 3)H \implies (1 2 3)H = (1 3)H.$ 
        \item $H(1 3) = \{e(1 3), (1 2) \cdot (1 3)\} = \{(1 3), (1 3 2)\} \neq  \{(1 2 3), (2 3)\} = 
        \{e \cdot (1 2 3), (1 2) \cdot (1 2 3)\} = H(1 2 3).$
    \end{enumerate}
\end{example}

\begin{definition}
    Подгруппа $H$ группы $G$ называется нормальной подгруппой, если левостороннее разложение $G$ 
    по $H$ совпадает с правосторонним:
    $$G = \bigsqcup_{i \in I} x_iH = \bigsqcup_{j \in J} Hy_j,$$ То есть разбиение группы на 
    непересекающиеся классы слева и справа одно и то же.\\
    Так же используются названия: "нормальный делитель" и "инвариантная подгруппа".\\
    Обозначение: $H \vartriangleleft G$.
\end{definition}

\begin{theorem}[Критерий нормальности]
    \label{th2.1}
    Пусть $H \leq G$, тогда $H \vartriangleleft G$ тогда и только тогда, когда для всех $x \in G$ верно 
    $xH = Hx$ (что равносильно $x^{-1}Hx = H$ и $xHx^{-1} = H$).
\end{theorem}

\begin{proof}~
    \begin{enumerate}
        \item Необходимость. Пусть $H \vartriangleleft G, x \in G$. Тогда верно
        $\underset{i \in I}{\bigsqcup} x_iH = \underset{j \in J}{\bigsqcup} Hy_j$.

        Рассмотрим $x$, лежащий в некоторых классах $x_iH$ и $Hy_j$. В таком случае, по определению нормальности 
        подгруппы $H$, эти классы совпадают. Тогда верна цепочка равенств:
        $$xH = x_iH = Hy_j = Hx.$$
        \item  Достаточность. Искомые совпадающие разложения можно получить следующим образом:
        $$xH = Hx \implies \bigsqcup_{i \in I} x_iH = \bigsqcup_{i \in I} Hx_i.$$
    \end{enumerate}
\end{proof}
    
\begin{proposition}
    Пусть $H_1 \vartriangleleft G, H_2 \vartriangleleft G$. Тогда $H_1 \cap H_2 \vartriangleleft G$.
\end{proposition}

\begin{proof}
    $\forall x \in G: x^{-1}(H_1 \cap H_2)x = x^{-1}H_1x \cap x^{-1}H_2x = H_1 \cap H_2$
\end{proof}

\begin{example}~
    \begin{enumerate}
        \item Если $G$ -- абелева группа, то из $H \leq G $ следует $ H \vartriangleleft G$. Действительно,
        $xH = Hx$, так как $xh = hx$ для всех $h \in H$.
        \item Пусть $(G : H) = 2$, тогда $H \vartriangleleft G$. Действительно,
        левостороннее разложение получается как: $$G = H \sqcup (G \backslash H).$$ 
        Правостороннее разложение можно получить аналогично, а значит $H \vartriangleleft G$.
        \item 
        $GL_n(F)$ -- группа невырожденных матриц размера $n \times n$ над полем $F$, \\
        $SL_n(F)$ -- группа матриц размера $n \times n$ с единичным определителем.

        Тогда $SL_n(F) \vartriangleleft GL_n(F)$.

        Действительно, пусть $X \in GL_n(F)$, $A \in SL_n(F)$. Тогда верно: $$\det (X^{-1}AX) = \det (X^{-1}) \cdot \det A \cdot \det X = 1.$$
        \item $A_n$ -- группа четных подстановок, то есть таких что  $\epsilon(\sigma) = 1$. 
        Тогда $A_n \vartriangleleft S_n$.

        Действительно, $|A_n| = \frac{n!}{2} = \frac{|S_n|}{2}$, а значит, из соображений мощности,
        в разложении на смежные классы будет всего два элемента: $A_n$ и $S_n \backslash A_n$.
        \item Если $\tau$ -- транспозиция, то $\langle\tau\rangle \ntriangleleft S_3$ 
        
        Действительно, любое произведение транспозиций в $S_3$ не коммутирует: 
        \begin{gather*}
            (a b) (b c) = (a b c) \neq (a c b) = (b c) (a b).
        \end{gather*}
        Тогда $\sigma \langle \tau \rangle = \{\sigma, \sigma \cdot \tau\} \neq \{\tau, \tau \cdot \sigma\} = \langle\tau\rangle \sigma$.
    \end{enumerate}
\end{example}

\begin{theorem}[О произведении нормальной подгруппы на подгруппу]~
    \label{th2.3}
    Пусть $G$ -- группа, $H \vartriangleleft G, K \leq G$, тогда:
    \begin{enumerate}
        \item $H \cdot K \leq G$,
        \item если $K \vartriangleleft G$, то $H \cdot K \vartriangleleft G$.
    \end{enumerate}
\end{theorem}

\begin{proof}~
    \begin{enumerate}
        \item Покажем, что $HK \cdot HK = HK$: 
        $$HK \cdot HK = H(K \cdot H)K = H(H \cdot K)K = (H \cdot H)(K \cdot K) = H \cdot K.$$
        
        Тогда: 
        $$K \cdot H = \bigcup_{k \in K} kH = \bigcup_{k \in K}Hk = H \cdot K.$$
        \item $x^{-1}HKx = x^{-1}Hxx^{-1}Kx = H \cdot K$, значит $HK \vartriangleleft G$ (по теореме 2)
    \end{enumerate}
\end{proof}

\section{Гомоморфизмы}
\subsection{Сопряженность в группе}

\begin{definition}
    Пусть $G$ -- группа, $a, x \in G$. Тогда элемент $x^{-1}ax$ называется сопряженным к элементу 
    $a$ с помощью элемента $x$. Обозначение: $a^{x}$.
\end{definition}

\begin{proposition}
    Свойства операции $({\uparrow}^x)$
    \begin{enumerate}
        \item $a^{xy} = (a^x)^y$
        \item $a^x \cdot b^x = (ab)^x$
        \item Операции $\uparrow^x$ и $\uparrow^{-1}$ коммутируют.
    \end{enumerate}    
\end{proposition}

\begin{proof}~
    \begin{enumerate}
        \item $a^{xy} = (xy)^{-1}a(xy) = y^{-1}(x^{-1}ax)y = y^{-1}(a^x)y = (a^x)^y$
        \item $(ab)^x = x^{-1}abx = x^{-1}ax x^{-1}bx = a^x b^x$
        \item $(a^{-1})^x \cdot a^x = (a^{-1} \cdot a)^x = e^x = e$.
    \end{enumerate}
\end{proof}

\begin{note}
    Отображение $\phi: G \rightarrow G$, переводящее  $a$ в $a^x$ при фиксированном $x$ -- автоморфизм
    (изоморфизм $G \to G$).
\end{note}

\begin{proof}~
    \begin{enumerate}
        \item Из второго свойства операции сопряжения следует, что $\phi$ -- гомоморфизм. 
        \item Покажем, что $\phi$ инъективен. Действительно, если $a^x = b^x$, то $x^{-1}ax = x^{-1}bx$, 
        откуда $a = b$.
        \item Покажем, что $\phi$ сюрьективен. Зафиксируем элемент $a \in G$. Тогда $x a x^{-1} \in G$, 
        и верно: $$\phi(x a x^{-1}) = x^{-1} (x a x^{-1}) x = a.$$ Таким образом, для любого элемента есть прообраз, 
        а значит, $\phi$ сюрьективен.
    \end{enumerate}
\end{proof}

\begin{remarkfrom}
    Определение автоморфизма и других морфизмов будет дано в следующем разделе. Доказательство 
    факта выше добавлено автором и на лекции представлено не было.
\end{remarkfrom}

\begin{note}
    Сопряжение $\sim$ является отношением эквивалентности.
\end{note}

\begin{proof}~
    \begin{enumerate}
        \item Рефлексивность $a \sim a$.
        
        Любой элемент $a$ сопряжен сам с собой при помощи нейтрального элемента $e$: 
        $$e^{-1} a e = a.$$
        \item Симметричность $a \sim b \iff b \sim a$.
        
        Пусть $a = b^x$. Тогда: 
        $$b = x x^{-1} \cdot b \cdot x x^{-1} = x \cdot x^{-1} b x \cdot x^{-1} = a^{x^{-1}}.$$ 
        Отсюда $b$ сопряжен с $a$ при помощи $x^{-1}$.
        \item Транзитивность $a \sim b,\, b \sim c \Rightarrow a \sim c$.
        
        Пусть $b = a^x$ и $c = b^y$. Тогда $c = b^y = (a^x)^y = a^{xy}$, откуда $a$ сопряжен с $c$
        при помощи элемента $xy$.
    \end{enumerate}
\end{proof}

\begin{corollary}
    Группа $G$ распадается на непересекающиеся классы сопряженных элементов (классы сопряженности)
\end{corollary}


\begin{definition}
    Обозначим $a^G = \{a^x \,|\, x \in G\}$ -- класс, содержащий $a$
\end{definition}

\begin{proposition}
    \label{pr2.1}
    $H \leq G$. $H$ является нормальной подгруппой в $G$ тогда и только тогда, когда $H$ является дизъюнктным объединением каких-то из классов сопряженных элементов (или $H$ вместе с каждым своим элементом $a$ содержит целиком класс сопряженности $a^G$)
\end{proposition}

\begin{proof}~
    \begin{enumerate}
        \item Необходимость. $H \lhd G$ \\
    $a_1 \in H$, по \ref{th2.1} $x^{-1}Hx = H \implies a_{1}^G \subseteq H$ \\
    Далее по индукции рассматриваем для каждого из элементов подгруппы $H$: $x^{-1}Hx = H, a_2 \in H \backslash (a_{1}^G) \implies a_{2}^G \subseteq H \dots$, поэтому $H$ разложится в прямую сумму $H = a_{1}^G \sqcup a_{2}^G \sqcup \ldots$
        \item Достаточность. Пусть $H = a_{1}^G \sqcup a_{2}^G \sqcup \ldots \sqcup a_{k}^G \sqcup \ldots$ \\
    $H^x = x^{-1}Hx = ((a_{1}^G)^x \sqcup (a_{2}^G)^x \sqcup \ldots \sqcup (a_{k}^G)^x \sqcup \ldots) = a_{1}^G \sqcup a_{2}^G \sqcup \ldots \sqcup a_{k}^G \sqcup \ldots = H$, тогда по \ref{th2.1} $H \lhd G$
    \end{enumerate}
\end{proof}

\begin{example}[Описание классов сопряженностей в $S_n$]~
    Пусть $\sigma \in S_n$, $$\sigma = (a_1, a_2, \ldots, a_k)(b_1, b_2, \ldots, b_l)(c_1 \ldots)$$ 
    Тогда $$\tau^{-1} \sigma \tau = \tau^{-1} (a_1 \ldots a_k) \tau \tau^{-1} (b_1 \ldots b_l) \tau \tau^{-1}(c_1 \ldots) \ldots$$
\end{example}

\begin{proposition}
    $\tau^{-1} (a_1 \ldots a_k) \tau = (\tau^{-1}(a_1) \tau^{-1}(a_2) \ldots)$
\end{proposition}

\begin{proof}
    $\tau^{-1} (a_1 \ldots a_k) \tau (\tau^{-1}(a_i)) = \tau^{-1} (a_1 \ldots a_k)(a_i) = \tau^{-1}(a_{i+1})$
\end{proof}

\begin{corollary}
    При действии сопряжения на цикл длины $k$ получается цикл такой же длины. $\tau^{-1} \sigma \tau = (a'_1 a'_2 \ldots a'_k)(b'_1 b'_2 \ldots b'_l)(c'_1 \ldots) = \sigma'$ \\
    При сопряжении циклический тип подставноки $\sigma$ сохраняется. Чтобы $\sigma'$ перешла в $\sigma$ при сопряжении необходимо взять такое $\tau$, что $\tau(a'_i) = a_i, \tau(b'_j) = b_j, \ldots$
    
    Вывод: множество всех классов сопряженности в группе $S_n$ находятся в биективном соответствии с множеством всех циклических типов для $\sigma \in S_n$
\end{corollary}

\begin{example}~
    $P(n)$ -- число циклических типов подстановки из $S_n$ (функция разбиения). $P(2) = 2, P(3) = 3, P(4) = 5, \ldots$ \\
    Так как $S_2 = \{e, | (1 2)\}$ -- абелева, то $a^x = a$ \\
    $S_3 = \{e, | (1 2), (1 3), (2 3), | (1 2 3), (1 3 2)\}$ \\
    $S_4 = \{e\} \sqcup \{(1 2), \ldots\} \sqcup \{(1 2)(3 4) \ldots\} \sqcup \{(1 2 3) \ldots\} \sqcup \{(1 2 3 4)\}$
\end{example}

\begin{definition}
    $P(n)$ -- число способов представить $n$ в виде суммы натуральных чисел (порядок не важен). Оказывается, что для нее нет регулируемого соотношения(явной формулы).
\end{definition}

\begin{example}
    Если $n = 2$, то $2 = 1 + 1$, $2 = 2$ \\
    При $n = 3: n = 1 + 1 + 1 = 1 + 2 + 3$ \\
    И, наконец, для $n = 4: n = 1 + 1 + 1 + 1 = 1 + 1 + 2 = 2 + 2 = 1 + 3 = 4$
\end{example}

\subsection{Гомоморфизмы групп}

\begin{definition}
    \label{def2.4}
    Пусть есть группы $(G_1, \cdot), (G_2, \circ)$. \\ 
    Отображение $\phi: G_1 \rightarrow G_2$ называется гомоморфизмом, если $\forall a, b \in G_1: \phi(a \cdot b) = \phi(a) \circ \phi(b)$
\end{definition}

\begin{proposition}(Свойства гомоморфизмов)
    \begin{enumerate}
        \item $\phi(e_1) = e_2$
        \item $\phi(a^{-1}) = \phi(a)^{-1}$
    \end{enumerate}
\end{proposition}

\begin{proof}~
    \begin{enumerate}
        \item $\phi(e_1 \cdot e_1) = \phi(e_1) \cdot {\phi}^{-1}(e_1) = \phi(e_1) \cdot \phi(e_1) \cdot {\phi}^{-1}(e_1) 
        \implies \phi(e_1) = e_2$
    \end{enumerate}
\end{proof}

