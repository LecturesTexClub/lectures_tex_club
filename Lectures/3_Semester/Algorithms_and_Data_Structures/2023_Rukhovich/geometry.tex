\section{Геометрия}
Скалярное произведение определяем через косинус, косинус через единичную окружность.
Свойства: симметрия, однородность, аддитивность (заметим, что скалярное произведение --- это длина проекции).

\textbf{Утверждение.} Если ко свойствам выше добавить, что $((1, 0), (1, 0)) = ((0, 1), (0, 1)) = 1$ и $((1, 0), (0, 1)) = 0$, то из этих свойств будет следовать определение из алгема, и функция будет определена однозначно.
Для доказательства просто разложим два вектора по базису.

Дополнительные свойства: скалярное произведение равно нулю только для перпендикулярных векторов, по знаку можно определить остроту угла.

Векторное произведение вводится через синус, свойства обычные.
Линейность из соображения, что это проекция на перпендикуляр к вектору.

\textbf{Утверждение.} Если зафиксировать значения на базисных векторах, то векторное произведение тоже определяется однозначно.

Свойства: равно нулю только для параллельных, позволяет понимать, по часовой или против часовой стрелки векторы идут.

\subsection{Многоугольники}
\subsubsection{Площадь}
\textbf{Определение.} Ориентированная площадь --- площадь, если многоугольник против часовой стрелки, и отрицательная площадь иначе.
Выпуклый многоугольник можно разбить на треугольники и найти сумму их площадей.
Положим $f(A) = \frac{1}{2} \sum_{i=0}^{n-1} [p_i - A, p_{i+1} - A]$ --- формула площади для выпуклых, где $A$ --- вершина многоугольника.

\textbf{Утверждение.} Для всех многоугольников $f(A)$ --- константа.
Раскроем по линейности и всё сократим.

Тогда для выпуклых многоугольников можно считать площадь через $f(A)$ для любой точки.
Но это работает и для произвольных многоугольников.
Аналогично проверке принадлежности проведём диагональ и найдём ориентированную площадь двух полученных многоугольников.
Но можно и по-другому.
Возьмём точку вне многоугольника и посмотрим на ориентированные площади, которые мы суммируем.
Тогда область снаружи сократится, а внутри --- останется.
Чтобы было чуть более строго, возьмём точку вне многоугольника и рассмотрим углы, которые смотрят на отрезки, такие что вершин нет внутри углов.
Тогда у отрезков, пересекающих этот угол, чередуются направления, так что вновь учтётся только внутренность.

Альтернативно, метод трапеций тоже существует.

\subsubsection{Проверка принадлежности точки}
Проверим, лежит ли точка $(x, y)$ в произвольном многоугольнике.
Проведём произвольный луч из точки и посчитаем число пересечений.
Будет больно, если он пересечёт вершины многоугольника, это можно избежать тремя способами: провести новый случайный луч, взять направление, у которого угол с осью X меньше углов для всех векторов в вершины многоугольника, провести луч в $(+\infty, y + 1)$.

А можно просто решить эту проблему. Будем учитывать только те отрезки $((x_1, y_1), (x_2, y_2))$, для которых верно $\min(y_1, y_2) \le y < \max(y_1, y_2)$.
Доказывается разбором случаев, или просто по факту мы поднимаем луч вверх на $\varepsilon$.

\subsubsection{Площадь пересечения многоугольника и круга}
Площадь многоугольника мы можем посчитать через треугольники.
Теперь пересечём каждый из треугольников с кругом и получим искомую площадь.

\subsection{Выпуклые многоугольники}
\textbf{Определение.} Множество точек $A$ называется \textit{выпуклым}, если все отрезки $(p_1, p_2)$ с $p_1, p_2, \in A$ лежат в $A$.

\textbf{Свойство 1.} Выпуклые множества замкнуты относительно любого пересечения.

\textbf{Теорема 1.} Это определение эквивалентно тому, что все рёбра при ориентации против часовой стрелки поворачивают налево.

\textbf{Доказательство.} $\Rightarrow$: от противного, тогда легко найти отрезок.

$\Leftarrow$: от противного, пусть какой-то отрезок не лежит полностью.
Переходя к подотрезку, можно считать, что вершины лежат на рёбрах, а внутренность --- вне многоугольника.
Будем двигать отрезок вверх (считая, что он горизонтален).
Тогда мы либо придём в вершину, и тогда нашли поворот направо, либо многоугольник закончится с одной из сторон, тогда пойдём вниз.
Если и внизу закончился, то многоугольник несвязен.

\textbf{Теорема 2.} Сумма углов многоугольника равна $\pi(n - 2)$.

\textbf{Доказательство.} Пройдём многоугольник против часовой стрелки.
Тогда в каждом повороте мы получим внутренний угол $\alpha$ и внешний угол $\beta$.
Известно, что $\alpha + \beta = \pi$, поэтому запишем сумму и заметим, что сумма $\beta$ равна $2\pi$.

\subsubsection{Проверка принадлежности точки}
По классике триангуляцией и бинарным поиском.

\subsubsection{Касательная}
\textbf{Определение.} Касательная ко множеству --- прямая, которая проходит хотя бы через одну точку множества и относительно которой множество лежит по одну сторону.
Можно ещё о ней думать, как о прямой, которая крутится вокруг многоугольника, касаясь стороны или вершины.

Присвоим рёбрам \textit{тип}. Разберём случай, когда первая вершина находится по другую сторону от точки, из которой строим касательную.
Рёбра от начала до первой точки касания --- первого типа, от первой до второй --- второго типа, от второй до конца --- третьего типа.
Теперь тип ребра можно понять векторным произведением, так что применим бинарный поиск.

\subsubsection{Общая касательная к двум многоугольникам}
Вращающимся сканлайном за $O(n \log(n))$, но это какая-то жесть.
Можно ещё параллельно крутить две касательные и отлавливать моменты, когда они совпадают (обгоняют друг друга), получится $O(n)$.

\subsection{Окружности}
\subsubsection{Пересечение окружностей}
Можно найти радиальную ось, можно сделать инверсию, но можно по-другому.
Сначала проверим, что точки пересечения вообще есть.
Пусть $C_0$, $C_1$ --- центры и $r_0$, $r_1$ --- радиусы окружностей.
Тогда нужно найти треугольник со сторонами $r_0$, $r_1$ и $|C_1 - C_0|$ с вершинами в центрах окружностей и точке пересечения.
Теперь возьмём угол между отрезком, соединяющим центры окружностей, и радиусом и найдём его по теореме косинусов.

\subsubsection{Касательная к окружности}
Найдём расстояние от точки до точки касания теоремой Пифагора или через степень точки.
Угол между касательной и отрезком, соединяющим точку и центр окружности, найти несложно, поэтому остаётся повернуть прямую.

\subsubsection{Общие касательные к двум окружностям}
Их четыре штуки: две внешние и две внутренние.
Внешние находятся через построение прямоугольной трапеции и нахождение угла между касательной и отрезком, соединяющим центры.
Внутренние --- увеличим радиус одной из окружностей до $r_1 + r_2$.

\subsection{Сканирующая прямая}
\subsubsection{Проверка принадлежности нескольких точек произвольному многоугольнику}
Классика,

\subsubsection{Объединение прямоугольников}
Классика,

\subsubsection{Проверка, пересекаются ли два отрезка}
Будем при добавлении отрезка во множество проверять, пересекается ли он с соседями.
Аналогично при удалении --- с бывшими соседями.

\subsection{Выпуклая оболочка}
\textbf{Определение.} Выпуклая оболочка множества --- пересечение всех выпуклых множеств, содержащих данное.

\subsubsection{Алгоритм Джарвиса (метод заворачивания подарка)}
Возьмём самую левую точку $p_0$ (она точно лежит в выпуклой оболочке).
Теперь наивно найдём точку $p_1$, такую что все точки из множества лежат слева от направленной прямой $p_0 p_1$.
Продолжаем, пока не надоест.
Работает за $O(nh)$, где $h$ --- размер выпуклой оболочки.

\subsubsection{Алгоритм Грэхема}
Классика (отсортируем по полярному углу и добавим).
Доказывается индукцией по количеству шагов в алгоритме.

\subsubsection{Алгоритм Грэхема-Эндрю}
Отсортируем по первой координате и соберём нижнюю и верхнюю огибающую.

\subsubsection{Алгоритм Киркпатрика}
Можно построить выпуклую оболочку через разделяй-и-властвуй: разделим на две части, построим и объединим через вращающуюся сканирующую прямую.

\subsubsection{Алгоритм Чена}
Пусть мы уже знаем, чему равно $h$, размер ответа.
Разделим точки на множества размера $h$ и найдём в них выпуклые оболочки алгоритмом Джарвиса за $O(n \log(h))$ суммарно.
Теперь через заворачивание подарка найдём объединение этих выпуклых оболочек.
А именно, пусть $p_0$ --- самая левая точка.
Найдём касательные ко всем $\frac{n}{h}$ множествам за $O(\log(h))$ к каждому, то есть за $O \left( \frac{n}{h} \log(h) \right)$ суммарно.
Возьмём $p_1$ среди точек касания, как в алгоритме Джарвиса.
Повторим $h$ раз --- по итогу $O(n \log(h))$.

Но откуда достать $h$?
На самом деле, если мы взяли слишком маленький $h$, то мы просто не достроим оболочку до конца, и тогда можно будет перезапуститься с бóльшим $h$.
Для хорошей асимптотики берём $h = 1, 2, 2^2, 2^{2^2}, 2^{2^3}, 2^{2^4}, \dots$.

То есть для случайного набора точек это работает за $O(n \log(\log(n)))$.

\subsection{Convex Hull Trick}
\subsubsection{На скалярных произведениях}
Пусть мы хотим посчитать динамику вида
\[
    dp_i = \min_{1 \le j < i} (a_i \cdot dp_j + b_i \cdot c_j + d_i).
\]
Заметим, что по модулю константы $d_i$ это является скалярным произведением векторов $((a_i, b_i), (dp_j, c_j))$.
Мы хотим иметь чёрный ящик, который поддерживает две операции: добавить вектор $(x, y)$ и по данному вектору $(x, y)$ найти вектор в ящике, дающий минимальное скалярное произведение с ним.

Пусть мы делаем запрос для вектора $(a, b)$.
Вспомним, что скалярное произведение --- это длина вектора $(a, b)$, умноженная на ориентированную длину проекции вектора $(x, y)$ в чёрном ящике на $(a, b)$.
Таким образом, чтобы найти минимум, нужно найти вектор, ориентированная длина проекции которого на $(a, b)$ минимальна.
Теперь нарисуем картинку и увидим, что все такие векторы лежат на выпуклой оболочке точек $(x, y)$ в чёрном ящике: пройдём вдоль прямой, содержащей $(a, b)$, упрёмся в выпуклую оболочку и т.д.

Как же поддерживать выпуклую оболочку и делать к ней запросы?
Будем решать задачу для $(a, b)$ при $b > 0$.
Тогда можно поддерживать только нижнюю огибающую, для векторов с $b < 0$ аналогично.
Нетрудно заметить, что подходящую точку можно найти просто бинарным поиском, смотря на знак скалярного произведения.

С запросами разобрались, теперь про добавление точек.
Если точки добавляются в порядке возрастания первой координаты, то можно просто строить выпуклую оболочку алгоритмом Эндрю.
Теперь общий случай, пусть мы добавляем точку $(x_0, y_0)$.
Найдём ближайшую слева и справа точки $A$ и $B$.
Если $(x_0, y_0)$ находится выше $AB$, то добавлять не нужно.
В противном случае придётся её вставить и поудалять соседние точки, если они нарушают выпуклость.

Но для этого нам нужна структура, позволяющая удалять из середины --- возьмём какое-нибудь дерево поиска и вместо бинарного поиска будем делать спуск, поддерживая следующую и предыдущую вершину в каждой.

\subsubsection{На прямых}
Вернёмся к классике: в чёрный ящик мы добавляем прямые $y = kx + b$ и хотим найти минимум среди прямых для данного $x$.

Если прямые добавляются в порядке убывания $k$, то вновь алгоритм Эндрю.
Если это не так, то будем вновь строить касательные по направлению прямых и удалять все точки над новой прямой.

\subsubsection{Дерево Ли-Чао}
Пусть нам заранее известны координаты $x$, в которых мы вычисляем значение прямых (в целом, можно и неявно, но тогда будет $\log(C)$ вместо $\log(n)$).
Тогда можно хранить для каждой точки прямые, в которых минимум потенциально достигается.
Этот инвариант мы просто будем поддерживать в ходе работы алгоритма, то есть нужно лишь доказать, что он сохраняется при добавлении новой прямой.

\subsection{Тернарный поиск}
\textbf{Определение.} Функция $f: [a, b] \to \mathbb R$ называется \textit{унимодальной}, если существует $x_0 \in [a, b]$, такой что $f$ строго убывает на $[a, x_0]$ и строго возрастает на $[x_0, b]$.

Тогда для поиска этой самой точки можно использовать тернарный поиск.
Чтобы найти минимум с точностью до $\varepsilon$, нам понадобится $O \left( \log_{1.5} \left( \frac{b - a}{\varepsilon} \right) \right)$ операций.
Для оптимизации можно использовать более оптимальные функции выбора новых границ, например: $x_l = \frac{3a + 4b}{7}$, $x_r = \frac{4a + 3b}{7}$.

Ещё можно делить по золотому сечению, чтобы делать $\log_\phi \left( \frac{b - a}{\varepsilon} \right)$ итераций.

\subsubsection{Выпуклые функции}
Обычно для доказательства того, что функция унимодальна и можно использовать тернарный поиск, доказывают её выпуклость.

\textbf{Определение.} Функция выпукла вниз, если её надграфик является выпуклым множеством точек.

\textbf{Утверждение.} Выпуклая вниз на $(a, b)$ немонотонная функция $f$ унимодальная.

\textbf{Доказательство.} В силу немонотонности существуют точки $a' < c' < b'$, такие что $f(c') < f(a')$ и $f(c') < f(b')$.
Тогда в силу непрерывности $f$ достигает минимум на отрезке $[a', b']$, и теперь в силу выпуклости вниз это и будет глобальным минимумом унимодальной функции (от противного).

\textbf{Утверждение.} Пусть $f$ и $g$ выпуклы вниз на $(a, b)$.
Тогда $\max(f, g)$, $f + g$ выпуклы вниз на $(a, b)$.

\textbf{Пример.} Задача про гонки на колесницах.
Есть $n \le 10^5$ спортсменов, которые стартуют в координатах $x_i$ и движутся со скоростью $v_i$.
Нужно найти момент времени $t$, когда они все находятся в минимально возможном отрезке.
Запишем формулу длины искомого отрезка: $\max(x_i + v_i \cdot t) - \min(x_i + v_i \cdot t)$.
Заметим, что это разность выпуклой вниз функции и выпуклой вверх --- результат выпуклый вниз.

\textbf{Пример.} Дан выпуклый многоугольник, найти максимальный радиус окружности, которую можно вписать в него.
Будем искать точку с максимальным минимальным расстоянием до отрезков многоугольника, а можно не до отрезков, а до прямых.
Заметим, что можно теперь делать вложенный тернарный поиск.
Возникает проблема с тем, что мы можем вылезти за многоугольник, но для её решения можно ориентировать прямые многоугольника и искать ориентированное расстояние до них (в левой полуплоскости положительное, в правой --- отрицательное).

\subsubsection{Пересечение полуплоскостей}
Научимся пересекать полуплоскости и находить полученное множество.
Сразу ограничимся каким-то большим прямоугольником, чтобы не работать с бесконечными координатами.
Тогда можно добавлять полуплоскости по-одному и пересекать их с результатом, который поначалу является этим большим прямоугольником.
По факту нужно только пересекать прямую с многоугольником и брать нужную полуплоскость.

Но это как-то сложно.
Разделим полуплоскости на направленные вверх и направленные вниз, их пересечём, получим верхнюю и нижнюю огибающие, которые можно пересечь сканирующей прямой.

\textbf{Пример.} Найти точку в выпуклом многоугольнике, такую что при удалении $k$ точек она останется в выпуклой оболочке, а при $k + 1$ уже вылетит, причём $k$ максимально.
Сделаем бинарный поиск и пересечём полуплоскости, полученные удалением всех $k$ последовательных вершин.

\textbf{Определение.} Сумма Минковского множеств $A$, $B$ --- это $A + B = \{a + b: a \in A, b \in B\}$.
