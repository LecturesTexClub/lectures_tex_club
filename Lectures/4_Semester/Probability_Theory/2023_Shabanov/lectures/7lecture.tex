\subsection{Формулы вычисления математического ожидания}

\begin{theorem}
	Пусть случайный вектор $\xi \colon \Omega \to \R^m$ имеет распределение $P_\xi$, причём оно само имеет (обобщённую) плотность $p(t)$ по $\sigma$-конечной мере $\mu$ на $(\R^m, \B(\R^m))$. Тогда для любой борелевской функции $g \colon \R^m \to \R$ выполнена формула:
	\[
	\E g(\xi) = \int_{\R^m} g(x)P_\xi(dx) = \int_{\R^m} g(x)p(x)\mu(dx)
	\]
\end{theorem}

\begin{proof}
	Снова произведём доказательство по пути определения интеграла Лебега:
	\begin{enumerate}
		\item $g(x) =\chi_A(x)$, где $A \in \B(\R^m)$. Тогда
		\[
			\E g(\xi) = P(\xi \in A) = P_\xi(A) = \int_A p(x)\mu(dx) = \int_{\R^m} \chi_A(x)p(x)\mu(dx) = \int_{\R^m} g(x)p(x)\mu(dx)
		\]
		
		\item $g$ --- простая функция. Тогда заметим, что обе части доказываемого равенства линейны по $g$, то есть всё автоматически доказано по первому пункту.
		
		\item Если $g \ge 0$, то рассмотрим возрастающую последовательность неотрицательных простых функций $\{g_n\}_{n = 1}^\infty$, подходящих к $g$. По определению математического ожидания:
		\[
			\E g(\xi) = \lim_{n \to \infty} \E g_n(\xi) = \lim_{n \to \infty} \int_{\R^m} g_n(x)p(x)\mu(dx) = \int_{\R^m} g(x)p(x)\mu(dx)
		\]
		Последний переход возможен по теореме Леви.
		
		\item $g$ --- произвольная функция. Тогда пользуемся разложением $g = g^+ - g^-$, линейностью интеграла Лебега и определениями.
	\end{enumerate}
\end{proof}

\begin{corollary}
	Пусть $\xi$ --- дискретная случайная величина, сосредоточенная на множестве $X$. Тогда
	\[
		\E g(\xi) = \sum_{x \in X} g(x)P(\xi = x)
	\]
\end{corollary}

\begin{proof}
	Воспользуемся $\mu$ --- считающей мерой на $X$. Тогда плотность распределения $P_\xi$ имеет следующий вид:
	\[
		p(x) = P_\xi(\{x\}) = P(\xi = x)
	\]
	Остаётся лишь воспользоваться доказанными результатами:
	\[
		\E g(\xi) = \int_\R g(x)p(x)d\mu(x) = \sum_{x \in X} g(x)P(\xi = x)
	\]
\end{proof}

\begin{corollary}
	Пусть $\xi$ --- абсолютно непрерывная случайная величина с плотностью $p(x)$ (то есть эта плотность принадлежит $P_\xi$). Тогда верна формула:
	\[
		\E g(\xi) = \int_\R g(x)p(x)d\mu(x)
	\]
	где $\mu$ --- стандартная мера Лебега на $\R$.
\end{corollary}

\begin{corollary}
	Пусть $\xi \colon \Omega \to \R^m$ --- случайный вектор с плотностью $p(\vv{x})$. Тогда верна формула:
	\[
		\E g(\xi) = \int_{\R^m} g(\vv{x})p(\vv{x})d\vv{x} = \int_{\R^m} g(\vv{x})p(\vv{x})d\mu(\vv{x})
	\]
	где $d\vv{x} := d\mu(\vv{x})$ обозначает обычную меру Лебега в $\R^m$.
\end{corollary}

\begin{note}
	В частности, для $\forall B \in \B(\R^m)$ можно рассмотреть $\chi_B \colon \R^m \to \R$. Тогда есть формула:
	\[
		P(\xi \in B) = \int_B p(\vv{x})d\vv{x}
	\]
\end{note}

\begin{note}
	Если мы находимся пространстве $(\R, \B(\R), P)$ и $F$ --- это функция распределения меры $P$, то $dP(x) = dF(x)$. Если помимо всего у $P$ есть плотность $p(x)$, то $dF(x) = p(x)d\mu(x)$, где $\mu$ --- стандартная мера Лебега на $\R$. Далее до конца главы мы живём в таком вероятностном пространстве.
\end{note}

\begin{note}
	Если $F$ --- функция распределения меры $P$, то интеграл по борелевской функции $g$ можно посчитать так, если вспомнить про теорему о разложении:
	\[
		\int_\R g(x)P(dx) = \int_\R g(x)dF(x) = \sum_{i = 1}^3 \alpha_i \int_\R g(x)dF_i(x)
	\]
\end{note}

\begin{example}
	Пусть случайная величина $\xi$ имеет функцию распределения следующего вида (экспоненциальное распределение):
	\[
		F(x) = \System{
			&{0,\ x < 0}
			\\
			&{1 - e^{-x},\ x \in \lsi{0; 1}}
			\\
			&{1,\ x \ge 1}
		}
	\]
	Найдём $\E\xi$:
	\[
		\E\xi = \int_0^1 xP_\xi(dx) + 1 \cdot P(\xi = 1) = \int_0^1 xF'(x)dx + (1 - (1 - e^{-1})) = (-xe^{-x} - e^{-x})|_0^1 + \frac{1}{e} = 1 - \frac{1}{e}
	\]
\end{example}

\section{Прямое произведение вероятностных пространств}

\begin{definition}
	Пусть $(\Omega_1, \F_1, P_1)$ и $(\Omega_2, \F_2, P_2)$ --- вероятностные пространства. Их \textit{прямым произведением} называется вероятностное пространство $(\Omega, \F, P)$, где объекты заданы следующим образом:
	\begin{enumerate}
		\item $\Omega = \Omega_1 \times \Omega_2$
		
		\item $\F = \F_1 \otimes \F_2 = \sigma(B_1 \times B_2 \colon B_i \in \F_i)$ --- $\sigma$-алгебра, порождённая прямоугольниками
		
		\item $P = P_1 \times P_2$ --- вероятностная мера на $(\Omega, \F)$ такая, что
		\[
			P(B_1 \times B_2) = P_1(B_1) \cdot P_2(B_2)
		\]
	\end{enumerate}
\end{definition}

\begin{lemma}
	Мера $P$ из определения прямого произведения существует, действительно является вероятностной мерой и единственна.
\end{lemma}

\begin{proof}
	Рассмотрим $\cA$ --- система множеств, состоящая из конечного объединения непересекающихся прямоугольников $B_1 \times B_2$. Тогда $\cA$ тривиально является алгеброй, причём $\sigma(\cA) = \F$.
	
	Определим $P$ на $\cA$ по конечной аддитивности. Для того, чтобы продлить построенную меру на $\sigma(\cA) = \F$, остаётся проверить счётную аддитивность этой меры на $\cA$. Пусть $C = \bscup_{i = 1}^\infty C_i$, где $C_i, C \in \cA$. Нужно показать такое равенство:
	\[
		P(C) = \sum_{i = 1}^\infty P(C_i)
	\]
	Так как любой элемент $\cA$ есть объединение конечного числа непересекающихся прямоугольников, то достаточно проверить лишь этот случай (то есть когда все вышеупомянутые множества являются просто прямоугольниками). Итак, $C = A \times B$, $C_i = A_i \times B_i$. Чтобы добится результата, мы будем интегрировать индикаторы:
	\[
		\chi_{A \times B}(\omega_1, \omega_2) = \sum_{i = 1}^\infty \chi_{A_i \times B_i}(\omega_1, \omega_2) \Lra \chi_A(\omega_1)\chi_B(\omega_2) = \sum_{i = 1}^\infty \chi_{A_i}(\omega_1)\chi_{B_i}(\omega_2)
	\]
	Зафиксируем $\omega_1 \in \Omega_1$ и возьмём математическое ожидание от обеих частей равенства в пространстве $(\Omega_2, \F_2, P_2)$ (ибо при фиксации мы как раз получаем случайные величины из этого пространства):
	\[
		\chi_A(\omega_1) \cdot P_2(B) = \sum_{i = 1}^\infty \chi_{A_i}(\omega_1) \cdot P_2(B_i)
	\]
	Аналогично проделываем в пространстве $(\Omega_1, \F, P)$, чем и завершаем доказательство.
\end{proof}

\begin{theorem} (Фубини, без доказательства)
	Пусть $(\Omega, \F, P) = (\Omega_1, \F_1, P_1) \otimes (\Omega_2, \F_2, P_2)$ и есть случайная величина $\xi \colon \Omega \to \R$ такая, что $\E\xi = \int_\Omega \xi dP$ конечен. Тогда $\forall i \in \{1, 2\}$ интеграл $\int_{\Omega_i} \xi(\omega_1, \omega_2) P_i(d\omega_i)$ конечен почти всюду по мере $P_{3 - i}$, является $\F_{3 - i}$-измеримой функцией и, кроме того, имеют место равенства:
	\begin{multline*}
		\int_\Omega \xi(\omega_1, \omega_2) P(d\omega_1, d\omega_2) =
		\\
		\int_{\Omega_1} \ps{\int_{\Omega_2} \xi(\omega_1, \omega_2)P_2(d\omega_2)} P_1(d\omega_1) =
		\\
		\int_{\Omega_2} \ps{\int_{\Omega_1} \xi(\omega_1, \omega_2)P_1(d\omega_1)} P_2(d\omega_2)
	\end{multline*}
\end{theorem}

\begin{proposition}
	Если $\xi, \eta$ --- независимые случайные величины, то $P_{(\xi, \eta)} = P_\xi \times P_\eta$.
\end{proposition}

\begin{proof}
	\[
		P_{(\xi, \eta)}(B_1 \times B_2) = P((\xi, \eta) \in B_1 \times B_2) = P(\xi \in B_1 \wedge \eta \in B_2) = P_\xi(B_1) \cdot P_\eta(B_2)
	\]
\end{proof}

\begin{note}
	То же самое верно и для случайных векторов.
\end{note}

\begin{lemma} (о свёртке)
	Пусть $\xi, \eta$ --- это независимые случайные величины с соответствующими функциями распределения $F_\xi, F_\eta$. Тогда, случайная величина $\xi + \eta$ имеет следующую функцию распределения $F_{\xi + \eta}$:
	\[
		F_{\xi + \eta}(z) = \int_\R F_\xi(z - x)dF_\eta(x) = \int_\R F_\eta(z - x)dF_\xi(x)
	\]
\end{lemma}

\begin{proof}
	Рассмотрим случайную величину $(\xi, \eta) \colon \Omega \to \R^2$ и плоскость принимаемых значений (по горизонтали $\xi$, по вертикали $\eta$). Тогда точка $(x, y)$ соответствует событиям $\{t \in \R \colon \xi(t) = x \wedge \eta(t) = y\}$. Условие $\xi + \eta \le z$ можно интерпретировать как множество всех точек плоскости под прямой $y = z - x$. Обозначим это множество за $A \subset \R^2$, тогда
	\[
		F_{\xi + \eta}(z) = P(\xi + \eta \le z) = P((\xi, \eta) \in A) = P_{(\xi, \eta)}(A)
	\]
	Осталось воспользоваться теоремой Фубини для прямого произведения пространств \\ $(\R^2, \B(\R^2), P_{(\xi, \eta)}) = (\R, \B(\R), P_\xi) \otimes (\R, \B(\R), P_\eta)$:
	\begin{multline*}
		F_{\xi + \eta}(z) = P_{(\xi, \eta)}(A) = \int_A dP_{(\xi, \eta)}(x, y) = \int_{\R^2} \chi_A dP_{(\xi, \eta)}(x, y) =
		\\
		\int_\R \underbrace{\ps{\int_\R \chi\{x + y \le z\} P_\xi(dx)}}_{P(\xi \le z - y) = F_\xi(z - y)} P_\eta(dy)
	\end{multline*}
\end{proof}

\begin{corollary} (Формула свёртки)
	Пусть $\xi, \eta$ --- независимые случайные величины с плотностями $p_\xi, p_\eta$ соответственно. Тогда $\xi + \eta$ тоже имеет плотность $p_{\xi + \eta}$, вычисляемую по следующей формуле:
	\[
		p_{\xi + \eta}(z) = \int_\R p_\xi(z - x)p_\eta(x)dx = \int_\R p_\eta(z - x)p_\xi(x)dx
	\]
\end{corollary}

\begin{proof}
	Нужно воспользоваться леммой о свёртке и расписать функцию распределения:
	\begin{multline*}
		F_{\xi + \eta}(z) = \int_\R F_\xi(z - x)dF_\eta(x) =
		\\
		\int_\R \ps{\int_{-\infty}^{z - x} p_\xi(y)dy} p_\eta(x)dx = \int_\R \ps{\int_{-\infty}^z p_\xi(t - x)dt}p_\eta(x)dx =
		\\
		\int_{-\infty}^z \underbrace{\ps{\int_\R p_\xi(t - x)p_\eta(x)dx}}_{p_{\xi + \eta}(t)}dt
	\end{multline*}
\end{proof}

\section{Дисперсия и ковариация}

\subsection{Для случайных величин}

\begin{definition}
	\textit{Дисперсией случайной величины} $\xi$, $\E\xi < \infty$ называется следующая величина:
	\[
		D\xi = \E(\xi - E\xi)^2
	\]
\end{definition}

\begin{note}
	Смысл дисперсии состоит в том, что это среднее квадратичное отклонение от среднего значения, степень разброса случайной величины. Поэтому при $c > 0$ вероятность $\xi$ попасть в отрезок $[\E\xi - c\sqrt{D\xi}; \E\xi + c\sqrt{D\xi}]$ всегда большая.
\end{note}

\begin{definition}
	\textit{Ковариацией случайных величин $\xi$ и $\eta$}, $\E\xi, \E\eta < \infty$ называется следующее значение:
	\[
		\cov(\xi, \eta) = \E(\xi - \E\xi)(\eta - \E\eta)
	\]
\end{definition}

\begin{definition}
	Случайные величины $\xi, \eta$ называются \textit{некоррелирующими}, если $\cov(\xi, \eta) = 0$.
\end{definition}

\begin{definition}
	\textit{Коэффициентом корреляции случайных величин} $\xi$ и $\eta$, $0 < D\xi, D\eta < +\infty$, называется следующая величина:
	\[
		\corr(\xi, \eta) = \rho(\xi, \eta) = \frac{\cov(\xi, \eta)}{\sqrt{D\xi \cdot D\eta}}
	\]
\end{definition}