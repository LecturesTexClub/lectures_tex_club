\begin{lemma} (Свойства дисперсии и ковариации)
	\begin{enumerate}
		\item Ковариация является билинейной симметричной формой
		
		\item Дисперсия является квадратичной формой ковариации, $D\xi = \cov(\xi, \xi)$
		
		\item Имеет место формула для ковариации:
		\[
			\cov(\xi, \eta) = \E\xi\eta - (\E\xi)\E\eta
		\]
		
		\item Имеет место формула для дисперсии:
		\[
			D\xi = \E\xi^2 - (\E\xi)^2
		\]
		
		\item Справедливо неравенство Коши-Буняковского:
		\[
			|\E\xi\eta| \le \sqrt{\E\xi^2 \cdot \E\eta^2}
		\]
		Причём равенство достигается тогда и только тогда, когда $\xi$ и $\eta$ линейно зависимы почти всюду.
		
		\item $|\corr(\xi, \eta)| \le 1$, причём равенство достигается тогда и только тогда, когда случайные величины $\xi - \E\xi$ и $\eta - \E\eta$ линейно независимы почти всюду
	\end{enumerate}
\end{lemma}

\begin{proof}~
	\begin{enumerate}
		\item Очевидно
		
		\item Тривиально
		
		\item Распишем скобки ковариации:
		\[
			\cov(\xi, \eta) = \E(\xi - \E\xi)(\eta - \E\eta) = \E(\xi\eta - 2\E\xi\eta + (\E\xi)\E\eta) = \E\xi\eta - (\E\xi)\E\eta
		\]
		
		\item Очевидно из предыдущего свойства
		
		\item Возможно 2 случая:
		\begin{itemize}
			\item $\E\xi^2 \cdot \E\eta^2 \neq 0$. Тогда, произведём нормировку случайных величин:
			\[
				\ol{\xi} = \frac{\xi}{\sqrt{\E\xi^2}};\ \ \ol{\eta} = \frac{\eta}{\sqrt{\E\eta^2}}
			\]
			По классическому неравенству КБШ, \(|\ol{\xi} \cdot \ol{\eta}|^2 \le \ol{\xi}^2 \cdot \ol{\eta}^2\). По определению можно проверить, что $\E\ol{\xi} = \E\ol{\eta} = 1$. Если применить матожидание на неравенство, то получим $\E|\ol{\xi} \cdot \ol{\eta}| \le 1$. Отсюда уже явной подстановкой $\ol{\xi}, \ol{\eta}$ и свойством $|E\xi\eta| \le \E|\xi\eta|$ получаем требуемое.
			
			\item $\E\xi^2 \cdot \E\eta^2 = 0$. Не умаляя общности, пусть $\E\xi^2 = 0$. Распишем математическое ожидание по определению:
			\[
				\E\xi^2 = \int_\Omega \xi^2(w)dP(w) = 0
			\]
			Такое возможно тогда и только тогда, когда $\xi^2$ является $P$-почти всюду нулём, то есть $P(\xi^2 = 0) = P(\xi = 0) = 1$. Тогда $|\E\xi\eta| = 0$ и утверждение тривиально
		\end{itemize}
	
		\item Отцентрируем и отнормируем наши исходные величины так, что их матожидание ноль, а дисперсия --- единица:
		\[
			\xi' = \frac{\xi - \E\xi}{\sqrt{D\xi}};\ \ \eta' = \frac{\eta - \E\eta}{\sqrt{D\eta}}
		\]
		Остаётся посмотреть на значение дисперсии $\xi' \pm \eta'$:
		\[
			0 \le D(\xi' \pm \eta') = D\xi' + D\eta' \pm 2\cov(\xi', \eta') = 2\ps{1 \pm \frac{\cov(\xi, \eta)}{\sqrt{D\xi} \cdot \sqrt{D\eta}}}
		\]
		Отсюда уже очевидно требуемое
	\end{enumerate}
\end{proof}

\begin{corollary}
	Если $\{\xi_k\}_{k = 1}^n$ --- попарно некоррелирующие (или независимые) случайные величины с конечными дисперсиями, то верна формула для дисперсии суммы случайных величин:
	\[
		D(\xi_1 + \ldots + \xi_n) = \sum_{k = 1}^n D\xi_k
	\]
\end{corollary}

\begin{proof}
	Распишем дисперсию через ковариацию (из независимости следует некорреляция):
	\[
		D(\xi_1 + \ldots + \xi_n) = \cov(\xi_1 + \ldots + \xi_n, \xi_1 + \ldots + \xi_n) = \sum_{i = 1}^n \sum_{j = 1}^n \cov(\xi_i, \xi_j) = \sum_{t = 1}^n \cov(\xi_t, \xi_t)
	\]
\end{proof}

\subsection{Для случайных векторов}

\begin{definition}
	Пусть $\xi = (\xi_1, \ldots, \xi_n)$ --- случайный вектор. Тогда \textit{матожиданием случайного вектора $\xi$} называется вектор, состоящий из математических ожиданий компонент:
	\[
		\E\xi = (\E\xi_1, \ldots, \E\xi_n)
	\]
\end{definition}

\begin{definition}
	\textit{Дисперсией (матрицей ковариаций) случайного вектора $\xi$} называется матрица $D\xi$, определённая следующим образом:
	\[
		D\xi = (\cov(\xi_i, \xi_j))
	\]
\end{definition}

\begin{proposition}
	Матрица ковариаций $D\xi$ является симметричной и неотрицательно определённой.
\end{proposition}

\begin{proof}
	Прямое следствие того, что ковариация является симметричной и неотрицательно определённой формой.
\end{proof}

\begin{proposition} (Неравенство Маркова)
	Если случайная величина $\xi \ge 0$ и $a > 0$, то имеет место следующее неравенство:
	\[
		P(\xi > a) \le \frac{\E\xi}{a}
	\]
\end{proposition}

\begin{proof}
	Распишем матожидание $\xi$ и сделаем оценку снизу:
	\[
		\E\xi = \int_{\Omega} \xi(\omega)dP(\omega) = \int_{\Omega_1} \xi(\omega)dP(\omega) + \int_{\Omega \bs \Omega_1} \xi(\omega)dP(\omega)
	\]
	где $\Omega_1 = \{\omega \in \Omega \colon \xi(w) > a\}$. Тогда:
	\[
		\E\xi \ge \int_{\Omega_1} \xi(\omega)dP(\omega) \ge a \cdot P(\Omega_1) = a \cdot P(\xi > a)
	\]
\end{proof}

\begin{proposition} (Неравенство Чебышёва)
	Если дисперсия $D\xi$ случайной величины $\xi$ конечна, то верно неравенство для отклонения этой величины от матожидания:
	\[
		\forall \eps > 0\ \ P(|\xi - \E\xi| \ge \eps) \le \frac{D\xi}{\eps^2}
	\]
\end{proposition}

\begin{proof}
	Достаточно свести к неравеству Маркова. Введём $\eta = (\xi - \E\xi)^2$. Тогда $\E\eta = D\xi$, а поэтому применение неравенства Маркова для $\eta$ и $\eps^2$ делает всё очевидным.
\end{proof}

\begin{proposition} (Неравенство Йенсена)
	Пусть $g(x)$ --- выпуклая вниз функция и $\E\xi$ конечно, то верно неравенство
	\[
		\E g(\xi) \ge g(\E\xi)
	\]
\end{proposition}

\begin{proof}
	Если $g(x)$ выпуклая вниз, то в каждой точке $g$ есть касательная, проходящая ниже этой функции:
	\[
		\forall x_0 \in \R\ \ \exists \lambda(x_0) \such \forall x \in \R\ \ g(x) \ge g(x_0) + \lambda(x_0)(x - x_0)
	\]
	Теперь подставим $x_0 = \E\xi$ и $x = \xi$:
	\[
		\exists \lambda \such g(\xi) \ge g(\E\xi) + \lambda \cdot (\xi - \E\xi)
	\]
	Если применить к обеим частям математическое ожидание, то получим требуемое.
\end{proof}

\section{Сходимости случайных величин}

\begin{definition}
	Последовательность случайных величин $\{\xi_n\}_{n = 1}^\infty$ \textit{сходится к случайной величине $\xi$ с вероятностью 1} (или говорят \textit{почти наверное}), если выполнено равенство:
	\[
		P(\{\omega \in \Omega \colon \lim_{n \to \infty} \xi_n(\omega) = \xi(\omega)\}) = 1
	\]
	Обозначается как $\xi_n \to^{\text{P п.н.}} \xi$
\end{definition}

\begin{definition}
	Последовательность случайных величин $\{\xi_n\}_{n = 1}^\infty$ \textit{сходится к случайной величине $\xi$ по вероятности}, если
	\[
		\forall \eps > 0\ \ P(|\xi_n - \xi| \ge \eps) \xrightharpoondown[n \to \infty]{} 0
	\]
	Обозначается как $\xi_n \to^P \xi$
\end{definition}

\begin{definition}
	Последовательность случайных величин $\{\xi_n\}_{n = 1}^\infty$ \textit{сходится к случайной величине $\xi$ в среднем порядка $p > 0$}, если имеет место сходимость в пространстве $L_p$:
	\[
		\E|\xi_n - \xi|^p \xrightarrow[n \to \infty]{} 0
	\]
	Обозначается как $\xi_n \to^{L_p} \xi$
\end{definition}

\begin{definition}
	Последовательность случайных величин $\{\xi_n\}_{n = 1}^\infty$ \textit{сходится к случайной величине $\xi$ по распределению}, если $\forall f \colon \R \to \R$ --- непрерывной, ограниченной функции выполнено следующее:
	\[
		\E f(\xi_n) \xrightarrow[n \to \infty]{
		} \E f(\xi)
	\]
	Обозначается как $\xi_n \to^d \xi$ (где $d$ от слова $distribution$).
\end{definition}

\begin{theorem} (Критерий сходимости почти наверное)
	Последовательность случайных величин $\{x_n\}_{n = 1}^\infty$ сходится к $\xi$ почти наверное тогда и только тогда, когда выполнено утверждение:
	\[
		\forall \eps > 0\ \ P(\sup_{k \ge n} |\xi_k - \xi| > \eps) \xrightarrow[n \to \infty]{} 0
	\]
\end{theorem}

\begin{proof}
	\textcolor{red}{Придумать повествование. Дописать}
\end{proof}

\begin{theorem} (Взаимоотношение видом сходимости)
	\begin{enumerate}
		\item $\xi_n \to^{\text{P п.н.}} \xi \Ra \xi_n \to^P \xi$
		
		\item $\xi_n \to^{L_p} \xi \Ra \xi_n \to^P \xi$
		
		\item $\xi_n \to^P \xi \Ra \xi_n \to^d \xi$
	\end{enumerate}
\end{theorem}

\begin{proof}
	\textcolor{red}{Дописать}
\end{proof}

\begin{note}
	\textcolor{red}{Сюда надо картинку со сходимостями}
\end{note}

\begin{lemma} (Достаточные условия сходимости почти наверное)
	Если для последовательности случайных величин $\{\xi_n\}_{n = 1}^\infty$ и $\xi$ выполнено утверждение
	\[
		\forall \eps > 0\ \ \sum_{n = 1}^\infty P(|\xi_n - \xi| \ge \eps) < +\infty \text{ --- ряд сходится}
	\]
	Тогда $\xi_n \to^{P \text{ п.н.}} \xi$
\end{lemma}

\textcolor{red}{Дописать}