\begin{lemma} (Свойства дисперсии и ковариации)
	\begin{enumerate}
		\item Ковариация является билинейной симметричной формой
		
		\item Дисперсия является квадратичной формой ковариации, $D\xi = \cov(\xi, \xi)$
		
		\item Имеет место формула для ковариации:
		\[
			\cov(\xi, \eta) = \E\xi\eta - (\E\xi)\E\eta
		\]
		
		\item Имеет место формула для дисперсии:
		\[
			D\xi = \E\xi^2 - (\E\xi)^2
		\]
		
		\item Справедливо неравенство Коши-Буняковского:
		\[
			|\E\xi\eta| \le \sqrt{\E\xi^2 \cdot \E\eta^2}
		\]
		Причём равенство достигается тогда и только тогда, когда $\xi$ и $\eta$ линейно зависимы почти всюду.
		
		\item $|\corr(\xi, \eta)| \le 1$, причём равенство достигается тогда и только тогда, когда случайные величины $\xi - \E\xi$ и $\eta - \E\eta$ линейно зависимы почти всюду
	\end{enumerate}
\end{lemma}

\begin{proof}~
	\begin{enumerate}
		\item Очевидно
		
		\item Тривиально
		
		\item Распишем скобки ковариации:
		\[
			\cov(\xi, \eta) = \E(\xi - \E\xi)(\eta - \E\eta) = \E(\xi\eta - 2\E\xi\eta + (\E\xi)\E\eta) = \E\xi\eta - (\E\xi)\E\eta
		\]
		
		\item Очевидно из предыдущего свойства
		
		\item Возможно 2 случая:
		\begin{itemize}
			\item $\E\xi^2 \cdot \E\eta^2 \neq 0$. Тогда, произведём нормировку случайных величин:
			\[
				\ol{\xi} = \frac{\xi}{\sqrt{\E\xi^2}};\ \ \ol{\eta} = \frac{\eta}{\sqrt{\E\eta^2}}
			\]
			По классическому неравенству КБШ, \(|\ol{\xi} \cdot \ol{\eta}|^2 \le \ol{\xi}^2 \cdot \ol{\eta}^2\). По определению можно проверить, что $\E\ol{\xi}^2 = \E\ol{\eta}^2 = 1$. Если применить матожидание на неравенство, то получим $\E|\ol{\xi} \cdot \ol{\eta}| \le 1$. Отсюда уже явной подстановкой $\ol{\xi}, \ol{\eta}$ и свойством $|E\xi\eta| \le \E|\xi\eta|$ получаем требуемое.
			
			\item $\E\xi^2 \cdot \E\eta^2 = 0$. Не умаляя общности, пусть $\E\xi^2 = 0$. Распишем математическое ожидание по определению:
			\[
				\E\xi^2 = \int_\Omega \xi^2(w)dP(w) = 0
			\]
			Такое возможно тогда и только тогда, когда $\xi^2$ является $P$-почти всюду нулём, то есть $P(\xi^2 = 0) = P(\xi = 0) = 1$. Тогда $|\E\xi\eta| = 0$ и утверждение тривиально
		\end{itemize}
	
		\item Отцентрируем и отнормируем наши исходные величины так, что их матожидание ноль, а дисперсия --- единица:
		\[
			\xi' = \frac{\xi - \E\xi}{\sqrt{D\xi}};\ \ \eta' = \frac{\eta - \E\eta}{\sqrt{D\eta}}
		\]
		Остаётся посмотреть на значение дисперсии $\xi' \pm \eta'$:
		\[
			0 \le D(\xi' \pm \eta') = D\xi' + D\eta' \pm 2\cov(\xi', \eta') = 2\ps{1 \pm \frac{\cov(\xi, \eta)}{\sqrt{D\xi} \cdot \sqrt{D\eta}}}
		\]
		Отсюда ясна оценка на корреляцию. Осталось выяснить, почему есть линейная зависимость при $\corr(\xi, \eta) = \pm 1$. Действительно, если, например, $\corr(\xi, \eta) = 1$, то $D(\xi' - \eta') = 0$. Значит, $\exists c \in \R \colon \xi' - \eta' = c$ почти наверное. Расписывая $\xi'$ и $\eta'$, получим линейную зависимость исходных величин
	\end{enumerate}
\end{proof}

\begin{corollary}
	Если $\{\xi_k\}_{k = 1}^n$ --- попарно некоррелирующие (или независимые) случайные величины с конечными дисперсиями, то верна формула для дисперсии суммы случайных величин:
	\[
		D(\xi_1 + \ldots + \xi_n) = \sum_{k = 1}^n D\xi_k
	\]
\end{corollary}

\begin{proof}
	Распишем дисперсию через ковариацию (из независимости следует некорреляция):
	\[
		D(\xi_1 + \ldots + \xi_n) = \cov(\xi_1 + \ldots + \xi_n, \xi_1 + \ldots + \xi_n) = \sum_{i = 1}^n \sum_{j = 1}^n \cov(\xi_i, \xi_j) = \sum_{t = 1}^n \cov(\xi_t, \xi_t)
	\]
\end{proof}

\subsection{Для случайных векторов}

\begin{definition}
	Пусть $\xi = (\xi_1, \ldots, \xi_n)$ --- случайный вектор. Тогда \textit{матожиданием случайного вектора $\xi$} называется вектор, состоящий из математических ожиданий компонент:
	\[
		\E\xi = (\E\xi_1, \ldots, \E\xi_n)
	\]
\end{definition}

\begin{definition}
	\textit{Дисперсией (матрицей ковариаций) случайного вектора $\xi$} называется матрица $D\xi$, определённая следующим образом:
	\[
		D\xi = (\cov(\xi_i, \xi_j))
	\]
\end{definition}

\begin{proposition}
	Матрица ковариаций $D\xi$ является симметричной и неотрицательно определённой.
\end{proposition}

\begin{proof}
	Прямое следствие того, что ковариация является симметричной и неотрицательно определённой формой.
\end{proof}

\begin{proposition} (Неравенство Маркова)
	Если случайная величина $\xi \ge 0$ и $a > 0$, то имеет место следующее неравенство:
	\[
		P(\xi > a) \le \frac{\E\xi}{a}
	\]
\end{proposition}

\begin{proof}
	Распишем матожидание $\xi$ и сделаем оценку снизу:
	\[
		\E\xi = \int_{\Omega} \xi(\omega)dP(\omega) = \int_{\Omega_1} \xi(\omega)dP(\omega) + \int_{\Omega \bs \Omega_1} \xi(\omega)dP(\omega)
	\]
	где $\Omega_1 = \{\omega \in \Omega \colon \xi(w) > a\}$. Тогда:
	\[
		\E\xi \ge \int_{\Omega_1} \xi(\omega)dP(\omega) \ge a \cdot P(\Omega_1) = a \cdot P(\xi > a)
	\]
\end{proof}

\begin{proposition} (Неравенство Чебышёва)
	Если дисперсия $D\xi$ случайной величины $\xi$ конечна, то верно неравенство для отклонения этой величины от матожидания:
	\[
		\forall \eps > 0\ \ P(|\xi - \E\xi| \ge \eps) \le \frac{D\xi}{\eps^2}
	\]
\end{proposition}

\begin{proof}
	Достаточно свести к неравеству Маркова. Введём $\eta = (\xi - \E\xi)^2$. Тогда $\E\eta = D\xi$, а поэтому применение неравенства Маркова для $\eta$ и $\eps^2$ делает всё очевидным.
\end{proof}

\begin{proposition} (Неравенство Йенсена)
	Пусть $g(x)$ --- выпуклая вниз функция и $\E\xi$ конечно, то верно неравенство
	\[
		\E g(\xi) \ge g(\E\xi)
	\]
\end{proposition}

\begin{proof}
	Если $g(x)$ выпуклая вниз, то в каждой точке $g$ есть касательная, проходящая ниже этой функции:
	\[
		\forall x_0 \in \R\ \ \exists \lambda(x_0) \such \forall x \in \R\ \ g(x) \ge g(x_0) + \lambda(x_0)(x - x_0)
	\]
	Теперь подставим $x_0 = \E\xi$ и $x = \xi$:
	\[
		\exists \lambda \such g(\xi) \ge g(\E\xi) + \lambda \cdot (\xi - \E\xi)
	\]
	Если применить к обеим частям математическое ожидание, то получим требуемое.
\end{proof}

\section{Сходимости случайных величин}

\begin{definition}
	Последовательность случайных величин $\{\xi_n\}_{n = 1}^\infty$ \textit{сходится к случайной величине $\xi$ с вероятностью 1} (или говорят $P$\textit{-почти наверное}), если выполнено равенство:
	\[
		P(\{\omega \in \Omega \colon \lim_{n \to \infty} \xi_n(\omega) = \xi(\omega)\}) = 1
	\]
	Обозначается как $\xi_n \to^{\text{P п.н.}} \xi$
\end{definition}

\begin{definition}
	Последовательность случайных величин $\{\xi_n\}_{n = 1}^\infty$ \textit{сходится к случайной величине $\xi$ по вероятности}, если
	\[
		\forall \eps > 0\ \ P(|\xi_n - \xi| \ge \eps) \xrightarrow[n \to \infty]{} 0
	\]
	Обозначается как $\xi_n \to^P \xi$
\end{definition}

\begin{definition}
	Последовательность случайных величин $\{\xi_n\}_{n = 1}^\infty$ \textit{сходится к случайной величине $\xi$ в среднем порядка $p > 0$}, если имеет место сходимость в пространстве $L_p$:
	\[
		\E|\xi_n - \xi|^p \xrightarrow[n \to \infty]{} 0
	\]
	Обозначается как $\xi_n \to^{L_p} \xi$
\end{definition}

\begin{definition}
	Последовательность случайных величин $\{\xi_n\}_{n = 1}^\infty$ \textit{сходится к случайной величине $\xi$ по распределению}, если $\forall f \colon \R \to \R$ --- непрерывной, ограниченной функции выполнено следующее:
	\[
		\E f(\xi_n) \xrightarrow[n \to \infty]{
		} \E f(\xi)
	\]
	Обозначается как $\xi_n \to^d \xi$ (где $d$ от слова $distribution$).
\end{definition}

\begin{theorem} (Критерий сходимости почти наверное)
	Последовательность случайных величин $\{x_n\}_{n = 1}^\infty$ сходится к $\xi$ $P$-почти наверное тогда и только тогда, когда выполнено утверждение:
	\[
		\forall \eps > 0\ \ P(\sup_{k \ge n} |\xi_k - \xi| > \eps) \xrightarrow[n \to \infty]{} 0
	\]
\end{theorem}

\begin{proof}
	Будем доказывать эквивалентность исходя из того, что $P(\xi_n \centernot\to \xi) = 0$ эквивалентно $P$-почти всюду сходимости. Итак, введём события $A_n^\eps$ и выразим через них вероятность несходимости:
	\[
		A_n^\eps = \bigcup_{k \ge n} \{\omega \in \Omega \colon |\xi_k - \xi| > \eps\}
	\]
	Тогда:
	\[
		\{\xi_n \centernot\to \xi\} = \bigcup_{m = 1}^\infty \bigcap_{n = 1}^\infty A_n^{1 / m}
	\]
	Остаётся лишь увидеть эквивалентные утверждения:
	\begin{multline*}
		P(\xi_n \centernot\to \xi) = P\ps{\bigcup_{m = 1}^\infty \bigcap_{n = 1}^\infty A_n^{1 / m}} = 0 \Lra
		\\
		\forall m \in \N\  P\ps{\bigcap_{n = 1}^\infty A_n^{1 / m}} = 0 \Lra
		\\
		\forall m \in \N\ P(\lim_{n \to \infty} A_n^{1 / m}) = 0 = \lim_{n \to \infty} P(A_n^{1 / m}) \Lra
		\\
		\forall \eps > 0\ \ \lim_{n \to \infty} P(A_n^\eps) = 0
	\end{multline*}
	Последнее равенство эквивалентно условию напрямую, можно легко показать равенство событий в скобках.
\end{proof}

\begin{theorem} (Взаимоотношение видов сходимости)
	\begin{enumerate}
		\item $\xi_n \to^{\text{P п.н.}} \xi \Ra \xi_n \to^P \xi$
		
		\item $\xi_n \to^{L_p} \xi \Ra \xi_n \to^P \xi$
		
		\item $\xi_n \to^P \xi \Ra \xi_n \to^d \xi$
	\end{enumerate}
\end{theorem}

\begin{proof}~
	\begin{enumerate}
		\item Зафиксируем $\eps > 0$. Покажем нужный предел за счёт оценки сверху:
		\[
			P(|\xi_n - \xi| > \eps) \le P(|\xi_n - \xi| \ge \eps / 2) \le P(\sup_{k \ge n} |\xi_k - \xi| \ge \eps / 2) \xrightarrow[n \to \infty]{} 0
		\]
		
		\item Снова зафиксируем $\eps > 0$. Тогда для любого $p > 0$ верна оценка:
		\[
			P(|\xi_n - \xi| > \eps) = P(|\xi_n - \xi|^p \ge \eps^p) \le \frac{\E |\xi_n - \xi|^p}{\eps^p} \xrightarrow[n \to \infty]{} 0
		\]
		
		\item Пусть $f$ --- произвольная ограниченная непрерывная функция. Тогда $\exists M > 0 \colon \forall x \in \R\ |f(x)| \le M$. Тогда нужно доказать следующий факт:
		\[
			\forall \eps > 0\ \exists N \in \N \such \forall n > N\ |\E f(\xi_n) - \E f(\xi)| = |\E(f(\xi_n) - f(\xi))| < \eps
		\]
		Итак, зафиксируем $\eps > 0$. Разобьём $\Omega$ на такие множества, что мы можем хорошо оценить величину $|\E(f(\xi_n) - f(\xi))| \le \E|f(\xi_n) - f(\xi)|$. Для начала, потребуем $\exists N \in \N \colon P(|\xi| \ge N) \le \eps / (8M)$ (такое $N$ существует, коль скоро $P(|\xi| \ge N) = P(\xi \in [-N; N]) = F_\xi(N) - F_\xi(-N)$, а пределы этих слагаемых равны 1 и 0 соответственно). При этом $f$ непрерывна на любом отрезке, то есть равномерно непрерывна на нём. Тогда, на $[-N; N]$ можем сказать следующее:
		\[
			\exists \delta > 0 \such \forall x, y \in [-N; N],\ |x - y| < \delta\ \ |f(x) - f(y)| \le \frac{\eps}{2}
		\]
		Итак, разобьём $\Omega$ на три части:
		\begin{align*}
			&{A_1 = \{\omega \in \Omega \colon |\xi_n - \xi| \le \delta \wedge |\xi| \le N\}}
			\\
			&{A_2 = \{\omega \in \Omega \colon |\xi_n - \xi| \le \delta \wedge |\xi| > N\}}
			\\
			&{A_3 = \{\omega \in \Omega \colon |\xi_n - \xi| > \delta\}}
		\end{align*}
		Разложим нашу оценку матожидания:
		\[
			|\E (f(\xi_n) - f(\xi))| \le \E|f(\xi_n) - f(\xi)| = \sum_{i = 1}^3 \E(|f(\xi_n) - f(\xi)| \cdot \chi_{A_i})
		\]
		А теперь разберёмся с каждым слагаемым по отдельности:
		\begin{itemize}
			\item На $A_1$ всё просто: $|f(\xi_n) - f(\xi)| \le \eps / 2$, поэтому
			\[
				\E(|f(\xi_n) - f(\xi)| \cdot \chi_{A_1}) \le \eps / 2 \cdot P(A_i) \le \frac{\eps}{2}
			\]
			
			\item Для $A_{2, 3}$ мы уже не можем воспользоваться неравенством со значениями, поэтому сделаем консервативную оценку $\forall x, y \in \R\ \ |f(x) - f(y)| \le 2M$. Итак:
			\[
				\sum_{i = 2}^3 \E(|f(\xi_n) - f(\xi)| \cdot \chi_{A_i}) \le 2M(P(A_2) + P(A_3)) \le 2M(P(|\xi| > N) + P(|\xi_n - \xi| > \delta))
			\]
			У первого слагаемого в скобках оценка уже дана --- это $\eps / (8M)$. А второе уходит за счёт основного условия: наличие сходимости по мере. Стало быть
			\[
				\forall \eps > 0\ \exists T \in \N \such \forall n > T\ \ |\E f(\xi_n) - \E f(\xi)| < \frac{\eps}{2} + \frac{\eps}{4} + \frac{\eps}{4} = \eps
			\]
		\end{itemize}
	\end{enumerate}
\end{proof}

\begin{note}
	\textcolor{red}{Сюда надо картинку со сходимостями}
\end{note}

\begin{lemma} (Достаточные условия сходимости почти наверное)
	Если для последовательности случайных величин $\{\xi_n\}_{n = 1}^\infty$ и $\xi$ выполнено утверждение
	\[
		\forall \eps > 0\ \ \sum_{n = 1}^\infty P(|\xi_n - \xi| \ge \eps) < +\infty \text{ --- ряд сходится}
	\]
	Тогда $\xi_n \to^{P \text{ п.н.}} \xi$
\end{lemma}

\begin{proof}
	Воспользуемся критерием сходимости почти наверное. Зафиксируем $\eps > 0$ и оценим вероятность для супремума:
	\[
		P(\sup_{k \ge n} |\xi_k - \xi| > \eps) = P\ps{\bigcup_{k = n}^\infty \{|\xi_k - \xi| > \eps\}} \le \sum_{k = n}^\infty P(|\xi_k - \xi| > \eps) \le \sum_{k = n}^\infty P(|\xi_k - \xi| \ge \eps) \xrightarrow[n \to \infty]{} 0
	\]
	Последний предел верен, коль скоро мы вышли на остаток сходящегося ряда из условия.
\end{proof}

\begin{corollary}
	Если $\xi_n \to^P \xi$, то существует подпоследовательность $\{\xi_{n_k}\}_{k = 1}^\infty$ такая, что она сходится $P$-почти наверное к $\xi$.
\end{corollary}

\begin{proof}
	Всё интуитивно: достаточно взять такую подпоследовательность $\{n_k\}_{k = 1}^\infty$, чтобы нужный ряд всегда оценивался сходящимся рядом, скажем, $\{2^{-k}\}_{k = 1}^\infty$. Итак, за счёт сходимости по мере возьмём такую подпоследовательность:
	\[
		\forall k \in \N\ \exists n_k \such n_k > n_{k - 1} \wedge P\ps{|\xi_{n_k} - \xi| \ge \frac{1}{k}} \le 2^{-k}
	\]
	Тогда понятно, что для любого $\eps > 0$ найдётся $K \in \N$ такое, что $1 / K \le \eps$, благодаря чему исходный ряд распадётся на конечную <<голову>> и бесконечный хвост, который спокойно оценивается по построению рядом $\sum_{k = 1}^\infty 2^{-k}$. Следовательно, все достаточные условия сходимости почти всюду выполнены.
\end{proof}

\section{Закон Больших Чисел}

\subsection{Базовая теорема}

\begin{theorem} (Закон Больших Чисел в форме Чебышёва)
	Пусть $\{\xi_n\}_{n = 1}^\infty$ --- случайные величины, на которые наложены следующие ограничения:
	\begin{enumerate}
		\item $\xi_n$ попарно некоррелируют
		
		\item Существует константа $C > 0$ такая, что $\forall n \in \N\ \ D\xi_n \le C$
	\end{enumerate}
	Тогда, если принять обозначение $S_n = \xi_1 \plusdots \xi_n$, верен предел:
	\[
		\frac{S_n - \E S_n}{n} \xrightarrow[n \to \infty]{P} 0
	\]
\end{theorem}

\begin{note}
	В чём смысл ЗБЧ? Это теоретическое обоснование принципа устойчивых частот. Если положить за $\xi_i$ индикатор того, что в $i$-м эксперименте произошло событие $A$, то
	\[
		\nu_n(A) = \frac{\xi_1 + \ldots + \xi_n}{n} \xrightarrow[n \to \infty]{P \text{п.н.}} \E\xi_1 = P(A)
	\]
	где $\nu_n(A)$ --- это частота появления $A$.
	
	Несложно заметить, как произошёл лёгкий обман. В теореме мы показываем сходимость по мере, которая не эквивалентна ожидаемой сходимости почти всюду. Это исправляет усиленная версия теоремы ниже.
\end{note}

\begin{theorem} (Усиленный Закон Больших Чисел в форме Кантелли)
	Пусть $\{\xi_n\}_{n = 1}^\infty$ --- случайные величины, на которые наложены следующие ограничения:
	\begin{enumerate}
		\item $\xi_n$ попарно независимы
		
		\item $\exists C > 0 \such \forall n \in \N\ \ \E(\xi_n - \E\xi_n)^4 \le C$
	\end{enumerate}
	Тогда, если принять обозначение $S_n = \xi_1 \plusdots \xi_n$, верен предел:
	\[
		\frac{S_n - \E S_n}{n} \xrightarrow[n \to \infty]{P \text{ п.н.}} 0
	\]
\end{theorem}

\begin{proof}
	Для начала перейдём к таким случайным величинам $\xi'_n$, для которых математическое ожидание ноль: $\xi'_n = \xi_n - \E\xi_n$. Тогда $\E S'_n = 0$ и мы можем не беспокоиться о разности в доказываемом пределе. Мы будем пытаться показать, что для $S'_n / n$ и 0 сходится ряд из леммы о достаточных условиях сходимости почти наверное. Значит, при фиксированном $\eps > 0$ нам нужно получить оценку на $P(|S'_n / n| \ge \eps)$:
	\[
		P\ps{\md{\frac{S'_n}{n}} \ge \eps} = P\ps{\frac{{S'}_n^4}{n^4} \ge \eps^4} \le \frac{\E {S'}_n^4}{n^4 \eps^4}
	\]
	Мистическая степень 4, естественно, появилась из-за условия на четвёртый момент. Теперь мы будем оценивать появившееся из неравенства Маркова матожидание:
	\[
		\E {S'}_n^4 = \sum_{i, j, k, l = 1}^n \E \xi'_i\xi'_j\xi'_k\xi'_l
	\]
	Тут-то и раскрывается наш переход к $\xi'$. Эти величины точно так же, как и исходные, являются независимыми. Это значит, что если среди четвёрки чисел есть хотя бы одно уникальное, то это произведение обязательно станет нулём.Стало быть, остаются лишь 2 возможные схемы: либо все 4 индекса одинаковы, либо это 2 + 2:
	\[
		E {S'}_n^4 = \sum_{i = 1}^n \E {\xi'}_i^4 + C_4^2\sum_{1 \le i < j \le n} \E{\xi'}_i^2 \cdot \E{\xi'}_j^2
	\]
	Сразу из условия, $\E {\xi'}_i^4 \le C$. Для квадратов же применим неравенство Йенсена:
	\[
		(\E {\xi'}_i^2)^2 \le \E {\xi'}_i^4 \Ra \E{\xi'}_i^2 \le \sqrt{\E{\xi'}_i^4} \le \sqrt{C}
	\]
	Таким образом, $\E{S'}_n^4 \le nC + 6 C_n^2 C = O(n^2)$. Стало быть, проверяемая вероятность стремится к нулю и ряд сходится при любом $\eps > 0$, то есть $S'_n / n = (S_n - \E S_n) / n \to^{P \text{ п.н.}} 0$
\end{proof}

\begin{note}
	Можно ли отказаться от условия на наличие четвёртого момента? Оказывается, можно, но для этого необходимо развить целый аппарат.
\end{note}