\begin{note}
	Поздравляю, мы доказали Усиленный Закон Больших Чисел в одной из самых приятных формулировок. Теперь встаёт не менее важный прикладной вопрос: <<А можно ли что-то сказать про скорость сходимости УЗБЧ? Иначе говоря, какое $n$ нужно взять, чтобы добиться определённой точности в отклонении от матожидания?>> Это мы и будем изучать далее, для ответа на вопрос нужна Центральная Предельная Теорема
\end{note}

\section{Слабая сходимость вероятностных мер}

\begin{definition}
	Пусть $\{F_n\}_{n = 1}^\infty$ --- функции распределения на $\R$. \textit{Последовательность $F_n$ слабо сходится к функции распределения $F$}, если выполнено условие:
	\[
		\forall f \colon \R \to \R \text{ --- непрерывная ограниченная}\ \ \int_\R f(x)dF_n(x) \xrightarrow[n \to \infty]{} \int_\R f(x)dF(x)
	\]
	Обозначается как $F_n \to^w F$
\end{definition}

\begin{anote}
	$w$ от слова $weak$, полагаю.
\end{anote}

\begin{definition}
	\textit{Последовательность $\{F_n\}_{n = 1}^\infty$ функций распределений на $\R$ сходится в основном к функции распределения $F$}, если выполнено утверждение:
	\[
		\forall x \in C(F)\ \ F_n(x) \xrightarrow[n \to \infty]{} F(x)
	\]
	где $C(F)$ --- точки непрерывности $F$. Сходимость обозначается как $F_n \Ra F$
\end{definition}

\begin{definition}
	\textit{Последовательность вероятностных мер $\{P_n\}_{n = 1}^\infty$ слабо сходится к вероятностной мере $P$ (все в пространстве $(\R^m, \B(\R^m))$)}, если выполнено условие:
	\[
		\forall f \colon \R^m \to \R \text{ --- непрерывная ограниченная}\ \ \int_{\R^m} f(x)dP_n(x) \xrightarrow[n \to \infty]{} \int_{\R^m} f(x)dP(x)
	\]
	Обозначается как $P_n \to^w P$
\end{definition}

\begin{definition}
	\textit{Последовательность вероятностных мер $\{P_n\}_{n = 1}^\infty$ сходится в основном к вероятностной мере $P$ (все в пространстве $(\R^m, \B(\R^m))$)}, если выполнено условие:
	\[
		\forall B \in \B(\R^m), P(\vdelta B) = 0\ \ \ P_n(B) \xrightarrow[n \to \infty]{} P(B)
	\]
	Обозначается как $P_n \Ra P$
\end{definition}

\begin{note}
	Методом пристального взгляда можно заметить, что если $\{\xi_n\}_{n = 1}^\infty$ --- последовательность случайных величин, то верна эквивалентность:
	\[
		\xi_n \to^d \xi \Lra F_{\xi_n} \to^w F_\xi \Lra P_{\xi_n} \to^w P_\xi
	\]
	Это верно в силу теоремы о замене переменных в интеграле:
	\[
		\E f(\xi) = \int_\R f(x)dP_\xi(x) = \int_\R f(x)dF_\xi(x)
	\]
\end{note}

\begin{note}
	Основной факт, который известен про описанные выше слабые и в основном сходимости, состоит в их эквивалентности. Часть этого факта можно получить в следующей мощной теореме.
\end{note}

\begin{theorem} (Александрова, без доказательства)
	Пусть $\{P_n\}_{n = 1}^\infty, P$ --- вероятностные меры на пространстве $(\R^m, \B(\R^m))$. Тогда следующие условия эквивалентны:
	\begin{enumerate}
		\item $P_n \to^w P$
		
		\item $\varlimsup_{n \to \infty} P_n(A) \le P(A)$ для любого замкнутого $A \subseteq \R^m$
		
		\item $\varliminf_{n \to \infty} P_n(G) \ge P(G)$ для любого открытого $G \subseteq \R^m$
		
		\item $P_n \Ra P$
	\end{enumerate}
\end{theorem}

\begin{theorem} (об эквивалентности сходимостей)
	Пусть $\{P_n\}_{n = 1}^\infty, P$ --- вероятностные меры на $(\R, \B(\R))$, а $\{F_n\}_{n = 1}^\infty, F$ --- соответствующие функции распределения. Тогда следующие условия эквивалентны:
	\begin{enumerate}
		\item $P_n \to^w P$
		
		\item $P_n \Ra P$
		
		\item $F_n \to^w F$
		
		\item $F_n \Ra F$
	\end{enumerate}
\end{theorem}

\begin{proof}
	По теореме Александрова, уже $1 \Lra 2$. Более того, $1 \Lra 3$ эквивалентно по определениям. Осталось присоединить последнюю сходимость к остальным:
	\begin{itemize}
		\item[$2 \Ra 4$] Пусть $x \in C(F)$. Рассмотрим множество $B = \rsi{-\infty; x}$, тогда $\vdelta B = \{x\} \Ra P(\vdelta B) = F(x) - F(x) = 0$. Стало быть, в силу условия
		\[
			\forall x \in C(F) \quad F_n(x) = P\rsi{-\infty; x} \xrightarrow[n \to \infty]{} P\rsi{-\infty; x} = F(x)
		\]
		
		\item[$4 \Ra 2$] Воспользуемся третьим эквивалентным свойством из теоремы Александрова. Пусть $G \subseteq \R$ --- произвольное открытое множество. Тогда известно, что $G$ есть не более чем счётное число непересекающихся интервалов $G = \bscup_{k = 1}^\infty (a_k; b_k)$. Зафиксируем $\eps > 0$ и для любого $k \in \N$ подберём полуинтервал $\rsi{a'_k; b'_k} \subset (a_k; b_k)$ такой, что $P(a_k; b_k) \le P\rsi{a'_k; b'_k} + \eps / 2^k$ и дополнительно $a'_k, b'_k \in C(F)$. Такой выбор возможен, коль скоро $P$ непрерывна, а точек разрыва $F$ не более чем счётное число (в силу монотонности). Осталось явно проверить требуемое ($N$ --- произвольное натуральное число):
		\begin{multline*}
			\varliminf_{n \to \infty} P_n(G) = \varliminf_{n \to \infty} \sum_{k = 1}^\infty P_n(a_k; b_k) \ge
			\\
			\varliminf_{n \to \infty} \sum_{k = 1}^N P_N(a_k; b_k) \ge \sum_{k = 1}^N \varliminf_{n \to \infty} P_n(a_k; b_k) \ge
			\\
			\sum_{k = 1}^N \varliminf_{n \to \infty} P_n\rsi{a'_k; b'_k} = \sum_{k = 1}^N \varliminf_{n \to \infty} (F_n(b'_k) - F_n(a'_k)) =
			\\
			\sum_{k = 1}^N (F(b'_k) - F(a'_k)) \ge \sum_{k = 1}^N \ps{P(a_k; b_k) - \frac{\eps}{2^k}} \ge \sum_{k = 1}^N P(a_k; b_k) - \eps
		\end{multline*}
		Устремляя $N$ в бесконечность, а затем и $\eps$ к нулю, получаем требуемое.
	\end{itemize}
\end{proof}

\begin{corollary}
	Имеет место эквивалентность (которое часто берут за определение сходимости по распределению): $\xi_n \to^d \xi \Lra \forall x \in C(F)\ \ F_{\xi_n}(x) \to F_\xi(x)$
\end{corollary}

\begin{note}
	В чём состоит смысл сходимости по распределению? Это позволяет делать аппроксимацию этих самых распределений.
	
	Пусть мы хотим вычислить распределение случайной величины $\xi$. Тогда, пусть $\eta_n \to^d \eta$, где распределение $\eta$ известно. Если окажется, что начиная с некоторого $n_0$ распределения $\eta_n$ и $\xi$ равны (или сильно близки), то можно считать, что $F_\xi(x) \sim F_\eta(x)$
\end{note}

\begin{example}
	В курсе ОВиТМа уже встречалось это соображение в виде теоремы Пуассона.
\end{example}

\section{Характеристические функции}

\begin{definition}
	Пусть $\xi$ --- случайная величина. Тогда \textit{характеристической функцией случайной величины $\xi$} называется комплексная функция $\phi_\xi(t) \colon \R \to \Cm$:
	\[
		\forall t \in \R\ \ \phi_\xi(t) := \E e^{i\xi t}
	\]
\end{definition}

\begin{note}
	Интеграл (матожидание) от комплексной функции берётся по такой формуле:
	\[
		\E e^{it\xi} = \E\cos(t\xi) + i\E\sin(t\xi)
	\]
\end{note}

\begin{definition}
	Пусть $\xi \colon \R^n \to \R$ --- случайный вектор. Тогда \textit{характеристической функцией случайного вектора $\xi$} называется комплекснозначная функция $\phi_\xi \colon \R^n \to \R$:
	\[
		\forall t \in \R^n\ \ \phi_\xi(t) := \E e^{i\tbr{\xi, t}}
	\]
\end{definition}

\subsubsection*{Примеры характеристических функций}

\begin{enumerate}
	\item Пусть $\xi \sim Bin(n, p)$. Тогда 
	\[
		\phi_\xi(t) = \E e^{it\xi} = \sum_{k = 0}^n e^{itk}P(\xi = k) = \sum_{k = 0}^n e^{itk} C_n^k p^k(1 - p)^{n - k} = (e^{it}p + 1 - p)^n
	\]
	
	\item Пусть $\xi \sim Exp(\alpha)$. Тогда
	\[
		\phi_\xi(t) = \E e^{it\xi} = \int_0^{+\infty} e^{itx}\alpha e^{-\alpha x}dx = \alpha \int_0^{+\infty} e^{(it - \alpha)x}dx = \frac{\alpha}{\alpha - it}
	\]
	
	\item Пусть $\xi \sim N(0, 1)$. Тогда
	\[
		\phi_\xi(t) = \E e^{it\xi} = \int_{-\infty}^{+\infty} e^{itx} \frac{1}{\sqrt{2\pi}} e^{-x^2 / 2}dx = 
	\]
	Если мы разложим интеграл на действительную и мнимую часть, то заметим, что в мнимой части подыинтегральная функция будет нечётной, стало быть ноль. Итого:
	\[
		\phi_\xi(t) = \int_{-\infty}^{+\infty} \cos(tx) \frac{1}{\sqrt{2\pi}} e^{-x^2 / 2}dx
	\]
	Получился очень хорошо параметризованный интеграл (подыинтегральная функция бесконечно гладкая), поэтому будем считать его через производную:
	\[
		\phi'_\xi(t) = \frac{1}{\sqrt{2\pi}}\int_{-\infty}^{+\infty} \sin(tx) \underbrace{(-x) e^{-x^2 / 2}}_{(e^{-x^2 / 2})'}dx = \frac{1}{\sqrt{2\pi}} \ps{0 - t\int_{-\infty}^{+\infty} \cos(tx)e^{-x^2 / 2}dx} = -t\phi_\xi(t)
	\]
	Отсюда $\phi_\xi(t) = Ce^{-t^2 / 2}$. Начальное условие $\phi_\xi(0) = 1$, в итоге $\phi_\xi(t) = e^{-t^2 / 2}$
\end{enumerate}

\begin{definition}
	Пусть $P$ --- вероятностная мера на $(\R^m, \B(\R^m))$. Тогда \textit{характеристической функцией меры $P$} называется $\phi_P \colon \R^m \to \R$, определённая следующим образом:
	\[
		\forall t \in \R^m\ \ \phi_P(t) := \int_{\R^m} e^{i\tbr{t, x}}dP(x)
	\]
\end{definition}

\begin{note}
	Несложно заметить, что определения характеристических функций для меры и случайных векторов согласованы, а именно (в случае случайных величин):
	\[
		\text{Хар-ая функция $\xi$} = \E e^{it\xi} = \int_\R e^{itx} dP_\xi(x) = \text{ хар-ая функция распределения $\xi$}
	\]
\end{note}