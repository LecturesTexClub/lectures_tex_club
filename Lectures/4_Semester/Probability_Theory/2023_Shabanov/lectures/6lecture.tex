\begin{note}
	То же самое утверждение верно и для случайных векторов:
	\[
		\xi_1, \ldots, \xi_n \text{ --- независ. в совокуп.} \Longleftrightarrow \forall \vv{x}_1, \ldots, \vv{x}_n\ \ P(\xi_1 \le \vv{x}_1 \wedge \ldots \wedge \xi_n \vv{x}_n) = \prod_{k = 1}^n P(\xi_k \le \vv{x}_k)
	\]
\end{note}

\begin{lemma} (Сохранение независимости в совокупности у композиций)
	Пусть $\{\xi_k\}_{k = 1}^n$ --- независимые в совокупности случайные векторы, $\xi_i \in \R^{d_i}$, $d_i \in \N$, а также есть $f_i \colon \R^{d_i} \to \R^{m_i}$ --- борелевские функции. Положим $\eta_i = f_i(\xi_i)$. Тогда $\{\eta_k\}_{k = 1}^n$ являются независимыми в совокупности.
\end{lemma}

\begin{proof}
	На самом деле, нам просто надо воспользоваться всем доказанным без дополнительных выкладок. Действительно, $\{\eta_k\}_{k = 1}^n$ независимы в совокупности тогда и только тогда, когда $\F_{\eta_1}, \ldots, \F_{\eta_n}$ независимы в совокупности. Однако, так как $\eta_i = f_i(\xi_i)$, то $\F_{\eta_i} \subseteq \F_{\xi_i}$, откуда всё и следует.
\end{proof}

\begin{corollary}
	Рассмотрим $\{\xi_t\}_{t = 1}^{n_1 + \ldots + n_k}$ независимых в совокупности случайных величин, разбитых соответствующим образом на блоки (отдельно первые $n_1$, затем $n_2$ и так далее). Дополнительно рассмотрим набор из $k$ борелевских функций $f_i \colon \R^{n_i} \to \R$. Тогда случайные величины $f_1(\xi_1, \ldots, \xi_{n_1}), \ldots, f_k(\xi_{n_1 + \ldots + n_{k - 1} + 1}, \ldots, \xi_{n_1 + \ldots + n_k})$ тоже независимы в совокупности.
\end{corollary}

\begin{proof}
	Доказывается аналогично лемме, просто надо положить за $\xi_i$ случайный вектор, образованный блоком.
\end{proof}

\begin{theorem} (Дополнительное свойство математического ожидания)
	Пусть $\xi, \eta$ --- независимые случайные величины, причём $\E\xi, \E\eta$ существуют и конечны. Тогда $\E\xi\eta$ тоже существует и вычисляется следующим образом:
	\[
		\E\xi\eta = (\E\xi) \cdot \E\eta
	\]
\end{theorem}

\begin{proof}
	Так как матожидание является интегралом Лебега, то мы будем доказывать эту теорему по соответствующим этапам:
	\begin{enumerate}
		\item Пусть $\xi, \eta$ --- простые случайные величины, то есть
		\begin{align*}
			&{\xi = \sum_{t = 1}^n x_t \chi\{\xi = x_t\}}
			\\
			&{\eta = \sum_{l = 1}^m y_l \chi\{\eta = y_l\}}
		\end{align*}
		Честно перемножим эти случайные величины:
		\[
			\xi\eta = \sum_{t = 1}^n \sum_{l = 1}^m x_t y_l \chi\{\xi = x_t \wedge \eta = y_l\}
		\]
		Теперь вычисляем матожидание от частей:
		\begin{multline*}
			\E(\xi\eta) = \sum_{t = 1}^n \sum_{l = 1}^m x_t y_l P(\xi = x_t \wedge \eta = y_l) =
			\\
			\sum_{t = 1}^n \sum_{l = 1}^m P(\xi = x_t) \cdot P(\eta = y_l) =
			\\
			\ps{\sum_{t = 1}^n P(\xi = x_t)} \cdot \ps{\sum_{l = 1}^m P(\eta = y_l)} = (\E\xi) \cdot \E\eta
		\end{multline*}
		
		\item Теперь $\xi, \eta \ge 0$ --- неотрицательные случайные величины. Рассмотрим подходящие снизу последовательности неотрицательных простых случайных величин $\{\xi_n\}_{n = 1}^\infty$, \\ $\{\eta_n\}_{n = 1}^\infty$. Так как $0 \le \xi_n \le \xi$, то $\xi_n$ является $\F_\xi$-измеримой, аналогично с $\eta_n$. Из вышесказанного получается, что $\xi_n\eta_n \sua \xi\eta$, причём $\forall n \in \N\ \ \xi_n \indep \eta_n$. Теперь, мы просто воспользуемся определением математического ожидания:
		\[
			\E\xi\eta = \lim_{n \to \infty} \E\xi_n\eta_n = \lim_{n \to \infty} ((\E\xi_n) \cdot \E\eta_n) = \lim_{n \to \infty} \E\xi_n \cdot \lim_{n \to \infty} \E\eta_n = (\E\xi) \cdot \E\eta
		\]
		
		\item $\xi, \eta$ --- произвольные случайные величины. Тогда $\xi^{\pm}$ и $\eta^{\pm}$ --- независимы (как композиции с независимыми). Выясним, что есть знакопостоянная часть для $\xi\eta$:
		\[
			(\xi\eta)^+ = \xi^+ \eta^+ + \xi^-\eta^-;\ \ (\xi\eta)^- = \xi^-\eta^+ + \xi^+\eta^-
		\]
		По определению матожидания в этом случае:
		\[
			\E\xi\eta = \E(\xi\eta)^+ - \E(\xi\eta)^-
		\]
		Дальнейшие алгебраические действия тривиальны.
	\end{enumerate}
\end{proof}

\section{Замена переменных в интеграле Лебега}

\begin{note}
	В этой главе мы живём в вероятностном пространстве $(\Omega, \F, P)$.
\end{note}

\subsection{Картина мира}

\begin{note}
	$\xi$ --- это отображение из $(\Omega, \F, P)$ в $(\R, \B(\R))$. По $P$ мы получили $P_\xi$, а по $\B(\R)$ мы получили $\F_\xi$.
	\textcolor{red}{Сюда очень хочется простую картинку с 6й лекции, около 46 минуты}
	Кортеж $(\R, \B(\R), P_\xi)$ образует вероятностное пространство. Случайные величины на нём --- борелевские функции. Коль скоро у нас есть вероятностная мера, мы имеем право определить интеграл Лебега от произвольной борелевской функции $g$:
	\[
		\int_\R g(x)P_\xi(dx) = \int_\R g(x)dP_\xi
	\]
\end{note}

\subsection{Замена переменной в интеграле Лебега}

\begin{theorem} (о замене переменной в интеграле Лебега)
	Пусть $\xi$ --- случайный вектор из $\R^m$, а $P_\xi$ --- его распределение. Тогда для любой борелевской функции $g \colon \R^m \to \R$ выполнено соотношение:
	\[
		\E g(\xi) = \int_\Omega g(\xi)dP = \int_{\R^m} g(x) P_\xi(dx)
 	\]
\end{theorem}

\begin{proof}
	Снова придётся помучаться с разбором случаев из построения интеграла:
	\begin{enumerate}
		\item Пусть $g(x) = \chi_A(x)$, где $A \in \B(\R^m)$. Тогда
		\[
			\E g(\xi) = \E \chi_{\xi \in A} = P(\xi \in A) = P_\xi(A) = \int_A P_\xi(dx) = \int_{\R^m} \chi_A(x)P_\xi(dx) = \int_{\R^m} g(x)P_\xi(dx)
		\]
		
		\item $g$ --- простая борелевская функция. Так как $g$ есть линейная комбинация индикаторов, а равенство из теоремы линейно по $g$, то всё доказано автоматически по первому пункту.
		
		\item Если $g \ge 0$, то рассмотрим последовательность неотрицательных простых функций $g_n \sua g$. По определению математического ожидания, $\E g(\xi) = \lim_{n \to \infty} \E g_n(\xi)$. Если мы заменим матожидание простой функции на соответствующий интеграл из теоремы, то такая последовательность сходится к интегралу, в котором $g_n$ заменена на $g$. В силу единственности предела получаем требуемое равенство.
		
		\item Если $g$ --- произвольная борелевская функция, то раскладываем её на $g = g^+ - g^-$ и пользуемся линейностью матожидания.
	\end{enumerate}
\end{proof}

\begin{corollary}
	\begin{enumerate}
		\item Для вычисления математического ожидания $\E g(\xi)$ не нужно знать $\xi$ в явном виде, достаточно знать только распределение $P_\xi$.
		
		\item Если распределения $\xi, \eta$ совпадают, то для любой борелевской функции $g$ будет выполнено равенство, что $\E g(\xi) = \E g(\eta)$.
		
		\item Пусть $\xi$ --- случайная величина, тогда
		\[
			\E\xi = \int_\R xP_\xi(dx)
		\]
	\end{enumerate}
\end{corollary}

\begin{note}
	А как посчитать интеграл $\int_\R g(x)P_\xi(dx)$? Мы знаем теорему Лебега о разложении функции распределения на 3 части, но даже там с сингулярной составляющей без случайного чуда ничего не сделать.
\end{note}

\begin{definition}
	Пусть $P$ --- вероятностная мера на $(\R^m, \B(\R^m))$, а $\mu$ --- это $\sigma$-конечная мера на $(\R^m, \B(\R^m))$. Тогда, функция $p(t) \ge 0$ называется \textit{обобщённой плотностью меры $P$ по мере $\mu$}, если выполнено равенство:
	\[
		\forall B \in \B(\R^m)\ \ P(B) = \int_B p(t)\mu(dt)
	\]
\end{definition}

\begin{example}
	Приведём некоторые примеры $\sigma$-конечных мер:
	\begin{itemize}
		\item $\mu$ --- мера Лебега на $\R$. Тогда для абсолютно непрерывного распределения $P$ у нас $p(t)$ --- это просто плотность $P$.
		
		\item $\mu$ --- считающая мера на $X$, где $X$ --- не более чем счётное множество. Она устроена так:
		\begin{align*}
			&{\forall x \in X\ \ \mu(\{x\}) = 1}
			\\
			&{\mu(\R \bs X) = 0}
		\end{align*}
		Так как $P$ дискретна, то $p(t) = P(\{t\})$. Тогда $P(B)$ запишется следующим образом:
		\[
			P(B) = \sum_{t \in X \cap B} P(\{t\}) = \int_B p(t)\mu(dt)
		\]
	\end{itemize}
\end{example}