\begin{note}
    Теорема о единственности говорит, что характеристическая функция однозначно определяет распределение случайной величины. Явный вид этого распределения можно получить через формулу обращения.
\end{note}

\begin{theorem} (Формула обращения, без доказательства)
    Пусть $\phi(t)$ --- характеристическая функция случайной величины $\xi$ с функцией распределения $F_\xi$. Тогда:
    \begin{enumerate}
        \item $\forall a < b$, $a, b \in C(F_\xi)$ выполнено равенство:
        \[
            F_\xi(b) - F_\xi(a) = \frac{1}{2\pi} \lim_{c \to +\infty} \int_{-c}^{c}
            \frac{e^{-ita} - e^{-itb}}{it} \phi(t) dt
        \]
        \item Если $\int_\R |\phi(t)| dt < +\infty$, то случайная величина $\xi$ имеет плотность
        \[
            p(x) = \frac{1}{2\pi} \int_\R e^{-itx} \phi(t) dt
        \]
    \end{enumerate}
\end{theorem}

\begin{note}
    Внимательный читатель увидит здесь прямое и обратное преобразование Фурье. Теперь же попробуем ответить на вопрос: <<Как выяснить, является ли функция характеристической (для некоторой случайной величины)?>>
\end{note}

\begin{definition}
    Функция $f: \R \to \Cm$ называется \textit{неотрицательно определённой}, если выполнено неравенство:
    \[
        \forall n \in \N \ \ \forall t_1, \ldots, t_n \in \R \ \ \forall z_1, \ldots, z_n \in \Cm \ \
        \sum_{i, j = 1}^n f(t_i - t_j) z_i \ol{z_j} \ge 0
    \]
    Это утверждение можно записать более просто на языке алгебры:
    \[
    	\forall n \in \N\ \forall t_1, \ldots, t_n \in \R\ \forall \vv{z} \in \Cm^n \quad \vv{z} \Matrix{
    		f(t_1 - t_1) & \cdots & f(t_1 - t_n) \\
   			\vdots & \ddots & \vdots \\
   			f(t_n - t_1) & \cdots & f(t_n - t_n)
   		}
   		\vv{z}^* \ge 0
    \]
    где $\Cm^n$ надо рассматривать как линейное пространство кортежей (то есть строк), а операция $*$ - это композиция транспонирования и сопряжения.
\end{definition}

\begin{theorem} (Бохнера-Хинчина)
    Пусть $\phi: \R \to \Cm$ такая, что $\phi(0)=1$ и $\phi$ непрерывна в нуле. Тогда следующие утверждения эквивалентны:
    \begin{enumerate}
        \item $\phi$ является характеристической функцией некоторого распределения
        \item $\phi$ неотрицательно определена
    \end{enumerate}
\end{theorem}

\begin{proof}~
    \begin{itemize}
        \item[$1 \Ra 2$] Пусть $\phi$ является характеристической функцией распределения $P$. Тогда:
        \[
            \phi(t) = \int_\R e^{itx} dP(x)
        \]
        Нужно проверить определение неотрицательной определённости. Запишем сумму оттуда в другом виде:
        \begin{multline*}
            \sum_{k, l = 1}^n \phi(t_k - t_l) z_k \ol{z_l} = \sum_{k, l = 1}^n \ps{\int_\R e^{i(t_k - t_l)x} dP(x)} z_k \ol{z_l} =
            \\
            \int_\R \ps{\sum_{k, l = 1}^n e^{it_kx} z_k \ol{e^{it_lx} z_l}} dP(x) =
            \int_\R \ps{\sum_{k=1}^n e^{it_kx}z_k} \ol{ \ps{\sum_{l=1}^n e^{it_lx}z_l}} dP(x) =
            \\
            \int_\R \md{\sum_{k=1}^n e^{it_kx}z_k}^2 dP(x) \ge 0
        \end{multline*}
        
        \item[$2 \Ra 1$] Без доказательства.
    \end{itemize}
\end{proof}

\section{Метод характеристических функций}

\begin{note}
    Один из способов применения характеристических функций --- теорема о единственности --- о том, что характеристическая функция однозначно определяет распределение. Оказывается, что зависимость характеристических функций и распределений не только биективна, но и в некотором смысле непрерывна: организуя сходимость характеристических функций, получаем сходимость распределений.
\end{note}

\subsection{Теорема Прохорова}

\begin{note}
    Под $\gA$ снова нужно понимать произвольное множество индексов.
\end{note}

\begin{definition}
    Пусть $\{P_\alpha\}_{\alpha \in \gA}$ --- семейство распределений на $(\R^m, \B(\R^m))$.
    Семейство $\{P_\alpha\}_{\alpha \in \gA}$ называется \textit{относительно компактным}, если из любой последовательности $\{P_{\alpha_n}\}_{n = 1}^\infty \subseteq \{P_\alpha\}_{\alpha \in \gA}$ можно выбрать слабо сходящуюся подпоследовательность.
\end{definition}

\begin{definition}
    Пусть $\{P_\alpha\}_{\alpha \in \gA}$ --- семейство распределений на $(\R^m, \B(\R^m))$.
    Семейство $\{P_\alpha\}_{\alpha \in \gA}$ называется \textit{плотным}, если выполнено условие:
    \[
        \forall \eps > 0 \ \ \exists K \subseteq \R^m \text{ - компакт} \such \sup_{\alpha \in \gA} P_\alpha(\R^m \setminus K) \le \eps
    \]
\end{definition}

\begin{note}
	Другими словами, семейство мер будет плотным, если найдётся компакт, на котором все меры принимают значение хотя бы $1 - \eps$.
\end{note}

\begin{note}
    Относительная компактность --- более интересное свойство, но плотность проще проверить. Оказывается, что они эквивалентны.
\end{note}

\begin{theorem} (Прохорова)
    Семество распределений относительно компактно тогда и только тогда, когда оно плотно.
\end{theorem}

\begin{proof}
    Докажем для случая $m=1$, в общем случае доказывается аналогично, но надо чуть больше повозиться.
    \begin{itemize}
        \item[$\Ra$] Предположим противное. Пусть семейство распределений $\{P_\alpha\}_{\alpha \in \gA}$ относительно компактно и не плотно. Второй факт означает следующее:
        \[
            \exists \eps > 0 \ \ \forall K \text{ --- компакт в } \R \ \
            \sup_{\alpha \in \gA} P_\alpha(\R \setminus K) > \eps
        \]

        Зафиксируем $\eps > 0$. Рассмотрим последовательность компактов $[-n; n] \subset \R, n \in \N$ и выберем для для неё соответствующую последовательность вероятностных мер по сказанному выше:
        \[
            \forall n \in \N\ \alpha_n \in \gA \wedge P_{\alpha_n}(\R \setminus [-n, n]) > \eps
        \]

        $\{P_\alpha\}_{\alpha \in \gA}$ относительно компактно, тогда из $\{P_{\alpha_n}\}_{n = 1}^\infty$ можно выбрать слабо сходящуюся подпоследовательность:
        \[
            \{P_{\alpha_{n_k}}\}_{k = 1}^\infty \subseteq \{P_{\alpha_n}\}_{n = 1}^\infty \such
            P_{\alpha_{n_k}} \to^w Q, \ Q \text{ --- вероятностная мера}
        \]
        По теореме Александрова:
        \[
            P_{\alpha_{n_k}} \to^w Q \Leftrightarrow
            \varlimsup_{k \to \infty} P_{\alpha_{n_k}}(A) \le Q(A)
            \text{ для любого замкнутого } A \subseteq \R
        \]
        Дальше идея состоит в том, что за счёт неравенств с $\eps$ и предела мы покажем, что на самом деле $\eps = 0$, чего не может быть. Итак, за счёт неравенств с $\eps$ верно следующее:
        \[
        	\forall k \in \N\ \ \eps \le P_{\alpha_{n_k}}(\R \bs [-n_k; n_k]) \Lora \eps \le \varlimsup_{k \to \infty} P_{\alpha_{n_k}}(\R \bs [-n_k; n_k])
        \]
        Найдём замкнутое множество под теорему Александрова. Например, подойдёт $\R \bs (-n; n)$ с любым $n \in \N$, ибо всегда найдётся $k_0 \in \N$ с которого будет вложение $\R \bs [-n_k; n_k] \subseteq \R \bs (-n; n)$:
        \begin{multline*}
        	\forall n \in \N\ \exists k_0 \in \N \such \forall k \ge k_0\ \R \bs [-n_k; n_k] \subseteq \R \bs (-n; n) \Ra
        	\\
        	\forall n \in \N\ \ \varlimsup_{k \to \infty} P_{\alpha_{n_k}}(\R \bs [-n_k; n_k]) \le \varlimsup_{k \to \infty} P_{\alpha_{n_k}}(\R \bs (-n; n)) \le Q(\R \bs (-n; n))
        \end{multline*}
        С учётом всего, получаем:
        \[
        	\forall n \in \N\ \eps \le Q(\R \bs (-n; n)) \Lora 0 < \eps \le \lim_{n \to \infty} Q(\R \bs (-n; n)) = 0
        \]
        Получили искомое противоречие.

        \item[$\La$] Пусть $\{P_{\alpha_n}\}_{n = 1}^\infty$ --- последовательность в плотном семействе $\{P_\alpha\}_{\alpha \in \gA}$. Нужно извлечь из неё слабо сходящуюся подпоследовательность. Обозначим $F_n$ --- функции распределения вероятностных мер $P_{\alpha_n}$. В силу теоремы об эквивалентности сходимостей, мы будем искать такую функцию $G$, что выполнена сходимость в основном для функций распределения (если найдём, то теорема автоматически доказана):
        \[
            \forall x \in C(G)\ \ G(x) = \lim_{k \to \infty} F_{n_k}(x)
        \]
        Нам понадобится счётное всюду плотное множество в $\R$ --- возьмём $\Q$. Занумеруем все рациональные числа $\Q = \{q_1, q_2, \ldots\}$. Теперь рассмотрим последовательность $\{F_n(q_1),\ n \in \N\}$. Функции распределения принимают значения на $[0, 1]$, поэтому эта последовательность ограничена. Тогда из неё можно выбрать сходящуюся подпоследовательность $\{n_{1, t}\}_{t = 1}^\infty$, $\exists \lim_{t \to \infty} F_{1, t}(q_1)$. Аналогично берём построенную последовательность и выделяем в ней подполедовательность $\{n_{2, t}\}_{t = 1}^\infty \subseteq \{n_{1, t}\}_{t = 1}^\infty$, для которой сойдётся предел $\lim_{t \to \infty} F_{2, t}(q_2)$, и так далее. Наличие этих последовательностей позволяет рассмотреть диагональную последовательность (по сути диагональ Кантора) $\{n_{t, t}\}_{t = 1}^\infty$. Для неё верно уже более сильное утверждение:
        \[
        	\forall m \in \N\ \ \exists \lim_{t \to \infty} F_{n_{t, t}}(q_m)
        \]
        Введём теперь $G$, но пока на рациональных числах. По этой причине обозначим её $G_\Q$:
        \[
            \forall x \in \Q\ \ G_{\Q}(x) := \lim_{t \to \infty} F_{n_{t, t}}(x)
        \]

        Заметим, что $G_{\Q}$ неубывает по построению. Действительно, для $G_\Q$ это просто предельный переход из следующего факта:
        \[
        	\forall x, y \in \Q, x < y\ \forall t \in \N\ \ F_{n_{t, t}}(x) \le F_{n_{t, t}}(y)
        \]
		Теперь мы можем продолжить $G_\Q$ до искомой $G$:
        \[
            \forall x \in \R\ \ G(x) := \inf_{y \in \Q \colon x < y} G_{\Q}(y)
        \]
        Утверждается, что $G$ --- искомая функция распределения, а $F_{n_{t, t}}$ --- искомая подпоследовательность функций распределения, сходящаяся к $G$ в основном. Внимательный читатель мог догадаться, что $G$ определена подобным образом именно для непрерывности справа. Так или иначе, мы проверим все свойства явно:
        \begin{itemize}
        	\item $G$ монотонна. Это тривиально по определению
        	
        	\item $G$ непрерывна справа. Так как монотонность уже имеется, рассмотрим монотонно подходящую к $x$ последовательность точек $x_k \sda x$. За счёт той же монотонности, мы уже имеем неравенство $\lim_{k \to \infty} G(x_k) \ge G(x)$. Предположим, что оно оказалось строгим. Тогда
        	\[
        		\exists y \in \Q \such x < y \wedge G_\Q(y) < \lim_{k \to \infty} G(x_k)
        	\]
        	Мы также знаем, что $\lim_{k \to \infty} x_k = x < y$, то есть существует $k_0 \in \N$ такое, что $\forall k \ge k_0\ x_k < y$. Но тогда $G(x_k) \le G_\Q(y)$. Отсюда противоречие:
        	\[
        		G_\Q(y) < \lim_{k \to \infty} G(x_k) \le G_\Q(y)
        	\]
        	
        	\item Имеют место пределы к 0 и 1 на соответствующих бесконечностях. Заметим, что до этого мы ни разу не воспользовались плотностью наших мер. По определению это даёт нам такой факт:
        	\[
        		\forall \eps > 0\ \exists K \subseteq \R^m \text{ --- компакт} \such \inf_{\alpha \in \gA} P_\alpha(K) \ge 1 - \eps
        	\]
        	Зафиксируем $\eps > 0$ и компакт $K$. Так как $G$ монотонна, то она имеет не более чем счётное число точек разрыва, а значит можно выбрать её точки непрерывности $a < b$ так, что $K \subseteq \rsi{a; b}$. Тогда разность значений на концах полуинтервала можно оценить следующим образом:
        	\begin{multline*}
        		G(b) - G(a) = \lim_{t \to \infty} F_{n_{t, t}}(b) - \lim_{t \to \infty} F_{n_{t, t}}(a) = \lim_{t \to \infty} (F_{n_{t, t}}(b) - F_{n_{t, t}}(a)) =
        		\\
        		\lim_{t \to \infty} P_{\alpha_{n_{t, t}}} \rsi{a; b} = \varliminf_{t \to \infty} P_{\alpha_{n_{t, t}}} \rsi{a; b} \ge \varliminf_{t \to \infty} P_{\alpha_{n_{t, t}}}(K) \ge 1 - \eps
        	\end{multline*}
        	Значения $F_{n_{t, t}} \in [0; 1]$, значит и у $G_\Q$ они заключены в $[0; 1]$, а это уже даёт аналогичное ограничение на значения $G$. С учётом монотонности $G$ и последнего неравенства, остаётся только одна возможность:
        	\[
        		\lim_{x \to +\infty} G(x) = 1;\ \ \lim_{x \to -\infty} G(x) = 0
        	\]
        \end{itemize}
    	Нам осталось проверить, что во всех точках непрерывности $x \in C(G)$ выполнено предельное равенство, о котором мы оговорились в самом начале пути. Итак, пусть $x_0 \in C(G)$. Покажем, что $G(x_0) = \lim_{t \to \infty} F_{n_{t, t}}(x_0)$. Будем разбираться с верхним и нижним пределом отдельно:
    	\begin{itemize}
    		\item Для верхнего предела достаточно рассмотреть $y \in \Q, x_0 < y$:
    		\[
    			\varlimsup_{t \to \infty} F_{n_{t, t}}(x_0) \le \varlimsup_{t \to \infty} F_{n_{t, t}}(y) = \lim_{t \to \infty} F_{n_{t, t}}(y) = G_\Q(y)
    		\]
    		Так как это верно для любого такого $y$, то мы можем обобщить утверждение:
    		\[
    			\varlimsup_{t \to \infty} F_{n_{t, t}}(x_0) \le \inf_{y \in \Q \colon x_0 < y} G_\Q(y) = G(x_0)
    		\]
    		
    		\item Теперь возьмём $x_1 \in \R$, $y \in \Q$ такие, что $x_1 < y < x_0$. С учётом уже доказанного можем написать такое неравенство:
    		\[
    			G(x_1) \le G_\Q(y) = \lim_{t \to \infty} F_{n_{t, t}}(y) = \varliminf_{t \to \infty} F_{n_{t, t}}(y) \le \varliminf_{t \to \infty} F_{n_{t, t}}(x_0)
    		\]
    		Так как $x_1$, опять же, произвольный, то мы вправе устремить его к $x_0$. Тогда выходит неравенство $G(x_0 - 0) \le \varliminf_{t \to \infty} F_{n_{t, t}}(x_0)$.
    	\end{itemize}
    	Если вспомнить, что $x_0 \in C(G)$, то мы всё и доказали:
    	\[
    		G(x_0) = G(x_0 - 0) \le \varliminf_{t \to \infty} F_{n_{t, t}}(x_0) \le \varlimsup_{t \to \infty} F_{n_{t, t}}(x_0) \le G(x_0) \Lora G(x_0) = \lim_{t \to \infty} F_{n_{t, t}}(x_0)
    	\]
    \end{itemize}
\end{proof}

\subsection{Теорема о непрерывности. Центральная предельная теорема}

\begin{lemma} (Первая лемма)
    Пусть $\{P_n\}_{n = 1}^\infty$ --- последовательность распределений в $\R^m$. Если она плотная и любая её слабосходящаяся подпоследовательность слабо сходится к одной и той же вероятностной мере $Q$, то $P_n \xrightarrow{w} Q$.
\end{lemma}

\begin{proof}
    Семейство $\{P_n\}_{n = 1}^\infty$ --- плотное, а значит по теореме Прохорова оно относительно компактное. Следовательно, из любой его последовательности можно выделить слабо сходящуюся подпоследовательность. Предположим, что $P_n \centernot{\xrightarrow{w}} Q$. Если записать это утверждение через определение, то получится следующее:
    \begin{multline*}
        \exists f \colon \R^m \to \R \text{ --- непрерывная ограниченная }
        \\
        \exists \eps > 0 \ \
        \exists \{n_k\}_{k = 1}^\infty \subseteq \N\ \ \md{\int_{\R^m} f(x) dP_{n_k}(x) - \int_{\R^m} f(x) dP(x)} \ge \eps
    \end{multline*}

    В силу относительной компактности выберем из $\{n_k\}_{k = 1}^\infty$ слабо сходящуюся подпоследовательность $\{n_{k_l}\}_{l \in \N}$. Это означает, что $P_{n_{k_l}} \xrightarrow{w} Q$. В частности, для функции $f$:
    \[
        \lim_{l \to \infty} \int_{\R^m} f(x) dP_{n_{k_l}}(x) = \int_{\R^m} f(x) dP(x)
    \]
    В итоге получаем противоречие:
    \[
        0 < \eps \le \md{\int_{\R^m} f(x) dP_{n_{k_l}}(x) - \int_{\R^m} f(x) dP(x)} \xrightarrow[l \to \infty]{} 0
    \]
\end{proof}

\begin{lemma} (Вторая лемма)
    Пусть $\{P_n\}_{n = 1}^\infty$ --- последовательность распределений на $\R$, $\{\phi_n\}_{n = 1}^\infty$ --- соответствующая последовательность характеристических функций. Если $\{P_n\}_{n = 1}^\infty$ --- плотная, то верна эквивалентность:
    \[
        P_n \text{ слабо сходится} \Leftrightarrow \forall t \in \R \ \ \exists \lim_n \phi_n(t)
    \]
\end{lemma}

\begin{proof}~
    \begin{itemize}
        \item[$\Ra$] $P_n \xrightarrow{w} Q$, где $Q$ --- вероятностная мера на $\R$. То есть:
        \[
            \forall f \colon \R \to \R \text{ --- непрерывная ограниченная } \int_\R f(x) dP_n(x) \xrightarrow[n \to \infty]{} \int_\R f(x)dQ(x)
        \]
        Возьмём в качестве $f(x)$ функции $\cos(tx)$, $\sin(tx)$ при фиксированном $t \in \R$. Получим:
        \[
        	\System{
        		&{\int_\R \cos(tx) dP_n(x) \xrightarrow[n \to \infty]{} \int_\R \cos(tx)dQ(x)}
        		\\
        		&{\int_\R \sin(tx) dP_n(x) \xrightarrow[n \to \infty]{} \int_\R \sin(tx)dQ(x)}
        	}
        	\Ra
        	\int_\R e^{itx} dP_n(x) \xrightarrow[n \to \infty]{} \int_\R e^{itx}dQ(x)
        \]
		Последнее в точности означает сходимость характеристических функций:
        \[
            \forall t \in \R\ \ \phi_n(t) \xrightarrow[n \to \infty]{} \phi(t)
        \]
        где $\phi(t)$ --- характеристическая функция вероятностной меры $Q$.

        \item[$\La$] Обозначим $\phi(t) := \lim_{n \to \infty} \phi_n(t)$. Рассмотрим в $\{P_n\}_{n = 1}^\infty$ произвольную слабо сходящуюся подпоследовательность $\{P_{n_k}\}_{k = 1}^\infty$, $P_{n_k} \xrightarrow{w} Q$. В силу уже доказанной части, верно следующее утверждение:
        \[
            \phi(t) = \lim_{k \to \infty} \phi_{n_k}(t) = \psi(t), \text{ где } \psi \text{ --- характеристическая функция } Q
        \]

        Значит, $\phi$ --- характеристическая функция $Q$. Тогда, в силу теоремы о единственности для характеристических функций, все слабо сходящиеся подпоследовательности последовательности $\{P_n\}_{n = 1}^\infty$ имеют один и тот же предел $Q$. Тогда выполнены условия первой леммы, следовательно $P_n \xrightarrow{w} Q$.
    \end{itemize}
\end{proof}

\begin{note}
    В частности, мы доказали, что если в условиях теоремы $P_n \xrightarrow{w} Q$, $\phi(t) = \lim_{n \to \infty} \phi_n(t)$, то $\phi$ --- характеристическая функция вероятностной меры $Q$.
\end{note}

\begin{note}
    В первых двух леммах активно пользовались плотностью семейства распределений. Пора научиться проверять, что у семейства она есть.
\end{note}

\begin{lemma} (Третья лемма)
    Пусть $\phi(t)$ --- характеристическая функция вероятностной меры $P$ на $\R$. Тогда
    \[
        \forall a > 0\ \ P\ps{\R \setminus \sbr{-\frac{1}{a}, \frac{1}{a}}} \le \frac{7}{a} \int_0^a (1 - \re\phi(t))dt
    \]
\end{lemma}

\begin{proof}
    Сначала отметим, что $\phi$ непрерывна. Стало быть, $1 - \re\phi(t)$ тоже непрерывна, то есть на отрезке $[0, a]$ интегрируема как по Риману, так и по Лебегу. Вся идея состоит в том, чтобы просто оценить интеграл нужной вероятностью снизу:
    \begin{multline*}
        \frac{1}{a} \int_0^a (1 - \re\phi(t))dt = \frac{1}{a} \int_0^a \ps{1 - \re \int_\R e^{ixt}dP(x)}dt =
        \\
        \frac{1}{a} \int_0^a \ps{1 - \int_\R \cos(xt)dP(x)}dt = [1] = \frac{1}{a} \int_0^a \int_\R (1 - \cos(xt)) dP(x) dt = [2] =
        \\
        \frac{1}{a} \int_\R \int_0^a (1 - \cos(xt))dt dP(x) = [3] \ge 
        \frac{1}{a} \int_{\R \bs \sbr{ -\frac{1}{a}, \frac{1}{a}}} \int_0^a (1 - \cos(xt))dt dP(x) =
        \\
        \frac{1}{a} \int_{\R \bs \sbr{  -\frac{1}{a}, \frac{1}{a}}} \ps{a - \frac{\sin(xt)}{x} \bigg|_0^a}dP(x) = \int_{\R \bs \sbr{-\frac{1}{a}, \frac{1}{a}}} \ps{1 - \frac{\sin(ax)}{ax}}dP(x)
    \end{multline*}
    Обоснуем все отмеченные переходы:
    \begin{enumerate}
    	\item Просто воспользовались свойством меры $\int_\R dP(x) = P(\R) = 1$
    	
    	\item Если коротко, то пользуемся теоремой Фубини. Если формально, то нужно знать, что теорема всё выводит из существования конечного интеграла по произведению мер, а здесь просто повторный интеграл. Поэтому нужно сослаться на теорему Тонелли (формулировка и доказательство есть, например, в книге Богачёв В.И., Смолянов О.Г. <<Действительный и функциональный анализ>>). Действительно, $1 - \cos(xt)$ --- неотрицательная измеримая относительно произведения вероятностной меры и меры Лебега функция, так как обе рассматриваются на борелевской $\sigma$-алгебре в $\R$. Повторный интеграл конечен, стало быть существует и конечен интеграл по произведению мер, и только после этого применяется теорема Фубини. Ну и ещё стоит отметить, что мы сделали несколько неявных переходов от интеграла Римана к интегралу Лебега и обратно.
    	
    	\item Косинус --- чётная функция, а максимальное по модулю значение, которое он принимает под интегралами на отрезке $\sbr{-\frac{1}{a}; \frac{1}{a}}$, равно $\cos(1) < 1$, то есть это будет заведомо положительная часть интеграла, которую мы можем выкинуть
    \end{enumerate}
    Из курса математического анализа мы теперь возьмём три факта:
    \begin{align*}
        &{\forall x \in \R\ \ \frac{\sin x}{x} \le 1}
        \\
        &{\forall x \in \R, |x| \ge 1\ \ \frac{\sin x}{x} \le \frac{\sin 1}{1} = \sin 1}
        \\
        &{1 - \sin 1 \ge \frac{1}{7}}
    \end{align*}
    С учётом этого можно продолжить цепочку неравенств:
    \begin{multline*}
        \frac{1}{a} \int_0^a (1 - \re\phi(t))dt \ge \int_{\R \bs \sbr{-\frac{1}{a}, \frac{1}{a}}} \ps{1 - \frac{\sin(ax)}{ax}} dP(x) \ge
        \\
        \inf_{|x| \ge \frac{1}{a}} \ps{1 - \frac{\sin(ax)}{ax}} P\ps{\R \bs \sbr{-\frac{1}{a}; \frac{1}{a}}} = \inf_{|x| \ge 1} \ps{1 - \frac{\sin x}{x}} P\ps{\R \bs \sbr{-\frac{1}{a}; \frac{1}{a}}} =
        \\
        = (1 - \sin 1) P\ps{\R \bs \sbr{-\frac{1}{a}; \frac{1}{a}}} \ge \frac{1}{7} P\ps{\R \bs \sbr{-\frac{1}{a}; \frac{1}{a}}}
    \end{multline*}
\end{proof}

\begin{theorem} (о непрерывности для характеристических функций)
    Пусть $\{P_n\}_{n = 1}^\infty$ --- последовательность распределений на $\R$, $\{\phi_n\}_{n = 1}^\infty$ --- соответствующая последовательность характеристических функций. Имеют место следующие утверждения:
    \begin{enumerate}
        \item Если $P_n \xrightarrow{w} P$, то $\forall t \in \R\ \exists \lim_{n \to \infty} \phi_n(t) = \phi(t)$, причём $\phi$ --- характеристическая функция меры $P$
        \item Пусть $\forall t \in \R\ \exists \lim_{n \to \infty} \phi_n(t) = \phi(t)$, где $\phi(t)$ непрерывна в нуле. Тогда $\phi(t)$ является характеристической функцией некоторой меры $P$, причём $P_n \xrightarrow{w} P$
    \end{enumerate}
\end{theorem}

\begin{proof}~
    \begin{enumerate}
        \item В точности повторяет доказательство следствия слева направо из второй леммы.
        \item Если докажем, что $\{P_n\}_{n = 1}^\infty$ --- плотная последовательность распределений, то вторая лемма и замечание к ней полностью завершат доказательство этого пункта. Итак, применим третью лемму для характеристических функций $\phi_n$ вероятностных мер $P_n$. Получим:
        \[
            \forall a > 0\ \ P_n \ps{\R \bs \sbr{-\frac{1}{a}, \frac{1}{a}}} \le \frac{7}{a} \int_0^a (1 - \re\phi_n(t)) dt
        \]
        Коль скоро $\phi_n \to \phi$, то и $(1 - \re\phi_n) \to (1 - \re\phi)$, причём все функции мажорируются константой 2 из оценки значений характеристической функции. Тогда можно применить теорему Лебега:
        \[
            \frac{7}{a} \int_0^a (1 - \re\phi_n(t)) dt \xrightarrow[n \to \infty]{} \frac{7}{a} \int_0^a (1 - \re\phi(t))dt
        \]
        В серии неравенств выше перейдем к пределу:
        \[
            \forall a > 0\ \forall n \in \N\ \ P_n \ps{\R \bs \sbr{-\frac{1}{a}, \frac{1}{a}}} \le \frac{7}{a} \int_0^a (1 - \re\phi(t))dt
        \]
        Покажем, что интеграл на самом деле можно оценить произвольным $\eps$ при подходящем $a$:
        \begin{enumerate}
        	\item $\phi(0) = \lim_{n \to \infty} \phi_n(0) = \lim_{n \to \infty} 1 = 1$, коль скоро $\phi_n$ уже характеристические функции по условию
        	
        	\item $\lim_{t \to 0} \re\phi(t) = 1$, опять же за счёт непрерывности в нуле функции $\phi$ по условию. По определению это означает следующее:
        	\[
        		\forall \eps > 0\ \exists a > 0 \such \forall t, |t| \le a\ \ 1 - \re\phi(t) = |1 - \re\phi(t)| \le \frac{\eps}{7}
        	\]
        	
        	\item За счёт утверждения последнего пункта, мы можем дать оценку на интеграл:
        	\[
        		\forall \eps > 0\ \exists a > 0 \such \frac{7}{a} \int_0^a (1 - \re\phi(t))dt = \frac{7}{a} \int_0^a |1 - \re\phi(t)|dt \le \frac{7}{a} \cdot a \cdot \frac{\eps}{7} = \eps
        	\]
        \end{enumerate}
        Таким образом, мы получили следующий факт:
        \[
            \forall \eps > 0\ \exists a > 0\ \forall n \in \N \ \ P_n\ps{\R \bs \sbr{-\frac{1}{a}, \frac{1}{a}}} \le \eps
        \]
        Это в точности и есть определение плотности последовательности распределений $\{P_n\}_{n = 1}^\infty$.
    \end{enumerate}
\end{proof}

\begin{theorem} (Центральная Предельная Теорема)
    Пусть $\{\xi_n\}_{n = 1}^\infty$ независимые одинаково распределённые случайные величины, а также $\E\xi_1 \neq \infty$ и $0 < D\xi_1 < \infty$. Тогда, если обозначить $S_n = \xi_1 \plusdots \xi_n$, имеет место сходимость:
    \[
    	\frac{S_n - \E S_n}{\sqrt{DS_n}} \xrightarrow[n \to \infty]{d} N(0, 1)
    \]
    Если обозначить за $a = \E\xi_1$ и $\sigma^2 = D\xi_1$, то это утверждение можно записать таким образом:
    \[
    	\frac{S_n - an}{\sigma\sqrt{n}} \xrightarrow[n \to \infty]{d} N(0, 1)
    \]
\end{theorem}

\begin{proof}
	Обозначим за $T_n$ элемент последовательности, для которой доказываем сходимость:
	\[
		T_n = \frac{S_n - \E S_n}{\sqrt{DS_n}} = \frac{S_n - an}{\sigma \sqrt{n}}
	\]
	Вспомним, что есть тривиальная эквивалентность сходимостей:
	\[
		T_n \xrightarrow{d} T \Lra F_{T_n} \xrightarrow{w} F_T \Lra P_{T_n} \xrightarrow{w} P_T
	\]
	Характеристическая функция для $N(0, 1)$ известна, это $e^{-x^2 / 2}$. По теореме о непрерывности харфункций достаточно доказать, что выполнено утверждение:
	\[
		\forall t \in \R\ \ \exists \lim_{n \to \infty} \phi_{T_n}(t) = e^{-t^2 / 2}
	\]
	Итак, отнормируем наши $\xi$, чтобы их матожидание было в нуле, а дисперсия равнялась единице: $\eta_j = (\xi_j - a) / \sigma$. Так как $a$, $\sigma$ одинаковы при всех $j$, а также преобразование $\xi_j$ является борелевской функцией, то $\eta_j$ тоже независимые одинаково распределённые случайные величины. Тогда $T_n$ выражается следующим образом:
	\[
		T_n = \frac{S_n - an}{\sigma\sqrt{n}} = \frac{\eta_1 \plusdots \eta_n}{\sqrt{n}}
	\]
	С новыми случайными величинами намного проще посчитать работать с видом харфункции $T_n$. Дальше вся идея состоит в том, чтобы воспользоваться определённой одинаковостью всех случайных величин и вспомнить следствие из теоремы о производных для характеристических функций (про разложение с о-малым):
	\begin{multline*}
		\phi_{T_n}(t) = \E e^{iT_n t} = \E \exp\ps{i\frac{t}{\sqrt{n}}(\eta_1 \plusdots \eta_n)} = [\text{с учётом независимости}] =
		\\
		\prod_{k = 1}^n \E e^{i\frac{t}{\sqrt{n}}\eta_k} = [\text{распределения одинаковы}] = \ps{\E e^{i\frac{t}{\sqrt{n}}\eta_1}}^n = \ps{\phi_{\eta_1}\ps{\frac{t}{\sqrt{n}}}}^n
	\end{multline*}
	Осталось применить уже упомянутое следствие:
	\begin{multline*}
		\forall t \in \R\ \ \phi_{T_n}(t) = \ps{1 - \frac{t^2}{2n} + o(1 / n)}^n =
		\\
		\exp\ps{n \ln\ps{1 - \frac{t^2}{2n} + o(1 / n)}} =
		\\
		\exp{n \cdot \ps{-\frac{t^2}{2n} + o(1 / n)}} = e^{-t^2 / 2 + o(1)}, n \to \infty
	\end{multline*}
	Это и означает, что $\lim_{n \to \infty} \phi_{T_n}(t) = e^{-t^2 / 2}$. Требуемое доказано.
\end{proof}