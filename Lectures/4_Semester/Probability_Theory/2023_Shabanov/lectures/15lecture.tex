\begin{theorem} (Свойства условного математического ожидания)
	Далее $\cC$, $\cC_1$, $\cC_2$ --- всегда под-$\sigma$-алгебры $\F$. Верны следующие утверждения:
    \begin{enumerate}
        \item Пусть $\E|\xi| < +\infty$. Пусть $\xi$ --- $\cC$--измеримая случайная величина. Тогда:
        \[
            \E(\xi | \cC) = \xi
        \]

        \item (Линейность) Пусть $\E|\xi|, \E|\eta| < +\infty$. Тогда:
        \[
            \forall a, b \in \R\ \ \E(a\xi + b\eta | \cC) = a\E(\xi | \cC) + b\E(\eta | \cC)
        \]

        \item (Формула полной вероятности) Пусть $\E|\xi| < +\infty$. Тогда:
        \[
            \E(\E(\xi | \cC)) = \E\xi
        \]

        \item Пусть $\E|\xi| < +\infty$, а также $\xi$ независима с $\cC$, то есть порождённая ей $\sigma$-алгебра $\F_\xi$ независима с $\cC$. Тогда:
        \[
            \E(\xi | \cC) = \E\xi
        \]

        \item (Сохранение отношения порядка) Пусть $\E|\xi|, \E|\eta| < +\infty$. Если $\xi \le \eta$, то:
        \[
            \E(\xi | \cC) \le \E(\eta | \cC) \text{ почти наверное} 
        \]

        \item Пусть $\E|\xi| < +\infty$. Тогда:
        \[
            |\E(\xi | \cC)| \le \E(|\xi| \ | \ \cC) \text{ почти наверное}
        \]

        \item (Телескопическое свойство) Пусть $\E|\xi| < +\infty$. Если $\cC_1 \subseteq \cC_2$, то:
        \begin{enumerate}
            \item $\E(\E(\xi | \cC_1) | \cC_2) = \E(\xi | \cC_1)$
            
            \item $\E(\E(\xi | \cC_2) | \cC_1) = \E(\xi | \cC_1)$
        \end{enumerate}

        \item (Предельный переход) Пусть $\{\xi_n\}_{n = 1}^\infty$ --- последовательность случайных величин, причём $\E|\xi_n| < +\infty$.
        \begin{enumerate}
            \item (Монотонная сходимость) Если $0 \le \xi_n \sua \xi$ почти наверное и $\sup_{n \in \N} \E\xi_n < +\infty$, то:
            \[
                \E(\xi_n | \cC) \xrightarrow{P\text{ п.н.}} \E(\xi | \cC)
            \]

            \item (Мажорируемая сходимость) Если $\xi_n \to^{P\text{ п.н.}} \xi$, $|\xi_n| \le \eta$, $\E\eta < +\infty$, то:
            \[
                \E(\xi_n | \cC) \xrightarrow{P\text{ п.н.}} \E(\xi | \cC)
            \]
        \end{enumerate}

        \item Пусть $\E|\xi| < +\infty$, $\E|\xi \eta| < +\infty$. Если $\eta$ --- $\cC$--измеримая случайная величина, то:
        \[
            \E(\xi\eta | \cC) = \eta \E(\xi | \cC)
        \]

        \item (Неравенство Йенсена) Пусть $\phi: \R \to \R$ --- выпуклая вниз борелевская функция, $\E|\xi| < +\infty$, $\E|\phi(\xi)| < +\infty$. Тогда:
        \[
            \E(\phi(\xi) | \cC) \ge \phi(\E(\xi | \cC)) \text{ почти наверное}
        \]
    \end{enumerate}
\end{theorem}

\begin{note}
    Во всех свойствах для всех условных матожиданий выполнена теорема о существовании и единственности УМО, если это где-то неочевидно, то будет пояснено в доказательстве. Стоит отметить, что все равенства из свойств можно заменить на равенство почти наверное, так как в таком случае УМО определено однозначно с точностью до множества нулевой вероятности. Практически все свойства будут доказываться так: есть кандидат в условное матожидание, проверим для него выполнение свойств из определения УМО.
\end{note}

\begin{proof}~
    \begin{enumerate}
        \item Выполнены условия теоремы о существовании и единственности УМО. Проверим свойства из определения УМО:
        \begin{itemize}
        	\item Свойство измеримости выполнено по условию теоремы.
        	
        	\item Интегральное свойство очевидно.
        \end{itemize}

        \item Так как $\E|a\xi + b\eta| \le |a| \cdot \E|\xi|+|b| \cdot \E|\eta| < +\infty$, то для всех УМО выполнены условия теоремы о существовании и единственности УМО. Проверим для $a\E(\xi | \cC) + b\E(\eta | \cC)$ свойства из определения УМО $\E(a\xi + b\eta | \cC)$:
        \begin{itemize}
        	\item Свойство измеримости: $\cC$-измерима как линейная комбинация $\cC$-измеримых.
        	
        	\item Интегральное свойство: пусть $A \in \cC$:
        	\begin{multline*}
	        	\E((a\xi + b\eta)\chi_A) = a\E(\xi \chi_A) + b\E(\eta \chi_A) = \text{[инт. свойство]} =
	        	\\
	        	= a\E(\E(\xi | \cC) \chi_A) + b\E(\E(\eta | \cC) \chi_A) = \E((a\E(\xi | \cC) + b\E(\eta | \cC))\chi_A)
        	\end{multline*}
        \end{itemize}

        \item Для УМО выполнены условия теоремы о существовании и единственности УМО. Так как $\cC$ --- $\sigma$-алгебра подмножеств $\Omega$, то $\Omega \in \cC$, поэтому в силу интегрального свойства:
        \[
            \E(\E(\xi | \cC)) = \E(\E(\xi | \cC)\chi_\Omega) = \E(\xi \chi_\Omega) = \E\xi
        \]

        \item Выполнены условия теоремы о существовании и единственности УМО. Проверим свойства из определения УМО:
        \begin{itemize}
        	\item Свойство измеримости для константы $\E\xi$, конечно, выполнено.
        	
        	\item Интегральное свойство: пусть $A \in \cC$. Тогда, так как $\F_{\chi_A} = \sigma(A) \subseteq \cC$, то просто по условию теоремы и по определению $\chi_A$ и $\xi$ независимы. Тогда:
        	\[
        		\E(\xi \chi_A) = \text{[независимость]} = (\E\xi) \E \chi_A = \E((\E\xi) \chi_A)
        	\]
        \end{itemize}

        \item Для всех УМО выполнены условия теоремы о существовании и единственности УМО.
        
        Возьмём произвольное $A \in \cC$. Тогда:
        \begin{multline*}
            \xi \le \eta \Ra \xi \chi_A \le \eta \chi_A \Ra \E(\xi \chi_A) \le \E(\eta \chi_A) \Ra \text{[инт. свойство УМО]} \Ra
            \\
            \Ra \E(\xi \chi_A) = \E(\E(\xi | \cC) \chi_A) \le \E(\E(\eta | \cC) \chi_A) = \E(\eta \chi_A)
        \end{multline*}

        Получили, что $\forall A \in \cC \ \ \E(\E(\xi | \cC) \chi_A) \le \E(\E(\eta | \cC) \chi_A)$.

        По свойству измеримости УМО $\E(\xi | \cC)$ и $\E(\eta | \cC)$ $\cC$-измеримы. Мы бы хотели воспользоваться соответствующим свойством математического ожидания про неравенство, но его условия более сильные. Однако, $\E(\eta | \cC) - \E(\xi | \cC)$ - тоже $\cC$-измеримая случайная велична, поэтому множество $A = \{\E(\eta | \cC) - \E(\xi | \cC) < 0\} \in \cC$. Далее доказательство повторяет то, что было проделано для свойства математического ожидания.

        \item Снова для всех УМО выполнены условия теоремы о существовании и единственности УМО. Так как $\xi \le |\xi|$ и $-\xi \le |\xi|$, то в силу предыдущего свойства:
        \begin{align*}
            &{\E(\xi | \cC) \le \E(|\xi| \ | \ \cC) \text{ почти наверное}}
            \\
            &{-\E(\xi | \cC) \le \E(|\xi| \ | \ \cC) \text{ почти наверное}}
        \end{align*}
        Стало быть, $|\E(\xi | \cC)| = \max(\E(\xi | \cC), -\E(\xi | \cC)) \le \E(|\xi|\ | \ \cC)$ почти наверное

        \item По третьему свойству (формуле полной вероятности) $\E(\E(\xi | \cC_1)) = \E(\E(\xi | \cC_2)) = \E\xi$, поэтому все эти матожидания конечны. Тогда для всех УМО выполнены условия теоремы о существовании и единственности УМО. Осталось проверить равенства:
        \begin{enumerate}
            \item Так как по свойству измеримости УМО $\E(\xi | \cC_1)$ $\cC_1$-измерима и $\cC_1 \subseteq \cC_2$, то $\E(\xi | \cC_1)$ $\cC_2$-измерима, после чего всё следует из первого доказанного свойства.

            \item Покажем, что $\E(\xi | \cC_1)$ удовлетворяет определению УМО $\E(\E(\xi | \cC_2) | \cC_1)$. $\E(\xi | \cC_1)$, конечно, $\cC_1$-измерима, теперь пусть $A \in \cC_1$, проверим интегральное свойство:
            \begin{multline*}
                \E( \E(\xi | \cC_2) \chi_A ) = \text{[инт. свойство, $A \in \cC_2 \supset \cC_1$]} = \E(\xi \chi_A) =
                \\
                = \E(\xi \chi_A) = \text{[инт. свойство, $A \in \cC_1$]} = \E( \E(\xi | \cC_1) \chi_A )
            \end{multline*}
        \end{enumerate}

        \item Здесь мы доказываем аналоги теоремы Леви о монотонной сходимости и теоремы Лебега о мажорируемой сходимости. В силу этих теорем в обоих случаях $\E\xi$ конечно. Поэтому для всех УМО выполнены условия теоремы о существовании и единственности УМО. Теперь проверяем, что действительно выполнены сходимости.
        \begin{itemize}
            \item[(a)] По условию теоремы и пятому свойству (сохранение отношения порядка):
            \begin{align*}
                & 0 \le \xi_1 \le \xi_2 \le \ldots \le \xi
                \\
                & \Downarrow
                \\
                & 0 \le \E(\xi_1 | \cC) \le \E(\xi_2 | \cC) \le \ldots \le \E(\xi | \cC)
            \end{align*}
            Все неравенства выполнены почти наверное.

            Обозначим $\eta = \lim_{n \to \infty} \E(\xi_n | \cC)$, из монотонной сходимости $\eta$ корректно определена и почти наверное $\eta \le \E(\xi | \cC)$. Хотим доказать, что $\eta = \E(\xi | \cC)$ п.н., тогда всё будет доказано. Заметим, что $\eta$ $\cC$-измерима как предел $\cC$-измеримых случайных величин $\E(\xi_n | \cC)$. Осталось проверить интегральное свойство, возьмём $A \in \cC$.

            Применим теорему Леви о монотонной сходимости к двум соотношениям:
            \begin{align*}
                & 0 \le \xi_n \chi_A \uparrow \xi \chi_A \text{ п.н.}
                \\
                & 0 \le \E(\xi_n | \cC) \chi_A \uparrow \eta \chi_A \text{ п.н.}
                \\
                & \Downarrow
                \\
                & \E(\xi \chi_A) = \lim_n \E(\xi_n \chi_A) = \text{[инт. свойство]} = \lim_n \E(\E(\xi_n | \cC) \chi_A) = \E(\eta \chi_A)
            \end{align*}

            Последняя строчка --- это в точности интегральное свойство, которое нам надо было проверить.

            \item[(b)] Хотим здесь сослаться на предыдущий пункт про монотонную сходимость. Обозначим $\eta_n = \sup_{m \ge n} |\xi_m - \xi|$. Тогда $\eta_n \xrightarrow{P\text{ п.н.}} 0$ как верхний предел для существующего обычного предела.

            Теперь распишем:
            \begin{multline*}
                | \E(\xi_n | \cC) - \E(\xi | \cC) | = | \E(\xi_n - \xi | \cC) | \le \text{[свойство 6 УМО]} \le \E(|\xi_n - \xi| \ | \ \cC) \le
                \\
                \le \text{[монотонность]} \le \E(\eta_n | \cC)
            \end{multline*}
                
            Так как $\eta_n \le \sup_{m \ge n} |\xi_m| + |\xi| \le 2\eta$ (так как $\eta$ --- мажоранта), получаем, что все матожидания $\E\eta_n$ конечны и ограничены одной константой. С учётом ещё и того, что $0 \le \eta_n \downarrow 0 \text{ п.н.}$, то не совсем по предыдущему пункту, но по тому, что из него мгновенно следует, получаем $\E(\eta_n | \cC) \xrightarrow[n \to \infty]{} \E(0 | \cC) = 0$ п.н.

            Отсюда следует, что
            \[
                | \E(\xi_n | \cC) - \E(\xi | \cC) | \le \E(\eta_n | \cC) \xrightarrow[n \to \infty]{} 0 \text{ п.н.}
            \]
            А это в точности то, что нам было нужно.
        \end{itemize}

        \item Для всех УМО выполнены условия теоремы о существовании и единственности УМО. Проверим для $\eta \E(\xi | \cC)$ свойства из определения УМО $\E(\xi\eta | \cC)$. Так как по условию теоремы и по свойству измеримости УМО $\eta$ и $\E(\xi | \cC)$ $\cC$-измеримы, то их произведение тоже $\cC$-измеримо. Поэтому осталось проверить только интегральное свойство УМО, пусть $A \in \cC$, нужно доказать, что
        \[
            \E(\xi\eta \chi_A) = \E(\eta \E(\xi | \cC) \chi_A) \ \ (\#)
        \]

        Будем доказывать (\#), рассматривая различные случайные величины $\eta$. Пусть сначала $\eta = I_B$, $B \in \cC$. Тогда получим:
        \[
            \E(\xi\eta \chi_A) = \E(\xi I_B \chi_A) = \E(\xi I_{A \cap B}) = \text{[инт. свойство]} = \E(\E(\xi | \cC) I_{A \cap B}) = \E(\eta \E(\xi | \cC) \chi_A)
        \]

        Теперь, так как равенство (\#), безусловно, линейно по $\eta$, получаем, что (\#) выполнена и для простых $\cC$-измеримых случайных величин $\eta$. Теперь пусть $\eta$ --- неотрицательная $\cC$-измеримая случайная величина, удовлетворяющая условию теоремы. Представляем её в виде монотонного предела неотрицательных простых $\cC$-измеримых случайных величин, переходим к пределу в равенстве (\#), получаем, что и в этом случае (\#) тоже выполнена.

        Чуть подробнее скажем, как в этом случае мы переходим к пределу. Разложим $\xi = \xi^+ - \xi^-$ и перепишем равенство (\#) как:
        \[
            \E(\xi^+ \eta \chi_A) - \E(\xi^- \eta \chi_A)= \E(\eta \E(\xi^+ | \cC) \chi_A) - \E(\eta \E(\xi^- | \cC) \chi_A) \ \ (\#\#)
        \]
        Так как $\xi^+, \xi^- \ge 0$, то и $\E(\xi^+ | \cC), \E(\xi^- | \cC) \ge 0$, последнее неравенство выполнено почти наверное. Теперь, для всех частей равенства (\#\#), под всеми матожиданиями получаем монотонную сходимость, применяем теорему Леви о монотонной сходимости.

        Остался последний случай: $\eta$ --- произвольного знака $\cC$-измеримая случайная величина, удовлетворяющая условию теоремы. Этот случай очевиден из доказанного ранее, разложения $\eta = \eta^+ - \eta^-$, и линейности.

        \item Для всех УМО выполнены условия теоремы о существовании и единственности УМО.
        
        Так как $\phi(x)$ --- выпуклая вниз функция, то
        \[
            \forall x \in \R \ \ \exists \Lambda(x) \in \R \text{ т.ч. } \forall y \in \R \ \ \phi(y) \ge \phi(x) + \Lambda(x)(y-x)
        \]

        Возьмём $x = \E(\xi | \cC)$, $y = \xi$, получим:
        \[
            \phi(\xi) \ge \phi(\E(\xi | \cC)) + \Lambda(\E(\xi | \cC))(\xi-\E(\xi | \cC))
        \]

        \textcolor{red}{Далее во всех переходах хорошо бы обосновать теорему о существовании и единственности УМО. И точно нужно знание, что $\Lambda$ --- борелевская.}

        Возьмём от обеих частей неравенства условное матожидание относительно $\cC$. Получим, раскрыв по линейности и воспользовавшись сохранением отношения порядка:
        \[
            \E(\phi(\xi) | \cC) \stackrel{P\text{ п.н.}}{\ge} \E(\phi(\E(\xi | \cC)) | \cC) + \E(\Lambda(\E(\xi | \cC))(\xi-\E(\xi | \cC)) | \cC)
        \]

        Так как $\E(\xi | \cC)$ --- $\cC$-измеримая случайная величина, $\phi$ и $\Lambda$ --- борелевские функции, то $\phi(\E(\xi | \cC))$, $\Lambda(\E(\xi | \cC))$ --- $\cC$-измеримые функции. В таком случае, воспользовавшись свойствами 1 и 9 УМО:
        \[
            \E(\phi(\xi) | \cC) \stackrel{P\text{ п.н.}}{\ge} \phi(\E(\xi | \cC)) + \Lambda(\E(\xi | \cC)) \E((\xi-\E(\xi | \cC)) | \cC)
        \]

        С учётом того, что:
        \[
            \E((\xi-\E(\xi | \cC)) | \cC) = \E(\xi | \cC) - \E(\E(\xi | \cC) | \cC) = \E(\xi | \cC) - \E(\xi | \cC) = 0 \text{ п.н}
        \]
        получим, что:
        \[
            \E(\phi(\xi) | \cC) \stackrel{P\text{ п.н.}}{\ge} \phi(\E(\xi | \cC))
        \]
        А это в точности то, что нужно было доказать.
    \end{enumerate}
\end{proof}

\section{Условные распределения}

\begin{note}
    Изучили свойства УМО. Теперь хотим понять, как его считать.
\end{note}

\begin{note}
    Зафиксируем вероятностное пространство $(\Omega, \F, P)$.
\end{note}

\begin{definition}
    $P(A | \cC) := \E(\chi_A | \cC),\ A \in \F,\ \cC$ --- $\sigma$-подалгебра в $\F$ --- условная вероятность события $A$ относительно $\sigma$-алгебры $\cC$.
\end{definition}

\begin{definition}
    $\E(\xi | \eta) := \E(\xi | \F_\eta)$ --- условное математическое ожидание случайной величины $\xi$ относительно случайной величины $\eta$.
\end{definition}

\begin{definition}
    Условным математическим ожиданием $\xi$ при условии $\eta = y$, обозначается как $\E(\xi | \eta = y)$, называется такая борелевская функция $\phi(y)$, что:
    \[
        \forall B \in \B(\R) \ \ \E(\xi I\{\eta \in B\}) = \int_{B} \phi(y) P_\eta(dy)
    \]
\end{definition}

\begin{theorem} (Существование и единственность $\E(\xi | \eta = y)$, без доказательства)
    Если $\E|\xi| < +\infty$, то $\E(\xi | \eta = y)$ существует и единственно $P_\eta$-почти наверное.
\end{theorem}

\begin{note}
    Здесь доказательство также опирается на теорему Радона-Никодима.
\end{note}

\begin{note}
    Свойства $\E(\xi | \eta = y)$ практически полностью повторяют свойства УМО $\E(\xi | \cC)$, поэтому не будем приводить ни формулировки, ни доказательства. Отметим, что в частности выполнены линейность, сохранение отношения порядка, предельный переход.
\end{note}

\begin{note}
    Поймём, как связаны $\E(\xi | \eta)$ и $\E(\xi | \eta = y)$.
\end{note}

\begin{proposition}
    Пусть $\xi, \eta$ --- случайные величины, $\phi$ --- борелевская функция. Тогда
    \[
        \E(\xi | \eta = y) = \phi(y) \Leftrightarrow \E(\xi | \eta) = \phi(\eta)
    \]
\end{proposition}

\begin{note}
    Здесь в принципе даже не нужно требовать единственность, просто, если $\phi(y)$ удовлетворяет левому определению, то $\phi(\eta)$ удовлетворяет правому определению, и наоборот. И так как $\F_\eta$, конечно, является под-$\sigma$-алгеброй в $\F$, то слева и справа условия теоремы о существовании и единственности выполнены или не выполнены одновременно.
\end{note}

\begin{proof}
    Заметим, что $\phi(\eta)$, так как $\phi$ --- борелевская, является $\F_\eta$-измеримой. Поэтому достаточно проверить равносильность выполнения интегральных свойств. Зафиксируем произвольное $B \in \B(\R)$. Заметим, что множества $\{\eta \in B\}$ по всем таким $B$ --- это в точности все множества из $\F_\eta$. 
    \begin{align*}
        & \E(\xi I\{\eta \in B\}) = \text{[опр-е слева]} = \int_{B} \phi(y) P_\eta(dy)
        \\
        & \E(\xi I\{\eta \in B\}) = \text{[опр-е справа]} = \E(\phi(\eta) I\{\eta \in B\})
        \\
        & \E(\phi(\eta) I\{\eta \in B\}) = \text{[зам. перем-х]} = \int_{B} \phi(y) P_\eta(dy)
    \end{align*}

    Отсюда получаем равносильность выполнения интегральных свойств слева и справа для произвольного борелевского множества. С учётом того, что так перечислим все необходимые множества, получаем просто равносильность выполнения интегральных свойств слева и справа.
\end{proof}

\begin{definition}
    Условным распределением случайной величины $\xi$ относительно случайной величины $\eta$ называется функция $P(B, y),\ B \in \B(\R),\ y \in \R$, удовлетворяющая трём свойствам:
    \begin{enumerate}
        \item При фиксированном $B$ функция $P(B, y)$ является борелевской функцией от $y$
        \item При фиксированном $y$ функция $P(B, y)$ является вероятностной мерой на $(\R, \B(\R))$
        \item $\forall A, B \in \B(\R) \ \ P(\xi \in B,\ \eta \in A) = \int_A P(B, y) P_\eta(dy)$
    \end{enumerate}
    Обозначение: $P(\xi \in B | \eta = y) := P(B, y)$.
\end{definition}

\begin{note}
    Если условное распределение $P(\xi \in B | \eta = y)$ существует, то при фиксированном борелевском множестве $B$ оно является условным математическим ожиданием $\E(I\{\xi \in B\} | \eta=y\}$.

    Действительно, при фиксированном $B \in \B(\R)$ определение условного распределения $P(B, y) = P(\xi \in B | \eta = y)$ вырождается в:
    \begin{align*}
        & P(B, y) \text{ является борелевской функцией от } y
        \\
        & \forall A \in \B(\R) \ \ P(\xi \in B,\ \eta \in A) = \int_A P(B, y) P_\eta(dy)
    \end{align*}

    Обозначим $\phi(y) = P(B, y)$ и перепишем в несколько другом виде:
    \begin{align*}
        & \phi(y) \text{ является борелевской функцией от } y
        \\
        & \forall A \in \B(\R) \ \ \E(I\{\xi \in B\} I\{\eta \in A\}) = P(\xi \in B,\ \eta \in A) = \int_A \phi(y) P_\eta(dy)
    \end{align*}

    Посмотрим на определение условного математического ожидания $\E(I\{\xi \in B\} | \eta=y\}$ и поймём, что это в точности оно и есть:
    \[
        \E(I\{\xi \in B\} | \eta=y\} = P(\xi \in B,\ \eta = y)
    \]
\end{note}

\begin{note}
    Почему нельзя было определить $P(\xi \in B | \eta = y)$ как $\E(I\{\xi \in B\} | \eta=y\}$? Из такого определения по таким же рассуждениям, что и в предыдущем замечании, мгновенно следуют пункты 1 и 3 определения условного распределения. Но для каждого конкретного $B$ эта штука неоднозначно определена, с точностью до множества нулевой вероятности, поэтому здесь плохо говорить, как пункт 2, о вероятностной мере, могут возникнуть проблемы, например, со счётной аддитивностью.
\end{note}

\begin{note}
    Тем не менее, утверждается, доказывать это не будем, что всегда можно выбрать условное матожидание $\E(I\{\xi \in B\} | \eta=y\}$ так, что оно будет являться условным распределением $P(\xi \in B | \eta = y)$. Отсюда в частности, при $\E|\xi| < +\infty$ получаем теорему о существовании и единственности условного распределения.
\end{note}

\begin{theorem} (Существование и единственность условного распределения, без доказательства)
    При $\E|\xi| < +\infty$ условное распределение $P(\xi \in B | \eta = y)$ существует и единственно, единственность можем утверждать $P_\eta$-почти наверное при каждом фиксированном $B$.

    \textcolor{red}{По идее, так как в определении условного распределения больше ограничений, единственность может быть сильнее, но быстро утверждений на этот счёт я не нашёл.}
\end{theorem}

\begin{definition}
    Функция $f_{\xi | \eta}(x | y)$ называется условной плотностью случайной величины $\xi$ относительно случайной величины $\eta$ (по мере $\mu$), если:
    \begin{align*}
        & f \text{ неотрицательна}
        \\
        & \forall B \in \B(\R) \ \ \forall y \in \R \ \ P(\xi \in B,\ \eta = y) = \int_B f_{\xi | \eta}(x | y) \mu(dx)
    \end{align*}

    Формально второе требование можно сформулировать так: нужно, чтобы такой интеграл от условной плотности удовлетворял определению условного распределения.
\end{definition}

\begin{note}
    Интересные нам случаи:
    \begin{itemize}
        \item Дискретный: $\mu$ --- считающая мера --- получаем ряд
        \item Абсолютно непрерывный: $\mu$ --- обычная мера Лебега на прямой --- получаем обычный интеграл Лебега на прямой
    \end{itemize}
\end{note}

\begin{theorem} (О вычислении условного математического ожидания)
    Если существует условная плотность $f_{\xi | \eta}(x | y)$ случайной величины $\xi$ относительно случайной величины $\eta$, то для любой борелевской функции $g: \R \to \R$ выполнено:
    \[
        \E(g(\xi) | \eta = y) = \int_\R g(x) f_{\xi | \eta}(x | y) \mu(dx)
    \]

    Здесь предполагаем, что $\E g(\xi)$ конечно, чтобы была выполнена теорема о существовании и единственности УМО. Единственность, как помним, есть лишь почти наверное, оговаривать это в доказательстве не будем.
\end{theorem}

\begin{proof}
    Докажем аналогично построению интеграла Лебега: будем рассматривать различные борелевские фунцкии $g(x)$. Пусть сначала $g(x) = I\{x \in B\},\ B \in \B(\R)$.
    
    Условное распределение выражается через условную плотность:
    \[
        P(\xi \in B,\ \eta = y) = \int_B f_{\xi | \eta}(x | y) \mu(dx)
    \]

    По замечанию к определению условного распределения, последнее удовлетворяет определению условного математического ожидания:
    \[
        \E(I\{\xi \in B\} | \eta = y) = P(\xi \in B,\ \eta = y)
    \]
    
    Тогда получаем:
    \begin{multline*}
        \E(g(\xi) | \eta=y) = \E(I\{\xi \in B\} | \eta = y) = P(\xi \in B,\ \eta = y) =
        \\
        = \int_B f_{\xi | \eta}(x | y) \mu(dx) = \int_\R I\{x \in B\} f_{\xi | \eta}(x | y) \mu(dx) = \int_\R g(x) f_{\xi | \eta}(x | y) \mu(dx)
    \end{multline*}

    Доказали утверждение теоремы для индикаторов. Так как оно, очевидно, линейно по $g$, то доказали его и для простых функций. Следующий шаг --- неотрицательные функции, после чего по линейности обобщается на функции произвольного знака. Здесь нигде при спуске к уже доказанному не теряли свойство конечности матожидания $\E g(\xi)$ и того, что $g$ --- борелевская.

    Поясним подробнее переход к неотрицательным функциям. Представим неотрицательную $g$ в виде возрастающего предела неотрицательных простых $g_n$. Перепишем доказываемое равенство в другом виде, разнеся условную плотность на положительную и отрицательную части:
    \[
        \E(g(\xi) | \eta = y) = \int_\R g(x) f^+_{\xi | \eta}(x | y) \mu(dx) - \int_\R g(x) f^-_{\xi | \eta}(x | y) \mu(dx)
    \]

    Так как в силу условия теоремы и построения функций $g_n$ выполнено, что все матожидания $\E g(\xi),\ \E g_n(\xi)$ конечны, слева можем применять предельный переход для монотонной сходимости для условных матожиданий. Справа тоже применяем теорему Леви о монотонной сходимости, так как внутри интегралов сходимость монотонна. Причём в силу сохранения конечности матожиданий и борелевости спуск корректен. Всё доказано.
\end{proof}

\begin{note}
    Осталось ответить на вопрос: как вычислить условную плотность?
\end{note}

\begin{theorem} (Достаточное условие наличия условной плотности)
    Пусть случайные величины $\xi$, $\eta$ таковы, что существует совместная плотность $f_{\xi, \eta}(x, y)$ случайного вектора $(\xi, \eta)$ по мере $\mu \times \lambda$. Тогда
    \[
        f_{\xi | \eta}(x | y) = \System{
            & \frac{f_{\xi, \eta}(x, y)}{f_\eta(y)},\ f_\eta(y) > 0
            \\
            & 0,\ f_\eta(y) = 0
        }
    \]
    является условной плотностью $\xi$ относительно $\eta$.

    Здесь $f_\eta(y)$ --- плотность случайной величины $\eta$ по мере $\lambda$. Она, конечно, существует, так как существует совместная плотность.
\end{theorem}

\begin{proof}
    Так как кандидат в условную плотность, безусловно, неотрицателен, необходимо и достаточно проверить, что
    \[
        Q(B, y) = \int_B f_{\xi | \eta}(x | y) \mu(dx)
    \]
    удовлетворяет определению условного распределения $\xi$ относительно $\eta$.

    Плотность $\eta$ можно выразить через совместную плотность $(\xi, \eta)$:
    \[
        f_\eta(y) = \int_\R f_{\xi, \eta}(x, y) \mu(dx)
    \]
    
    Тогда $Q(B, y)$ можем записать в виде:
    \[
        Q(B, y) = \System{
            & \frac{\int_B f_{\xi, \eta}(x, y) \mu(dx)}{\int_\R f_{\xi, \eta}(x, y) \mu(dx)},\  f_\eta(y) > 0
            \\
            & 0,\ f_\eta(y) = 0
        }
    \]

    Проверим все три свойства из определения условного распределения для $Q(B, y)$:
    \begin{enumerate}
        \item Зафиксируем $B$. $f_{\xi, \eta}$ является плотностью, в частности, интегрируема по мере $\mu \times \lambda$. Тогда $f_{\xi, \eta} I_B$ тоже интегрируема по мере $\mu \times \lambda$. При этом все интегралы конечны. Тогда, в силу теоремы Фубини, интегралы
        \[
            \int_B f_{\xi, \eta}(x, y) \mu(dx),\ \ f_\eta(y) = \int_\R f_{\xi, \eta}(x, y) \mu(dx)
        \]
        интегрируемы по мере $\lambda$, в частности, являются $\lambda$-измеримыми функциями. Из этого следует, что $Q(B, y)$ является $\lambda$-измеримой функцией. Так как $\lambda$ --- мера на борелевской $\sigma$-алгебре в $\R$ (считающая мера, мера Лебега, или какая-то ещё), то $Q(B, y)$ является борелевской функцией от $y$.

        \item Зафиксируем $y$. Если $f_\eta(y) > 0$, то в силу неотрицательности совместной плотности и счётной аддитивности интеграла Лебега, $Q(B, y)$ удовлетворяет определению вероятностной меры на $(\R, \B(\R))$.

        А вот если $f_\eta(y) = 0$, то $Q(B, y) \equiv 0$, и утверждение про вероятностную меру вообще неправда. В конце доказательства поясним, почему это не является проблемой.
        
        \item В третьем свойстве нужно проверить, что:
        \[
            \forall A, B \in \B(\R) \ \ P(\xi \in B,\ \eta \in A) = \int_A Q(B, y) P_\eta(dy)
        \]

        Проверяем:
        \begin{multline*}
            P(\xi \in B,\ \eta \in A) = \text{[совм. пл-сть]} = \int_{B \times A} f_{\xi, \eta}(x, y) \ \mu \times \lambda (dx, dy) = \text{[теор. Фубини]} =
            \\
            = \int_A \left( \int_B f_{\xi, \eta}(x, y) \mu(dx) \right) \lambda(dy) = \int_A Q(B, y) f_\eta(y) \lambda(dy) = \int_A Q(B, y) P_\eta(dy)
        \end{multline*}
    \end{enumerate}

    Теперь поясним возможную проблему свойства 2. Заметим, что по доказательству, если при всех $y$ таких, что $f_\eta(y) = 0$ переопределить $Q(B, y)$ любым измеримым образом, то свойства 1 и 3 не нарушатся. В частности, можем переопределить так, чтобы свойство 2 тоже было выполнено, и с формальной точки зрения всё было хорошо.

    Теперь почему это не очень важно. Посмотрим, какую вероятность имеет множество проблемных точек $y$ (это множество, конечно, измеримо):
    \[
        P_\eta(y \colon f_\eta(y) = 0) = \int_{\R} I\{y \colon f_\eta(y) = 0\} dP_\eta(y) = \int_{\R} I\{y \colon f_\eta(y) = 0\} f_\eta(y) d\lambda(y) = 0
    \]

    То есть $P_\eta$-почти наверное все точки $y$ хорошие. Теперь подумаем, где нам нужна условная плотность? При вычислении условного математического ожидания в предыдущей теореме. Но условное математическое ожидание определено однозначно с точностью до множества $P_\eta$-меры нуль, и переопределение УМО на множестве $P_\eta$-меры нуль не влияет на выполнение условий его определения. Поэтому при условной плотности из данной теоремы тоже получим условное математическое ожидание.
\end{proof}