\subsection{Классификация вероятностных мер}

\subsubsection{Дискретные вероятностные меры}

\begin{definition}
	Пусть $P$ --- вероятностная мера на $(\R, \B(\R))$. Она называется \textit{дискретной}, если выполнено условие:
	\[
		\exists X \subseteq \R \text{ --- не более чем счётное множество} \such P(\R \bs X) = 0 \wedge \forall x \in X\ P(\{x\}) > 0
	\]
	При этом говорят, что \textit{вероятностная мера $P$ сосредоточена на $X$}.
\end{definition}

\begin{note}
	\textit{Распределением (вероятностей)} называется сама вероятностная мера $P$.
\end{note}

\begin{definition}
	Пусть дискретная вероятностная мера $P$ сосредоточена на $X = \{x_k\}_{k = 1}^\infty \subset \R$. Обозначим $p_k = P(\{x_k\})$, тогда \textit{набор $(p_1, p_2, \ldots)$ образует распределение вероятностей на $X$}.  
\end{definition}

\begin{note}
	Если $P$ --- дискретная вероятстная мера, сосредоточенная на $X = \{x_k\}_{k = 1}^\infty \subset \R$, то функция распределения $F(x)$ имеет такой вид:
	\[
		F(x) = P\lsi{-\infty; x} = \sum_{k \colon x_k \le x} P(\{x_k\})
	\]
\end{note}

\subsubsection*{Примеры дискретных распределений}

\begin{enumerate}
	\item Константа $Const(x)$, $x \in \R$. $X = \{x\}$, $P(\{x\}) = 1$.
	
	\item Распределение Бернулли $Bern(p)$, $p \in (0; 1)$. $X = \{0, 1\}$, $p_1 = P(\{1\}) = p$, $p_0 = 1 - p$ (является моделью броска идеальной монетки)
	
	\item Биномиальное распределение $Bin(n, p)$, $n \in \N_0$, $p \in (0; 1)$. $X = \{0, 1, \ldots, n\}$, $p_k = C_n^k p^k (1 - p)^{n - k}$ для $k \in \range{0}{n}$ (является моделью $n$ независимых бросков идеальной монетки)
	
	\item Пуассоновское распределение $Poiss(\lambda)$, $\lambda > 0$. $X = \N_0$, $p_k = \frac{\lambda^k}{k!}e^{-\lambda}$ для $k \in \N_0$ (используется для моделирования \textit{редких событий})
\end{enumerate}

\subsubsection{Абсолютно непрерывные веротностные меры}

\begin{definition}
	Пусть $P$ --- вероятностная мера на $(\R, \B(\R))$, а $F$ -- её функция распределения. Эта мера называется \textit{абсолютно непрерывной}, если выполнено условие:
	\[
		\exists p \colon \R \to \R \such \int_\R p(t)d\mu(t) = 1 \wedge F(x) = \int_{\rsi{-\infty; x}} p(t)d\mu(t)
	\]
	В этом случае $p(t)$ называется \textit{плотностью функции распределения $F$ (меры $P$)}.
\end{definition}

\begin{note}
	Достаточно часто задача складывается так, что интеграл Лебега спокойно заменяется на интеграл Римана (функция позволяет).
\end{note}

\begin{proposition}
	Пусть $P$ --- абсолютно непрерывная вероятностная мера на $(\R, \B(\R))$. Тогда имеет место формула:
	\[
		\forall B \in \B(\R)\ \ P(B) = \int_B p(x)d\mu(x)
	\]
\end{proposition}

\begin{proof}
	Рассмотрим $Q(B) = \int_B p(x)d\mu(x)$. В силу свойств интеграла Лебега, $Q$ является вероятностной мерой на измеримом пространстве $(\R, \B(\R))$, причём совпадающая с $P$ на полукольце полуинтервалов. Значит, по теореме Каратеодори $P = Q$.
\end{proof}

\begin{note}
	Естественно, если вероятностная мера абсолютно непрерывна, то и функция распределения абсолютно непрерывна. В частности, это означает 2 вещи:
	\begin{enumerate}
		\item $F$ непрерывна
		
		\item $F'(x) = p(x)$ почти всюду по мере Лебега
	\end{enumerate}
\end{note}

\subsubsection*{Примеры абсолютно непрерывных мер}

\textcolor{red}{Хорошо бы графики добавить}

\begin{enumerate}
	\item Равномерное распределение $U[a; b]$, $-\infty < a < b < +\infty$. Тогда
	\begin{align*}
		&{p(x) = \frac{1}{b - a}\chi_{[a; b]}}
		\\
		&{F(x) = \System{
			&{0,\ x < a}
			\\
			&{\frac{x - a}{b - a},\ x \in [a; b]}
		}}
	\end{align*}
	
	\item Нормальное (гауссовское) распределение $N(a, \sigma^2)$, $a \in \R$, $\sigma > 0$. Тогда
	\begin{align*}
		&{p(x) = \frac{1}{\sigma\sqrt{2\pi}}e^{-\frac{(x - a)^2}{\sigma^2}}}
		\\
		&{F(x) = \int_{-\infty}^x p(y)dy \text{ --- неберущийся интеграл}}
	\end{align*}
	Эта мера используется для моделирования ошибок измерения
	
	\item Экспоненциальное (показательное) распределение $Exp(\alpha), \alpha > 0$. Тогда
	\begin{align*}
		&{p(x) = \alpha e^{-\alpha x}\chi_{\{x > 0\}}}
		\\
		&{F(x) = \System{
			&{0,\ x \le 0}
			\\
			&{1 - e^{-\alpha x},\ x > 0}
		}}
	\end{align*}
	
	\item Гамма распределение $\Gamma(\alpha, \lambda)$, $\alpha, \lambda > 0$. Тогда
	\begin{align*}
		&{p(x) = \frac{x^{\lambda - 1}\alpha^\lambda}{\Gamma(\lambda)}e^{-\alpha x} \chi_{\{x > 0\}}}
		\\
		&{F(x) = \int_{-\infty}^x p(y)dy \text{ --- неберущийся интеграл}}
	\end{align*}
	где $\Gamma$ --- это \textit{гамма-функция}, являющаяся непрерывным обощением факториала. Верны следующие факты:
	\begin{enumerate}
		\item \(\forall \lambda > 0\ \ \Gamma(\lambda) = \int_0^{+\infty} x^{\lambda - 1}e^{-x}dx\)
		
		\item \(\Gamma(\lambda + 1) = \lambda\Gamma(\lambda)\)
		
		\item \(\forall n \in \N\ \ \Gamma(n) = (n - 1)!\)
	\end{enumerate}

	\item Распределение Коши $K(\sigma)$, $\sigma > 0$. Тогда
	\begin{align*}
		&{p(x) = \frac{\sigma}{\pi(x^2 + \sigma^2)}}
		\\
		&{F(x) = \frac{1}{2} + \frac{1}{\pi} \arctg \frac{x}{\sigma}}
	\end{align*}
	(появляется, например, при исследовании вероятностной меры точки пересечения луча, проведённого из фиксированной точки, не лежащей на данной прямой)
\end{enumerate}

\subsubsection{Сингулярные вероятностные меры}

\begin{definition}
	Пусть $F$ --- функция распределения на $\R$. Точка $x$ называется \textit{точкой роста $F$}, если выполнено утверждение:
	\[
		\forall \eps > 0\ F(x + \eps) - F(x - \eps) > 0
	\]
\end{definition}

\begin{definition}
	Функция распределения $F$ (и соответствующая ей мера $P$) называется \textit{сингулярной}, если $F$ непрерывна и множество её точек роста имеет нулевую меру по Лебегу.
\end{definition}

\begin{example}
	Канторова лестница, дополненная нулём на интервале $(-\infty; 0)$ и единицей на $(1; +\infty)$ является сингулярной функцией распределения.
	
	Интересный факт здесь состоит в том, что мера всех горизонтальных отрезков равна единице, при этом вероятность попасть в любой из отрезков --- ноль (потому что вероятность отрезка = разность значений на его концах, а они одинаковы).
\end{example}

\begin{theorem} (Лебега, без доказательства)
	Пусть $F$ --- функция распределения на $\R$. Тогда существует разложение этой функции следующего вида:
	\[
		F(x) = \alpha_1 F_1(x) + \alpha_2 F_2(x) + \alpha_3 F_3(x)
	\]
	где $\alpha_i \ge 0$ и $\alpha_1 + \alpha_2 + \alpha_3 = 1$, а $F_1$ --- дискретная, $F_2$ --- абсолютно непрерывная, $F_3$ --- сингулярная функции распределения.
\end{theorem}