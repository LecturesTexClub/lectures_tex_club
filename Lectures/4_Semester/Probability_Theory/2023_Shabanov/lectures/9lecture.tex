\textcolor{red}{Выше находятся лекции до 8й включительно плюс доказательство УЗБЧ}

\begin{note}
	В чём смысл УЗБЧ? Это теоретическое обоснование принципа устойчивых частот. Если положить за $\xi_i$ индикатор того, что в $i$-м эксперименте произошло событие $A$, то
	\[
		\nu_n(A) = \frac{\xi_1 + \ldots + \xi_n}{n} \xrightarrow[n \to \infty]{P п.н.} \E\xi_1 = P(A)
	\]
	где $\nu_n(A)$ --- это частота появления $A$
\end{note}

\begin{note}
	Можно ли отказаться от условия на наличие четвёртого момента? Оказывается, можно, но для этого необходимо развить целый аппарат.
\end{note}

\section{Усиленный закон больших чисел}

\begin{definition}
	Последовательность случайных величин $\{\xi_n\}_{n = 1}^\infty$ фундаментальна с вероятностью 1, если
	\[
		P(\{\xi_n\}_{n = 1}^\infty \text{ фундаментальна}) = 1
	\]
\end{definition}

\begin{proposition}
	Последовательность $\{\xi_n\}_{n = 1}^\infty$ сходится почти наверное по мере $P$ тогда и только тогда, когда она фундаментальна с вероятностью 1.
\end{proposition}

\begin{proof}~
	\begin{itemize}
		\item[$\Ra$] Пусть $\xi_n \to^{P\text{ п.н.}} \xi$, тогда
		\[
			P(\{\xi_n\}_{n = 1}^\infty \text{ фундаментальна}) \ge P(\xi_n \to \xi) = 1
		\]
		
		\item [$\La$] Обозначим $A = \{\{\xi_n\}_{n = 1}^\infty \text{ фундаментальна}\}$. Тогда $\forall \omega \in A$ последовательность имеет некоторый предел, который мы положим за $\xi(\omega)$. Если $\omega \notin A$, то $\xi(\omega) := 0$. В результате становится верным равенство:
		\[
			\xi(\omega) = \lim_{n \to \infty} (\xi_n(\omega) \cdot I_A(\omega))
		\]
		Стало быть, $\xi$ является случайной величиной как предел последовательности случайных величин. Ну и остаётся увидеть соотношение:
		\[
			P(\xi_n \to \xi) \ge P(A) = 1
		\]
	\end{itemize}
\end{proof}

\begin{theorem} (Критерий фундаментальности с вероятностью 1)
	Последовательность $\{\xi_n\}_{n = 1}^\infty$ фундаментальна с вероятностью 1 тогда и только тогда, когда верно утверждение:
	\[
		\forall \eps > 0\ \ P(\sup_{k \ge n} |\xi_k - \xi_n| > \eps) \xrightarrow[n \to \infty]{} 0
	\]
\end{theorem}

\begin{theorem} (Неравенство Колмогорова)
	Пусть $\{\xi_k\}_{k = 1}^n$ --- независимые случайные величины, причём $\E\xi_k = 0$ и $\E\xi_k^2 < +\infty$. Обозначим $S_k = \xi_1 + \ldots + \xi_k$. Тогда верно утверждение:
	\[
		\forall \eps > 0\ \ P(\max_{1 \le k \le n} |S_k| \ge \eps) \le \frac{\E S_n^2}{\eps^2}
	\]
\end{theorem}

\begin{proof}
	Обозначим $A = \{\max_{1 \le k \le n} |S_k| \ge \eps\}$ и аналогичным образом $A_k$:
	\[
		A_k = \{|S_k| \ge \eps, \forall i \in \range{1}{k}\ \ |S_i| < \eps\}
	\]
	Тогда $A = \bscup_{k = 1}^n A_k$ \textcolor{red}{Дописать}
\end{proof}