\textcolor{red}{Отсюда качетсво конспект не гарантируется, пока что.}

\begin{definition}
	Функция $\phi \colon \R^n \to \R^m$ называется \textit{борелевской}, если выполнено условие:
	\[
		\forall B \in \B(\R^m)\ \ \phi^{-1}(B) = \{x \in \R^n \colon \phi(x) \in B\} \in \B(\R^n)
	\]
\end{definition}

\begin{anote}
	По сути просто измеримая функция из $(\R^n, \B(\R^n))$ в $(\R^m, \B(\R^m))$.
\end{anote}

\begin{proposition}
	Пусть $\xi \colon \Omega \to \R^n$ --- случайный вектор, $\phi \colon \R^n \to \R^m$ --- борелевская функция. Тогда $\phi(\xi)$ --- тоже случайный вектор.
\end{proposition}

\begin{proof}
	Рассмотрим любое $B \in \B(\R^m)$ и его прообраз для $\phi(\xi)$:
	\[
		(\phi(\xi))^{-1}(B) = \{\omega \in \Omega \colon \phi(\xi(\omega)) \in B\} = \{\omega \in \Omega \colon \xi(\omega) \in \phi^{-1}(B)\} \in \F
	\]
	Включение в $\F$ верно, поскольку $\phi^{-1}(B) \in \B(\R^n)$, а $\xi^{-1}(\phi^{-1}(B)) \in \F$ по определению случайного вектора.
\end{proof}

\begin{note}
	Какие функции являются борелевскими?
	\begin{enumerate}
		\item Непрерывные
		
		\item Индикаторы
		
		\item Кусочно-непрерывные функции
	\end{enumerate}
\end{note}

\begin{reminder}
	Если $A \in \F$ --- какое-то событие, то индикатором $A$ мы называем такую функцию:
	\[
		\chi_A(x) = \System{
			&{1,\ x \in A}
			\\
			&{0,\ x \notin A}
		}
	\]
\end{reminder}

\begin{corollary} (Арифметические операции над случайными величинами)
	Если $\xi, \eta$ --- случайные величины, то $\xi + \eta$, $\xi \cdot \eta$ и $\xi / \eta$ (если $\eta(\omega) \neq 0$ для $\forall \omega \in \Omega$) являются тоже случайными величинами.
\end{corollary}

\begin{proposition}
	Если $\{\xi_n\}_{n = 1}^\infty$ --- последовательности случайных величин, то $\sup_{n \in \N} \xi_N$, $\inf_n{n \in \N} \xi_n$, $\varlimsup_{n \to \infty} \xi_n$ и $\varliminf{n \to \infty} \xi_n$ являются случайными величинами \textit{в расширенном смысле} (то есть могут принимать значения $\pm\infty$).
\end{proposition}

\subsection{Характеристики случайных величин и векторов}

\begin{definition}
	\textit{Распределением случайной величины} $\xi$ называется вероятностная мера $P_\xi$ на измеримом пространстве $(\R, \B(\R))$, определённая по следующему правилу:
	\[
		\forall B \in \B(\R)\ \  P_\xi(B) = P(\xi \in B)
	\]
\end{definition}

\begin{note}
	Определение распределения случайного вектора из $\R^n$ делается аналогичным образом.
\end{note}

\begin{exercise}
	Докажите, что $P_\xi$ действительно является вероятностной мерой на $(\R^n, \B(\R^n))$, $n \in \N$.
\end{exercise}

\begin{definition}
	\textit{Функцией распределения} случайной величины $\xi$ называется функция распределени $F_\xi$, соответствующая $P_\xi$:
	\[
		F_\xi(x) = P(\xi \le x) = P_\xi\rsi{-\infty; x}
	\]
\end{definition}

\begin{definition}
	Если $\xi = (\xi_1, \ldots, \xi_n)$ --- случайный вектор из $\R^n$, то его \textit{функцией распределения} называется такая $F_\xi$:
	\[
		F_\xi(x_1, \ldots, x_n) = P_\xi(\rsi{-\infty; x_1} \times \ldots \times \rsi{-\infty; x_n}) = P(\xi_1 \le x_1 \wedge \ldots \wedge \xi_n \le x_n)
	\]
\end{definition}

\subsubsection*{Классификация случайных величин}

Случайная величина $\xi$ является
\begin{enumerate}
	\item \textit{Дискретной}, если таково её распределение.
	
	\item \textit{Абсолютно непрерывной}, если таково её распределение. В этом случае у случайной величины также есть \textit{плотность} $p_\xi \ge 0$
	
	\item \textit{Сингулярной}, если таково её распределение.
\end{enumerate}

\subsection{Порождённая $\sigma$-алгебра}

\begin{definition}
	Пусть $\xi$ --- случайная величина на вероятностном пространстве $(\Omega, \F, P)$. Тогда $\sigma$-алгеброй, порождённой $\xi$, называется такой объект:
	\[
		\F_\xi = \{\{\xi \in B\} \colon B \in \B(\R)\}
	\]
\end{definition}

\begin{note}
	Аналогичным образом определяется $\sigma$-алгебра, порождённая вектором. Естественно, тот факт, что $\F_\xi$ является $\sigma$-алгеброй, нужно обосновать, но мы это опустим. Отдельно отметим, что $\F_\xi \subset \F$.
\end{note}

\begin{definition}
	Случайная величина (вектор) $\eta$ является \textit{$\F_\xi$-измеримой}, если $\F_\eta \subset \F_\xi$.
\end{definition}

\begin{lemma}
	$\eta$ является $\F_\xi$-измеримой тогда и только тогда, когда существует борелевская функция $\phi$ такая, что $\eta = \phi(\xi)$.
\end{lemma}

\begin{proof}~
	\begin{itemize}
		\item[$\Ra$] \textcolor{red}{Придумать, должно быть просто}
		
		\item [$\La$] Пусть $\eta \colon \Omega \to \R^n$, $\xi \colon \Omega \to \R^m$ и $\eta = \phi(\xi)$. Рассмотрим $B \in \B(\R^n)$ и оценим прообраз по $\eta$:
		\[
			\eta^{-1}(B) = \{\omega \in \Omega \colon \eta(\omega) \in B\} = \{\omega \in \Omega \colon \xi(\omega) \in \phi^{-1}(B)\} \in \F_\xi
		\]
		Последнее действительно так, ибо $\phi^{-1}(B) \in \B(\R^m)$.
	\end{itemize}
\end{proof}

\begin{exercise}
	Если $A \in \F$ --- событие, то $\chi_A$ является случайной величиной.
\end{exercise}

\begin{definition}
	Случайная величина $\xi$ называется \textit{простой}, если она принимает конечное число значений. В этом случае её можно записать в таком виде:
	\[
		\xi = \sum_{k = 1}^n x_k \chi_{A_k}
	\]
	где $x_i$ --- различные принимаемые значения, а $A_i = \xi^{-1}(x_k)$.
\end{definition}

\begin{note}
	Из определения простой случайной величины следует, что $\bscup_{k = 1}^n A_k = \Omega$.
\end{note}

\begin{proposition}
	Пусть $\xi$ --- это случайная величина. Если положить $\xi^+ = \max(\xi, 0)$ и $\xi^- = \max(-\xi, 0)$, то $\xi = \xi^+ - \xi^-$.
\end{proposition}

\begin{proof}
	Тривиально.
\end{proof}

\begin{theorem} (о приближении простыми функциями)
	\begin{enumerate}
		\item Пусть $\xi \ge 0$ --- случайная величина. Тогда существует последовательность $\F_\xi$-измеримых простых случайных величин $\{\xi_n\}_{n = 1}^\infty$ такая, что $0 \le \xi_n \sua \xi$. Это означает следующее:
		\begin{enumerate}
			\item $\forall \omega \in \Omega, n \in \N\ \ 0 \le \xi_n(\omega) \le \xi_{n + 1}(\omega)$
			
			\item $\xi_n \to \xi$ --- сходится поточечно
		\end{enumerate}
		
		\item Если $\xi$ --- произвольная случайная величина, то существует последовательность $\F_\xi$-измеримых простых случайных величин таких, что
		\begin{enumerate}
			\item $\forall n \in \N\ \ |\xi_n| \le |\xi|$
			
			\item $\xi_n \to \xi$
		\end{enumerate}
	\end{enumerate}
\end{theorem}

\begin{proof}~
	\begin{enumerate}
		\item Предъявим $\xi_n$ в явном виде:
		\[
			\xi_n = \sum_{k = 1}^{n \cdot 2^n} \frac{k - 1}{2^n} \chi\set{\frac{k - 1}{2^n} \le \xi < \frac{k}{2^n}}
		\]
		Легко увидеть, что все требуемые свойства выполнются. Исключением может быть $\F_\xi$-измеримость, но и тут достаточно заметить, что $\xi_n = f_n(\xi) \Ra$ сразу является $\F_\xi$-измеримой случайной величиной.
		
		\item Применим уже доказанную часть к $\xi^+$ и $\xi^-$, а затем возьмём их разность.
	\end{enumerate}
\end{proof}

\subsection{Математическое ожидание случайной величины}

\begin{definition}
	Пусть $\xi$ --- случайная величина на вероятностном пространстве $(\Omega, \F, P)$. Тогда \textit{математическим ожиданием} случайной величины $\xi$ называется следующий интеграл Лебега от $\xi$ по мере $P$:
	\[
		\E\xi := \int_\Omega \xi(\omega)P(d\omega) = \int_\Omega \xi dP
	\]
\end{definition}

\subsubsection*{Построение математического ожидания}

\begin{enumerate}
	\item Если $\xi$ --- простая случайная величина, записываемая в виде
	\[
		\xi = \sum_{k = 1}^n x_k\chi_{A_k}
	\]
	то матожидание вычисляется по готовой формуле:
	\[
		\E\xi = \sum_{k = 1}^n x_kP(A_k)
	\]
	
	\item Если $\xi \ge 0$ --- неотрицательная случайная величина, то матожидание есть $\E\xi = \lim_{n \to \infty} \E\xi_n$, где $0 \le \xi_n \sua \xi$
	
	\item Если $\xi$ --- произвольная случайная величина, то 
	\begin{enumerate}
		\item $\E\xi = \E\xi^+ - \E\xi^-$, если оба числа справа конечны
		
		\item $\E\xi = +\infty$, если $\E\xi^+ = +\infty$ и $\E\xi^-$ конечно
		
		\item $\E\xi = -\infty$, если $\E\xi^- = +\infty$ и $\E\xi^+$ конечно
		
		\item $\E\xi$ не определено, если $\E\xi^+ = \E\xi^- = +\infty$
	\end{enumerate}
\end{enumerate}

\subsubsection*{Свойства математического ожидания}

\begin{note}
	Здесь мы считаем, что все математические ожидания рассматриваемых величин конечны.
\end{note}

\begin{definition}
	Говорят, что событие $A \in \F$ выполнено \textit{почти наверное}, если $P(A) = 1$
\end{definition}

\begin{enumerate}
	\item Если $0 \le \xi \le \eta$ и $\E\eta$ --- конечно, то и $\E\xi$ тоже конечно
	
	\item Если $|\xi| \le \eta$ и $\E\eta$ конечно, то и $\E\xi$ тоже конечно
	
	\item Если $c \in \R$, то $\E(c\xi) = c\E\xi$
	
	\item Для любых случайных величин $\E(\xi + \eta) = \E\xi + \E\eta$
	
	\item (Сохранение порядка) Если $\xi \le \eta$, то и $\E\xi \le \E\eta$
	
	\item (Неравенство с модулем) $|\E\xi| \le \E|\xi|$
	
	\item Если $\xi = 0$ почти наверное, то и $\E\xi = 0$
	
	\item Если $\xi = \eta$ почти наверное, то $\E\xi = \E\eta$
	
	\item Если $\xi \ge 0$ и $\E\xi = 0$, то $\xi = 0$ почти наверное
	
	\item Предположим, что $\forall A \in \F$ выполнено следующее неравенство для случайных величин $\eta, \xi$:
	\[
		\E(\xi\chi_A) \le \E(\eta\chi_A)
	\]
	Тогда $\xi \le \eta$ почти наверное.
\end{enumerate}

\begin{proof}
	Все свойства, кроме последнего, уже встречались в курсе ОВиТМа. Разберёмся с последним: рассмотрим $A = \{\xi > \eta\}$. Тогда $(\xi - \eta)\chi_A \ge 0$, а значит и $\E((\xi - \eta)\chi_A) \ge 0$. Однако, по условию $\E((\xi - \eta)\chi_A) \le 0$, то есть $\E((\xi - \eta)\chi_A) = 0$. По предпоследнему свойству получаем, что $(\xi - \eta)\chi_A = 0$ почти наверное. Из определения $A$ следует, что когда $\chi_A = 1$, то обязательно $\xi \neq \eta$. Значит, $P(A) = 0$.
\end{proof}

\subsection{Независимость случайных величин и векторов}

\begin{definition}
	Случайные векторы $\{\xi_\alpha\}_{\alpha \in \gA}$ называются \textit{независимыми в совокупности}, если независимы в совокупности порождённые ими $\sigma$-алгебры $\{\F_\alpha\}_{\alpha \in \gA}$.
\end{definition}

\begin{corollary}
	Случайные векторы $\{\xi_k\}_{k = 1}^n$, $\xi_k \in \R^{d_k}$ независимы в совокупности тогда и только тогда, когда
	\[
		\forall B_1 \in \B(\R^{d_1}), \ldots, B_n \in \B(\R^{d_n})\ \ P(\xi_1 \in B_1 \wedge \ldots \wedge \xi_n \in B_n) = \prod_{k = 1}^n P(\xi_k \in B_k)
	\]
\end{corollary}

\begin{lemma} (Критерий независимости случайных величин в терминах функции распределения)
	Случайные величины $\{\xi_k\}_{k = 1}^n$ независимы в совокупности тогда и только тогда, когда выполнено условие:
	\[
		\forall x_k \in \R\ \ P(\xi_1 \le x_1 \wedge \ldots \wedge \xi_n \le x_n) = \prod_{k = 1}^n P(\xi_k \le x_k) = \prod_{k = 1}^n F_{\xi_k}(x_k)
	\]
\end{lemma}

\begin{note}
	Можно сформулировать лемму несколько иначе: функция распределения случайного вектора $\xi = (\xi_1, \ldots, \xi_n)$ распадается в произведение функций распределений его компонент.
\end{note}

\begin{proof}~
	\begin{itemize}
		\item[$\Ra$] Тривиально из следствия выше.
		
		\item[$\La$] Вспомним о критерии независимости $\sigma$-алгебр. Если мы найдём некоторые $\pi$-системы $\cM_i$ такие, что $\sigma(\cM_i) = \F_{\xi_i}$, и докажем их независимость, то успех нам обеспечен. Посмотрим на такие системы:
		\[
			\cM_i = \{\{\xi_i \le x\},\ x \in \R\}
		\]
		Очевидно, что эти системы являются независимыми $\pi$-системами и, более того, $\sigma(\cM_i) \subseteq \F_{\xi_i}$. За счёт последнего свойства мы можем сказать, что любое множество из $\cM_i$ есть прообраз некоторого борелевского множества по $\xi_i$. Осталось показать, что на самом деле у нас есть все борелевские множества. Введём систему подходящих борелевских множеств $\cD_i$:
		\[
			\cD_i = \{B \in \B(\R) \colon \{\xi_i \in B\} \in \sigma(\cM_i)\}
		\]
		Покажем, что $\cD_i$ является $\sigma$-алгеброй:
		\begin{enumerate}
			\item $\R \in \cD_i$. Действительно, $\{\xi_i \in \R\} = \Omega \in \sigma(\cM_i)$
			
			\item Пусть $B_1, B_2 \in \cD$. Тогда $\{\xi_i \in B_1 \cap B_2\} = \{\xi_i \in B_1\} \cap \{\xi_i \in B_2\} \in \sigma(\cM_i)$. Из этого следует, что $B_1 \in \cD$
			
			\item Пусть $B \in \cD$. Тогда же $B \in \sigma(\cM_i) \Ra \ol{B} \in \sigma(\cM_i)$. Более того, $\{\xi_i \in \ol{B}\} = \{\xi_i \in \R \bs B\} = \{\xi_i \in \R\} \bs \{\xi_i \in B\} \in \sigma(\cM_i)$, ну и конечно же $\ol{B} \in \B(\R)$. Отсюда снова получаем $\ol{B} \in \cD$
			
			\item Пусть $\{B_n\}_{n = 1}^\infty \subseteq \cD$. Обозначим $B = \bigcup_{n = 1}^\infty B_n$ и воспользуемся всё тем же подходом:
			\[
				\{\xi_i \in B\} = \set{\xi_i \in \bigcup_{n = 1}^\infty B_n} = \bigcup_{n = 1}^\infty \{\xi_i \in B_n\} \in \sigma(\cM_i)
			\]
			Стало быть, $B \in \cD$.
		\end{enumerate}
		По построению, $\forall x \in \R\ \ \rsi{-\infty; x} \in \cD$, а это автоматически значит, что $\B(\R) = \cD$, то есть $\F_{\xi_i} \subseteq \sigma(\cM_i)$, что и требовалось. Осталось сказать, что независимость $\cM_i$ в совокупности эквивалентна формуле из условия, ибо любой элемент $\cM_i$ задаётся просто действительным числом.
	\end{itemize}
\end{proof}