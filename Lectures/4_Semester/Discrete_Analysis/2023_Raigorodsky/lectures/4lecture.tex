\begin{theorem}
	Пусть $H$ --- это $n$-однородный гиперграф, причём $|E(H)| < 2^{n - 1}$. Тогда $\chi(H) = 2$
\end{theorem}

\begin{proof}
	Рассмотрим случайную раскраску вершин нашего гиперграфа $H$ в красный и синий цвета (выбор цвета происходит с вероятностью 0.5). Сопоставим каждому ребру $e \in E$ событие $A_e$, что $e$ --- одноцветное множество. Тогда
	\[
		P(A_e) = 2^{1 - n}
	\]
	Сделаем тривиальную оценку на то, что ни одно ребро не является одноцветным (это и будет доказательством):
	\[
		P\ps{\bigcup_{e \in E} A_e} \le \sum_{e \in E} P(A_e) < 2^{n - 1} \cdot 2^{1 - n} = 1 \Ra P\ps{\bigcap_{e \in E} \overline{A_e}} > 0
	\]
\end{proof}

\begin{theorem}
	Пусть $H$ --- это $n$-однородный гиперграф, где степень каждой вершины не превосходит $n$. Если $n \ge 9$, то $\chi(H) = 2$
\end{theorem}

\begin{proof}
	Вводим всё то же вероятностное пространства и события $A_e$, как и в предыдущей теореме. Для доказательства этой теоремы мы применим симметричный случай ЛЛЛ, поэтому нужно ответить на вопрос: <<От каких $A_{e'}$ не зависит $A_e$?>> Для кого-то удивительно, но события, пересекающиеся по одной вершине, будут всё ещё независимы. Но если, вдруг, у нас есть 3 события, которые пересекаются попарно по 1 вершине, то в совокупности они зависимы. Самая простая оценка --- это всё же оценить число рёбер, которые пересекают наше хотя бы по одной вершине:
	\[
		d \le n \cdot \underbrace{(n - 1)}_{\forall v \in V\ \deg v \le n - 1}
	\]
	Осталось проверить неравенство $e \cdot (n(n - 1) + 1) \cdot 2^{1 - n} \le 1$. Понятно, что левая часть стремится к нулю, а требование $n \ge 9$ обеспечило нам уже верность этого неравенства.
\end{proof}

\begin{theorem}
	Пусть $H$ --- это $n$-однородный гипеграф. Тогда верна следующая оценка снизу на хроматическое число:
	\[
		\chi(H) \le \sqrt[n - 1]{e(n(\Delta - 1) + 1)}
	\]
	где $\Delta$ --- наибольшая степень вершины в $H$.
\end{theorem}

\begin{proof}
	Результат, который указан в теореме, получается только через ЛЛЛ, по крайней мере сегодня.
	
	Снова рассмотрим вероятностное пространство со случайными раскрасками графа в $r$ цветов. Тогда $A_e$ это всё то же событие, что вершины $e$ покрашены в одинаковый цвет, а вероятность этого события уже $P(A_e) = r (1 / r)^n = r^{1 - n}$. Оценку на число зависимых с каким-то выбранным событий можно записать как $d \le n \cdot (\Delta - 1)$. Снова остаётся разобраться с неравенством из симметричного случая ЛЛЛ:
	\[
		e(n(\Delta - 1) + 1) \cdot r^{1 - n} \le 1 \Lra e(n(\Delta - 1) + 1) \le r^{n - 1} \Lra r \ge \sqrt[n - 1]{e(n(\Delta - 1) + 1)}
	\]
	То есть достаточно взять за $r$ округлённую вверх целую часть от правого выражения, чтобы найти хорошую раскраску.
\end{proof}

\begin{note}
	Можно сравнить полученный результат в случае обыкновенных графов (то есть 2-однородных гиперграфов) с тривиальной оценкой: $\chi(G) \le \Delta(G) + 1$. Тогда продвинутый результат говорит, что $\chi(G) \le \sqrt{2e\Delta - 2e + 1}$, что для графа с $\Delta(G) = 3$ даст улучшение в порядка 1.5 раза.
\end{note}

\subsection{Конструктивные оценки снизу числа Рамсея}

\begin{note}
	Большая часть оценок, которая была нами получена, сделана через вероятностный метод. Он позволяет установить существование объекта, но не даёт конкретных примеров. Естественно, профессиональные математики это тоже заметили и стали искать конструкции. Об этих результатах мы говорим далее.
\end{note}

\begin{example}
	Можно вспомнить граф, который возникает как предельный случай теоремы Турана. Если взять $s - 1$ компоненту, каждая из которых является $K_{s - 1}$, то получится граф, в котором $\alpha(G) = s - 1$ и $\omega(G) = s - 1$. Стало быть, $R(s, s) > (s - 1)^2$ (сама оценка глобально смешна, но важна сама конструкция и какую оценку она даёт)
\end{example}

\begin{example}
	Рассмотрим $G(n, 3, 1)$. Тогда мы уже знаем, что $\alpha(G(n, 3, 1)) \le n$, а что можно сказать про $\omega(G(n, 3, 1))$? Его значение можно найти точно:
	\[
		\omega(G) = \floor{\frac{n - 1}{2}} (< n \text{ при } n \ge 8)
	\]
	Как это получается? Мы берём любую из вершин, а оставшиеся 2 это 2 последовательных числа из тех $n - 1$, что остались.
	Как известно, в таком графе $C_n^3 \sim n^3 / 6$ вершин. Если ввести обозначение $s = n + 1$, то получаем конструкцию, доказывающую оценку $R(s, s) > (1 + o(1))\frac{s^3}{6}$
\end{example}

\textcolor{red}{Дальше начинается конструктивная теорема Франкла-Уилсона для чисел Рамсея. В нашей программе это снова вопрос на отл, поэтому отсутствие конспекта понятно}