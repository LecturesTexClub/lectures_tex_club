\textcolor{red}{Тут начинается 8я лекция, и должна быть ещё какая-то часть про СОПы}

\subsection{Размерность Вапника-Червоненкиса и эпсилон-сети}

\begin{note}
	Изначально данная тема появилась для доказательства некоторых оценок в машинном обучении, но в итоге развилась в более красивую и широкую теорию.
\end{note}

\begin{note}
	Рассмотрим следующую задачу на плоскости: даны $n$ точек на плоскости. Рассмотрим множество всех возможных подмножеств этих точек, которые можно выделить, пересекая любые треугольники на плоскости. Верно ли, что тогда мы можем получить любое подмножество? На самом деле нет, достаточно взять такое расположение 4х точек, что 3 из них образуют равносторонний треугольник, а последняя является его центром (тогда нельзя выделить эти 3 вершины треугольника отдельно).
	
	Мы могли бы облегчить свою задачу, сказав, что нас интересуют только те подмножества вершин (или рёбра гиперграфа на плоскости на $n$ вершинах), у которых доля от общего числа вершин не менее $\eps$.
	
	Если нам вдруг повезло, что мы смогли получить все такие рёбра, то далее можно поставить вопрос о минимальном размере СОП для такой системы. Понятно, что тогда размер такой СОП не может превосходить $n - \eps n + 1$
\end{note}

\begin{theorem}
	Имеет место утверждение:
	\[
		\forall n \in \N\ \forall X \subset \R^2 \colon |X| = n\ \forall \eps \in (0; 1)\ \exists \text{СОП мощности } \le \frac{500}{\eps}\log_2 \frac{500}{\eps}
	\]
\end{theorem}

\begin{definition}
	\textit{Ранжированным пространством (ranged space)} мы назовём пару $(X, \cR)$, где $X \subseteq 2^X$.
\end{definition}

\begin{example}
	Если $\mathcal{H}$ --- это все возможные полупространства $\R^n$, то есть ранжированное пространство $(\R^n, \mathcal{H})$
\end{example}

\begin{definition}
	Пусть $(X, \cR)$ --- ранжированное пространство, $A \subseteq X$, $|A| < \infty$ \textit{дробится $\cR$}, если мы можем получить пересечениями $\cR$ с $A$ все возможные подмножества $A$:
	\[
		\{r \cap A \colon r \in \cR\} = 2^A
	\]
\end{definition}

\begin{definition}
	Пусть $(X, R)$ --- ранжированное пространство. Тогда \textit{размерностью Вапника-Червоненкиса этого пространства} называется мощность максимального дробящегося при помощи $\cR$ подмножества этого пространства:
	\[
		VC(X, R) = \sup\{m \in \N \colon \exists A \subseteq X, |A| = m \wedge A \text{ дробится } \cR\}
	\]
\end{definition}

\begin{example}
	Исследуем размерность Вапника-Червоненкиса для $(\R^n, \mathcal{H})$:
	\begin{itemize}
		\item $n = 1$ Тут размерность 2, потому что для 3х точек мы уже не сможем при помощи лучей отделить только центральную
		
		\item $n = 2$ Тут размерность 3, потому что для 4х точек общего положения (это такие, где никакие 3 не лежат на одной прямой) возможно как минимум 2 контрпримера:
		\begin{enumerate}
			\item 4 точки образуют что-то похожее на прямоугольник. Тогда мы не сможем отделить подмножество из диагональных точек
			
			\item 4 точки образуют уже упомянутую конструкцию с равнобедренным треугольником и центром. Тогда мы не сможем отделить этот самый центр
		\end{enumerate}
		
		\item $n = 3$
	\end{itemize}
\end{example}

\begin{theorem} (Радона, без доказательства)
	Пусть $\vv{x}_1, \ldots, \vv{x}_m$, $m \ge n + 2$ --- произвольные точки в $\R^n$. Тогда существуют подмножества $A, B$ этих точек такие, что
	\begin{enumerate}
		\item $A \cup B = \{\vv{x}_1, \ldots, \vv{x}_m\}$
		
		\item $A \cap B = \emptyset$
		
		\item $A \neq \emptyset$
		
		\item $B \neq \emptyset$
		
		\item $\conv A \cap \conv B \neq \emptyset$, где $\conv$ --- это выпуклая оболочка
	\end{enumerate}
	Иными словами, можно найти такие 2 непересекающихся многогранника, которые разобьют наше множество на 2.
\end{theorem}

\begin{lemma}
	Пусть $(X, \cR)$ --- ранжированное пространство, причём $VC(X, \cR) = d$, $|X| = n$. Тогда $\cR \le g(n, d) := \sum_{k = 0}^d C_n^k$
\end{lemma}

\begin{proof}
	Воспользуемся индукцией по $(n, d)$:
	\begin{itemize}
		\item База $d = 0$: такое возможно тогда и только тогда, когда $\cR = \emptyset$.
		
		\item Переход $d > 0$: несложно заметить, что для $g$ верна формула треугольника Паскаля: \(g(n, d) = g(n - 1, d) + g(n, d - 1)\). Это является подсказкой, к чему нужно сводить доказательство перехода. Посмотрим на ранжированные пространства $(X \bs \{x\}, \cR_1)$ и $(X \bs \{x\}, \cR_2)$, где
		\begin{align*}
			&{\cR_1 = \{r \bs \{x\}, r \in \cR\}}
			\\
			&{\cR_2 = \{r \in \cR \colon x \notin r, \{x\} \cup r \in \cR\}}
		\end{align*}
		Тогда должно быть тривиально, что $|\cR| = |\cR_1| + |\cR_2|$. Тогда надо показать, что полученные ранжированные пространства удовлетворяют нашей лемме и оцениваются ровно теми числами, что записаны в равенстве для $g$.
		\begin{itemize}
			\item Несложно понять, что $VC(X \bs \{x\}, \cR_1) \le d$. Тогда, по предположению индукции мы сразу получаем требуемое $|\cR_1| \le g(n - 1, d)$.
			
			\item Предположим противное, что $VC(X \bs \{x\}, \cR_2) = d$. Тогда, существует $A \subseteq X \bs \{x\}$ такое, что $|A| = d$ и $A$ дробится $\cR_2$. Покажем, что $A \cup \{x\}$ дробится $\cR$. Действительно, проблема возникает лишь с теми подмножествами, которые включают $x$. Однако $r \in \cR_2 \Lra r \cup \{x\} \in \cR_2$, поэтому мы вначале находим $r$ для подмножества без $x$, а потом везде его добавляем. Получили множество мощности $d + 1$, которое дробится $\cR$, а это противоречие с определением размерности Вапника-Червоненкиса.
		\end{itemize}
	\end{itemize}
\end{proof}

\begin{definition}
	Пусть $(X, \cR)$ --- ранжированное пространство, $A \subset X$. Тогда \textit{проекцией $\cR$ на $A$} назовём следующее множество:
	\[
		\pr_A(\cR) = \{r \cap A \colon r \in \cR\}
	\]
\end{definition}

\begin{corollary}
	В условиях леммы мы установили, что $|\pr_A(\cR)| \le g(n, d)$
\end{corollary}

\begin{definition}
	Пусть $(X, \cR)$ --- ранжированное пространство, $h \in \N, h \ge 2$. Тогда назовём \textit{$h$-измельчением $\R$} следующую систему множеств:
	\[
		\R_h := \{r_1 \cap \ldots \cap r_h \colon \forall i \in \range{h}\ r_i \in \cR\}
	\]
\end{definition}

\begin{lemma}
	Пусть $VC(X, \cR) = d$, $h \in \N$, $h \ge 2$. Тогда $VC(X, \R_h) \le 2dh\log_2(dh)$
\end{lemma}

\begin{proof}
	Пусть $A \subseteq X$, $|A| = n$ и $A$ дробится $\R_h$.
	\begin{itemize}
		\item С одной стороны, $|\pr_A(\R)| \le g(n, d) \le n^d$. Тогда $|\pr_A(\R_h)| \le n^{dh}$.
		
		\item С другой стороны, $|\pr_A(\R_h)| = 2^n$ в силу дробления $A$
	\end{itemize}
	Всё, что утверждаем лемма --- это то, что при $n = 2dh\log_2(dh)$ мы окажемся на всё ещё допустимой границе для размерности $\R_h$.
\end{proof}

\begin{definition}
	Пусть $(X, \cR)$ --- ранжированное пространство, $\eps \in (0; 1)$ и $A \subseteq X$. Назовём \textit{эпсилон-сетью для $A$} любое множество $N \subseteq A$ такое, что $N$ имеет непустое пересечение с каждым элементом $r \cap A$ проекции $\pr_A(\cR)$ таким, что $|r \cap A| \ge \eps|A|$.
\end{definition}

\begin{note}
	Эпсилон-сеть --- это та самая СОП, которую мы пробовали искать вначале в частном случае с треугольниками.
\end{note}

\begin{theorem} (Вапника-Червоненкиса об эпсилон-сетях)
	Если $VC(X, \cR) = d$, то
	\[
		\forall \eps > 0\ \forall A \subseteq X\ \exists \text{$\eps$-сеть } N \colon |N| \le \frac{8d}{\eps} \log_2 \frac{8d}{\eps}
	\]
\end{theorem}

\begin{note}
	Теорема для СОП треугольников --- это просто оценка по лемме 2 размерности $VC(\R^2, \mathcal{H}_3)$, а затем применение уже теоремы Вапника-Червоненкиса
\end{note}

\begin{proof}
	\textcolor{red}{Спасибо, но это вопрос на отл10}
\end{proof}