\begin{definition}
	Говорят, что \textit{$f$ удовлетворяет условию Гёльдера порядка $\alpha \in \rsi{0; 1}$} в точке $x_0$, если выполнены условия:
	\begin{enumerate}
		\item $\exists f(x_0 \pm 0) := \lim_{x \to x_0 \pm 0} f(x)$
		
		\item $\exists \delta > 0, C > 0 \such \forall t \in (0; \delta)\ |f(x_0 + t) - f(x_0 + 0)| < Ct^\alpha \wedge |f(x_0 - t) - f(x_0 - 0)| < Ct^\alpha$
	\end{enumerate}
\end{definition}

\begin{note}
	Далее \textit{односторонними производными} мы будем называть несколько необычные пределы:
	\[
		f'_+(x_0) = \lim_{t \to 0+} \frac{f(x_0 + t) - f(x_0 + 0)}{t}; \quad f'_-(x_0) = \lim_{t \to 0+} \frac{f(x_0 - t) - f(x_0 - 0)}{t}
	\]
	Несложно заметить, что существование таких односторонних производных приводит $f$ к тому, что она удовлетворяет условию Гёльдера при $\alpha = 1$ (оно ещё называется \textit{условием Липшица})
\end{note}

\begin{theorem} (Признак Липшица)
	Если $f \in L_{2\pi}$ и удовлетворяет условию Гёльдера порядка $\alpha$ в точке $x_0$, то тригонометрический ряд Фурье функции $f$ сходится к $\frac{1}{2}(f(x_0 + 0) - f(x_0 - 0))$ в точке $x_0$.
\end{theorem}

\begin{note}
	В частности, тригонометрический ряд сходится к $f(x_0)$, если $f$ непрерывна в $x_0$
\end{note}

\begin{proof}
	Проверим условия признака Дини с $s = \frac{1}{2}(f(x_0 + 0) - f(x_0 - 0))$, для этого посмотрим на соответствующий интеграл:
	\begin{multline*}
		\int_{[0; \delta]} \phi_x\ps{t, \frac{f(x_0 + 0) - f(x_0 - 0)}{2}}d\mu(t) =
		\\
		\int_{[0; \delta]} \frac{f(x_0 + t) - f(x_0 + 0)}{t}d\mu(t) + \int_{[0; \delta]} \frac{f(x_0 - t) - f(x_0 - 0)}{t}d\mu(t)
	\end{multline*}
	Остаётся воспользоваться условиём Гёльдера и оценить интегралы:
	\[
		\md{\int_{[0; \delta]} \frac{f(x_0 \pm t) - f(x_0 \pm 0)}{t}d\mu(t)} \le \int_{[0; \delta]} \frac{Ct^\alpha}{t}d\mu(t) < \infty
	\]
	Пояснение к последнему переходу: да, это почти всегда несобственный интеграл с особенностью в нуле. Однако, по теореме Леви можно показать, что значение этого интеграла всё же конечно.
\end{proof}

\begin{theorem} (Признак Жордана)
	Если функция $f \in L_{2\pi}$ имеет ограниченную вариацию на $[a; b]$, $b - a \le 2\pi$, то в любой точке $x \in (a; b)$ тригонометрический ряд Фурье функции $f$ сходится к $\frac{1}{2}(f(x - 0) + f(x + 0))$. Если, кроме того, $f$ непрерывна на $[a; b]$, то соответствующий тригонометрический ряд Фурье равномерно сходится к $f$ на любом подотрезке $[a'; b'] \subset (a; b)$.
\end{theorem}

\begin{reminder}
	Любая функция ограниченной вариации представима как разность двух монотонных, а любая монотонная функция имеет разрывы не более, чем первого рода (то есть односторонние пределы всегда есть).
\end{reminder}

\begin{proof}
	По основе признака Дини, тригонометрический ряд Фурье сходится к $\frac{1}{2}(f(x - 0) + f(x + 0))$ тогда и только тогда, когда
	\[
		\exists \delta > 0 \such \int_{[0; \delta]} \frac{f(x + t) + f(x - t) - (f(x - 0) + f(x + 0))}{t}\sin(nt)d\mu(t) \xrightarrow[n \to \infty]{} 0
	\]
	Коль скоро $x \in (a; b)$, то существует такое $\delta > 0$, что $x \pm \delta \in [a; b]$. Без ограничения общности, пусть $f$ неубывает. Зафиксируем $\eps > 0$. Тогда
	\[
		\exists \delta_1 \in (0; \delta) \such \forall t \in \rsi{0; \delta_1}\ \ f(x + t) - f(x + 0) < \eps
	\]
	Применим вторую теорему о среднем к следующему интегралу:
	\[
		\exists \delta_2 \in [0; \delta_1] \such \int_0^{\delta_1} \frac{f(x + t) - f(x + 0)}{t}\sin(nt)dt = (f(x + \delta_1) - f(x + 0)) \int_{\delta_2}^{\delta_1} \frac{\sin(nt)}{t}dt
	\]
	Чтобы оценить последний интеграл, нам нужно убрать $n$ из подыинтегрального выражения и воспользоваться известной сходимостью:
	\begin{enumerate}
		\item $\int_0^{+\infty} \frac{\sin t}{t}dt$ --- сходится. Значит:
		\[
			\exists C > 0 \such \forall A > 0\ \ \md{\int_0^A \frac{\sin t}{t}dt} < C
		\]
		
		\item \(\md{\int_0^{\delta_1} \frac{\sin(nt)}{t}dt} = \md{\int_0^{n\delta_1} \frac{\sin(u)}{u}du} < C\)
	\end{enumerate}
	Всё это позволяет оценить модуль интеграла следующим образом:
	\[
		\md{\int_0^{\delta_1} \frac{f(x + t) - f(x + 0)}{t}\sin(nt)dt} = \md{(f(x + \delta_1) - f(x + 0)) \int_{\delta_2}^{\delta_1} \frac{\sin(nt)}{t}dt} <
		\\
		\eps \cdot 2C
	\]
	С другой стороны, интеграл от $\delta_1$ до $\delta$ будет стремится к нулю при $n \to \infty$ по теореме Римана об осциляции. Стало быть:
	\[
		\exists N \in \N \such \forall n > N\ \ \int_0^{\delta} \frac{f(x + t) - f(x + 0)}{t} \sin(nt)d\mu(t) < \eps(1 + 2C)
	\]
	Первая часть доказана. Далее $f \in C[a; b]$ (помним, что по теореме Кантора $f \in \hat{C}[a; b]$) и мы покажем равномерную сходимость. По основе признака Дини, нам нужен такой предел:
	\[
		\int_{[0; \delta]} \frac{f(x + t) + f(x - t) - 2f(x)}{t}\sin(nt)d\mu(t) \rra 0,\ n \to \infty
	\]
	Доказательство по сути повторяет то, что мы уже сделали ранее. Только вот теперь $\delta_1$ такая:
	\[
		\exists \delta_1 \in (0; \delta) \such \forall x \in [a'; b']\ \forall t \in \rsi{0; \delta_1}\ \ f(x + t) - f(x + 0) < \eps
	\]
	Тогда интеграл от $\delta_1$ до $\delta$ равномерно сходится к нулю, а оценка на интеграл от 0 до $\delta_1$ остаётся старой (она и так не зависит от функции)
\end{proof}

\begin{definition}
	\textcolor{red}{Определение кусочно гладкой функции?}
\end{definition}

\begin{theorem} (Оценка скорости сходимости тригонометрического ряда Фурье)
	Если $f$ --- непрерывная, кусочно гладкая и $2\pi$-периодическая, то тригонометрический ряд Фурье этой функции сходится к $f$ равномерно на $\R$, причём справедлива такая оценка:
	\[
		\exists C = C(f) > 0 \such |S_n(f, x) - f(x)| \le C\frac{\ln n}{n}
	\]
\end{theorem}

\begin{proof}
	Распишем разность между частичной суммой и функцией через интеграл с ядром Дирихле:
	\[
		S_n(f, x) - f(x) = \frac{1}{\pi} \ps{\int_0^\delta + \int_\delta^\pi} \underbrace{\frac{f(x + t) + f(x - t) - 2f(x)}{2\sin\frac{t}{2}}}_{g_x(t)}\sin\ps{\ps{n + \frac{1}{2}}t}d\mu(t)
	\]
	Обозначим соответствующие интегралы как $I_{1, 2}$. Для первого интеграла оценим $|g_x(t)|$:
	\begin{itemize}
		\item $M = \max_{x \in \R} |f'(x)|$, подразумевая вместо производной в точках разрыва её односторонние версии
		
		\item По теореме Лагранжа имеет место оценка $|f(x + t) - f(x)| \le Mt$ (если на отрезке $[x; x + t]$ есть $N$ точек разрыва производной, то расписываем модуль по неравенству треугольника на сумму отрезков непрерывности. Их общая оценка даёт ту же оценку)
		
		\item Имеет место неравенство $\frac{2}{\pi}\alpha \le \sin \alpha \le \alpha$ для $0 \le \alpha \le \pi / 2$. Покажем, почему верна оценка снизу. Для этого рассмотрим производную функции $\sin \alpha / \alpha$ при соответствующих значениях аргумента:
		\[
			\ps{\frac{\sin \alpha}{\alpha}}' = \frac{\cos \alpha}{\alpha^2}(\alpha - \tg \alpha) < 0,\ 0 < \alpha < \pi / 2
		\]
		С учётом того, что в нуле у этой функции максимум, то при $\pi / 2$ получаем минимум на отрезке, отсюда и оценка:
		\[
			\frac{\sin \alpha}{\alpha} \ge \frac{\sin(\pi / 2)}{\pi / 2} = \frac{2}{\pi}
		\]
	\end{itemize}
 	С учётом всего вышесказанного, имеет место оценка:
 	\[
 		|g_x(t)| \le \frac{2Mt}{2 \cdot \frac{2}{\pi} \cdot \frac{t}{2}} = \pi M \Lora |I_1| \le \frac{1}{\pi} \cdot \pi M \cdot \delta = M\delta
 	\]
 	Второй интеграл возьмём по частям:
 	\[
 		\int_\delta^\pi g_x(t)\sin\ps{(n + 1 / 2)t}dt = -\frac{g_x(t)\cos((n + 1 / 2)t)}{n + 1 / 2}\Big|_\delta^\pi + \int_\delta^\pi \frac{d}{dt}(g_x(t)) \frac{\cos((n + 1 / 2)t)}{n + 1 / 2}dt
 	\]
 	Оценим возникшую производную:
 	\begin{multline*}
 		\md{\frac{d}{dt}(g_x(t))} =
 		\\
 		\md{\frac{(f'(x + t) - f'(x - t))\sin(t / 2) - (f(x + t) + f(x - t) - 2f(x))\frac{1}{2}\cos(t / 2)}{2\sin^2(t / 2)}} \le
 		\\
 		\frac{M}{2 \cdot \frac{2}{\pi} \cdot \frac{t}{2}} + \frac{Mt}{2 \cdot \frac{4}{\pi^2} \cdot \frac{t^2}{4}} \le \frac{M_1}{t}
 	\end{multline*}
 	Итого, для интеграла $I_2$ верна оценка:
 	\[
 		|I_2| \le \frac{1}{\pi} \cdot \ps{\frac{1}{n + 1 / 2} \cdot 2\pi M + \frac{M_1}{n + 1 / 2} \cdot (\ln \pi - \ln \delta)} \le \frac{2M}{n} + \frac{M_1}{n}(\ln \pi + \ln(1 / \delta))
 	\]
 	Общая оценка получается такой:
 	\[
 		|S_n(f, x) - f(x)| \le |I_1| + |I_2| \le M\delta  + \frac{2M}{n} + \frac{M_1}{n}(\ln \pi + \ln(1 / \delta))
 	\]
 	Положив $\delta = 1 / n$, получим требуемое.
\end{proof}

\subsection{Действие с рядами Фурье}

\begin{theorem} (Почленное дифференцирование рядов Фурье) Если $f$ --- это $2\pi$-периодичекая, абсолютно непрерывная на любом периоде функция, то тригонометрический ряд Фурье функции $f'$ получается почленным дифференцированием тригонометрического ряда Фурье функции $f$.
\end{theorem}

\begin{proof}
	В силу абсолютной непрерывности $f$, её производная существует и непрерывна почти всюду. Стало быть, $f' \in L_1[-\pi; \pi]$. Распишем по стандартным формулам коэффициенты, которыми должен обладать ряд Фурье для производной:
	\[
		a_n(f') = \frac{1}{\pi} \int_{[-\pi; \pi]} f'(x)\cos(nx)d\mu(x) = \underbrace{\frac{1}{\pi} f(x)\cos(nx)\Big|_{-\pi}^\pi}_{0} + \frac{n}{\pi} \int_{[-\pi; \pi]} f(x)\sin(nx)d\mu(x) = nb_n(f)
	\]
	Аналогичная формула верна и для $b_n(f') = -na_n(f)$. Остаётся подставить коэффициенты по итоговым формулам в ряд:
	\[
		f'(x) \sim \sum_{n = 1}^\infty (-na_n(f)\sin(nx) + nb_n(f)\cos(nx)) = \ps{\frac{a_0(f)}{2} + \sum_{n = 1}^\infty (a_n(f)\cos(nx) + b_n(f)\sin(nx))}'
	\]
\end{proof}

\begin{corollary}
	Если $f$ --- это $2\pi$-периодическая функция такая, что $f', \ldots, f^{(k - 1)}$ абсолютно непрерывны на любом периоде и $2\pi$-периодичны, то верны оценки:
	\begin{align*}
		&{a_n(f) = o\ps{\frac{1}{n^k}}, n \to \infty}
		\\
		&{b_n(f) = o\ps{\frac{1}{n^k}}, n \to \infty}
	\end{align*}
\end{corollary}

\begin{proof}
	Если мы посчитаем $a_n(f^{(k)})$ или $b_n(f^{(k)})$, то индуктивно получим равенство, которое с модулем запишется как \(|a_n(f^{(k)})| = n^k|g_k|\), где $g_k$ --- это соответствующий коэффициент из $\{a_n(f), b_n(f)\}$. Стало быть, мы можем выразить коэффициенты ряда Фурье в этом равенстве и получить требуемую оценку (потому что коэффициент тригонометрического ряда Фурье для высшей производной тоже сходится к нулю по теореме Римана об осцилляции).
\end{proof}

\begin{note}
	В частных случаях оценку на коэффициенты можно улучшить. Например, как показывает слеудющая теорема, это возможно для функций ограниченной вариации.
\end{note}

\begin{theorem} (Оценка коэффициентов Фурье функции ограниченной вариации)
	Если $f$ --- это $2\pi$-периодическая функция, обладающая ограниченной вариацией на любом периоде, то верны оценки:
	\begin{align*}
		&{|a_n(f)| = O\ps{\frac{1}{n}}, n \to \infty}
		\\
		&{|b_n(f)| = O\ps{\frac{1}{n}}, n \to \infty}
	\end{align*}
\end{theorem}

\begin{proof}
	Для начала стоит объяснить, почему мы вообще имеем право говорить об интегрировании такой функции. Действительно, если взять $f$ на своём периоде, то за счёт ограниченной вариации она представима в виде разности двух неубывающих функций. Как известно, монотонная функция имеет не более чем счётное число разрывов первого рода, то есть интегрируема на этом отрезке. Значит, интегрируема и сама $f$. Теперь, будем аккуратно сводить оценку модуля коэффициента $|a_n(f)|$ к неравенству с полной вариацией:
	\begin{multline*}
		|a_n(f)| = \frac{1}{\pi}\md{\int_{-\pi}^\pi f(x)\cos(nx)dx} = \{1\} =
		\\
		\md{-\frac{1}{2\pi} \int_{-\pi}^\pi \ps{f\ps{x + \frac{\pi}{n}} - f(x)}\cos(nx)dx} = \{2\} =
		\\
		\frac{1}{2\pi} \md{\int_{-\pi}^\pi \ps{f\ps{x + \frac{k\pi}{n}} - f\ps{x + \frac{(k - 1)\pi}{n}}}dx} \le
		\\
		\frac{1}{2\pi} \int_{-\pi}^\pi \md{f\ps{x + \frac{k\pi}{n}} - f\ps{x + \frac{(k - 1)\pi}{n}}}dx
	\end{multline*}
	Пояснения к действиям:
	\begin{enumerate}
		\item Сделали замену $x = t + \pi / n$, сдвинули пределы интегрирования обратно к $[-\pi; \pi]$, взяли среднее от двух интегралов
		
		\item Повторили то же самое ещё произвольное $k - 1$ число раз, но среднее уже не брали
	\end{enumerate}
	Итак, посмотрим на $n|a_n(f)|$ и соберём вариацию:
	\[
		n|a_n(f)| \le \frac{1}{2\pi} \int_{-\pi}^\pi \sum_{k = 1}^n \md{f\ps{x + \frac{k\pi}{n}} - f\ps{x + \frac{(k - 1)\pi}{n}}}dx \le V(f)
	\]
	Из этого неравенства очевидна оценка на коэффициент ряда Фурье
\end{proof}

\begin{theorem} (Лебега об интегрировании рядов Фурье)
	Если $f \in L_{2\pi}$ имеет тригонометрический ряд Фурье $\frac{a_0}{2} + \sum_{n = 1}^\infty (a_n\cos(nx) + b_n\sin(nx))$, то неопределенный интеграл Лебега $F(x) = \int_{[x_0; x]} f(t)d\mu(t)$ представим по следующей формуле:
	\[
		F(x) = \frac{a_0x}{2} + C + \sum_{n = 1}^\infty \frac{a_n\sin(nx) - b_n\cos(nx)}{n}
	\]
	где ряд в правой части является рядом Фурье для функции $F(x) - \frac{a_0x}{2}$ и сходится равномерно на $[-\pi; \pi]$.
\end{theorem}

\begin{note}
	Важно отметить, что этот ряд для интеграла будет сходится независимо от того, сходится ли ряд для $f$.
\end{note}

\begin{proof}
	Уже известен факт, что $F(x)$ --- абсолютно непрерывная функция. Более того, $F(x) - \frac{a_0x}{2}$ --- тоже абсолютно непрерывная, причём ещё и $2\pi$-периодическая. Проверим это:
	\[
		F(\pi) - F(-\pi) = \int_{-\pi}^\pi f(t)d\mu(t) = \pi a_0 \Lora \ps{F(x) - \frac{a_0x}{2}}\Big|_{-\pi}^\pi = 0
	\]
	Разложим её в ряд Фурье:
	\[
		F(x) - \frac{a_0}{2}x = \frac{A_0}{2} + \sum_{n = 1}^\infty (A_n\cos(nx) + B_n\sin(nx))
	\]
	Теперь применим теорему о почленном дифференцировании:
	\[
		f(x) - \frac{a_0}{2} = \ps{F(x) - \frac{a_0}{2}x}' \sim \sum_{n = 1}^\infty (-A_n\sin(nx) + B_n\cos(nx))
	\]
	Из соотношения ниже получаем связь между коэффициентами рядов:
	\[
		f(x) \sim \frac{a_0}{2} + \sum_{n = 1}^\infty (-A_n\sin(nx) + B_n\cos(nx))
	\]
\end{proof}