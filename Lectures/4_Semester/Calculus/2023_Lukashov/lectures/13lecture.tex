\begin{theorem} (Дифференцирование преобразования Фурье)
	Если $f(x)$ и $xf(x)$ лежат в классе $L_1(\R)$, то $\wdh{f}(\lambda)$ будет дифференцируема всюду на $\R$, причём верны равенства:
	\[
		(\wdh{f})'(\lambda) = \wdh{(-ixf(x))}(\lambda);\ \ \lim_{\lambda \to \infty} |\wdh{f}'(\lambda)| = 0
	\]
\end{theorem}

\begin{proof}
	Мы имеем дело с несобственным интегралом:
	\[
		\wdh{f}(\lambda) = \frac{1}{\sqrt{2\pi}} \int_\R f(x)e^{-\lambda x}d\mu(x)
	\]
	Поэтому формально происходит следующее: интеграл по $\R$ расписывается как предел, дальше мы получаем собственные интегралы, где подыинтегральная функция суммируема и есть оценка на производную:
	\[
		\md{\pd{}{\lambda}\ps{f(x)e^{-\lambda x}}} = |f(x)(-ix)e^{-\lambda x}| \le |xf(x)|,\ |xf(x)| \in L_1(\R)
	\]
	Стало быть, просто используем теорему о дифференцируемости собственного интеграла по параметру под пределом и возвращаем предел в форму интеграла:
	\[
		(\wdh{f})'(\lambda) = \frac{1}{\sqrt{2\pi}} \int_\R (-ix)f(x)e^{-i\lambda x}d\mu(x) = (\wdh{-ixf(x)})(\lambda)
	\]
	По свойству преобразования Фурье мы также заключаем, что $\lim_{\lambda \to \infty} |(\wdh{f})'(\lambda)| = 0$.
\end{proof}

\begin{definition}
	\textit{Свёрткой функций} $f, g \in L_1(\R)$ называется функция $f * g$, определяемая поточечно следующим образом:
	\[
		(f * g)(x) = \int_\R f(t)g(x - t)d\mu(t)
	\]
\end{definition}

\begin{proposition}
	Если $f, g \in L_1(\R)$, то и $f * g \in L_1(\R)$
\end{proposition}

\begin{proof}
	Воспользуемся хитрым фактом: $f(t) \cdot g(y) \in L_1(\R^2)$. Стало быть, применима теорема Фубини:
	\begin{multline*}
		\int_{\R^2} f(t)g(y)d\mu(t, y) = \int_\R f(t)d\mu(t) \cdot \int_\R g(y)d\mu(y) = \{y = x - t\} =
		\\
		\int_\R f(t) \int_\R g(x - t)d\mu(x)d\mu(t) = \int_\R d\mu(x) \int_\R f(t)g(x - t)d\mu(t) \Lora f * g \in L_1(\R)
	\end{multline*}
\end{proof}

\begin{theorem} (Преобразование Фурье для свёртки)
	Если $f, g \in L_1(\R)$, то имеет место равенство:
	\[
		\wdh{f * g}(\lambda) = \sqrt{2\pi}\wdh{f}(\lambda) \cdot \wdh{g}(\lambda)
	\]
\end{theorem}

\begin{proof}
	Подставим определение свёртки в определение преобразования Фурье, разделим интегралы:
	\begin{multline*}
		\wdh{f * g}(\lambda) = \frac{1}{\sqrt{2\pi}} \int_\R (f * g)(x)e^{-i\lambda x}d\mu(x) = \frac{1}{\sqrt{2\pi}} \int_\R \int_\R f(t)g(x - t)d\mu(t)e^{-i\lambda x}d\mu(x) =
		\\
		\frac{1}{\sqrt{2\pi}} \int_\R f(t) \int_\R g(x - t)e^{-i\lambda x}d\mu(x)d\mu(t) =
		\\
		\sqrt{2\pi} \cdot \frac{1}{\sqrt{2\pi}} \cdot \int_\R f(x)e^{-i\lambda t}d\mu(t) \cdot \frac{1}{\sqrt{2\pi}} \int_\R g(y)e^{-i\lambda y}d\mu(y) = \sqrt{2\pi} \wdh{f}(\lambda) \wdh{g}(\lambda)
	\end{multline*}
\end{proof}

\begin{example}
	Пусть $f(x) = e^{-\alpha x^2},\ \alpha > 0$. Найдём $\wdh{f}$:
	\textcolor{red}{Дописать}
\end{example}

\begin{note}
	Преобразование Фурье имеет различные применения. Рассмотрим 2 частых:
	\begin{enumerate}
		\item Решение уравнения теплопроводности: $\pd{u}{t} = a^2 \pd{^2u}{x^2},\ u(x, 0) = f(x),\ a > 0$. Применим к обеим частям преобразование Фурье:
		\[
			\pd{\wdh{u}}{t} = a^2 (-\lambda^2) \wdh{u}(\lambda, t)
		\]
		отсюда $\wdh{u}(\lambda, t) = C(\lambda) e^{-a^2\lambda^2 t}$. После преобразования Фурье начальное условие принимает вид $\wdh{u}(\lambda, 0) = \wdh{f}(\lambda)$, то есть
		\begin{multline*}
			\wdh{u}(\lambda, t) = \wdh{f}(\lambda) e^{-a^2\lambda^2 t} \Ra
			\\
			u(x, t) = F^{-1}[\wdh{f}(\lambda) e^{-a^2\lambda^2 t}](x, t) = \frac{1}{\sqrt{2\pi}} f * \wdt{e^{-a^2\lambda^2 t}} = \frac{1}{\sqrt{2\pi}} f * \ps{\frac{1}{\sqrt{2ta^2}} \cdot e^{-x^2 / (4a^2 t)}}
		\end{multline*}
		
		\item Теорема Котельникова (Найквиста-Шеннона, теорема отсчётов): для восстановления сигнала $f(x)$ с финитным спектром, сосредоточенным в полосе частот $|\omega| \le a$, достаточно передать его значения в точках $k\Delta$, где $k \in \Z$ и $\Delta = \pi / a$, причём верно равенство:
		\[
			f(x) = \sum_{k = -\infty}^{+\infty} f\ps{k\frac{\pi}{a}} \frac{\sin\ps{x - \frac{k\pi}{a}}}{a\ps{x - \frac{k\pi}{a}}}
		\]
		\textcolor{red}{Вряд ли кто-то в здравом уме спросит ФИВТа доказательство матфиз теоремы, ибо никто нам пока не рассказывал ни финитный спектр, ни понятие полосы частот. Пока что отложено}
	\end{enumerate}
\end{note}

\subsection{Пространство Шварца $S$}

\begin{definition}
	Будем называть функцию $f \colon \R \to \R$ \textit{быстро убывающей}, если выполнено утверждение:
	\[
		\forall k, n \in \N_0\ \ \exists C_{k, n} \such \forall x \in \R\ (1 + |x|^k)f^{(n)}(x) \le C_{k, n}
	\]
\end{definition}

\begin{anote}
	Иначе гворя, любая производная быстро убывающей функции убывает быстрее любого полинома.
\end{anote}

\begin{definition}
	Множество быстро убывающих функций $S$ называется \textit{пространством Шварца}.
\end{definition}

\begin{theorem}
	Преобразование Фурье является биекцией на $S$
\end{theorem}

\begin{proof}
	\textcolor{red}{Нужно больше формализма}
\end{proof}

\begin{theorem} (Формулы Парсеваля)
	Если $f_1, f_2 \in S$, то имеют место следующие формулы:
	\begin{enumerate}
		\item \(\int_{-\infty}^\infty \wdh{f}_1(x)f_2(x)dx = \int_{-\infty}^{+\infty} f_1(x)\wdh{f}_2(x)dx\)
		
		\item \(\int_{-\infty}^{+\infty} f_1(x)\overline{f_2(x)}dx = \int_{-\infty}^{+\infty} \wdh{f}_1(\lambda)\overline{\wdh{f}_2(\lambda)}d\lambda\)
	\end{enumerate}
\end{theorem}

\begin{proof}~
	\begin{enumerate}
		\item Нужно просто раскрыть преобразования Фурье в равенстве справа и слева и удостоверится, что полученное новое равенство действительно выполено. \textcolor{red}{Расписать подробнее}
		
		\item Сведём всё к первому равенству, введя такую $h(x)$, то $\wdh{h}(x) = \overline{f_2(x)}$. \textcolor{red}{Дописать}
	\end{enumerate}
\end{proof}