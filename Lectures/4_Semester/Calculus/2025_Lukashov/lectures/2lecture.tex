\begin{lemma} \label{average_continuous}
	Каждая суммируемая на $\R$ функция $f \colon \R \to \R$ непрерывна в среднем, то есть имеет место предел:
	\[
		\exists \lim_{t \to 0} \int_\R |f(x + t) - f(x)|d\mu(x) = 0
	\]
\end{lemma}

\begin{proof}[Доказательство 1]
	Разберёмся с простым случаем, а сложный сведём к нему:
	\begin{enumerate}
		\item $f \in L_1[a; b]$. Докажем наличие такого предела:
		\[
			\exists \lim_{t \to 0} \int_{[a + t; b - t]} |f(x + t) - f(x)|d\mu(x) = 0
		\]
		По лемме о всюду плотности множества непрерывных функций верно следующее:
		\[
			\forall \eps > 0\ \exists g \in C[a; b] \such \int_{[a; b]} |f(x) - g(x)|d\mu(x) < \frac{\eps}{3}
		\]
		Коль скоро $g$ будет непрерывной на отрезке, то по теореме Кантора она ещё и равномерно непрерывна:
		\[
			\forall \eps > 0\ \exists \delta > 0 \such \forall x, y \in [a; b], |x - y| < \delta\ \ |g(x) - g(y)| < \frac{\eps}{3(b - a)}
		\]
		Рассмотрим достаточно малое $-\delta < t < \delta$ и оценим соответствующий интеграл:
		\begin{multline*}
			\int_{[a + t; b - t]} |f(x + t) - f(x)|d\mu(x) \le \int_{[a + t; b - t]} |f(x + t) - g(x + t)|d\mu(x) +
			\\
			\int_{[a + t; b - t]} |g(x + t) - g(x)|d\mu(x) + \int_{[a + t; b - t]} |g(x) - f(x)|d\mu(x) < \eps
		\end{multline*}
		Оценка первого и последнего интегралов как $\eps / 3$ по отдельности следует из выбора $g$, а интеграл посередине оценивается так же из-за равномерной непрерывности.
	
		\item $f \in L_1(\R)$ --- сведём этот случай к уже доказаному выше. Из суммируемости на множестве бесконечной меры следует, что
		\[
			\forall \eps > 0\ \exists N \in \N \such \int_{\R \bs [-N; N]} |f(x)|dx < \frac{\eps}{3}
		\]
		Зафиксируем $\eps > 0$ и соответствующее $N \in \N$. Для произвольного $-1/2 < t < 1/2$ мы можем разбить интеграл на такие два:
		\begin{multline*}
			\int_\R |f(x + t) - f(x)|dx = \int_{[-N - 1 + t; N + 1 - t]} |f(x + t) - f(x)|dx +
			\\
			\int_{\R \bs [-N - 1 + t; N + 1 - t]} |f(x + t) - f(x)|dx
		\end{multline*}
		Первый интеграл уменьшением $t$ доводится до $< \eps / 3$ (за счёт уже доказанного выше), а второй интеграл разбивается на два по неравенству треугольника и тоже оцениваются как $\eps / 3$ каждый за счёт свойства выбранного $N$.
	\end{enumerate}
\end{proof}

\begin{proof}[Доказательство 2] (набросок от автора)
	Заметим, что множество финитных функций всюду плотно в $L_1(\R)$ (определение финитности и всюду плотности встречается дальше в конспекте, второе фактически позволяет применить неравенство треугольника, вставив под модуль "промежуточную более хорошую" \ функцию). $f(x+t) - f(x)$ суммируема, поэтому
	\[
		\forall \eps > 0 \ \exists N : \int_{\R \setminus [-N; \ N]} |f(x + t) - f(x)| d\mu(x) < \eps
	\]
	Значит, достаточно доказать утверждение для финитных функций. Дальше разобьём $f$ на положительную и отрицательную части:
	\[
		|f(x + t) - f(x)| = |f_+(x + t) - f_-(x + t) - f_+(x) 	+ f_-(x)| \leq |f_+(x + t) - f_+(x)| + |f_-(x + t) - f_-(x)|
	\]
	Свели ситуацию к финитным неотрицательным функциям. В тех, в свою очередь, плотны ограниченные функции (по теореме о пределе последовательности интегралов от срезок). А если носитель функции вложен в $[-N; N]$, сама функция ограничена числом $M$, то утверждение для неё очевидно из теоремы  Лебега, ибо можно взять естественную мажоранту:
	\[
		\int_\R |f(x + t) - f(x)| d\mu(x) \leq \int_\R |f(x + t)| + |f(x)| d\mu(x) \leq 4NM
	\]
	Отсюда
	\[
		\lim_{t \to 0} \int_\R |f(x + t) - f(x)|d\mu(x) = \int_\R \lim_{t \to 0} |f(x + t) - f(x)|d\mu(x) = 0
	\]
	Дальше формально нужно много-много раз применить неравенство треугольника, каждый раз приближая друг дружкой вышеописанные функции с точностью порядка $\frac{\eps}{6}$.
\end{proof}

\begin{theorem} (Римана об осцилляции)
	Если $f \in L_1(I)$, где $I$ --- промежуток в $\R$, то существуют следующие пределы:
	\[
		\exists \lim_{\lambda \to \infty} \int_I f(x)\cos(\lambda x)d\mu(x) = \lim_{\lambda \to \infty} \int_I f(x)\sin(\lambda x)d\mu(x) = 0
	\]
\end{theorem}

\begin{proof}
	Интегралы суть одинаковы, поэтому мы покажем доказательство лишь для одного из них (того, что с косинусом). Разберём случаи:
	\begin{itemize}
		\item $I = \R$. Сделаем замену $x = t + \pi / \lambda$:
		\[
			\int_\R f(x)\cos(\lambda x)d\mu(x) = -\int_\R f\ps{t + \frac{\pi}{\lambda}}\cos(\lambda t)d\mu(t)
		\]
		А теперь оценим модуль интеграла, взяз его среднее арифметическое с собой же:
		\begin{multline*}
			\md{\int_\R f(x)\cos(\lambda x)d\mu(x)} = \md{\frac{1}{2}\ps{\int_\R f(t)\cos(\lambda t)d\mu(t) - \int_\R f\ps{t + \frac{\pi}{\lambda}}\cos(\lambda t)d\mu(t)}} = \\
			\md{-\frac{1}{2}\int_\R \ps{f\ps{t + \frac{\pi}{\lambda}} - f(t)}\cos(\lambda t)d\mu(t)} \le \frac{1}{2}\int_\R \md{f\ps{t + \frac{\pi}{\lambda}} - f(t)}d\mu(t) \xrightarrow[\lambda \to \infty]{} 0
		\end{multline*}
		
		\item $I \neq \R$. Тогда дополним $f$ нулём за границами промежутка до $\R$. Понятно, что интеграл по $I$ тогда совпадает с интегралом по $\R$, и мы просто ссылаемся на доказанное в предыдущем пункте.
	\end{itemize}
\end{proof}

\begin{corollary}
	Для любой функции, суммируемой на промежутке длиной $2\pi$, её тригонометрические коэффициенты Фурье образуют бесконечно малые последовательности:
	\[
		\lim_{n \to \infty} a_n = \lim_{n \to \infty} b_n = 0
	\]
\end{corollary}

\begin{proposition}
	Если $f \in L_1[-\pi; \pi]$, то имеет место 2 утверждения:
	\begin{itemize}
		\item Если $f$ --- чётная функция, то $b_n = 0$ и $a_n = 2/\pi \int_0^\pi f(x) \cos(nx) d\mu(x)$. Ряд Фурье в таком случае называется \textit{обобщённым косинус-рядом Фурье}
		
		\item Если $f$ --- нечётная функция, то $a_n = 0$  и $b_n = 2/\pi \int_0^\pi f(x) \sin(nx) d\mu(x)$. Ряд Фурье в таком случае называется \textit{обобщённым синус-рядом Фурье}
	\end{itemize}
\end{proposition}

\begin{proof}
	Посчитаем соответствующие интегралы из ряда Фурье для чётной функции, а для нечётной всё то же самое с заменой $a_n$ на $b_n$. Заметим, что если $f$ --- чётная функция, то $f(x)\sin(nx)$ --- нечётная функция, а $f(x)\cos(nx)$ - чётная. Отсюда следуют равенства с точностью до знака интегралов по левую и правую стороны от нуля:
	\begin{multline*}
		\pi b_n = \int_{-\pi}^\pi f(x)\sin(nx)d\mu(x) = \int_{-\pi}^0 f(x)\sin(nx)d\mu(x) + \int_0^\pi f(x)\sin(nx)d\mu(x) = 0 \\
		\pi a_n = \int_{-\pi}^\pi f(x)\cos(nx)d\mu(x) = \int_{-\pi}^0 f(x)\cos(nx)d\mu(x) + \int_0^\pi f(x)\cos(nx)d\mu(x) = \\ 2 \int_0^\pi f(x)\cos(nx)d\mu(x) 
	\end{multline*}
\end{proof}

\begin{note}
	Более того, формулы коэффициентов работают даже тогда, когда $f \notin L_1[0; \pi]$. Достаточно лишь того, чтобы $f(x)\sin x \in L_1[0; \pi]$. Действительно, мы можем переписать интеграл для $f(x)\sin(nx)$ таким образом:
	\[
		\int_{[0; \pi]} f(x)\sin(nx)d\mu(x) = \int_{[0; \pi]} f(x)\sin x \cdot \ps{\frac{\sin(nx)}{\sin x}}d\mu(x)
	\]
	Функция, которая была выделена в скобки, является композицей \textit{многочлена Чебышёва} и тригонометрической функции:
	\[
		\frac{\sin(nx)}{\sin x} = U_{n - 1}(\cos x)
	\]
\end{note}

\begin{example}
	Посчитаем тригонометрический ряд Фурье для $f(x) = \ctg \frac{x}{2} \in L_1[0; \pi]$. Коль скоро это нечётная функция, то нам нужны лишь $b_n$:
	\[
		b_1 = \frac{2}{\pi} \int_0^\pi \frac{\cos(x / 2)}{\sin(x / 2)} \sin xdx = 1
	\]
	Чтобы вычислить остальные члены ряда, посчитаем разницу между следующим и предыдущим коэффициентом:
	\begin{multline*}
		b_{n + 1} - b_n = \frac{2}{\pi} \int_0^\pi \frac{\cos(x / 2)}{\sin(x / 2)}(\sin((n + 1)x) - \sin(nx))dx =
		\\
		\frac{4}{\pi}\int_0^\pi \frac{\cos(x / 2)}{\sin(x / 2)}(\cos((n + 1/2)x) \cdot \sin(x / 2))dx = \frac{2}{\pi}\int_0^\pi (\cos((n + 1)x) + \cos(nx))dx = 0
	\end{multline*}
	То есть котангенсу половины аргумента сопоставлен очень милый ряд:
	\[
		\ctg \frac{x}{2} \sim \sin(x) + \sin(2x) + \sin(3x) + \ldots
	\]
\end{example}

\subsubsection*{Классические ортогональные многочлены и ортогональные системы, с ними связанные \red{(материал прошлого года)}}

\begin{enumerate}
	\item Зафиксируем $\alpha, \beta > -1$. Тогда рассмотрим систему из многочленов вида $(1 - x)^{\alpha / 2}(1 + x)^{\beta / 2}P_n^{(\alpha, \beta)}(x)$, $n \in \N_0$, где
	\[
		P_n^{(\alpha, \beta)}(x) = c_n(1 - x)^{-\alpha}(1 + x)^{-\beta} \frac{d^n}{dx^n}\ps{(1 - x)^{\alpha + n}(1 + x)^{\beta + n}} \text{ --- \textit{многочлены Якоби}}
	\]
	При этом система рассматривается на $[-1; 1]$.
	\begin{itemize}
		\item При $\alpha = \beta$ элементы этой системы называются \textit{многочленами Гегенбауэра, или же ультрасферическими многочленами}
		
		\item При $\alpha = \beta = 0$ --- \textit{многочлены Лежандра}
		
		\item При $\alpha = \beta = -\frac{1}{2}$ --- \textit{многочлены Чебышёва первого рода}
		
		\item При $\alpha = \beta = \frac{1}{2}$ --- \textit{многочлены Чебышёва второго рода}
	\end{itemize}

	\item Зафиксируем $\alpha > -1$ и рассмотрим систему, состоящую из элементов вида $x^{\alpha / 2}e^{-x / 2} L_n^{(\alpha)}(x)$, $n \in \N_0$, где
	\[
		L_n^{(\alpha)}(x) = c_n x^{-\alpha}e^x \frac{d^n}{dx^n}(x^{\alpha + n}e^{-x}) \text{ --- \textit{многочлены Лагерра}}
	\]
	При этом система рассматривается на $\lsi{0; +\infty}$.
	
	\item Рассмотрим систему, состоящую из элементов вида $e^{-x^2 / 2}H_n(x)$, $n \in \N_0$, где
	\[
		H_n(x) = c_ne^{x^2} \frac{d^n}{dx^n}(e^{-x^2}) \text{ --- \text{многочлены Эрмита}}
	\]
	При этом система рассматривается на $(-\infty; \infty)$
\end{enumerate}

\begin{note} (Комплексная форма тригонометрического ряда Фурье)
	Тригонометрический ряд Фурье можно переписать следующим образом, используя комплексную экспоненту:
	\begin{multline*}
		\frac{a_0}{2} + \sum_{n = 1}^\infty a_n\cos(nx) + b_n\sin(nx) = \frac{a_0}{2} + \sum_{n = 1}^\infty a_n\frac{e^{inx} + e^{-inx}}{2} + b_n\frac{e^{inx} - e^{-inx}}{2i} =
		\\
		\frac{a_0}{2} + \sum_{n = 1}^\infty \ps{\frac{e^{inx}}{2}(a_n - ib_n) + \frac{e^{-inx}}{2}(a_n + ib_n)} = \sum_{-\infty}^{+\infty} c_ne^{inx}
	\end{multline*}
	Бесконечную сумму надо понимать как 2 суммы от 1 до бесконечности и ещё отдельный ноль. Коэффициенты этого ряда выглядят так:
	\[
		\forall n \in \N \quad c_n = \frac{1}{2\pi} \int_{[-\pi; \pi]} f(x)e^{-inx}d\mu(x)
	\]
\end{note}

\subsection{Сходимость тригонометрических рядов Фурье}

\begin{note}
	Для упрощения записи мы будем говорить, что $f \in L_{2\pi}$, если $f$ --- $2\pi$-периодическая и $f \in L_1[-\pi; \pi]$
\end{note}

\begin{anote}
	$2\pi$-периодичность подразумевает, что равенство $f(x + 2\pi) = f(x)$ верно для всех разумных $x$. В частности, если рассматривать $f$ как функцию только на отрезке $[a; a + 2\pi]$, то она $2\pi$-периодична по определению при одном лишь условии $f(a) = f(a + 2\pi)$. Однако далее следует предполагать, что $f$ определена в некоторой окрестности отрезка $[-\pi; \pi]$, если её значения берутся вне этого отрезка.
\end{anote}

\begin{lemma} (общая формулировка вынесена у лектора в параграф про приближение функций полиномами) \label{fourier_partial_sum}
	Если $f \in L_{2\pi}$, то для любых $n \in \N, \ \{\rho_{n,k}\}_{k=1}^n \subset \R$ имеет место равенство:
	\[
		\widetilde{\sigma}_n(f, x) := \frac{a_0}{2} + \sum_{k = 1}^n \rho_{n,k} (a_k\cos(kx) + b_k\sin(kx)) = \frac{1}{\pi}\int_{[-\pi; \pi]} f(u)K_n(u - x)d\mu(u)
	\]
	где $K_n(t)$ есть следующее выражение:
	\[
		K_n(t) = \frac{1}{2} + \sum_{k = 1}^n \rho_{n,k} \cos(kt)
	\]
\end{lemma}

\begin{proof}
	Подставим все формулы коэффициентов в сумму:
	\begin{multline*}
		\wdt\sigma_n(f, x) = \frac{1}{2\pi}\int_{[-\pi; \pi]} f(u)d\mu(u) \ +
		\\
		\sum_{k = 1}^n \rho_{n,k} \ps{\frac{\cos(kx)}{\pi}\int_{[-\pi; \pi]} f(u)\cos(ku)d\mu(u) + \frac{\sin(kx)}{\pi}\int_{[-\pi; \pi]} f(u)\sin(ku)d\mu(u)} =
		\\
		\frac{1}{\pi} \int_{[-\pi; \pi]} f(u)\ps{\frac{1}{2} + \sum_{k = 1}^n \rho_{n,k} \ps {\cos(kx)\cos(ku) + \sin(kx)\sin(ku)}}d\mu(u) =
		\\
		\frac{1}{\pi} \int_{[-\pi; \pi]} f(u)\ps{\frac{1}{2} + \sum_{k = 1}^n \rho_{n,k} \cos(k(u - x))}d\mu(u) = \frac{1}{\pi} \int_{[-\pi; \pi]} f(u)K_n(u - x)d\mu(u)
	\end{multline*}
	Остальные равенства получаются аналогичным образом.
\end{proof}

\begin{note}
	В ближайшее время нас будет интересовать только случай, когда все $\rho_{n,k}$ равны 1, который даёт формулу для частичных сумм ряда Фурье, обозначаемых $S_n(f, x)$. Величина $K_n(t)$ при этом называется \textit{ядром Дирихле} и обозначается $D_n(t)$.
\end{note}

\begin{lemma}
	Ядро Дирихле представимо в следующем виде:
	\[
		D_n(t) := \frac{1}{2} + \sum_{k = 1}^n \cos(kt) = \frac{\sin\ps{(n + \frac{1}{2})t}}{2\sin\ps{t/2}}
	\]
\end{lemma}

\begin{proof}
	Есть два пути: первый --- домножить числитель и знаменатель на $\sin(t/2)$, представить каждое слагаемое как сумму синусов и получить телескопическую сумму; второй --- представить косинус как комплексную экспоненту. Сделаем вторым способом:
	\begin{multline*}
		\frac{1}{2} + \sum_{k = 1}^n \cos(kt) = \frac{1}{2} + \frac{1}{2}\sum_{k = 1}^n (e^{ikt} + e^{-ikt}) = \frac{1}{2} \sum_{k = -n}^n e^{ikt} = \frac{1}{2} e^{-int} \frac{e^{i(2n + 1)t} - 1}{e^{it} - 1} = \\
		\frac{1}{2} \frac{e^{i(n + 1)t} - e^{-int}}{e^{it} - 1} = \frac{1}{2}\frac{(e^{i(n + 1/2)t} - e^{-i(n + 1/2)t})e^{it/2}}{(e^{it/2} - e^{-it/2})e^{it/2}} = \frac{1}{2}\frac{2i\sin((n + 1/2)t)}{2i\sin(t/2)} = \frac{\sin((n + 1/2)t)}{2\sin(t/2)}
	\end{multline*}
\end{proof}

\begin{note}
	Формулу с ядром Дирихле можно записать ещё несколькими способами:
	\begin{itemize}
		\item Заменой $u - x = t$ и сдвигом новых пределов интеграла обратно до $[-\pi; \pi]$ в силу $2\pi$-периодичности можно получить такую формулу:
		\[
			S_n(f, x) = \frac{1}{\pi} \int_{[-\pi; \pi]} f(x + t)D_n(t)d\mu(t)
		\]
		
		\item Заметим, что в форме выше $D_n(t)$ --- чётная функция, поэтому замена $u = -t$ приведёт к такой формуле (потом $u$ я просто заменил на $t$):
		\[
			S_n(f, x) = \frac{1}{\pi} \int_{[-\pi; \pi]} f(x - t)D_n(t)d\mu(t)
		\]
		
		\item Так как выше мы выразили одно и то же при помощи разных формул, то можно, например, взять среднее от них:
		\[
			S_n(f, x) = \frac{S_n(f, x) + S_n(f, x)}{2} = \frac{1}{2\pi} \int_{[-\pi; \pi]} (f(x + t) + f(x - t))D_n(t)d\mu(t)
		\]
		
		\item Дополнительно заметим, что $g(t) = f(x + t) + f(x - t)$ является чётной функцией. Стало быть, можно получить ещё и такую формулу:
		\[
			S_n(f, x) = \frac{1}{\pi} \int_{[0; \pi]} (f(x + t) + f(x - t))D_n(t)d\mu(t)
		\]
	\end{itemize}
\end{note}