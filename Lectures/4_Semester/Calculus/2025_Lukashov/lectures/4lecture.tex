\begin{definition}
	Если функция $F \colon E \to \sxR$ лежит в $L_{2\pi}$, то мы называем \textit{интегральным модулем непрерывости функции $F$} значение следующей функции:
	\[
		\omega_1(\delta, F) = \sup_{0 \le h \le \delta} \int_{[-\pi; \pi]} |F(x + h) - F(x)|d\mu(x)
	\]
\end{definition}

\begin{lemma} (от автора)
	Для $f \in L_{2\pi}$ имеет место следующий предел:
	\[
		\lim_{\delta \to 0+} \omega_1(\delta, f) = 0
	\]
\end{lemma}

\begin{proof}
	По лемме \ref{average_continuous}
	\[
		\forall g \in L_1(\R)\ \ \exists \lim_{t \to 0} \int_\R |g(x + t) - g(x)|d\mu(x) = 0
	\]
	Из функции $f \in L_{2\pi}$ легко сделать функцию $g \in L_1(\R)$ (например, расширить до некоторой окрестности $[-\pi; \pi]$ по периодичности и занулить вне этой окрестности). Тогда при достаточно маленьком $\delta > 0$ (таком, что $f$ и $g$ совпадают на $[-\pi; \pi + \delta]$) верно следующее:
	\[
		\omega_1(\delta, f) \leq \sup_{0 \le h \le \delta} \int_\R |g(x + h) - g(x)|d\mu(x)
	\]
	Перейдём к пределу в неравенстве и получим требуемое:
	\[
		\lim_{\delta \to 0+} \omega_1(\delta, f) \leq \lim_{\delta \to 0+} \sup_{0 \le h \le \delta} \int_\R |g(x + h) - g(x)|d\mu(x) = \lim_{t \to 0} \int_\R |g(x + t) - g(x)|d\mu(x) = 0
	\]
\end{proof}

\begin{lemma} \label{uniconv_lemma}
	Пусть $f \in L_{2\pi}$, а $g$ --- измеримая, $2\pi$-периодическая и ограниченная на $\R$ функция. Тогда коэффициенты Фурье функции $\chi_x(t) = f(x + t)g(t)$ стремятся к нулю равномерно по $x$ на $\R$.
\end{lemma}

\begin{note}
	Аналогичное утверждение верно для $\kappa_x(t) = f(x - t)g(t)$
\end{note}

\begin{proof}
	По основной теореме об интеграле Лебега, функция $g$ является суммируемой на $[-\pi; \pi]$. По признаку суммируемости будет суммируема и $\chi$. Зафиксируем $x \in \R$ и распишем стандартным образом коэффициенты ряда Фурье для $\chi$ (для $b_n$ аналогично):
	\begin{multline*}
		a_n(\chi) = \frac{1}{\pi} \int_{[-\pi; \pi]} f(x + t)g(t)\cos(nt)d\mu(t) = \{\text{замена } u = t - \pi / n\} =
		\\
		-\frac{1}{\pi} \int_{[-\pi; \pi]} f\ps{x + u + \frac{\pi}{n}}g\ps{u + \frac{\pi}{n}}\cos(nu)d\mu(u) =
		\\
		\{\text{усреднение двух предыдущих равных интегралов}\} =
		\\
		-\frac{1}{2\pi} \int_{[-\pi; \pi]} \Bigg(f\ps{x + t + \frac{\pi}{n}}g\ps{t + \frac{\pi}{n}} - f(x + t)g(t)\Bigg)\cos(nt)d\mu(t) = I_1(x) + I_2(x)
	\end{multline*}
	где $I_{1, 2}$ --- это следующие интегралы (добавили и отняли промежуточное слагаемое $f(x + t)g(t + \pi/n)$ и разбили получившийся интеграл на два):
	\begin{align*}
		&{I_1(x) = -\frac{1}{2\pi}\int_{[-\pi; \pi]} \ps{f\ps{x + t + \frac{\pi}{n}} - f(x + t)}g\ps{t + \frac{\pi}{n}}\cos(nt)d\mu(t)}
		\\
		&{I_2(x) = -\frac{1}{2\pi}\int_{[-\pi; \pi]} \ps{g\ps{t + \frac{\pi}{n}} - g(t)}f(x + t)\cos(nt)d\mu(t)}
	\end{align*}
	Коль скоро $g$ ограничена, пусть $\forall x \in \R \ |g(x)| \le M \in \R$. Покажем, что $I_1$ равномерно сходится к нулю:
	\[
		|I_1| \le \frac{M}{2\pi} \int_{[-\pi; \pi]} \md{f\ps{x + t + \frac{\pi}{n}} - f(x + t)}d\mu(t) \le \frac{M}{2\pi} \omega_1\ps{\frac{\pi}{n}, f} \to 0,\ n \to \infty
	\]
	Так как мы получили оценку на интеграл вида константы на функцию, не зависящую от $x$, а только от $n$, то $I_1(x) $ сходится к нулю равномерно. Теперь займёмся $I_2$ \red{а не получится ли так же оценить $g$ константой, а равномерную сходимость $\int f(x+t)\cos(nt) d\mu(t)$ показать так же, как в теореме об осцилляции?}: по лемме о всюду плотности множества непрерывных функций, если зафиксировать $\eps > 0$, то мы можем разложить $f$ следующим образом:
	\[
		f = f_1 + f_2,\ f_1 \in C(\R),\ \int_{[-\pi; \pi]} |f_2(x + t)|d\mu(t) < \eps
	\]
	Так как $f_1$ непрерывна на отрезке $[x - \pi; x + \pi]$, то по теореме Вейерштрасса её можно ограничить константой на этом отрезке. Однако эта константа в общем случае будет зависеть от выбора $x$, что нарушит равномерную сходимость. Но поскольку нас интересует приближение $f$ только на отрезке длиной $2\pi$, можно считать, что $f_1$ (а следовательно и $f_2$) также имеет период $2\pi$. С учётом этого скажем, что $\forall x \in \R \ |f_1| \le B$. Соответствующим образом мы можем теперь разбить интеграл $I_2 = I_{2, 1} + I_{2, 2}$ ($I_{2, i}$ относится к $I_2$, но с $f_i$ вместо $f$) и разобраться с каждым по отдельности:
	\begin{align*}
		&{|I_{2, 1}| \le \frac{B}{2\pi} \omega_1\ps{\frac{\pi}{n}, g} \to 0,\ n \to \infty}
		\\
		&{|I_{2, 2}| \le \frac{M}{\pi}\eps}
	\end{align*}
	Обе оценки стремятся к нулю при $\eps \to 0$, $n \to \infty$ и не зависят от $x$, чего достаточно для завершения доказательства.
\end{proof}

\begin{theorem} (Принцип локализации Римана-Лебега)
	Пусть $f \in L_{2\pi}$ такова, что $f = 0$ на некотором интервале $I$. Тогда $\forall x \in I$ тригонометрический ряд Фурье функции $f$ сходится к нулю в точке $x$, причём равномерно на любом отрезке $S \subset I$.
\end{theorem}

\begin{proof}
	Выберем произвольный отрезок $S \subset I$. Так как $I$ --- это интервал, то есть открытое множество, то верно утверждение:
	\[
		\exists \eta > 0 \such \forall x \in S\ \forall h, |h| < \eta\ \ x + h \in I
	\]
	Рассмотрим любой $x \in I$ и его $\eta$-окрестность. Введём хитрый индикатор (он по сути отражает тот факт, что $f$ зануляется в окрестности $x$, когда мы посмотрим на коэффициенты ряда Фурье):
	\[
		\lambda(t) = \System{
			&{0,\ |t| < \eta}
			\\
			&{1,\ \eta \le |t| \le \pi}
		}
		= \chi_{[-\pi; \pi] \bs (-\eta; \eta)}(t)
	\]
	Используя лемму о виде частичной суммы ряда Фурье с ядром Дирихле, запишем такую цепочку равенств (во втором переходе мы добавили $\lambda(t)$, ибо это ни на что не влияет: $f(x + t)$ гарантированно зануляется в тех же местах):
	\begin{multline*}
		S_n(f, x) = \frac{1}{\pi} \int_{[-\pi; \pi]} f(x + t)D_n(t)d\mu(t) = \frac{1}{\pi} \int_{[-\pi; \pi]} f(x + t)\lambda(t)D_n(t)d\mu(t) =
		\\
		\frac{1}{2\pi} \int_{[-\pi; \pi]} f(x + t)\underbrace{\lambda(t) \frac{\cos(t / 2)}{\sin(t / 2)}}_{g_1(t)}\sin(nt)d\mu(t) + \frac{1}{2\pi} \int_{[-\pi; \pi]} f(x + t)\underbrace{\lambda(t)}_{g_2(t)}\cos(nt)d\mu(t)
	\end{multline*}
	Заметим, что $g_1$ и $g_2$ --- это ограниченные и $2\pi$-периодические функции на $[-\pi; \pi]$. Стало быть, коэффициенты рядов Фурье для $f(x + t)g_1(t)$ и $f(x + t)g_2(t)$ равномерно сходятся к 0 по $x$, а это ровно те интегралы, которые мы видим выше. Стало быть, $S_n(f, x) \rra 0$ при $n \to \infty$, что и требовалось.
\end{proof}

\begin{corollary}
	Если $f_1 = f_2$ на интервале $I$, причём $f_1, f_2 \in L_{2\pi}$, то тригонометрические ряды Фурье функций $f_1$ и $f_2$ равномерно сходятся или не сходятся одновременно на любом отрезке $S \subset I$. В частности, эти ряды равномерно сходятся $\forall x \in I$.
\end{corollary}

\begin{lemma} (Основа признака Дини)
	Пусть $f \in L_{2\pi}$, а $\phi_x(t, s)$ представляет собой следующую функцию:
	\[
		\phi_x(t, s) = \frac{f(x + t) + f(x - t) - 2s}{t}
	\]
	Тогда
	\begin{enumerate}
	\item
	для сходимости тригонометрического ряда Фурье функции $f$ к $s$ в точке $x$ необходимо и достаточно, чтобы выполнялось утверждение:
	\[
		\exists \delta \in (0; \pi) \such \lim_{n \to \infty} \int_{[0; \delta]} \phi_x(t, s)\sin(nt)d\mu(t) = 0
	\]
	\item
	Для равномерной сходимости тригонометрического ряда Фурье функции $f$ к самой $f$ на $\R$ необходимо и достаточно, чтобы выполнялось утверждение:
	\[
		\exists \delta \in (0; \pi) \such \int_{[0; \delta]} \phi_x(t, f(x))\sin(nt)d\mu(t) \rightrightarrows 0, \ n \to \infty
	\]
	\end{enumerate}
\end{lemma}

\begin{proof}
	Идея состоит в том, чтобы рассмотреть разность $S_n(f, x) - s$ и расписать её при помощи интеграла с ядром Дирихле. Чтобы затащить $s$ под тот же интеграл, заметим важный факт:
	\[
		\int_{[-\pi; \pi]} D_n(t)d\mu(t) = \int_{[-\pi; \pi]} \frac{1}{2}d\mu(t) + \sum_{k = 1}^n \underbrace{\int_{[-\pi; \pi]} \cos(kt) d\mu(t)}_{\tbr{1, \cos(kt)} = 0} = \pi
	\]
	Стало быть, в силу чётности ядра Дирихле верна следующая формула:
	\[
		S_n(f, x) - s = \frac{1}{\pi}\int_{[0; \pi]} (f(x + t) + f(x - t) - 2s)D_n(t)d\mu(t)
	\]
	Далее мы распишем $D_n(t)$ так, чтобы получить отдельно слагаемое с интегралом из леммы:
	\begin{multline*}
		D_n(t) = \frac{\sin((n + 1 / 2)t)}{2\sin(t / 2)} =
		\\
		\frac{\sin(nt)\cos(t / 2) + \cos(nt)\sin(t / 2)}{2\sin(t / 2)} = \frac{\sin(nt)\cos(t / 2)}{2\sin(t / 2)} + \frac{\cos(nt)}{2} =
		\\
		\frac{\sin(nt)}{t} + \frac{\cos(nt)}{2} + \frac{\sin(nt)\cos(t / 2)}{2\sin(t / 2)} - \frac{\sin(nt)}{t}
	\end{multline*}
	Теперь мы бьём отрезок $[0; \pi] = [0; \delta] \cup [\delta; \pi]$, для первой части подставляем расписанное $D_n(t)$ с искусственным слагаемым, а для второй подставляем $D_n(t)$ без него. Получится следующее:
	\begin{multline*}
		S_n(f, x) - s = \frac{1}{\pi}\int_{[0; \delta]} \phi_x(t, s)\sin(nt)d\mu(t) +
		\\
		\frac{1}{\pi}\int_{[0; \delta]} (f(x + t) + f(x - t) - 2s) \sin(nt) \ps{\frac{\cos(t / 2)}{2\sin(t / 2)} - \frac{1}{t}}d\mu(t) +
		\\
		\frac{1}{\pi} \int_{[0; \pi]} (f(x + t) + f(x - t) - 2s)\frac{\cos(nt)}{2}d\mu(t) +
		\\
		\frac{1}{\pi} \int_{[\delta; \pi]} (f(x + t) + f(x - t) - 2s) \frac{\sin(nt)\cos(t / 2)}{2\sin(t / 2)}d\mu(t)
  	\end{multline*}
 	Разберёмся, почему и как каждый интеграл сходится к нулю:
 	\begin{enumerate}
 		\item Сходится в зависимости от условия (мы доказываем эквивалентность)
 		
 		\item Заметим, что у нас тут $g(t) = (f(x + t) + f(x - t) - 2s)(\frac{1}{2}\ctg(t / 2) - 1 / t) \in L_1[0; \delta]$ (вторую скобку нужно разложить по формуле Тейлора до первого слагаемого и убедиться, что она ограничена около нуля). Стало быть, по теореме об осцилляции этот интеграл сходится к нулю
 		
 		\item Аналогично $g(t) = \frac{1}{2}(f(x + t) + f(x - t) - 2s) \in L_1[0; \pi]$
 		
 		\item Аналогично $g(t) = (f(x + t) + f(x - t) - 2s) \cdot \frac{\cos(t / 2)}{2\sin(t / 2)} \in L_1[\delta; \pi]$
 	\end{enumerate}
    Первая часть доказана. Пусть теперь $s = s(x) = f(x)$. Снова рассмотрим наши 4 интеграла:
    \begin{enumerate}
    	\item Равномерно по $x$ сходится в зависимости от условия (мы доказываем эквивалентность)
    	
    	\item Положим $u(t) = (f(x + t) + f(x - t) - 2f(x)) \in L_{2\pi}$, $v(t) = (\frac{1}{2}\ctg(t / 2) - 1 / t) \cdot \chi_{[0; \delta]}$. Тогда $v$ --- это $2\pi$-периодическая и ограниченная функция на $[-\pi; \pi]$, поэтому можно применить лемму \ref{uniconv_lemma}, по которой для $\chi(t) = u(x + t)v(t)$ соответствующий интеграл коэффициента будет равномерно сходиться к нулю
    	
    	\item Заметим, что $u(t) = \frac{1}{2}(f(x + t) + f(x - t) - 2f(x))$ является чётной функцией как и косинус, поэтому интеграл по $[0; \pi]$ есть половина интеграла от $[-\pi; \pi]$. Применим лемму \ref{uniconv_lemma} к $u(t)$ и $v(t) = 1$, тогда $\chi(t) = u(t)v(t)$ равномерно сходится к нулю
    	
    	\item Обоснование аналогично второму интегралу, только $v(t) = \frac{1}{2}\ctg(t / 2) \cdot \chi_{[\delta; \pi]}$
    \end{enumerate}
\end{proof}

\begin{theorem} (Признак Дини)
	Пусть $f \in L_{2\pi}$. Если для $s \in \R$ функция $\phi_x(t, s) \in L_1[0; \delta]$ при некотором $\delta > 0$, то тригонометрический ряд Фурье для $f$ сходится к $s$ в точке $x$
\end{theorem}

\begin{proof}
	Так как $\phi_x(t, s) \in L_1[0; \delta]$, то по теореме Римана об осцилляции имеем предел:
	\[
		\lim_{n \to \infty} \int_{[0; \delta]} \phi_x(t, s)\sin(nt)d\mu(t) = 0
	\]
	Что и требует основа признака Дини для соответствующей сходимости.
\end{proof}