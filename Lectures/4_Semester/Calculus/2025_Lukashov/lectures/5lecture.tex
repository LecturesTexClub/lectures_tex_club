\begin{definition}
	Говорят, что \textit{$f$ удовлетворяет условию Гёльдера порядка $\alpha \in \rsi{0; 1}$} в точке $x_0$, если выполнены условия:
	\begin{enumerate}
		\item $\exists f(x_0 \pm 0) := \lim_{x \to x_0 \pm 0} f(x)$
		
		\item $\exists \delta > 0, C > 0 \such \forall t \in (0; \delta)\ |f(x_0 + t) - f(x_0 + 0)| < Ct^\alpha \wedge |f(x_0 - t) - f(x_0 - 0)| < Ct^\alpha$
	\end{enumerate}
\end{definition}

\begin{definition}
	Далее \textit{односторонними производными} мы будем называть несколько необычные пределы:
	\[
		f'_+(x_0) = \lim_{t \to 0+} \frac{f(x_0 + t) - f(x_0 + 0)}{t}; \quad f'_-(x_0) = \lim_{t \to 0+} \frac{f(x_0 - t) - f(x_0 - 0)}{t}
	\]
\end{definition}

\begin{note}
	Несложно заметить, что существование таких односторонних производных приводит $f$ к тому, что она удовлетворяет условию Гёльдера при $\alpha = 1$ (оно ещё называется \textit{условием Липшица}).
\end{note}

\begin{theorem} (Признак Липшица)
	Если $f \in L_{2\pi}$ и удовлетворяет условию Гёльдера порядка $\alpha$ в точке $x_0$, то тригонометрический ряд Фурье функции $f$ сходится к $\frac{1}{2}(f(x_0 + 0) + f(x_0 - 0))$ в точке $x_0$.
\end{theorem}

\begin{note}
	В частности, если $f$ также непрерывна в $x_0$, то её тригонометрический ряд Фурье сходится к $f(x_0)$.
\end{note}

\begin{proof}
	Проверим условия признака Дини с $s = \frac{1}{2}(f(x_0 + 0) + f(x_0 - 0))$, для этого посмотрим на соответствующий интеграл:
	\begin{multline*}
		\int_{[0; \delta]} \phi_x\ps{t, \frac{f(x_0 + 0) + f(x_0 - 0)}{2}}d\mu(t) =
		\\
		\int_{[0; \delta]} \frac{f(x_0 + t) - f(x_0 + 0)}{t}d\mu(t) + \int_{[0; \delta]} \frac{f(x_0 - t) - f(x_0 - 0)}{t}d\mu(t)
	\end{multline*}
	Остаётся воспользоваться условием Гёльдера и оценить интегралы:
	\[
		\md{\int_{[0; \delta]} \frac{f(x_0 \pm t) - f(x_0 \pm 0)}{t}d\mu(t)} \le \int_{[0; \delta]} \frac{Ct^\alpha}{t}d\mu(t) < \infty
	\]
\end{proof}

\begin{theorem} (Признак Жордана)
	Если функция $f \in L_{2\pi}$ имеет ограниченную вариацию на $[a; b] \subset \R$, то в любой точке $x \in (a; b)$ тригонометрический ряд Фурье функции $f$ сходится к $\frac{1}{2}(f(x - 0) + f(x + 0))$. Если, кроме того, $f$ непрерывна на $[a; b]$, то соответствующий тригонометрический ряд Фурье равномерно сходится к $f$ на любом отрезке $[a'; b'] \subset (a; b)$.
\end{theorem}

\begin{reminder}
	Любая функция ограниченной вариации представима как разность двух монотонных, а любая монотонная функция имеет разрывы не более, чем первого рода (то есть односторонние пределы всегда есть). По этой причине достаточно доказать теорему для монотонных функций.
\end{reminder}

\begin{proof}
	По основе признака Дини тригонометрический ряд Фурье сходится к $\frac{1}{2}(f(x - 0) + f(x + 0))$ в точке $x$ тогда и только тогда, когда
	\[
		\exists \delta > 0 \such \int_{[0; \delta]} \frac{f(x + t) + f(x - t) - (f(x - 0) + f(x + 0))}{t}\sin(nt)d\mu(t) \xrightarrow[n \to \infty]{} 0
	\]
	Мы разделим интеграл на два слагаемых: один с $f(x + t) - f(x + 0)$, другой с $f(x - t) - f(x - 0)$, и докажем стремление к нулю только для первого из них, ибо второй сводится к тому же случаю домножением на $-1$.
	
	Обозначим $g(t) := f(x + t) - f(x + 0)$. Коль скоро $x \in (a; b)$, то существует такое $\delta > 0$, что $x \pm \delta \in [a; b]$. Без ограничения общности, пусть $f$ неубывает. Это также означает, что $g$ неубывающая и неотрицательная. Зафиксируем $\eps > 0$. Тогда по определению $f(x + 0)$
	\[
		\exists \delta_1 \in (0; \delta) \such \forall t \in \rsi{0; \delta_1}\ \ g(t) < \eps
	\]
	Поскольку $g$ монотонна, то она интегрируема по Риману, и мы можем разбить интеграл от $0$ до $\delta$ на два интеграла по $[0; \delta_1]$ и $[\delta_1; \delta]$. Применим вторую теорему о среднем к первому из них:
	\[
		\exists \delta_2 \in [0; \delta_1] \such \int_0^{\delta_1} \frac{g(t)}{t}\sin(nt)dt = g(\delta_1) \int_{\delta_2}^{\delta_1} \frac{\sin(nt)}{t}dt
	\]
	Хочется воспользоваться известным фактом: $\int_0^{+\infty} \frac{\sin t}{t}dt$  сходится. Это значит, что
	\[
		\exists C > 0 \such \forall A > 0\ \ \md{\int_0^A \frac{\sin t}{t}dt} < C
	\]
	Заменяя $u = nt$:
	\[
		\md{\int_0^{\delta_1} \frac{\sin(nt)}{t}dt} = \md{\int_0^{n\delta_1} \frac{\sin(u)}{u}du} < C
	\]
	Всё это позволяет оценить модуль интеграла следующим образом:
	\[
		\md{\int_0^{\delta_1} \frac{g(t)}{t}\sin(nt)dt} = \md{g(\delta_1) \int_{\delta_2}^{\delta_1} \frac{\sin(nt)}{t}dt} <
		\\
		\eps \cdot 2C
	\]
	Про оставшийся отрезок $[\delta_1; \delta]$ можно сказать, что $g(t)/t$ на нём ограничена при фиксированном $\delta_1$:
	\[
		\forall t \in [\delta_1, \delta] \ \frac{g(t)}{t} \leq \frac{\eps}{\delta_1} 
	\]
	Стало быть, по теореме Римана об осцилляции
	\[
		\exists N \in \N \such \forall n > N\ \ \int_0^{\delta} \frac{g(t)}{t} \sin(nt)d\mu(t) < \eps(1 + 2C)
	\]
	Первая часть доказана. Далее $f \in C(a; b)$, и мы покажем равномерную сходимость. По основе признака Дини, нам нужен такой предел:
	\[
		\int_{[0; \delta]} \frac{f(x + t) + f(x - t) - 2f(x)}{t}\sin(nt)d\mu(t) \rra 0,\ n \to \infty
	\]
	Причём мы всё так же можем рассматривать только слагаемое с $(f(x+t) - f(x))/t$
	Помним, что по теореме Кантора $f$ равномерно непрерывна на $[a'; b']$, и $\delta_1$ теперь будет такая:
	\[
		\exists \delta_1 \in (0; \delta) \such \forall x \in [a'; b']\ \forall t \in \rsi{0; \delta_1}\ \ f(x + t) - f(x + 0) < \eps
	\]
	Тогда интеграл от $\delta$ до $\delta_1$ равномерно сходится к нулю, поскольку $\delta_1$ и константа $C$ никак не зависят от $x$. Что касается второго интеграла, разобьём его на два слагаемых:
	\[
		\md{\int_{\delta_1}^\delta \frac{f(x+t) - f(x)}{t} \sin(nt) dt} \leq \md{\int_{\delta_1}^\delta \frac{f(x+t)}{t} \sin(nt) dt} + \md{f(x) \int_{\delta_1}^\delta \frac{\sin(nt)}{t} dt}
	\]
	По теореме Вейерштрасса $f$ ограничена на $[a; b]$, поэтому оба слагаемых равномерно сходятся к нулю по лемме \ref{uniconv_lemma}.
\end{proof}

\subsection{Действия с рядами Фурье}

\begin{lemma} \label{abs_int_product}
	Если $F$, $G$ абсолютно непрерывны на $[a, b]$, то
	\[
		\int_{[a, b]} F(x)G'(x)d\mu(x) = F(b)G(b) - F(a)G(a) - \int_{[a, b]} F'(x)G(x)d\mu(x)
	\]
\end{lemma}

\begin{proof}
	Достаточно доказать, что $FG$ абсолютно непрерывна на $[a, b]$. Во-первых, из абсолютной непрерывности $F$ и $G$ на отрезке следует их ограниченность:
	\[
		\exists C : \forall x \in [a, b] \ F(x) < C, G(x) < C
	\]
	Во-вторых, распишем по определению абсолютную непрерывность:
	\begin{multline*}
		\forall \eps > 0 \ \exists \delta > 0 \ps{ \forall a = x_0 < x_1 < \ldots < x_n = b \such \sum_{k=1}^n a_k - a_{k-1} < \delta} \\ \sum_{k=1}^n |F(a_k) - F(a_{k-1})| < \eps, \ \sum_{k=1}^n |G(a_k) - G(a_{k-1})| < \eps
	\end{multline*}
	То же неравенство для $FG$ выводится стандартным приёмом с добавление слагаемого:
	\begin{multline*}
		\sum_{k=1}^n |F(a_k)G(a_k) - F(a_{k-1}G(a_{k-1}))| \leq \\ \sum_{k=1}^n |F(a_k) - F(a_{k-1})||G(a_k)| + |G(a_k) - G(a_{k-1})||F(a_{k-1})| < 2C\eps 
	\end{multline*}
\end{proof}

\begin{theorem} (Почленное дифференцирование рядов Фурье) Если $f$ --- это $2\pi$-периодичекая, абсолютно непрерывная на любом периоде функция, то тригонометрический ряд Фурье функции $f'$ получается почленным дифференцированием тригонометрического ряда Фурье функции $f$.
\end{theorem}

\begin{proof}
	В силу абсолютной непрерывности $f$ её производная существует и непрерывна почти всюду. Стало быть, $f' \in L_1[-\pi; \pi]$. Распишем по стандартным формулам коэффициенты, которыми должен обладать ряд Фурье для производной:
	\[
		a_n(f') = \frac{1}{\pi} \int_{[-\pi; \pi]} f'(x)\cos(nx)d\mu(x) = \underbrace{\frac{1}{\pi} f(x)\cos(nx)\Big|_{-\pi}^\pi}_{0} + \frac{n}{\pi} \int_{[-\pi; \pi]} f(x)\sin(nx)d\mu(x) = nb_n(f)
	\]
	Аналогичная формула верна и для $b_n(f') = -na_n(f)$. Остаётся подставить коэффициенты по итоговым формулам в ряд:
	\[
		f'(x) \sim \sum_{n = 1}^\infty (-na_n(f)\sin(nx) + nb_n(f)\cos(nx)) = \frac{a_0(f)'}{2} + \sum_{n = 1}^\infty (a_n(f)\cos(nx)' + b_n(f)\sin(nx)')
	\]
\end{proof}

\begin{corollary}
	Если $f$ --- это $2\pi$-периодическая функция такая, что $f, f', \ldots, f^{(k - 1)}$ абсолютно непрерывны на любом периоде и $2\pi$-периодичны, то верны оценки:
	\begin{align*}
		&{a_n(f) = o\ps{\frac{1}{n^k}}, n \to \infty}
		\\
		&{b_n(f) = o\ps{\frac{1}{n^k}}, n \to \infty}
	\end{align*}
\end{corollary}

\begin{proof}
	Если мы посчитаем $a_n(f^{(k)})$ или $b_n(f^{(k)})$, то индуктивно получим равенство, которое с модулем запишется как \(|a_n(f^{(k)})| = n^k|g_k|\), где $g_k$ --- это соответствующий коэффициент из $\{a_n(f), b_n(f)\}$. Стало быть, мы можем выразить коэффициенты ряда Фурье в этом равенстве и получить требуемую оценку (потому что коэффициент тригонометрического ряда Фурье для высшей производной тоже сходится к нулю по теореме Римана об осцилляции).
\end{proof}

\begin{theorem} (Оценка коэффициентов Фурье функции ограниченной вариации)
	Если $f$ --- это $2\pi$-периодическая функция, обладающая ограниченной вариацией на любом периоде, то верны оценки:
	\begin{align*}
		&{|a_n(f)| = O\ps{\frac{1}{n}}, n \to \infty}
		\\
		&{|b_n(f)| = O\ps{\frac{1}{n}}, n \to \infty}
	\end{align*}
\end{theorem}

\begin{proof}
	Ещё раз напомним, что любая функция ограниченной вариации интегрируема на отрезке, так как раскладывается в разность монотонных, а для тех утверждение уже доказывалось. Будем аккуратно сводить оценку модуля коэффициента $|a_n(f)|$ к неравенству с полной вариацией:
	\begin{multline*}
		|a_n(f)| = \frac{1}{\pi}\md{\int_{-\pi}^\pi f(x)\cos(nx)dx} = \{1\} =
		\\
		\md{-\frac{1}{2\pi} \int_{-\pi}^\pi \ps{f\ps{x + \frac{\pi}{n}} - f(x)}\cos(nx)dx} = \{2\} =
		\\
		\frac{1}{2\pi} \md{\int_{-\pi}^\pi \ps{f\ps{x + \frac{k\pi}{n}} - f\ps{x + \frac{(k - 1)\pi}{n}}}\cos(nx)dx} \le
		\\
		\frac{1}{2\pi} \int_{-\pi}^\pi \md{f\ps{x + \frac{k\pi}{n}} - f\ps{x + \frac{(k - 1)\pi}{n}}}dx
	\end{multline*}
	Пояснения к действиям:
	\begin{enumerate}
		\item Сделали замену $x = t + \pi / n$, сдвинули пределы интегрирования обратно к $[-\pi; \pi]$, взяли среднее от двух интегралов
		
		\item Повторили то же самое ещё произвольное $k - 1$ число раз, но среднее уже не брали
	\end{enumerate}
	Итак, посмотрим на $n|a_n(f)|$ и соберём вариацию:
	\[
		n|a_n(f)| \le \frac{1}{2\pi} \int_{-\pi}^\pi \sum_{k = 1}^n \md{f\ps{x + \frac{k\pi}{n}} - f\ps{x + \frac{(k - 1)\pi}{n}}}dx \le V(f)
	\]
	Из этого неравенства очевидна оценка на коэффициент ряда Фурье
\end{proof}

\begin{theorem} (Лебега об интегрировании рядов Фурье)
	Если $f \in L_{2\pi}$ имеет тригонометрический ряд Фурье $\frac{a_0}{2} + \sum_{n = 1}^\infty (a_n\cos(nx) + b_n\sin(nx))$, то неопределенный интеграл Лебега функции $F(x) = \int_{[x_0; x]} (f(t) - \frac{a_0}{2})d\mu(t)$ равномерно сходится к $F$ на $[-\pi; \pi]$ и представим по следующей формуле:
	\[
		F(x) = C + \sum_{n = 1}^\infty \frac{a_n\sin(nx) - b_n\cos(nx)}{n}
	\]
\end{theorem}

\begin{note}
	Важно отметить, что ряд для интеграла будет сходиться независимо от того, сходится ли ряд для $f$.
\end{note}

\begin{proof}
	Уже известен факт, что $F(x)$ --- абсолютно непрерывная функция. Проверим, что она $2\pi$-периодическая:
	\[
		F(\pi) - F(-\pi) = \int_{-\pi}^\pi \ps{f(t) - \frac{a_0}{2}}d\mu(t) = \pi a_0 - \pi a_0 = 0
	\]
	Из абсолютной непрерывности и $2\pi$-периодичности уже следует равномерная сходимость на любом отрезке по признаку Жордана. Найдём теперь коэффициенты Фурье для $F$, пользуясь леммой \ref{abs_int_product}:
	\begin{multline*}
		a_n(F) = \frac{1}{\pi} \int_{-\pi}^\pi F(x)\cos(nx)d\mu(x) = \\ \frac{1}{\pi}\ps{\frac{F(\pi)\sin(n\pi)}{n} - \frac{F(-\pi)\sin(-n\pi)}{n} - \frac{1}{n} \int_{-\pi}^\pi f(x)\sin(nx)d\mu(x)} = -\frac{b_n}{n}
	\end{multline*}
	Аналогично получается формула для $a_n$.
\end{proof}