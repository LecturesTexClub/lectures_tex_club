\begin{proposition}
	Рассмотрим $L_2[-\pi; \pi]$ --- гильбертово пространство. Тогда система $\{\frac{1}{2}, \cos(kx), \sin(kx)\}_{k = 1}^\infty$ полна в нём
\end{proposition}

\begin{proof}
	Мы уже знаем, что указанная система $\{e_n\}_{n = 1}^\infty$ является ортогональной системой. Если мы покажем, что мы можем приблизить любую функцию из $L_2[-\pi; \pi]$, то за счёт эквивалентного свойства всё доказано.
	
	Итак, вспомним, что множество непрерывных функций всюду плотно в $L_2[-\pi; \pi]$. Стало быть, всюду плотно и множество $2\pi$-периодических непрерывных функций. Тогда по теореме Фейера получаем требуемое (опять же, из равномерной сходимости следует сходимость в $L_2[-\pi; \pi]$).
\end{proof}

\begin{corollary}
	Имеет смысл уточнить некоторые свойства и показать новые:
	\begin{itemize}
		\item Верно равенство Парсеваля:
		\[
			\forall f \in L_2[-\pi; \pi]\ \ \|f\|^2 = \sum_{k = 1}^\infty f_k^2\|e_k\|^2 \Lra \frac{1}{\pi} \int_{[-\pi; \pi]} |f(t)|^2d\mu(t) = \frac{a_0^2}{2} + \sum_{n = 1}^\infty (a_n^2 + b_n^2)
		\]
		
		\item Для любой $f \in L_2[-\pi; \pi]$ её тригонометрический ряд Фурье сходится к ней по норме $L_2[-\pi; \pi]$.
		
		\item Из теоремы Рисса-Фишера следует, что для любого сходящегося ряда $\frac{a_0^2}{2} + \sum_{n = 1}^\infty (a_n^2 + b_n^2)$ найдётся функция $f \in L_2[-\pi; \pi]$ такая, что элементы этого ряда будут коэффициентами тригонометрического ряда Фурье для $f$.
	\end{itemize}
\end{corollary}

\begin{theorem} (Хаусдорфа-Юнга, без доказательства)
	Верны следующие утверждения:
	\begin{enumerate}
		\item Если $1 < p \le 2$, то для любой $f \in L_p[-\pi; \pi]$ её ряд коэффициентов Фурье $\sum_{n = 1}^\infty (|a_n|^q + |b_n|^q)$ сходится (где $1 / p + 1 / q = 1$).
		
		\item Если $1 < p \le 2$ и ряд $\frac{a_0^p}{2} + \sum_{n = 1}^\infty (|a_n|^p + |b_n|^p)$ сходится, то найдётся функция $f \in L_q[-\pi; \pi]$ (снова $1 / p + 1 / q = 1$) такая, что для неё эти элементы последовательности будут коэффициентами Фурье.
	\end{enumerate}
\end{theorem}

\section{Интегралы, зависящие от параметров}

\subsection{Собственные интегралы, зависящие от параметра}

\begin{theorem} (о непрерывности собственного интеграла по параметру)
	Пусть $A \subseteq \R^n$ --- множество значений параметра, $\alpha_0$ --- предельная точка $A$, $E \subseteq \R^m$ --- измеримое множество и задана функция $f \colon E \times A \to \R$. Если наложены следующие условия:
	\begin{enumerate}
		\item Для любого $\alpha \in A$ функция $f(x, \alpha)$ измерима на $E$
		
		\item Почти всюду на $E$ выполнено $|f(x, \alpha)| \le \phi(x)$, где $\phi \in L_1(E)$, для любого $\alpha \in A$
		
		\item Почти всюду на $E$ имеет место сходимость $f(x, \alpha) \to f(x, \alpha_0)$ при $\alpha \to \alpha_0$, $\alpha \in A$
	\end{enumerate}
	Тогда интеграл $\int_E f(x, \alpha)d\mu(x)$ непрерывен в точке $\alpha_0$, то есть имеется предел:
	\[
		\lim_{\alpha \to \alpha_0} \int_E f(x, \alpha)d\mu(x) = \int_E f(x, \alpha_0)d\mu(x)
	\]
\end{theorem}

\begin{proof}
	Рассмотрим произвольную последовательность Гейне $\{\alpha_n\}_{n = 1}^\infty \subseteq A$, \\ $\lim_{n \to \infty} \alpha_n = \alpha_0$. Тогда последовательность функций $f(x, \alpha_n)$ удовлетворяет теореме Лебега о мажорируемой сходимости, то есть имеется предел:
	\[
		\lim_{n \to \infty} \int_E f(x, \alpha_n)d\mu(x) = \int_E f(x, \alpha_0)d\mu(x)
	\]
	Это и есть определение непрерывности по Гейне.
\end{proof}

\begin{corollary}
	Если $f(x, y) \colon \R^2 \to \R$ непрерывна на прямоугольнике $[a; b] \times [c; d]$, то функция $I(y) = \int_a^b f(x, y)dx$ непрерывна на $[c; d]$
\end{corollary}

\begin{theorem} (о дифференцируемости собственного интеграла по параметру)
	Пусть даны $f \colon E \times (c; d) \to \R$, где $E \subseteq \R^m$ --- измеримое множество. Если выполнены следующие условия :
	\begin{enumerate}
		\item Для любого $\alpha \in (c; d)$ функция $f(x, \alpha)$ суммируема на $E$
		
		\item Почти всюду на $E$ $f$ дифференцируема по $\alpha$ и $\md{\pd{f}{\alpha}(x, \alpha)} \le \phi(x)$, где $\phi \in L_1(E)$
	\end{enumerate}
	Тогда взятие производной по параметру коммутирует с интегралом по $E$:
	\[
		\forall \alpha_0 \in (c; d)\ \ \pd{}{\alpha} \ps{\int_E f(x, \alpha)d\mu(x)}\at_{\alpha=\alpha_0} = \int_E \pd{}{\alpha}f(x, \alpha_0)d\mu(x)
	\]
\end{theorem}

\begin{proof}
	Пусть $\alpha_0 \in (c; d)$. Рассмотрим произвольную последовательность Гейне $\{\alpha_n\}_{n = 1}^\infty \subset \mc{U}(\alpha_0)$, $\lim_{n \to \infty} \alpha_n = \alpha_0$. Раз нам дано, что почти всюду на $E$ у нас есть производная по параметру, то почти всюду к $f$ применима теорема Лагранжа относительно $\alpha$:
	\[
		\frac{f(x, \alpha_n) - f(x, \alpha_0)}{\alpha_n - \alpha_0} = \pd{f}{\alpha}(x, \xi_n(x)) =: \psi_n(x)
	\]
	где $\xi_n(x)$ --- соответствующая точка между $\alpha_n$ и $\alpha_0$. Осталось заметить, что последовательность функций $\psi_n$ удовлетворяет условиям теоремы Лебега о мажорируемой сходимости:
	\begin{enumerate}
		\item $\psi_n \to \pd{f}{\alpha}(x, \alpha_0)$ почти всюду на $E$
		
		\item $|\psi_n(x)| \le \phi(x)$ почти всюду на $E$
		
		\item $\phi \in L_1(E)$
	\end{enumerate}
	Стало быть, есть следующее равенство:
	\[
		\int_E \pd{f}{\alpha}(x, \alpha_0)d\mu(x) = \int_E \lim_{n \to \infty} \psi_n(x) d\mu(x) = \lim_{n \to \infty} \int_E \psi_n(x) d\mu(x)
	\]
	При этом последний интеграл переписывается так:
	\begin{multline*}
		\lim_{n \to \infty} \int_E \psi_n(x) d\mu(x) = \lim_{n \to \infty} \int_E \frac{f(x, \alpha_n) - f(x, \alpha_0)}{\alpha_n - \alpha_0}d\mu(x) =
		\\
		\lim_{n \to \infty} \frac{\int_E f(x, \alpha_n)d\mu(x) - \int_E f(x, \alpha_0)d\mu(x)}{\alpha_n - \alpha_0} = \pd{}{\alpha} \ps{\int_E f(x, \alpha)d\mu(x)}\at_{\alpha=\alpha_0}
	\end{multline*}
\end{proof}

\begin{corollary}
	Если $f(x, y)$ и $\pd{f}{y}(x, y)$ непрерывны  на прямоугольнике $[a; b] \times [c; d]$, то справедливо \textit{правило Лейбница}:
	\[
		\forall y \in (c; d)\ \ \pd{}{y}\ps{\int_a^b f(x, y)dx} = \int_a^b \pd{f}{y}(x, y)dx
	\]
\end{corollary}

\begin{corollary} (из предыдущего следствия)
	Если добавить к условиям предыдущего следствия функции $\phi$ и $\psi$, которые дифференцируемы на $[c; d]$ и в любой точке $y \in [c; d]$ удовлетворяют неравенствам $a \le \phi(y) \le \psi(y) \le b$, то формула Лейбница принимает следующий вид:
	\[
		\pd{}{y}\ps{\int_{\phi(y)}^{\psi(y)} f(x, y)dx} = f(\psi(y), y) \cdot \psi'(y) - f(\phi(y), y) \cdot \phi'(y) + \int_{\phi(y)}^{\psi(y)} \pd{f}{y}(x, y)dx
	\]
\end{corollary}

\begin{proof}
	Введём следующую функцию:
	\[
		F(y, u, v) := \int_u^v f(x, y) dx
	\]
	Тогда
	\[
		\ps{\int_{\phi(y)}^{\psi(y)} f(x, y)dx} = F(y, \phi(y), \psi(y))
	\]
	И по формуле дифференцирования сложной функции многих переменных получаем
	\begin{multline*}
		\pd{}{y}\ps{\int_{\phi(y)}^{\psi(y)} f(x, y)dx}\!(y) = \ps{\pd{F}{y} + \pd{F}{\phi}\pd{\phi}{y} + \pd{F}{\psi}\pd{\psi}{y}}(y, \phi(y), \psi(y)) = \\
		\int_{\phi(y)}^{\psi(y)} \pd{f}{y}(x, y)dx -f(\phi(y), y)\phi'(y) + f(\psi(y), y)\psi'(y)
	\end{multline*}
	Где первое слагаемое получается по предыдущему следствию, а вторые два по свойству интеграла с переменным верхним пределом.
\end{proof}

\subsection{Несобственные интегралы Римана, зависящие от параметра}

\begin{anote}
	Во втором семестре мы познакомились с таким объектом, как числовые ряды. Затем это понятие было обобщено в двух независимых направлениях: как функциональные ряды и как несобственные интегралы. Теперь мы готовы замкнуть цепочку несобственными интегралами Римана, зависящими от параметра.
	
	Здесь важно держать в голове правильную интуицию. С одной стороны, мы интегрируем в несобственном смысле некоторую функцию $f \colon \lsi{a; b} \times Y \to \R$ по $\lsi{a; b}$, при этом делаем это как бы независимо для каждого $y \in Y$. С другой стороны, чтобы увидеть аналогию с функциональными рядами, имеет смысл рассматривать множество функций $Y \to \R$, определённых для каждого $t \in \lsi{a; b}$ как $\int_a^t f(x, y) dx$. В этом смысле $\lsi{a; b}$ служит некоторым метризованным множеством индексов. Это позволяет изучать разные свойства, которые уже встречались раньше, но теперь для континуального набора функций.
\end{anote} 

\begin{designation}
	Обозначим $NR_{[a; b]}^Y$ класс функций $f \colon \lsi{a; b} \times Y \to \R$ таких, что для любых $y \in Y$ и $\tilde{b} \in \lsi{a; b}$ верно, что $f(\cdot, y) \in R[a; \tilde{b}]$.
\end{designation}

\begin{designation}
	Аналогично введём класс $NR_{[a; b]}$ функций $f \colon \lsi{a; b} \to \R$ таких, что для любого $\tilde{b}$ верно $f \in R[a; \tilde{b}]$
\end{designation}

\begin{definition} 
	Пусть задана функция $f \in NR_{[a; b]}^Y$. Тогда \textit{несобственным интегралом Римана, зависящим от параметра}, называется следующий предел, если он существует:
	\[
		\int_a^b f(x, y)dx := \lim_{\tilde{b} \to b-0} \int_a^{\tilde{b}} f(x, y)dx
	\]
\end{definition}

\begin{definition}
	Если для $f \in NR_{[a; b]}^Y$ несобственный интеграл Римана конечен для каждого $y \in Y$, то говорят, что он \textit{сходится (поточечно)}.
\end{definition}

\begin{note}
	Поточечную сходимость несобственного интеграла Римана с параметром можно формально записать так (по определению предела Коши):
	\[
		\forall y \in Y\ \forall \eps > 0\ \exists B \in \lsi{a; b} \such \forall \tilde{b} \in (B; b) \ \md{\int_{\tilde{b}}^b f(x, y)dx} < \eps
	\]
\end{note}

\begin{definition}
	Пусть $f \in NR_{[a; b]}^Y$. Тогда говорят, что \textit{несобственный интеграл Римана от $f$ сходится равномерно на $Y$}, если выполнено условие:
	\[
		\forall \eps > 0\ \exists B \in \lsi{a; b} \such \forall y \in Y\ \forall \tilde{b} \in (B; b) \ \md{\int_{\tilde{b}}^b f(x, y)dx} < \eps
	\]
\end{definition}

\begin{theorem} (Критерий Коши равномерной сходимости несобственного интеграла Римана, зависящего от параметра)
	Несобственный интеграл Римана от $f \in NR_{[a; b]}^Y$ сходится равномерно по $Y$ тогда и только тогда, когда выполнено условие:
	\[
		\forall \eps > 0\ \exists B \in \lsi{a; b} \such \forall y \in Y\ \forall B_1, B_2 \in (B; b) \ \md{\int_{B_1}^{B_2} f(x, y)dx} < \eps
	\]
\end{theorem}

\begin{proof}~
	\begin{itemize}
		\item[$\Ra$] Утверждение теоремы совпадает с определением равномерной сходимости до первых трёх кванторов, а дальше рассуждения очевидны:
		\[
			\forall B_1, B_2 \in (B; b) \ \md{\int_{B_1}^{B_2} f(x, y)dx} = \md{\int_{B_1}^b f(x, y)dx - \int_{B_2}^b f(x, y)dx} < 2\eps
		\]
		\item[$\La$] Если выполнено условие Коши, то при любом $y \in Y$ выполнен критерий Коши о сходимости просто несобственного интеграла Римана, то есть существует поточечный предел. Это даёт право устремить $B_2$ к $b$, зафиксировав остальные переменные, и получить тем самым эквивалентное требуемому условие:
		\[
			\forall \eps > 0\ \exists B \in \lsi{a; b} \such \forall y \in Y\ \forall B_1 \in (B; b) \ \md{\int_{B_1}^b f(x, y)dx} \le \eps
		\]
	\end{itemize}
\end{proof}