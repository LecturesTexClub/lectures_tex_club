\documentclass[a4paper,12pt]{article}
%%% Работа с русским языком
\usepackage{cmap}                   % поиск в PDF
\usepackage{mathtext}               % русские буквы в фомулах
\usepackage[T2A]{fontenc}           % кодировка
\usepackage[utf8]{inputenc}         % кодировка исходного текста
\usepackage[english,russian]{babel} % локализация и переносы

    
\usepackage[normalem]{ulem}

%размеры полей
\usepackage[left=1.5cm,right=1cm, top=2cm,bottom=2cm,bindingoffset=0cm]{geometry}

%%% Дополнительная работа с математикой
\usepackage{amsfonts,amssymb,amsthm,mathtools} % AMS
\usepackage{amsmath}
\usepackage{icomma} % "Умная" запятая: $0,2$ --- число, $0, 2$ --- перечисление

\usepackage{graphicx} % Вставка Картинок
\usepackage{wrapfig}
\usepackage{sidecap}
\usepackage{tikz-cd}
\usepackage{tkz-euclide}
\usepackage{stackengine}
\usepackage{pgfplots}
\pgfplotsset{compat=1.18}

\usepackage{listings}

\usepackage{xcolor}

%New colors defined below
\definecolor{codegreen}{rgb}{0,0.6,0}
\definecolor{codegray}{rgb}{0.5,0.5,0.5}
\definecolor{codepurple}{rgb}{0.58,0,0.82}
\definecolor{backcolour}{rgb}{255, 255, 255}
\definecolor{keywordscolour}{rgb}{0, 0, 200}

\newcommand{\topics}[1]{\large\texttt{#1}\\\normalsize}

%Code listing style named "mystyle"
\lstdefinestyle{mystyle}{
  backgroundcolor=\color{backcolour}, commentstyle=\color{codegreen},
  keywordstyle=\color{keywordscolour},
  numberstyle=\tiny\color{codegray},
  stringstyle=\color{codepurple},
  basicstyle=\ttfamily\footnotesize,
  breakatwhitespace=false,         
  breaklines=true,                 
  captionpos=b,                    
  keepspaces=true,                 
  numbers=left,                    
  numbersep=5pt,                  
  showspaces=false,                
  showstringspaces=false,
  showtabs=false,                  
  tabsize=2
}




\renewcommand{\epsilon}{\ensuremath{\varepsilon}}
\renewcommand{\phi}{\ensuremath{\varphi}}
\renewcommand{\kappa}{\ensuremath{\varkappa}}
\renewcommand{\le}{\ensuremath{\leqslant}}
\renewcommand{\leq}{\ensuremath{\leqslant}}
\renewcommand{\ge}{\ensuremath{\geqslant}}
\renewcommand{\geq}{\ensuremath{\geqslant}}
\renewcommand{\emptyset}{\ensuremath{\varnothing}}

\DeclareMathOperator{\sgn}{sgn}
\DeclareMathOperator{\Ker}{Ker}
\DeclareMathOperator{\im}{Im}
\DeclareMathOperator{\re}{Re}
\DeclareMathOperator{\rank}{rank}
\DeclareMathOperator{\grad}{grad}
\DeclareMathOperator{\mes}{mes}
\DeclareMathOperator{\supp}{supp}
\newcommand{\vpint}{\mathop{\mathrm{v.p.}}\!\int}
\DeclareMathOperator*\lowlim{\underline{lim}}
\DeclareMathOperator*\uplim{\overline{lim}}

\newcommand{\N}{\mathbb{N}}
\newcommand{\Z}{\mathbb{Z}}
\newcommand{\Q}{\mathbb{Q}}
\newcommand{\R}{\mathbb{R}}
\newcommand{\D}{\mathcal{D}}
\newcommand{\Cm}{\mathbb{C}}
\newcommand{\F}{\mathbb{F}}
\newcommand{\id}{\mathrm{id}}
\newcommand{\DIF}{\mathcal{DIF}}
\newcommand{\Rim}{\mathcal{R}}
\newcommand{\M}{\mathcal{M}}
\newcommand{\cN}{\mathcal{N}}
\newcommand{\G}{\mathcal{G}}
\newcommand{\B}{\mathcal{B}}
\newcommand{\A}{\mathcal{A}}
\newcommand{\E}{\mathcal{E}}
\newcommand{\PP}{\mathcal{P}}
\newcommand{\MM}{\mathfrak{M}}
\newcommand{\NN}{\mathfrak{N}}
\newcommand{\LL}{\mathcal{L}}
\newcommand{\LIP}{\mathcal{LIP}}
\DeclareMathOperator{\loclip}{loc\mathcal{LIP}}
\DeclareMathOperator{\graph}{graph}
\DeclareMathOperator{\rot}{rot}
\DeclareMathOperator{\dive}{div}
\renewcommand{\div}{\dive}
\DeclareMathOperator*{\esssup}{ess\,sup}
% привееет

\newcommand{\incirc}[1]{\stackMath\mathbin{\stackinset{c}{0ex}{c}{0ex}{#1}{\bigcirc}}}


\def\Xint#1{\mathchoice
    {\XXint\displaystyle\textstyle{#1}}%
    {\XXint\textstyle\scriptstyle{#1}}%
    {\XXint\scriptstyle\scriptscriptstyle{#1}}%
    {\XXint\scriptscriptstyle\scriptscriptstyle{#1}}%
      \!\int}
\def\XXint#1#2#3{{\setbox0=\hbox{$#1{#2#3}{\int}$}
    \vcenter{\hbox{$#2#3$}}\kern-.5\wd0}}
\def\dashint{\Xint-}

\def\Yint#1{\mathchoice
    {\YYint\displaystyle\textstyle{#1}}%
    {\YYYint\textstyle\scriptscriptstyle{#1}}%
    {}{}%
    \!\int}
\def\YYint#1#2#3{{\setbox0=\hbox{$#1{#2#3}{\int}$}
    \lower1ex\hbox{$#2#3$}\kern-.46\wd0}}
\def\YYYint#1#2#3{{\setbox0=\hbox{$#1{#2#3}{\int}$}
    \lower0.35ex\hbox{$#2#3$}\kern-.48\wd0}}
\def\lowdashint{\Yint-}

\def\Zint#1{\mathchoice
    {\ZZint\displaystyle\textstyle{#1}}%
    {\ZZZint\textstyle\scriptscriptstyle{#1}}%
    {}{}%
    \!\int}
\def\ZZint#1#2#3{{\setbox0=\hbox{$#1{#2#3}{\int}$}
    \raise1.15ex\hbox{$#2#3$}\kern-.57\wd0}}
\def\ZZZint#1#2#3{{\setbox0=\hbox{$#1{#2#3}{\int}$}
    \raise0.85ex\hbox{$#2#3$}\kern-.53\wd0}}
\def\highdashint{\Zint-}




\newcommand{\imp}[2]{
    (#1\,\,$\ra$\,\,#2)\,\,
}
\newcommand{\System}[1]{
    \left\{\begin{aligned}#1\end{aligned}\right.
}
\newcommand{\Root}[2]{
    \left\{\!\sqrt[#1]{#2}\right\}
}

\renewcommand\labelitemi{$\triangleright$}

\let\bs\backslash
\let\Lra\Leftrightarrow
\let\lra\leftrightarrow
\let\Ra\Rightarrow
\let\ra\rightarrow
\let\La\Leftarrow
\let\la\leftarrow
\let\emb\hookrightarrow
\let\rr\rightrightarrows
\newcommand{\nrr}{\bcancel{\rightrightarrows}}

\newcommand{\xrr}{\underset{X}{\rightrightarrows}}

\newcommand{\xnrr}{\underset{X}{\bcancel{\rightrightarrows}}}




\newcommand{\dint}{\displaystyle\int}



%%% Теоремы
\theoremstyle{plain}
\newtheorem{theorem}{Теорема}[section]
\newtheorem{theoremdefinition}{Теорема-определение}[section]
\newtheorem{lemma}{Лемма}[section]
\newtheorem{proposition}{Утверждение}[section]
\newtheorem*{exercise}{Упражнение}
\newtheorem*{problem}{Задача}
\newtheorem*{question}{Вопрос}
\newtheorem*{answer}{Ответ}
\newtheorem*{explanation}{Пояснение}
\newtheorem*{fact}{Факт}
\newtheorem*{interpret}{Геометрическая интерпертация}


\theoremstyle{definition}
\newtheorem{definition}{Определение}[section]
\newtheorem*{corollary}{Следствие}
\newtheorem*{note}{Примечание}
\newtheorem*{reminder}{Напоминание}
\newtheorem*{example}{Пример}
\newtheorem*{examples}{Примеры}
\newtheorem*{counterexample}{Контрпример}
% \newtheorem*{solution}{Решение}
\newtheorem*{properties}{Свойства}
\newtheorem*{remark}{Замечание}


\theoremstyle{remark}
\newtheorem*{solution}{Решение}










%"mystyle" code listing set
\lstset{style=mystyle}

% Настройка оглавления и ссылок
\usepackage[unicode]{hyperref}
\hypersetup{
    unicode=true,            % русские буквы в раздела PDF
    colorlinks=true,         % Цветные ссылки вместо ссылок в рамках
    linkcolor=blue, % Внутренние ссылки
    citecolor=green,         % Ссылки на библиографию
    filecolor=magenta,       % Ссылки на файлы
    urlcolor=olive,       % Ссылки на URL
}

% Колонтикулы
\usepackage{titleps}
\newpagestyle{main}{
    \setheadrule{0.4pt}
    \sethead{Гармонический анализ}{}{\hyperlink{intro}{\;Назад к содержанию}}
    \setfootrule{0.4pt}                       
    \setfoot{ФПМИ МФТИ, \today}{}{\thepage} 
}
\pagestyle{main}  




\numberwithin{equation}{subsection}


\usepackage{comment}

\newcommand\ff{
	\stackon[-6pt]{f}{\rule{.8ex}{.08ex}}
}

