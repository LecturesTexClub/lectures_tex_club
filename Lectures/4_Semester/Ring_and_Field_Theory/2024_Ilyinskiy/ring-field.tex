\section{Кольца}
\subsection{Основные понятия}
\textbf{Определение.} $(K, +, \cdot)$ называется \textit{кольцом}, если $(K, +)$ является абелевой группой и выполняется дистрибутивность.

\textbf{Определение.} $K$ называется \textit{коммутативным кольцом}, если оно является кольцом с ассоциативностью и коммутативностью умножения и единицей.

\textbf{Определение.} Пусть $a, b \in K$.
Говорят, что \textit{$a$ делится на $b$}, или же $b~|~a$, если найдётся $c \in K$, такой что $a = bc$.

\textbf{Определение.} $a \in K \setminus \{0\}$ называется \textit{делителем нуля}, если найдётся $b \in  K \setminus \{0\}$, такой что $bc = 0$.

\textbf{Определение.} Пусть $K$ --- кольцо. Будем обозначать через $K[x]$ кольцо многочленов с коэффициентами из $K$.
Если же $K \subset L$ и $a \in L$, то $K[a]$ --- это либо многочлены из $K[x]$, в которые подставили $a$, либо пересечение всех надколец $K$, содержащих $a$.

\textbf{Определение.} \textit{Область целостности} --- это коммутативное кольцо без делителей нуля.

\textbf{Утверждение.} Если $K$ --- область целостности, то из $ac = bc$ при $c \ne 0$ следует $a = b$. Очевидно.

\textbf{Определение.} $K^*$ --- это множество всех обратимых элементов $K$, то есть делителей единицы.

\textbf{Утверждение.} Пусть $M_a$ --- множество делителей элемента $a \in K$.
Тогда $M_a = M_b \iff \exists c \in K^*: a = bc$. Очевидно.

\textbf{Следствие.} Группа $K^*$ действует на множество $K$ умножениями.
Здесь орбиты называются \textit{классами ассоциированности}, то есть $a \sim b$, если $\exists c \in K^*: a = bc$.

\textbf{Утверждение.} Ассоциированность является отношением эквивалентности, очевидно.

\textbf{Определение.} Элемент $a$ кольца $K$ называется \textit{неразложимым}, если $a \ne 0$, $a \not\in K^*$, и если мы смогли разложить $a$ на множители $a = bc$, то $b \in K^*$ или $c \in K^*$.

\textbf{Пример.} Рассмотрим кольцо $\mathbb Z[i]$.
Заметим, что оно представляет из себя числа вида $a + bi$ для $a, b \in \mathbb Z$.
Тогда неразложимыми элементами являются $\pm 1$, $\pm i$, а остальные --- нет из-за того, что модуль больше единицы.
Глобально, можно пользоваться понятием \textit{нормы}, $N(a + bi) = a^2 + b^2$, но это позже.

\textbf{Определение.} Область целостности $K$ называется \textit{факториальным кольцом}, если:
\begin{enumerate}
    \item Для любого $x \in K \setminus \{0\}$ существует разложение $x = u \cdot p_1 p_2 \dots p_s$, где $u \in K^*$ и $p_1, \dots, p_s$ неразложимы.

    \item Если $x \ne 0$ удалось разложить двумя способами: $x = u \cdot p_1 \dots p_s = w \cdot q_1 \dots q_s$, то можно перенумеровать неразложимые так, что все $p_i \sim q_i$.
\end{enumerate}

\textbf{Пример.} Второе свойство не всегда идёт вместе с первым: например, в кольце $\mathbb Z[2i]$ есть два разложения $4 = 2 \cdot 2 = (2i) \cdot (-2i)$.
Элементы $2$ и $2i$ не ассоциированы, так как мы не в кольце $\mathbb Z[i]$.

\textbf{Пример.} Первое тоже не всегда есть: например, кольцо корней многочленов из $\mathbb Z[x]$ со старшим коэффициентом, равным единице.
Тогда $\sqrt 2 = \sqrt[4]{2} \cdot \sqrt[4]{2} = \sqrt[8]{2} \dots$.

\textbf{Утверждение.} Поле является факториальным кольцом. Очевидно, так как все элементы неразложимы.

\textbf{Определение.} $\mathbb Z[i]$ называется кольцом \textit{гауссовых чисел}, $\mathbb Z[\omega]$ называется \textit{числами Эйзенштейна}, где $\omega$ --- корень третьей степени из единицы.
Обычно берут $\omega = -\frac{1}{2} + \frac{\sqrt 3}{2} i$.

Как доказывается факториальность кольца?
Во-первых, нужно доказать существования разложения, то есть ограничить глубину разложения, например, норма в $\mathbb Z[i]$.
Во-вторых, единственность, но здесь обычно можно доказать следующее свойство: если $x~|~ab$ и $x$ неразложим, то $x~|~a$ или $x~|~b$.
Из него следует единственность разложения.

\textbf{Определение.} $x \in K$ называется \textit{простым}, если $x \ne 0$, $x \not\in K^*$, и если $x~|~ab$, то $x~|~a$ или $x~|~b$.

\textbf{Утверждение.} Простой элемент неразложим, очевидно.

\textbf{Утверждение.} В факториальном кольце любой неразложимый элемент прост, очевидно.

\textbf{Теорема.} Пусть $K$ --- область целостности. Если любой неразложимый элемент прост и выполнено условие (1) факториальности кольца, то оно факториально.

\textbf{Доказательство.} По индукции можно доказать, что если $x~|~a_1 \dots a_t$, то найдётся $j$, такое что $x~|~a_j$.
Пусть у нас есть два разложения $y = u \cdot p_1 \dots p_s = w \cdot q_1 \dots q_l$.
Будем делать то же, что и в натуральных числах.
Сократим все ассоциированные, тогда все $p_i \not\sim q_j$.
Получаем, что $q_1~|~u \cdot p_1 \dots p_s$, теперь из неразложимости получаем, что один из сомножителей делится на $q_1$.
Если $q_1~|~u$, то по транзитивности делимости $q_1~|~1$, то есть $q_1 \in K^*$ --- противоречие.

Иначе $q_1~|~p_i$ для какого-то $i$.
По определению найдётся $a \in K$, такое что $p_i = a q_1$, а из неразложимости $a \in K^*$ или $q_1 \in K^*$.
Второе быть верно не может, а из первого следует, что $p_i \sim q_1$, но мы изначально сократили ассоциированные --- противоречие.

\QED

\subsection{Евклидовы кольца}
\textbf{Определение.} Область целостности $K$ называется \textit{евклидовым кольцом}, если существует $N: K \setminus \{0\} \to \mathbb Z_{\ge 0}$, такая что:
\begin{enumerate}
    \item $N(ab) \ge N(a)$.
    \item Деление с остатком: для любых $a, b$ найдутся частное $q$ и остаток $r$, такие что $a = bq + r$, причём $r = 0$ или $N(r) < N(b)$.
\end{enumerate}

\textbf{Замечание.} Может захотеться доопределить норму $N$ в нуле, чтобы в делении с остатком не было двух случаев, но тогда придётся везде писать $a, b \ne 0$, и это, в целом, не всегда удобно (например, в кольце многочленов).
Однако на практике в большинстве случаев норму можно сделать мультипликативной, и тогда с доопределением проблем не будет.
Например, у многочленов можно ввести $N(f) = 2^{\deg(f)}$.


\textbf{Пример.} Докажем, что кольцо гауссовых чисел является евклидовым.
Положим $N(a + bi) = a^2 + b^2$.
Первое свойство следует из того, что норма вне нуля положительна.
Разберёмся с делением с остатком: если $u = qv + r$, то $\frac{u}{v} = q + \frac{r}{v}$.
Так как $\left| \frac{r}{v} \right| < 1$, достаточно найти $q \in \mathbb Z[i]$, которое будет близко к $\frac{u}{v}$.
Это можно сделать геометрически: $\frac{u}{v}$ --- это какая-то точка на плоскости, а нам нужна точка с целочисленными координатами, недалёкая от неё.
Из соображений длины диагонали квадрата можно найти точку на расстоянии не более $\frac{\sqrt 2}{2}$.

\textbf{Утверждение.} Если в области целостности $K$ выполняется деление с остатком по норме $N$, то можно подкрутить норму так, чтобы выполнялось первое свойство.

\textbf{Доказательство.} Мы хотим сделать так, чтобы норма $a$ не превосходила норму всего, что можно из него получить умножением.
Так и определим:
\[
    \tilde N(a) = \min_{b \in K \setminus \{0\}}(N(ab)).
\]
Проверим свойства. 1) Сравним $\tilde N(ab)$ и $\tilde N(a)$.
Существует $c \in K \setminus \{0\}$, такое что $\tilde N(ab) = N(abc) \ge \min_{x \in K \setminus \{0\}}(N(ax)) = \tilde N(a)$.

2) Пусть $\tilde N(b) = N(bc)$.
Разделим $a$ на $bc$ с остатком по норме $N$: существуют $q$ и $r$, такие что $a = (bc)q + r$ и либо $r = 0$, либо $N(r) < N(bc)$.
Первый случай очевиден, рассмотрим второй случай:
\[
    \tilde N(r) \le N(r) < N(bc) = \tilde N(b).
\]
Остаётся сказать, что $cq$ --- это частное и $r$ --- остаток (так можно делать в силу ассоциативности).

\QED

\textbf{Теорема.} Евклидово кольцо факториально.

\textbf{Доказательство.} Индукцией по норме.
От противного: пусть $x \in K \setminus \{0\}$, и для него не существует разложения, причём среди всех таких $x$ рассмотрим элемент с минимальной нормой.
Пусть $x = ab$. Если $N(x) > N(a)$ и $N(x) > N(b)$, то по индукции $a$ и $b$ разложимы и $x$ разложим, как произведение.
Если $N(x) = N(a)$, то $N(ab) = N(a)$.
Разделим $a$ на $ab$ с остатком (это единственный способ выбрать делимое и делитель так, чтобы было нетривиально).
Тогда $a = ab \cdot q + r$.
Если $r = 0$, то $bq = 1$ и $b \in K^*$.
Иначе $N(r) < N(ab) = N(a)$.
Но $r = a(1 - bq)$, так что по свойству 1 $N(r) \ge N(a)$ --- противоречие.
Следовательно, $b$ обратим.
Таким образом, либо $x$ разложим, либо $b \in K^*$ и $x$ неразложим по определению.

Теперь докажем, что разложение единственно.
Для этого докажем лемму: если $p$ неразложимо и $p~|~ab$, то $p~|~a$ или $p~|~b$.
А для этого будем использовать обычновенный алгоритм Евклида: $r_1 = a$, $r_2 = b$, $r_n = q_{n+2} r_{n+1} + r_{n+2}$.
Так как нормы уменьшаются, это рано или поздно закончится: пусть $r_{n+1} = 0$.
Как и в натуральных числах $\exists x, y: r_n = ax + by$. Это доказывается индукцией по $n$.
Теперь индукцией по $n$ доказываем, что все НОДы $r_k$ и $r_{k+1}$ равны.

Обратно к лемме: пусть $p~|~ab$, но $p\nmid~a$.
Тогда $\gcd(a, p) = 1$ с точностью до ассоциированных.
Как известно, все делители $p$ --- это ассоциированные с ним или обратимые.
Тогда это же верно и для делителей $\gcd(a, p)$, то есть существуют $x$ и $y$, такие что $ax + py = 1$.
Остаётся домножить на $b$ и получить $p~|~b$.

\QED

\textbf{Замечание.} $r_n$ делится на любой общий делитель $a$ и $p$. Доказывается рассмотрением разложения $r_n = ax + py$.

\textbf{Пример.} Рассмотрим $R = \mathbb Z[i]$.
Как в нём искать неразложимые элементы, а для разложимых как искать это самое разложение?
Возьмём обычную меру $N(a + bi) = a^2 + b^2$, тогда она мультипликативна.
В этом случае $N(z) = 1 \iff z \in R^*$.
Более того, если $N(z) = p$ --- простое целое число (не элемент кольца), то $z$ неразложимо (доказывается прямой проверкой).
Есть ещё один случай неразложимых элементов: если $N(z) = p^2$ и нет элементов нормы $p$, то $z \sim p$ (ассоциированы) неразложим.

Интересный факт: все простые нормы вида $4k + 3$ дают неразложимые элементы.

\subsection{Неразложимые элементы}
\textbf{Обозначение.} $D$ --- одно из колец $\mathbb Z[i]$, $\mathbb Z[\sqrt 2 i]$, $\mathbb Z[\omega]$, 
На них норма --- $N(z) = z \cdot \overline z$: мультипликативна и $N(z) = 1\ \iff z \in D^*$.

\textbf{Теорема.} (Об описании неразложимых элементов) В кольце $D$ элемент $z$ неразложим тогда и только тогда, когда выполнено одно из:
\begin{itemize}
    \item $N(z) = p$, где $p$ --- простое целое число (не обязательно элемент кольца).
    \item $z \sim p$. В этом случае не существует элемента $w$, такого что $N(w) = p$.
\end{itemize}
В случае $D = \mathbb Z[i]$ числа $p$ в двух случаях имеют вид ($2$ или $4k + 1$) и $4k + 3$ соответственно, в случае $\mathbb Z[\omega]$ --- (3 или $3k + 1$) и $3k + 2$ соответственно.

\textbf{Доказательство.} $\Rightarrow$. Пусть $z$ неразложим, тогда $z$ прост.
Норма $N(z)$ --- это целое число, поэтому его можно разложить на простые множители: $N(z) = p_1 \cdots p_s$.
Причём $z~|~N(z)$, поэтому какое-то $p_j$ делится на $z$, то есть $p_j = xz$.
В силу мультипликативности нормы $N(p_j) = N(x) N(z)$, то есть делится на $N(z)$.
Следовательно, $p_j^2 = N(p_j)$ делится на $N(z)$, то есть $N(z) \in \{1, p_j, p_j^2\}$.
В первом случае $z$ обратим, это неинтересно.
А два оставшихся случая как раз дают следствие теоремы.

Докажем несуществование $w$, такого что $N(w) = p$: пусть $N(z) = p_j^2$, то есть $z \sim p_j$.
По условию $z$ неразложим, поэтому и $p_j$ неразложим.
Допустим, что существует $w$, такой что $N(w) = p_j$.
Так как $w~|~N(w)$, $p_j = N(w) = w \cdot u$.
Применим $N$ к обеим частям: $N(p_j) = N(w) N(u)$, подставляя, $p_j^2 = p_j \cdot N(u)$, то есть $N(u) = p_j$.
Таким образом, $p_j = w \cdot u$, и $N(u) = N(w) = p_j$ --- оба неразложимы, противоречие с неразложимостью $p_j$.

$\Leftarrow$. Пусть $z \in D$, и $N(z) = p$.
Разложим на множители: $z = a \cdot b$, тогда $N(a) \cdot N(b) = N(z) = p$, то есть $N(a) = 1$ или $N(b) = 1$.
Получается, что $a$ или $b$ обратим, то есть $z$ неразложим.

Пусть $z \sim p$ и не существует $w$, таких что $N(w) = p$.
Если $z = a \cdot b$, то $N(a)N(b) = N(z)$.
По определению ассоциированности $z = dp$ для $d \in D^*$.
То есть $N(z) = N(dp) = N(p) = p^2$.
Это равно произведению $N(a)$ и $N(b)$, поэтому либо $N(a) = 1$, либо $N(b) = 1$ и $z$ неразложим.

Теперь докажем часть про вид простых чисел в $\mathbb Z[i]$.
Если $p = N(z) = a^2 + b^2$, то, так как $a^2, b^2 \equiv 0, 1 \mod 4$, сумма не может иметь вид $4k + 3$.
Пусть $p = 4k + 1$, докажем, что существует разложение.
Найдём символ Лежандра... $\left( \frac{-1}{p} \right) = (-1)^{\frac{p-1}{2}} = 1$, то есть существует $x \in \mathbb Z$, такой что $x^2 + 1$ делится на $p$.
Это значит, что $(x + i)(x - i)$ делится на $p$.
Если бы $p$ было неразложимо, то одна из скобок бы делилась на $p$.
Заметим, что, глобально, если $p~|~(a + bi)$, то $p~|~a$ и $p~|~b$.
Поэтому $p~|~\pm 1$ --- противоречие.

Теперь для $\mathbb Z[\omega]$. Если $p = 3k + 2$, то $(a + b)^2 \equiv a^2 - ab + b^2 \mod 3$ --- получаем плохой остаток.
Если $p = 3k + 2$, то заметим, что $3 = \lambda \overline \lambda$, где $\lambda = 1 - \omega$.
Найдём символ Лежандра:
\[
    \left( \frac{-3}{p} \right) = \left( \frac{-1}{p} \right) \left( \frac{3}{p} \right) = (-1)^{\frac{p-1}{2}} \left( \frac{p}{3} \right) (-1)^{\frac{p-1}{2}} = \left(\frac{1}{3} \right) = 1.
\]
Следовательно, существует $x$, такое что $x^2 + 3$ делится на $p$, то есть $p~|~(x + \sqrt i)(x - \sqrt 3i)$.
Аналогично предыдущему случаю доказываем от противного, здесь $p~|~\pm 2$, ибо $\sqrt 3i = 2\omega + 1$.

\QED

\textbf{Следствие.} (Рождественская теорема Ферма) Число $n$ представимо в виде суммы двух квадратов тогда и только тогда, когда в разложении $n$ на простые делители все простые числа вида $4k + 3$ входят в чётных степенях.

\textbf{Доказательство.} 
Разделим $n$ на все квадраты простых чисел, входящих в разложение, теперь $n$ будет произведением $p_1 \cdots p_s$ различных простых чисел.
Также будем доказывать через гауссовы числа.

$\Leftarrow$. Рассмотрим одно из простых чисел в разложении $p$.
По доказанному $p$ разложимо, причём $N(p) = p^2$, то есть оно представимо в виде произведения двух чисел $(a + bi)(a - bi)$ нормы $p$.
Вот, собственно, и сумма --- $(a + bi)(a - bi) = a^2 + b^2$.
Теперь для представления $n$ в виде суммы двух квадратов воспользуемся тождеством
\[
    (a^2 + b^2)(c^2 + d^2) = (ac - bd)^2 + (ad - bc)^2.
\]

$\Rightarrow$. От противного: $n = a^2 + b^2$, но следствие нарушается.
Заметим, что $a$ и $b$ взаимно просты (иначе бы $n$ делилось на квадрат).
Пусть, без ограничения общности, $p_1 = 4k + 3$.
Так как $p_1~|~n = a^2 + b^2$, имеем $p_1~|~N(a + bi)$.
Если $p_1^2~|~N(a + bi)$, то $p_1^2~|~n$ и получим противоречие с первым шагом доказательства.
Иначе разложим $a + bi$ на неразложимые: пусть $c + di$ --- делитель, норма которого делится на $p_1$.
По теореме об описании неразложимых $N(c + di) = p_1^2$ --- противоречие.

\QED

\textbf{Следствие 2.} Алгоритм разложения на неразложимые в кольце $\mathbb Z[i]$.
Рассмотрим $z \in \mathbb Z[i]$, пусть $N(z) = p_1 \cdots p_s$.
Простые вида $4k + 3$ неразложимы, их пропускаем.
Простые вида $4k + 1$ или $2$ раскладываются дальше, например, $5 = (1 + 2i)(1 - 2i)$.

\textbf{Фан факт.} Положим $\mathcal O(d)$ --- корни многочленов из $\mathbb Z[x]$ со старшим коэффициентом, равным $1$, в $\mathbb Q[\sqrt d]$.
Тогда при $d = -1, -2, -7, -11$ это Евклидовы кольца, а при $d = -19, -43, -67$ и ещё одном значении получаются не Евклидовы, но ОТА там работает.
При остальных $d$ ОТА работает только для идеалов.

\subsection{Пифагоровы тройки}
Научимся перечислять решения уравнения $x^2 + y^2 = z^2$.
Сразу будем считать, что $x$ и $y$ взаимно просты, и перейдём в гауссовы числа, тогда $(x + yi)(x - yi) = z^2$.
Заметим, что $(x + yi, x - yi) = (2x, x + yi)$.

Пусть этот НОД равен $d$.
Заметим, что если $d~|~x$, то $d~|~y$, поэтому в этом случае $d = 1$ с точностью до ассоциированности.
Иначе пусть $d \nmid x$, тогда $N(d) > 1$.
Более того, $N(d)~|~4x^2$, откуда $N(d)$ чётно (если нечётно, то $d^2~|~x^2$).
Следовательно, $d$ чётно.
Но тогда получается, что $d~|~x + yi$, то есть $2~|~x, y$ --- противоречие со взаимной простотой.

Следовательно, остаётся лишь случай $d = 1$ (с точностью до ассоциированности).
Так как $(x + yi)(x - yi) = z^2$, множители можно записать в виде $x + yi = a_1 b_1^2$, $x - yi = a_2 b_2^2$, где $a_1, a_2$ свободны от квадратов.
Допустим, что $a_1 \not\in \mathbb Z[i]^*$, тогда существует неразложимый $p~|~a_1$.
Так как $z^2 = a_1 a_2 b_1^2 b_2^2$, $p~|~\frac{z^2}{b_1^2 b_2^2}$ --- это полный квадрат, поэтому $p^2~|~\frac{z^2}{b_1^2 b_2^2} = a_1 a_2$.
По условию $a_1$ свободно от квадратов, откуда $p~|~a_2$.
Следовательно, НОД чисел $x + yi$, $x - yi$ делится на $p$, то есть его норма больше единицы --- противоречие.

Итак, $a_1, a_2 \in \mathbb Z[i]^*$, то есть равны $\pm 1$ или $\pm i$.
Распишем $b_1$ через целые числа: $b_1 = u + vi$.
Тогда $x + yi = a_1 (u^2 - v^2 + 2uvi)$.
Если $a_1 = \pm 1$, то получаем $x = \pm (u^2 - v^2)$ и $y = \pm 2uv$, если же $a_1 = \pm i$, то наоборот.
В обоих случаях $z = u^2 + v^2$, и, как можно проверить подстановкой, все такие тройки подходят.

\section{Идеалы}
\subsection{Кольца главных идеалов}
\textbf{Определение.} Идеал $I$ в коммутативном кольце $K$ --- это подмножество, такое что
\begin{enumerate}
    \item $(I, +)$ --- абелева группа.
    \item $\forall a \in K~\forall x \in I~ax \in I$.
\end{enumerate}

\textbf{Утверждение.} $I \subset K$ является идеалом тогда и только тогда, когда оно замкнуто относительно сложения и второе свойство.

\textbf{Доказательство.} $\Rightarrow$ очевидно. $\Leftarrow$. Проверим, что $0 \in I$: возьмём $a = 0$, $x \in I$, тогда их произведение $0 \in I$.
Проверим, что обратный лежит: возьмём $a = -1$, $x \in I$, тогда $ax = -x \in I$.

\QED

\textbf{Определение.} Пусть $x \in K$. Тогда \textit{главный идеал} --- это $(x) = \{ax~|~a \in K\}$.

\textbf{Определение.} Пусть $x_1, \dots, x_n \in K$.
Тогда $(x_1, \dots, x_n) = \{a_1x_1 + \dots + a_n x_n~|~a_1, \dots, a_n \in K\}$ --- \textit{конечно порождённый идеал}.

\textbf{Замечание.} Не все идеалы являются главными: возьмём кольцо $\mathbb Q[x, y]$ и идеал $(x, y)$.

\textbf{Замечание 2.} И не все идеалы конечно порождены: возьмём кольцо многочленов над $\mathbb Q$ со счётным числом переменных.

\textbf{Определение.} \textit{Кольцо главных идеалов} --- область целостности, в которой все идеалы главные.

\subsection{Идеалы и делимость}
Заметим, что $(a)$ --- это все элементы, которые делятся на $a$.

\textbf{Свойства.}
\begin{itemize}
    \item $a~|~b \iff (a) \supset (b)$.
    \item Если $a \sim b$, то $(a) = (b)$, так как делятся друг на друга.
    \item $(a) + (b) = ((a, b))$ (это НОД).
    \item $(a) \cap (b) = ([a, b])$ (это НОК).
\end{itemize}

Цель параграфа --- доказать, что кольца главных идеалов лежат между факториальными и евклидовыми кольцами.
Для этого переформулируем определение евклидовых колец: $K$ евклидово, если существует норма $N$, такая что работает деление с остатком: для $a, b \ne 0$ либо $b~|~a$, либо найдётся $g$, такой что $N(a - bg) < N(b)$.
А теперь обобщим:

\textbf{Определение.} Область целостности $K$ \textit{обладает нормой Дедекине-Хассе}, если существует норма $N: K \setminus \{0\}: \to \mathbb Z_{\ge 0}$, такая что для любых $a, b \ne 0$: либо $b~|~a$, либо найдутся $x, y$, такие что $N(ax - by) < N(b)$.
В частности, $ax - by \ne 0$, так как $N(ax - by)$ определена.

То есть коэффициент при $a$ теперь не обязан быть единицей.

\textbf{Теорема.} $K$ является кольцом главных идеалов тогда и только тогда, когда $K$ обладает нормой Дедекине-Хассе.

\textbf{Доказательство.} $\Leftarrow$. Рассмотрим идеал $I \subset K$, а в нём --- элемент с наименьшей нормой $d$.
Теперь любой элемент либо делится на $d$, либо разделим с остатком и противоречие с выбором $d$.
Следовательно, $I = (d)$.
Доказательство в другую сторону будет позже.

\QED

\textbf{Теорема.} Кольцо главных идеалов факториально.

\textbf{Доказательство.} Докажем существование разложения.
От противного: здесь нужно аккуратно сформулировать, что это значит.
Пусть у элемента $a_0$ нет разложения.
Тогда при любом разложении $a_0 = b_1 \cdot c_1$ один из множителей можно бесконечно раскладывать, б.о.о $a_1$, а другой необратим, то есть $b_1 \not\in K^*$.
Повторяем для $a_1$: $a_1 = a_2 \cdot b_2$ и так далее, получаем цепочку $a_0, a_1, a_2, \dots$.
Тогда их идеалы вложены друг в друга: $(a_0) \subset (a_1) \subset (a_2) \subset \dots$.
Докажем, что эта цепочка стабилизируется: положим $I = \bigcup_{i=0}^{\infty} (a_i)$.
Очевидно, что это идеал, а значит, $I = (c)$ для $c \in K$ (так как мы в кольце главных идеалов).
Но тогда $c \in (a_N)$ для какого-то $N$, и $(a_{N}) = (a_{N+1}) = \dots$.
Следовательно, при $n \ge N$ мы брали обратимые $b_n$ --- противоречие.

Докажем единственность, а именно, лемму Евклида: если $p~|~ab$, где $p$ неразложим и $p \nmid~a$, то $p~|~b$ --- неразложимый элемент прост.
Положим $I = \{x \in K~|~p~|~ax\}$.
Это идеал (прямой проверкой), причём $1 \not\in I$ и $b, p \in I$, а ещё $I = (d)$, так как кольцо главных идеалов.
$d$ необратим, так как нет единицы, поэтому $d~|~p$, то есть $p \sim d$ и $I = (p)$.
$b \in I$, поэтому $p~|~b$.

\QED

Обратно к предыдущей теореме. Зная, что $K$ факториально, возьмём с потолка норму и докажем, что подходит.
Положим
\[
    N(x) = 2^{\text{количество простых в разложении $x$ с повторами}}.
\]
Возьмём $a, b \ne 0$, такие что $b\nmid~a$.
Пусть $(a, b) = (d)$ (как идеалы).
Тогда $d~|~a, b$, так как $a, b \in (d)$.
Отсюда $N(b) \ge N(d)$.
Более того, $b \ne d$, откуда неравенство строгое.
Остаётся взять разложение $d = ax - by$ из того, что $d \in (a, b)$.

\QED

\textbf{Пример.} Кольцо главных идеалов, не являющееся евклидовым:
\[
    \mathbb Z \left[ -\frac{1}{2} + \frac{\sqrt{19} i}{2} \right].
\]
Доказательство занимает около 40 минут и пропущено.

\subsection{Факторкольца}
\textbf{Определение.} Пусть $K, L$ --- кольца. Отображение $f: K \to L$ называется \textit{гомоморфизмом}, если оно сохраняет операции и единицу.

\textbf{Свойства.}
\begin{itemize}
    \item Если $I \subset K$ --- идеал, то $f(I)$ не обязательно идеал в $L$. Пример: $K = \mathbb Z$, $L = \mathbb Q$, $f(x) = x$.
    \item Но $f(I)$ является идеалом в $f(K)$.
    \item Если $I \subset L$ --- идеал, то $f^{-1}(I)$ --- идеал в $K$.
\end{itemize}

\textbf{Определение.} Факторкольцо --- $K/I$, всё, как обычно.

\textbf{Теорема.} (О гомоморфизмах) $K / \Ker(f) \cong \im(f)$.

\textbf{Теорема.} (О гомоморфизмах, 2) Если $I \subset J \subset K$, то $(K / I) / (J / I) \cong (K / J)$.
Смысл её в том, что мы можем факторизовать по очереди.
Например, для $\mathbb Z[x] / (5, x - 2)$ можно сначала найти $\mathbb Z[x] / (5)$, потом по $(x - 2)$, причём порядок не важен.

Пусть $I \subset K$, $K$ --- область целостности.
Когда $K/I$ является областью целостности?
Когда для всех $a, b \in K$ верно, что если $[a] \cdot [b] = [0]$, то $[a] = [0]$ или $[b] = [0]$ (здесь $[x] = x + I$).
То есть если $ab \in I$, то $a \in I$ или $b \in I$.
Отсюда вытекает

\textbf{Определение.} Идеал $I$ \textit{простой}, если $I$ нетривиальный и для всех $ab \in I$ верно $a \in I$ или $b \in I$.

\textbf{Утверждение.} Пусть $I$ --- нетривиальный идеал.
Тогда $K / I$ --- область целостности тогда и только тогда, когда $I$ простой.
Доказательство двумя абзацами выше.

\textbf{Утверждение.} Число $x$ простое тогда и только тогда, когда $(x)$ простой.

Когда $K/I$ является полем? Например,

\textbf{Утверждение.} Пусть $F$ --- коммутативное кольцо. Тогда $F$ является полем тогда и только тогда, когда в $F$ нет нетривиальных идеалов.

\textbf{Доказательство.} $\Rightarrow$. Пусть $I \subset F$.
Либо $I = (0)$, что нас устраивает, либо $a \in I \setminus \{0\}$, тогда $1 = a \cdot a^{-1} \in I$ и $I = F$, что нас вновь устраивает.

$\Leftarrow$. Пусть $a \in F$, $a \ne 0$.
Рассмотрим $(a)$. $(a) = (0)$ быть не может, поэтому $(a) = F$.
Следовательно, $a~|~1$, то есть существует $b \in F$, такой что $ab = 1$ --- обратим.

\QED

\textbf{Утверждение.} Пусть $K$ --- коммутативное кольцо.
$K/I$ --- поле тогда и только тогда, когда в $K$ нет нетривиальных идеалов, строго содержащих $I$.
Следует из того, что можно установить биекцию между идеалами $K/I$ и идеалами, содержащими $I$, --- по доказанному в поле идеалов нет.

\textbf{Определение.} Идеал $I$ \textit{максимальный}, если нет идеала $I \subsetneq J \subsetneq K$.

\textbf{Следствие.} Любой максимальный идеал простой.
Следует из того, что $K/I$ --- поле.

\textbf{Утверждение.} В кольце главных идеалов любой простой идеал максимальный.

\textbf{Утверждение.} В кольце главных идеалов все нетривиальные простые идеалы максимальны.

\textbf{Доказательство.} Если $I = (x)$ --- простой идеал, то $x$ простой, а значит, если $x \subset (y)$, то $y~|~x$.
Тогда либо $x \sim y$, либо $(y) = K$, оба случая нам подходят.

\QED

\textbf{Следствие.} Если идеал $I$ лежит в области целостности $K$, причём $K/I$ --- это область целостности, но не поле, то $K$ не является кольцом главных идеалов и не евклидово.
Пример --- $\mathbb Q[x, y] / (y) \cong \mathbb Q[x]$, поэтому $\mathbb Q[x, y]$ --- не кольцо главных идеалов.

\textbf{Следствие 2.} Пусть $F$ --- поле, $f(x)$ --- неприводимый многочлен над $F$.
Тогда $F[x] / f(x)$ --- поле.

\textbf{Доказательство.} Так как $F[x]$ --- кольцо главных идеалов, $f(x)$ является простым, откуда $(f(x))$ простой, максимальный, а значит, $F[x] / (f(x))$ --- поле.

\QED

\textbf{Теорема.} (б/д, доказательство --- на отл.(10)) Если $K$ --- факториальное кольцо, то $K$ является кольцом главных идеалов тогда и только тогда, когда любой нетривиальный простой идеал максимален.

\textbf{Следствие.} Если для любого идеала $I \subset K$ верно, что $K/I$ конечен, то $K$ факториально тогда и только тогда, когда $K$ --- кольцо главных идеалов.
Здесь мы пользуемся тем, что конечная область целостности является полем.

\textbf{Оффтоп.} Из теории групп мы знаем, что конечно порождённые абелевы группы можно разложить на $\mathbb Z \oplus \dots \oplus \mathbb Z \oplus \mathbb Z_{p_1^{\alpha^1}} \oplus \dots \oplus \mathbb Z_{p_s^{\alpha_s}}$.
Для колец есть похожая вещь: если $K$ --- кольцо главных идеалов, то конечно порождённый модуль представим в виде $K \oplus \dots K \oplus K/(p_1^{\alpha_1}) \oplus \dots$.
А теперь возьмём $K = \mathbb C[T]$, где $T$ --- линейный оператор над $V$.
Тогда
\[
    V = \mathbb C[T] / (T - \lambda_1)^{m_1} \oplus \dots \oplus \mathbb C[T]/(T - \lambda_s)^{m_s}
\]
--- жорданова нормальная форма.

\subsection{Нётеровы кольца}
\textbf{Теорема.} Пусть $K$ --- область целостности.
Следующие условия эквивалентны:
\begin{enumerate}
    \item Любой идеал конечно порождён.
    \item Любая цепочка возрастающих идеалов стабилизируется.
    \item Любая цепочка вида $(a_0) \subset (a_0, a_1) \subset (a_0, a_1, a_2) \subset \dots$ стабилизируется.
\end{enumerate}

\textbf{Доказательство.} $2 \Rightarrow 3$ очевидно. 

$3 \Rightarrow 1$. Пусть $I \subset K$ --- идеал.
Возьмём $a_0 \in I$. Есть 2 варианта: либо $I = (a_0)$, либо есть $a_1 \in I \setminus (a_0)$.
Повторяем --- рано или поздно мы стабилизируемся.


$1 \Rightarrow 2$. Рассмотрим цепочку $I_0 \subset \dots \subset I_n \subset \cdots$.
Возьмём объединение всех идеалов $J = \bigcup_{j=0}^{\infty} I_j$ --- легко проверить, что это идеал.
Так как он конечно порождён, $J = (x_0, \dots, x_n)$.
Теперь возьмём первый идеал из цепочки, который содержит все порождающие элементы --- дальше все будут совпадать.

\QED

\textbf{Определение.} Такие кольца называются \textit{нётеровыми}.

\textbf{Утверждение.} (б/д) Если $K$ нётерово, то $K/I$ нётерово.

\textbf{Теорема.} (Гильберта о базисе) Если $K$ нётерово, то $K[x]$ нётерово.

\textbf{Доказательство.} Пусть $I \subset K[x]$, докажём его конечную порождённость, от противного.
Возьмём многочлен с минимальной степенью $f_0$ из $I$, $f_1$ --- из $I \setminus (f_0)$, и так далее.
Обозначим за $d_0, d_1, \dots$ степени взятых многочленов, $a_i$ --- их старшие коэффициенты.

Теперь возьмём идеалы на $a_i$, тогда по нётеровости кольца $K$ получим, что $(a_0, a_1, \dots) = (a_0, \dots, a_N)$.
В частности, мы можем выразить $a_{N+1}$ через предыдущие: $a_{N+1} = a_0 b_0 + \dots + a_N b_N$.
Теперь уменьшим степень $f_{N+1}$, вычев из него
\[
    b_0 f_0 x^{d_{N+1} - d_0} + \dots + b_N f_N x^{d_{N+1} - d_N}.
\]
Мы получили многочлен $g$, лежащий в $I$, причём он не лежит в $(f_0, \dots, f_N)$ (иначе там же лежит $f_N$).
Следовательно, пришли к противоречию с минимальностью степени $f_{N+1}$.

\QED

\textbf{Доказательство в лоб.} Пусть $I \subset K[x]$ --- идеал.
Положим $J$ --- множество старших коэффициентов многочленов в $I$ --- это, очевидно, идеал в $K$, так что его можно конечно породить $J = (a_1, \dots, a_n)$.
($J$ замкнут относительно сложения, так как два многочлена из $I$ можно привести к одной степени, после чего сложить)

Пусть $f_1, \dots, f_n \in K[x]$ --- многочлены, старшие коэффициенты которых породили $J$.
Зафиксируем $d := \max \{\deg(f_1), \dots, \deg(f_n)\}$.
Теперь положим $J_k$ --- множество старших коэффициентов многочленов в $I$, степень которых не превосходит $k$.
Как и с $J$, это идеал, и можно взять многочлены $f_{k,1}, \dots, f_{k,n_k} \in J_k$.

Пусть $I^* = \left<J \cup \{f_{k,i}~|~i \le n_k, k \le d\} \right>$, он, очевидно, конечно порождён и вложен в $I$.
Докажем, что $I \subset I^*$, от противного: допустим, что нашёлся $g \in I \setminus I^*$, возьмём с минимальной степенью.
Дополнительно скажем, что $a$ --- старший коэффициент многочлена $g$.

Если $\deg(g) > d$, то $a \in J$, то есть $a = \sum_{i=1}^{n} \lambda_i a_i$.
Рассмотрим многочлен $g - \sum_{i=1}^{n} \lambda_i x^{\deg(g) - \deg(f_i)} f_i$: он всё ещё не лежит в $I^*$, но его степень строго меньше --- противоречие.

Иначе $a \in J_k$, где $k = \deg(g)$. Аналогично.

\QED

\textbf{Минутка пафоса.} Если у нас есть семейство многочленов $f_k(x_1, \dots, x_n)$ на $n$ переменных над $\mathbb C$, и мы рассмотрим систему
\[
    \begin{cases}
        f_1(x_1, \dots, x_n) = 0 \\
        f_2(x_1, \dots, x_n) = 0 \\
        \vdots
    \end{cases},
\]
то все эти многочлены образуют идеал в кольце $\mathbb C[x_1, \dots, x_n]$.
По теореме Гильберта о базисе можно выделить конечное число порождающих элементов (отсюда базис в названии), и исходная система эквивалентна новой конечной системе.

\textbf{Теорема.} (Ласкера-Нётер, б/д) Любой идеал в нётеровом кольце является конечным пересечением примарных идеалов.
Примарный идеал --- простой идеал в какой-то степени (умножение идеалов определено, степень получаем интуитивным образом).

\textbf{Теорема.} Если $K$ факториально, то $K[x]$ факториально. Доказательство чуть позже.

\textbf{Определение.} Хотелось бы построить поле из кольца, чтобы строить разложения.
Будем это делать аналогично рациональным числам. \textit{Поле частных} --- это
\[
    \Quot(K) = \{(a, b)~|~a \in K, b \in K \setminus \{0\}\} / \sim,
\]
где $(a, b) \sim (b, c) \iff ad = bc$.

\textbf{Лемма.} Это отношение эквивалентности порождается заменами $(a, b) \to (xa, xb)$, где $x \ne 0$.

Теперь вводим операции очевидным образом (как на дробях) и доказываем, что получилось поле.

Итак, положим $F = \Quot(X)$, мы знаем, что $K$ факториально, $F$ --- поле, $F[x]$ факториально, и теперь мы хотим что-то сказать про факториальность $K[x]$.
Аналогия: $K = \mathbb Z$, $F = \mathbb Q$.

\textbf{Утверждение.} Если $p$ --- неразложимый элемент в $K$, то $p$ --- простой элемент в $K[x]$ (как многочлен степени 0).

\textbf{Доказательство.} Рассмотрим $K[x] / (p)$: оно изоморфно $K/(p)[x]$, поэтому оба кольца являются областями целостности.
Следовательно, $p$ прост в $K[x]$.

\QED

\textbf{Определение.} Пусть $f \in K[x]$. Его \textit{носителем} называется $C(f) = gcd(a_n, \dots, a_0)$ --- НОД его коэффициентов.

\textbf{Определение.} Многочлен $f \in K[x]$ называется \textit{примитивным}, если $C(f) \sim 1$.

\textbf{Утверждение.} Если $A f_1 \cdots f_k = B g_1 \cdots g_l$, где $f_i$ и $g_i$ --- примитивные над $K$ и $A, B \in K$, то $f_1 \cdots f_k$ --- примитивный многочлен и $A \sim B$.
То есть многочлены можно сократить.

\textbf{Доказательство.} Докажем примитивность произведения: пусть $f_1 f_2$ не примитивен.
Тогда $f_1 f_2$ делится на какой-то просто $p \in K$.
Значит, один из них делится на $p$, что противоречит их исходной примитивности.

Теперь заметим, что $C(Af_1 \cdots f_k) = A$ и $C(Bg_1 \cdots g_l) = B$, откуда $A \sim B$.

\QED

\textbf{Обозначение.} $f(x)$ --- многочлен над $K[x]$, $\widehat f(x)$ --- примитивная компонента $f$, $\widetilde f(x)$ --- многочлен над $F[x]$, где $F = \Quot(K)$.

Тогда любой многочлен $\widetilde f$ можно написать в виде $\widetilde f(x) = \frac{A}{B} \widehat f(x)$.

\textbf{Утверждение 2.} Если $f(x) = \widetilde g(x) \cdot \widetilde h(x)$, то $\widehat f(x) \sim \widehat g(x) \cdot \widehat h(x)$ (доказывается прямой проверкой со свойством выше).

Теперь наконец-то можно перейти к доказательству факториальности $K[x]$.
Нужно доказать две вещи: у любого элемента существует разложение и неразложимый элемент прост.

Пусть $f(x) \in K[x]$ неразложим.
Если $\deg(f) = 0$, то $f \in K$, откуда простота следует из простоты в $K$ --- факториальном кольце.
Пусть $\deg(f) > 0$, тогда $C(f) \sim 1$ (иначе есть очевидное разложение), то есть $f$ примитивен.

\textbf{Утверждение 3.} Пусть $f(x) \in K[x]$, $\deg(f) > 0$. Тогда $f(x)$ неприводим в $K[x]$ тогда и только тогда, когда $f(x)$ примитивен и неприводим в $F[x]$.

\textbf{Доказательство.} $\Rightarrow$. Примитивность доказали, докажем неприводимость.
Допустим, что нашлось: пусть $f(x) = \overline g(x) \overline h(x)$. По утверждению 2 $\widehat f(x) \sim \widehat g(x) \widehat h(x)$.
Следовательно, $\widehat f(x)$ приводим в $K[x]$ --- противоречие.

$\Leftarrow$. Пусть $f(x) = g(x) h(x)$.
Если $\deg(g), \deg(H) > 0$, то $F[x]$ приводим в $F[x]$ --- противоречие.
Иначе одно из $g(x), h(x)$ обратимо, что следует из примитивности $f$.

\QED

\textbf{Утверждение 4.} Если $\deg(f) > 0$ и $f$ неприводим в $K[x]$, то $f(x)$ прост.

\textbf{Доказательство.} $f(x)$ примитивен и неприводим: пусть $g(x) h(x)$ делится на $f(x)$ в $K[x]$.
Без ограничения общности $g(x) = f(x) \widehat q(x)$ (разделили над $F[x]$).
Тогда $\widehat g(x) \sim \widehat f(x) \widehat q(x)$, а значит, $g(x) = A f(x) \widehat q(x)$ и $f(x)~|~g(x)$.

\QED

На этом моменте мы доказали всё необходимое для факториальности $K[x]$.

\subsection{Признак неприводимости Эйзенштейна}
Пусть $a_n x^n + \dots + a_0 = f(x) \in K[x]$, $I \subset K$ --- простой идеал.
Если $a_n \not\in I$, $a_{n-1}, \dots, a_0 \in I$ и $a_0 \not\in I^2$, то $f(x)$ неприводим в $\Quot(K[x])$.

\textbf{Доказательство.} Рассмотрим $K/I[x]$ --- область целостности, так как $I$ прост.
Пусть $f(x)$ приводим, то есть $f(x) = g(x) h(x)$.
Рассмотрим их над $K/I[x]$: получится $[a_n] x^n = [b_k] x^k [c_l] x^l$ --- слева все остальные члены исчезли, так как они по условию лежат в идеале.
Отсюда следует, что $[b_0] = [c_0] = [0]$, так как в равенстве они исчезли, поэтому $a_0 = b_0 \cdot c_0 \in I^2$ --- противоречие.

\QED

\textbf{Другой признак.} Пусть $f(x) \in \mathbb Z[x]$, $[f](x) \in \mathbb Z/(p)[x]$.
Если $[f](x)$ неприводим и $p\nmid a_n$, то $f(x)$ неприводим.

\textbf{Другой признак.} Пусть $f(x) \in K[x]$, $I$ --- максимальный идеал, $a_n \not\in I$.
Тогда если $[f](x)$ неприводим в $K/I[x]$, то $f(x)$ неприводим в $\Quot(K)[x]$.

\textbf{Доказательство.} Пусть $f(x) = g(x) h(x)$, тогда $[f](x) = [g](x) \cdot [h](x)$, откуда $[a_n] = [b_m] \cdot [c_l]$, где $b_m$ и $c_l$ --- старшие коэффициенты.
Тогда $[b_m] \cdot [c_l] \not\in 0$, то есть нашли разложение в $K/I[x]$.

\QED

\textbf{Упражнение.} Придумать многочлен из $\mathbb Z[x]$ со старшим коэффициентом 1, такой что $f(x)$ неприводим, а $f(x) \mod p$ приводим для всех $p$.

\textbf{Напоминание.} Пусть $F$ --- поле. Если $\deg(f) = 1$, то $f$ неприводим.
Если $\deg(f) = 2$ или $3$, то $f(x)$ неприводим тогда и только тогда, когда у $f(x)$ нет корней.

\textbf{Утверждение.} (Из школы) Если $a_n x^n + \dots + a_0 \in \mathbb Z[x]$ и $\frac{p}{q}$ --- корень, где $(p, q) = 1$, то $q~|~a_n$ и $p~|~a_0$.

\textbf{Утверждение.} Пусть $p$ --- простое целое число.
Тогда $x^{p-1} + \dots + x + 1$ неприводим над $\mathbb Q$.

\textbf{Доказательство.} Многочлен равен $\frac{x^p - 1}{x - 1}$, поэтому хочется сделать замену $t = x - 1$.
Тогда это равно
\[
    (t + 1)^{p-1} + \dots + (t + 1) + 1 = \frac{(t + 1)^p - 1}{t}.
\]
Можно заметить, что это будет
\[
    t^{p-1} + C_p^1 t^{p-2} + \dots + C_p^{p-1}.
\]
По признаку Эйзенштейна многочлен неприводим, идеал --- $(p)$.

\QED

\section{Расширение полей}
Вопросы этого параграфа: разрешимость в радикалах, основная теорема алгебры и замечательная

\textbf{Теорема.} Правильный $m$-угольник можно построить цикрулем и линейкой тогда и только тогда, когда $\phi(m) = 2^k$.

\textbf{Напоминание.} Характеристика поля --- ноль или простое число. При гомоморфизме $\phi: F \to L$ выполнено $\Char(F) = \Char(L)$.
Любой гомоморфизм --- это вложение, ибо ядро либо равно нулю, либо всему полю.

\textbf{Утверждение.} Пусть $F \subset L$ --- расширение поля. Тогда $L$ --- линейное пространство над $F$.

\textbf{Определение.} $[L:F]$ --- \textit{степень расширения}, то есть размерность $L$, как линейного пространства.

\textbf{Утверждение.} Пусть $F \subset L \subset K$ --- ``башня расширений``.
Тогда $[K:F] = [K:L] \cdot [L:F]$ (при условии, что всё конечно).

\textbf{Определение.} Расширение \textit{конечно}, если $[L:F] < \infty$, иначе \textit{бесконечно}.

\textbf{Определение.} $F(\alpha)$ --- поле с элементом $\alpha \in L \setminus F$.
Можно определять двумя способами:
\begin{itemize}
    \item $\Quot(F[\alpha])$.
    \item Пересечение всех полей $K$, таких что $F \subset K \subset L$ и $\alpha \in K$.
\end{itemize}

Расширения $F \subset L$ глобально делятся на конечные и бесконечные.
Первые можно записать в виде $L = F(\alpha_1, \dots, \alpha_k)$.

\textbf{Определение.} Алгебраический элемент $\alpha \in L$ --- такой элемент, что $[F(\alpha) : F] < \infty$ или, эквивалентно, найдётся многочлен $f \in F[x]$, такой что $f(\alpha) = 0$.

\textbf{Определение.} Минимальный многочлен $m_{\alpha, F}(x)$ --- многочлен, корнем которого является $\alpha$, и одно из эквивалентных:
\begin{itemize}
    \item Его степень минимальна.
    \item Он неприводим.
    \item $(m_{\alpha, F})$ совпадает с идеалом $\{g(x)~|~g(\alpha) = 0\}$.
\end{itemize}

\textbf{Утверждение.} $F[\alpha] \cong F(\alpha) \cong F[x] / (m_{\alpha, F}(x))$.

\subsection{Поле разложения многочлена}
\textbf{Определение.} $L$ --- поле разложения многочлена $f(x)$, если:
\begin{itemize}
    \item $f(x)$ линейно факторизуем над $L$.
    \item $L$ минимально по включению.
\end{itemize}

\textbf{Утверждение.} Для поля $F$ и $f(x) \in F[x]$ поле разложения $L$ многочлена $f$ существует и $[L : F] \le (\deg(f(x)))!$.

\textbf{Примеры.} Все над $\mathbb Q$:
\begin{itemize}
    \item $x - 1$, $L = \mathbb Q$, степень --- 1.
    \item $x^2 - 2$, $L = \mathbb Q(\sqrt 2)$, степень --- 2.
    \item $x^3 - 2$, $L = \mathbb Q(\sqrt[3]{2}, \sqrt[3]{2} \omega, \sqrt[3]{2} \omega^2)$, степень --- 6.
    \item $(x - 1)(x - 2)(x - 3)$, $L = \mathbb Q$, степень --- 1.
\end{itemize}

\textbf{Доказательство.} Индукцией по $\deg(f(x))$.
Разложим $f$ на неприводимые: $f(x) = g_1(x) \cdot \ldots \cdot g_s(x)$.
Положим $L_0 = F$, $L_1 = F[x] / (g_1(x))$.
По утверждению с предыдущей лекции в $L_1$ есть корень $\alpha_1$ многочлена $g_1(x)$.
Тогда $f(x) = (x - \alpha_1) h_1(x)$ над $L_1$.
Применим предположение индукции к $h_1$ и $L_1$, откуда $[L : L_0] = [L : L_1] \cdot [L_1 : L_0] \le n \cdot (n - 1)! = n!$.

\QED

\textbf{Теорема.} (б/д) Поле разложения единственно с точностью до изоморфизма.

\textbf{Рассуждения.}
Пусть $F$ --- конечное поле, $\Char(F) = p$, $[F : \mathbb Z_p] = m$.
Тогда $|F| = p^m$, $|F^*| = p^m - 1$, откуда для всех $a \in F^*$ выполнено $a^{p^m - 1} = 1$, то есть $a$ является корнем многочлена $x^{p^m - 1} - 1$, а значит, все $a \in F$ являются корнями многочлена $x^{p^m} - x$, то есть $F$ является полем разложения многочлена $x^{p^m} - x$ над $\mathbb Z_p$.
По утверждению выше такое поле единственно с точностью до изоморфизма.

\subsection{Алгебраически замкнутые поля}
\textbf{Определение.} Алгебраическое расширение поля $F$ --- расширение, в котором все элементы алгебраические относительно $F$.

\textbf{Определение.} Поле $F$ называется \textit{алгебраически замкнутым}, если выполнено одно из следующих эквивалентных условий:
\begin{enumerate}
    \item Любой многочлен имеет корень.
    \item Любой многочлен линейно факторизуем.
    \item Любое конечное расширение $F$ тривиально.
    \item Любое алгебраическое расширение $F$ тривиально.
    \item Поле разложения любого многочлена совпадает с $F$.
    \item (На лекции не было, но было в листочке) Все неприводимые над $F$ многочлены имеют степень 1.
\end{enumerate}

\textbf{Корректность.}
$1 \Rightarrow 2$. Очевидно.

$2 \Rightarrow 4$. Пусть $L \supset F$ --- алгебраическое расширение, $\alpha \in L$.
Тогда $m_{\alpha, F}(x)$ неприводим, откуда его степень равна единице, то есть $\alpha \in F$.

$4 \Rightarrow 3$. Любое конечное расширение является алгебраическим.

$3 \Rightarrow 5$. Поле разложения конечно по утверждению выше, тривиально по условию.

$5 \Rightarrow 1$. Рассмотрим многочлен, полем его разложения является $F$, откуда все его корни лежат в $F$.

$1 \iff 6$. Очевидно.

\QED

\textbf{Определение.} Пусть $F$ --- поле.
Тогда $\overline F$ --- это \textit{алгебраическое замыкание} $F$, то есть алгебраическое расширение, которое алгебраически замкнуто.

\textbf{Утверждение.} (И корректность) Пусть $F \subset L \subset K$ --- башня алгебраических разложений, то есть $F \subset L$ и $L \subset K$ алгебраические.
Тогда расширение $F \subset L$ алгебраическое.

\textbf{Доказательство.} Пусть $\alpha \in K$.
\sloppy Из алгебраичности существуют $\beta_0, \dots, \beta_m \in L$, такие что $\beta_0 + \dots + \beta_m \alpha^m = 0$, причём по условию они все алгебраические.
Теперь мы хотим заменить $L$ на поля с конечным расширением.
А именно, $F \subset F(\beta_0, \dots, \beta_m) \subset F(\beta_0, \dots, \beta_m)(\alpha)$ --- башня конечных расширений.
Значит, $\alpha$ лежит в конечном расширении $F$, то есть в алгебраическом.

\QED

\subsection{Построение алгебраического замыкания для счётного поля}
Если $F$ не более, чем счётно, то можно перечислить все неприводимые многочлены $f_1, f_2, \dots$ и строим цепочку $F = F_0 \subset F_1 \subset \dots$, где $F_N$ --- поле разложения $f_N$ над $F_{N-1}$.
Тогда $\overline F = \bigcup_{k=0}^{\infty} F_k$.
То, что это является полем, очевидно, алгебраичность следует из того, что каждое промежуточное расширение конечно, то есть алгебраично, теперь по утверждению каждое из полей алгебраично относительно $F$.

Для более чем счётных полей делается то же самое, но там начинается аксиома выбора.

Другой вариант: можно рассмотреть алгебраически замкнутое $K \supset F$, тогда $\overline F = \{\alpha \in L~|~\text{$\alpha$ --- алгебраическое над $F$}\}$.
Корректность: пусть $\alpha, \beta \in \overline F$, тогда $\alpha, \beta \in K$, а значит, $F(\alpha, \beta) \supset F$ --- конечное расширение.
Значит, это расширение алгебраическое, и $\alpha + \beta, \alpha \cdot \beta, \alpha^{-1} \in \overline F$.
Алгебраичность аналогично доказательству выше.

\textbf{Теорема.} (б/д) $\overline F$ единственно с точностью до изоморфизма.

\subsection{Построение алгебраического замыкания в общем случае}
Пусть $M$ --- множество многочленов положительной степени над $F$.
Введём для каждого $f \in M$ формальную переменную $x_f$.
Пусть $I \subset F[x_f]$ --- идеал, порождённый всеми многочленами вида $f(x_f)$.
Заметим, что $I$ нетривиален, так как в нём не может быть единицы, ибо у всех многочленов положительная степень.
Расширим его до максимального идеала, тогда $F[x_f] / I$ будет полем.

Рассмотрим $f \in M$ над полем $F[x_f] / I$.
Заметим, что у него есть корень $\overline{x_f}$, ибо $f(x_f) \in I$.
Следовательно, все многочлены из $F[x]$ имеют корень над этим полем.
Положим $F_1 = F[x_f] / I$, $F_2$ --- сделаем то же самое, но над $F$, и так далее.
Получается цепочка $F_1 \subset F_2 \subset \dots$, пусть $K$ --- её объединение.

Тогда над $K$ любой многочлен имеет корень (пусть мы его добавили на $i$-ом шаге, тогда на $(i + 1)$-ом шаге мы добавили его корень) и $F \subset K$, откуда, по определению, $K = \overline F$.

\subsection{Теорема о примитивном элементе}
Если $\Char(F) = 0$ и $F \subset L$ конечно, то существует $\alpha \in L$, такое что $L = F(\alpha)$.

\textbf{Доказательство.} Понятно, что $L = F(\alpha_1, \dots, \alpha_s)$.
По индукции достаточно доказать, что $F(\alpha, \beta) = F(\gamma)$.
Будем искать $\gamma$ в виде $\alpha + c \cdot \beta$.
Так как $F(\gamma) \subset F(\alpha, \beta)$, достаточно доказать обратное включение.
А именно, хочется сделать так, чтобы $\beta \in F(\gamma)$.
Во-первых, $\beta$ --- корень $m_{\beta, F}(x)$.
Во-вторых, $\beta$ --- корень $m_{\alpha, F}(\gamma - cx) \in F(\gamma)[x]$.

Теперь мы хотим доказать, что НОД многочленов $m_{\beta, F}(x)$ и $m_{\alpha, F}(\gamma - cx))$ имеет степень 1, то есть $\beta$ --- их единственный общий корень.
Пусть $\beta_1, \dots, \beta_N$ ($\beta_1 = \beta$) --- корни $m_{\beta, F}(x)$ и $\alpha_1, \dots, \alpha_s$ ($\alpha_1 = \alpha$) --- корни $m_{\alpha, F}(x)$.
Сейчас мы рассматриваем многочлены над $\overline F$, так что это все возможные корни.
Рассмотрим какой-то корень $\beta_j$ многочлена $m_{\alpha, F}(\gamma - cx)$.
Допустим, что нашёлся ещё один общий корень $\gamma - c\beta_j = \alpha_k$.
Тогда
\[
    \gamma = \alpha_k + c\beta_j = \alpha + c\beta \Rightarrow c = \frac{\alpha_k - \alpha}{\beta - \beta_j}.
\]
Можно взять $c$ так, чтобы это равенство не случилось, ибо мы выкинули конечное число элементов (в определении $\beta_i$ и $\alpha_j$ элемент $c$ не фигурировал).
Следовательно, $\beta$ --- единственный общий корень, однако есть проблема: кратные корни в $m_{\beta, F}(x)$ (если они есть, то произойдёт деление на ноль).
Но в этом случае можно взять НОД с производной (тогда не единица) и получить противоречие с неприводимостью.

Таким образом, кратных корней нет и единственный общий корень --- это $\beta$, то есть НОД многочленов лежит в $F(\gamma)[x]$ и равен $x - \beta$.
Отсюда $\beta \in F(\gamma)$.

\QED

\subsection{Построение цикрулем и линейкой}
Что значит ``построить точку цикрулем и линейкой``?
Изначально нам даны точки $0$ и $1$ в $\mathbb C$.
Теперь, если у нас есть точки $p$ и $q$, мы можем совершать следующие действия:
\begin{itemize}
    \item Провести между ними прямую.
    \item Провести окружность с центром в $p$, проходящую через $q$.
    \item Отметить пересечение двух окружностей или прямых.
\end{itemize}

Будем обозначать через $M_\mathbb C$ множество построимых точек.
Также положим $M_\mathbb R = M_\mathbb C \cap \mathbb R$.

\textbf{Примеры.}
\begin{itemize}
    \item Можно построить медиану точек.
    \item Можно найти перпендикуляр из точки на прямую.
    \item А оттуда и параллельные прямые.
\end{itemize}

\textbf{Следствие.} Так как мы можем строить перпендикуляры на оси, то есть проекции, $M_\mathbb C = \{a + bi~|~a, b \in M_\mathbb R\}$.

\textbf{Утверждение.} $M_\mathbb R$ замкнуто относительно сложения, вычитания, умножения, деления и взятия квадратного корня.

\textbf{Доказательство.} Упражнение.
Например, разность: даны $a$ и $b$, можно окружностью и прямой построить $-b$, после этого найти их медиану ($\frac{a - b}{2}$) и удвоить.

\QED

\textbf{Следствие.} $M_\mathbb R$ --- поле, а $M_\mathbb C = M_\mathbb R(i)$.

\textbf{Теорема.} (Критерий принадлежности $M_\mathbb R$) Пусть $\delta \in \mathbb R$.
Тогда $\delta \in M_\mathbb R$ тогда и только тогда, когда найдутся $\delta_1, \dots, \delta_n \in \mathbb R$, такие что для всех $k$ выполняется $[\mathbb Q(\delta_1, \dots, \delta_k) : \mathbb Q(\delta_1, \dots, \delta_{k-1})] = 2$ и $\delta \in \mathbb Q(\delta_1, \dots, \delta_n)$.

\textbf{Доказательство.} $\Leftarrow$. Очевидно, так как при каждом расширении мы получаем корни квадратных многочленов, а брать корень мы умеем.

$\Rightarrow$. Пусть мы на $k$-ом шаге строили точку $(x_1, y_1)$.
Будем строить поля $F_k = \mathbb Q(x_1, y_1, x_2, \dots, x_k, y_k)$.
Тогда для всех $k$ выполнено $[F_{k+1} : F_k] \in \{1, 2\}$.
Докажем разбором случаев.
\begin{itemize}
    \item $p_{k+1} = (x_{k+1}, y_{k+1})$ появилась, как пересечение двух прямых.
        Тогда она выражается через точки на этих двух прямых с коэффициентами из поля, так что поле не изменилось.
    \item $p_{k+1}$ является пересечением окружности и прямой, то есть, без ограничения общности,
        \[
            \begin{cases}
                y_{k+1} = ax_{k+1} + b \\
                x_{k+1}^2 + y_{k+1}^2 + cx_{k+1} + dy_{k+1} + e = 0
            \end{cases}
        \]
        В этом случае координаты находятся решением квадратного уравнения, откуда $F_{k+1}$ можно получить добавлением в поле корня из дискриминанта.
    \item Получилось пересечением окружностей, аналогично, или можно свести к предыдущему случаю радиальной осью.
\end{itemize}

\QED

\textbf{Следствие.} (Критерий принадлежности $M_\mathbb C$)
$z \in M_\mathbb C$ тогда и только тогда, когда найдутся $z_1, \dots, z_n$, такие что для всех $k$ выполнено $[\mathbb Q(z_1, \dots, z_{k+1}) : \mathbb Q(z_1, \dots, z_k)] = 2$ и $z \in \mathbb Q(z_1, \dots, z_k)$.

\textbf{Следствие.} Если $z \in M_\mathbb C$, то $[\mathbb Q(z) : \mathbb Q] = 2^k$ для $k \in \mathbb N$.
Иными словами, мы умеем строить только очень специфичные алгебраические числа.
В частности, $\sqrt \pi \not\in M_\mathbb C$, так что квадратура круга не построима.
Также $\sqrt[3]{2} \not\in M_\mathbb C$, так что удвоение куба не построить.
Можно попытаться построить правильные $n$-угольники, это эквивалентно построению корней $n$-ой степени из единицы.
Например, восемнадцатиугольник не построим, так как $x^9 + 1$ --- один из многочленов, обнуляющих $\xi_{18}$, его можно разложить в $(x^3 + 1)(x^6 - x^3 + 1)$, и второй многочлен неприводим, ибо можно сдвинуть на единицу и применить признак Эйзенштейна.
В частности, если $n$ простое, то степень минимального многочлена $\xi_n$ равна $n - 1$, откуда

\textbf{Следствие.} Если правильный $p$-угольник построим, то $p = 2^k + 1$.

\textbf{Определение.} Многочлен деления круга --- это
\[
    \Phi_n(x) = \prod_{\substack{1 \le m \le n \\ (m, n) = 1}} (x - \xi_n^m),
\]
где $\xi_n$ --- примитивный корень $n$-ой степени из единицы.

\textbf{Утверждение.} (Доказательство позже) Коэффициенты $\Phi_n$ целые, и он неприводим над $\mathbb Q$.

\textbf{Следствие.} Если правильный $n$-угольник построим, до $\phi(n) = 2^k$.

\textbf{Итог.} Мораль в том, что построение циркулем и линейкой --- это расширение поля $\mathbb Q$ корнями квадратных многочленов.

\subsection{Автоморфизмы расширений}
\textbf{Определение.} Пусть $K \subset F$ --- поле.
\textit{Группа автоморфизмов расширения} $K \subset F$ --- это
\[
    \Aut_K(F) = \{\psi \in \Aut(F)~|~\phi|_K = id\},
\]
то есть автоморфизмы, которые не двигают подполе.

\textbf{Пример.} $\Aut_\mathbb Q (\mathbb Q(\sqrt 2))$.
Заметим, что если $\psi$ подходит, то $\psi(a + b \sqrt 2) = \psi(a) + \psi(b \sqrt 2) = a + b \cdot \psi(\sqrt 2)$.
Куда может переходить $\sqrt 2$? Заметим, что $\sqrt 2$ является корнем $x^2 - 2$, а значит, если применить к многочлену $\psi$, то $\psi(\sqrt 2)$ будет корнем $x^2 - \psi(2) = 0$, то есть $\psi(\sqrt 2) = \pm \sqrt 2$.

\textbf{Определение.} Пусть $K \subset F$ --- расширение, $a, b \in \overline K$.
Тогда $a$ и $b$ называются \textit{сопряжёнными}, если их минимальные многочлены над $K$ совпадают.

\textbf{Утверждение.} (О мощности $\Aut_F(F(\gamma))$) Пусть $\gamma$ --- алгебраический элемент над $F$.
Тогда $|\Aut_F(F(\gamma))|$ равно количеству сопряжённых к $\gamma$ элементов из $F(\gamma)$, где $\alpha \sim \beta$ сопряжены над $F$, если $m_{\alpha, F} = m_{\beta, F} \iff m_{\alpha, F}(\beta) = 0$.

\textbf{Доказательство.} Будем строить биекцию.
Пусть $\phi \in \Aut_F(F(\gamma))$, тогда $m_{\gamma, F}(\phi(\gamma)) = \phi(m_{\gamma, F}(\gamma)) = \phi(0) = 0$, то есть $\phi(\gamma) \sim \gamma$.

Обрано, пусть $\gamma \sim \gamma'$, тогда $m_{\gamma} = m_{\gamma'}$, то есть $F(\gamma) \cong F[x] / (m_\gamma) \cong F[x] / (m_{\gamma'}) \cong F(\gamma')$.

\QED

\textbf{Следствие.} $|\Aut_F(F(\gamma))| \le \deg(m_{\gamma, F})$.

\textbf{Пример 2.} $\Aut_{\mathbb Q}(\mathbb Q(\sqrt[3]{2})) = \{id\}$.
Следует напрямую из утверждения.

\subsection{Сепарабельные расширения}
Пусть $f(x) \in K[x]$ и $f$ неприводим над $K$.
Разложим его над алгебраическим замыканием: $f(x) = \prod_{i=1}^n (x - x_i)$.
Верно ли, что все $x_i$ различны?
Это будет верно тогда и только тогда, когда $\Char(K) = 0$ или $\Char(K) = p$ и можно брать корень $p$-ой степени, то есть отображение $a \mapsto a^p$ сюръективно.

\textbf{Определение.} Расширение $K \supset F$ называется \textit{сепарабельным}, если для любого $\alpha \in K$ многочлен $m_{\alpha, F}$ не имеет кратных корней.

\textbf{Замечание.} Не сепарабельные расширения встреачаются крайне редко, ибо нам нужно бесконечное поле с конечной характеристикой.
Пример: $\mathbb Z_p(x)$, добавим туда корень многочлена $x^p - a$ для $a \in \mathbb Z_p$, то есть $\sqrt[p]{x}$.
Тогда $x^p - a = (x - \sqrt[p]{a})^p$.

\subsection{Расширения Галуа}
\textbf{Теорема.} Пусть $F \subset K$ --- конечное расширение и ($\Char(F) = 0$ или $K$ конечное).
Иными словами, выполнена теорема о примитивном элементе.
Следующие условия эквивалентны:
\begin{enumerate}
    \item $K$ --- поле разложения некоторого $f(x) \in F[x]$.
    \item Если $\gamma \in K$, то все сопряжённые с $\gamma$ над $F$ элементы лежат в $K$.
    \item $|\Aut_F(K)| = [K:F]$.
    \item $K^{\Aut_F(K)} = F$ (множество неподвижных точек при действии $\Aut_F(K)$).
\end{enumerate}
Первые два условия называются \textit{нормальным расширением}, а последние два --- \textit{расширением Галуа}, то есть нормальное и сепарабельное.

\textbf{Доказательство.} $2 \Rightarrow 3$. Имеем $K = F(\gamma)$ по теореме о примитивном элементе, тогда $|\Aut_F(F(\gamma))|$ равно количеству сопряжённых к $\gamma$ в $F(\gamma)$ по утверждению о мощности, а по условию это равно $\deg(m_\gamma)$, что в точности равно $[F(\gamma) : F]$.

$3 \Rightarrow 4$. Пусть $K^{\Aut_F(K)} = L \supset F$, тогда $\Aut_F(K) = \Aut_L(K)$ по построению $L$.
Теперь по условию $[K : F] = |\Aut_F(K)|$, далее $|\Aut_F(K)| = |\Aut_L(K)| \le [K : L]$, так как по утверждению о мощности $|\Aut_L(K)| \le \deg(m_{\gamma, L}) = [K : L]$, где $K = L(\gamma)$.
Наконец, $[K : L] \le [K : F]$, поэтому все неравенства обращаются в равенства.

$4 \Rightarrow 2$. Пусть $\gamma \sim \gamma'$, тогда $\gamma' \in K$.
Рассмотрим
\[
    f(x) = \prod_{g \in \Aut_F(K)} (x - g(\gamma)).
\]
Утверждается, что $f(x) \in F[x]$.
Действительно, если мы подействуем на $f(x)$ любым автоморфизмом $h \in \Aut_F(K)$, то многочлен не изменится, откуда по условию $f(x) \in F[x]$.
Более того, один из автоморфизмов в произведении из определения $f$ --- тождественный, что даёт $f(\gamma) = 0$.
Следовательно, $f$ делится на $m_{\gamma}(x)$. Теперь все корни $m_{\gamma}$ являются корнями $f(x)$, а они лежат в $K$ (потому что область значений всех $g \in \Aut_F(K)$ совпадает с $K$).
Следовательно, если $\gamma' \sim \gamma$, то он является корнем $m_{\gamma}$ и лежит в $K$.

$2 \Rightarrow 1$. $m_{\gamma, F}$ линейно факторизуем над $K = F(\gamma)$, ибо все корни там лежат, а минимальность очевидна.

$1 \Rightarrow 2$. Зафиксируем $f(x) \in F[x]$.
Пусть $\gamma \in K$ и $\tilde{\gamma} \in \overline F$ сопряжено.
У нас есть изоморфизм $\phi: F(\gamma) \to F(\tilde \gamma)$, сохраняющий $F$ и переводящий $\gamma$ в $\tilde \gamma$ (задача с семинара).
Мы хотим построить гомоморфизм $\psi: K \to \overline F$, продолжающий $\phi$.

\textbf{Утверждение.} (О продолжении гомоморфизма) Пусть $F$ --- поле, $L \supset F$ --- конечное расширение, $\phi: L \to \overline F$ --- гомоморфизм, сохраняющий $F$, и $\alpha$ --- алгебраический над $L$.
Если $\beta$ --- корень $\phi(m_{\alpha, L})$, то существует единственный гомоморфизм $\tilde \phi: L(\alpha) \to \overline F$, такой что $\tilde \phi|_L = \phi$ и $\tilde \phi(\alpha) = \beta$.

\textbf{Пример.} Рассмотрим башню $\mathbb Q(\sqrt[3]{2}) \subset \mathbb Q(\sqrt[3]{2} \omega) \subset \overline {\mathbb Q}$.
Гомоморфизм $\phi: \mathbb Q(\sqrt[3]{2}) \to \mathbb Q(\sqrt[3]{2} \omega)$ можно продолжить до $\mathbb Q(\sqrt[3]{2}, \omega) \to \mathbb Q(\sqrt[3]{2}, \omega)$.

\textbf{Доказательство.} Положим $\tilde L = \phi(L)$, тогда $\phi: L \to \tilde L$ --- изоморфизм.
Тогда $\phi: L[x] \to \tilde L[x]$ --- изоморфизм, так что $L(\alpha) \cong L[x] / (m_{\gamma, L}) \cong \tilde L[x] / (\phi(m_{\alpha, L})) \cong \tilde L(\beta)$.

\QED

Возьмём продолжение изоморфизма $\tilde \phi: K \to \overline F$, оно существует по утверждению и теореме о примитивном элементе.
Пусть $f(x) = (x - \gamma)(x - \gamma_1) \dots (x - \gamma_s)$, по условию $\gamma, \gamma_1, \dots \gamma_s \in K$.
Заметим, что $f(x) = \tilde \phi(f(x)) = (x - \gamma')(x - \tilde \phi(\gamma_1)) \dots (x - \tilde \phi(\gamma_s))$.
Следовательно, $\gamma', \tilde \phi(\gamma_1), \dots, \tilde \phi(\gamma_s) \in K$, и, в частности, $\gamma' \in K$.

\QED

\textbf{Лемма.} (О собственной подгруппе группы Галуа) Если $K \supset F$ --- расширение Галуа и $H \subsetneq \Aut_F(K)$, то $K^H \ne F$.

\textbf{Доказательство.} От противного: $K^H = F$. По теореме о примитивном элементе $K = F(\gamma)$, теперь рассмотрим $f(x) = \prod_{h \in H}(x - h(\gamma))$.
Теперь, как и в $(4 \Rightarrow 2)$ основной теоремы Галуа, $f(x) \in K^H[x] = F[x]$.
Так как $f(\gamma) = 0$, $\deg(f(x)) \ge \deg(m_\gamma)$.
Наконец, $|H| = \deg(f(x))$, а $|\Aut_F(K)| = |\Aut_F(F(\gamma))| \le \deg(m_\gamma)$ по лемме о мощности $\Aut_F(F(\gamma))$.
Собирая неравенства вместе, получаем $|H| \ge |\Aut_F(K)|$ --- противоречие с $H \ne \Aut_K(F)$.

\QED

\textbf{Лемма.} (О башне нормальных расширений) Пусть $K \supset L \supset F$ --- башня расширений, причём $K \supset F$ нормальное.
Тогда $K \supset L$ тоже нормальное.

\textbf{Доказательство.} Действительно, по первому условию $K$ является полем разложения некоторого $f(x) \in F[x] \subset L[x]$.

\QED

\textbf{Теорема.} (Основная теорема теории Галуа) Пусть $K \supset F$ --- расширение Галуа.
Тогда существует биекция между конечными расширениями и группой Галуа $\Aut_F(K)$, а именно, подгруппе $H \subset \Aut_F(K)$ сопоставляется расширение $F \subset K^H$, и расширению $F \subset L$ сопоставляется подгруппа $\Aut_L(K)$.
Причём $H \triangleleft \Aut_F(K)$ тогда и только тогда, когда $L \supset F$ нормальное (где $H$ и $L$ сопоставлены биекцией).
Более того, если $L \mapsto H$, то $|H| = [K : L]$ и $[\Aut_F(K) : H] = [L : F]$.

\begin{figure}[ht]
    \centering
    \incfig{galois}{0.9\linewidth}
    \caption{Пример биекции из основной теоремы Галуа}
\end{figure}

\textbf{Доказательство.} Мы хотим доказать три вещи. Первая, если $F \subset L$ --- конечное расширение, то $K^{\Aut_L(K)} = L$.
Вторая, если $H \subset \Aut_F(K)$, то $\Aut_{K^H}(K) = H$.
Третья, про нормальность.

Первое совсем очевидно: по лемме о башне нормальных расширений $L \subset K$ является расширением Галуа, после чего четвёртое условие даёт искомое.

Второе: заметим, что $K^H = K^{\Aut_{K^H}(K)}$ и $H \subset \Aut_{K^H}(K)$.
По лемме о башне нормальных расширений $K^H \subset K$ является расширением Галуа, откуда по контрапозиции с леммой о собственной подгруппе Галуа $H = \Aut_{K^H}(K)$.

Третье: проверим, что $K^{g H g^{-1}} = g \circ K^H$. Это просто факт из теории групп, и он проверяется в лоб:
\[
    K^{gHg^{-1}} = \{x \in K~|~\forall h \in H~ghg^{-1}(x) = x\} = 
\]
\[
    = \{x \in K~|~\forall h \in H~hg^{-1}(x) = g^{-1}(x)\} =
\]
\[
    = \{x \in K~|~g^{-1}(x) \in K^H\} = \{x \in K~|~x \in g \circ K^H\} = g \circ K^H.
\]
Теперь заметим, что нормальность $K^H$ над $F$ это то же самое, что для всех $g \in \Aut_F(K)$ выполняется $g \circ K^H = K^H$ (доказано ниже).
Последнее эквивалентно тому, что для всех $g \in \Aut_F(K)$ выполнено $K^{gHg^{-1}} = K^H$.
По первым двум пунктам у нас уже есть биекция, так что это эквивалентно тому, что $gHg^{-1} = H$, а это уже нормальность $H$ в $\Aut_{F}(K)$.

Обратно к утверждению: $F \subset K^H$ нормальное тогда и только тогда, когда для всех $g \in \Aut_F(K)$ выполнено $g \circ K^H = K^H$.

$\Rightarrow$: рассмотрим $\gamma \in K^H$, по нормальности все сопряжённые тоже лежат в $K^H$.
Пусть $g \in \Aut_F(K)$, тогда $g(m_\gamma(x)) = m_\gamma(x)$.
Раскладывая $m_\gamma$ на множители, получаем, что $\gamma$ переходит в сопряжённый ему, то есть $g(\gamma) \in K^H$.

$\Leftarrow$: пусть $\gamma \sim \gamma'$, тогда найдётся изоморфизм $g: F(\gamma) \to F(\gamma')$, такой что $g(\gamma) = \gamma'$.
По теореме о продолжении гомоморфизма его можно продолжить до $K$, а значит, $g \circ K^H = K^H$.
Следовательно, $g(\gamma) \in K^H$.

\QED

\textbf{Замечание.} До этого мы говорили, что башни расширений хорошие: башня конечных конечная, башня алгебраических алгебраическая.
Однако если в башне $K \supset L \supset F$ расширения $K \supset L$ и $L \supset F$ нормальные, то $K \supset F$ не обязательно нормальное, здесь критерием является то, что $\Aut_L(K) \triangleleft \Aut_F(K)$.

Пример: $\mathbb Q(\sqrt[4]{2}) \supset \mathbb Q(\sqrt 2) \supset \mathbb Q$.
Каждый этаж расширения нормальный, так как степень равна единице.
Но всё расширение --- нет, так как вместе с $\sqrt[4]{2}$ нужны ещё сопряжённые $\pm \sqrt[4]{2} i$.

\subsection{Следствия из основной теоремы Галуа}
\textbf{Определение.} Пусть $f(x) \in F[x]$. $\Gal_F(f) = \Aut_F(K)$ --- это группа Галуа поля разложения $K$ многочлена $f(x)$.

\textbf{Факт из теории групп.} Если $|G| = 2^k$, то найдётся цепочка $G = G_0 \supset G_1 \supset \dots \supset G_k = \{e\}$, такая что все $[G_i : G_{i+1}] = 2$.
Для доказательства берётся центр группы, фактор по нему и индукция.

\textbf{Теорема.} (Основная теорема алгебры) $\mathbb C$ алгебраически замкнуто.

\textbf{Доказательство.} Пусть $L \supset \mathbb R$, $[L : \mathbb R] = 2k + 1$ --- конечное расширение нечётной степени.
По теореме о примитивном элементе $L = \mathbb R(\gamma)$, откуда $\deg(m_\gamma) = 2k + 1$.
Как известно, многочлен нечётной степени над $\mathbb R$ всегда имеет корень, откуда $\deg(m_\gamma) = 1$, то есть $L = \mathbb R$ (здесь просто какая-нибудь непрерывность).

Второй известный факт: если $L \supset \mathbb C$, такое что $[L : \mathbb C] \le 2$, то $L = \mathbb C$.

Теперь пусть $K \supset \mathbb C$ --- какое-то конечное расширение, продолжим его до нормального над $\mathbb R$ (если $K = \mathbb R(\gamma)$, то возьмём $K$ --- поле разложения $m_\gamma$ над $\mathbb R$).
Пусть $|\Gal_\mathbb R(K)| = 2^k \cdot M$, где $M$ нечётно.
По теоремам Силова найдётся силовская подгруппа $H \subset \Gal_\mathbb R(K)$, такая что $|H| = 2^k$.
Пусть ей соответствует расширение $K \supset L \supset \mathbb R$, тогда $[L : \mathbb R] = M$.
По первому факту $M = 1$.
Следовательно, $|\Aut_\mathbb R(K)| = 2^k$ и $|\Aut_\mathbb C(K)| = 2^{k-1}$.

Если $k = 1$, то победа, иначе рассмотрим подгруппу $\tilde H \subset \Aut_\mathbb C(K)$ индекса 2, то есть $|\tilde H| = 2^{k-2}$.
Пусть ей сопоставлено расширение $K \supset L \supset \mathbb C$.
По основной теореме Галуа $[L : \mathbb C] = [\Aut_\mathbb C(K) : \tilde H] = 2$, откуда по второму известному факту $L = \mathbb C$, то есть $\tilde H = \Aut_\mathbb C(K)$ --- противоречие.

Следовательно, $K = \mathbb C$, что эквивалентно алгебраической замкнутости.

\QED

\textbf{Утверждение.} $\alpha$ построимо циркулем и линейкой тогда и только тогда, когда $|\Gal(m_\alpha)| = 2^k$.

\textbf{Доказательство.} $\Leftarrow$. По факту из теории групп найдётся цепочка $\Gal(m_\gamma) = G_0 \supset G_1 \supset \dots \supset G_n = \{e\}$.
По основной теореме Галуа им можно сопоставить цепочку $K = K_n \supset \dots \supset K_0 = \mathbb Q$.
Теперь все $[K_i : K_{i-1}] = [G_i : G_{i-1}] = 2$, что эквивалентно построимости.

$\Rightarrow$. Остаётся в качестве вопроса на отл.(10).

\QED

\textbf{Пример.} Для примитивного корня $n$-ой степени из единицы $\xi_n$ всё довольно просто, ибо $\mathbb Q(\xi_n)$ является полем разложения $m_{\xi_n}$, то есть $[\mathbb Q(\xi_n) : \mathbb Q] = |\Gal(\xi_n)|$.
Засим правильный $n$-угольник построим $\iff$ $\xi_n$ построим $\iff$ $\deg(m_{\xi_n}) = 2^k$ $\iff$ $\phi(n) = 2^k$.
Откуда взялось $\phi$?

\textbf{Теорема.} $\deg(m_{\xi_n}) = \phi(n)$.

\textbf{Доказательство.} Для начала вспомним, как выглядит $m_{\xi_n}$, а именно, выпишем первые несколько многочленов: $x - 1$, $x + 1$, $x^2 + x + 1$, $x^2 + 1$, $x^4 + x^3 + x^2 + x + 1$, $x^2 - x + 1$.
Может показаться, что коэффициенты всегда равны $\pm 1$, но уже для $m_{\xi_{105}}$ это неверно.

Положим $\Phi(x) = \prod_{(k, n) = 1} (x - \phi_n^k)$.
Сначала докажем, что $\Phi(x) \in \mathbb Z[x]$, по индукции.
Заметим, что $x^n - 1 = \Phi(x) \cdot \prod_{d|n, d \ne n} m_{\xi_d}(x)$, доказано.

Если $\alpha$ --- корень $m_{\xi_n}(x)$, то $\alpha^p$ тоже является его корнем для всех простых $p$, таких что $(p, n) = 1$.
От противного: $x^n - 1 = m_{\xi_n}(x) \cdot g(x)$, такие что $\alpha$ является корнем $m_{\xi_n}(x)$, а $\alpha^p$ является корнем $g(x)$.
Тогда $\alpha$ является корнем $g(x^p)$, а отсюда $g(x^p) = m_{\xi_n}(x) \cdot h(x)$.
Беря по модулю $p$, получаем $(\overline x)^n - 1 = \overline m_{\xi_n}(x) \cdot \overline g(x)$.
Но $\overline g(x^p) = (\overline g(x))^p = \overline m_{\xi_n}(x) \cdot \overline h(x)$.
Отсюда получается, что $m_{\xi_n}(x)~|~(\overline g(x))^p$, то есть у $\overline m_{\xi_n}(x)$ и $\overline g(x)$ есть общий корень.
Так как $(\overline x)^n - 1$ делится на их произведение, у него есть кратный корень --- противоречие, ибо их нет.

Наконец, если $\alpha$ --- корень $m_{\xi_n}(x)$, то $\alpha^k$ является корнем $m_{\xi_n}(x)$ для $k$, взаимно простого с $n$.
Пусть $k = p_1 \dots p_s$, тогда $\alpha$, $\alpha^{p_1}$, $\dots$, $\alpha_{p_1 \dots p_s} = \alpha_k$ являются корнями.

Отсюда получаем, что $\deg(m_{\xi_n}) \ge \phi(n)$.
Итак, $\Phi(x)$ --- это неприводимый многочлен степени $\phi(n)$ (доказательство не случилось), корнем которого является $\xi_n$.
Следовательно, он является минимальным.

\QED

\textbf{Утверждение.} Если $\deg(f(x)) = n$, то $\Gal(f) \subset S_n$.
То есть любой автоморфизм переставляет корни местами.

\textbf{Утверждение.} Если $f(x)$ неприводим, то $\Gal(f) \subset S_n$ транзитивна, то есть для любых $i, j$ найдётся $\sigma \in \Gal(f)$, такая что $\sigma(i) = j$.
Очевидно, так как мы можем любой корень перевести в любой другой и продолжить изоморфизм до $K$.

\section{Симметрические многочлены}
Пусть $K$ --- область целостности.
Рассмотрим многочлены в $K[x_1, \dots, x_n]$, которые сохраняются при перестановке переменных, они называются \textit{симметрическими}.
Иными словами, для любой $\sigma \in S_n$ выполнено $f(x_1, \dots, x_n) = f(x_{\sigma(1)}, \dots, x_{\sigma(n)})$.

\textbf{Определение.} Элементарные симметрические многочлены:
\[
    \sigma_1 = x_1 + \dots + x_n
\]
\[
    \sigma_2 = x_1x_2 + x_1x_3 + \dots + x_{n-1}x_n
\]
\[
    \vdots
\]
\[
    \sigma_k = \sum_{\{i_1, \dots, i_k\} \subset \{1, \dots, n\}} x_{i_1} \cdot \ldots \cdot x_{i_k}.
\]

\textbf{Теорема.} (Множество неподвижных точек относительно действия группой $S_n$ --- это) $K[x_1, \dots, x_n]^{S_n} = K[\sigma_1, \dots, \sigma_n]$.
Иными словами, любой симметрический многочлен единственным образом представляется в виде многочлена от $\sigma_1, \dots, \sigma_n$.
Например, $x_1^2 + \dots + x_n^2 = \sigma_1^2 - 2\sigma_2$ и $(x_1 - x_2)^2 = \sigma_1^2 - 4\sigma_2$.
Доказательство позже.

Введём лексикографический порядок на мономах:
\[
    x_1^{a_1} \dots x_n^{a_n} \ge x_1^{b_1} \dots x_n^{b_n} \iff (a_1, \dots, a_n) \ge_{lex} (b_1, \dots, b_n).
\]
Будем называть \textit{старшим мономом} наибольший в таком порядке моном.

\textbf{Лемма.} (О старшем мономе) Если $f$ симметрический, то старший моном имеет вид $x_1^{a_1} \dots x_n^{a_n}$, где $a_1 \ge \dots \ge a_n$.

\textbf{Доказательство.} От противного, пусть он имеет такой же вид, но найдутся $a_i < a_j$ для $i < j$.
Применяя к многочлену транспозицию $(i, j)$, получаем его же, но теперь старший моном стал строго больше --- противоречие.

\QED

\textbf{Лемма 1.} Для любых $a_1 \ge \dots \ge a_n$ найдётся многочлен $g(x_1, \dots, x_n)$, такой что старший моном $g(\sigma_1, \dots, \sigma_n)$ равен $x_1^{a_1} \dots x_n^{a_n}$.

\textbf{Доказательство.} Явно построим $g$. Старший моном должен получаться из произведения вида $(x_1 + \dots + x_n)^{b_1} (x_1x_2 + \dots)^{b_2} \dots (x_1 \dots x_n)^{b_n}$.
Теперь заметим, что по лемме о старшем мономе из этого произведения старший моном получается взятием как можно большей степени $x_1$, потом $x_2$ и так далее.
Как максимизировать степень при $x_1$? Выбрать из всех скобок слагаемое с $x_1$.
Теперь $x_2$: первая скобка уже зафиксирована, но в остальных можно выбрать слагаемое с $x_2$, и так далее.
Получается
\[
    x_1^{b_1 + \dots + b_n} x_2^{b_2 + \dots + b_n} \dots x_n^{b_n}.
\]
Коэффициенты $b_1, \dots, b_n$ восстанавливаются однозначно.

\QED

\textbf{Доказательство теоремы.} Индукцией по старшему моному.
База: если $f = 0$, то $g = 0$.

Переход: пусть старший моном $f$ имеет вид $A \cdot x_1^{a_1} \dots x_n^{a_n}$.
По лемме 1 подбираем многочлен $g$ с таким старшим мономом и вычитаем его.
Старший моном станет строго меньше, дальше по предположению индукции.

Докажем единственность от противного: пусть $g_1(\sigma_1, \dots, \sigma_n) = g_2(\sigma_1, \dots, \sigma_n)$, рассмотрим их разность $h(\sigma_1, \dots, \sigma_n)$ --- сей многочлен не является тождественным нулём, но если раскрыть определения $\sigma_1, \dots, \sigma_n$, то тождественный ноль всё-таки получится.
Найдём среди мононов многочлена $h$ моном вида $\sigma_1^{b_1}, \dots, \sigma_n^{b_n}$, у которого $b_1 + \dots + b_n$ максимально, среди таких --- у которого $b_2 + \dots + b_n$ максимально, и так далее.
Раскрывая скобки, мы получим, среди прочего, моном $B \cdot x_1^{b_1 + \dots + b_n} x_2^{b_2 + \dots + b_n} \dots x_n^{b_n}$.
Если бы он с чем-то сократился, то получится противоречие с тем, как мы взяли $\sigma_1^{b_1} \dots \sigma_n^{b_n}$.

\QED

\textbf{Замечание.} Может показаться плохим, что в переходе индукции у нас может быть бесконечно много многочленов с меньшим старшим мономом.
Однако в силу фундированности множества старших мононов это не проблема.

\subsection{Решение уравнений второй степени}
Только в случае, когда у нас поле $F$ и $\Char(F) \ne 2, 3$.
Пусть у нас есть уравнение $x^2 + ax + b = 0$.
Тогда у него имеется два корня $\alpha_1, \alpha_2$.
Мы знаем, что $\alpha_1 + \alpha_2 = -a \in F$ и $\alpha_1 \alpha_2 = b \in F$.
Рассмотрим дискриминант $(\alpha_1 - \alpha_2)^2 = a^2 - 4b \in F$.

Если $\alpha_1 - \alpha_2 \in F$, то и $\alpha_1 \alpha_2 \in F$, так что оба корня лежат в $F$ и легко находятся.
В противном случае корни лежат в расширении $F(\sqrt D)$, и здесь группа Галуа изоморфна $S_2$.

\subsection{Решение уравнений третьей степени}
Пусть у нас есть уравнение $x^3 + ax + b = 0$ и $\alpha_1, \alpha_2, \alpha_3$ --- его корни.
(Коэффициент при $x^2$ можно убрать подходящей заменой)
Пусть
\[
    \begin{cases}
        \alpha_1 + \alpha_2 + \alpha_3 = \beta_1 \in F \\
        \alpha_1 + \omega \alpha_2 + \omega^2 \alpha_3 = \beta_2 \\
        \alpha_1 + \omega^2 \alpha_2 + \omega \alpha_3 = \beta_3
    \end{cases}
    .
\]
Если мы найдём $\beta_1, \beta_2, \beta_3$, то мы найдём корни, решив вышеуказанную систему.
Применим $S_3$ к $\alpha_1, \alpha_2, \alpha_3$, после чего $\beta_2$ будет равно одному из $\{\beta_2, \omega \beta_2, \omega^2 \beta_2, \beta_3, \omega \beta_3, \omega^2 \beta_3\}$, так как нужно внимательно посмотреть на то, как меняются последние два уравнения.
Грустно, так как слишком много вариантов.
Возведём в куб, тогда останется только $\beta_2^3$ и $\beta_3^3$.
Заметим, что при любых перестановках значения $\beta_2^3 + \beta_3^3 \in F$ и $\beta_2^3 \beta_3^3 \in F$ не меняются, то есть получились симметрические многочлены от $\alpha_1, \alpha_2, \alpha_3$.
Теперь их можно найти (а-ля по теореме Виета), как корни многочлена второй степени и решить исходное уравнение.

Группа Галуа здесь --- $S_3$ или $A_3$.
Также если $f$ неприводим, то $\Gal(f)$ транзитивна (как доказывалось в прошлом параграфе).

\textbf{Утверждение.} Для уравнений степени $n$: $\sqrt D \in F \iff \Gal(f) \subset A_n$, где $D = \prod_{i \ne j} (\alpha_i - \alpha_j)^2$ --- дискриминант.

\textbf{Доказательство.} Запишем эквивалентные утверждения:
\[
    \sqrt D \in F
\]
\[
    \forall \sigma \in \Gal(f)~\sigma(\sqrt D) = \sqrt D
\]
\[
    \forall \sigma \in \Gal(f)~\sigma \in A_n
\]
\[
    \Gal(f) \subset A_n
\]

\QED

В частности, для $n = 3$ получится $\Gal(f) = A_3$.

\subsection{Решение уравнений четвёртой степени}
Пусть у нас есть уравнение $x^4 + ax^2 + bx + c = 0$, опять же обозначим корни за $\alpha_i$.
Мы вновь хотим найти три переменные $\beta_i$, такие что многочлен $r_3(f) = (x - \beta_1)(x - \beta_2)(x - \beta_3)$ является симметрическим.

Возьмём с потолка 
\[
    \begin{cases}
        \beta_1 = \alpha_1 \alpha_2 + \alpha_3 \alpha_4 \\
        \beta_2 = \alpha_1 \alpha_3 + \alpha_2 \alpha_4 \\
        \beta_3 = \alpha_1 \alpha_4 + \alpha_2 \alpha_3
    \end{cases}
    .
\]
\textbf{Определение.} Многочлен $r_3(f)$ называется \textit{кубической резольвентой}.

Теперь остаётся построить кубическое уравнение с корнями $\beta_1$, $\beta_2$, $\beta_3$, найти все $\alpha_i$, пользуясь тем, что $\alpha_1 + \alpha_2 + \alpha_3 + \alpha_4 = 0$ (коэффициент при $x^3$).

Вся эта магия сработала из-за того, что в $S_4$ есть нормальная подгруппа
\[
    \{e, (12)(34), (13)(24), (14)(23)\},
\]
что делает $S_4$ разрешимой, то есть для бóльших степеней уже не прокатит.

Остаётся построить классификацию группы $\Gal(f)$.
Если $\sqrt D \in F$, то по утверждению выше $\Gal(f)$ --- это $A_4$ или $V_4$.
Иначе $S_4$, $D_4$ или $\mathbb Z / 4 \mathbb Z$.
А именно, варианты $A_4$ и $S_4$ возможны тогда и только тогда, когда $r_3(f)$ неприводим.

\textbf{Утверждение.} $3~|~|\Gal(f)| \iff r_3(f)$ неприводим.

\textbf{Доказательство.} $\Leftarrow$. Если $r_3(f)$ неприводим, то $\Gal(r_3(f)) \subset \Gal(f)$, причём $3~|~|\Gal(r_3(f))|$.

$\Rightarrow$. Рассмотрим $(1~2~3) \in \Gal(f)$.
Тогда $(1~2~3) \beta_1 = \beta_3$, $(1~2~3) \beta_3 = \beta_2$ и $(1~2~3) \beta_3 = \beta_1$.
Следовательно, $\beta_1 \sim \beta_2 \sim \beta_3$, откуда $r_3(f)$ неприводим.

\QED

\section{Разрешимость в радикалах}
Пусть $F$ --- поле, $f(x) \in F[x]$.
Мы хотим понять, разрешим ли $f(x)$ в радикалах.

\textbf{Определение.} $f(x)$ \textit{разрешим в радикалах}, если его корни можно получить операциями `+`, `-`, `$\cdot$`, `/`, `$\sqrt[n]{}$`.
Более формально:
\begin{enumerate}
    \item Поле разложения $f(x)$, обозначим за $L$, удовлетворяет условию
        \[
            L \subset K = K_0 \supset K_1 \supset K_2 \supset \dots \supset K_s = F,
        \]
        где $K_i$ получено из $K_{i+1}$ добавлением корня $x^{n_i} - a_{i + 1}$, где $a_{i+1} \in K_{i+1}$.

    \item Альтернативно, что эквивалентно, можно сказать, что все $K_i \supset K_{i+1}$ являются полями разложения $x^{n_i} - a_i$, где $a_i \in K_i$.
        
    \item Ещё альтернатива: $K \supset F$ --- расширение Галуа и все $K_{i-1} \supset K_i$ --- поля разложения $x^{n_i} - a_i$.
\end{enumerate}

\textbf{Теорема.} (2) и (3) эквивалентны.
Идея в том, чтобы взять второе свойство и рассмотреть поле разложения произведения минимальных многочленов для $a_1, \dots, a_s$, после чего доказать, что оно нормальное.
Детали --- на отл.(10).

Почему это не тривиально: рассмотрим расширения $\mathbb Q \subset L \subset M$, где в $L$ добавили корень $x^2 - 2$, а в $M$ --- корень $x^2 - \sqrt 2$.
Получается, что $M = \mathbb Q(\sqrt[4]{2})$, но это расширение не нормальное, так что нужно следить за сопряжёнными корнями.
Как полагается, далее мы будем пользоваться самым сильным, третьим, определением.

Пусть $\Gal(f) \subset G_s \supset G_{s-1} \supset \dots \supset G_0 = \{e\}$ --- соответствующая цепочке полей разложения цепочка групп Галуа.

\textbf{Лемма.} $K_{i-1} \supset K_i$ --- поле разложения тогда и только тогда, когда $G_{i-1} \triangleleft G_i$.
Доказательства не было.

\textbf{Утверждение.} $\Aut_{K_i}(K_{i-1}) \cong G_i / G_{i-1}$.
Доказательства не было.

\textbf{Утверждение.} Пусть $K \supset L \supset F$, причём все три расширения нормальные.
Тогда $\Aut_F(L) \cong \Aut_F(K) / \Aut_L(K)$.

\textbf{Доказательство.} Положим $\phi: \Aut_F(K) \to \Aut_F(L)$, такой что $\phi(g) = g|_L$.
Тогда это сюръективный гомоморфизм, причём $\Ker(\phi) = \Aut_L(K)$.
По ОТГ получаем искомое.

\QED

\textbf{Теорема.} $f(x)$ разрешим в радикалах тогда и только тогда, когда $\Gal(f)$ разрешима.

\textbf{Доказательство.} $\Rightarrow$. Мы знаем, что все $G_{i-1} \triangleleft G_i$ и $G_i / G_{i-1} = \Gal(x^{n_i} - a_i)$.
Теперь мы хотим, чтобы все $\Gal(x^{n_i} - a_i)$ были абелевы, а отсюда по утверждению из теории групп $\Gal(f)$ разрешима.

$\Leftarrow$. Так как $\Gal(f)$ разрешима можно построить цепочку $\Gal(f) = G_s \supset G_{s-1} \supset \dots \supset G_0 = \{e\}$, такую что все $G_i / G_{i-1}$ циклические.
Отсюда все $K_{i-1} \supset K_i$ являются полями разложения $x^{n_i} - a_i$.

В доказательстве остались дырки, которые вынесены в отдельные утверждения.

\QED

\textbf{Замечание.} Дырка в $\Rightarrow$: слишком много хотим.
А именно, например, $\Gal(x^3 - 2) = S_3$, $\Gal(x^4 - 2) = D_4$ --- совсем не абелевы.
Решается расширением полей в два шага: сначала добавим $x^n - 1$, получится циклическая группа Галуа, потом $x^n - a$, с корнями из единицы это уже будет абелева.
Дырка в $\Leftarrow$ не особо серьёзная, а именно, называется теорией Куммера.

\textbf{Утверждение.} Если $L \supset F$ --- поле разложения $x^n - 1$ и $(\Char(F), n) = 1$, то $\Gal_F(L)$ абелева.

\textbf{Доказательство.} В частности, $L = F(\xi_n)$.
Заметим, что любой элемент $\Gal_F(L)$ задаётся, как $\xi_n \mapsto \xi_n^k$, сохраняющий $F$.
Рассмотрим $g, h \in \Gal_F(L)$, $g(\xi_n) = \xi_n^k$ и $h(\xi_n) = \xi_n^l$, тогда $g \circ h(\xi_n) = \xi_n^{kl}$.

Теперь возьмём $\phi: \Gal_F(L) \to \mathbb Z_n^*$, $\phi(g)$ --- степень $\xi_n$ в выражении $g(\xi_n)$.
По доказанному это гомоморфизм, так что $\Gal_F(L)$ абелева.

\QED

\textbf{Утверждение.} Если $L \supset F$ --- поле разложения $x^n - a$ и в $F$ если все корни $x^n - 1$, то $\Gal_F(L)$ вкладывается в $\mathbb Z_n$, то есть $\Gal_F(L)$ циклическая.

\textbf{Доказательство.} Возьмём один из корней $\alpha$: $\alpha^n = a$ и $g \in \Aut_F(L)$.
Тогда $\frac{g(\alpha)}{\alpha} \in F$, так как $g$ переводит $\alpha$ в один из корней $x^n - a$, а это в точности $\alpha$, умноженное на корень $n$-ой степени из единицы, который лежит в поле.
Теперь можно положить $\phi: \Aut_F(L) \to F^*$, сопоставляющий отображениям $g$ элемент $\frac{g(\alpha)}{\alpha}$.
Это гомоморфизм, так как
\[
    \frac{h \circ g(\alpha)}{\alpha} = \frac{h(g(\alpha))}{g(\alpha)} \cdot \frac{g(\alpha)}{\alpha} = \frac{h(g(\alpha))}{\alpha}.
\]
Теперь из этого гомоморфизма можно построить вложение в $\mathbb Z_n$: так как $\frac{g(\alpha)}{\alpha} = \xi_n^k$, достаточно взять этот самый $k$.
Во-первых, определение корректно, так как если взять другой корень $\alpha \cdot \xi_n^m$, то получится 
\[
    \frac{g(\alpha \xi_n^m)}{\alpha \xi_n^m} = \frac{\xi_n^m g(\alpha)}{\xi_n^m \alpha} = \frac{g(\alpha)}{\alpha}.
\]
Во-вторых, степени суммируются при композиции: пусть $\frac{g(\alpha)}{\alpha} = \xi_n^k$, $\frac{h(\alpha)}{\alpha} = \xi_n^l$, тогда
\[
    g \circ h(\alpha) = g(\xi_n^l \cdot \alpha) = \xi_n^k g(\alpha) = \alpha \cdot \xi_n^{k+l}.
\]

\QED

\textbf{Замечание.} В доказательстве теоремы о разрешимости мы теперь будем специально брать поля разложения так, чтобы $G_i / G_{i-1} = \Gal(x^n - 1)$, тогда будет абелева, или $G_i / G_{i-1} = \Gal(x^n - a)$, и все корни из единицы уже есть.

\textbf{Пример.} Пусть $f(x) = x^5 - 4x + 2$ над $\mathbb Q$.
По признаку Эйзенштейна он неприводим, по матанализу у него есть 3 вещественных корня.
Из первого следует, что $\Gal(f)$ транзитивна, из второго --- что комплексное сопряжение является автоморфизмом.
Получается, что $\Gal(f) \subset S_5$ и найдётся транспозиция $(i, j) \in \Gal(f)$.

\textbf{Утверждение.} Если $G \subset S_p$ транзитивна и найдётся $(i, j) \in G$, то $G = S_p$.

\textbf{Доказательство.} Рассмотрим граф транспозиций.
Мы хотим доказать, что во всех компонентах связности одинаковое число элементов.
Рассмотрим какую-то вершину $k$, теперь докажем, что $\deg(1) \le \deg(k)$.
Без ограничения общности $(1~2) \in G$, также зафиксируем $i \in \{1, \dots, n\}$, тогда найдётся $\sigma \in G$, такая что $\sigma(1) = i$ (в силу транзитивности).
Тогда в группе есть элемент $\sigma (1~2) \sigma^{-1} = (i~\sigma(2))$.
Следовательно, любой транспозиции из единицы мы сопоставили транспозицию из $i$, откуда $\deg(1) \le \deg(i)$.

Таким образом, все компоненты связности содержат одинаковое число элементов, откуда в силу того, что $p$ простое и группа не пустая, все эти размеры равны $p$.

\QED

\textbf{Пример.} (Уравнение, не разрешимое в радикалах) Будем строить так, чтобы группа Галуа $\Gal(f)$ была равна $S_5$, ибо $S_5' = A_5$ и $A_5' = A_5$.
Построим многочлен с ровно тремя вещественными корнями, тогда два будут комплексными.
Тогда у нас будет автоморфизм, соответствующий транспозиции, --- комплексное сопряжение.
Более того, любые два корня можно перевести друг в друга автоморфизмом, откуда группа Галуа транзитивна.
Следовательно, $\Gal(f) = S_5$.
Пример подходящего многочлена: $x^5 - 4x + 2$.


\textbf{Теорема.} (Теория Куммера) Пусть $K \supset F$ --- расширение Галуа и $F$ содержит все корни $x^n - 1$, причём $(n, \Char(F)) = 1$.
Тогда $\Gal_F(K) \cong \mathbb Z_n$ циклическая тогда и только тогда, когда найдётся $a \in F$, такое что $K$ является полем разложения $x^n - a$ над $F$.

\textbf{Доказательство.}
$\Leftarrow$ доказали ранее.
$\Rightarrow$. Пусть $\Gal_F(K) = \left<g\right>_n$.
Мы хотим найти $\alpha \in K$, такой что $\frac{g(\alpha)}{\alpha} = \gamma$, где $\gamma$ --- корень $n$-ой степени из единицы.
Это нужно, чтобы получить, что все $g^k(\alpha) = \gamma^k \cdot \alpha$ лежат в поле, а это и есть все корни уравнения $x^n - \alpha^n = 0$.
Но почему $\alpha^n$ лежит в $F$?
Всё просто: $g(\alpha^n) = g(\alpha)^n = \gamma^n \alpha^n = \alpha^n$, то есть $\alpha^n \in K^{\Gal_F(K)} = F$.

Теперь найдём $\alpha$: для этого возьмём $b \in F$ и запишем очень интересную линейную комбинацию
\[
    \alpha = \lambda_0 b + \lambda_1 g(b) + \dots + \lambda_{n-1}g^{n-1}(b).
\]
Для пафоса применим к ней ещё раз $g$:
\[
    g(\alpha) = \lambda_0 g(b) + \lambda_1 g^2(b) + \dots + \lambda_{n-1} b,
\]
в конце просто $b$, так как по условию $g^n = id$.
Теперь в линейной комбинации возьмём $\lambda_i = \gamma^i$, получится
\[
    g(\alpha) = g(b) + \gamma g^2(b) + \dots + \gamma^{n-1} b = \gamma^{-1}(\gamma g(b) + \gamma g^2(b) + \dots + \gamma^n b) =
\]
(просто поставим последнее слагаемое в начало)
\[
    = \gamma^{-1}(b + \gamma g(b) + \dots + \gamma^{n-1} g^{n-1}(b)).
\]
Заметим, что это в скобках получилась в точности линейная комбинация, написанная в начале, то есть $g(\alpha) = \gamma^{-1} \alpha$ --- а это и есть то, что мы хотели.

\QED

\textbf{Замечание.} В доказательстве есть дырка --- $\alpha$ может быть равно нулю.
Но можно подобрать так, чтобы проблем не было.

\textbf{Определение.} Пусть $G$ --- группа, $F$ --- поле. \textit{Характером} называется гомоморфизм $\chi: G \to F^*$.

\textbf{Теорема.} (Артена о характерах) Пусть $\chi_1, \dots, \chi_n$, $\chi_i: G \to F^*$, --- различные характеры, где $G$ конечна. Тогда они линейно независимы.

\textbf{Доказательство.} Индукцией по $k$. При $k = 1$ имеем $\lambda \chi_1(g) = 0$, тогда $\lambda = 0$, так как в $F^*$ нет нуля.

Переход $k \to k + 1$: пусть для всех $g$
\[
    \tag{$*$}
    \lambda_1 \chi_1(g) + \dots + \lambda_{k+1} \chi_{k+1}(g) = 0. \hfill 
\]
Без ограничения общности будем считать, что $\lambda_1 \ne 0$.
Так как характеры различны, найдётся $g_0 \in G$, такой что $\chi_1(g_0) \ne \chi_{k+1}(g_0)$.
Домножим (*) на $\chi_{k+1}(g_0)$:
\[
    \lambda_1 \chi_1(g) \chi_{k+1}(g_0) + \dots + \lambda_{k+1} \chi_{k+1}(g) \chi_{k+1}(g_0) = 0.
\]
Теперь подставим в (*) $g = g + g_0$:
\[
    \lambda_1 \chi_1(g) \chi_1(g_0) + \dots + \lambda_{k+1} \chi_{k+1}(g) \chi_{k+1}(g_0) = 0.
\]
Наконец, вычтем:
\[
    \lambda_1 \chi_1(g)(\chi_{k+1}(g_0) - \chi_1(g_0)) + \dots + \lambda_k \chi_k(g)(\chi_{k+1}(g_0) - \chi_k(g_0)) = 0.
\]
Так как $\chi_{k+1}(g_0) - \chi_i(g_0)$ --- это какие-то константы, их можно загнать в $\lambda_i$ и дальше по индукции.
По построению сверху не всё сократилось, так как $\chi_{k+1}(g_0) \ne \chi_1(g_0)$.

\QED

\textbf{Замечание.} Беря $G = F^*$, мы получаем, что различные автоморфизмы $F^*$ линейно независимы.
Это закрывает дырку в теории Куммера: если для всех $b$ мы получаем $\alpha = 0$, то автоморфизмы $g, g^2, \dots, g^n$ линейно зависимы.

\section{Великая теорема Ферма при $n = 3$}
Мы хотим доказать, что уравнение $x^3 + y^3 = z^3$ не имеет нетривиальных решений в целых числах.
Докажем более сильное утверждение: нетривиальных решений нет в числах Эзенштейна $\mathbb Z[\omega]$.
Тогда уравнение переписывается в виде $(x + y)(x + \omega y)(x + \omega^2 y) = z^3$.
Сразу скажем, что $x, y, z$ взаимно просты, иначе сократим.

Исследуем НОДы: $(x + y, x + \omega y) = (x + y, (1 - \omega) y)$, $(x + \omega y, x + \omega^2 y) = (x + \omega y, \omega(1 - \omega) y)$, $(x + y, x + \omega^2 y) = (x + y, (1 + \omega)(1 - \omega) y)$.

Далее мы будем доказывать, что все вышеуказанные НОДы равны $1 - \omega$.
Для этого будем считать сумму $\pm x^3 \pm y^3 = \pm z^3$ по модулю 9.
Почему? Потому что $(1 - \omega)^2 \sim 3$, то есть $9 \sim (1 - \omega)^4$.
Положим $\lambda = 1 - \omega$.

\textbf{Утверждение.} Если один из $x + y, x + \omega y, x + \omega^2 y$ делится на $1 - \omega$, то и все делятся на $1 - \omega$.
Действительно, они отличаются друг от друга на $(1 - \omega) y$ или $(1 - \omega) \omega y$.

\textbf{Лемма 1.} Пусть $x \in \mathbb Z[\omega]$. Тогда либо $\lambda~|~x$, либо $x \equiv r \mod 3$, где $r \in \mathbb Z[\omega]^*$.

\textbf{Доказательство.} Так как $(1 - \omega)^2 \sim 3$, получаем $3~|~(x - r) \iff (1 - \omega)^2~|~(x - r)$.
Теперь можно перебрать все 9 вариантов того, чему равен $r$: а именно, $a + b\omega$, где $a, b \in \{0, \pm 1\}$, и увидеть, что это действительно так.
В частности, $1, 1 + \omega, \omega, -1, \omega^2, 1 + \omega^2$ обратимы, так как лежат на единичной окружности, а $1 - \omega, 0, -1 + \omega$ делятся на $\lambda$.

\QED

\textbf{Лемма 2.} Если $x \in \mathbb Z[\omega]$ и $\lambda \nmid~x$, то $x^3 \equiv \pm 1 \mod 9$.

\textbf{Доказательство.} По лемме 1 $x = 3z + u$, где $u \in \mathbb Z[\omega]^*$.
Возведём в куб: $x^3 = u^3$ (снова можно перебрать все остатки и убедиться).

\QED

\textbf{Лемма 3.} Если $\pm x^3 \pm y^3 \pm z^3 = 0$ и $x, y, z \in \mathbb Z[\omega]$, причём $(x, y) = (x, z) = (y, z) = 1$, то $\lambda~|~xyz$.

\textbf{Доказательство.} Действительно, в противном случае по лемме 2 получаем $\pm x^3 \pm y^3 \pm z^3 \equiv \pm 1 \pm 1 \pm 1 \equiv 0 \mod 9$, что быть не может.

\QED

Теперь рассмотрим уравнение $r_x x^3 + r_y y^3 + r_z z^3 = 0$, где $r_x, r_y, r_z \in \mathbb Z[\omega]^*$.
По лемме 3 можно сичтать, что $\lambda~|~z$.
Вынесем $\lambda$ из $z$: $r_x x^3 + r_y y^3 = r_z \lambda^{3k} \cdot \overline z^3$, где $\lambda \nmid~xy\overline z$ (так можно считать в силу взаимной простоты).

\textbf{Утверждение.} Если существует решение этого уравнения, то существует решение уравнения $x^3 + y^3 = \tilde r_z \lambda^{3k} z^3$, где $k \ge 2$.

\textbf{Доказательство.} Рассмотрим наше уравнение по модулю 9.
По лемме 1 $x^3 = y^3 = z^3 = \pm 1 \mod 9$, откуда $\pm r_x \pm r_y \pm r_z \lambda^{3k} \equiv 0 \mod 9$.
Заметим, что $|\lambda^3| = 3 \sqrt 3 \approx 5$, и от него нельзя отойти на расстояние 2 (то есть прибавить $\pm r_x \pm r_y$, модуль которых равен единице) до числа, делящегося на 9.
Немного рукомахательно, но теперь $k \ge 2$, откуда $\pm r_x \pm r_y \equiv 0 \mod 9$, значит, $\pm r_x \pm r_y = 0$, так как они обратимы, то есть остаток по модулю 9 их однозначно определяет.
Остаётся подставить и сократить.

\QED

Теперь пусть $(x, y, z)$ --- решение $x^3 + y^3 = z^3$.
Перейдём к решению уравнения $x^3 + y^3 = \tilde r_z \lambda^{3k} z^3$, возьмём такое, что $k$ минимально.
Тогда $(x + y)(x + \omega y)(x + \omega^2 y) = \tilde r_z \lambda^{3k} z^3$.
Так как произведение слева делится на $\lambda^{3k}$, хотя бы один из них делится на $\lambda$, а по первому утверждению они все делятся на $\lambda$.
Нетрудно заметить, что если НОД выражений под скобками не равен $\lambda$, то НОД $x, y, z$ будет не равен единице.
Отсюда имеет место запись
\[
    \begin{cases}
        x + y = r_{x_0} \lambda x_0^3 \\
        x + \omega y = r_{y_0} \lambda y_0^3 \\
        x + \omega^2 y = r_{z_0} \lambda^{3k - 2} z_0^3
    \end{cases} .
\]
Оставшуюся степень можно запихнуть в последнее без ограничения общности, ибо достаточно сделать замену.
Теперь заметим, что
\[
    0 = (x + y) + \omega(x + \omega y) + \omega^2(x + \omega^2 y) =
\]
\[
    = r_{x_0} \lambda x_0^3 + \omega r_{y_0} \lambda y_0^3 + \omega^2 r_{z_0}^3 \lambda^{3k - 2} z_0^3.
\]
Сократим $\lambda$ и перегруппируем слагаемые:
\[
    r_{x_0} x_0^3 + \tilde r_{y_0} y_0^3 + \tilde r_{z_0} \lambda^{3k - 3} z_0^3 = 0.
\]
Итак, нашлась новая тройка $(x_0, y_0, z_0)$, такая что $(x_0, y_0, z_0) = 1$ и $k$ строго меньше --- противоречие со взятием $k$.

\QED

\section{Теорема Гильберта о нулях}
Пусть $F$ --- поле, $K \supset F$ --- алгебраически замкнутое алгебраическое расширение.
Положим для идеала $I \subset F[x_1, \dots, x_n]$ отображение $V(I)$ --- множество таких $(t_1, \dots, t_n) \subset K^n$, что для всех $f \in I$ выполняется $f(t_1, \dots, t_n) = 0$.
Иными словами, $V(I)$ --- множество наборов переменных, обнуляющих все многочлены в $I$.

\textbf{Теорема.} (Nullstellensatz, сильная форма, доказательство на отл.(10)) Пусть $f \in F[x_1, \dots, x_n]$, $I \subset F[x_1, \dots, x_n]$ --- идеал, такой что $f(x) = 0$ для всех $x \in V(I)$.
Тогда найдётся $r \in \mathbb N$, такое что $p^r \in I$.

\textbf{Теорема.} (Nullstellensatz, слабая форма) Пусть $I \subset F[x_1, \dots, x_n]$ --- идеал. $I$ содержит единицу тогда и только тогда, когда $V(I)$ пусто.

\textbf{Доказательство.} $\Rightarrow$: идеал совпадает с $F[x_1, \dots, x_n]$, очевидно пусто.

$\Leftarrow$. От противного: пусть $I$ не содержит единицу, тогда его можно расширить до максимального идеала $J$.
Как известно, $J = (x_1 - a_1, \dots, x_m - a_m)$, где $a_1, \dots, a_m \in K$.
Тогда заметим, что все многочлены в $J$ обнуляются на $(a_1, \dots, a_m)$, значит, и все многочлены в $I$ обнуляются, то есть $(a_1, \dots, a_m) \in V(I)$ --- противоречие.

\QED
