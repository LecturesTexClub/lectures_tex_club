\section{Введение}
\epigraph{На самом деле, вам очень повезло с лектором по ТФКП.}{\textit{Владимир Лысов}}

Будем определять $\mathbb C$, как расширение поля $\mathbb R$ корнем уравнения $x^2 + 1$, обозначаемым $i$.
Если $z = x + iy \in \mathbb C$, то $x = \re(z)$, $y = \im(z)$ --- вещественная и мнимая часть соответственно.

Дальше некоторые свойства из ТКИПа, неравенство треугольника (равенство при $z_1 \overline z_2 \in \mathbb R_{\ge 0}$), в общем, ничего нового.

Также комплексное число $z$ можно записать в виде $z = r(\cos(\phi) + \sin(\phi))$, где $r = |z|$, $\phi \in \Arg\{z\} = \{\arg(z) + 2\pi k~|~k \in \mathbb Z\}$.

\textbf{Утверждение.} Пусть $z_j = r_j (\cos(\phi_j) + \sin(\phi_j))$.
Тогда $z_1z_2 = r_1r_2(\cos(\phi_1 + \phi_2) + i \cdot \sin(\phi_1 + \phi_2))$.

\textbf{Обозначение.} Пусть $z$ --- корень уравнения $z^n = c = r(\cos(\phi) + \sin(\phi))$.
Тогда 
\[
    z \in \{\sqrt[n]{c}\} = \left\{\sqrt[n]{r} \left( \cos \left( \frac{\phi + 2\pi k}{n} \right) + i \cdot \sin \left( \frac{\phi + 2 \pi k}{n} \right) \right), k \in \mathbb Z \right\}.
\]

\textbf{Определение.} $O_r(a)$ --- открытая окрестность, $\overline{O_r(a)}$ --- замкнутая, $\dot O_r(a)$ --- проколотая открытая, $\dot O_R(\infty)$ --- числа с модулем более $R$.
Последовательность $z_n \to a$, если $\forall \varepsilon > 0~\exists N > 0: \forall n > N~z_n \in O_\varepsilon(a) \Rightarrow |z_n - a| < \varepsilon$, далее критерий Коши.
Расширенная плоскость --- $\overline{\mathbb C} = \mathbb C \cup \{\infty\}$.

\section{Комплексная дифференцируемость}
\epigraph{Голоморфные функции, вообще, — это отвратительная вещь.}{\textit{Владимир Лысов}}

\textbf{Теорема.} (Принцип Больцано--Вейерштрасса, б/д) Из любой последовательности комплексных чисел можно выбрать сходящуюся подпоследовательность в $\overline{\mathbb C}$.

% \textbf{Утверждение.} (Рукомахательное) Если рассмотреть стереографическую проекцию сферы на плоскость, то всем точкам, кроме северного полюса, можно сопоставить точку на плоскости, а северному полюсу --- $\infty \in \overline{\mathbb C}$.

\textbf{Определение.} Пусть $f: \dot O_r(a) \to \mathbb C$.
Тогда $\lim_{z \to a} f(z) = A$, если $\forall \varepsilon > 0~\exists \delta  > 0: \forall z \in \dot O_\delta(a)~f(z) \in O_\varepsilon(A)$.

Если $f$ дополнительно определена в $a$, то она называется \textit{непрерывной} в точке $a$, если $\lim_{z \to a} f(z) = f(a)$.
Как следствие, если $f$ непрерывна в точке $a$, то $\re(f), \im(f), |f|$ в ней непрерывны.

$f$ называется (комплексно) \textit{дифференцируемой} в точке $a$, если существует предел
\[
    \lim_{\Delta z \to 0} \left( \frac{f(a + \Delta z) - f(a)}{\Delta z} \right) \in \mathbb C.
\]
Этот предел называется \textit{производной} и обозначается через $f'(a)$.

\textbf{Определение.} Функция $f: \dot O(a) \to \mathbb C$ называется \textit{бесконечно малой}, если $\lim_{z \to a} f(z) = 0$.
$o(1), z \to a$ --- класс таких функций.

\textbf{Теорема.} Функция $f(x + iy) = u(x, y) + i \cdot v(x, y)$ ($u = \re(f), v = \im(f)$) комплексно дифференцируема в точке $z_0 = x_0 + i \cdot y_0$ тогда и только тогда, когда $u$ и $v$ дифференцируемы в точке $(x_0, y_0)$, а также выполнено условие Коши-Римана:
\[
    \begin{cases}
        \frac{\partial u}{\partial x}(x_0, y_0) = \frac{\partial v}{\partial y}(x_0, y_0) \\
        \frac{\partial u}{\partial y}(x_0, y_0) = - \frac{\partial v}{\partial x}(x_0, y_0) \\
    \end{cases}
\]

\textbf{Доказательство.} Обозначим $\Delta f = f(z_0 + \Delta z) - f(z_0)$, $\Delta u = u(x_0 + \Delta x, y_0 + \Delta y) - u(x_0, y_0)$, аналогично $\Delta v$.
В частности, $\Delta f = \Delta u + i \cdot \Delta v$.
Тогда комплексная дифференцируемость эквивалентна тому, что $\Delta f = f'(z_0) \Delta z + o(\Delta z)$.
Также заметим, что $\mathbb R$--дифференцируемость функции $u$ эквивалентна тому, что $\Delta u = u_x' \Delta x + u_y' \Delta y + o(\Delta z)$, аналогично для $v$.

$\Rightarrow$. Пусть $f'(z_0) = \alpha + i \cdot \beta$.
Время арифметики:
\[
    \Delta u + i \cdot \Delta v = \Delta f = \alpha \Delta x - \beta \Delta y + i(\alpha \Delta y + \beta \Delta x) + o(\Delta z).
\]
Отсюда
\[
    \Delta u = \alpha \Delta x - \beta \Delta y + o(\Delta z)
\]
и
\[
    \Delta v = \alpha \Delta y + \beta \Delta x + o(\Delta z),
\]
что и требовалось.

$\Leftarrow$. Распишем через дифференциалы: $\dif u = u'_x \dif x + u'_y \dif y$, $\dif v = v'_x \dif x + v'_y \dif y$.
Тогда 
\[
    \dif f = \dif u + i \dif v = (u'_x + i \cdot v'_x) \dif x + (u'_y + i \cdot v'_y) \dif y.
\]
Положим $f'_x = u'_x + i \cdot v'_x$ и $f'_y = u'_y + i \cdot v'_y$.
Из формул $\dif z = \dif x + i \cdot \dif y$ и $\dif \overline z = \dif x - i \cdot \dif y$ можно выразить $\dif x$ и $\dif y$, тогда выражение выше переписывается в виде
\[
    \dif f = f'_x \cdot \frac{\dif z + \dif \overline z}{2} + f'_y \cdot \frac{\dif z - \dif \overline z}{2i} = \frac{1}{2} (f'_x - i \cdot f'_y) \dif z + \frac{1}{2} (f'_x + i \cdot f'_y) \dif \overline z.
\]
Обозначим первое слагаемое за $f'_z$, а второе --- за $f'_{\overline z}$.
Заметим, что из условия Коши--Римана следует $f'_{\overline z} = 0$, так как
\[
    0  = u'_x + i \cdot v'_x + i(u'_y + i \cdot v'_y) = u'_x - v'_y + i(u'_y + v'_x) = 2f'_{\overline z}.
\]
Следовательно, 
\[
    \dif f = \frac{1}{2} (f'_x - i \cdot f'_y) \dif z,
\]
то есть производная существует.

\QED

\textbf{Определение.} Функция $f: O(z_0) \to \mathbb C$ называется \textit{(комплексно) дифференцируемой} в точке $z_0$, если
\[
    \Delta f := f(z) - f(z_0) = f'(z_0) \Delta z + o(\Delta z), \Delta z := z - z_0 \to 0.
\]
Нетрудно заметить, что это определение эквивалентно предыдущему.

\textbf{Утверждение.} Если $f$ комплексно дифференцируема в точке $z_0$, а $g$ комплексно дифференцируема в точке $f(z_0)$, то $h := g \circ f$ комплексно дифференцируема в точке $z_0$, и $h'(z_0) = f'(z_0) \cdot g'(f(z_0))$.

\textbf{Доказательство.} Положим $w_0 := f(z_0)$. 
По определению дифференцируемости $\Delta f = f'(z_0) \Delta z + o(\Delta z)$ и $\Delta g = g'(w_0) \Delta w + o(\Delta w)$ для любых приращений.
Теперь положим $\Delta w = \Delta f$ и распишем определение дифференцируемости для $h$:
\[
    \Delta h = \Delta (g \circ f) = g(f(z)) - g(f(z_0)) = g'(w_0) \Delta f + o(\Delta f) =
\]
Раскроем $\Delta f$ и $o(\Delta f)$ через определение дифференцируемости для $f$:
\[
    = (g'(w_0) f'(z_0) \Delta z + g'(w_0) \cdot o(\Delta z)) + (f'(z_0) o(\Delta z) + o(o(\Delta z))) =
\]
\[
    = g'(w_0) f'(z_0) \Delta z + o(\Delta z).
\]

\QED

\textbf{Определение.} Множество $D \subset \overline {\mathbb C}$ называется \textit{открытым}, если любая точка входит с окрестностью.

\textbf{Определение.} Функция $u: D \to \mathbb R$ называется \textit{гармонической}, если $u \in C^2(D)$ и $\Delta u = u''_{xx} + u''_{yy} \equiv 0$ на $D$.

\textbf{Следствие.} Пусть функция $f$ комплексно дифференцируема на открытом множестве $D$.
Тогда $u := \re(f)$ и $v := \im(f)$ дважды дифференцируемы, и они являются гармоническими в $D$.

\textbf{Доказательство.} Из дифференцируемости
\[
    \begin{cases}
        u'_x = v'_y \\
        u'_y = -v'_x
    \end{cases} .
\]
Продифференцируем верхнее уравнение по $x$, а второе --- по $y$:
\[
    \begin{cases}
        u''_{xx} = v''_{yx} \\
        u''_{yy} = -v''_{xy}
    \end{cases} .
\]
Сложим их. Заметим, что из двойной дифференцируемости $v''_{yx} = v''_{xy}$, что завершает доказательство.

\QED

\textbf{Определение.} Множество $E \subset {\mathbb C}$ называется \textit{связным}, если не существует пары открытых множеств $G_1$ и $G_2$, таких что:
\begin{enumerate}
    \item $E \subset G_1 \cup G_2$.
    \item $E \cap G_1 \cap G_2 = \varnothing$.
    \item $E \cap G_1 \ne \varnothing$, $E \cap G_2 \ne \varnothing$.
\end{enumerate}
Иными словами, множество не разбивается на два непересекающихся куска.

\textbf{Утверждение.} Отрезок $E$ является связным множеством.

\textbf{Доказательство.} От противного: нашлись $G_1, G_2$.
Тогда найдутся $a, b \in E$, такие что $a \in G_1$ и $b \in G_2$.
Рассмотрим середину отрезка $c := \frac{a + b}{2}$.
Тогда выполнено строго одно из $c \in G_1$ или $c \in G_2$.
Теперь перейдём к подотрезку, одним из концов которого является $c$, а в качестве другого выберем $a$ или $b$ так, чтобы концы полученного подотрезка лежали в разных множествах.

Будем продолжать этот процесс, по принципу Кантора найдётся точка в пересечении всех этих отрезков.
Назовём её $c$, без ограничения общности $c \in G_1$.
Так как $G_1$ открыто, найдётся окрестность $O(c) \subset G_1$.
А это значит, что с какого-то момента в построении подотрезков оба конца лежали в $G_1$ --- противоречие.

\QED

\textbf{Определение.} $D \subset \overline {\mathbb C}$ называется \textit{областью}, если $D$ открыто, связно и непусто.

\textbf{Замечание.} Когда мы говорим про область $D \subset \mathbb C$, мы имеем в виду ограниченную область, так как иначе она бы содержала $\infty$ и лежала в $\overline {\mathbb C}$.

\textbf{Теорема.} (Критерий связности) Открытое множество $D$ является связным тогда и только тогда, когда для всех $z_1, z_2 \in D$ найдётся (конечная, но, возможно, уходящая в бесконечность) ломаная, соединяющая $z_1$ и $z_2$, лежащая в $D$.
Более того, можно считать, что звенья ломаной параллельны осям координат.

\textbf{Пример.} Дополнение до ``полосы`` в $\overline{\mathbb C}$ является связным множеством, так как любые две точки можно соединить через бесконечность (иллюстрация на рисунке 1).
\begin{figure}[ht]
    \centering
    \incfig{connected-strip}{0.35\linewidth}
    \caption{Дополнение до полосы}
\end{figure}

\textbf{Доказательство.}
$\Rightarrow$. Для пустого тривиально. Иначе пусть найдётся $a \in D$.
Положим $G_1 \subset D$ --- множество точек, которые можно соединить с $a$ ломаной, а $G_2$ --- которые нельзя.
Заметим, что $G_1$ открыто: рассмотрим $z \in G_1$, пусть она лежит в $D$ с окрестностью $O(z)$, тогда можно продолжить ломаную от $a$ до $z$ двумя звеньями до всех точек этой окрестности.
Аналогично $G_2$ открыто.

Заметим, что для множеств $G_1$, $G_2$ выполнены пункты (1) и (2) определения связности, поэтому одно из них должно быть пусто, ибо $D$ связно.
Но в $G_1$ лежит $a$, поэтому $G_2$ пусто.

$\Leftarrow$. 
Допустим, что $D$ несвязно, тогда найдутся $G_1$ и $G_2$.
Возьмём $z_1 \in G_1$ и $z_2 \in G_2$ соединим их ломаной.
Внимательно посмотрим на эту ломаную: у неё должно найтись звено, концы которого лежат в разных множествах среди $G_1$ и $G_2$.
Но звено является отрезком, то есть мы получили, что отрезок несвязен.

\QED

\textbf{Определение.} Функция $f$ называется \textit{голоморфной в области $D \subset \mathbb C$}, если она комплексно дифференцируема в каждой точке $D$.

\textbf{Определение.} Функция $f$ \textit{голоморфна на множестве $E$}, если существует область $D \supset E$, в которой $f$ голоморфна.

\textbf{Теорема.} (О голоморфной функции с нулевой производной) Пусть $f$ голоморфна в области $D$ и $f'(z) \equiv 0$.
Тогда $f \equiv const$.

\textbf{Доказательство.} Зафиксируем $z_1, z_2 \in D$, докажем, что $f(z_1) = f(z_2)$.
Соединим эти точки ломаной, звенья которой параллельны осям координат (по критерию связности так можно).
Из того, что $f' \equiv 0$, следует $u'_x \equiv u'_y \equiv v'_x \equiv v'_y \equiv 0$.
Получается, что $f$ постоянна вдоль оси $Ox$ и $Oy$ из матанализа, а значит, она постоянна на ломаной.

\QED

\section{Степенные ряды}
Пока что у нас не так много голоморфных функций: только многочлены.
Поэтому мы перейдём к более широкому классу: ``бесконечные`` многочлены, то есть степенные ряды.

\subsection{Воспоминания из матанализа}
\textbf{Определение.} Последовательность функций $f_n: E \to \mathbb C$ \textit{сходится поточечно} к функции $f$ на $E$, если
\[
    \forall z \in E, \varepsilon > 0~\exists N: \forall n > N~|f_n(z) - f(z)| < \varepsilon.
\]
Равномерно, если 
\[
    \forall \varepsilon > 0~\exists N: \forall n > N, z \in E~|f_n(z) - f(z)| < \varepsilon.
\]

\textbf{Утверждение.} Если $f_n$ непрерывны на $E$ и $f_n \rightrightarrows^E f$, то $f$ непрерывна на $E$.

\textbf{Утверждение.} (Критерий Коши) $f_n \rightrightarrows^E f \iff \forall \varepsilon > 0~\exists N: \forall n, m > N, z \in E~|f_n(z) - f_m(z)| < \varepsilon$.

\textbf{Утверждение.} (Признак Вейерштрасса) Пусть $S_n = \sum_{k=1}^{n} f_n$, причём $|f_n| \le \alpha_n$, где ряд из $\alpha_n$ сходится.
Тогда ряд из $f_n$ сходится абсолютно и равномерно.

\subsection{Конец воспоминаний}
Зафиксируем ряд
\begin{equation}
    \sum_{n=0}^{\infty} a_n z^n
\end{equation}
и константу
\[
    R = \frac{1}{\overline{\lim_{n \to \infty}} \sqrt[n]{|a_n|}}.
\]
Обозначим за $\mathcal O(E)$ класс голоморфных функций на множестве $E$.

\textbf{Теорема.} (О степенных рядах)
\begin{enumerate}
    \item Если $R > 0$, $r \in (0, R)$, то ряд (1) сходится абсолютно и равномерно в $\overline O_r(0)$.
    \item Если $|z| > R$, то ряд (1) расходится.
    \item Если $R > 0$, то ряд $f(z) = (1) \in \mathcal O(O_R(0))$, причём $f'(z) = \sum_{n=1}^{\infty} n a_n z^{n-1}$.
\end{enumerate}

\textbf{Доказательство.}
\begin{enumerate}
    \item Зафиксируем $\rho \in (r, R)$, тогда $\frac{1}{r} > \frac{1}{\rho} > \frac{1}{R}$.
        Так как $\overline{\lim_{n \to \infty}} \sqrt[n]{|a_n|} = \frac{1}{R}$, найдётся $N$, такое что для $n > N$ выполнено $|a_n| \le \frac{1}{\rho^n}$.
        Значит, для таких $n$ и для $|z| \le r$ выполнено $|a_n z^n| \le \left( \frac{r}{\rho} \right)^n$.
        Следовательно, по признаку Вейерштрасса ряд сходится равномерно и сбсолютно.

    \item Зафиксируем $z$ и $\rho$, такие что $|z| > \rho > R$.
        Тогда $\frac{1}{\rho} < \frac{1}{R}$.
        Из определения верхнего предела получаем, что найдётся подпоследовательность $n_k$, такая что $|a_{n_k}| \ge \frac{1}{\rho^{n_k}}$.
        Тогда $|a_{n_k} z^{n_k}| \ge \frac{\rho^{n_k}}{\rho^{n_k}} = 1$, то есть ряд расходится по необходимому условию.

    \item Положим $g(z) = \sum_{n=1}^{\infty} n a_n z^{n-1}$.
        Она определена корректно, так как $\overline {\lim_{n \to \infty}} \sqrt[n]{n a_n} = \frac{1}{R}$.
        Пусть $S_n = \sum_{k=0}^{n-1} a_k z^k$ и $S_n' = \sum_{k=1}^{n-1} k a_k z^{k-1}$, а также $E_n := f(z) - S_n$.
        Зафиксируем $z_0$, такой что $|z_0| < R$, и $r \in (|z_0|, R)$.
        Рассмотрим
        \[
            \frac{f(z) - f(z_0)}{z - z_0} - g(z_0) = \frac{S_n(z) - S_n(z_0)}{z - z_0} + \frac{E_n(z) - E_n(z_0)}{z - z_0} - g(z_0) =
        \]
        \[
            = \left( \frac{S_n(z) - S_n(z_0)}{z - z_0} - S_n'(z_0) \right) + \left( \frac{E_n(z) - E_n(z_0)}{z - z_0} \right) + \left(S_n'(z_0) - g(z_0) \right).
        \]
        Теперь отдельно докажем малость каждого из слагаемых.
        \[
            \frac{E_n(z) - E_n(z_0)}{z - z_0} = \sum_{k=n}^{\infty} a_k \cdot \frac{z^k - z^k_0}{z - z_0} = 
        \]
        \[
            = \sum_{k=n}^{\infty} a_k(z^{k-1} + z^{k-2} z_0 + \dots + z_0^{k-1}) \le
        \]
        В $k$--ом слагаемом у нас сумма из $k$ степеней $z$ и $z_0$, они оба не превоходят $r$, так что
        \[
            \le \sum_{k=n}^{\infty} |a_k| \cdot k r^{k-1}.
        \]
        Этот ряд сходится, поэтому можно сделать оценку
        \[
            \exists N: \forall n > N~\left| \frac{E_n(z) - E_n(z_0)}{z - z_0} \right| < \frac{\varepsilon}{3}
        \]
        для всех $z, z_0 \in O_r(0)$, как хвоста сходящегося ряда.
        Аналогично найдётся $N'$, такое что для $n > N'$ выполнено $|S_n'(z_0) - g(z_0)| < \frac{\varepsilon}{3}$.

        Теперь из дифференцируемости многочлена $S_n$ следует, что найдётся $\delta > 0$, такое что для $z \in \dot O_\delta(z_0)$ выполнено
        \[
            \left|\frac{S_n(z) - S_n(z_0)}{z - z_0} - S_n'(z_0) \right| < \frac{\varepsilon}{3}.
        \]
        Следовательно, модуль всей суммы меньше $\varepsilon$ при $n > \max(N, N')$, что завершает доказательство.
\end{enumerate}

\QED

\textbf{Следствие.} Последовательно применяя теорему выше, мы можем получить, что 
\[
    f^{(n)}(z) = \frac{n!}{0!} a_n + \frac{(n + 1)!}{1!} a_{n+1} z + \frac{(n + 2)!}{2!} a_{n+2} z^2 + \dots
\]

\subsection{Экспонента}
\textbf{Определение.}
\[
    e^z := \sum_{n=0}^{\infty} \frac{z^n}{n!}.
\]

\textbf{Замечание.} Так как $\lim_{n \to \infty} \sqrt[n]{n!} = \infty$, радиус сходимости этого ряда равен $R = \infty$.

\textbf{Утверждение.}
\begin{enumerate}
    \item $e^0 = 1$.
    \item $(e^z)' = e^z$.
    \item $e^z$ является целой функцией, то есть голоморфной в $\mathbb C$.
    \item $\overline{e^z} = e^{\overline z}$.
    \item $e^{z_1 + z_2} = e^{z_1} \cdot e^{z_2}$.
    \item $e^z e^{-z} \equiv 1$, то есть $e^z$ не обращается в ноль.
    \item $|e^z| = e^{\re(z)}$.
\end{enumerate}

\textbf{Доказательство.} 1-4 очевидно.

5)
Положим $g(z) = e^z e^{a - z}$.
Тогда $g' = e^z e^{a - z} - e^z e^{a - z} = 0$.
Следовательно, $g$ является константой.
Подставим ноль: $g(0) = e^a$.

6 очевидно, 7 --- прямой проверкой.

\QED

\textbf{Замечание.} Первых двух свойств достаточно, чтобы однозначно восстановить экспоненту.

\textbf{Определение.}
\[
    \cos(z) = \frac{e^{iz} + e^{-iz}}{2}, \sin(z) = \frac{e^{iz} - e^{-iz}}{2i}.
\]

\textbf{Следствие.} $e^{iz} = \cos(z) + i \cdot \sin(z)$ и $e^{iz}$ является $2\pi$--периодичной функцией.
Более того, $e^z = e^{z + a} \iff a = 2\pi k i$ для $k \in \mathbb Z$.

\textbf{Теорема.} (Об обратной функции) Пусть $f$ голоморфна в области $D$ и $f'$ непрерывна в $D$ (это условие избыточно, но мы пока не умеем это доказывать).
Пусть у нас есть точка $z_0$, такая что $f'(z_0) \ne 0$ и $f(z_0) = w_0$.
Тогда найдётся окрестность $U$ точки $z_0$ и $V$ точки $w_0$, такие что $f$ является биекцией между множествами $U$ и $V$, причём $g = f^{-1}$ является голоморфной на $V$ и $g'(w_0) = \frac{1}{f'(z_0)}$.

\textbf{Доказательство.} Из непрерывности $f'$ следует непрерывность $|f'|$.
Следовательно, существует окрестность $O_r(z_0)$, в которой $|f'| > 0$.
Теперь рассмотрим $f$, как отображение из $\mathbb R^2$ в $\mathbb R^2$: $f(x + iy) = u(x, y) + i \cdot v(x, y)$.
Тогда якобианом $f$ будет
\[
    J =
    \begin{vmatrix}
        u_x' & u_y' \\
        v_x' & v_y'
    \end{vmatrix}
    = u_x' v_y' - u_y' v_x' = u_x^2 + v_x^2 = |f'_x|^2 > 0
\]
в окрестности $O_r(z_0)$.
Переход к квадратам получен из условия Коши--Римана.

По одноимённой теореме из матанализа найдётся окрестность $U$ точки $z_0$ и $V$ --- точки $w_0$, такие что $f$ является биекцией и $g = f^{-1}$ непрерывно дифференцируема.
Непрерывная дифференцируемость --- это не совсем то, что нужно; докажем голоморфность.
Рассмотрим $z_0 \in U$, $z \in \dot O_\delta(z_0) \cap U$, $w = f(z)$.
Так как $f$ является биекцией, $w \in \dot O(w_0)$ (не равна $w_0$).

Положим $\Delta z = z - z_0$, $\Delta w = w - w_0$.
Тогда $\Delta z = 0 \iff \Delta w = 0$ и $\Delta z \to 0 \iff \Delta w \to 0$.
Теперь
\[
    \lim_{\Delta w \to 0} \frac{\Delta g}{\Delta w} = \lim_{\Delta w \to 0} \frac{\Delta z}{\Delta w} = \lim_{\Delta w \to 0} \frac{1}{\Delta w / \Delta z} = \lim_{\Delta z \to 0} \frac{1}{\Delta w / \Delta z} =
\]
\[
    = \frac{1}{\lim_{\Delta z \to 0} \frac{\Delta w}{\Delta z}} = \frac{1}{f'(z_0)} = g'(w_0).
\]

\QED

\textbf{Определение.} Многозначная функция $\Ln\{z\} = \ln|z| + i \cdot \Arg\{z\}$ называется \textit{логарифмом} $z$, и она даёт все решения $e^w = a \ne 0$.

\textbf{Определение.} Функция $f$, голоморфная в области $D \subset \overline {\mathbb C}$, называется \textit{однолистной}, если она инъективна.

\textbf{Пример.} Функция $f(w) = e^w$ однолистна на любой горизонтальной полосе, то есть на множествах $\{w: \im(w) \in (\phi_0, \phi_0 + 2\pi)\}$.
Её образом будет всё, кроме луча, полученного из $\phi_0$, то есть $\{z: z \ne 0, \arg(z) \in [\dots]\}$.

Теперь мы хотим её обратить, но логарифм, который мы определяли, многозначен, поэтому придётся ввести ещё несколько формальных определений.

\textbf{Определение.} Функция $F: D \to 2^{\mathbb C}$ называется \textit{многозначной}.

\textbf{Определение.} Функция $f: D \to \mathbb C$ называется \textit{ветвью} многозначной функции $F$, если $f$ непрерывна в $D$ и для всех $z \in D$ выполнено $f(z) \in F(z)$.

\textbf{Пример.} $\Ln\{z\}$ --- это многозначная функция в области $\mathbb C \setminus \{0\}$.
Тогда в любой области без луча, описанного в предыдущем примере, существует ветвь.
Продифференцируем эту ветвь:
\[
    (\ln^{(B)}(z))' = \frac{1}{(e^w)'} = \frac{1}{e^w} = \frac{1}{z}.
\]
Здесь буква $B$ означает ветвь, и это надо всегда добавлять, если в качестве луча не используется $\{z: \arg(z) = \pi\}$.

Теперь разложим в ряд Тейлора:
\[
    (\ln(1 + z))' = \frac{1}{z + 1} = \sum_{n=0}^{\infty} (-1)^n z^n, |z| < 1,
\]
значит, интегрируя, можно получить
\[
    \ln(1 + z) = \sum_{n=1}^{\infty} \frac{(-1)^{n+1} z^n}{n}, |z| < 1.
\]

\section{Интегрирование}
\subsection{Понятие кривой}
Для начала вспомним парочку определений из матанализа.

\textbf{Определение.} \textit{Путь} --- это функция $z: [\alpha, \beta] \to \mathbb C$, $z(t) = x(t) + i \cdot y(t)$, где $x, y \in C[\alpha, \beta]$.
Также известен, как параметризация кривой.

\textbf{Определение.} Два пути $z: [\alpha, \beta] \to \mathbb C$ и $\tilde z: [\alpha_1, \beta_1] \to \mathbb C$ называются \textit{эквивалентными}, если существует непрерывная $\tau \in C[\alpha, \beta]$, такая что $\tau$ строго монотонно возрастает, $\tau([\alpha, \beta]) = [\alpha_1, \beta_1]$ и для всех $z \in [\alpha, \beta]$ выполнено $\tilde z(\tau(t)) = z(t)$.

\textbf{Определение.} \textit{Кривая} --- класс эквивалентности путей.
Обозначение --- $\gamma \subset \mathbb C$.

\textbf{Определение.} \textit{Образ кривой} --- это множество $z([\alpha, \beta])$.

\textbf{Замечание.} Образ кривой --- это компактное связное множество.

\textbf{Определение.} \textit{Длина кривой} --- это
\[
    l(\gamma) = \sup_T \sum_{k=1}^{n} |z(t_k) - z(t_{k-1})|,
\]
где супремум берётся по разбиениям $T = \{t_0, t_1, \dots t_n\}$.

\textbf{Определение.} Кривая называется \textit{спрямляемой}, если $l(\gamma) < \infty$.

\textbf{Определение.} Кривая называется \textit{жордановой}, если из $z(t_1) = z(t_2)$ следует $t_1 = t_2$ (от параметризации очевидным образом не зависит).

\textbf{Определение.} Кривая называется \textit{замкнутой}, если $z(\alpha) = z(\beta)$.

Получается не очень удобно, что замкнутая кривая не может быть жоржановой, поэтому введём прекрасное

\textbf{Определение.} Кривая называется \textit{замкнутожордановой}, если она замкнута и ``жорданова, не считая концы``.

\textbf{Определение.} Кривая $\gamma$ называется \textit{гладкой}, если у неё существует параметризация $z \in C^1[\alpha, \beta]$.
У гладких кривых существует \textit{касательный вектор} $\dot z(t)$, если он не равен нулю.

\textbf{Определение.} Кривая $\gamma$ называется \textit{кусочно гладкой}, если существует параметризация $z$ и разбиение отрезка $\alpha = t_0 < t_1 < \dots < t_n = \beta$, такое что все сужения $z|_{[t_k, t_{k+1}]}$ определяют гладкую кривую.

\textbf{Определение.} Пусть $\gamma$ --- кривая с параметризацией $z: [\alpha, \beta] \to \mathbb C$.
Кривую $-\gamma$ определим, как кривую с параметризацией $\tilde z(t) = z(-t)$.

\subsection{Понятие интеграла}
Определим интеграл комплексной функции по кривой.

\textbf{Определение.} Пусть $D \subset \mathbb C$, $f$ непрерывна в $D$, $\gamma$ --- спрямляемая кривая, $z: [\alpha, \beta] \to \mathbb C$ --- её параметризация.
Рассмотрим отмеченное разбиение $\alpha = t_0 < t_1 < \dots < t_n = \beta$ с точками $\tau_k \in [t_{k-1}, t_k]$.
Положим $\Delta z_k = z(t_k) - z(t_{k-1}) = \Delta x_k + i \cdot \Delta y_k$, $f = u + i \cdot v$.
Тогда
\[
    \sum_{k=1}^{n} f(z(\tau_k)) (z(t_k) - z(t_{k-1})) = 
\]
\[
    = \sum_{k=1}^{n} \bigg( u(z(\tau_k)) \Delta x_k - v(z(\tau_k)) \Delta y_k \bigg) + i \cdot \sum_{k=1}^{n} \bigg( u(z(\tau_k)) \Delta y_k + v(z(\tau_k)) \Delta x_k \bigg)
\]
будет интегральной суммой, как в вещественном случае.
Устремляя мелкость разбиения $\max_k (t_{k-1}, t_k)$ к нулю, получаем по определению два вещественных интеграла
\[
    \int_\gamma(u \dif x - v \dif y) + i \cdot \int_\gamma (u \dif y + v \dif x).
\]
Определим сумму выше, как $\int_\gamma f \dif z$.
Таким образом, мы получили естественное определение комплексного интеграла, и про все эти интегральные суммы можно забыть.

Дополнительно определим интеграл первого рода:
\[
    \int_\gamma f(z) |\dif z| = \int_\gamma u \dif s + i \cdot \int_\gamma v \dif s,
\]
где $\dif s = \sqrt{\dif x^2 + \dif y^2}$ --- длина кривой.

\textbf{Свойства.}
\begin{enumerate}
    \item Пусть $\gamma$ --- кусочно гладкая кривая, $z: [\alpha, \beta] \to \mathbb C$.
        Тогда
        \[
            \int_\gamma f(z) \dif z = \int_\alpha^\beta = f(z(t)) z'(t) \dif t.
        \]

    \item Линейность.

    \item Аддитивность по $\gamma$: если у нас есть две кривые $\gamma_1, \gamma_2$ с параметризациями $z_1: [\alpha, \beta] \to \mathbb C$ и $z_1: [\beta, \beta'] \to \mathbb C$ и $z_1(\beta) = z_2(\beta)$, то положим объединение этих кривых $\gamma_1 + \gamma_2$, и тогда
        \[
            \int_{\gamma_1 + \gamma_2} f \dif z = \int_{\gamma_1} f \dif z + \int_{\gamma_2} f \dif z.
        \]

    \item Без комментариев:
        \[
            \int_{-\gamma} f \dif z = - \int_\gamma f \dif z.
        \]
        \[
            \int_\gamma f |\dif z| = \int_{-\gamma} f |\dif z|.
        \]

    \item Неравенство треугольника:
        \[
            \left| \int_\gamma f \dif z \right| \le \int_\gamma |f| \cdot |\dif z| \le \max_{z \in \gamma} |f| \int_\gamma |\dif z| = \max_{z \in \gamma} |f| \cdot l(\gamma).
        \]
\end{enumerate}

\subsection{Полные дифференциалы}

\textbf{Определение.} Функция $F$, голоморфная в области $D \subset \mathbb C$, называется \textit{первообразной функции $f$}, если $F'(z) = f(z)$ для всех $z \in D$, то есть если $f \dif z = \dif F$.
В этом случае $f$ называется \textit{полным дифференциалом}.

\textbf{Теорема.} (Критерий полного дифференциала) Пусть $f$ непрерывна в области $D$.
Следующие утверждения эквивалентны:
\begin{enumerate}
    \item $f$ является полным дифференциалом.
    \item Для любой кусочно гладкой замкнутой кривой в $D$ выполнено $\int_\gamma f \dif z = 0$.
    \item Для любой замкнутой ломаной $\gamma$ в $D$ выполнено $\int_\gamma f \dif z = 0$.
\end{enumerate}

\textbf{Доказательство.} $1 \Rightarrow 2$. Напишем интеграл по произвольной кривой:
\[
    \int_\gamma f \dif z = \int_\alpha^\beta f(z(t)) z'(t) \dif t = \int_\alpha^\beta \frac{\dif}{\dif z} F(z(t)) z'(t) \dif t =
\]
\[
    \int_\alpha^\beta \frac{\dif}{\dif t} F(z(t)) \dif t = F(z(\beta)) - F(z(\alpha)) = 0,
\]
так как кривая замкнута.

$2 \Rightarrow 3$ очевидно. $3 \Rightarrow 1$.
Рассмотрим точку $a \in D$.
По связности для любой $z \in D$ существует ломаная $\gamma_z$, полностью лежащая в $D$ и соединяющая $a$ с $z$.
Положим
\[
    F(z) := \int_{\gamma_z} f(\zeta) \dif \zeta.
\]
Докажем, что это и есть искомая первообразная.
Рассмотрим $z_0 \in D$ и зафиксируем $\varepsilon > 0$.
Из открытости найдётся $\delta > 0$, такое что $O_\delta(z_0) \subset D$.
Дополнительно по непрерывности $f$ уменьшим $\delta$ так, чтобы $|f(z) - f(z_0)| < \varepsilon$ при $z \in O_\delta(z_0)$.
Теперь зафиксируем $z \in \dot O_\delta(z_0)$.
Тогда
\[
    F(z) - F(z_0) = \left( \int_{\gamma_z} - \int_{\gamma_{z_0}} \right) f(\zeta) \dif \zeta = \int_{[z_0, z]} f(\zeta) \dif \zeta.
\]
Последний переход получен из соображения, что разность двух интегралов --- это интеграл по какой-то ломаной, соединяющей $z_0$ и $z$.
Так как они обе лежат в $O_\delta(z_0) \subset D$, в качестве ломаной можно просто взять отрезок, их соединяющий, ибо по условию интегралы по всем таким ломаным равны.

Ненадолго забудем равенство выше, заметим, что
\[
    \int_{[z_0, z]} f(z_0) \dif \zeta = f(z_0) \int_{[z_0, z]} \dif \zeta = f(z_0)(z - z_0).
\]
Теперь мы хотим доказать, что у $F$ будет правильная производная, для этого оценим разность:
\[
    \frac{F(z) - F(z_0)}{z - z_0} - f(z_0) = \frac{\int_{[z_0, z]} (f(\zeta) - f(z_0)) \dif \zeta}{z - z_0} \le
\]
\[
    \le \frac{\int_{[z_0, z]} |f(\zeta) - f(z_0)| \cdot |\dif \zeta|}{|z - z_0|} \le \frac{\max_{\zeta \in [z_0, z]} |f(\zeta) - f(z_0)| \cdot |z - z_0|}{|z - z_0|} \le \varepsilon.
\]

\QED

\textbf{Примеры.}
\begin{enumerate}
    \item $f(z) = (z - a)^n$, где $a \in \mathbb C$ и $n \in \mathbb Z \setminus \{-1\}$.
        Тогда $F(z) = \frac{(z - a)^{n+1}}{n+1}$, а интеграл $\int_\gamma (z - a)^n \dif z = 0$ по любой замкнутой кривой $\gamma$.

    \item $f(z) = \frac{1}{z - b}$. Рассмотрим кусочно гладкую кривую $\gamma$, такую что для какого-то $\theta_0$ выполнено $\gamma \cap \{b + te^{i \theta_0}, t \ge 0\} = \varnothing$.
        Тогда на множестве $\mathbb C$ без луча будет $F(z) = \ln(z - b)$, и вновь интеграл по кривой равен нулю.

    \item $f(z) = \int_{|z - b| = \rho} \frac{\dif z}{z - b}$.
        Здесь можно параметризовать $z = b + \rho e^{it}$ для $t \in [0, 2\pi]$, после чего переписать интеграл:
        \[
            \int_0^{2\pi} \frac{\rho \cdot i \cdot e^{it} \dif t}{\rho e^{it}} = i \int_0^{2\pi} \dif t = 2\pi i \ne 0.
        \]
        Следовательно, у данной функции не существует первообразной ни в каком кольце $\{z: r < |z - b| < R\}$, ибо мы нашли кривую, по которой интеграл не зануляется.
\end{enumerate}

\textbf{Теорема.} (Лемма Гурсá) Пусть $f$ голоморфна в области $D$, и в этой области есть треугольник $T_0$ (открытое множество, ограниченное тремя прямыми), замыкание которого лежит в $D$.
Тогда $\int_{\partial T_0} f(z) \dif z = 0$.

\textbf{Доказательство.} Будем фактически доказывать от противного, но формально будет напрямую.
Обозначим $I(T) = \int_{\partial T} f \dif z$.
Возьмём треугольник и разобьём его на четыре треугольника следующим образом, выбирая середины сторон (рисунок 2).
\begin{figure}[ht]
    \centering
    \incfig{triangles}{0.25\linewidth}
    \caption{Разбиение треугольника}
\end{figure}

Обзовём четыре новых треугольника, как $T_1, T_2, T_3, T_4$.
Тогда $I(T_0) = I(T_1) + I(T_2) + I(T_3) + I(T_4)$.
Следовательно, найдётся треугольник $T_j$, такой что $|I(T_j)| \ge \frac{|I(T_{j-1})|}{4}$.
Будем продолжать такое построение, получим последовательность треугольников $T_0, T_1, \dots$ (забудем про обозначение четырёх треугольников выше).

Пусть $z_0, z_1, \dots$ --- последовательность центров построенных треугольников, $z^*$ --- её предел.
Теперь по определению дифференцируемости для любого $\varepsilon > 0$ найдётся $\delta > 0$, такое что для всех $z \in O_\delta(z^*) \subset D$ будет выполняться
\[
    |f(z) - f(z^*) - f'(z^*)(z - z^*)| \le \varepsilon |z - z^*|.
\]
Заметим, что функции $f(z^*) \dif z$, $f'(z^*) (z - z^*) \dif z$ являются полными дифференциалами, ибо это константа и линейная функция, и для них это было доказано в примерах выше.
Тогда их можно ``вставить`` по линейности, ибо их интеграл равен нулю:
\[
    \left| \int_{\partial T_n} f(z) \dif z \right| = \left| \int_{\partial T_n} \left( f(z) - f(z^*) - f'(z^*) (z - z^*) \right) \dif z \right|
\]
Теперь это не превосходит
\[
    \varepsilon \int_{\partial T_n} |z - z^*| \cdot |\dif z| \le \varepsilon \cdot \frac{l}{2^n} \cdot \frac{l}{2^n} = \frac{\varepsilon l^2}{4^n}.
\]
Первая оценка оценивает расстояние от центра до точки на границе, которое мало, вторая --- периметр треугольника, тоже мал.
Вспоминая, что по построению мы брали $T_n$ так, чтобы $|I(T_n)| \ge \frac{|I(T_{n-1})|}{4}$, получаем оценку $|I(T_n)| \ge \frac{|I(T_0)|}{4^n}$, или же $|I(T_n)| \cdot 4^n \ge |I(T_0)|$.
Величина слева стремится к нулю, значит, величина справа равна равна нулю.

\QED

\textbf{Замечание.} В условии леммы Гурса голоморфность $f$ можно ослабить до непрерывности в $D$ и голоморфности в $D \setminus \{a\}$.

\textbf{Доказательство.} Действительно, возможны 4 случая:
\begin{itemize}
    \item Точка $a$ не попала в треугольник. Тогда доказательство никак не пострадало.
    \item Точка $a$ совпала с одной из вершин.
        Построим маленький подобный треугольник внутри исходного следующим образом:

        \begin{figure}[ht]
            \centering
            \incfig{triangle-congruent}{0.4\linewidth}
            \caption{Построение маленького треугольника}
        \end{figure}

        Пусть $\lambda$ --- коэффиециент подобия, $T'$ --- маленький треугольник.
        Триангулируя трапецию $z_1 z_2 z_2' z_1'$, получаем, что интеграл по ней равен нулю.
        Следовательно, интеграл по всему треугольнику равен интегралу по маленькому, а он равен
        \[
            \left| \int_{\partial T'} f \dif z \right| \le \max_{\partial T} |f| \cdot \lambda l,
        \]
        где $l$ --- периметр исходного треугольника.
        Устремляя $\lambda$ к нулю, получаем, что интеграл стремится к нулю, значит, и интеграл по исходному треугольнику равен нулю.

    \item Лежит на стороне.
        Проведём отрезок из точки $a$ в противоположную вершину треугольника и получим предыдущий случай для двух новых треугольников.
    \item Лежит внутри. Теперь проведём три отрезка в вершины треугольника и снова получим второй случай.
\end{itemize}

\QED

\textbf{Лемма.} (Достаточное условие полного дифференциала) Пусть $f$ непрерывна в выпуклой области $D$, и для любого треугольника $T$, такого что $\overline T \subset D$, выполнено $\int_{\partial T} f \dif z = 0$.
Тогда $f \dif z$ является полным дифференциалом.

\textbf{Доказательство.} Зафиксируем $a \in D$ и рассмотрим $z, z^* \in D$.
Пусть $T$ --- треугольник с вершинами в $a$, $z$, $z^*$, с ориентацией треугольника в таком порядке.
Тогда
\[
    \int_{\partial T} f \dif \zeta = \left( \int_{[a, z]} + \int_{[z, z^*]} - \int_{[a, z^*]} \right) f \dif \zeta.
\]

Рассмотрим функцию $F(z) = \int_{[a, z]} f \dif \zeta$.
Тогда выражение выше переписывается в виде
\[
    F(z) - F(z^*) = \int_{[z^*, z]} f \dif \zeta.
\]
Докажем, что это и есть первообразная $f$, для этого нужно доказать, что для любого $\varepsilon > 0$ выполнено
\[
    \left| \frac{F(z) - F(z^*)}{z - z^*} - f(z^*) \right| < \varepsilon.
\]
У внимательного читателя должно возникнуть дежавю: это уже было в доказательстве критерия полного дифференциала.
Здесь доказательство аналогично, засим оно пропущено.

\QED

\textbf{Замечание.} Выпуклость здесь нужна для того, чтобы убедиться, что треугольник на любых трёх точках полностью лежит внутри области.

\textbf{Следствие.} (Теорема Коши для выпуклой области)
Пусть $D$ --- выпуклая область в $\mathbb C$, $f$ непрерывна в $D$ и голоморфна в $D \setminus \{a\}$, $\gamma$ --- кусочно гладкая замкнутая кривая в $D$.
Тогда $\int_\gamma f \dif z = 0$.

\textbf{Доказательство.} Рекомендуется доказать в качестве упражнения, но на всякий случай: по лемме Гурса интеграл по контуру любого треугольника равен нулю, по достаточному условию полного дифференциала $f \dif z$ является полным дифференциалом, по критерию полного дифференциала получаем искомое.

\QED

\textbf{Следствие.} Пусть $f$ голоморфна в $O_r(z_0)$.
Тогда у неё есть первообразная в этой области.

\subsection{Интеграл Коши}
\textbf{Теорема.} (О свойствах интеграла Коши)
Пусть $\gamma$ --- кусочно гладкая кривая в $\mathbb C$, $\phi$ непрерывна на $\gamma$, $z \not\in \gamma$, $n \in \mathbb Z_{>0}$.
Положим
\[
    F_n(z, \phi) := \int_\gamma \frac{\phi(\zeta) \dif \zeta}{(\zeta - z)^n}.
\]
Тогда:
\begin{enumerate}
    \item $F_n(z, \phi)$ голоморфна в $\mathbb C \setminus \gamma$.
    \item $F_n'(z, \phi) = nF_{n+1}(z, \phi)$. Производная берётся по $z$.
        Иными словами, дифференциал заносится под интеграл.
\end{enumerate}

\textbf{Доказательство.} Важное тождество для ядер Коши:
\[
    \frac{1}{\zeta - z} - \frac{1}{\zeta - z_0} = \frac{z - z_0}{(\zeta - z)(\zeta - z_0)}.
\]

Теперь докажем по индукции. При $n = 1$:
\[
    F_1(z, \phi) - F_1(z_0, \phi) = (z - z_0) F_1 \left(z, \frac{\phi(\zeta)}{\zeta - z_0} \right).
\]
Последний переход получен по Важному тождеству.

Покажем, что $F_1$ непрерывна в $z_0 \in \mathbb C \setminus \gamma$.
Для этого оценим второй множитель справа: берём $\delta = \frac{\rho(z_0, \gamma)}{2} > 0$, $z \in \dot O_\delta(z_0)$.
Тогда
\[
    \left| F_1 \left(z, \frac{\phi(\zeta)}{\zeta - z_0)} \right) \right| = \left| \int_\gamma \frac{\phi(\zeta) \dif \zeta}{(\zeta - z)(\zeta - z_0)} \right| \le \frac{1}{2 \delta \cdot \delta} \int_\gamma |\phi(\zeta) \dif \zeta|,
\]
где $|\zeta - z_0| \ge \rho(z_0, \gamma) = 2\delta$ по построению, а $|\zeta - z| \ge |\zeta - z_0| - |z_0 - z| \ge 2\delta - \delta$.
Подставляя в исходную формулу, получаем
\[
    |F_1(z) - F_1(z_0)| \le |z - z_0| \cdot \frac{\int_\gamma |\phi \dif \zeta|}{2 \delta^2}.
\]
Сужая окрестность, из которой мы берём $z$, выражение справа можно сделать сколь угодно малым, что доказывает непрерывность.

Теперь проверим дифференцируемость:
\[
    \frac{F_1(z, \phi) - F_1(z_0, \phi)}{z - z_0} = F_1 \left(z, \frac{\phi(\zeta)}{\zeta - z_0} \right).
\]
Устремляя $z \to z_0$, получаем слева по определению $F_1'(z_0, \phi)$, а справа ---
\[
    \int_\gamma \frac{\phi(\zeta)}{(\zeta - z_0)(\zeta - z)} \dif \zeta \xrightarrow{z \to z_0} \int_\gamma \frac{\phi(\zeta)}{(\zeta - z_0)^2} \dif \zeta = F_2(z_0, \phi).
\]

Теперь переход от $n - 1$ к $n$.
Снова посмотрим на ядра Коши:
\[
    \frac{1}{(\zeta - z)^n} - \frac{1}{(\zeta - z_0)^n} = 
\]
\[
    = \frac{1}{(\zeta - z)^n} - \frac{1}{(\zeta - z)^{n-1}(\zeta - z_0)} + \frac{1}{(\zeta - z)^{n-1}(\zeta - z_0)} - \frac{1}{(\zeta - z_0)^n} =
\]
(вынесем общий множитель из первых двух и вторых двух слагаемых)
\[
    = \frac{1}{(\zeta - z)^n} \cdot \frac{z - z_0}{\zeta - z_0} + \frac{1}{\zeta - z_0} \left( \frac{1}{(\zeta - z)^{n-1}} - \frac{1}{(\zeta - z_0)^{n-1}} \right).
\]
Следовательно, подставляя в получающийся интеграл эту формулу,
\[
    F_n(z, \phi) - F_n(z_0, \phi) = 
\]
\[
    = (z - z_0) F_n \left(z, \frac{\phi}{\zeta - z_0} \right) + F_{n-1} \left(z, \frac{\phi(\zeta)}{\zeta - z_0} \right) - F_{n-1} \left(z_0, \frac{\phi(\zeta)}{\zeta - z_0} \right).
\]
По предположению индукции последние два слагаемых непрерывны, а непрерывность первого доказывается аналогично случаю $n = 1$.
Теперь докажем дифференцируемость:
\[
    \frac{F_n(z, \phi) - F_n(z_0, \phi)}{z - z_0} =
\]
\[
    = F_n \left(z, \frac{\phi}{\zeta - z_0} \right) + \frac{F_{n-1} \left(z, \frac{\phi(\zeta)}{\zeta - z_0} \right) - F_{n-1} \left(z_0, \frac{\phi(\zeta)}{\zeta - z_0} \right)}{z - z_0}.
\]
При $z \to z_0$ первое слагаемое стремится по непрерывности, а второе --- по дифференцируемости из предположения индукции:
\[
    \xrightarrow{z \to z_0} F_n \left(z_0, \frac{\phi}{\zeta - z_0} \right) + F_{n-1}' \left(z_0, \frac{\phi}{\zeta - z_0} \right) =
\]
\[
    = F_n \left(z_0, \frac{\phi}{\zeta - z_0} \right) + (n - 1) F_n \left( z_0, \frac{\phi}{\zeta - z_0} \right) = n F_n \left( z_0, \frac{\phi}{\zeta - z_0} \right) = n F_{n+1}(z_0, \phi).
\]
Эти равенства берутся из определения $F_n$, достаточно внимательно посмотреть или расписать явно.

\QED

\textbf{Следствие 1.} Пусть $\gamma$ --- кусочно гладкая кривая в $\mathbb C$, $\phi \in C(\gamma)$,
\[
    f(z) = \frac{1}{2\pi} \int_\gamma \frac{\phi(\zeta) \dif \zeta}{\zeta - z}.
\]
Тогда $f$ бесконечно дифференцируема в $\mathbb C \setminus \gamma$ и
\[
    f^{(n)}(z) = \frac{n!}{2\pi} \int_\gamma \frac{\phi(\zeta) \dif \zeta}{(\zeta - z)^{n+1}}.
\]
Следует напрямую многократным применением теоремы.

\textbf{Следствие 2.} Пусть $r > 0$. А чему равно
\[
    h(z) = \int_{|\zeta| = r} \frac{\dif \zeta}{\zeta - z}?
\]

\textbf{Доказательство.} По теореме $h$ голоморфна в $\mathbb C \setminus \{|z| = r\}$.
Ещё можно посчитать, что $h(0) = 2\pi i$.
Также
\[
    h'(z) = \int_\gamma \frac{\dif \zeta}{(\zeta - z)^2} = 0,
\]
так как это полный дифференциал.
Отсюда следует, что в круге $\{|z| < r\}$ функция является константой, значит, равна $h(0) = 2\pi i$.

Снаружи окружности можно заметить, что сама функция является интегралом полного дифференциала, а именно, $\frac{\dif}{\dif \zeta} \ln(\zeta - z)$, то есть тождественно равна нулю.

\QED

\textbf{Теорема.} (Формула Коши в круге) Пусть $f$ голоморфна в области $D$, $\overline {O_r(a)} \subset D$.
Положим $\gamma_r(a) = \partial O_r(a)$ --- окружность радиуса $r$, ориентированная против часовой стрелки.
Тогда для любого $z \in O_r(a)$ выполнено
\[
    f(z) = \frac{1}{2\pi i} \int_{\gamma_r(a)} \frac{f(\zeta) \dif \zeta}{\zeta - z}.
\]
Иными словами, любая голоморфная функция является интегралом Коши, и она восстанавливаются по значениям на окружности.

\textbf{Доказательство.}
Зафиксируем $z \in O_r(a)$, положим
\[
    g(\zeta) = 
    \begin{cases}
        \frac{f(\zeta) - f(z)}{\zeta - z}, & \zeta \in D \setminus \{z\} \\
        f'(z) & \zeta = z
    \end{cases} .
\]
Как всегда, определение именно такое, чтобы получить непрерывность.
Также она голоморфна в $D \setminus \{z\}$, как отношение двух голоморфных функций.

Итак, эта функция удовлетворяет условиям теоремы Коши для выпуклой области.
Но в какой области? Возьмём из открытости $D$ число $\delta > 0$, такое что $O_{r + \delta}(a) \subset D$, и на ней применим теорему, то есть
\[
    \int_{\gamma_r(a)} g(\zeta) \dif \zeta = 0.
\]
Подставим определение $g(\cdot)$:
\[
    \int_{\gamma_r(a)} \frac{f(\zeta) \dif \zeta}{\zeta - z} = f(z) \int_{\gamma_r(a)} \frac{\dif \zeta}{\zeta - z}.
\]
А интеграл справа мы умеем считать:
\[
    \int_{\gamma_r(a)} \frac{f(\zeta) \dif \zeta}{\zeta - z} = 2\pi i \cdot f(z).
\]

\QED

\textbf{Следствие.} Если $f$ голоморфна в области $D$, то она бесконечно дифференцируема в области $D$, и для любого $z \in O_r(a)$ выполнено
\[
    f^{(n)}(z) = \frac{n!}{2\pi i} \int_{\gamma_r(a)} \frac{f(\zeta) \dif \zeta}{(\zeta - z)^{n+1}}
\]
Для запоминания этой формулы рекомендуется взять тождество из формулы Коши в круге и внести под интеграл оператор дифференцирования.

Ранее мы доказывали достаточное условие существования голоморфной первообразной.
Теперь мы доказали, что если у функции есть голоморфная производная, то она голоморфна и бесконечно дифференцируема.
Если объединить эти два результата вместе, то получится

\textbf{Теорема.} (Морера) Пусть $f$ непрерывна в области $D$ и для любого замкнутого треугольника $\overline \triangle \subset D$ выполнено $\int_{\partial \triangle} f(z) \dif z = 0$.
Тогда $f$ голоморфна в $D$.

\textbf{Доказательство.} Итак, в той лемме была одна тонкость: она работает только в выпуклых областях.
Засим рассмотрим точку $z_0 \in D$, окрестность $O_r(z_0) \subset D$, в этом шаре применяем лемму.
Получаем, что $f \dif z$ --- полный дифференциал в шаре, то есть $f = F'(z)$ в шаре.
Следовательно, $f$ голоморфна в шаре, в точке $z_0$ и на всей области $D$.

\QED

\textbf{Теорема.} (О среднем) Пусть $f$ голоморфна в области $D$, содержащей круг $\overline{O_r(a)}$.
Тогда её значение в центре круга восстанавливается по значениям на окружности, то есть
\[
    f(a) = \frac{1}{2\pi} \int_0^{2\pi} f(a + re^{it}) \dif t.
\]

\textbf{Доказательство.} Действительно, по формуле Коши в круге
\[
    f(a) = \frac{1}{2\pi i} \int_{\gamma_r(a)} \frac{f(\zeta) \dif \zeta}{\zeta - a} =
\]
(параметризуем окружность: $\zeta = a + re^{it}$, $\dif \zeta = ri e^{it}$)
\[
    = \frac{1}{2\pi i} \int_0^{2 \pi} \frac{f(a + re^{it}) ri e^{it} \dif t}{re^{it}}.
\]

\QED

\subsection{Ряды Тейлора}
\textbf{Теорема.} Пусть функция $f$ голоморфна в области $D$, содержащей круг $O_R(z_0)$.
Тогда для любого $z \in O_R(z_0)$ выполнено
\[
    f(z) = \sum_{n=0}^{\infty} c_n(z - z_0)^n,
\]
где
\[
    c_n = \frac{f^{(n)}(z_0)}{n!} = \frac{1}{2\pi i} \int_{\gamma_r(z_0)} \frac{f(\zeta) \dif \zeta}{(\zeta - z_0)^{n+1}}
\]
для любого $r < R$.
Ура, любая голоморфная функция раскладывается в ряд Тейлора.

\textbf{Доказательство.} Зафиксируем $z \in O_R(z_0)$ и $r \in (0, R)$.
Теперь можно считать, что $z \in O_r(z_0)$, иначе увеличим $r$.
Снова поиграемся с ядрами Коши для $\zeta \in \gamma_r(z_0)$:
\[
    \frac{1}{\zeta - z} = \frac{1}{\zeta - z_0 - (z - z_0)} = \frac{1}{\zeta - z_0} \cdot \frac{1}{1 - \frac{z - z_0}{\zeta - z_0}} =
\]
Теперь заметим, что это можно разложить в ряд Тейлора, ибо $|z - z_0| < r$:
\[
    = \sum_{n=0}^{\infty} \frac{(z - z_0)^n}{(\zeta - z_0)^{n+1}},
\]
более того, сходимость равномерная.

Умножим сиё тождество на $f(\zeta)$ и проинтегрируем по $\zeta$.
Корректность интегрирования следует из равномерной сходимости и теоремы Вейерштрасса для вещественных чисел, ибо комплексный интеграл можно разбить на два вещественных.
То есть это равно
\[
    \int_{\gamma_r(z_0)} \frac{f(\zeta) \dif \zeta}{\zeta - z} = \int_{\gamma_r(z_0)} \sum_{n=0}^{\infty} \frac{f(\zeta) (z - z_0)^n \dif \zeta}{(\zeta - z_0)^{n+1}} = \sum_{n=0}^{\infty} \int_{\gamma_r(z_0)} \frac{f(\zeta) \dif \zeta}{(\zeta - z_0)^{n+1}} (z - z_0)^n.
\]

\QED

Данную теорему можно использовать вместо теоремы Коши--Адамара для нахождения радиуса сходимости ряда.

\textbf{Пример.} $f(z) = \frac{1}{\cos(z)}$.
Внутри круга радиуса $\frac{\pi}{2}$ эта функция голоморфна, то есть радиус сходимости --- хотя бы $\frac{\pi}{2}$.
Но в $\frac{\pi}{2}$ она разрывна, так что это и есть искомый радиус сходимости.

\textbf{Пример.} Почему у функции $\arctan(z)$ такой маленький радиус сходимости?
Можно продифференцировать и получить $\frac{1}{1 + z^2}$, то есть она разрывна в точке $i$.

\textbf{Следствие.} (Неравенство Коши для коэффициентов $c_n$) Пусть $f$ голоморфна в $O_r(z_0)$.
Обозначим $M_f(\rho) = \max_{|z - z_0| = \rho} |f(z)|$.
Тогда для любого $\rho \in (0, r)$ выполнено
\[
    |c_n| = \left| \frac{f^{(n)}(z_0)}{n!} \right| \le \frac{M_f(\rho)}{\rho^n}.
\]

\textbf{Доказательство.} Тривиальным образом воспользуемся интегральной формулой для $c_n$:
\[
    |c_n| = \frac{1}{2\pi} \left| \int_{\gamma_\rho} \frac{f(\zeta) \dif \zeta}{(\zeta - z_0)^{n+1}} \right| \le \frac{1}{2\pi} \int_{\gamma_\rho} \frac{|f(\zeta)| \cdot |\dif \zeta|}{|\zeta - z_0|^{n+1}} \le
\]
\[
    \le \frac{M_f(\rho)}{2\pi \rho^{n+1}} \cdot \int_{\gamma_\rho} |\dif \zeta| = \frac{M_f(\rho)}{\rho^n},
\]
так как интеграл равен длине окружности радиуса $\rho$.

\QED

\textbf{Определение.} Функция $f$ нызвается \textit{целой}, если она голоморфна в $\mathbb C$.

\textbf{Следствие.} (Теорема Лиувилля) Пусть $f$ целая и существуют $m \in \mathbb Z_{\ge 0}$, $R, A \in \mathbb R$, такие что для всех $z$, таких что $|z| > R$, выполнено $|f(z)| \le A \cdot |z|^m$.
Тогда $f$ является многочленом степени не более $m$.

\textbf{Доказательство.} Раз функция целая, её можно разложить в степенной ряд с бесконечным радиусом сходимости в любой точке.
Разложим в нуле: $f(z) = \sum_{n=0}^{\infty} c_n z^n$.
Теперь напишем неравенство Коши для коэффициентов: для $\rho > R$ выполнено
\[
    |c_n| \le \frac{M_f(\rho)}{\rho^n} \le \frac{A}{\rho^{n-m}}.
\]
Остаётся устремить $\rho$ в бесконечность.
Для $n > m$ коэффициенты оказываются равны нулю, а про остальные неизвестно.
Следовательно, разложение в степенной ряд имеет вид $f(z) = \sum_{n=0}^{m} c_n z^n$.

\QED

\textbf{Следствие.} Если $f$ целая и ограниченная, то она равна константе.
Это следствие тоже называют теоремой Лиувилля, но формально это не она.

\textbf{Определение.} Пусть $f$ голоморфна в области $D$.
Точка $a \in D$ называется \textit{нулём} функции $f$, если $f(a) = 0$.

\textbf{Определение.} Пусть $f$ голоморфна в области $D$, точка $a \in D$ --- нуль функции $f$.
Она называется \textit{нулём порядка $m$}, если $f(a), f'(a), \dots, f^{m-1}(a) = 0$ и $f^{(m)}(a) \ne 0$.

Точка $a$ называется \textit{нулём бесконечного порядка}, если цепочка нулей производных не обрывается.

\textbf{Замечание.} Если $O_r(a) \subset D$ и $a$ является нулём бесконечного порядка, то $f|_{O_r(a)} \equiv 0$.
Действительно, разложим в ряд Тейлора в точке $a$.

Если же $a$ является нулём порядка $m$, то в $O_r(a)$ функция равна
\[
    f(z) = \sum_{n=m}^{\infty} c_n (z - a)^n = (z - a)^m \underbrace{\sum_{n=0}^{\infty} c_{n+m} (z - a)^n}_{\phi(z)}.
\]
Заметим, что $\phi(\cdot)$ голоморфна в $O_r(a)$ и $\phi(a) = c_m \ne 0$.
Тогда найдётся окрестность точки $a$, в которой $\phi$ не обращается в ноль.

\textbf{Пример.}
\[
    f(x) =
    \begin{cases}
        e^{-\frac{1}{x^2}}, & x \ne 0 \\
        0, & x = 0
    \end{cases}
\]
У неё ноль является нулём бесконечного порядка, но при этом она не является тождественным нулём.
Проблема в том, что она хоть и дифференцируема на действительных числах, но голоморфной не является.

\textbf{Лемма.} (О неизолированном нуле) Пусть $f$ голоморфна в области $D$, а также существует последовательность $\{z_n\} \subset D$, такая что $z_n \to z_0 \in D$ и все $f(z_n) = 0$.
Тогда $z_0$ --- ноль бесконечного порядка.

\textbf{Доказательство.} От противного: пусть $m$ --- порядок нуля $f$.
Тогда $f = (z - z_0)^m g(z)$, где $g$ голоморфна на $D$ и $g(z_0) \ne 0$.
В силу непрерывности $g$ найдётся окрестность $O_\delta(z_0)$, в которой $|g| > 0$.

Теперь вспомним, что у нас была последовательность $z_n \to z_0$.
Найдётся $n$, такое что $z_n \in O_\delta(z_0)$, тогда по неравенству выше
\[
    |f(z_n)| = |z_n - z_0|^m |g(z_n)| > 0,
\]
противоречие.

\QED

\textbf{Теорема.} (О единственности) Пусть функции $f$ и $g$ голоморфны в области $D$, а $E \subset D$, причём хотя бы одна предельная точка $E$ лежит в $D$, обозначим её за $a$.
Если $f \equiv g$ на $E$, то $f \equiv g$ на $D$.

\textbf{Доказательство.} Рассмотрим функцию $h(z) = f(z) - g(z)$ --- тождественный ноль на $E$, голоморфна в $D$.
Положим $H$ --- множество нулей функции $h$, $G_1 = \Int(H)$, $G_2 = D \setminus G_1$.
Тогда $G_1$ открыто, но и $G_2$ --- тоже. Докажем от противного: найдётся $z_0 \in G_2$, такое что для всех $\delta$ выполнено $O_\delta(z_0) \not\subset G_2$.
Беря $\delta$ достаточно малым, получим, что $O_\delta(z_0) \subset D$.

Следовательно, найдётся $\{z_n\} \subset G_1$, такая что $z_n \to z_0$, то есть $z_0$ является неизолированным нулём функции $h$, а значит, это ноль бесконечного порядка.
Из голоморфности получаем, что $h \equiv 0$ в некоторой окрестности $z_0$ --- противоречие.

Получаем, что $D = G_1 \cup G_2$, причём $G_1$ пусто или $G_2$ пусто.
Но мы знаем, что $a \in G_1$: по условию найдётся последовательность $\{z_n\} \subset E$, такая что $z_n \to a$, поэтому $a$ является неизолированным нулём.
То есть $G_2$ пусто, а значит, $D = G_1$.

\QED

\section{Индекс и приращение аргумента}
\subsection{Определение и свойства}
Сейчас будет длинное обоснование корректности определения.

Пусть $\gamma$ --- кусочно гладкая кривая в $\mathbb C \setminus \{0\}$, $z: [\alpha, \beta] \to \mathbb C \setminus \{0\}$ --- кусочно гладкая функция, такая что
\[
    \int_\gamma \frac{\dif z}{z} = \int_\alpha^\beta \frac{z'(t) \dif t}{z(t)}.
\]
Положим $d = \rho(\gamma, 0) > 0$.
Так как $z(\cdot)$ равномерно непрерывна на $[\alpha, \beta]$, найдётся $\delta > 0$, такое что для всех $t', t''$ выполнено $|t' - t''| < \delta \Rightarrow |z(t') - z(t'')| < \frac{d}{2}$.
Разобьём отрезок $[\alpha, \beta]$: $\alpha = t_0 < t_1 < \dots < t_n = \beta$ так, что для всех $k$ выполнено $0 < t_k - t_{k-1} < \delta$, возьмём сужения $\gamma_k: [t_{k-1}, t_k] \to \mathbb C \setminus \{0\}$ кривой на подотрезки, то есть $\gamma = \sum_{k=1}^{n} \gamma_k$.

Из равномерной непрерывности мы знаем, что $\gamma_k$ лежит в $O_{\frac{d}{2}}(z(t_k))$.
На этом моменте можно посмотреть на рисунок 4, чтобы лучше понять конструкцию.

\begin{figure}[ht]
    \centering
    \incfig{integral}{0.5\linewidth}
    \caption{Геометрический смысл интеграла}
\end{figure}

Заметим, что теперь существует непрерывная ветвь $\arg_k(z)$ в $O_{\frac{d}{2}}(z(t_k))$ и непрерывная ветвь $\ln_k(z) = \ln|z| + i \cdot \arg_k(z)$.
Итак, интеграл записывается в виде
\[
    \int_\gamma \frac{\dif z}{z} = \sum_{k=1}^{n} \int_{\gamma_k} \frac{\dif z}{z} = \sum_{k=1}^{n} \int_{\gamma_k} \dif \ln_k(z) = \sum_{k=1}^{n} \left( \ln_k(z_k) - \ln_k(z_{k-1}) \right) =
\]
\[
    = \sum_{k=1}^{n} \ln \left| \frac{z_k}{z_{k-1}} \right| + i \cdot \sum_{k=1}^{n} (\arg_k(z_k) - \arg_k(z_{k-1})) =
\]
\[
    = \ln \left| \frac{z(\beta)}{z(\alpha)} \right| + i \cdot \sum_{k=1}^{n} \arg \left( \frac{z_k}{z_{k-1}} \right).
\]
В переходе аргументов мы воспользовались тем, что константа функции $\arg_k$ сокращается в разности, поэтому в частном можно перейти просто к $\arg$.
Заметим, что по построению $\arg$ принимает значения в $(-\pi, \pi)$, так как шары, в которых мы брали разности, достаточно маленькие.

Итак, положим 
\[
    \Delta_\gamma \arg(z) := \im \int_\gamma \frac{\dif z}{z},
\]
то есть сумма аргументов в формуле выше.
Если $\gamma$ замкнута, то
\[
    \Delta_\gamma \arg(z) = \frac{1}{i} \int_\gamma \frac{\dif z}{z} = \im \int_\gamma \frac{\dif z}{z}.
\]
Также в этом случае $\Delta_\gamma \arg(z)$ кратно $2\pi$, так как $\sum_{k=1}^{n} (\arg_k(z_k) - \arg_k(z_{k-1}))$ является телескопической суммой по модулю $2\pi$.

\textbf{Определение.} Пусть $D$ --- область в $\mathbb C$, $f$ голоморфна в $D$, $\gamma$ --- кусочно гладкая кривая в $D$, $f \ne 0$ на $\gamma$.
Положим $\Gamma = f(\gamma)$, тогда это будет кусочно гладкой кривой на $[\alpha, \beta]$.
Получаем
\[
    \Delta_\gamma \arg(f) := \Delta_\Gamma \arg(w) = \im \int_\Gamma \frac{\dif w}{w} = \im \int_\alpha^\beta \frac{f'(z(t)) z'(t) \dif t}{f(z(t))} = \im \int_\gamma \frac{f' \dif z}{f},
\]
и вот это называется \textit{приращением аргумента}.

\textbf{Свойства.}
\begin{enumerate}
    \item $\Delta_\gamma \arg(f_1 f_2) = \Delta_\gamma \arg(f_1) + \Delta_\gamma \arg(f_2)$.
        Следует из того, что
        \[
            \frac{(f_1 f_2)'}{f_1f_2} = \frac{f_1'}{f_1} + \frac{f_2'}{f_2}.
        \]

    \item $\Delta_{-\gamma} \arg(f) = \Delta_\gamma \arg \left( \frac{1}{f} \right) = -\Delta_\gamma \arg(f)$.
\end{enumerate}

\textbf{Определение.} Пусть $\gamma$ --- замкнутая кусочно гладкая кривая, $a \not\in \gamma$.
\textit{Индексом точки $a$} называется $J_\gamma(a) = \frac{1}{2\pi} \Delta_\gamma \arg(z - a)$, то есть количество оборотов, которое делает кривая вокруг.

Альтернативное определение, $\frac{1}{2\pi i} \int_\gamma \frac{\dif z}{z - a}$, эквивалентно предыдущему.

Иными словами, для кусочно гладкой кривой $\gamma \subset \mathbb C \setminus \{0\}$ существует параметризация $z: [\alpha, \beta] \to \mathbb C$.
Но ещё можно ввести функцию $\phi: [\alpha, \beta] \to \mathbb R$, такую что она непрерывна и для любого $t \in [\alpha, \beta]$ выполнено $\phi(t) \in \Arg(z(t))$.

Однако функция $\phi$ определена неоднозначно.
Более того, для любых $\phi_1, \phi_2$, удовлетворяющих этим двум свойствам, найдётся $k$, такое что $\phi_2 - \phi_1 = 2\pi k$.
Справа налево очевидно, а обратно по второму свойству.
Соответственно мы определяем $\Delta_\gamma \arg(z) = \phi(\beta) - \phi(\alpha)$, и здесь уже однозначности нет, и равно это
\[
    \im \int_\gamma \frac{\dif z}{z} = \frac{1}{i} \int_\gamma \frac{\dif z}{z},
\]
а индекс определяется похожим образом, но для произвольной точки, то есть
\[
    J_\gamma(a) = \frac{1}{2\pi i} \int_\gamma \frac{\dif z}{z - a}.
\]

\textbf{Определение.} Пусть $\gamma$ --- замкнутая кусочно гладкая кривая в $\mathbb C$.
\textit{Компонентой связности} точки $z \in \mathbb C \setminus \gamma$ называется любое связное множество в $\mathbb C \setminus \gamma$, содержащее $z$.
Более того, без доказательства $\mathbb C \setminus \gamma$ разбивается на конечное число максимальных компонент связности.

\textbf{Утверждение.} Пусть $\gamma$ --- кусочно гладкая замкнутая кривая.
Тогда
\begin{enumerate}
    \item $J_\gamma$ на $\mathbb C \setminus \gamma$ действует в целые числа.
    \item $J_\gamma$ является константой на каждой компоненте связности.
    \item $J_\gamma \equiv 0$ в той компоненте связности, в которой лежит бесконечность.
\end{enumerate}

\textbf{Доказательство.} 1) По построению.
2) Заметим, что $J_\gamma$ голоморфна, как интеграл Коши, причём для всех $z \in \mathbb C \setminus \gamma$
\[
    J'_\gamma(z) = \frac{1}{2\pi i} \int_\gamma \frac{\dif \zeta}{(\zeta - z)^2} = 0,
\]
как полный дифференциал.
Следовательно, это константа в пределах каждой компоненты связности.

3) Оценим индекс. Пусть кривая лежит в круге радиуса $R$, рассмотрим $z$, такое что $|z| > R$.
Тогда
\[
    |J_\gamma(z)| = \left| \frac{1}{2\pi} \int_\gamma \frac{\dif \zeta}{\zeta - z} \right| \le \frac{l(\gamma)}{2\pi |z - R|}.
\]
При $z \to\ \infty$ получаем ноль.

\QED

\subsection{Общая теорема Коши}
\textbf{Лемма.} Пусть $f$ голоморфна в $D$ --- области $\mathbb C$.
Положим
\[
    g(\zeta, z) =
    \begin{cases}
        \frac{f(\zeta) - f(z)}{\zeta - z}, & \zeta \ne z \\
        f'(z), & \zeta = z
    \end{cases}.
\]
Тогда $g$ непрерывна в $D \times D$, а функция
\[
    h(z) = \int_\gamma g(\zeta, z) \dif \zeta
\]
голоморфна в $D$ для любой замкнутой кусочно гладкой кривой $\gamma$ в $D$.

\textbf{Доказательство.} Докажем непрерывность.
Рассмотрим $(\zeta_0, z_0) \in D \times D$.
Если $\zeta_0 \ne z_0$, то получаем отношение двух непрерывных функций, что непрерывно.
При $\zeta_0 = z_0$.
Зафиксируем круг $\overline{O_r(z_0)} \subset D$.
Разложим $f$ в ряд Тейлора в окрестности $z_0$, то есть для любого $z \in \overline{O_r(z_0)}$
\[
    f(z) = c_0 + c_1(z - z_0) + \sum_{n =2}^{\infty} c_n(z - z_0)^n.
\]
Выразим $g$ через этот ряд: пусть $\zeta, z \in \overline{O_r(z_0)}$, тогда
\[
    f(z) - f(\zeta) = f'(z_0)(z - \zeta) + \sum_{n=2}^{\infty} c_n \cdot \left( (z - z_0)^n - (\zeta - z_0)^n \right).
\]
Теперь
\[
    g(\zeta, z) - g(z_0, z_0) = \sum_{n=2}^{\infty} c_n \cdot \frac{(\zeta - z_0)^n - (z - z_0)^n}{\zeta - z} = 
\]
\[
    = \sum_{n=2}^{\infty} c_n \left((z - z_0)^{n-1} + (z - z_0)^{n-2}(\zeta - z_0) + \dots + (\zeta - z_0)^{n-1} \right) \le
\]
(так как $|\zeta - z_0| \le r$)
\[
    \le \sum_{n=2}^{\infty} |c_n| \cdot n \cdot r^{n-1} =
\]
Справа что-то немного странное, но, на самом деле, это практически ряд для производной $f$, и он должен всюду сходится:
\[
    = r \sum_{n=2}^{\infty} |c_n| \cdot n \cdot r^{n-2}.
\]
Обозначим эту величину за $C(r, f)$, тогда получается, что для всех $\delta < r$ и $\zeta, z \in O_\delta(z_0)$ выполнено $|g(\zeta, z) - g(z_0, z_0)| \le \delta \cdot C(r, f)$.

Докажем голоморфность функции $h$ в круге $O_r(z_0) \subset D$.
Непрерывность очевидна, теперь рассмотрим треугольник $\overline \triangle \subset O_r(z_0)$.
Тогда
\[
    \int_{\partial \triangle} h(z) \dif z = \int_{\partial \triangle} \left( \int_\gamma g(\zeta, z) \dif \zeta \right) \dif z = \int_\gamma \left( \int_{\partial \triangle} g(\zeta, z) \dif z \right) \dif \zeta.
\]
Зафиксируем $\zeta$. Функция $g(\zeta, \cdot)$ непрерывна в $O_r(z_0)$ и голоморфна в $O_r(z_0) \setminus \{\zeta\}$, откуда $\int_{\partial \triangle} g(\zeta, z) \dif z = 0$.
Следовательно, $\int_{\partial \triangle} h(z) \dif z = 0$, и по теореме Морера она голоморфна.

\QED

\textbf{Определение.} Пусть $\gamma_1, \dots, \gamma_n$ --- замкнутые кусочно гладкие кривые в области $D \subset \mathbb C$, $f$ непрерывна в $D$, $k_1, \dots, k_n \in \mathbb Z$.
Положим $\Gamma = \sum_{j=1}^{n} k_j \gamma_j$ --- \textit{цикл} и
\[
    \int_\Gamma f(z) \dif z := \sum_{j=1}^{n} k_j \int_{\gamma_j} f(z) \dif z.
\]
Тогда $\Gamma$ лежит в $D$, так что можно определить \textit{индекс точки $a \in D \setminus \Gamma$ относительно цикла $\Gamma$}:
\[
    J_\Gamma(a) = \frac{1}{2\pi i} \int_\Gamma \frac{\dif z}{z - a} = \sum_{j=1}^{n} k_j J_{\gamma_j}(z) = \frac{1}{2\pi} \sum_{j=1}^{n} k_j \Delta_{\gamma_j} \arg(z - a).
\]

\textbf{Определение.} Пусть $\Gamma$ --- цикл в области $D \subset \mathbb C$.
Тогда цикл $\Gamma$ \textit{эквивалентен нулю относительно области $D$}, если $J_\Gamma(z) = 0$ для всех $z \not\in D$.
Обозначение --- $\Gamma \sim 0 \mod D$.

\textbf{Пример.} $D$ --- односвязная область (то есть $\mathbb C \setminus D$ связно).
Тогда любой цикл в $D$ будет эквивалентен нулю относительно $D$.

\textbf{Пример.} Рассмотрим кольцо $D = \{r < |z| < R\}$, цикл $\gamma_{\rho_1} - \gamma_{\rho_2}$ --- две окружности радиусов $\rho_1, \rho_2 \in (r, R)$, ориентированные одинаково.
Тогда $\Gamma \sim 0 \mod D$.

\textbf{Теорема.} (Общая теорема Коши)
Пусть $D$ --- область в $\mathbb C$, $f$ голоморфна в $D$, $\Gamma$ --- цикл в $D$, $\Gamma \sim 0 \mod D$.
Тогда для $z \in D \setminus \Gamma$ выполнена формула Коши
\[
    J_\Gamma(z) f(z) = \frac{1}{2\pi i} \int_\Gamma \frac{f(\zeta) \dif \zeta}{\zeta - z},
\]
а также $\int_\Gamma f(z) \dif z = 0$.

\textbf{Доказательство.} Вспоминая предыдущую лемму, положим $h(z) = \int_\Gamma g(\zeta, z) \dif \zeta$ --- голоморфная в $D$ функция.
Ещё положим
\[
    \widetilde h(z) = \int_\Gamma \frac{f(\zeta) \dif \zeta}{\zeta - z}
\]
--- голоморфная в $\mathbb C \setminus \Gamma$ функция, как интеграл Коши.
Теперь для $z \in D \setminus \Gamma$ выполнено 
\[
    h(z) = \int_\Gamma \frac{f(\zeta) - f(z)}{\zeta - z} \dif \zeta = \widetilde h(z) - f(z) \int_\Gamma \frac{\dif \zeta}{\zeta - z} = \widetilde h(z) - f(z) (2\pi i) J_\Gamma(z).
\]
Положим $Q = \{z \in \mathbb C \setminus \Gamma~|~J_\Gamma(z) = 0\}$.
Мы знаем, что это открытое множество, как объединение компонент связности, и оно содержит в себе $\mathbb C \setminus D$, ибо по условию $\Gamma \sim 0 \mod D$.
Тогда, в частности, $\widetilde h$ голоморфна в $Q$.

Заметим, что при $z \in Q \cap D$ выполнено $J_\Gamma(z) = 0$, то есть 
\[
    h(z) = \widetilde h(z) - f(z) (2\pi i) J_\Gamma(z) = \widetilde h(z).
\]
Этот факт доказывает корректность следующего определеления:
\[
    H(z) =
    \begin{cases}
        h(z), & z \in D \\
        \widetilde h(z), & z \in Q \supset \mathbb C \setminus D
    \end{cases},
\]
Теперь докажем, что $H$ стремится к нулю при $z \to \infty$.
Действительно, так как $D \subset \mathbb C$ --- ограниченная область, 
\[
    \lim_{z \to \infty}(H(z)) = \lim_{z \to \infty}(\widetilde h(z)) = 0.
\]
Теперь остаётся заметить, что $H$ ещё и целая, так что по теореме Лиувилля $H \equiv 0$.
Значит, $h \equiv 0$ на $D$, что доказывает первое утверждение, а для доказательства второго можно для $z \in D \setminus \Gamma$ рассмотреть функцию $\widetilde f(\zeta) = (\zeta - z) f(z)$, тогда
\[
    J_\Gamma(z) \widetilde f(z) = \frac{1}{2\pi i} \int_\Gamma \frac{\widetilde f(\zeta) \dif \zeta}{\zeta - z}
\]
или же
\[
    J_\Gamma(z) f(z) (z - z) = \frac{1}{2\pi i} \int_\Gamma f(\zeta) \dif \zeta.
\]

\QED

\section{Ряды Лорана}
До этого у нас были простые ряды вида $f(z) = \sum_{n=0}^{\infty} c_n z^n$, которые сходятся в круге с центром в нуле, и такими рядами представляются в точности все голоморфные функции.

\textbf{Определение.} \textit{Порядок нуля} $\ord_0(f)$ --- это наибольшее $n$, такое что функция $f$ представляется в виде $z^n g(z)$, где $g(0) \ne 0$.

Тогда альтернативное определения ряда Тейлора --- это такие многочлены $T_n$ степени $n$, что $\ord_0(f(z) - T_n(z)) \ge n + 1$ для всех $z$.

Теперь пусть у нас степени отрицательны, и ряд имеет вид $\sum_{n=0}^{\infty} c_n z^{-n}$, тогда ряд будет сходиться на внешности какого-то круга
А если рассмотреть сумму этих рядов --- ряд $\sum_{n=-\infty}^{\infty} c_n (z - a)^n$, то получится \textit{ряд Лорана}.

\textbf{Определение.} Будем считать, что Ряд Лорана \textit{сходится}, если сходятся ряды $\sum_{n=0}^{\infty} c_n (z - a)^n$ и $\sum_{n=1}^{\infty} \frac{c_n}{(z - a)^n}$.

Как нетрудно понять, ряд Лорана сходится в пересечении круга и внешности круга, то есть в \textit{кольце} вида $K = \{z: r < |z - a| < R\}$.

\textbf{Теорема.} Пусть $f$ голоморфна в $K = \{z: r < |z - a| < R\}$, причём $0 \le r < R \le +\infty$.
Тогда в $K$ её можно представить в виде сходящегося ряда $f(z) = \sum_{n=-\infty}^{+\infty} c_n (z - a)^n$, причём его коэффициенты находятся по формуле
\[
    c_n = \frac{1}{2\pi i} \int_{\gamma_\rho} \frac{f(z) \dif z}{(z - a)^{n+1}},
\]
где, как обычно, $\gamma_\rho$ --- это окружность с центром в точке $a$, а $\rho \in (r, R)$.

\textbf{Доказательство.} Как видно по формуле, доказывать мы будем через общую теорему Коши.
Сначала покажем, что формула не зависит от $\rho$.
Рассмотрим $r < r' < R' < R$, положим $\Gamma = \gamma_{R'} - \gamma_{r'}$.
Заметим, что $\Gamma \sim 0 \mod K$: действительно, возьмём $z \in \mathbb C \setminus K$ и проверим индекс (рисунок 5 иллюстрирует происходящее).
\begin{figure}[ht]
    \centering
    \incfig{ring-zero}{0.5\linewidth}
    \caption{Окружности в кольце}
\end{figure}

При $|z| \le r$ получаем $J_{\gamma_{R'}}(z) = J_{\gamma_{r'}}(z) = 1$.
При $|z| \ge R$ получаем $J_{\gamma_{R'}}(z) = J_{\gamma_{r'}}(z) = 0$, следует чисто из картинки.

Теперь, так как функция $\frac{f(\zeta)}{(\zeta - a)^{n+1}}$ голоморфна в $K$, её интеграл по $\Gamma$ равен нулю, значит, её интегралы по $\gamma_{R'}$ и по $\gamma_{r'}$ равны, что доказывает независимость значения от выбора $\rho$.

Теперь рассмотрим $f$ на кольце $K' = \{z: r' < |z - a| < R'\}$.
На нём из геометрических соображений $J_\Gamma \equiv 1$, тогда по общей формуле Коши
\[
    J_\Gamma(z) f(z) = f(z) = \frac{1}{2\pi i} \int_\Gamma \frac{f(\zeta) \dif \zeta}{(\zeta - z)} = \frac{1}{2\pi i} \left( \int_{\gamma_{R'}} \dots - \int_{\gamma_{r'}} \dots \right) = f_1(z) + f_2(z).
\]
Рассмотрим эти самые функции $f_1, f_2$.
Как интеграл Коши, $f_1$ голоморфна в круге $|z - a| < R'$, поэтому она раскладывается в ряд $f(z) = \sum_{n=0}^{\infty} c_n (z - a)^n$, где
\[
    c_n = \frac{f_1^{(n)}(a)}{n!} = \frac{1}{2\pi i} \int_{\gamma_{R'}} \frac{f(\zeta) \dif \zeta}{(\zeta - a)^{n+1}},
\]
как производная интеграла Коши.

Теперь посмотрим на $f_2(z) = - \frac{1}{2\pi i} \int_{\gamma_{r'}} \frac{f(\zeta) \dif \zeta}{\zeta - z}$.
Перепишем знаменатель:
\[
    -\frac{1}{\zeta - z} = -\frac{1}{\zeta - a - (z - a)} = \frac{1}{z - a} \cdot \frac{1}{1 - \frac{\zeta - a}{z - a}} = \sum_{n=1}^{\infty} \frac{(\zeta - a)^{n-1}}{(z - a)^n}.
\]
Это тождество верно при $|\zeta - a| < |z - a|$, чтобы геометрическая прогрессия сходилась.
Теперь вспомним, что в кольце $K$ выполнено $|z - a| > r'$, а $\zeta$ берётся из $\gamma_{r'}$, то есть $|z - a| > r' = |\zeta - a|$, и для наших целей эта формула верна всегда.

Наконец, подставим это всё в $f_2$:
\[
    f_2(z) = - \frac{1}{2\pi i} \int_{\gamma_{r'}} \frac{f(\zeta) \dif \zeta}{z - \zeta} = \frac{1}{2\pi i} \sum_{n=1}^{\infty} \frac{1}{(z - a)^n} \left( \int_{\gamma_{r'}} f(\zeta) (\zeta - a)^{n-1} \dif \zeta \right).
\]
Иными словами,
\[
    f_2(z) = \sum_{n=1}^{\infty} \frac{c_{-n}}{(z - a)^n},
\]
где
\[
    c_{n} = \frac{1}{2\pi i} \int_{\gamma_{r'}} \frac{f(\zeta) \dif \zeta}{(\zeta - a)^{n+1}}.
\]
($(\zeta - a)^{n+1}$ переехал в знаменатель, так как здесь уже $n < 0$)

\QED

\textbf{Теорема.} (О единственности ряда Лорана) Пусть
\[
    f(z) = \sum_{n=-\infty}^{+\infty} c_n(z - a)^n
\]
в кольце $K = \{z: r < |z - a| < R\}$, то есть в нём ряд сходится.
Тогда $f$ голоморфна в $K$ и
\[
    c_n = \frac{1}{2\pi i} \int_{\gamma_{\rho}} \frac{f(\zeta) \dif \zeta}{(\zeta - a)^{n+1}}.
\]

\textbf{Доказательство.} Голоморфность относительно очевидна: раз ряд сходится, то сходятся его отрицательная и положительная части, каждая из которых по отдельности голоморфны, как ряды Тейлора.
Теперь найдём коэффициент $c_{-1}$, проинтегрировав $f$:
\[
    \int_{\gamma_\rho} f(z) \dif z = \sum_{n=\infty}^{+\infty} c_n \int_{\gamma_\rho} (z - a)^n \dif z.
\]
Заметим, что в этой сумме почти все слагаемые равны нулю.
Действительно, мы же интегрируем полный дифференциал по замкнутой кривой, так что выживает лишь $c_{-1}$:
\[
    \int_{\gamma_\rho} f(z) \dif z = c_{-1} \int_{\gamma_\rho} (z - a)^{-1} \dif z = 2\pi i \cdot c_{-1}.
\]
Теперь нужно доказать для остальных коэффициентов, это делается домножением на $(z - a)^m$:
\[
    (z - a)^m f(z) = \sum_{n=-\infty}^{+\infty} c_n(z - a)^{n - m} = \sum_{n=-\infty}^{+\infty} c_{n+m} (z - a)^n.
\]
Вновь интегрируем и получаем искомое.

\QED

\subsection{Изолированные особые точки}
\textbf{Определение.} Точка $a$ называется \textit{изолированной особой точкой} $f$, если $f$ голоморфна в $\dot O_R(a)$.
Более того, она называется
\begin{itemize}
    \item \textit{Устранимой особой точкой}, если существует предел $\lim_{z \to a} f(z) \in \mathbb C$.
    \item \textit{Полюсом}, если существует предел $\lim_{z \to a} f(z) = \infty$.
    \item \textit{Существенно особой точкой}, иначе.
\end{itemize}

\textbf{Утверждение.} (Неравенство Коши) Пусть $M_f(\rho) = \max_{\gamma_\rho} |f|$.
Тогда для коэффициентов ряда Лорана выполнено неравенство $|c_n| \le \frac{M_f(\rho)}{\rho^n}$.

Доказательство аналогично случаю ряда Тейлора, так как формула на коэффициенты та же самая.

\textbf{Утверждение.} Точка $a$ является УОТ функции $f$ тогда и только тогда, когда $f$ ограничена в $\dot O(a)$.

\textbf{Доказательство.} $\Rightarrow$ очевидно. $\Leftarrow$: по неравенству Коши $|c_n| \le \frac{M_f(\rho)}{\rho^n}$.
Теперь из ограниченности возьмём $M$, такое что $|f| \le M$ на $\dot O_R(a)$, а значит, неравенство Коши можно продолжить и получить $|c_n| \le \frac{M}{\rho^n}$.
При $n < 0$ это стремится к нулю при $\rho \to 0$.

Значит, $f(z) = \sum_{n=0}^{\infty} c_n (z - a)^n$, то есть отрицательных степеней нет, и при $z \to a$ это стремится к $c_0 \in \mathbb C$.

\QED

\textbf{Утверждение.} Точка $a$ является полюсом тогда и только тогда, когда найдётся $m \in \mathbb N$, такое что $c_{-m} \ne 0$, но при $n > m$ выполнено $c_{-n} = 0$ для разложения $f$ в ряд Лорана с центром в точке $a$.

\textbf{Доказательство.} $\Rightarrow$. Так как $a$ является полюсом, $\lim_{z \to a} f(z) = \infty$.
По определению это означает, что $\forall \varepsilon > 0~\exists \delta > 0$, такие что $|f| > \frac{1}{\varepsilon}$ на $\dot O_\delta(a)$.
Положим $g = \frac{1}{f}$, тогда $|g| < \varepsilon$ на $\dot O_\delta(a)$, то есть $g$ имеет устранимую особенность в точке $a$, причём $\lim_{z \to a} g(z) = 0 = g(0)$, а это значит, что у функции $g$ ненулевой порядок нуля, то есть найдётся $m$, такое что существует представление $g(z) = (z - a)^m h(z)$, где $h(a) \ne 0$.

Отсюда следует, что $\frac{1}{h}$ голоморфна в $O(a)$, и её можно разложить в ряд Тейлора
\[
    \frac{1}{h(z)} = \sum_{n=0}^{\infty} b_n (z - a)^n
\]
в окрестности точки $a$.
Наконец, подставим это в $g$:
\[
    f = \frac{1}{g} = \frac{1}{(z - a)^m h} = \sum_{n=0}^{\infty} (z - a)^{n-m} b_n = \sum_{n=-m}^{\infty} (z - a)^n b_{n+m}.
\]

$\Leftarrow$. Пусть
\[
    f(z) = \frac{c_{-m}}{(z - a)^m} + \dots + \frac{c_{-1}}{z - a} + \sum_{n=0}^{\infty} c_n (z - a)^n.
\]
Утверждение почти очевидно, но нужно ещё доказать, что бесконечности не сократятся.
Положим $\tilde f(z) = (z - a)^m f(z) = \sum_{n=0}^{\infty} (z - a)^n c_{n-m}$.
Сия функция уже голоморфна в $\dot O(a)$, причём у неё устранимая особенность в точке $a$, так что её можно доопределить до $O(a)$.
Теперь
\[
    f(z) = \frac{\tilde f(z)}{(z - a)^m}.
\]
Штука сверху ограничена, а штука снизу стремится к нулю при $z \to a$.

\QED

\section{Вычеты}
\subsection{Определение и свойства}
\textbf{Определение.} Пусть $a \in \mathbb C$, $f$ голоморфна в $\dot O_r(a)$.
\textit{Вычетом} функции $f$ в точке $a$ называется 
\[
    \res_a(f) = \frac{1}{2\pi i} \int_{\gamma_\rho(a)} f(z) \dif z
\]
для $\rho \in (0, r)$.

\textbf{Утверждение.}
Пусть $f(z) = \sum_{n=-\infty}^{\infty} c_n(z - a)^n$, $0 < |z - a| < r$.
\begin{itemize}
    \item $\res_a(f) = c_{-1}$.
    \item Если $\Gamma$ --- цикл в $\dot O_r(a)$, то
        \[
            \frac{1}{2\pi i} \int_\Gamma f(z) \dif z = J_\Gamma(a) \res_a(f).
        \]
    \item Если $a$ --- устранимая особая точка, то $\res_a(f) = 0$.
    \item Если $a$ --- полюс $n$--го порядка, то
        \[
            \res_a(f) = \lim_{z \to a} \left( \frac{1}{(n-1)!} (f(z) (z - a)^{n-1})^{(n-1)} \right).
        \]
\end{itemize}

\textbf{Доказательство.} 1) Проинтегрируем $f$:
\[
    \int_{\gamma_\rho(a)} f(z) \dif z = \sum_{-\infty}^{\infty} c_n \int_{\gamma_\rho(a)} (z - a)^n \dif z.
\]
В этой сумме, как обычно, все интегралы, кроме интеграла при $(n-1)$--ом слагаемом, равны нулю, как полные дифференциалы, так что остаётся ровно $c_{-1}$.

2) Если поменять $\gamma_\rho(a)$ на произвольный цикл $\Gamma$ и записать сумму выше, то получится то же самое, но в оставшемся слагаемом будет $J_\Gamma(a) (2\pi i)$ вместо единицы по общей теореме Коши.

3) По определению устранимой особенности $c_{-1} = 0$.

4) По условию
\[
    f(z) = \frac{c_{-n}}{(z - a)^m} + \dots + \frac{c_{-1}}{z - a} + \sum_{n=0}^{\infty} c_n(z - a)^n.
\]
Положим $g(z) = (z - a)^n f(z)$, тогда она голоморфна в точке $a$, значит, существует предел из условия, и он в точности равен $(n-1)$--ому коэффициенту ряда Тейлора функции $g$.

\QED

\textbf{Определение.} Область $D$ \textit{ограничена циклом} $\Gamma = \gamma_0 - \gamma_1 - \dots - \gamma_n$, если выполнены три свойства:
\begin{enumerate}
    \item $\gamma_0, \dots, \gamma_n$ --- замкнутые кусочно гладкие жордановые положительно ориентированные кривые.
    \item $\Gamma = \partial D$.
    \item
        \[
            J_\Gamma(z) =
            \begin{cases}
                1, & z \in D \\
                0, & z \not\in \overline D
            \end{cases} .
        \]
\end{enumerate}

Неформально область представляет из себя множество с дырками, и эти дырки не пересекаются по внутренности, пример на рисунке 6.

\begin{figure}[ht]
    \centering
    \incfig{loop-limited}{0.75\linewidth}
    \caption{Область, ограниченная циклом}
\end{figure}

\textbf{Теорема.} (Теорема Коши для многосвязной области) Пусть область $D$ ограничена циклом $\Gamma$, $f$ голоморфна в области $D'$, содержащей $\overline D$.
Тогда
\begin{itemize}
    \item $\int_\Gamma f(z) \dif z = 0$.
    \item Для любого $z \in D$ выполнено $f(z) = \frac{1}{2\pi i} \int_\Gamma \frac{f(\zeta) \dif \zeta}{\zeta - z}$.
    \item Для любого $z \in D$ выполнено $f^{(n)}(z) = \frac{n!}{2\pi i} \int_\Gamma \frac{f(\zeta) \dif \zeta}{(\zeta - z)^{n+1}}$.
\end{itemize}

\textbf{Доказательство.} Нужно доказать, что $\Gamma \sim 0 \mod D'$.
Действительно, если $z \not\in D'$, то $J_\Gamma(z) = 0$ по определению ограниченности (если точка лежит в дырке, то две кривые вокруг неё сократятся, а если снаружи, то тем более ноль).
Теперь отсюда всё следует по общей теореме Коши.

\QED

\subsection{Теорема Коши о вычетах и её следствия}
\textbf{Теорема.} (Коши о вычетах) Пусть $D$ --- область, ограниченная циклом $\Gamma$, $A = \{a_1, \dots, a_N\} \subset D$, $f$ голоморфна на $D \setminus A$.
Тогда
\[
    \frac{1}{2\pi i} \int_\Gamma f(z) \dif z = \sum_{j=1}^{N} \res_{a_j} (f).
\]

\textbf{Доказательство.} Вырежем все точки $A$ с небольшой областью, чтобы подогнать под теорему Коши о многосвязной области.
Найдётся $r > 0$, такое что все $\overline{O_r(a_j)}$ попарно не пересекаются и вложены в $D$.
Положим $\lambda_k = \partial O_r(a_k)$, как положительно ориентированные кривые, $G := D \setminus \bigcup_{j=1}^N \overline{O_r(a_j)}$.
Тогда $G$ ограничена циклом $\Gamma - \sum_{k=1}^{N} \lambda_k$, и $f$ голоморфна в $\overline G$.

Следовательно, можно применить общую теорему Коши для многосвязной области, и получить
\[
    \int_{\partial G} f(z) \dif z = \left( \int_\Gamma - \sum_{k=1}^{N} \int_{\lambda_k} \right) f(z) \dif z = 0.
\]
Перенося сумму в правую часть, получаем
\[
    \int_\Gamma f(z) \dif z = \sum_{k=1}^{N} \int_{\lambda_k} f(z) \dif z = 2\pi i \sum_{k=1}^{N} \res_{a_k}(f).
\]

\QED

\textbf{Определение.} Пусть $f$ голоморфна в $\mathbb C \setminus \overline{O_R(0)}$. \textit{Вычетом в точке} $\infty$ называется
\[
    \res_{\infty}(f) = \frac{1}{2\pi i} \int_{-\gamma_\rho} f(z) \dif z,
\]
где $\rho > R$.
Окружность берётся именно с отрицательной ориентацией, чтобы она ``смотрела`` в бесконечность.
Более того, из доказанного это равняется $-c_{-1}$ --- коэффициенту разложения в ряд Лорана.

\textbf{Теорема.} (О сумме вычетов) Пусть $f$ голоморфна в $\mathbb C \setminus A$, где $A = \{a_1, \dots, a_n\} \subset \mathbb C$.
Тогда
\[
    \sum_{j=1}^{N} \res_{a_j}(f) + \res_{\infty}(f) = 0.
\]

\textbf{Доказательство.} Рассмотрим область, ограниченную циклом, состоящим из большой окружности $\gamma_\rho$ из определения вычета в $\infty$.
По теореме Коши о вычетах
\[
    \int_{\gamma_\rho} f(z) \dif z = 2\pi i \sum_{j=1}^{N} \res_{a_j}(f),
\]
и по определению вычета в бесконечности это же равно $-\res_\infty(f)$.

\QED

\textbf{Определение.} Пусть $f$ голоморфна в области $G$, $K$ --- компакт в $G$.
Тогда $f$ имеет конечное число нулей в $K$, если она не является тождественным нулём (по теореме о единственности).
Обозначим их за $b_1, \dots, b_N$, а за $n_1, \dots, n_N$ --- их кратности.
Тогда \textit{числом нулей} функции $f$ на компакте $K$ называется $N_K(f) = \sum_{j=1}^{N} n_j$.

\textbf{Определение.} \textit{Числом полюсов} функции $f$ называется $P_K(\frac{1}{f}) = \sum_{j=1}^{N} n_j$.

\textbf{Теорема.} (Принцип аргумента)
Пусть $D$ --- область, ограниченная циклом $\Gamma$, $A$ --- конечное множество, $A = \{a_1, \dots, a_M\} \subset D$, $f$ голоморфна в $\overline D \setminus A$, $a_j$ --- полюса $f$, и $f$ не обнуляется на $\Gamma$.

Тогда 
\[
    N_D(f) - P_D(f) = \frac{1}{2\pi} \Delta_\Gamma \arg(f) = \frac{1}{2\pi i} \int_\Gamma \frac{f'(z)}{f(z)} \dif z = J_{f(\Gamma)}(0).
\]
Эти три формулы эквивалентны по определению аргумента, поэтому мы будем доказывать лишь для одной из них.

\textbf{Доказательство.} Пусть $p_j$ --- кратности полюсов, тогда $f$ записывается в виде
\[
    f(z) = g(z) \frac{\prod_{j=1}^N (z - b_j)^{n_j}}{\prod_{j=1}^M (z - a_j)^{p_j}},
\]
где $g$ --- голоморфная на $\overline D$ функция, не обращающаяся в ноль на $D$.

Теперь упражнение:
\[
    \frac{f'}{f} = \frac{g'}{g} + \sum_{j=1}^{N} \frac{n_j}{z - b_j} - \sum_{j=1}^{M} \frac{p_j}{z - a_j}.
\]
Предлагаемый способ доказательства --- прологарифмировать тожество выше и взять производную.

В любом случае это доказывает, что $\frac{f'}{f}$ голоморфна в $\overline D \setminus (A \cup B)$.
Тогда
\[
    \frac{1}{2\pi i} \int_\Gamma \frac{f'}{f} \dif z = \sum_{j=1}^{N} \res_{b_j} \frac{f'}{f} + \sum_{j=1}^{M} \res_{a_j} \frac{f'}{f}  = N_D(f) - P_D(f).
\]
Последний переход проще всего увидеть, если расписать вычет по определению через интеграл: в нём сокращается всё, кроме нужного слагаемого суммы, как полный дифференциал.

\QED

\textbf{Теорема.} (Рушé) Пусть $D$ --- область, ограниченная циклом $\Gamma$, функции $f$, $g$ голоморфны в $\overline D$ и $|f| > |g|$ на $\Gamma$.
Тогда $N_D(f) = N_D(f + g)$.

\textbf{Доказательство.} Заметим, что
\[
    f + g = f \cdot \left( 1 + \frac{g}{f} \right),
\]
причём на $\Gamma$ здесь даже не будет деления на ноль по условию.
Поэтому
\[
    \Delta_\Gamma \arg(f + g) = \Delta_\Gamma(f) + \Delta_\Gamma \arg \left(1 + \frac{g}{f} \right).
\]
Докажем, что второе слагаемое равно нулю.
Заметим, что $(1 + \frac{g}{f})(\Gamma) \subset O_1(1)$.
Пусть $\Gamma = \gamma_0 - \gamma_1 - \dots - \gamma_n$, рассмотрим кривую $\gamma_i$.
Положим $\gamma' = (1 + \frac{g}{f})(\gamma) \subset O_1(1)$.
Заметим, что, как бы эта кривая ни крутилась, она совершит ровно ноль оборотов вокруг начала координат.
Следовательно, $\Delta_{\gamma'} \arg(z) = 0$, и, как сумма таких нулей по всем кривым $\gamma_i$, всё второе слагаемое равно нулю.

\QED

\textbf{Теорема.} Основная теорема алгебры.

\textbf{Доказательство.} Пусть многочлен $P$ имеет вид $z^n + c_{n-1} z^{n-1} + \dots + c_0$.
Обозначим за $f$ старший член, а за $g$ --- остальные члены.
Тогда найдётся $R$, такое что $|f| > |g|$ при $|z| \ge R$.
Следовательно, по теореме Руше количество нулей многочлена совпадает с количеством нулей $f$, то есть равно $n$.

\QED

\subsection{Интегрирование}
Всего нужно уметь считать три вида интегралов, которые встретятся на семестровой контрольной, на экзамене, и, вообще, будут преследовать всю жизнь.

Первый тип --- $I = \int_0^\pi R(\cos(\phi), \sin(\phi)) \dif \phi$.
Решается заменой $R_1 = \frac{1}{z} R(\cdot, \cdot)$, где $z = e^{i\phi}$.
Далее остаётся проинтегрировать по единичной окружности, получится $2\pi \sum_{|a| < 1} \res_a(R_1)$.

Второй тип --- $\int_{-\infty}^{+\infty} \frac{P}{Q}(x) \dif x$, где $Q \ne 0$ на $\mathbb R$ и $\deg(Q) \ge \deg(P) + 2$.
Он равен $2\pi i \sum_{\im(a) > 0} \res_a \frac{P}{Q}$, если нарисовать картинку с действительной осью и дугой над ней.

Третий тип --- $\int_{-\infty}^{+\infty} \frac{P}{Q}(x) \sin(\alpha x) \dif x$.
Он считается леммой Жордана, она будет чуть позже (в самом конце).

\subsection{Теорема о локальной структуре отображения и её следствия}
\textbf{Теорема.} (О локальной структуре отображения) Пусть $f$ голоморфна в области $D$, $z_0 \in D$, такая что $f(z_0) = w_0$ с кратностью $n$, то есть $f'(z_0) = f''(z_0) = \dots = f^{(n-1)}(z_0) = 0$, а $f^{(n)}(z_0) \ne 0$.
Тогда существует $R > 0$, такое что $O_R(z_0) \subset D$ и для всех $r < R$ найдётся $\rho > 0$, такое что для всех $w^* \in \dot O_\rho(w_0)$ уравнение $f(z) = w^*$ имеет ровно $n$ простых (кратности 1) различных корней в $\dot O_r(z_0)$.
Например, при $n = 1$ это будет теоремой об обратной функции.

\textbf{Доказательство.} Так как $f^{(n)}(z_0) \ne 0$, функция не является константой, а значит, найдётся $R > 0$, такое что $\overline{O_R(z_0)} \subset D$ и для $z \in \overline{O_R(z_0)} \setminus \{z_0\}$ выполнено $f(z) - w_0 \ne 0$ и $f'(z) \ne 0$, так как нули голоморфной функции изолированы.

Зафиксируем $r < R$. Положим $\gamma = \partial O_r(z_0)$ --- окружность радиуса $r$ и $\Gamma = f(\gamma)$.
Тогда $\gamma$ лежит в $\overline{O_R(z_0)}$, а значит, $f \ne w_0$ на $\gamma$.
Засим можно определить $\rho := \frac{1}{2} \dist(\Gamma, w_0) > 0$.

Рассмотрим $w^* \in \dot O_\rho(w_0)$, тогда $|w^* - w_0| < \rho < |f(z) - w_0|$.
Теперь перепишем исходное уравнение:
\[
    f(z) - w^* = (f(z) - w_0) + (w_0 - w^*).
\]
Наконец, по теореме Руше $f(z) - w_0$ и $f(z) - w^*$ имеют одинаковое количество нулей с учётом кратности в $O_r(z_0)$
У функции $f(z) - w_0$ по условию имеется ровно $n$ нулей, так что остаётся проверить, что $n$ нулей $f(z) - w^*$ различны.

Действительно, мы выбирали $w^*$ так, что $f(z_0) \ne w^*$, а значит, все нули лежат в $\dot O_r(z_0)$.
Теперь вспомним, что в самом начале мы выбрали $R$ так, чтобы производная не обнулялась в этой окрестности.
Значит, у всех нулей кратность --- ровно 1.

\QED

\textbf{Определение.} Пусть $f$ голоморфна в области $D$.
Она называется \textit{однолистной}, если из $f(z_1) = f(z_2)$ следует $z_1 = z_2$.

$f$ называется \textit{локально однолистной}, если для любой $z_0 \in D$ найдётся окрестность $O(z_0)$, в которой она однолистна.

\textbf{Следствие.} Пусть $f$ голоморфна в $D$.
Она является локально однолистной в $D$ тогда и только тогда, когда $f'(z) \ne 0$ в $D$.

\textbf{Доказательство.} Рекомендуется доказать в качестве упражнения.
$\Rightarrow$. От противного: пусть нашлась точка $z_0 \in D$, такая что $f'(z_0) = 0$.
Обозначим $w_0 = f(z_0)$.
Теперь есть два случая: $f(z_0) = w_0$ либо с бесконечной кратностью, либо с конечной.
Если с бесконечной, то $z_0$ --- ноль бесконечного порядка функции $f - w_0$, значит, $f - w_0 \equiv 0$ в окрестности $z_0$ --- противоречие с однолистностью.

Если конечного, то найдётся $n$, такой что $f'(z_0) = f''(z_0) = \dots = f^{(n-1)}(z_0) = 0$ и $f^{(n)} \ne 0$.
По теореме о локальной структуре отображения найдётся $r > 0$ и $w^* \ne w_0$, такие что уравнение $f(z) = w^*$ имеет ровно $n$ различных корней в $\dot O_r(z_0) \subset D$.
$n \ge 2$ по допущению --- противоречие с однолистностью.

$\Leftarrow$. Здесь просто по теореме об обратной функции для любого $z_0 \in D$ функция $f$ будет биекцией в какой-то окрестности, что сразу даёт однолистность.

\QED

У непрерывных функций есть классические свойства, такие как прообраз открытого открыт и образ связного связен.
Получим подобный результат для голоморфных функций.

\textbf{Теорема.} (Принцип открытости и сохранения области)
Пусть $f$ голоморфна, и не является константой в области $D$.
Тогда $f(D)$ является областью.

\textbf{Доказательство.} Связность очевидна, докажем открытость: пусть $G = f(D)$.
Рассмотрим $w_0 \in G$, докажем, что он входит с окрестностью.
Так как $w_0$ лежит в образе, найдётся $z_0 \in D$, такое что $f(z_0) = w_0$.
Пусть $n$ --- кратность $f(z_0) = w_0$, то есть $f^{(n)}(z_0) \ne 0$, а до этого нули.
Из открытости $D$ найдётся $r > 0$, такой что $O_r(z_0) \subset D$, а из теоремы о локальной структуре отображения найдётся $\rho > 0$, такое что $f(z) = w^*$ имеет ровно $n$ различных корней в $O_r(z_0)$ для всех $w^* \in O_\rho(w_0)$.

А из этого следует, что $O_\rho(w_0) \subset f(O_r(z_0)) \subset G$, так как $n \ge 1$, то есть теорема гласит, что у всех $w^*$ есть прообраз.

\QED

\textbf{Определение.} Пусть $D$ --- область, $\phi: D \to \mathbb R$.
Точка $z_0 \in D$ называется \textit{точкой максимума} (нестрогого локального), если найдётся $r > 0$, такое что для всех $z \in O_r(z_0)$ выполнено $\phi(z_0) \ge \phi(z)$.

\textbf{Теорема.} (Принцип максимума) Пусть функция $f$ голоморфна в области $D$, и она отлична от константы.
Тогда $|f(z)|$, $\pm \re(f)$, $\pm \im(f)$ не имеют точку максимума в $D$.

\textbf{Замечание.} Если $D$ ограничена, то теорема означает, что максимум достигается на границе.

\textbf{Доказательство.} От противного: пусть $z_0$ --- точка максимума $|f|$, $w_0 = f(z_0)$.
Вытащим $r$ из определения точки максимума, тогда в $O_r(z_0)$ получаем, что $|f| \le |f(z_0)| = |w_0|$.
Иными словами, $f(O_r(z_0)) \subset \overline{O_{|w_0|}(0)}$.

Что это значит? Мы взяли открытое множество, отобразили его в замкнутое множество, и центр образа неожиданно оказался на границе сего замкнутого множества.
Но образ открытого открыт, что даёт противоречие.

Теперь докажем про вещественную часть.
Можно доказывать аналогичным образом, а можно рассмотреть $|e^f| = e^{\re(f)}$.

\QED

\textbf{Лемма.} (Шварца) Пусть $\mathbb D = \{z: |z| < 1\}$ --- единичный круг, $f$ в голоморфна в нём, $f(0) = 0$ и $|f| \le 1$.
Тогда для любого $z \in \mathbb D$ выполнено $|f(z)| \le |z|$, а ещё $|f'(0)| \le 1$.

Более того, если хотя бы одно из неравенств обращается в равенство ($|f'(0)| = 1$ или для какого-то $z_0 \in \mathbb D \setminus \{0\}$ оказалось $|f(z_0)| = |z_0|$), то $f(z) = z \cdot e^{i\theta}$, то есть функция поворота.

\textbf{Доказательство.} Положим $g(z) = \frac{f(z)}{z}$. Что мы про неё знаем?
\begin{itemize}
    \item Она голоморфна в $\mathbb D \setminus \{0\}$.
    \item Ноль является устранимой особенностью, так что можно доопределить и получить голоморфность на всём $\mathbb D$.
    \item $g(0) = f'(0)$.
\end{itemize}
Зафиксируем $r \in (0, 1)$, $z \in O_r(0)$.
Тогда
\[
    |g(z)| \le \max_{|z| = r} |g| \le \frac{1}{r} \max_{|z| = r} |f| \le \frac{1}{r},
\]
первое неравенство получено по принципу максимума (максимум не может достигаться на внутренности).

А теперь сделаем наоборот: зафиксируем $z \in \mathbb D$, тогда для всех $r \in (|z|, 1)$ выполнено $|g(z)| \le \frac{1}{r}$.
Устремляя $r$ к единице, получаем $|g(z)| \le 1$.
Теперь напрямую из определения $g$ получаем $|f(z)| \le |z|$ и $|f'(0)| = |g(0)| \le 1$.

Докажем вторую часть: если где-то $|f(z_0)| = |z_0|$, то в этой точке $|g(z_0)| = 1$, то есть по контрапозиции с принципом максимума $g \equiv const$.
Следовательно, $g \equiv e^{i\theta}$ для $\theta \in \mathbb R$.
А если $|f'(0)| = 1$, то $|g(0)| = 1$, далее аналогично.

\QED

\textbf{Замечание.} (Душное) Можно портребовать и $|f(z)| < 1$, так как иначе противоречие с принципом максимума.

\textbf{Теорема.} (Сохоцкого) Пусть $a$ --- существенная особенность функции $f$, голоморфной в $\dot O(a)$.
Тогда для любого $A \in \overline{\mathbb C}$ найдётся последовательность $z_n \to a$ из $\dot O(a)$, такая что $f(z_n) \to A$.

\textbf{Доказательство.} От противного: допустим, что найдётся $A$, такое что для любой последовательности $z_n \to a$ последовательность $f(z_n) \not\to A$.

Есть два случая. Первый --- $A = \infty$, тогда $f$ ограничена в $\dot O(a)$, а значит, $a$ является устранимой особой точкой, противоречие.

Второй: $A \in \mathbb C$. Заметим, что допущенное от противного эквивалентно тому, что найдутся $\varepsilon, \delta > 0$, такие что $O_{\varepsilon}(A) \cap f(\dot O_\delta(a)) = \varnothing$.
Положим $g = \frac{1}{f - A}$ --- голоморфна и ограничена в $\dot O(a)$ по рассуждению выше.
Значит, $a$ --- устранимая особая точка $g$, и её можно доопределить.

Выразим $f$: $f = A + \frac{1}{g}$.
Если $g(a) = 0$, то $f$ имеет полюс в точке $a$.
А если $g(a) \ne 0$, то $f$ имеет устранимую особенность в $a$.

\QED

\section{Асимптотические разложения, лемма Ватсона}
\subsection{Напоминание}
Гамма--функция Эйлера существует, 
\[
    \Gamma(z) = \int_0^\infty t^{z-1} e^{-t} \dif t,
\]
определена только при $\re(z) > 0$.
Почему? Потому что $t^{z-1} e^{(z-1) \ln(t)}$, то есть
\[
    |t^{z-1}| = \exp((\re(z) - 1) \ln(t)) = t^{\re(z) - 1},
\]
откуда по интегральному признаку интеграл конечен только в правой полуплоскости.

Свойства:
\begin{itemize}
    \item $\Gamma(z + 1) = z \cdot \Gamma(z)$.
    \item $\Gamma(n + 1) = n!$ для $n \in \mathbb N$.
    \item Упражнение --- $n! \int_0^1 (-\ln(x))^n \dif x$, делается заменой переменной $s = e^{-t}$.
    \item $\Gamma(\frac{1}{2}) = \sqrt \pi$, откуда $\Gamma(n + \frac{1}{2}) = \frac{\sqrt \pi (2n - 1)!!}{2^n}$.
\end{itemize}

\subsection{Асимптотические разложения}
Пусть $E \subset \mathbb C$, $\infty$ --- предельная точка $E$, а также на нём определена последовательность функций $f_0, f_1, f_2, \dots: E \to \mathbb C$, такие что они не обращаются в ноль, а также $f_{n+1}(z) = o(f_n(z))$ при $z \to \infty$.
Иными словами,
\[
    \lim_{E \ni z \to \infty} \left( \frac{f_{n+1}}{f_n} (z) \right) = 0.
\]

\textbf{Определение.} Пусть $F: E \to \mathbb C$, тогда $F$ \textit{допускает асимптотическое разложение по $f_n$}, $F \sim \sum_{n=0}^{\infty} c_n f_n$, если для всех $n$
\[
    \left( F - \sum_{n=0}^{n - 1}  c_k f_k \right)(z) = o(f_{n-1}(z)) = O(f_n(z)), z \to \infty.
\]

\textbf{Замечание.} Такие разложения можно складывать и умножать на скаляр.
Чем-то это всё похоже на степенные ряды, но тут нет единственности.

\textbf{Пример.} $F(x) = e^{-x}$, $E = \mathbb R_+$, $f_n(x) = x^{-n}$.
Тогда
\[
    e^{-x} \sim \sum_{n=0}^{\infty} 0 \cdot x^{-n},
\]
но таким разложением ещё обладает ноль.

\textbf{Пример.} 
\[
    F(z) = \sum_{n=p}^{-\infty} c_n z^{-n} = \sum_{k=0}^{\infty} c_{p-k} z^{p-k},
\]
то есть ряд Лорана.

\textbf{Пример.}
\[
    F(z) = \int_0^\infty \frac{e^{-t} \dif t}{z + t}
\]
на множестве $E = \mathbb R_+$.
Заметим, что
\[
    \frac{1}{z + t} = \sum_{n=0}^{N - 1} \frac{(-1)^n t^n}{z^{n+1}} + \frac{(-1)^N t^N}{z^N (z + t)},
\]
то есть разложение в ``конечный ряд Тейлора``.
Подставим это в $F$:
\[
    F(z) = \sum_{n=0}^{N-1} \frac{(-1)^n}{z^{n+1}} \int_0^\infty t^n e^{-t} \dif t + \frac{(-1)^N}{z^N} \int_0^\infty \frac{t^N e^{-t} \dif t}{z + t}.
\]
Обозначим второе слагаемое за $R_N$, тогда эта сумма будет равна
\[
    \sum_{n=0}^{N-1} \frac{(-1)^n n!}{z^{n+1}} + R_N.
\]
Значится нам остаётся доказать, что интеграл --- это $o(1)$ при $z \to \infty$.
И это довольно тривиально, так как $z$ там встречается лишь в знаменателе:
\[
    |R_N| \le \frac{1}{|z^N|} \int_0^\infty \frac{t^N e^{-t} \dif t}{z} = \frac{N!}{z^{N+1}} = O \left( \frac{1}{z^{N+1}} \right).
\]
Следовательно,
\[
    F \sim \sum_{n=0}^{\infty} \frac{(-1)^n n!}{z^{n+1}},
\]
хотя ряд в правой части нигде не сходится.

\textbf{Упражнение.}
\[
    \operatorname{Erf}(x) = \frac{2}{\sqrt \pi} \int_x^\infty e^{-t^2} \dif t \sim \sum_{n=0}^{\infty} c_n x^{-n},
\]
найти $c_n$ для $E = \mathbb R_+$.

\subsection{Лемма Ватсона}
Пусть
\[
    F(\lambda) = \int_0^a = \phi(t) e^{-\lambda t^\alpha} \dif t
\]
для $\alpha > 0$, $\lambda > 0$, $a \in (0, +\infty]$.
Дополнительно предположим, что
\begin{enumerate}
    \item $F(\lambda_0)$ сходится абсолютно для какого-то $\lambda_0 > 0$.
    \item $\phi$ голоморфна в $\overline {O_\delta(0)}$.
        Как следствие, её можно разложить в ряд Тейлора
        \[
            \phi(t) = \sum_{n=0}^{\infty} c_n t^n, |t| \le \delta.
        \]
\end{enumerate}

Тогда
\[
    F(\lambda) \sim \sum_{n=0}^{\infty} \frac{c_n}{\alpha} \Gamma \left( \frac{n+1}{\alpha} \right) \lambda^{-\frac{n+1}{\alpha}}, \lambda \to +\infty.
\]

\textbf{Доказательство.} Пусть $\lambda > \lambda_0$, оценим модуль интеграла
\[
    \left| \int_\delta^a \phi(t) e^{-\lambda t^\alpha} \dif t \right|.
\]
Заметим, что
\[
    e^{-\lambda t^\alpha} = e^{-\lambda_0 t^\alpha} e^{-(\lambda - \lambda_0) t^\alpha}.
\]
Как следствие, при $t \ge \delta$ модуль интеграла не превосходит
\[
    \exp(-(\lambda - \lambda_0) \delta^\alpha) \int_\delta^a |\phi(t)| e^{-\lambda_0 t^\alpha} \dif t \sim e^{-(\lambda - \lambda_0) \delta^\alpha} \int_\delta^a e^{-\lambda_0 t^\alpha} \dif t = O(e^{-\lambda \delta^\alpha}).
\]
Штука под $O$--большим меньше, чем функции, по которым мы раскладываем, поэтому этим интегралом можно пренебречь, и рассматривать только интеграл от нуля до $\delta$.

Немного арифметки:
\[
    \int_0^\infty t^n e^{-\lambda t^\alpha} \dif t = \int_0^\infty \left( \frac{s}{\lambda} \right)^{\frac{n}{\alpha}} e^{-s} \frac{\dif s}{\alpha \lambda} \left( \frac{s}{\lambda} \right)^{\frac{1}{\alpha} - 1} =
\]
\[
    = \frac{1}{\alpha} \left( \frac{1}{\lambda} \right)^{\frac{n+1}{\alpha}} \int_0^\infty s^{\frac{n+1}{\alpha} - 1} e^{-s} \dif s = \frac{1}{\alpha} \lambda^{-\frac{n+1}{\alpha}} \Gamma \left( \frac{n+1}{\alpha} \right).
\]

Ранее мы доказали, что интеграл можно рассматривать лишь от нуля до $\delta$, вместо $a$, а теперь докажем, что можно его взять до беконечности.
Зафиксируем $\beta > 0$, $x > 0$, тогда
\[
    \int_x^\infty t^\beta e^{-t} \dif t =
\]
(замена: $t - x = \tau$)
\[
    = e^{-x} \int_0^\infty (t + x)^\beta e^{-\tau} \dif \tau = e^{-x} \left( \int_0^x + \int_x^{\infty} \right) (\tau + x)^\beta e^{-\tau} \dif \tau \le
\]
\[
    \le e^{-x} \left((2x)^\beta \int_0^x e^{-\tau} \dif \tau + \int_x^{\infty} (2\tau)^\beta e^{-\tau} \dif \tau \right) \le 
\]
\[
    \le e^{-x} ((2x)^\beta + 2^\beta \Gamma(\beta + 1)) = O(e^{-x/2}).
\]

Теперь зафиксируем частичное разложение $\phi$ в ряд Тейлора
\[
    \phi(t) = \sum_{n=0}^{N-1} c_n t^n + t^N \widetilde \phi(t).
\]

Наконец, оценим $F(\lambda)$ при $\lambda > \lambda_0$
\[
    F(\lambda) = \sum_{n=0}^{N-1} c_n \int_0^\delta t^n e^{-\lambda t^\alpha} \dif t + \int_0^\delta t^N \widetilde \phi(t) e^{-\lambda t^\alpha} \dif t + O(e^{-\lambda \delta^\alpha}) =
\]
(Замена: $-\lambda t^\alpha = s$)
\[
    = \sum_{n=0}^{N-1} c_n \left( \int_0^{\lambda \delta^\alpha} s^{\frac{n+1}{\alpha} - 1} e^{-s} \dif s \right) \frac{1}{\alpha} \lambda^{-\frac{n+1}{\alpha}} + \int_0^\delta t^N \widetilde \phi(t) e^{-\lambda t^\alpha} \dif t + O(e^{-\lambda \delta^\alpha}) =
\]
\[
    \sum_{n=0}^{N-1} c_n \frac{1}{\alpha} \Gamma \left( \frac{n+1}{\alpha} \right) \lambda^{-\frac{n+1}{\alpha}} + O \left(e^{-\frac{\lambda \delta^\alpha}{2}} \right) + R_N,
\]
где $R_N$ --- интеграл второго слагаемого.

Оценим $R_N$:
\[
    |R_N| \le C \int_0^\delta t^N e^{-\lambda t^\alpha} \dif t \le C \int_0^\infty t^N e^{-\lambda t^\alpha} \dif t = \frac{C}{\alpha} \lambda^{-\frac{N+1}{\alpha}} \Gamma \left( \frac{N+1}{\alpha} \right),
\]
всё сошлось.

\QED

\section{Конформность}
Пусть функция $f$ дифференцируема в точке $z_0$, положим $w_0 = f(z_0)$.
Рассмотрим кривую $\gamma$, проходящую через точку $z_0$, пусть $\Gamma = f(\gamma)$.
Параметризуем эту кривую --- $\gamma: z(t)$, $t \in (0, 1)$, $z_0 = z(t_0)$ и положим $w(t) = f(z(t))$, тогда $w'(t_0) = f'(z_0) z'(t_0)$.
Разделим равенство и возьмём модуль
\[
    \left| \frac{w'(t_0)}{z'(t_0)} \right| = |f'(z_0)|.
\]

\textbf{Определение.} Число $|\frac{w'(t_0)}{z'(t_0)}|$ называется \textit{коэффициентом искажения}, и, как мы сейчас показали, оно не зависит от выбора кривой.

Более того, $\arg(w'(t_0)) = \arg(f'(z_0)) + \arg(z'(t_0))$, то есть отображение $f$ сохраняет углы между кривыми.

\textbf{Определение.} Отображение $O(z_0) \mapsto O(w_0)$ называется \textit{конформным}, если оно переводит гладкие кривые в гладкие кривые и сохраняет углы между кривыми.

\textbf{Теорема.} Пусть $D$ --- область, $f = u + iv$, где $u, v \in C^1(D)$.
Тогда $f$ конформна в точке $z_0$ тогда и только тогда, когда $f$ дифференцируема в $z_0$, и её производная не равна нулю.
Иными словами, $f'_{\overline z}(z_0) = 0$ и $f'_z(z_0) \ne 0$.

\textbf{Доказательство.} $\Leftarrow$ доказали в начале. $\Rightarrow$: зафиксируем кривую $\gamma$, её параметризацию $z(t) = x(t) + i y(t)$ и посчитаем:
\[
    w'(t_0) = f_x' x'(t) + f_y' y'(t),
\]
где $f_x' = u_x' + iv_x'$, $f_y' = u_y' + iv_y'$, или же это равняется
\[
    f_z' z'(t_0) + f_{\overline z}' \overline{z'(t_0)},
\]
где $f_z' = \frac{1}{2} (f_x' - i f_y')$ и $f_{\overline z}' = \frac{1}{2} (f_x' + if_y')$.
В итоге
\[
    \frac{w'(t_0)}{z'(t_0)} = f_z' + f_{\overline z}' \frac{\overline {z'(t_0)}}{z'(t_0)}.
\]
Это упражнение из задавальника. Чтобы правая часть была определена однозначно, нужно, чтобы второе слагаемое было равно нулю, то есть при $f_{\overline z}' = 0$.
Условие на $f_z' \ne 0$ нам нужно для того, чтобы производная не стала константой, так как иначе нарушится гладкость.

\QED

\subsection{Конформность в $\overline{\mathbb C}$}
Посмотрим, какие случаи не были покрыты в прошлом пункте: конечность преходит в бесконечность, бесконечность в конечность и бесконечность в бесконечность.

Пусть $f(a) = \infty$, $f$ голоморфна в $\dot O(a)$.
Тогда в точке $a$ имеется полюс, так что функция $g(z) = \frac{1}{f(z)}$ имеет устранимую особенность в точке $a$.
Конформность $f$ и $g$ эквивалентна, поэтому из предыдущего пункта мы хотим, чтобы $g'(a) \ne 0$, то есть у $g$ должен быть ноль первого порядка в точке $a$, значит, $f$ должна иметь полюс первого порядка.

Пусть $f(\infty) = A \in \mathbb C$, то есть $\infty$ --- устранимая особая точка функции $f$.
Похожим образом берём $g(\zeta) = f(\frac{1}{\zeta})$, теперь конформность $f$ в бесконечности эквивалента конформности $g$ в нуле, то есть $g'(0) \ne 0$.
Значится, мы хотим
\[
    0 \ne \lim_{\zeta \to 0} \left( \frac{g(\zeta) - A}{\zeta} \right) = \lim_{z \to \infty} \left( \frac{f(z) - f(\infty)}{1/z} \right) = -\res_\infty(f).
\]

И, наконец, последний случай: $f(\infty) = \infty$.
Здесь просто комбинируем два предыдущих случая:
\[
    g(\zeta) = \frac{1}{f(\frac{1}{\zeta})}.
\]
Получаем устранимую особенность в нуле, причём $g(0) = 0$, а конформность $f$ эквивалентна $g'(0) \ne 0$.
Это означает, что
\[
    0 \ne \lim_{\zeta \to 0} \left( \frac{g(\zeta) - 0}{\zeta} \right) = \lim_{z \to \infty} \left( \frac{z}{f(z)} \right) = \left( \lim_{z \to \infty} \left( \frac{f(z)}{z} \right) \right)^{-1}.
\]
Итак, конформность $f$ эквивалентна тому, что у неё полюс первого порядка в $\infty$.

\subsection{Дробно--линейные отображения}
Отображения вида $L(z) = \frac{az + b}{cz + d}$ для $a, b, c, d \in \mathbb C$, $ad - bc \ne 0$.

Свойства:
\begin{enumerate}
    \item Они образуют группу относительно композиции, упражнение из задавальника теории групп.
    \item Они являются конформными отображениями $\overline{\mathbb C} \mapsto \overline{\mathbb C}$ (а позже мы докажем и обратное вложение).

        Пусть $L$ --- ДЛО, тогда
        \[
            L'(z) = \frac{ad - bc}{(cz + d)^2},
        \]
        нулю не равно, так как в числителе как раз условие на невырожденность $L$.
        Но это только $\mathbb C$, а для $\overline{\mathbb C}$ нужно ещё рассмотреть два случая: $cz + d = 0$ и $z = \infty$, как в предыдущем пункте.

        Первый случай --- подставим $z = -\frac{d}{c}$ в $L$, получаем полюс первого порядка, подходит.
        Второй случай --- $L(\infty) = \frac{a}{c}$.
        Если $c \ne 0$, то будет устранимая особенность, засим нужно посчитать вычет
        \[
            \res_\infty(L) = \lim_{z \to \infty} \left( z \cdot \left( \frac{az + b}{cz + d} - \frac{a}{c} \right) \right) = \lim_{z \to \infty} \left( \frac{z(bc - ad)}{(cz + d)c} \right) = \frac{bc - ad}{c^2} \ne 0.
        \]
        А если $c = 0$, то $L$ --- это просто линейная функция, и у неё очевидным образом в бесконечности полюс первого порядка.

    \item 
        \textbf{Определение.} Окружность на $\overline{\mathbb C}$ --- это окружность в $\mathbb C$ или прямая (как предел окружностей очень большого радиуса).

        Круговое свойство --- ДЛО переводит окружности в $\overline{\mathbb C}$ в окружности.
        
        Для доказательства нам пригодится разложение ДЛО в три компоненты:
        \[
            L(z) = \frac{az + b}{cz + d} = \frac{a}{c} - \frac{ad - bc}{c(cz + d)} = L_1 \circ L_2 \circ L_3(z),
        \]
        где $L_1 = Az + B$, $L_2 = \frac{1}{z}$, $L_3 = z + \beta$, точные константы не особо важны.
        Теперь докажем, что каждое из $L_1, L_2, L_3$ переводит окружность в окружность.
        Для сдвига и для линейного преобразования очевидно, докажем для обратного.

        Рассмотрим общее уравнение окружности в $\overline{\mathbb C}$:
        \[
            E(x^2 + y^2) + F_1 x + F_2 y + G = 0
        \]
        при $E^2 + F_1^2 + F_2^2 > 0$.
        Сделаем замену координат $(x, y) \mapsto (z, \overline z)$:
        \[
            Ez \overline z + \overline F z + F \overline z + G = 0,
        \]
        где $F = \frac{F_1 + iF_2}{2}$.
        После преобразования $w = \frac{1}{z}$ это уравнение перейдёт в 
        \[
            G\overline w w + \overline F \overline w + Fw + E = 0,
        \]
        снова уравнение окружности.

    \item \textbf{Определение.} Точки $z$ и $z^*$ \textit{симметричны} относительно окружности $(z_0, R)$, если $\arg(z - z_0) = \arg(z^* - z_0)$ и $|z - z_0| \cdot |z^* - z_0| = R^2$.

        Аналогично симметрия относительно прямой.

        Свойство симметрии --- если точки $z$ и $z^*$ симметричны относительно окружности $\gamma$ в $\overline{\mathbb C}$, то $L(z)$ и $L(z^*)$ симметричны относительно $L(\gamma)$.

        Можно написать альтернативное определение симметрии: для любой $\gamma'$, содержащей $z$ и $z^*$, выполнено $\gamma' \bot \gamma$.
        Доказательство через школьную геометрию, а именно, так называемую теорему о секущей и касательной.
        Дальше пользуемся тем, что конформное отображение сохраняет углы и завершаем доказательство.

    \item Пусть $z_1, z_2, z_3 \in \mathbb C$ --- три различные точки, $w_1, w_2, w_3 \in \mathbb C$ --- ещё какие-то три различные точки.
        Тогда найдётся единственное ДЛО, переводящее $z_j$ в $w_j$.
        Рассмотрим
        \[
            L_1(z) = \frac{z - z_1}{z - z_2} \cdot \frac{z_3 - z_2}{z_3 - z_1}
        \]
        и
        \[
            L_2(w) = \frac{w - w_1}{w - w_2} \cdot \frac{w_3 - w_2}{w_3 - w_1}.
        \]
        Как нетрудно проверить, $L_2^{-1} \circ L_1$ подходит.

        Докажем единственность: пусть $L$ --- какое-то подходящее ДЛО.
        Рассмотрим $L_2 \circ L \circ L_1^{-1}$.
        Можно проверить, что оно переводит ноль в ноль, единицу --- в единицу, бесконечность --- в бесконечность.

        Раз бесконечность переходит в бесконечность, оно является линейным преобразованием $Az + B$.
        Раз ноль в ноль, $B = 0$.
        Единицу в единицу, $A = 1$.
        Получили тождественное преобразование.
\end{enumerate}

\section{Метод перевала}
\textbf{Теорема.} Пусть
\[
    F(\lambda) = \int_\gamma \phi(z) e^{\lambda f(z)} \dif z,
\]
где $\gamma$ --- кусочно гладкая кривая в $\mathbb C$, а $\phi, f: \gamma \to \mathbb C$ --- непрерывные функции.
Пусть дополнительно существует $z_0 \in \dot \gamma$ (внутренность $\gamma$), такое что:
\begin{itemize}
    \item $\re(f(z_0)) > \re(f(z))$ для всех $z \in \gamma \setminus \{z_0\}$.
    \item $f, \phi$ голоморфны в $O(z_0)$, причём $f'(z_0) = 0$, а $f''(z_0) \ne 0$.
\end{itemize}
Тогда
\[
    F(\lambda) \sim e^{\lambda f(z_0)} \sum_{n=0}^{\infty} \Gamma \left( n + \frac{1}{2} \right) a_{2n} \lambda^{-n - \frac{1}{2}}
\]
при $\lambda \to +\infty$.
В частности, главный член разложения --- это
\[
    \Gamma \left( \frac{1}{2} \right) a_0 = \phi(z_0) \sqrt{ \frac{2\pi}{-f''(z_0)} }.
\]
Здесь написано корень из комплесного числа, какое из двух значений нам нужно будет понятно по ходу доказательства.

\textbf{Доказательство.} Разложим $f$ в окрестности $z_0$:
\[
    f(z) = f_0 + \frac{f''(z_0)}{2} (z - z_0)^2 + \dots
\]
при $|z - z_0| \le \delta$.
Немного магии:
\[
    f_0 - f(z) = (z - z_0)^2 \widetilde f(z),
\]
причём $\widetilde f(\cdot)$ голоморфна при $|z - z_0| \le \delta$, а также $\widetilde f(z_0) = -\frac{f''(z_0)}{2} \ne 0$.
Иными словами, $\widetilde f(z) = c + c_1(z - z_0) + \dots$, где $c = -\frac{f''(z_0)}{2} \ne 0$.

Теперь нам хочется ввести новую переменную $w$, так что $w^2 = f_0 - f(z)$, но, как и в формулировке, тут проблема со взятием корня.
Так как $\widetilde f$ голоморфна, найдётся $\delta' < \delta$, такое что $|\widetilde f| > \frac{|c|}{2}$ в $O_{\delta'}(z_0)$.
Итак, засчёт того, что мы отделились от нуля, найдётся голоморфная ветвь логарифма $\ln(\widetilde f)$ в этой самой окрестности.
Наконец, положим 
\[
    w(z) = (z - z_0) \exp \left( \frac{1}{2} \ln(\widetilde f(z)) \right).
\]
Логарифм определён с точностью до $2 \pi k i$, поэтому $w$ определена с точностью до знака --- пока что не будем его фиксировать.

Разложим $w$ на вещественную и мнимые части: $w = u + iv$, $w^2 = u^2 - v^2 + 2i \cdot uv$.
Мнимая часть квадрата равна нулю тогда и только тогда, когда $u = 0$ или $v = 0$, а вещественная --- когда $|u| = |v|$.
Посмотрим, что в это время происходит с переменной $z$.
Функция $w$ голоморфна в $O_{\delta'}(z_0)$, причём
\[
    w'(z_0) = \exp \left( \frac{1}{2} \ln(\widetilde f(z_0)) \right) \in \sqrt c \ne 0.
\]
Раз не равна нулю, то существует обратная функция $z(w)$, такая что $z(0) = z_0$, а $z'(0) = \frac{1}{\sqrt c}$, опять же знак всё ещё не фиксирован.
И верно это в $\varepsilon$--окрестности $\{|w| \le \varepsilon\} \mapsto O_{\delta'}(z_0)$ (справа написана вся $\delta'$--окрестность, а не её сужение, как в теореме об обратной функции, ибо мы не требуем сюръекцию).

Найдём в этих координатах кривую $\gamma$ из условия.
На ней $\re(f(z)) < \re(f_0)$, то есть $\re(w^2) > 0$.
Более того, при $w \in (-\varepsilon, \varepsilon)$ выполнено $z'(0) = |z'(0)| e^{i\theta}$, и здесь, наконец, мы фиксируем знак: сделаем так, чтобы
\[
    \arg \left( \frac{\gamma'(t_0)}{z'(0)} \right) < \frac{\pi}{2},
\]
где $\gamma(t_0) = z_0$.
Разложим кривую $\gamma$ в две: $\gamma_0 = \{z(w)~|~ w \in [-\varepsilon, \varepsilon]\}$, $\gamma_1$ --- продолжение $\gamma$ с двумя дугами окружностей.
Рисунок 7 показывает, как это примерно выглядит.

\begin{figure}[ht]
    \centering
    \incfig{steepest-descent}{0.95\linewidth}
    \caption{Построение кривых}
\end{figure}

В силу компактности $\gamma_1$ найдётся $h > 0$, такое что $\re(f(z)) < \re(f(z_0)) - h$ на $\gamma_1$.
А именно, при $z$ на $\gamma$ это выполнено по условию, а при $z$ на дугах это верно засчёт того, что они вложены в секторы $\re(w^2(z)) > 0$.
Наконец, к исходному интегралу:
\[
    \left| \int_{\gamma_1} \phi(z) e^{\lambda f(z)} \dif z \right| = \left| e^{\lambda f_0} \int_{\gamma_1} \phi(z) e^{\lambda(f - f_0)} \dif z \right| \le
\]
\[
    \le \exp(\lambda \re(f_0)) \int_{\gamma_1} |\phi| \cdot e^{-\lambda h}\cdot |\dif z| = \exp(\lambda \re(f_0)) \cdot O(e^{-\lambda h}).
\]
По $\gamma_0$:
\[
    \int_{\gamma_0} \phi(z) e^{\lambda f(z)} \dif z\bigg|_{z = z(w)} = e^{\lambda f_0} \int_{-\varepsilon}^{\varepsilon} \phi(z(w)) z'(w) e^{-\lambda w^2} \dif w =
\]
Положим $\psi(w) = \phi(z(w)) z'(w)$ и сделаем замену:
\[
    = e^{\lambda f_0} \int_0^{\varepsilon} (\psi(w) + \psi(-w)) e^{-\lambda w^2} \dif w.
\]
Пусть
\[
    \psi(w) = \sum_{n=0}^{\infty} a_n w^n,
\]
причём 
\[
    a_0 = \phi(z_0) z'(0) = \frac{\phi(z_0)}{\sqrt{ -\frac{f''(z_0)}{2} }} = \frac{\phi(z_0)}{ \left| \frac{f''(z_0)}{2} \right|^{1/2}} e^{i\theta}.
\]
Тогда
\[
    \psi(w) + \psi(-w) = 2 \sum_{n=0}^{\infty} a_{2n} w^{2n},
\]
Наконец, применяя лемму Ватсона для $\alpha = 2$, получаем, что выражение выше раскладывается в
\[
    e^{\lambda f_0} \sum_{n=0}^{\infty} \Gamma \left( \frac{2n + 1}{2} \right) \frac{2a_{2n}}{2} \lambda^{\frac{-2n - 1}{2}}.
\]
По итогу интеграл по $\gamma$, как сумма интегралов по $\gamma_0$ и $\gamma_1$ (потому что интеграл по ``восьмёрке``, которая образовалась вокруг $z_0$, равен нулю, ибо это замкнутый контур), раскладывается в
\[
    e^{\lambda f_0} \left( \sum + O(e^{-\lambda h}) \right).
\]

\QED

\textbf{Замечание.} Смысл асимптотических рядов в том, что, даже если они не сходятся, при помощи них можно считать интегралы: мы сначала фиксируем количество членов в сумме, которыми будем делать оценку, а потом непосредственно оцениваем.

Обосновывать применимость данной теоремы очень тяжело, поэтому люди обычно либо не обосновывают (физики), либо по-быстрому считают ответ, а потом доказывают его корректность другим способом (математики).

\textbf{Пример.} (Формула Стирлинга) 
\[
    \Gamma(\lambda + 1) = \sqrt{2\pi \lambda} \left( \frac{\lambda}{e} \right)^\lambda \left(1 + O \left( \frac{1}{\lambda} \right) \right).
\]
Запишем интеграл
\[
    \int_0^\infty t^\lambda e^{-t} \dif t = \lambda^{\lambda + 1} \tau^{\lambda} e^{-\lambda \tau} \dif \tau = \lambda^{\lambda + 1} \int_0^{\infty} e^{\lambda f(\tau)} \dif \tau,
\]
где $f(\tau) = \ln(\tau) - \tau$, тогда $f'(\tau) = \frac{1}{\tau} - 1$ --- единственный ноль в единице, $f''(\tau) = \frac{-1}{\tau^2} < 0$.
Отсюда по методу перевала интеграл асимптотически равен
\[
    \lambda^{\lambda + 1} e^{\lambda f(1)} \sqrt{2\pi} \left(1 + O \left(\frac{1}{\lambda} \right) \right) \lambda^{-\frac{1}{2}}.
\]

\section{Конформные отображения}
\textbf{Определение.} Области $D_1$ и $D_2$ \textit{конформно эквивалентны}, если существует биекция $f: D_1 \to D_2$, такая что $f$ голоморфна и $f' \ne 0$.
Иными словами, $f$ однолистна.

\textbf{Замечание.} Конформная эквивалентность является отношением эквивалентности.

\textbf{Определение.} $\Aut(D)$ --- группа конформных отображений из $D$ на $D$.

Попытаемся найти группы конформный отображений для простых областей.
Обозначения: $\mathbb C_+ = \{z: \im(z) > 0\}$, $\mathbb D = \{z: |z| < 1\}$.

\textbf{Теорема.} 
\[
    \Aut(\mathbb D) = \left\{e^{i\theta} \cdot \frac{z - a}{1 - z \overline a}, a \in \mathbb D, \theta \in \mathbb R \right\}.
\]

\textbf{Доказательство.} Докажем, что все дробнолинейные автоморфизмы имеют такой вид.
Пусть $L \in \Aut(\mathbb D)$ --- ДЛО.
Тогда найдётся $a \in \mathbb D$, такое что $L(a) = 0$.
Так как $a$ симметрична $\frac{1}{\overline a}$ относительно единичной окружности, $\frac{1}{\overline a}$ переходит в $\infty$, ибо она симметрична нулю.
Засим
\[
    L(z) = A \frac{z - a}{z - 1/\overline a} = \varkappa \frac{z - a}{1 - \overline a z}.
\]
На границе круга $|z| = z \overline z = 1$, откуда
\[
    1 = |L(z)| = |\varkappa| \cdot \left| \frac{z - a}{\overline z - \overline a} \right| = |\varkappa|.
\]
Поэтому можно сделать замену $\varkappa = e^{i\theta}$ и получить искомую формулу.

В обратную сторону, пусть $f \in \Aut(\mathbb D)$, $a = f(0)$.
Тогда $|a| < 1$.
Рассмотрим отображение $L = \frac{z - a}{1 - \overline a z}$ и композицию $g = L \circ f$.
Тогда $g(0) = 0$, $|g| < 1$, и оно голоморфно в $\mathbb D$.

По лемме Шварца $|g(z)| \le |z|$, а если бы нам удалось доказать равенствов какой-то точке, то мы бы получили, что $g$ --- это поворот.
И это не так уж и сложно, просто применим лемму Шварца к $g^{-1}$, это нам сразу даёт равенство модулей во всех точках.
Следовательно, $g(z) = e^{i\theta} z$, то есть $L \circ f = e^{i\theta} z$ или же $f = L^{-1}(e^{i\theta} z)$ --- ДЛО.

\QED

\textbf{Упражнение.} Найти $\Aut(\mathbb C_+)$
Ответ ---
\[
    \left\{ \frac{az + b}{cz + d}~\bigg|~a, b, c, d \in \mathbb R, ad - bc > 0 \right\}.
\]
Действительно, все такие подходят, ибо прямую $\im(z) = 0$ они не трогают, а $\im( \frac{az + b}{cz + d}(i)) > 0$.
Обратное вложение делается переходом в $\mathbb D$: берём $L(z) = \frac{z - i}{z + i}$, тогда для любого автоморфизма $f$ верхней полуплоскости автоморфизм $g = L f L^{-1}$ является автоморфизмом единичного круга, то есть $g$ --- это ДЛО, то есть $f$ --- ДЛО.

\textbf{Теорема.} (Римана) Пусть $D$ --- односвязная область в $\overline{\mathbb C}$, $D \ne \mathbb C$, $D \ne \overline{\mathbb C}$ и $D \ne \overline{\mathbb C} \setminus \{a\}$ для $a \in \mathbb C$ (иными словами, граница $D$ содержит более одной точки).
Тогда для любого $z_0 \in D$ существует единственное конформное отображение $f: D \to \mathbb D$, такое что $f(z_0) = 0$ и $f'(z_0) > 0$.

\textbf{Доказательство.} Существование очевидно (без доказательства).
Докажем единственность: пусть $f_1, f_2$ подходят, тогда $g = f_1 \circ f_2^{-1} \in \Aut(\mathbb D)$.
Нам известно, что $g(0) = 0$ и $g'(0) = \frac{f_1'(z_0)}{f_2'(z_0)} > 0$.
Вспоминая, что $g(z) = e^{i\theta} \frac{z - a}{1 - z \overline a}$, подставим сюда всё, что мы знаем.
Раз $g(0) = 0$, то $g(z) = e^{i\theta} z$.
Раз производная в нуле больше нуля, то $e^{i\theta} > 0$, что выполнено только при $\theta = 0$.

\QED

\textbf{Теорема.} (Каратеодори, б/д) Пусть $D_1, D_2$ --- области, границы которых являются замкнутыми жордановыми кривыми, $f$ --- конформное отображение $D_1$ на $D_2$.
Тогда существует гомеоморфизм $F: \overline{D_1} \to \overline{D_2}$, такой что $F|_{D_1} = f$.

Иными словами, отображение можно гомеоморфно продолжить до границы области.
\subsection{Примеры конформных отображений}
\begin{itemize}
    \item Степенная функция $z^{\alpha}$, $\alpha > 0$, область определения --- $\mathbb C \setminus \mathbb R_+$.
        Так как $\ln(z) = \ln|z| + i \cdot \arg(z)$,
        \[
            z^\alpha = e^{\alpha \ln(z)} = \exp(\alpha(\ln|z| + i \cdot \arg(z))).
        \]
        Отсюда $\Arg(z^\alpha) = \alpha \cdot \arg(z) + 2\pi k$.
        Получается, что при $\alpha > 1$ такое отображение ``расширяет`` сектор $\arg(z) \in (0, \frac{2\pi}{\alpha})$ в $\mathbb C \setminus \mathbb R_+$, а при $\alpha < 1$ --- наоборот.

    \item Экспонента $e^z$.
        Она $2\pi i$--периодична, поэтому достаточно рассматривать полосу $\{\im(z) \in [0, 2\pi i)\}$.
        Посмотрим на кривую в полосе $z = x + iy$: её образом будет $e^z = e^x \cdot e^{iy}$
        Если $y$ --- константа, то получится прямая, а если $x$ --- то окружность.

    \item Функция Жуковского $f(z) = \frac{1}{2} \left(z + \frac{1}{z} \right)$.
        $f(0) = f(\infty) = \infty$ --- полюса первого порядка, то есть она там конформна.
        В остальных точках через производную:
        \[
            f'(z) = \frac{1}{2} \left(1 - \frac{1}{z^2} \right) = 0 \iff z = \pm 1.
        \]
        Посмотрим, какие точки эта функция склеивает: пусть $z_1 \ne z_2$, тогда
        \[
            f(z_1) - f(z_2) = \frac{1}{2}(z_1 - z_2) + \frac{1}{2} \frac{z_2 - z_1}{z_1 z_2} = 0 \iff z_1 z_2 = 1.
        \]
        Иными словами, нас интересуют области $D$, такие что $z \in D \Rightarrow \frac{1}{z} \not\in D$.
        Например, круг, диск, верняя полуплоскость.

        Посмотрим, куда переходит окружность: $z = r e^{i\phi}$, тогда
        \[
            f(z) = \frac{1}{2} \left(r e^{i\phi} + \frac{1}{r} e^{-i\phi} \right) = \frac{1}{2} \left(r + \frac{1}{r} \right) \cos(\phi) + \frac{i}{2} \left(r - \frac{1}{r} \right) \sin(\phi).
        \]
        Если записать это в виде $u + iv$, положить $a = \frac{1}{2} \left(r + \frac{1}{r} \right)$, $b = \frac{1}{2} \left(r - \frac{1}{r} \right)$, то $a^2 - b^2 = 1$, то есть мы получили эллипс $\frac{u^2}{a^2} + \frac{v^2}{b^2} = 1$ с фокусами $\pm 1$.

        Следовательно, $\mathbb D$ переходит в $\overline{\mathbb C} \setminus [-1, 1]$, туда же переходит $\overline{\mathbb C} \setminus \overline{\mathbb D}$.

        Найдём образ луча $z = r \cdot e^{i\phi}$.
        По формуле для окружности получаем
        \[
            \left( \frac{u}{\cos(\phi)} \right)^2 - \left( \frac{v}{\sin(\phi)} \right)^2 = 1
        \]
        --- гипербола с фокусами в $\pm 1$.
        Далее в зависимости от $\phi$ получаются разные ветви и направления.
        Как следствие, верхняя полуплоскость переходит в $\mathbb C \setminus ((-\infty, -1] \cup [1, +\infty))$.

        Интересный факт про функцию Жуковского, который любят спрашивать на экзамене: пусть $w = f(z)$, тогда
        \[
            \frac{w - 1}{w + 1} = \left( \frac{z-1}{z+1} \right)^2.
        \]
        Иными словами, $w(z) = L_1^{-1} \circ (\cdot)^2 \circ L_1$, где $L_1 = \frac{z - 1}{z + 1}$.
\end{itemize}

\section{Числа Белла и их асимптотика}
Пусть $A = \{a_1, \dots, a_n\}$ --- произвольное множество.
Будем рассматривать \textit{разбиения $A$ на классы}, то есть количество способов представить $A$ в виде неупорядоченного дизъюнктного объединения множеств.

\textbf{Определение.} \textit{Число Белла}, $B(n)$ --- это количество способов разбить $n$--элементное множество на классы.
Например, $B(1) = 1$, $B(2) = 2$, $B(3) = 5$.
Дополнительно доопределим $B(0) = 1$.

\textbf{Утверждение 1.}
\[
    B(n + 1) = \sum_{k=0}^{n} C_n^k B(k).
\]
Очевидно.

Рассмотрим экспоненциальную производящую функцию чисел Белла
\[
    G(z) = \sum_{n=0}^{\infty} \frac{B(n) z^n}{n!}.
\]
Тогда
\[
    G'(z) = \sum_{n=0}^{\infty} \frac{B(n + 1) z^n}{n!} = \sum_{n=0}^{\infty} \frac{z^n}{n!} \sum_{k=0}^{n} C_n^k B(k) = e^z G(z).
\]
Это уравнение с разделяющими переменными, засим, решая его, получаим $G(z) = e^{e^z - 1}$.

Вспонимая, что $\sum_{n=0}^{\infty} \frac{B(n) z^n}{n!}$ является рядом Лорана, мы можем выразить число Белла, как его член:
\[
    2\pi i \cdot \frac{B(n)}{n!} = \int_{\gamma_0} \frac{e^{e^z - 1} \dif z}{z^{n+1}}.
\]
Преобразуем формулу: пусть $F_n := \int_{\gamma_0} \frac{e^{e^z} \dif z}{z^n}$.
Тогда
\[
    F_n = \int_{\gamma_0} \exp \left(e^z - n \ln(z) \right) \dif z.
\]
Как следствие, $2\pi i \cdot B(n) \cdot e = n! \cdot F_{n+1}$, поэтому мы будем оценивать асимптотику $F_n$.
Положим $f(z) = e^z - n \ln(z)$, тогда $f'(z) = e^z - \frac{n}{z}$.
Найдём критические точки $f$ для метода перевала, то есть нули производной --- решения трансцендентного уравнения $z e^z = n$.

Изучим вещественные решения.
Так как $(xe^x)' = (x + 1)e^x$, эта функция является строго возрастающей при $x > -1$, поэтому при $x > -1$ решение существует и единственно.
Пусть $x e^x = t$, где $t >> 1$, тогда $x = \ln(t) - \ln(t)$.
При $x > 1$ это значит, что $1 < x < \ln(t)$, обозначим $x = \ln(t) - \tilde x$.
Тогда
\[
    \ln(t) - \tilde x = \ln(t) - \ln(\ln(t) - \tilde x),
\]
откуда
\[
    \tilde x = \ln(\ln(t) - \tilde x) \le \ln(\ln(t)).
\]
Значит,
\[
    \ln(t) - \ln(\ln(t)) < x < \ln(t).
\]
Хотелось бы продолжить это рассуждение, но дальше пойдут отрицательные степени, так что эта ветвь не доводится.
Однако теперь видно, как примерно ведёт себя $x$.
Как утверждение без доказательства, вещественное решение уравнения $xe^x = n$ имеет бóльшую высоту, чем комплексные, то есть в методе перевала нас интересует только оно.

\textbf{Лемма 0.} Пусть у уравнения $f(v) = 0$ существует единственное решение в $O_r(0)$.
Тогда оно находится по формуле
\[
    v = \frac{1}{2\pi i} \int_{\gamma_r} \frac{f'(z)}{f(z)} z \dif z.
\]

\textbf{Доказательство.} Действительно, сей интеграл равен
\[
    \res \left( \frac{f'(z)}{f(z)} z \right) = \frac{f'(v) v}{f'(v)} = v,
\]
так как это полюс первого порядка.

\QED

\textbf{Лемма.} Рассмотрим уравнение $xe^x = t$, где $t >> 1$, $x$ --- единственное его решение.
Тогда
\begin{itemize}
    \item Существует $t_0$, такое что для всех $t > t_0$ найдётся $v \in (0, \pi)$, такое что $x = \ln(t) - \ln(\ln(t)) + v$, то есть оценка, полученная выше, асимптотически точна.
    \item
        \[
            v = \sum_{m=0}^{\infty} \sum_{k=0}^{\infty} c_{m,k} \frac{(\ln(\ln(t)))^{m+1}}{(\ln(t))^{k+m+1}}.
        \]
\end{itemize}

\textbf{Доказательство.} Пусть $\alpha = \frac{1}{\ln(t)}$, $\beta = \frac{\ln(\ln(t))}{\ln(t)}$, положим $v = x - \ln(t) + \ln(\ln(t))$.
Найдём уравнение на $v$: мы знаем, что $x = \ln(t) - \ln(x)$, подставим сюда $v$:
\[
    \ln(t) - \ln(\ln(t)) + v = \ln(t) - \ln(\ln(t) - \ln(\ln(t)) + v) =
\]
\[
    = \ln(t) - \ln(\ln(t)) - \ln \left( 1 + \frac{-\ln(\ln(t)) + v}{\ln(t)} \right).
\]
Сократим:
\[
    v = -\ln \left(1 + \frac{-\ln(\ln(t)) + v}{\ln(t)} \right).
\]
Проэкспоненциируем:
\[
    e^{-v} = 1 + \alpha v - \beta.
\]
Здесь можно применить теорему Руше: берём $f(z) = e^{-z} - 1$, $(f + g)(z) = e^{-z} - 1 - \alpha z + \beta$, корнями $f$ будут
\[
    e^{-z} = 1 \iff z = 2\pi k i, k \in \mathbb Z.
\]
Значит, если $|z| = \pi$, то $f$ отделима от нуля: пусть $|e^{-z} - 1| > \delta$ для таких $z$.
Немного уменьшим $\delta$: $\delta_1 := \frac{\delta}{2(\pi + 1)}$.
Тогда для всех $\alpha, \beta \in O_{\delta_1}(0)$ выполнено $|\alpha z - \beta| < \frac{\delta}{2}$ для $|z| = \pi$.

Итак, можно применить теорему Руше для круга радиуса $\pi$: количество корней $f + g$ равно количеству корней $f$, а он ровно один --- $z = 0$.
По лемме 0
\[
    v = \frac{1}{2\pi i} \int_{\gamma_\pi} \frac{(-e^{-z} - \alpha) z \dif z}{e^{-z} - 1 - \alpha z + \beta}.
\]
Заметим, что
\[
    (e^{-z} - 1 - \alpha z + \beta)^{-1} = \sum_{k=0}^{\infty} \frac{(\alpha z - \beta)^k}{(e^{-z} - 1)^{k+1}},
\]
на окружности $\gamma_\pi$, здесь просто арифметика.
Более того, ряд сходится абсолютно в силу оценок, которые мы делали для применения теоремы Руше.
Продолжая равенство, получаем
\[
    \sum_{k=0}^{\infty} \sum_{m=0}^{\infty} \frac{(\alpha z)^k \beta^m (-1)^m C_{m+k}^k}{(e^{-z} - 1)^{m+k+1}} =
    \sum_{k=0}^{\infty} \sum_{m=0}^{\infty} \frac{(\alpha z)^k \beta^m \alpha_{m,k}}{(e^{-z} - 1)^{m+k+1}},
\]
где $\alpha_{m,k}$ --- просто обозначение для константы.
Подставим всё это в интеграл:
\[
    v = \frac{1}{2\pi i} \sum_{k=0}^{\infty} \sum_{m=0}^{\infty} \alpha^k \beta^m \int_{\gamma_\pi} \frac{\alpha_{m,k} (-e^{-z} - \alpha) z^{k+1} \dif z}{(e^{-z} - 1)^{m+k+1}} =
\]
\[
    = \sum_{k=0}^{\infty} \sum_{m=0}^{\infty} \alpha^k \beta^m \widetilde\alpha_{m,k}.
\]
И это почти то, что нам нужно, то остались ещё нулевые степени.
Докажем, что $\widetilde \alpha_{0,k} = 0$.
\[
    \widetilde \alpha_{0,k} = \int_{\gamma_\pi} \frac{(e^{-z} - \alpha) z^{k+1} \dif z}{(e^{-z} - 1)^{k+1}} = 0,
\]
так как эта функция голоморфна в круге: единственная потенциальная особенность находится в нуле, но в нуле порядок нуля знаменателя равен порядку нуля числителя, так что проблемы не возникают.

\QED

\textbf{Утверждение.}
\[
    F_n = \int_{x - i\infty}^{x + i\infty} \exp(e^z - n \ln(z)) \dif z.
\]

\textbf{Доказательство.} Будем расширять окружность и спрямлять её на прямой $\re(z) = x$:
\begin{figure}[ht]
    \centering
    \incfig{bell_fn}{0.75\linewidth}
    \caption{Спрямление окружности}
\end{figure}
Пусть $\Gamma_R = \partial O_R(0) \cap \{\re(z) < x\}$, $I_R = [-r \sqrt{R^2 - x^2}, i\sqrt{R^2 - u^2}]$.
Тогда
\[
    F_n = \left( \int_{\gamma_R} + \int_{I_R} \right) \frac{\exp(e^z)}{z^n} \dif z.
\]
Оценим интеграл на $\Gamma_R$:
\[
    |\exp(e^z)| = \exp(\re(e^z)) \le \exp(|e^z|).
\]
TODO: дальше не успел, но здесь хотя бы есть запись.

\QED

Запишем $e^z - n \ln(z)$ в другом виде:
\[
    e^z - n \ln(z) = e^{x + iy} - xe^x \ln(x + iy) = e^x (1 - x \ln(x)) + e^x \left(e^{iy} - 1 - x \ln \left(1 + \frac{iy}{x} \right) \right),
\]
так как $n = xe^x$, ибо $x$ является решением уравнения.

Обозначим
\[
    f(y) = e^{iy} - 1 - x \ln \left(1 + \frac{iy}{x} \right),
\]
к ней и будем применять метод перевала.
В этих обозначениях
\[
    F_n = i \exp(e^x (1 - x \ln(x))) \int_{-\infty}^{+\infty} \exp(e^x f(y)) \dif y.
\]
Что мы знаем про $f$? $f(0) = 0$, 
\[
    f'(y) = i \left( e^{iy} - \frac{1}{1 + \frac{iy}{x}} \right), f'(0) = 0.
\]
\[
    f''(y) = i \left( ie^y + \frac{i/x}{\left(1 + \frac{iy}{x} \right)^2} \right), f''(0) = -\left(1 + \frac{1}{x} \right) < 0.
\]
Следовательно, ноль --- точка максимума.

\textbf{Лемма 2.} Для любого $\delta > 0$ найдётся $A$, такое что
\[
    F(x) := \int_\delta^{+\infty} \exp(e^x f(y)) \dif y = \mathcal O(\exp(-Ae^x))
\]
при $x \to \infty$.

\textbf{Доказательство.} Заметим, что
\[
    \re(f) = \cos(y) - 1 - \frac{x}{2} \ln \left(1 + \frac{y^2}{x^2} \right).
\]
Теперь рассмотрим случаи.
Первый: $x > y > \delta$, тогда
\[
    \ln \left(1 + \frac{y^2}{x^2} \right) > \frac{y^2}{2x^2},
\]
значит,
\[
    \left| \int_\delta^{x} \exp(e^x f(y)) \dif y \right| \le \left| \int_{\delta}^x \exp\left( -e^x \cdot \frac{x}{2} \cdot \frac{y^2}{2x^2} \right) \dif y \right| \le 
\]
\[
    \le \left| x \exp\left(-e^x \frac{\delta^2}{4x} \right) \right| = \mathcal O(\exp(-e^x A)).
\]
Теперь при $y > x > \delta$, здесь 
\[
    1 + \frac{y^2}{x^2} \ge \frac{2y}{x}.
\]
Тогда
\[
\left| \int_x^\infty \exp(e^x f(y)) \dif y \right| \le x \int_1^\infty \exp \left(e^x \left( -\frac{x}{2} \ln(2t) \right) \right) \dif t =
\]
\[
    = x \int_1^\infty (2t)^{-\frac{x}{2} e^x} \dif t = \mathcal O \left(x \exp\left(-\frac{xe^x}{4} \right) \right)
\]
по оценке
\[
    \int_1^\infty (2t)^{-p} \dif t = \mathcal O(e^{-\frac{p}{2}})
\]
для $p > 2$.

Остаётся рассмотреть маленькие $y$, то есть $y \in O_\delta(0)$.
Разложим $f$ в ряд Тейлора с центром в нуле
\[
    f(y) = -\frac{1}{2} \left(1 + \frac{1}{x} \right) y^2 \left(1 + a_1 y + a_2 y^2 + \dots \right),
\]
где $1 + a_1y + a_2y^2 + \dots = g \left(y, \frac{1}{x} \right)$.
Теперь немного поточнее:
\[
    x \ln \left(1 + \frac{iy}{x} \right) = \sum_{n=1}^{\infty} \frac{(iy)^n (-1)^n}{x^{n-1}n}.
\]
Значит, $g$ раскладывается по неотрицательным степеням $y$ и $\frac{1}{x}$.
Уменьшим $\delta$ при необходимости, чтобы $g$ была близка к единице (так как $g(0, \frac{1}{x}) = 1$), то есть $|g - 1| < \frac{1}{2}$ при $|y| < \delta$ и $|x| > x_0$.

Теперь мы хотим извлечь корень: положим
\[
    w = y \exp \left( \frac{1}{2} \ln(g) \right),
\]
тогда $f(y) = -\frac{1}{2} \left(1 + \frac{1}{x} \right) w^2(y)$.
Обозначим за $\tilde f$ второй множитель (корень), то есть $w = y \cdot \tilde f(y, \frac{1}{x})$.
Так как $\frac{\dif w}{\dif y}(0) = 1 \ne 0$, можно применить теорему об обратной функции.
Более того, как уже доказывалось ранее, решение трансцендентного уравнения записывается в виде
\[
    y \left(w, \frac{1}{x} \right) = \frac{1}{2\pi i} \int_{\gamma_0} \frac{(\eta \tilde f(\eta, \frac{1}{x}))'_\eta \dif \eta}{\eta \tilde f(\eta, \frac{1}{x}) - w}.
\]
Эта страшная формула нужна для того, чтобы сказать, что
\[
    y'\left(w, \frac{1}{x} \right) = 1 + \gamma_1 \left(\frac{1}{x} \right) w^1 + \gamma_2 \left( \frac{1}{x} \right) w^2 + \dots,
\]
где $\gamma_i$ --- ряды, сходящиеся абсолютно при $|u| > u_0$.
Выделим окрестность $(-\delta_1, \delta_2)$ такое, чтобы $w$ лежала в $(-\varepsilon, \varepsilon)$.
Тогда
\[
    \int_{-\delta_1}^{\delta_2} \exp(e^x f(y)) \dif y = \int_{-\varepsilon}^{\varepsilon} \exp \left( e^x \left(-\frac{1}{2} \right) \left(1 + \frac{1}{x} \right) w^2 \right) y'(w) \dif w.
\]
По лемме Ватсона это асимптотически равняется
\[
    \sum_{n=0}^{\infty} \gamma_{2n} \left( \frac{1}{x} \right) \Gamma\left(n + \frac{1}{2} \right) U^{-n - \frac{1}{x}},
\]
где $U = \frac{1}{2} e^x \left(1 + \frac{1}{x} \right)$ --- показатель экспоненты без $w^2$.

Наконец,
\[
    B(n) = \frac{n!}{2\pi e} F_{n+1} =
\]
\[
    = \frac{n!}{2\pi e} \exp (e^x (1 - x \ln(x))) \sum_{k=0}^{\infty} \gamma_{2k} \left( \frac{1}{x} \right) \Gamma \left(k + \frac{1}{2} \right) \left(\frac{1}{2} \left(1 + \frac{1}{x} \right) e^x \right)^{-k - \frac{1}{2}}.
\]
На асимптотику влияет лишь первый член, так что
\[
    B(n) \sim \frac{n!}{2\pi e} \exp(e^x(1 - x \ln(x))) \sqrt{2\pi} \left(\left(1 + \frac{1}{x} \right) e^x \right)^{-\frac{1}{2}} (1 + \mathcal O(e^{-x})),
\]
где $x$ --- вещественное решение уравнения $x e^x = n + 1$.

\textbf{Следствие.}
\[
    \frac{1}{n} \ln(B(n)) \sim \ln(n) - \ln(\ln(n)) - 1 + \frac{\ln(\ln(n))}{\ln(n)} + \frac{1}{\ln(n)} + \frac{1}{2} \left( \frac{\ln(\ln(n))}{\ln(n)} \right)^2.
\]
Для сравнения
\[
    \frac{1}{n} \ln(n!) \sim \ln(n) - 1.
\]

\section{Лемма Жордана}
Самое время её доказать.
Пусть $g$ непрерывна в $\{z: \im(z) \ge 0, |z| \ge R_0\}$, а также $\max_{\Gamma_R} |g| \to 0$ при $R \to \infty$, $\alpha > 0$.
Тогда
\[
    \lim_{R \to \infty} \left( \int_{\Gamma_R} g(z) e^{i \alpha z} \dif z \right) = 0.
\]

\textbf{Доказательство.} Докажем, что интеграл
\[
    \int_{\Gamma_R} |e^{i\alpha z}| \cdot |\dif z|
\]
ограничен.
Распишем:
\[
    \int_{\Gamma_R} |e^{i\alpha z}| \cdot |\dif z| = R \int_0^\pi e^{-\alpha R \sin(\phi)} \dif \phi = 2R \int_0^{\pi/2} e^{-\alpha R \sin(\phi)} \dif \phi \le
\]
Из вогнутости синуса при $\phi \in (0, \frac{\pi}{2})$ выполнено $\sin(\phi) \ge \frac{2\phi}{\pi}$.
А именно,
\[
    \left(1 - \frac{\phi}{\pi/2} \right) \sin(0) + \frac{\phi}{\pi/2} \sin \left( \frac{\pi}{2} \right) \le \sin \left( 0 + \frac{\phi}{\pi/2} \cdot \frac{\pi}{2} \right).
\]
Продолжаем:
\[
    \le 2R \int_0^{\pi/2} e^{-\alpha R \frac{2\phi}{\pi}} \dif \phi = \frac{2R \pi}{2\alpha R} \int_0^{\alpha R} e^{-t} \dif t \le,
\]
сделав замену $t = 2\phi \cdot \frac{\alpha R}{\pi}$,
\[
    \le \frac{\pi}{\alpha} \int_0^\infty e^{-t} \dif t = \frac{\pi}{\alpha}.
\]
Наконец,
\[
    \left| \int_{\Gamma_R} g e^{i\alpha z} \dif z \right| \le \max_{\Gamma_R} |g| \int_{\gamma_R} |e^{i\alpha z}| \cdot |\dif z| \to 0.
\]

\QED
