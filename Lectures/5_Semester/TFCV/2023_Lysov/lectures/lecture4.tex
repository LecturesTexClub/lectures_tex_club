% Лекция от 9 октября.

\begin{proof}
	Рассмотрим случаи взаимного расположения $a$ и $\ol \triangle$ (замыкания треугольника).
	\begin{enumerate}
		\item Если $a \not\in \ol \triangle$. Проведем средние линии, получим $\triangle^{(1)}, \triangle^{(2)}, \triangle^{(3)}, \triangle^{(4)}$. Обозначим интеграл по границе треугольника (любого) как $I(\triangle')$:
		\[
			I(\triangle') = \int_{\partial \triangle'} f(z) dz.
		\]
		
		Тогда \begin{align}
			\label{gurse_middle_lines}
			I(\triangle) = I\jleft( \triangle^{(1)} \jright) + I\jleft( \triangle^{(2)} \jright) + I\jleft( \triangle^{(3)} \jright) + I\jleft( \triangle^{(4)} \jright)
		\end{align}
		(к этому равенству есть комментарий после доказательства, посмотрите).

		Оценим модуль интеграла по треугольнику (с помощью неравенства треугольника для модуля):
		\[
			\begin{aligned}
				\mds{I(\triangle)} & = && \mds{I\jleft( \triangle^{(1)} \jright) + I\jleft( \triangle^{(2)} \jright) + I\jleft( \triangle^{(3)} \jright) + I\jleft( \triangle^{(4)} \jright)} & \leq & \\
				& && \mds{I\jleft( \triangle^{(1)} \jright)} + \mds{I\jleft( \triangle^{(2)} \jright)} + \mds{I\jleft( \triangle^{(3)} \jright)} + \mds{I\jleft( \triangle^{(4)} \jright)} & \leq & \\
				& && 4 \mds{I\jleft( \triangle^{(k)} \jright)} && ,
			\end{aligned}
		\]
		где $k$ -- индекс треугольника с наибольшим по модулю интегралом по границе $I\jleft( \triangle^{(k)} \jright)$. Т.е.
		\[
			I \jleft( \triangle^{(k)} \jright) \geqslant \frac{1}{4} I(\triangle).
		\]
		Обозначим $\triangle_1 = \triangle$, $\triangle_2 = \triangle^{(k)}$, $\triangle_3$ и далее построим итеративно таким же образом из предыдущего треугольника в последовательности. (Прим. автора: можно ещё то же самое так сказать: строим последовательность $\triangle_n$, итеративно выбирая из разбиения средними линиями треугольник с максимальным по модулю интегралом по границе, $\triangle_1 = \triangle$.)
		
		Получили последовательность вложенных треугольников:
		\[
			\triangle_1 \subset \triangle_2 \subset \ldots \subset \triangle_n \subset \ldots
		\]
		
		В силу компактности $\ol \triangle$ (прим. автора: он компактен, потому что замкнут и ограничен, при замыкании треугольника, открытый он или замкнутый, мы получаем треугольник с границами; после теоремы есть комментарий, как мы из компактности получаем это свойство {\color{red} написать, не забыть; там будет ссылка на то, что есть эквивалентное определение компактности, номер 4 \href{https://en.wikipedia.org/wiki/Compact_space#Characterization}{тут}, подмножества, которое подойдет нам.}):
		\[
			\exists z_0 \in \bigcup_{n = 1}^\infty \triangle_n.
		\]
		
		Т.к. $f$ дифференцируема в $z_0$, то по определению дифференцируемости
		\[
			\begin{aligned}
				f: & (\forall z \in \Cm) \,\, f(z) = f(z_0) + f'(z_0) (z - z_0) + o(z - z_0), \\
				o(z - z_0) : & (\forall \epsilon > 0) (\exists \delta_0(\epsilon) > 0) \jleft( \forall z \in B_{\delta_0}(z_0) \jright) \,\, \mds{o(z - z_0)} \leq \epsilon \mds{z - z_0}.
			\end{aligned}
		\]
		(Прим. автора: причем $o(z - z_0) = f(z) - f(z_0) - f'(z_0) (z - z_0)$, т.е. мы знаем, как она выглядит, просто на неё есть ограничения.)
		{\color{red} Использовать здесь наше определение дифференцируемости.}
		
		Рассмотрим интеграл по границе для произвольного $\triangle_n$.
		\[
			\int_{\triangle_n} f(z) dz = \int_{\triangle_n} f(z_0) dz + \int_{\triangle_n} f'(z_0) (z - z_0) dz + \int_{\triangle_n} o(z - z_0) dz.
		\]
		Первые два интеграла равны нулю, поскольку интеграл берется по замкнутой кривой и у интегрируемых функций есть первообразные (т.е. под интегралами полные дифференциалы):
		\[
			(z f(z_0))' = f(z_0), \jleft( f'(z_0) \frac{(z - z_0)^2}{2} \jright)' = f'(z_0) (z - z_0).
		\]
		Тогда
		\[
			\int_{\triangle_n} f(z) dz = \int_{\triangle_n} o(z - z_0) dz.
		\]
		
		Зафиксируем $\epsilon > 0$. Возьмём $N$ такое, что
		\[
			(\forall z \in \partial \triangle_N) \,\, \mds{z - z_0} < \delta_0(\epsilon).
		\]
		Тогда по определению $o$-малого
		\[
			(\forall z \in B_{\delta_0(\epsilon)}(z_0)) \,\, \mds{o(z - z_0)} \leq \epsilon \mds{z - z_0}.
		\]
		(Прим. автора: т.е. можем оценить значения функции, скрытой под знаком $o$-малого, на границе треугольника, т.к. граница входит в рассматриваемый шар.)
		Можем оценить интеграл по границе $\triangle_N$:
		\[
			\begin{aligned}
				\mds{\int_{\partial \triangle_N} f(z) dz} = \mds{\int_{\triangle_N} o(z - z_0) dz} \leq \\
				\max_{z \in \partial \triangle_N} \mds{o(z - z_0)} \mds{\partial \triangle_N} \leq \\
				\max_{z \in \partial \triangle_N} \epsilon \mds{z - z_0} \mds{\partial \triangle_N} = \epsilon \max_{z \in \partial \triangle_N} \mds{z - z_0} \mds{\partial \triangle_N} \leq \\
				\epsilon \mds{\partial \triangle_N} \mds{\partial \triangle_N}.
			\end{aligned}
		\]
		(Прим. автора: второй переход в произошел по свойству о неравенствах интеграла по кривой, третий переход произошел потому, что у нас есть оценка на значения функции за $o$-малым для границы, {\color{red} объяснить далее}.)
		
		Перед нами $\mds{\partial \triangle_N}$, это же длина кривой. Т.е. периметр треугольника. А при переходе к одному треугольников, составленных средними линиями, длины сторон уменьшаются в два раза (для центрального средние линии -- половины длин; для боковых треугольников две стороны исходного стали в два раза меньше, средняя линия создает сторону длиной в два раза меньше {\color{red} нужна картинка со средними линиями в треугольнике, можно не понять}.) Значит, мы знаем периметр по сравнению с исходным периметром. Обозначим периметр произвольного треугольника как $P(\triangle')$.
		\[
			P(\triangle_N) = \frac{1}{2^N} P(\triangle).
		\]
		Тогда
		\[
			\begin{aligned}
				\frac{1}{4^N} \mds{I(\triangle)} \leq \mds{I(\triangle_N)} \leq \epsilon \frac{P_0}{4^N}, \\
				\mds{I(\triangle)} \leq \epsilon P_0.
			\end{aligned}
		\]
		Исходный интеграл $\mds{I(\triangle)}$ можно сделать сколь угодно маленьким. Значит, он равен нулю.
		\item {\color{red} Ещё есть случаи, доказать потом..}
	\end{enumerate}
\end{proof}
\begin{anote}
	По всей видимости в теореме треугольник может быть и без границы, с частичной границей, может быть внутренность треугольника. Иначе мы бы не говорили про замыкание.
\end{anote}
\begin{anote}
	{\color{red} Здесь будет картинка для \ref{gurse_middle_lines}.}
\end{anote}
% Нарисовать картинку с треугольниками в tikz по координатам, объяснить, как получается равенство из свойств интеграла по кривой: надо воспользоваться аддитивностью.

% Высота в треугольнике не больше максимальной по длине стороны. Можем нарисовать множество точек, удаленных на не более, чем d, от стороны. Гарантированно включим весь треугольник, т.к. у него высота не больше.

\begin{lemma}
	Если $D$ -- выпуклая область (прим. автора: выпуклая, т.е. для любых двух точек области отрезок между ними тоже содержится), $f$ непрерывна на $D$, $(\forall \triangle \text{ -- треугольник}: \ol \triangle \subset D) \,\, \int_{\partial \triangle} f(z) dz = 0$, то $f(z) dz$ -- полный дифференциал (у $f$ есть первообразная).
\end{lemma}
\begin{proof}
	Зафиксируем точку $a$ в $D$. Мы снова от неё будем откладывать кривые. На этот раз -- отрезки. Введем $F(z)$ как интеграл от $f$ по отрезку от $a$ до $z$: 
	\[
		F(z) = \int_{[a, z]} f(z) dz.
	\]
	Хотим сказать, что $F$ -- первообразная $f$. Зафиксируем точку $z_0 \in D$ (прим. автора: она может и совпадать с $a$, $F$ будет равна нулю, в случае выбора другой точки -- будет отличаться на константу из-за линейности -- ожидаемо). Проверим, что $F'(z_0) = f(z_0)$.
	\[
		\int_{[a, z_0] \cup [z_0, z] \cup [z, a]} f(z) dz = \int_{[a, z_0]} f(z) dz + \int_{[z_0, z]} f(z) dz + \int_{[z, a]} f(z) dz.
	\]
	{\color{red} Продолжение следует.}
\end{proof}
\begin{anote}
	Если лемма Гурса была похожа на пункт $1$ теоремы о связи наличия первообразной и интеграла по кривой \ref{full_differential_curve_integral_relation} (я бы мог её так назвать), то эта лемма в каком-то смысле пункт $2$ этой теоремы.
		
\end{anote}