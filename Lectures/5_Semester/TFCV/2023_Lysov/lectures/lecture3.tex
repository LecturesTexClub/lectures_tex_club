% Лекция от 16 сентября.

\section{Интегрирование}

На лекции по ТФКП мы не определяли снова, что такое кривая. Дальше идет напоминание материала первого курса матана.

\begin{definition}
	Пусть $\alpha, \beta \in \R$, $\alpha \leq \beta$, $z: [\alpha, \beta] \to \Cm$, $(\forall t \in [\alpha, \beta]) \,\, z = x(t) + i y(t)$, $x$ и $y$ непрерывны; $z$ называется \it{параметризацией кривой}. (Модифицировано автором.)
\end{definition}

\begin{definition}
	Пусть $z_1: [\alpha_1, \beta_1] \to \Cm$, $z_2: [\alpha_2, \beta_2] \to \Cm$, $z_1$ и $z_2$ -- параметризации кривых; $z_1$ и $z_2$ эквивалентны, если существует непрерывная функция $\varphi: [\alpha_1, \beta_1] \to [\alpha_2, \beta_2]$ такая, что $z_1 = z_2 \circ \varphi$.
\end{definition}

\begin{anote}
	Получилось отношение эквивалентности: рефлексивно, достаточно взять тождественное отображение, оно непрерывно; симметрично, т.к. $\varphi^{-1}$ существует и непрерывна по свойствам непрерывных функций из $\R$ в $\R$, $z_1 \circ \varphi^{-1} = z_2 \circ \varphi \circ \varphi^{-1} = z_2$; транзитивно: если $z_1 = z_2 \circ \varphi_1$, $z_2 = z_3 \circ \varphi_2$, то $z_1 = (z_3 \circ \varphi_2) \circ \varphi_1 = z_3 \circ (\varphi_2 \circ \varphi_3)$, при этом $\varphi_2 \circ \varphi_3$ непрерывна как композиция непрерывных функций.
\end{anote}

\begin{definition}
	Класс эквивалентности параметризаций кривых по определенному выше отношению эквивалентности называется кривой.
\end{definition}

\begin{anote}
	У любых двух эквивалентных параметризаций одно и то же множество значений: $(\forall t \in [\alpha_1, \beta_1]) \, \, z_1(t) = (z_2 \circ \varphi)(t)$, тогда $z_1([\alpha_1, \beta_1]) = z_2(\varphi([\alpha_1, \beta_1])) \subset z_2([\alpha_2, \beta_2])$. Можно показать включение в другую сторону благодаря симметричности определенного нами отношения.
\end{anote}

\begin{definition}
	Пусть $z: [\alpha, \beta] \to \Cm$, $z$ -- параметризация кривой; \it{образом кривой} называется $z([\alpha, \beta])$.
\end{definition}

Дальше идет материал лекции.

{\color{red} Сказать, что такое минус кривая: это обход её в другом направлении, разворот параметризации; возможно, у нас на лекции кривые ориентированные были, в отношении эквивалентности должны быть неубывающие непрерывные. Определялось ли на матане это -- не знаю. }.

Пусть $D$ -- область в $\Cm$; $w: [\alpha, \beta] \to \Cm$, $w$ -- параметризация кривой $\gamma$, $\gamma \in \Cm$ (прим. автора: подразумевается, что образ кривой лежит в области; кривую считаем в том числе и множеством точек); $f: D \to \Cm$, $f$ непрерывна в $D$, $f(x + iy) = u(x, y) + iv(x, y)$ для любых $x, y \in \R$.

Возьмём разбиение $P = (t_0, \ldots, t_n)$ отрезка $[\alpha, \beta]$ (прим. автора: разбиение определяли на матане, во втором семестре; $t_0 = \alpha$, $t_n = \beta$, $\jleft( \forall k \in \ol{1, n} \jright) \,\, t_{k - 1} < t_k$; в разбиении скобочки, потому что порядок точек важен). Рассмотрим интегральную сумму.
\[
	\begin{aligned}
		S(f, P) = & && \sum_{k = 1}^n \underbrace{f(w(t_k))}_{(u+iv)(w(t_k))} \underbrace{\jleft( w(t_k) - w(t_{k - 1}) \jright)}_{\Delta x_k + i \Delta y_k} = \\
		& && \sum_{k = 1}^n \jleft[ u(w(t_k)) \Delta x_k - v(w(t_k)) \Delta y_k \jright] + \\
		& i && \sum_{k = 1}^n \jleft[ u(w(t_k)) \Delta y_k + v(w(t_k)) \Delta x_k \jright].
	\end{aligned}
\]
(Прим. автора: по ходу дела ввели приращения параметризации по $x$ и по $y$, вещественной и мнимой частям.) Устремим к нулю мелкость дробления $\Delta P$, получим интеграл по кривой.

\begin{definition}
	\it{Интегралом от функции $f$ по кривой $\gamma$} называется \[
		\int_\gamma f(z) dz = \lim_{\Delta(P) \to 0} S(f, P).
	\]
\end{definition}
\begin{note}
	При выполнении предположений выше, интеграл всегда существует и
	\begin{align*}
		& \int_\gamma f(z) dz = \\
		& \int_\gamma (u dx - v dy) + i \int_\gamma (v dx + u dy) = \\
		& \int_\alpha^\beta f(w(t)) w'(t) dt.
	\end{align*}
\end{note}
\begin{anote}
	Перед нами криволинейные интегралы второго рода (здесь мы пользуемся тем, что $\Cm = \R^2$, но с доопределённым умножением). Мы определяли их в третьем семестре матана аналогично криволинейным интегралам первого рода (в интегралах первого рода функции и кривые в трехмерном пространстве). Кому интересно, это страница 67 конспекта КТЛ лекций А. Л. Лукашова за осень 2022-2023 учебного года. На лекции мы не доказывали эти формулы, но в конспекте лекций Половинкина Е. С. 2020 года, страница 18, сначала выводится вторая формула, а потом из неё третья. 
\end{anote}

Рассмотрим ещё другую сумму, теперь домножаем на модуль приращения параметризации.
\[
	\begin{aligned}
		S'(f, P) = & && \sum_{k = 1}^n \underbrace{f(w(t_k))}_{(u+iv)(w(t_k))} \mds{ w(t_k) - w(t_{k - 1})}\\
		= & && \sum_{k = 1}^n \underbrace{f(w(t_k))}_{(u+iv)(w(t_k))} \mds{\Delta w(t_k)}.
	\end{aligned}
\]
При $\Delta P$ стремящемуся к нулю $S'(f, P)$ стремится к интегралу $\int_\gamma f(z) \mds{dz}$:
\[
	\int_\gamma f(z) \mds{dz} = \int_\gamma u ds + i \int_\gamma v ds,
\]
где $ds = \sqrt{{(z_1'(t))}^2 + {(z_2'(t))}^2} dt$.

\begin{anote}
	Комбинируя всё вместе, получим:
	\[
		\int_\gamma f(z) \mds{dz} = \int_\alpha^\beta f(z(t)) \sqrt{{(z_1'(t))}^2 + {(z_2'(t))}^2} dt.
	\]
\end{anote}

{\color{red} Вот бы примеры вычисления интеграла с $\mds{dz}$ от себя добавить, (ниже, после примера с интегралом с $dz$), но это как-нибудь потом. Проверить, по определению интеграла комплекснозначной функции из матана этот интеграл можно брать отдельно по вещественной и мнимой частям, а потом собирать исходный.}

\begin{note}
	Длину кривой можно вычислить так (обозначение $\mds{\gamma}$ от автора, не с лекции):
	\[
		\mds{\gamma} = \int_\gamma \mds{dz}
	\]
\end{note}
\begin{theorem}[свойства интеграла от функции по кривой, без доказательства]
	Верны следующие свойства.
	\begin{enumerate}
		\item $\int_\gamma f dz$ и $\int_\gamma f \mds{dz}$ не зависят от параметризации.
		\item $\int_{-\gamma} f dz = - \int_{\gamma} f dz$, $\int_\gamma f \mds{dz} = \int_{-\gamma} f \mds{dz}$.
		\item Линейность: $\int_\gamma (af + bg) dz = a \int_\gamma f dz + b \int_\gamma g dz$.
		\item Аддитивность: $\int_{\gamma_1 \cup \gamma_2} f dz = \int_{\gamma_1} f dz + \int_{\gamma_2} f dz$.
		\item Неравенства: $\mds{\int_\gamma f dz} \leq \int_\gamma \mds f \mds{dz} \leq \max_{p \in \gamma} \mds{f(p)} \mds{\gamma}$.
	\end{enumerate}
\end{theorem}
{\color{red} Написать комментарий от автора, почему пишем максимум. Имеем вещественную функцию двух переменных, непрерывную. А $\gamma$ -- компакт? Выглядит так, что да (ситуация похоже на ту, которую имеем с отрезком). Мб даже параметризовать отрезком можно, чтобы свести к известному.}

\begin{definition}
	Пусть $D$ -- область в $\Cm$, $f$ непрерывна в $D$, $F: D \to \Cm$; $F$ называется \it{первообразной} $f$, если $F$ голоморфна в $D$ и $(\forall z \in D) \,\, F'(z) = f(z)$
\end{definition}

\begin{definition}
	Выражение $f(z) dz$ называется \it{полным дифференциалом в области $G$}, если у $f$ существует первообразная (т.е. $f(z) dz = dF$, условно).
\end{definition}
\begin{anote}
	Если считать $f(z) dz$ дифференциальной формой, должно быть не условно, а даже формально верно.
\end{anote}

\begin{theorem}
	\label{full_differential_curve_integral_relation}
	Если $D$ -- область, $f$ непрерывна в $D$, тогда:
	\begin{enumerate}[label=(\arabic*)]
		\item если $f(z) dz$ -- полный дифференциал в области $D$, тогда для любой замкнутой кривой $\gamma \subset D$
		\[
			\int_\gamma f(z) dz = 0;
		\]
		\item если для любой замкнутой ломаной $\gamma \subset D$ $\int_\gamma f(z) dz = 0$, тогда $f(z) dz$ -- полный дифференциал.
	\end{enumerate}`
\end{theorem}
\begin{proof} \hfil
	\begin{enumerate}
		\item $f(z) dz$ -- полный дифференциал значит то, что у $f$ есть первообразная. Возьмём произвольную кривую $\gamma$, $w: [\alpha, \beta] \to \Cm$ -- её параметризация.
			\begin{align*}
				\int_\gamma f(z) dz = \int_\alpha^\beta f(w(t)) w'(t) dt = \int_\alpha^\beta F'(w(t)) w'(t) dt.
			\end{align*}
		Видим производную <<сложной>> функции $F(w(t))$. Знаем первообразную подынтегральной, воспользуемся ещё и замкнутостью кривой: $z(\alpha) = z(\beta)$. (Верно для любой параметризации, если верно для одной; упражнение, его решение можете присылать мне :).)
		\begin{align*}
			\int_\gamma f(z) dz = & \int_\alpha^\beta F'(w(t)) w'(t) dt = F(w(t)) \mid_\alpha^\beta = F(w(\alpha)) - F(w(\beta)) = \\
			& F(w(\alpha)) - F(w(\alpha)) = 0.
		\end{align*}
	
		\item Зафиксируем точку $a \in D$. Мы будем её использовать как начальную точку ломаных. Для произвольной точки $z \in D$ обозначим $\gamma_{a, z}$ -- некоторая ломаная с началом в $a$ и концом в $z$. Ломаная есть, поскольку область -- связное множество {\color{red} (прим. автора: связность же не обязательно влечет линейную связность (когда между двумя точками можно провести кривую), тем более связность с помощью ломаной; если удастся прояснить этот вопрос, сделать замечание автора об этом.)}. Введем интеграл по ломаной для произвольного конца:
		\[
			F(z) = \int_{\gamma_{a,z}} f(z) dz.
		\]
		Хотим сказать, что $(\forall z \in \Cm) \,\, F'(z) = f(z)$. Возьмём произвольную $z_0 \in \Cm$. Можем сказать, что $(\exists \epsilon > 0) \,\, B_r(z_0) \subseteq G$ (прим. автора: область -- открытое множество ($D$ -- область), это значит, любая точка внутренняя, в том числе $z_0$, т.е. содержится с некоторой своей окрестностью). Рассмотрим $\Delta z: 0 < \mds{\Delta z} < r$, тогда $z_0 + \Delta z \in D$ (прим. автора: всё ещё в окрестности, специально так брали $\Delta z$),
		\[
			F(z_0 + \Delta z) = \int_{\gamma_{a,z_0 + \Delta z}} f(z) dz = \int_{\gamma_{a, z_0} \cup [z_0, z_0 + \Delta z]} f(z) dz.
		\]
		Как $\gamma_{a, z_0} \cup [z_0, z_0 + \Delta z]$ обозначаем ломаную $\gamma_{a, z_0}$ к которой добавили ещё и звено $[z_0, z_0 + \Delta z]$, являющееся отрезком на комплексной плоскости. (Прим. автора: здесь используем то, что интеграл по $\gamma_{a, z}$ не зависит от того, какую ломаную из $a$ в $z$ брать; $\gamma_{a, z}$ была ``некоторой'' ломаной, не обязательно $\gamma_{a, z_0} \cup [z_0, z_0 + \Delta z]$.)
		Тогда
		\[
			\begin{aligned}
				F(z_0 + \Delta z) - F(z_0) & = & \int_{[z_0, z_0 + \Delta z]} f(z) dz, \\
				\frac{F(z_0 + \Delta z) - F(z_0)}{\Delta z} & = \frac{1}{\Delta z} & \int_{[z_0, z_0 + \Delta z]} f(z) dz.
			\end{aligned}
		\]
		Прим. автора: воспользовались аддитивностью интеграла по кривой.
		Хотим подогнать под определение предела, осталось немного: вычесть $f(z_0)$.
		\[
			\int_{[z_0, z_0 + \Delta z]} f(z_0) dz = f(z_0) \int_{[z_0, z_0 + \Delta z]} 1 dz = f(z_0) \Delta z.
		\]
		Прим. автора: здесь воспользовались линейностью интеграла по кривой, интеграл единицы для нашего отрезка равен его длине: пусть $w_1: [\alpha_1, \beta_1] \to D$ -- параметризация отрезка $[z_0, z_0 + \Delta z]$, тогда для интеграла от единицы по отрезку в интегральных суммах получится $w_1(\beta_1) - w_1(\alpha_1)$ (независимо от разбиения), а это $z_0 + \Delta z - z_0 = \Delta z$.
		Тогда
		\[
			\begin{aligned}
				\frac{F(z_0 + \Delta z) - F(z_0)}{\Delta z} - f(z_0) = \frac{1}{\Delta z} & \int_{[z_0, z_0 + \Delta z]} (f(z) - f(z_0)) dz.
			\end{aligned}
		\]
		Функция $f$ непрерывна в области $D$, потому непрерывна в точке $z_0$, т.е.
		\[
			(\forall \epsilon > 0) (\exists \delta_1(\epsilon) > 0) \jleft( \forall \Delta z \in \Cm: \mds{\Delta z} < \delta_1(\epsilon) \jright) \,\, \mds{f(z_0 + \Delta z) - f(z_0)} < \epsilon.
		\]
		Хотим получить определение производной $F$ в точке $z_0$ (показать, что она равна $f(z_0)$):
		\[
			(\forall \epsilon > 0) (\exists \delta_2 > 0) \jleft( \forall \Delta z \in \Cm: \mds{\Delta z} < \delta_2 \jright) \,\, \mds{\frac{F(z_0 + \Delta z) - F(z_0)}{\Delta z} - f(z_0)} < \epsilon.
		\]
		Рассмотрим произвольное $\varepsilon > 0$. Возьмём $\delta_2 = \min \{ \delta_1(\epsilon), r \}$.
		\[
			\begin{aligned}
				\mds{\frac{F(z_0 + \Delta z) - F(z_0)}{\Delta z} - f(z_0)} = & \frac{1}{\mds{\Delta z}} \mds{\int_{[z_0, z_0 + \Delta z]} (f(z) - f(z_0)) dz} && \leq \\
				& \frac{1}{\mds{\Delta z}} \max_{p \in [z_0, z_0 + \Delta z]} \mds{f(p) - f(z_0)} \mds{\Delta z} && < \epsilon.
			\end{aligned}
		\]
		Воспользовались свойством неравенства из предыдущего утверждения.
	\end{enumerate}
\end{proof}'

{\color{red} Доказать, что интеграл не зависит от выбора ломаной (мы просто замыкаем ломаные, докажу, но позже; смотрите конспект лекций Е. С. Половинкина 2020 года, страница 21; по всем вопросам пишите мне во вконтакте или телеграме). Написать замечание от автора, что первое свойство похоже на интеграл от точной формы; если вы готовы интерпретировать ситуацию как дифференциальную форму, был бы рад послушать, напишите мне.}

\begin{example}
	Вычислим $\int_{\mds{z - a} = \rho} {(z - a)}^n dz$ при $n \in \Z$. \\
	При $n \neq -1$
	\[
		F(z) = \frac{{(z - a)}^{n + 1}}{n + 1}, F'(z) = {(z - a)}^n.
	\]
	Тогда ${(z - a)}^n dz$ -- полный дифференциал (у ${(z - a)}^n$ есть первообразная), интеграл сразу равен нулю, по доказанному нами утверждению. \\
	При $n = -1$ выберем параметризацию $w(t) = a + \rho e^{it}$, $t \in [-\pi, \pi]$,
	\[
		\int_{\mds{z - a} = \rho} {(z - a)}^{-1} dz = \int_{-\pi}^\pi {(w(t) - a)}^{-1} w'(t) dt = \int_{-\pi}^\pi \rho^{-1} e^{-it} \rho e^{it} i dt = 2 \pi i.
	\]
\end{example}
\begin{anote}
	Здесь под $\mds{z - a} = \rho$ подразумевается обход множества точек против часовой стрелки. Для кривой без самопересечений её обход против часовой стрелки обычно считается положительным относительно области внутри кривой. \\
	В последнем равенстве случая $n = -1$ мы считали интеграл от $i$ по кривой $[-\pi, \pi]$ (отрезок). Мы знаем, что можно вытащить $i$ по линейности, а дальше интеграл единицы по отрезку дает его длину (замечание от автора в доказательстве теоремы про полный дифференциал, если интеграл по любой замкнутой ломаной равен нулю; это не сложно показать, в частичных суммах для любого разбиения всегда разность конечной точки и начальной).
\end{anote}

\section{Интегральная формула Коши}

{\color{red} Отметить и отмечать далее все теоремы, которые мы формулировали без доказательства на лекциях (это далеко не все те, которые не доказаны в конспекте), как таковые, чтобы у читателя не возникало вопросов.}

\begin{theorem}[лемма Гурса]
	Если $D$ -- область, $f$ голоморфна в $D \setminus \{ a \}$, непрерывна в $D$; $\triangle$ -- треугольник, $\ol \triangle \subset D$, тогда $\int_{\partial \triangle} f(z) dz = 0$ (прим. автора: здесь $\partial \triangle$ -- граница треугольника).
\end{theorem}
