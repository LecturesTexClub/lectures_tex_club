\section{Дифференцируемость в $\Cm$}

Зафиксируем до конца параграфа (если не сказано обратного в формулировке утверждения)
\begin{itemize}
	\item $z_0 \in \Cm$, $z_0 = x_0 + i y_0$, $x_0, y_0 \in \R$;
	\item $f: O_r(z_0) \to \Cm$, $f(z) = f(x + iy) = u(x, y) + i v(x, y)$
	\item $u, v: \R^2 \to \R$, $u(x, y) = \re f(x + i y)$, $v(x + iy) = v(x, y) = \im f(x + i y)$ (вне $O_r(z_0)$ положим их равными нулю).
\end{itemize}
Т.е. разложили комплекснозначную функцию на функции вещественной и мнимой части (если считать, что $\Cm = \R^2$, то $u, v: \Cm \to \R$).

\begin{definition}
	Функция $f$ $\Cm$-дифференцируема в точке $a$, если
	\[
	(\exists A' \in \Cm) \,\, \ulim{z \to a} \frac{f(z) - f(a)}{z - a} = A',
	\]
	(т.е. существует предел). $A'$ обозначается как $f'(z_0)$.
\end{definition}
\begin{note}
	Существование предела эквивалентно утверждению:
	\[
		(\exists A' \in \Cm) \,\, f(z_0 + \Delta z) - f(z_0) = \Delta f = A' \Delta z + o(\Delta z),
	\]
	где o-малое взято при $\Delta z \to 0$. Можно использовать его в определении.	
\end{note}
%TODO: доказать, хотя бы пару слов сказать.
\begin{anote}
	Как понимаю, выбираем такое название, потому что мы не хотим пересекаться с определением дифференцируемости вектор-функций. Нашу функция может быть воспринята как вектор-функция, но нам нужно другое определение.
\end{anote}


\begin{theorem}
	$f$ дифференцируема в точке $z_0$ тогда и только тогда, когда $u$ и $v$ дифференцируемы в $(x_0, y_0)$ и выполнено условие Коши-Римана:
	\[
		\begin{cases}
			{u'}_x(x_0, y_0) = {v'}_y(x_0, y_0), \\
			{u'}_y(x_0, y_0) = -{v'}_x(x_0, y_0).
		\end{cases}
	\]
\end{theorem}
\begin{proof}	
	Докажем слева направо: из дифференцируемости получим дифференцируемость компонент (как функций нескольких переменных) и условие Коши-Римана.
	
	Обозначим $f'(z_0) = \alpha + i \beta$, $\alpha, \beta \in \R$.
	Воспользуемся эквивалентным определением дифференцируемости:
	\[
		f(z_0 + \Delta z) - f(z_0) = (\alpha + i \beta) \Delta z + o(\Delta z)
	\]
	при $\Delta z \to 0$. По определению о-малого [прим. автора: которого у нас пока нет, появится]:
	\[
		\left( \exists g: \Cm \to \Cm: \lim_{\Delta z \to 0} \frac{g(\Delta z)}{\Delta z} = 0 \right) (\forall \Delta z : \lvert \Delta_z \rvert < r) \,\, f(z_0 + \Delta z) - f(z_0) = (\alpha + i \beta) \Delta z + g(\Delta z).
	\]
	Обозначим $\Delta z$ как $(\Delta x, \Delta y) = \Delta x + i \Delta y$. Тогда условие примет вид: \\
	\[
		f(z_0 + \Delta z) - f(z_0) = (\alpha \Delta x - \beta \Delta y) + i (\beta \Delta x + \alpha \Delta y) + g(\Delta z).
	\]
	Распишем $f$ по определению $u$ и $v$:
	\begin{align*}
		u(x_0 + \Delta x, y_0 + \Delta y) + i v(x_0 + \Delta x, y_0 + \Delta y) - 	u(x_0, y_0) - i v(x_0, y_0) = \\
		(\alpha \Delta x - \beta \Delta y) + i (\beta \Delta x + \alpha \Delta y) + g(\Delta z).
	\end{align*}

	Возьмём вещественную и мнимую части от выражения:
	\begin{align*}
		u(x_0 + \Delta x, y_0 + \Delta y) - u(x_0, y_0) &= (\alpha \Delta x - \beta \Delta y) + \re g(\Delta z) = \langle (\alpha, -\beta), \Delta z \rangle + \re g(\Delta z), \\
		v(x_0 + \Delta x, y_0 + \Delta y) - v(x_0, y_0) &= (\beta \Delta x + \alpha \Delta y) + \im g(\Delta z) = \langle (\beta, \alpha), \Delta z \rangle + \im g(\Delta z).
	\end{align*}

	Мы до этого считали, что $\Delta z \in \Cm$ ($\Cm$ -- это тоже самое $\R^2$, только разрешено перемножать векторы, делить векторы и т.п.). Теперь рассматриваем $\Delta z$ как элемент $\R^2$.
	Покажем, что $\re g(\Delta z)$ и $\im (\Delta z)$ являются $o(\lvert \Delta z \rvert)$. Тогда по определению дифференцируемости функции нескольких переменных получим дифференцируемость $u$ и $v$, это даст нам явный вид частных производных, т.к. градиент у нас уже есть.
	
	\begin{align*}
		\ulim{\Delta z \to 0} \frac{g(\Delta z)}{\Delta z} = 0 \Lra \ulim{\Delta z \to 0} \jleft\lvert \frac{g(\Delta z)}{\Delta z} \jright\rvert = 0 \Lra \ulim{\Delta z \to 0} \frac{\lvert g(\Delta z) \rvert}{\lvert \Delta z \rvert} = 0.
	\end{align*}

	Первый переход обосновывается из определения предела: \[
		\jleft\lvert \frac{g(\Delta z)}{\Delta z} - 0 \jright\rvert = \jleft\lvert \frac{g(\Delta z)}{\Delta z} \jright\rvert = \jleft\lvert \jleft\lvert \frac{g(\Delta z)}{\Delta z} \jright\rvert - 0 \jright\rvert.
	\]
	Второй переход верен, т.к. берем предел одной и той же функции (они всюду совпадают). \\
	Воспользуемся свойствами модуля относительно вещественной и мнимой части:
	\begin{align*}
		\begin{cases}
			-\mds{g(\Delta z)} \leq \re g(\Delta z) \leq \mds{g(\Delta z)}, \\
			-\mds{g(\Delta z)} \leq \im g(\Delta z) \leq \mds{g(\Delta z)},
		\end{cases}
		\Rightarrow
		\begin{cases}
			-\frac{\mds{g(\Delta z)}}{\mds{\Delta z}} \leq \frac{\re g(\Delta z)}{\mds{\Delta z}} \leq \frac{\mds{g(\Delta z)}}{\mds{\Delta z}}, \\
			-\frac{\mds{g(\Delta z)}}{\mds{\Delta z}} \leq \frac{\im g(\Delta z)}{\mds{\Delta z}} \leq \frac{\mds{g(\Delta z)}}{\mds{\Delta z}},
		\end{cases} \\
		\Rightarrow
		\begin{cases}
			\ulim{z \to 0} \frac{\re g(\Delta z)}{\mds{\Delta z}} = 0, \\
			\ulim{z \to 0} \frac{\im g(\Delta z)}{\mds{\Delta z}} = 0. \\
		\end{cases}
	\end{align*}
	Последний переход верен по теореме о зажатой функции для функций нескольких переменных. [Прим. автора: она ещё легко получается из определения, просто оценивается модуль.]
	
	Значит, $u$ и $v$ дифференцируемы как функции нескольких переменных. Знаем их градиенты, откуда выполнено и условие Коши-Римана: ${u'}_x = \alpha = {v'}_y$, ${u'}_y = -\beta = -{v'}_x$.
	
	{\color{red} В другую сторону тоже будет, скоро. } %TODO: в другую сторону! Там совсем быстро будет, думаю. Доказать только, что o(|z|) + io(|z|) = o(z)... 
\end{proof}


\begin{anote}
	Как запомнить условие Коши-Римана? Предлагаю способ. Давайте придумаем такую матрицу:
	\[
		\begin{pmatrix}
			{u'}_x & {u'}_y \\
			{v'}_x & {v'}_y
		\end{pmatrix}.
	\]
	Возьмём теперь её определитель: ${u'}_x {v'}_y - {u'}_y {v'}_x$. Пара сомножителей -- те компоненты, которые равны. Знак минус перед сомножителем указывает на то, что у компонент отличаются лишь знаком (и равны по модулю). \\
	Запомнить просто: <<у икс, у игрек, в икс, в игрек>>. Можно её не выписывать, просто в голове считать определитель. \\
	Матрица похожа на якобиан замены координат от $(u, v)$ к $(x, y)$. Можно придумать и другие способы.
\end{anote}

\begin{definition}
	Пусть $U$ -- открытое множество в $\overline{\Cm}$; $f$ называется \emph{гомоморфной} в $U$, если $f$ $\Cm$-дифференцируема в любой точке $U$. 
\end{definition}

{\color{red} Спросить лектора, как мы могли бы определить открытость в $\overline{\Cm}$. Тут открытость, если не включается бесконечность, просто открытость в $\R^2$? А открытость с бесконечностью -- открытость без бесконечности + объединение с $\{ \infty \}$?}

\begin{definition}
	Пусть $E$ -- множество точек в $\overline{\Cm}$, $E$ не открыто; $f$ называется \emph{голоморфной} в $E$, если существует открытое $U \supset E$, в котором $f$ голоморфна.
\end{definition}

\begin{definition}
	Пусть $g: U \to \R$, $U$ -- открытое в $\R^2$; функция $g$ называется \emph{гармонической}, если $g \in C^2(U)$ (дважды дифференцируема и вторая производная непрерывна) и $u_{xx} + u_{yy} = 0$ в любой точке $U$ [прим. автора: уравнение на вторые частные производные].
\end{definition}

\begin{proposition}
	Если $U$ -- открытое, $f$ голоморфна в $U$, $u, v \in C^2(U)$, то $u$, $v$ -- гармонические в $U$.
\end{proposition}
\begin{proof}
	Первое условие из определения гармонической функции уже выполнено по предположению утверждения: $u, v \in C^2(U)$. Докажем только для $u$, для $v$ будет всё аналогично. Зафиксируем точку $(x, y) = x_0 + i y_0$ из $U$. Функция $f$ дифференцируема в $x_0 + i y_0$ (по определению голоморфности). Выпишем условие Коши-Римана:
	\[
		\begin{cases}
			{u'}_x(x_0, y_0) = {v'}_{y}(x_0, y_0), \\
			{u'}_y(x_0, y_0) = -{v'}_{x}(x_0, y_0).
		\end{cases}
	\]
	Это верно для любой точки из $U$. Продифференцируем первое равенство по $x$, а второе -- по $y$. Получим
	\[
		\begin{cases}
			{u'}_{xy}(x_0, y_0) = {v'}_{yx}(x_0, y_0), \\
			{u'}_{yy}(x_0, y_0) = -{v'}_{xy}(x_0, y_0).
		\end{cases}
	\]
	Из теории вещественных функций нескольких переменных мы знаем, что если частные производные существуют в некоторой окрестности точки (а наше множество открытое, для любой точки можно найти окрестность, где функция дифференцируема) и непрерывны в точке ($u \in C^2(U)$, вторые частные производные непрерывны), то есть перестановочность порядка дифференцирования (\href{https://docs.yandex.ru/docs/view?url=ya-disk-public%3A%2F%2Fwxhx1QYHiV0ono%2FDz7ZjMLvQDgvtU6E96f8jJk7T%2FdQrDLFzJ7YEtdg%2B37yJNcdEq%2FJ6bpmRyOJonT3VoXnDag%3D%3D%3A%2FLectures%2F2_Semester%2FCalculus%2F2022_Lukashov.pdf&name=2022_Lukashov.pdf}{теорема Шварца, конспект Лекций А. Л. Лукашова за второй семестр, страница 28}). Значит, у нас есть уже и такая система условий:
	\[
		\begin{cases}
			{u'}_{xy}(x_0, y_0) = {v'}_{yx}(x_0, y_0), \\
			{u'}_{yy}(x_0, y_0) = -{v'}_{yx}(x_0, y_0).
		\end{cases}
	\]
	Складывая эти уравнения, получаем требуемое.
	{\color{red} Здесь важно, что мы считаем открытыми в $\overline{\Cm}$, потому спросить лектора про определение открытых надо. Спрошу.}
\end{proof}
\begin{anote}
	В доказательстве функции совпадали на открытом множестве, потому их производные тоже совпадают на этом множестве. Если определить дифференцируемость функции нескольких переменных как существование предела по совокупности переменных некоторого выражения (частное приращения значения и модуля приращения аргумента), то для каждой точки и для каждой достаточно малой окрестности (меньше радиуса, с которым эта точка внутренняя) выражения совпадают, потому им подходит одно и то же число, являющееся пределом для их обоих. {\color{red} Здесь можно будет подробнее написать, о чем речь.}
\end{anote}

\begin{proposition}
	Если $f: O_r(z_0) \to V$, $V$ -- открытое, $g: V \to \Cm$, $f$ голоморфна в $O_r(z_0)$, $g$ голоморфна в $V$, тогда $h = g \circ f$ голоморфна в $O_r(z_0)$, $h'(z_0) = g'(f(z_0)) f'(z_0)$.
\end{proposition}
\begin{proof}
	Будет.
\end{proof}

\begin{definition}
	Пусть $E \subset \ol \Cm$; $E$ называется связным, если не существует двух открытых (в $\Cm$?) $G_1$ и $G_2$ таких, что выполнены все условия [прим. автора: т.е. что-то да нарушится]:
	\begin{itemize}
		\item $E \subset G_1 \cup G_2$,
		\item $E \cap G_1 \cap G_2 = \varnothing$,
		\item $E \cap G_1 \neq \varnothing$, $E \cap G_2 \neq \varnothing$.
	\end{itemize}
\end{definition}

{\color{gray} Здесь было замечание, которое не удалось понять. Разберусь.} %TODO: непонятное замечание, спросить у Лизы.

\begin{proposition}
	Отрезок (множество точек комплексной плоскости между двумя заданными, или луч, или прямая) -- связное множество.
\end{proposition}
\begin{proof}
	Будет.
\end{proof}

\begin{proposition}
	Пусть $D \subset \ol \Cm$; $D$ называется \it{областью}, если открыто и связно в $\ol \Cm$.
\end{proposition}

\begin{theorem}
	Если $D$ -- открыто и не пусто, то $D$ связно тогда и только тогда, когда любые две точки $D$ можно соединить ломаной, лежащей в $D$. Более того, можно считать, что звенья ломаной параллельны оси $OX$ или $OY$.
\end{theorem}
\begin{proof}
	Будет.
\end{proof}

\begin{definition}
	Пусть $D$ -- область в $\ol \Cm$; $D$ называется \it{односвязной}, если $\ol \Cm \setminus D$ связно.
\end{definition}

\begin{theorem}
	Если $D$ -- область, $f$ голоморфна в $D$, $(\forall z \in D) \,\, f'(z) = 0$, тогда $f$ -- константа на $D$: $(\exists C \in \Cm) \, (\forall z \in D) \,\, f(z) = C$.
\end{theorem}
\begin{proof}
	Будет.
\end{proof}

\begin{theorem}[об обратных функциях]
	Если $D$ -- область, $f$ голоморфна в $D$, $f'$ непрерывна в $D$, $z_0 \in D$, $f'(z_0) \neq 0$, $f(z_0) = w_0$, тогда существуют окрестность $U$ точки $z_0$ и окрестность $V$ точки $w_0$ такие, что
	\begin{enumerate}
		\item $f$ -- биекция из $U$ в $V$,
		\item $g = f^{-1}$ голоморфна в $V$,
		\item $g'(w_0) = \frac{1}{f'(z_0)}$.
	\end{enumerate}
\end{theorem}
\begin{proof}
	Будет.
\end{proof}

{\color{red} Конспект переходит в режим <<дословно с доски>>, пока нет времени привести в вид, который я бы назвал идеалом, мой личный стандарт.} % TODO: как только буду приводить в нормальный вид, сдвигать эту плашку ниже и ниже.

\begin{definition}
	Пусть $f$ голоморфна в области $D$; $f$ \it{однолистна} в $D$, если $f(z_1) = f(z_2) \Ra z_1 = z_2 \,\, (\forall z_1, z_2 \in D)$. % TODO: переписать на кванторы, скзаать, что можно просто запомнить, что это голоморфная и ограничение на $D$ инъективно
\end{definition}

\begin{definition}
	Пусть $f$ голоморфна в области $D$; $f$ \it{локально однолистна} в $D$, если $(\forall z_0 \in D) \, (\exists O(z_0)) \,\, f \text{ однолистна в $O(z_0)$}$ . % TODO: здесь сказать, что существует радиус, а текстом сказать, у каждой точки есть окрестность, в которой функция однолистна.
\end{definition}

\begin{note}
	Если $f$ голоморфна в $D$, $f'$ непрерывна в $D$, $(\forall z \in D) \,\, f(z) \ne 0$, то $f$ локально однолистна в $D$.
\end{note}

\begin{example}
	$f(z) = z^2$, $D = \{ 1 < \mds z < 2 \}$. $f(z)$ локально однолистна, но не однолистна ($z = \pm \frac{3}{2}$).
\end{example}

\begin{definition}
	Пусть $f, f_n: E \to \Cm$, $E \subseteq \Cm$; $f_n \to f$ (сходится поточечно) на $E$, если $(\forall z \in E) \, (\forall \epsilon > 0) \, (\exists N) \, (\forall n \geq N) \,\, \mds{f_n(z) - f(z)} < \epsilon$. %TODO: сделать определение понятия нормальное, обозначение в скобки, указать, из каких множеств берутся элементы в кванторах.
\end{definition}

\begin{definition}
	Пусть $f, f_n: E \to \Cm$, $E \subseteq \Cm$; $f_n \rightrightarrows f$ (сходится равномерно) на $E$, если $(\forall \epsilon > 0) \, (\exists N) \, (\forall n \geq N) \, (\forall z \in E) \,\, \mds{f_n(z) - f(z)} < \epsilon$. %TODO: сделать определение понятия нормальное, обозначение в скобки, указать, из каких множеств берутся элементы в кванторах. Сказать, что все отличие в том, что квантор для любой точки стоит последним.
\end{definition}

\begin{theorem}[критерий Коши] % TODO: критерий Коши равномерной сходимости
	$f_n \overset{E}{\tto} f \Leftrightarrow (\forall \epsilon > 0) \, (\exists N) \, (\forall n, m > N) \, (\forall z \in D) \,\, \mds{f_n(z) - f_m(z)} < \epsilon$.
\end{theorem}
%TODO: спросить, почему мы многое не доказываем. Потому что это доказывается аналогично вещественному случаю? Если подтвердиться, написать это в конспекте.

\begin{definition}
	Ряд $\sum_{n = 1}^\infty g_n(z)$ сходится $\Lra$ последовательность $\sum_{n = 1}^N g_n(z)$ сходится. % Сказать, что сходится поточечно; сделать аналогичное определение для сходимости равномерно.
\end{definition}

\begin{definition}
	Ряд $\sum_{n = 1}^\infty g_n(z)$ сходится абсолютно $\Lra$ последовательность $\sum_{n = 1}^N \mds{g_n(z)}$ сходится. % Сказать, что сходится поточечно; сделать аналогичное определение для сходимости равномерно.
\end{definition}

\begin{theorem}[признак Вейерштрасса]
	Пусть $\mds{g_n(z)} < \alpha_n \,\, (\forall z \in E)$, $\sum_{n = 1}^\infty \alpha_n < \infty$ (ряд вещественных чисел сходится), тогда $\sum_{n = 1}^\infty g_n(z)$ сходится абсолютно и равномерно на $E$.
\end{theorem}

\begin{proposition}
	$\sum_{n = 0}^N z_n = 1 + z + \ldots + z^N = \frac{1 - z^{N + 1}}{1 - z}$; при $\mds{z} < 1$ $\sum_{n = 0}^\infty z^n = \frac{1}{1 - z}$. % запомнить формулу геометрической прогрессии просто, в числителе возводим частное прогресси q в степень количества членов, вычитаем 1, делим на q - 1 (тут просто домножили на $-1$).
\end{proposition}

\section{Степенные ряды и элементарные функции}

\begin{definition}
	Степенным рядом в точке $z_0$ называется ряд $\sum_{n = 0}^\infty a_n {(z - z_0)}^n$, $a_n \in \Cm$.
\end{definition}

\begin{theorem}
	Пусть $\ol{\ulim{n \to \infty}} \sqrt{\mds{a_n}} = \frac{1}{R}$, $R \in [0, +\infty]$, $a_n z^n$ -- степенной ряд, тогда
	\begin{enumerate}
		\item $\mds z \leq r < R \Ra$ ряд сходится равномерно и абсолютно % на $O_r(0)$.
		\item $\mds z > R$ -- ряд расходится.
		\item $f(z) = \sum_{n = 0}^\infty a_n z^n$ при $\mds z < R$, $f(z)$ голоморфна и $f'(z) = \sum_{n = 0}^\infty n a_n z^{n - 1}$.
	\end{enumerate}
	% перейти к произвольному ряду можно с помощью замены переменных, тогда в неравенствах из условия теоремы тоже нужно сделать замену. Попытаться наглядно объяснить.
\end{theorem}

\begin{proof}
	\begin{enumerate}
		\item Пусть $\rho \in (r, R), \frac{1}{R} < \frac{1}{\rho} < \frac{1}{r}$. Тогда $(\exists N) \, (\forall n > N) \, {\mds{a_n}}^{\frac{1}{n}} < \frac{1}{\rho}$. $\mds{a_n z^n} \leq \frac{1}{\rho^n} r^n$ при $\mds z \leq r$. По признаку Вейерштрасса (ряд $\sum q^n$ сходится при $q < 1$), ряд $\sum_{n = 0}^\infty a_m z^n$ сходится абсолютно и равномерно. % \rho = \mds z, \rho \in [r, R] по условию. Все ли работает в этом случае? Если да, включить r. Объяснить переход от верхнего предела в неравенству на модуль.
	\end{enumerate}
\end{proof}
