\begin{exercise}
	Доказать, что для любой точки $x_0 \in [-\pi; \pi]$ существует функция $f \in C_{2\pi}$ такая, что частные суммы Фурье $S_n(f, x_0)$ расходятся.
\end{exercise}

\section{Сопряжённое пространство}

\begin{note}
	Далее $E$ --- линейное нормированное пространство, $\K$ --- либо $\R$, либо $\Cm$.
\end{note}

\begin{definition}
	\textit{Сопряжённым пространством} $E^*$ называется пространство $\cL(E, \K)$.
\end{definition}

\begin{theorem} (Рисса-Фреше)
	Пусть $H$ --- гильбертово пространство. Тогда выполнено утверждение:
	\[
		\forall f \in H^*\ \exists! y_0 \in H \such \forall x \in H\ \ f(x) = (x, y_0)
	\]
\end{theorem}

\begin{note}
	Пафос теоремы состоит в том, что она даёт полное описание сопряжённого пространства в случае гильбертова пространства. К сожалению, такая роскошь недоступна в более общих случаях, но и там есть результат --- теорема Хана-Банаха.
\end{note}

\begin{proof} (безкоординатный метод)
	\begin{itemize}
		\item (Существование) Если $f = 0$, то можно взять $y_0 = 0$ и всё. Иначе $f \neq 0$, а значит $\ker f \neq H$. Ядро является подпространством, поэтому к нему применима теорема о проекции: $\ker f \oplus (\ker f)^\bot = H$. Стало быть, существует $x_0 \in (\ker f)^\bot$. Покажем, что $\ker f \oplus [x_0] = H$. Для этого нам надо представить произвольный $x \in H$ в виде $x = z + \alpha x_0$, где $z \in \ker f$ и $\alpha \in \K$. Покажем, что мы можем подобрать $\alpha$ так, чтобы $z := x - \alpha x_0$ действительно лежал в ядре:
		\[
			x = z + \alpha x_0 \Lora fx = fz + \alpha fx_0 = \alpha fx_0 \Lora \alpha = \frac{fx}{fx_0}
		\]
		Мы почти у цели, ведь в коэффициенте проявилось значение $fx$, которое нужно выразить. Применим к равенству выше слева скалярное произведение с $x_0$:
		\[
			(x, x_0) = (z, x_0) + \frac{fx}{\ole{fx_0}} \|x_0\|^2 = \frac{fx}{\ole{fx_0}} \|x_0\|^2 \Ra fx = \Bigg(x, \underbrace{\frac{\ole{fx_0}}{\|x_0\|^2} x_0}_{y_0}\Bigg)
		\]
		Искомый элемент $y_0$ найден.
		
		\item (Единственность) Если $f(x) = (x, y_1) = (x, y_2)$, то $(x, y_1 - y_2) = 0$ для любого $x \in H$. Если $y_1 \neq y_2$ (а мы это предполагаем), то можно рассмотреть $x = y_1 - y_2$, тогда $\|y_1 - y_2\|^2 = 0 \Lra y_1 = y_2$, противоречие.
	\end{itemize}
\end{proof}

\begin{proof} (координатный метод)
	Пусть $H$ --- сепарабельное гильбертово пространство. Тогда в нём существует ортонормированный базис $\{e_n\}_{n = 1}^\infty$, причём верно разложение:
	\[
		\forall x \in H\ \ x = \sum_{n = 1}^\infty (x, e_n)e_n = \lim_{N \to \infty} \underbrace{\sum_{n = 1}^N (x, e_n)e_n}_{S_N}
	\]
	Пусть $f \in H^*$, $x \in H$, тогда $\lim_{N \to \infty} f(S_N) = f(x)$. Заметим, что $f(S_N)$ можно переписать в таком виде:
	\[
		f(S_N) = \sum_{n = 1}^N (x, e_n)f(e_n) = \sum_{n = 1}^N (x, \ole{f(e_n)}e_n) = \ps{x, \sum_{n = 1}^N \ole{f(e_n)}e_n}
	\]
	Кандидатом в искомый элемент $y_0$ будет потенциально существующий предел частичных сумм $\lim_{N \to \infty} \sum_{n = 1}^N \ole{f(e_n)}e_n$.
	\begin{itemize}
		\item (Сходимость ряда, существование $y_0$) Так как $\{e_n\}_{n = 1}^\infty$ --- ортонормированный базис, то ряд $\sum_{n = 1}^\infty \alpha_n e_n$ в гильбертовом пространстве $H$ сходится тогда и только тогда, когда сходится ряд $\sum_{n = 1}^\infty |\alpha_n|^2$. Посмотрим на соответствующую частичную сумму у рассматриваемого ряда:
		\[
			\no{\sum_{n = 1}^N \ole{f(e_n)}e_n}^2 = \sum_{n = 1}^N |\ole{f(e_n)}|^2 = \sum_{n = 1}^N f(e_n)\ole{f(e_n)} = f\ps{\sum_{n = 1}^N \ole{f(e_n)}e_n}
		\]
		Если обозначить величину слева за $\sigma_N^2$, то верны оценки:
		\[
			\sigma_N^2 = \sum_{n = 1}^N |\ole{f(e_n)}|^2 \le \|f\| \cdot \no{\sum_{n = 1}^N \ole{f(e_n)}e_n} = \|f\|\sigma_N \Ra \sigma_N \le \|f\|
		\]
		Значит, сходится ряд $\sigma_\infty^2 = \sum_{n = 1}^\infty |\ole{f(e_n)}|^2$, а стало быть, сходится и $\sum_{n = 1}^\infty \ole{f(e_n)}e_n$.
		
		\item (Единственность $y_0$) Доказательство повторяет то, что было написано в безкоординатном методе.
	\end{itemize}
\end{proof}

\begin{exercise}
	Пусть $L \subset H$ --- подпостранство гильбертова пространства, $f$ --- линейный ограниченный функционал на $L$. Докажите, что существует и единственен функционал $\wdt{f} \in H^*$ такой, что он согласован с $f$:
	\begin{enumerate}
		\item $\wdt{f}|_L = f$
		
		\item $\|\wdt{f}\| = \|f\|$
	\end{enumerate}
\end{exercise}

\begin{task}
	Если $f$ --- линейный функционал и $\ker f$ замкнуто, то $f$ непрерывен.
\end{task}

\begin{note}
	Далее $E$ закрепляется за обозначением линейного нормированного пространства над полем $\K$.
\end{note}

\begin{theorem} (Хана-Банаха)
	Пусть $M \subseteq E$ --- линейное многоообразие, $f \in M^*$. Тогда существует (не обязательно единственный!) функционал $\wdh{f} \in E^*$ такой, что он является продолжением $f$ на всё пространство $E$:
	\begin{enumerate}
		\item $\wdh{f}|_M = f$
		
		\item $\|\wdh{f}\| = \|f\|$
	\end{enumerate}
\end{theorem}