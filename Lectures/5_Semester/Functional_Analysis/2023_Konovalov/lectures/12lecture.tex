\begin{exercise}
	Доказать, что для любой точки $x_0 \in [-\pi; \pi]$ существует функция $f \in C_{2\pi}$ такая, что частные суммы Фурье $S_n(f, x_0)$ расходятся.
\end{exercise}

\begin{proof} (теоремы \ref{long_theorem})
	\begin{enumerate}
		\item (Идея) Пусть $\wdt{A}$ --- некоторое продолжение оператора $A$ по условию теоремы. Тогда несложно заметить, что по условию выполнено утверждение:
		\[
			\forall x \in E_1\ \exists \{x_n\}_{n = 1}^\infty \subseteq D(A) \such \lim_{n \to \infty} x_n = x \wedge \wdt{A}x_n = Ax_n \xrightarrow[n \to \infty]{} \wdt{A}x
		\]
		Стало быть, нужно отталкиваться от поточечного определения $\wdt{A}$.
		
		\item (Существование) Определим $\wdt{A}$ согласно идее (оператор $A$ непрерывен, поэтому пределы всегда есть):
		\[
			\forall x \in E_1\ \forall \{x_n\}_{n = 1}^\infty \subseteq D(A) \such \lim_{n \to \infty} x_n = x \Ra \wdt{A}x := \lim_{n \to \infty} Ax_n
		\]
		Теперь, покажем корректность такого определения:
		\begin{itemize}
			\item Значение $\wdt{A}$ не зависит от рассматриваемой последовательности $\{x_n\}_{n = 1}^\infty \subseteq D(A)$, $\lim_{n \to \infty} x_n = x$. Действительно, рассмотрим 2 последовательности $\lim_{n \to \infty} x_{n, 1} = \lim_{n \to \infty} x_{n, 2} = x$. Тогда, можно написать следующее неравенство:
			\[
				\|Ax_{1, n} - Ax_{2, m}\| \le \|A\| \cdot \|x_{1, n} - x_{2, m}\| \xrightarrow[n, m \to \infty]{} 0
			\]
			Более строго, нужно поочерёдно устремить в бесконечность $n, m \to \infty$ и тем самым сделать 2 предельных перехода.
			
			\item Почему $\wdt{A}$ --- линейный оператор? Это тривиально из линейности предела:
			\begin{multline*}
				\forall x, y \in E_1,\ \alpha, \beta \in \K\ \ \wdt{A}(\alpha x + \beta y) = \lim_{n \to \infty} A(\alpha x_n + \beta y_n) =
				\\
				\alpha \lim_{n \to \infty} Ax_n + \beta \lim_{n \to \infty} Ay_n = \alpha Ax + \beta Ay
			\end{multline*}
			
			\item Почему $\wdt{A}$ --- ограниченный оператор? Воспользуемся старым добрым предельным переходом:
			\[
				\|\wdt{A}x_n\| = \|Ax_n\| \le \|A\| \cdot \|x_n\| \Ra \|\wdt{A}x\| \le \|A\| \cdot \|x\| \Ra \|\wdt{A}\| \le \|A\|
			\]
			При этом из определения $\wdt{A}$ сразу следует, что $\|\wdt{A}\| \ge \|A\|$. Таким образом, мы сразу установили равенство $\|\wdt{A}\| = \|A\|$
		\end{itemize}
	
		\item (Единственность) Предположим, есть 2 продолжающих оператора: $\wdh{A}$ и $\wdt{A}$. Как мы и требовали, они должны быть согласованы с $A$. Стало быть, можно записать следующее:
		\[
			\forall x \in E_1\ \ \wdh{A}x = \lim_{n \to \infty} Ax = \wdt{A}x
		\]
		Следовательно, $\wdh{A} = \wdt{A}$.
	\end{enumerate}
\end{proof}

\section{Сопряжённое пространтво}

\begin{note}
	Далее $E$ --- линейное нормированное пространство, $\K$ --- либо $\R$, либо $\Cm$.
\end{note}

\begin{definition}
	\textit{Сопряжённым пространством} $E^*$ называется пространство $\cL(E, \K)$.
\end{definition}

\begin{theorem} (Рисса-Фреше)
	Пусть $H$ --- гильбертово пространство. Тогда выполнено утверждение:
	\[
		\forall f \in H^*\ \exists! y_0 \in H \such \forall x \in H\ \ f(x) = (x, y_0)
	\]
\end{theorem}

\begin{note}
	Пафос теоремы состоит в том, что она даёт полное описание сопряжённого пространства в случае гильбертова пространства. К сожалению, такая роскошь недоступна в более слабых случаях, но и там есть свой результат.
\end{note}

\begin{lemma} (не по лектору)
	Если $H$ --- гильбертово пространство, $f \in H^*$, $f \neq 0$, то коразмерность ядра равна 1. Иначе говоря:
	\[
		\exists x_0 \in H \bs \ker f \such \ker f \oplus [x_0] = H
	\]
\end{lemma}

\begin{proof}
	Так как $f \neq 0$, то $\ker f \neq H$, причём ядро является подпространством. По теореме о проекции:
	\[
		H = \ker f \oplus (\ker f)^\bot
	\]
	Значит, есть $x_0 \in (\ker f)^\bot \bs \{0\}$. Покажем, что верен факт $H = \ker f \oplus [x_0]$. Для этого нам надо представить произвольный $x \in H$ в виде $x = z + \alpha x_0$, где $z \in \ker f$ и $\alpha \in \K$. Покажем, что мы можем подобрать $\alpha$ так, чтобы $z := x - \alpha x_0$ действительно лежал в ядре:
	\[
		x = z + \alpha x_0 \Ra fx = fz + \alpha fx_0 = \alpha fx_0 \Ra \alpha = \frac{fx}{fx_0}
	\]
	Требуемое показано, значит верно разложение $H = \ker f \oplus [x_0]$, значит коразмерность ядра равна 1.
\end{proof}

\begin{proof} (безкоординатный метод)
	\begin{itemize}
		\item (Существование) Если $f = 0$, то можно взять $y_0 = 0$ и всё. Иначе $f \neq 0$, воспользуемся леммой о коразмерности ядра. в ней мы получили такое разложение элементов $H = \ker f \oplus [x_0]$:
		\[
			\forall x \in H\ x = z + \alpha x_0,\ \alpha = \frac{fx}{fx_0},\ z \in \ker f
		\]
		Мы почти у цели, ведь в коэффициенте проявилось значение $fx$, которое нужно выразить. Применим к равенству скалярное произведение с $x_0$:
		\[
			(x, x_0) = (z, x_0) + \frac{fx}{\ole{fx_0}} \|x_0\|^2 = \frac{fx}{\ole{fx_0}} \|x_0\|^2 \Ra fx = \Bigg(x, \underbrace{\frac{\ole{fx_0}}{\|x_0\|^2} x_0}_{y_0}\Bigg)
		\]
		Искомый элемент $y_0$ найден.
		
		\item (Единственность) Если $f(x) = (x, y_1) = (x, y_2)$, то $(x, y_1 - y_2) = 0$ для любого $x \in H$. Если $y_1 \neq y_2$ (а мы это предполагаем), то можно рассмотреть $x = y_1 - y_2$, тогда $\|y_1 - y_2\|^2 = 0 \Lra y_1 = y_2$, противоречие.
	\end{itemize}
\end{proof}

\begin{proof} (координатный метод)
	Пусть $H$ --- сепарабельное гильбертово пространство. Тогда мы знаем, что в нём существует ортонормированный базис $\{e_n\}_{n = 1}^\infty$, причём верно разложение:
	\[
		\forall x \in H\ \ x = \sum_{n = 1}^\infty (x, e_n)e_n = \lim_{N \to \infty} \underbrace{\sum_{n = 1}^N (x, e_n)e_n}_{S_N}
	\]
	Пусть $f \in H^*$, $x \in H$, тогда $\lim_{N \to \infty} f(S_N) = f(x)$. Заметим, что $f(S_N)$ можно переписать в таком виде:
	\[
		f(S_N) = \sum_{n = 1}^N (x, e_n)f(e_n) = \sum_{n = 1}^N (x, \ole{f(e_n)}e_n) = \ps{x, \sum_{n = 1}^N \ole{f(e_n)}e_n}
	\]
	Кандидатом в искомый элемент $y_0$ будет потенциально существующий предел частичных сумм $\lim_{N \to \infty} \sum_{n = 1}^N \ole{f(e_n)}e_n$.
	\begin{itemize}
		\item (Сходимость ряда, существование $y_0$) Ряд $\sum_{n = 1}^\infty \alpha_n e_n$ в гильбертовом пространстве $H$ сходится тогда и только тогда, когда сходится ряд $\sum_{n = 1}^\infty |\alpha_n|^2$. Посмотрим на соответствующую частичную сумму у рассматриваемого ряда:
		\[
			\no{\sum_{n = 1}^N \ole{f(e_n)}e_n}^2 = \sum_{n = 1}^N |\ole{f(e_n)}|^2 = \sum_{n = 1}^N f(e_n)\ole{f(e_n)} = f\ps{\sum_{n = 1}^N \ole{f(e_n)}e_n}
		\]
		Если обозначить величину слева за $\sigma_N^2$, то верны оценки:
		\[
			\sigma_N^2 = \sum_{n = 1}^N |\ole{f(e_n)}|^2 \le \|f\| \cdot \no{\sum_{n = 1}^N \ole{f(e_n)}e_n} = \|f\|\sigma_N \Ra \sigma_N \le \|f\|
		\]
		Таким образом, сходится ряд $\sigma_\infty^2 = \sum_{n = 1}^\infty |\ole{f(e_n)}|^2$, а стало быть, сходится и $\sum_{n = 1}^\infty \ole{f(e_n)}e_n$.
		
		\item (Единственность $y_0$) Доказательство повторяет то, что было написано в безкоординатном методе.
	\end{itemize}
\end{proof}

\begin{exercise}
	Пусть $L \subset H$ --- подпостранство гильбертова пространства, $f$ --- линейный ограниченный функционал на $L$. Докажите, что существует и единственен функционал $\wdt{f} \in H^*$ такой, что он согласован с $f$:
	\begin{enumerate}
		\item $\wdt{f}|_L = f$
		
		\item $\|\wdt{f}\| = \|f\|$
	\end{enumerate}
\end{exercise}

\begin{exercise}
	Если $f$ --- линейный функционал и $\ker f$ замкнуто, то $f$ непрерывен.
\end{exercise}

\begin{note}
	Далее $E$ закрепляется за обозначением линейного нормированного пространства над полем $\K$.
\end{note}

\begin{theorem} (Хана-Банаха)
	Пусть $M \subseteq E$ --- линейное многоообразие, $f \in E^*$. Тогда существует (не обязательно единственный!) функционал $\wdh{f} \in E^*$ такой, что он является продолжением $f$ на всё пространство $E$:
	\begin{enumerate}
		\item $\wdh{f}|_M = f$
		
		\item $\|\wdh{f}\| = \|f\|$
	\end{enumerate}
\end{theorem}