\begin{theorem} (Полнота $\cL(E_1, E_2)$ относительно поточечной сходимости)
	Пусть $E_1, E_2$ --- банаховы пространства, $\forall n \in \N\ A_n \in \cL(E_1, E_2)$. Если для любого $x \in E_1$ последовательность $\{A_nx\}_{n = 1}^\infty$ фундаментальна в $E_2$, то существует такой оператор $A \in \cL(E_1, E_2)$, что $A_n \to A$.
\end{theorem}

\begin{proof}
	Зафиксируем $x \in E_1$. Раз $\{A_nx\}_{n = 1}^\infty \subseteq E_2$ фундаментальна, то у неё есть предел. Положим $Ax$ по определению этим пределом:
	\[
		Ax := \lim_{n \to \infty} A_nx
	\]
	\textcolor{red}{Из теоремы о норме пространства операторов?} $A$ является линейным, ограниченным и к нему сходится $A_n$, поэтому лежит в $\cL(E_1, E_2)$.
\end{proof}

\begin{theorem} (Критерий поточечной сходимости последовательности линейных ограниченных операторов)
	Пусть $\{A_n\}_{n = 1}^\infty$ --- последовательность линейных ограниченных операторов. Тогда $A_n$ поточечно сходится к некоторому оператору $A$ тогда и только тогда, когда выполнено 2 условия:
	\begin{itemize}
		\item Последовательность норм $\{\|A_n\|\}_{n = 1}^\infty$ ограничена
		
		\item $\exists Y \subseteq E_1 \colon \cl Y = E_1 \wedge \forall y \in Y\ \lim_{n \to \infty} A_ny = Ay$
	\end{itemize}
\end{theorem}

\begin{proof}
	\begin{itemize}
		\item[$\Ra$] Коль скоро $A_n$ сходятся, их последовательность является ограниченной. Стало быть, $\{\|A_n\|\}_{n = 1}^\infty$ ограничена и осталось найти $Y$. \textcolor{red}{Дописать}
		
		\item[$\La$] \textcolor{red}{Дописать}
	\end{itemize}
\end{proof}

\begin{corollary}
	Если в условиях последней теоремы $\{A_n\}_{n = 1}^\infty$ сходится, то предел тоже является линейным ограниченным оператором \textcolor{red}{Это бессмысленно,  должно быть сильно раньше}
\end{corollary}

\begin{example} (Применение теоремы Банаха-Штейнгауза)
	Как известно из курса матанализа, для $f \in C_{2\pi}$ можно сопоставить ряд Фурье:
	\[
		f \sim \frac{a_0}{2} + \sum_{k = 1}^\infty a_k\cos(kx) + b_k\sin(kx)
	\]
	Причём частичные суммы можно записать через ядро Дирихле:
	\[
		S_n(f; x) = \frac{1}{\pi}\int_{-\pi}^\pi D_n(x - t)f(t)dt
	\]
	Оказывается, что этот вид \textit{операторов} обобщаем. \textcolor{red}{Дописать}
\end{example}

\textcolor{red}{Пример стоит расширить до микротемы}