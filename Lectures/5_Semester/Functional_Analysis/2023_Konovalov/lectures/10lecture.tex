\begin{definition}
	\textit{Нормой оператора} $A$ называется $\|A\| := \inf K$ \textcolor{red}{Переписать}
\end{definition}

\begin{proposition}
	Для нормы оператор верны равенства:
	\[
		\|A\| = \inf K = \sup_{\|x\| \le 1} \|Ax\| = \sup_{\|x\| = 1} \|Ax\| = \sup_{x \neq 0} \frac{\|Ax\|}{\|x\|}
	\]
\end{proposition}

\begin{proof}
	\textcolor{red}{Доказать}
\end{proof}

\textcolor{red}{Здесь должен быть здравый смысл к определению нормы}

\begin{theorem}
	Пусть $A \colon E_1 \to E_2$ --- линейный оператор. Тогда $A$ ограничен тогда и только тогда, когда он непрерывен.
\end{theorem}

\begin{proof}~
	\begin{itemize}
		\item[$\Ra$] По условию мы знаем, что
		\[
			\forall x \in E_1\ \ \|Ax\| \le \|A\| \cdot \|x\|
		\]
		С другой стороны, рассмотрим последовательность $\{x_n\}_{n = 1}^\infty \subseteq E_1$, сходящуюся к $x \in E_1$. Тогда:
		\[
			\|A(x_n - x)\| \le \|A\| \cdot \|x_n - x\| \xrightarrow[n \to \infty]{} 0
		\]
		
		\item[$\La$] Пойдём от противного. Тогда оператор должен быть неограничен, а это можно записать так:
		\[
			\forall n \in \N\ \exists x_n \in E_1 \such \|Ax_n\| > n\|x_n\|
		\]
		\textcolor{red}{Дописать}
	\end{itemize}
\end{proof}

\begin{definition}
	\textit{Множество всех линейных ограниченных операторов} обозначается как $\cL(E_1, E_2)$.
\end{definition}

\begin{note}
	Укажем 2 важных частных случая:
	\begin{itemize}
		\item $\cL(E) := \cL(E, E)$
		
		\item $\cL(E, \R) = E^*$
	\end{itemize}
\end{note}

\begin{theorem}
	Имеет место 2 утверждения:
	\begin{enumerate}
		\item $\cL(E_1, E_2)$ --- нормированное пространство с нормой, порождённой нормой операторов
		
		\item Если $E_2$ --- банахово пространство, то и $\cL(E_1, E_2)$ --- банахово пространство
	\end{enumerate}
\end{theorem}

\begin{proof}~
	\begin{enumerate}
		\item Линейность пространства тривиальна. Для того, чтобы доказать корректность нормы, нам достаточно проверить неравенство треугольника (остальное просто очевидно):
		\[
			\|A_1 + A_2\| = \sup_{\|x\| = 1} \|A_1x + A_2x\| \le \sup_{\|x\| = 1} \|A_1x\| + \sup_{\|x\| = 1} \|A_2x\| = \|A_1\| + \|A_2\|
		\]
		
		\item Нужно показать полноту пространства. Пусть $\{A_n\}_{n = 1}^\infty \subseteq \cL(E_1, E_2)$ --- фундаментальная последовательность:
		\[
			\forall \eps > 0\ \exists N \in \N \such \forall n, m \ge N\ \ \|A_n - A_m\| < \eps
		\]
		Покажем, что есть поточечная сходимость. Действительно:
		\[
			\forall x \in S(0, 1)\ \ \|A_nx - A_mx\| \le \|A_n - A_m\| \cdot \|x\| = \|A_n - A_m\| < \eps
		\]
		Стало быть, последовательность $\{A_nx\}_{n = 1}^\infty \subseteq E_2$ фундаментальна при любом $x \in E_1$, а в силу банаховости $E_2$ сходится. Определим оператор $A$ следующим образом:
		\[
			\forall x \in E_1\ \ Ax := \lim_{n \to \infty} A_nx
		\]
		\textcolor{red}{Доказать его линейность, ограниченность и то, что оператор $A$ является пределом $A_n$ по норме}
	\end{enumerate}
\end{proof}

\begin{corollary}~
	\begin{itemize}
		\item Если $E$ --- банахово пространство, то и $\cL(E)$ банахово.
		
		\item Для любого линейного нормированного пространства $E$ верно, что $E^*$ --- банахово пространство
	\end{itemize}
\end{corollary}

\begin{example}
	Существует ли оператор, который был бы линейным, но не ограниченным?
	\textcolor{red}{В конечномерном случае нет, в бесконечномерным --- да, оператор дифференцировния в $\cL(E_1, E_2)$, $E_1 = C^1[0; 1]$ с нормой $C[0; 1]$, $E_2 = C[0; 1]$. Контрпример даёт $\frac{\sin nx}{n} \rra 0$}
\end{example}

\begin{theorem}
	Пусть $D(A) \subseteq E_1$ --- линейное многообразие. Тогда для любого линейного ограниченного оператора $A \colon D(A) \to E_2$ существует согласованный оператор $\tilde{A} \in \cL(E_1, E_2)$ такой, что
	\begin{enumerate}
		\item $\tilde{A}|_{D(A)} = A$
		
		\item $\|\tilde{A}\| = \|A\|$
	\end{enumerate}
\end{theorem}

\begin{proof}
	\textcolor{red}{Докаательство после кучи теорем дальше}
\end{proof}

\begin{theorem} (1927г. Банаха-Штейнгауза-Хана, принцип равномерной ограниченности)
	Пусть $E_1$ --- банахово пространство, $\{A_n\}_{n = 1}^\infty \subseteq \cL(E_1, E_2)$. Тогда верна импликация:
	\[
		\Big(\forall x \in E\ \ \sup_{n \in \N} \|A_n x\| < \infty\Big) \Ra \sup_{n \in \N} \|A_n\| < \infty
	\]
\end{theorem}

\begin{proof}
	Запишем утверждение теоремы в эквивалентной форме:
	\[
		\sup_{n \in \N} \|A_n\| = \infty \Ra \Big(\exists x \in E_1 \such \sup_{n \in \N} \|A_nx\| = \infty\Big)
	\]
	Именно в такой форме мы и докажем требуемое. Сделаем это в 2 шага:
	\begin{enumerate}
		\item \textcolor{red}{Дописать}
		
		\item \textcolor{red}{Дописать}
	\end{enumerate}
\end{proof}
