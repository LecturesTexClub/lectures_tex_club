\begin{definition}
	Линейное пространство $E$ над $K$ называется \textit{евклидовым}, если в нём определена билинейная симметричная (полуторалинейная) положительно определённая форма $(\cdot, \cdot) \colon E^2 \to \K$:
	\begin{enumerate}
		\item $\forall x \in E\ \ (x, x) \ge 0 \wedge \Big((x, x) = 0 \Lra x = 0\Big)$
		
		\item $\forall x, y \in E\ \ (x, y) = \ole{(y, x)}$
		
		\item $\forall \alpha \in \K\ \forall x, y \in E\ \ (\alpha x, y) = \alpha(x, y)$
		
		\item $\forall x_1, x_2, y \in E\ \ (x_1 + x_2, y) = (x_1, y) + (x_2, y)$
	\end{enumerate}
\end{definition}

\begin{anote}
	В комплексном случае нельзя забывать, что $(x, \alpha y) = \ole{(\alpha y, x)} = \ole{\alpha}(x, y)$. В этом и есть суть названия <<полуторалинейная>>.
\end{anote}

\begin{proposition}
	Всякое евклидово пространство является линейным нормированным с нормой $\|x\| := \sqrt{(x, x)}$.
\end{proposition}

\begin{proof}
	Проверим аксиомы нормы:
	\begin{enumerate}
		\item (Неотрицательность и единственность нуля) Так как $(x, x) \ge 0$ по определению, то и $\sqrt{(x, x)} \ge 0$. Если $\|x\| = 0$, то $(x, x) = 0$. А это эквивалентно равенству $x = 0$ по определению.
		
		\item (Линейность нормы) $\|\alpha x\| = \sqrt{(\alpha x, \alpha x)} = \sqrt{\alpha\ole{\alpha}(x, x)} = |\alpha|\sqrt{(x, x)} = |\alpha| \cdot \|x\|$
		
		\item (Неравенство треугольника) Пусть $x, y \in E$. Тогда, вместо рассмотрения нормы, мы посмотрим на её квадрат:
		\begin{multline*}
			\|x + y\|^2 = (x + y, x + y) = (x, x) + (y, y) + 2\re (x, y) \le (x, x) + (y, y) + 2|(x, y)| \le
			\\
			(x, x) + (y, y) + 2\sqrt{(x, x)} \cdot \sqrt{(y, y)} = \ps{\sqrt{(x, x)} + \sqrt{(y, y)}}^2 = (\|x\| + \|y\|)^2
		\end{multline*}
		Остаётся без зазрения совести снять квадрат и получится требуемое.
	\end{enumerate}
\end{proof}

\begin{definition}
	Линейное пространство называется \textit{гильбертовым}, если оно является полным евклидовым пространством.
\end{definition}

\begin{note}
	Иногда к этому определению добавляют ещё разные комбинации из следующих свойств:
	\begin{itemize}
		\item Бесконечномерность
		
		\item Сепарабельность
	\end{itemize}
\end{note}

\begin{reminder} (Неравенство Бесселя)
	Пусть $\{e_n\}_{n = 1}^\infty \subseteq E$ --- ортогональная система, $x \in E$. Тогда, если $\alpha_n = \frac{\tbr{x, e_n}}{\|e_n\|^2}$ (коэффициент проекции $x$ на $e_n$), то выполнено неравенство:
	\[
		\sum_{n = 1}^\infty \alpha_n^2\|e_n\|^2 \le \|x\|^2
	\]
\end{reminder}

\begin{proof} (от автора)
	Доказательство целиком опирается на так называемом \textit{равенстве Бесселя}:
	\begin{multline*}
		\no{x - \sum_{n = 1}^N \tbr{x, e_n}e_n}^2 = \tbr{x - \sum_{n = 1}^N \tbr{x, e_n}e_n, x - \sum_{n = 1}^N \tbr{x, e_n}e_n} =
		\\
		\|x\|^2 - 2\tbr{x, \sum_{n = 1}^N \tbr{x, e_n}e_n} + \no{\sum_{n = 1}^N \tbr{x, e_n}e_n}^2 =
		\\
		\|x\|^2 - 2\sum_{n = 1}^N |\tbr{x, e_n}|^2 + \sum_{n = 1}^N |\tbr{x, e_n}|^2 = \|x\|^2 - \sum_{n = 1}^N |\tbr{x, e_n}|^2
	\end{multline*}
	Осталось заметить, что квадрат нормы с самого начала является неотрицательной величиной, отсюда и неравенство:
	\begin{multline*}
		0 \le \no{x - \sum_{n = 1}^N \tbr{x, e_n}e_n}^2 = \|x\|^2 - \sum_{n = 1}^N |\tbr{x, e_n}|^2
		\\
		\Lora \sum_{n = 1}^N |\tbr{x, e_n}|^2 \le \|x\|^2 \Lora \sum_{n = 1}^\infty |\tbr{x, e_n}|^2 \le \|x\|^2
	\end{multline*}
	Приведение к искомому виду уже тривиально.
\end{proof}

\begin{proposition} (Неравенство Коши-Буняковского)
	Имеет место неравенство:
	\[
		\forall x, y \in E\ \ |(x, y)| \le \|x\| \cdot \|y\|
	\]
	Причём равенство достигается тогда и только тогда, когда $x, y$ линейно зависимы.
\end{proposition}

\begin{proof} (через неравенство Бесселя)
	Запишем неравенство Бесселя, если ортонормированная система состоит из одного вектора $e$:
	\[
		\forall v \in E\ \ |(v, e)| \le \sqrt{(v, v)} = \|v\|
	\]
	Откуда в неравенстве Коши-Буняковского получить единичный вектор (который и составит ортогональную систему)? Занести норму множителя справа в скалярное произведение!
	\[
		|(x, y)| \le \|x\| \cdot \|y\| \Lra \md{\ps{x, \frac{y}{\|y\|}}} \le \|x\|
	\]
	Второе неравенство тогда верно по неравенству Бесселя и всё доказано.
\end{proof}

\begin{theorem} (фон Нейман, Фреше, 1930 гг., без доказательства)
	Пусть $E$ --- линейное нормированное пространство. Тогда норма $\|\cdot\|_E$ согласована с каким-либо скалярным произведением на $E$ тогда и только тогда, когда выполнено равенство параллелограмма:
	\[
		\forall x, y \in E\ \ \|x + y\|_E^2 + \|x - y\|_E^2 = 2\|x\|_E^2 + 2\|y\|_E^2
	\]
\end{theorem}

\begin{note}
	Базис --- это довольно <<связывающая>> конструкция, он не всегда существует. Что можно использовать вместо него? Оказывается, что для большинства результатов нам нужна только связь скалярного произведения с нормой, равенство параллелограмма.
\end{note}

\begin{note}
	С возникшим вопросом о базисах есть ещё две связанные темы:
	\begin{itemize}
		\item \textit{Дополняемость}. Верно ли, что для некоторого линейного нормированного пространства $E$, если взять подпространство $E_1 \subset E$, то найдётся некоторое другое подпространство $E_2$, что $E_1 \oplus E_2 = E$?
		
		\item \textit{Существование базиса Шаудера}. Верно ли, что если линейное нормированное пространство $E$ хотя бы сепарабельно, то в нём найдётся базис Шаудера? (Базис Гамеля можно построить, если у нас есть аксиома выбора, поэтому тут вопрос неинтересный)
	\end{itemize}
	Оказывается, что для гильбертовых пространств на оба вопроса ответ положителен, а для банаховых --- отрицателен (таким образом, евклидовость является существенной характеристикой пространств).
\end{note}

\begin{definition}
	Пусть $S \subseteq E$ --- подпространство евклидова пространства. Тогда \textit{аннулятором $S$} называется следующее множество:
	\[
		S^\bot = \{y \in E \colon \forall x \in S\ (x, y) = 0\}
	\]
\end{definition}

\begin{definition}
	Пусть $S \subseteq E$ --- подпространство метрического пространства, $x \notin S$. Тогда \textit{элементом наилучшего приближения множества $S$} называется следующий элемент $y$:
	\[
		y \in S \such \rho(x, y) = \rho(x, S)
	\]
\end{definition}

\begin{note}
	Естественно, элемент наилучшего приближения не обязательно существует в произвольном случае.
\end{note}

\begin{theorem} (о проекции)
	Пусть $H$ --- гильбертово пространство, $M \subseteq H$ --- подпространство. Тогда верно равенство:
	\[
		H = M \oplus M^\bot
	\]
\end{theorem}

\begin{lemma} (о существовании и единственности элемента наилучшего приближения)
	Пусть $M \subseteq H$ --- подпространство гильбертова пространтва. Тогда:
	\[
		\forall h \in H\ \exists! x \in M \such \|h - x\| = \rho(h, x) = \rho(h, M)
	\]
\end{lemma}

\begin{proof}
	Докажем две отдельные части:
	\begin{enumerate}
		\item (Единственность) Пусть существуют элементы наилучшего приближения $x_1 \neq x_2$ для фиксированного $h$. Тогда, получим противоречие с равенством параллелограмма:
		\[
			0 < \|x_1 - x_2\|^2 = \|(x_1 - h) + (h - x_2)\|^2 = -\|x_1 + x_2 - 2h\|^2 + 2\|x_1 - h\|^2 + 2\|x_2 - h\|^2
		\]
		Обозначим $d = \rho(h, M)$. Тогда:
		\[
			0 < -4\no{\frac{x_1 + x_2}{2} - h}^2 + 4d^2 \Lora \no{\frac{x_1 + x_2}{2} - h} < d
		\]
		Требуемое противоречие достигнуто, ибо $\frac{x_1 + x_2}{2} \in M$
		
		\item (Существование) Пусть $d = \rho(h, M)$. Рассмотрим ещё два подслучая:
		\begin{enumerate}
			\item $d = 0$. Так как $M$ замкнуто, то $h \in M$ и является искомым элементов наилучшего приближения.
			
			\item $d > 0$. Так как $d = \inf_{x \in M} \rho(x, M)$, то найдётся последовательность $\{x_n\}_{n = 1}^\infty$ такая, что $\lim_{n \to \infty} \|h - x_n\| = d$. Покажем, что последовательность сходится, тогда её предел есть элемент наилучшего приближения. Будем доказывать фундаментальность, а для этого воспользуемся уже известным равенством параллелограмма:
			\[
				\|x_n - x_m\|^2 = -\|x_n + x_m - 2h\|^2 + 2\|x_n - h\|^2 + 2\|x_m - h\|^2
			\]
			Зафиксируем $\eps > 0$. Тогда найдётся такой номер $N \in \N$, что будет выполнено неравенство:
			\[
				\forall n \ge N\ \ d^2 \le \|x_n - h\|^2 \le d^2(1 + \eps)
			\]
			Пусть $n \le m$, тогда
			\[
				\|x_n - x_m\|^2 \le -4d^2 + 4d^2(1 + \eps) = 4d^2\eps
			\]
			Получили нужную оценку. Итак, $\{x_n\}_{n = 1}^\infty$ фундаментальна, а стало быть её предел $x$ --- кандидат в искомый элемент. Осталось проверить, что действительно на $x$ достигается равенство, а это просто в силу непрерывности нормы:
			\[
				\lim_{n \to \infty} \|x_n - h\| = \|x - h\| = d \text{ --- в силу единственности предела}
			\]
		\end{enumerate}
	\end{enumerate}
\end{proof}

\begin{proof} (теоремы о проекции)
	Наша задача состоит в том, чтобы показать, что любой элемент $h \in H$ представим в виде $h = x + y$, где $x \in M, y \in M^\bot$. Зафиксируем произвольный $h \notin M$ (иначе разложение тривиально). Пусть $x$ --- элемент наилучшего приближения $\rho(h, M)$.  Утверждается, что $y := h - x \in M^\bot$. Проверим по определению этот факт. Пусть $z \in M$. Тогда, будем работать с $\alpha z \in M$, $\alpha \in \K$. Воспользуемся определением $x$ (далее $d = \rho(h, M)$):
	\[
		0 \le d^2 = \|h - x\|^2 \le \|h - x - \alpha z\|^2 = d^2\underbrace{ - (y, \alpha z) - (\alpha z, y)}_{-2\re((\alpha(z, y))} + |\alpha|^2\|z\|^2
	\]
	Отсюда $2\re(\alpha(z, y)) \le \alpha^2\|z\|^2$. Если рассмотреть $\eps \in \R$, $\eps > 0$ и $\alpha = \pm \eps$, то получится оценка $|2 \re((z, y))| \le \eps \|z\|^2$. Так как $\eps$ произвольно, то такое возможно тогда и только тогда, когда $\re((z, y)) = 0$. Аналогично смотрим $\alpha = \pm i\eps$, что даёт оценку $|2\im((z, y))| \le \eps \|z\|^2$ и делаем вывод $\im((z, y)) = 0$. Стало быть, $(z, y) = 0$, что и требовалось доказать.
\end{proof}