\iffalse
\subsection{Цитатник}
``Жить без аксиоматики --- это как жить долго и счастливо, но без жареной картошки. Как жить без жареной картошки?`` (Перед доказательством эквивалентных определений непрерывной функции)

``Встретить на практике неизмеримое множество настолько сложно, как и услышать в деканате, что ваша стипенидия равна корню из трёх``

Всякая сказка начинается с ``в одном царстве, в одном государстве``. Так же должно быть и в теоремах.

Разница между косвенными и прямыми доказательствами аналогична разнице между традиционной борьбой в Римской империи и вольной борьбой.
В вольной борьбе можно ставить подножки, и борьба получается более зрелищной, из-за чего её все предпочитают, но это всё равно не по традиционным правилам.

Так не надо ничего доказывать, достаточно просто нарисовать.

Как вы думаете, на каком языке велась переписка египетского МИДа?
Армейский. (Очень важный вопрос во время лекции)

Приведите пример неполного метрического пространства.
*Предлагают несколько вариантов*
Нет, у нормальных людей неполное метрическое пространство --- это $(0, 1)$.

--- Как называется теорема о том, что непрерывная функция на компакте равномерно ограничена?
--- Кажется, Кантора.
--- Не надо говорить ``кажется``, просто теорема Кантора.

Чем отличаются большие кошки от маленьких? Вот, например, пума весит 120 кг и считается маленькой кошкой, а леопард весит 90 кг и считается большой.
Потому что большие кошки умеют рычать.

Отнесите, пожалуйста, доказательство теоремы Кантора на 17 этаж главного здания МГУ.

Вот я сейчас открою дверь, сюда зайдут какие-то люди, скажут, что они пришли приближать студентов Физтеха.

Польза кирпичей в том, что их мало, и они хорошо согласованы.
Если бы видов кирпичей было много, то в чём их польза?
Тогда бы мы все здания называли кирпичами.

Синус одного градуса весит больше одного грамма.

Доказывать неравенство КБШ через неравенство Гёльдера, это как стрелять из пушки по воробьям. Причём воробей один. И уже улетел.

--- Я аннулятор пола.
--- Это не совсем корректно, вы лежите в аннуляторе.
--- *молчание* Давайте подумаем.

В гильбертовом пространстве тождество параллелограмма, прямо как настойка чебреца, является панацеей.

Как Сергея Львовича звали? Ну что вы боитесь, так и звали --- Сергей Львович.

Говоря современным языком, норма оператора --- это степень его крутизны.

В Долгопрудном есть аудитория 415, в которой сидят студенты ФИВТа. В Тамбове есть такая же аудитория с такими же студентами.
В этом и есть смысл линейных топологических пространств --- в каждой точке происходит то же самое.

Теорема Банаха-Ш говорит, что прямо ехать --- живым не бывать, или, говоря на математическом языке, существует непрерывная периодическая функция, у которой ряд Фурье сходится неравномерно.

Комплекснозначное зверьё --- это круто.

Я не путаю тот пол, на котором я стою, с потолком, который может на меня обвалиться.

Ласка писал, что к своим 57 годам он смог хотя бы частично забыть то, что он знал, и после этого он почувствовал огромное облегчение.
К сожалению, он не поделился техникой забывания, но вам она, к счастью, и не нужна --- после экзамена всё равно сразу всё забудете.

Когда ты в джунглях встречаешь другого человека, что нужно ему говорить?
Мы с тобой одной крови.

Прямая сумма пола и линейной оболочки меня --- это всё пространство.

Квадрат Малевича --- это один из поздних вариантов изображения кривой Пеано.

Хорошая фраза --- это кочан капусты.
С него отрезаешь лист, отрезаешь, а он остаётся кочаном.

В теории вероятностей всё не как у людей: они напридумывали свои термины и пользуются только ими (сходится не по мере, а по вероятности; не почти всюду, а почти наверное).

Проблема борелевской меры --- вот ты берёшь кусок колбасы, взвешиваешь его, получаешь 0 грамм.
Потом отрезаешь от него кусочек, взвешиваешь, а весы взрываются.

Ры --- это рысь, Ра --- это бог солнца.

Средняя высота великана в европейских сказках --- 5 метров.
\fi
\subsection{Введение}
\epigraph{На каком языке велась переписка египетского МИДа? На арамейском.}{Сегрей Петрович}

В XVIII веке изучали локальные свойства функций, в XIX --- глобальные свойства, а в XX веке начал развиваться функциональный анализ, изучающий функции в целом.
Полезная литература --- Dieudonne, History of FA, 1983.
Опорная книга курса --- Колмогоров, Фамин.

\textbf{Определение.} Пусть $E$ --- линейное пространство над полем $K$, где $K$ --- это $\mathbb R$ или $\mathbb C$.
$E$ будем называть \textit{линейно-топологическим пространством}, если оно является топологическим пространством, и операции сложения и умножения на скаляр из линейного пространства являются непрерывными.
Иными словами, если мы взяли окрестности точек $x_1, x_2 \in E$, то суммы будут попадать в окрестность $x_1 + x_2$, аналогично с умножением.

\textbf{Обозначение.} $\mathbb R_+$ --- это \textit{неотрицательные} числа, а не положительные.

\section{Метрические и топологические пространства}
\epigraph{Жить без аксиоматики --- это как жить долго и счастливо, но без жареной картошки. Как жить без жареной картошки?}{Сергей Петрович}

\textbf{Определение.} Пара $(X, \rho)$ называется \textit{метрическим пространством}, если выполнены обычные три свойства (неотрицательность, симметричность, неравенство треугольника).

\textbf{Определение.} Пара $(X, \tau)$ называется \textit{топологическим пространством}, если выполнены обычные три свойства (лежат $\varnothing$ и $X$, любое объединение, конечное пересечение).

\textbf{Теорема.} (б/д) Если $(X, \rho)$ --- это метрическое пространство, то $G \subset X$ открыто тогда и только тогда, когда $X \setminus G$ замкнуто.

\textbf{Теорема.} (б/д) Пусть $(X, \rho)$ --- метрическое пространство.
Тогда любое объединение открытых открыто, а конечное пересечение открыто.
Для замкнутых --- наоборот.

\textbf{Интересный факт.} На любом метрическом пространстве можно ввести топологию (например, дискретную или топологию из всех подмножеств).
Но не любое топологическое пространство можно породить метрикой, в частности, не работает в бесконечномерном гильбертовом пространстве.

\textbf{Определение.} Для МП \textit{подпространством} $X$ называется любое подмножество $Y \subset X$ с метрикой из $X$.
Для ТП аналогично, но топологией будет $G \cap Y$ для $G \in \tau_X$.

\textbf{Определение.} Для МП множество $Y$ называется \textit{ограниченным}, если $\sup_{y_1, y_2 \in Y} \rho(y_1, y_2) < \infty$.
Для ТП аналогичного определения нет.

\textbf{Определение.} Для МП \textit{расстоянием между множествами} $X, Y$ называется $\inf_{x \in X, y \in Y} \rho(x, y)$.
Для ТП аналогичного определения нет.

\textbf{Определение.} Для МП \textit{шаром} называется множество $B(x, r) = \{y \in X~|~\rho(x, y) \le r\}$.
Для ТП аналогичного определения нет.

\textbf{Определение.} Для МП последовательность $x_n$ называется \textit{сходящейся}, если $\forall B(x)~\exists N: \forall n \ge N~x_n \in B$.
Иными словами, $\lim_{n \to \infty} \rho(x_n, x) \to 0$.
Это же определение работает и для ТП.

\textbf{Определение.} $x$ называется \textit{точкой прикосновения} множества $M$, если $\forall B(x)~B \cap M \ne \varnothing$.
Или существует $\{x_n\} \subset M$, такая что $x_n \to x$.
Или для любого открытого $C(x)$ выполнено $C \cap M \ne \varnothing$.
Или расстояние до множества равно нулю.

\textbf{Определение.} $\overline M$ --- замыкание множества, то есть все точки прикосновения $M$.

\textbf{Определение.} Замкнутое множество --- замыкание совпадает со множеством.

\textbf{Определение.} $x$ является внутренней точкой $M$, если она входит с окрестностью.
В топологических пространствах вместо окрестности берётся просто открытое множества.

\textbf{Определение.} $\Int(M)$ --- внутренность множества $M$, то есть все внутренние точки.

\textbf{Определение.} Открытое множество --- совпадает со своей внутренностью.

\textbf{Определение.} $A$ плотно в $B$, если $B \subset \overline A$.

\textbf{Определение.} $A$ всюду плотно в пространстве $X$, если $\overline A = X$.

\textbf{Определение.} $A$ нигде не плотно, если $A$ не плотно ни в одном шаре или если в $\overline A$ нет шаров.

\textbf{Определение.} $X$ сепарабельно, если в $X$ существует не более чем счётное всюду плотное множество.

\textbf{Определение.} $X$ не связно, если его можно представить в виде объединения $X = G_1 \sqcup G_2$ открытых множеств.

\textbf{Определение.} $\tau$ называется \textit{базой топологии}, если любое множество можно из них получить.

\textbf{Определение.} Топология $\tau_1$ \textit{сильнее} топологии $\tau_2$, если $\tau_1 \supset \tau_2$.

\textbf{Теорема.} (1.3) Пусть $(X, \rho)$ --- метрическое пространство, $r > 0$.
$B(x, r)$ --- это открытое множество.
$\Int(M)$ --- открытое множество.
$\overline M$ --- замкнутое множество.
$\overline B(x, r)$ --- замкнутое множество.

\textbf{Доказательство.} 1) Внимательно посмотрим на расстояние до границы.
2) Заметим, что если $M_1 \subset M_2$, то $\Int(M_1) \subset \Int(M_2)$.
Рассмотрим $x \in \Int(M)$, тогда есть окрестность $B(x) \subset M$.
А отсюда $\Int(B(x)) \subset \Int(M)$.
Но $\Int(B(x)) = B(x)$, так что $x$ является внутренней.
Остальное в качестве упражнения.

\QED

\subsection{Сравнение методов математического и функционального анализа}
\epigraph{Средняя высота великана в европейских сказках --- 5 метров}{Сергей Петрович}

\textbf{Задача 1.} Существует непрерывная функция, не дифференцируемая ни в одной точке.
В матанализе строят фракталы.
Функциональный анализ: теорема Бэра гласит, что полное метрическое пространство является множеством второй категории.
Множество имеет первую категорию, если оно представляется в виде объединения нигде не плотных множеств, иначе --- вторую.
А множество функций, у которых хотя бы в одной точке есть односторонняя производная, нигде не плотно.

\textbf{Задача 2.} Существует непрерывный периодический функциональный ряд Фурье, который не является равномерно сходящимся.
В матанализе строится какой-то жуткий пример, который никто не может запомнить.
В функциональном анализе мы вводим оператор $S_n: CP[-\pi, \pi] \to CP[-\pi, \pi]$, то есть сопоставляем функции $f(\cdot)$ её частичную сумму ряда Фурье $S_n(f, \cdot)$.
Тогда $S_n \rightrightarrows f$ и по теореме Банаха--Литейнгауза $S_n \to Id$ поточечно тогда и только тогда, когда $\sup_n \|S_n\| < \infty$.

\textbf{Задача 3.} Дифференциальное уравнение:
\[
    \begin{cases}
        y' = f(x, y) \\
        y(x_0) = y_0
    \end{cases} .
\]
В каких случаях у него есть решение?
В матанализе мы доказывали теорему о том, что $f$ должна быть непрерывна и и $f'_y$ должна быть непрерывна, делали мы это через теорему Банаха о сжимающих отображениях.
В функциональном анализе можно доказать и без непрерывности, правда, единственность гарантировать нельзя.

\subsection{Словарь для топологических пространств}
\epigraph{Встретить на практике неизмеримое множество настолько сложно, как услышать в деканате, что ваша стипендия равна корню из трёх}{Сергей Петрович}

\textbf{Утверждение.} Пусть $\{x_n\}$ --- последовательность в метрическом пространстве.
Следующие два утверждения эквивалентны:
\begin{enumerate}
    \item $\forall B(x)~\exists N: \forall n \ge N~x_n \in B(x)$.
    \item То же самое, но вместо $B(x)$ берём открытое множество.
\end{enumerate}
$2 \Rightarrow 1$ по теореме 1.3, обратно очевидно.

Аналогично доказывается эквивалентность определений для точки прикосновения множества.

\textbf{Утверждение.} В топологическом пространстве открытое множество совпадает со своей внутренностью.
$\Rightarrow$ очевидно, $\Leftarrow$ берём все точки с их окрестностями, в объединении получим исходное множество.

\textbf{Утверждение.} В топологическом пространстве замыкание множества замкнуто.

\textbf{Доказательство.} $\Rightarrow$. Рассмотрим $x \not\in \overline M$.
По определению не точки прикосновения найдётся $G(x)$ --- открытая окрестность, такая что $G(x) \cap \overline M = \varnothing$.
Следовательно, $X \setminus \overline M$ открыто.

$\Leftarrow$ очевидно.

\QED

\textbf{Определение.} Пусть $X, Y$ --- топологические пространства, $f: X \to Y$.
$f$ называется \textit{непрерывной в точке} $x_0 \in X$, если для любой $V(f(x_0))$ найдётся $U(x_0)$, такая что $f(U) \subset V$.

Функция просто непрерывна, если она непрерывна во всех точках $x \in X$.

\textbf{Теорема.} Пусть $X, Y$ --- топологические пространства, $f: X \to Y$.
Следующие утверждения эквивалентны:
\begin{enumerate}
    \item $f$ непрерывна.
    \item Прообраз открытого открыт.
    \item Прообраз замкнутого замкнут.
\end{enumerate}

\textbf{Доказательство.} $1 \Rightarrow 2$. 
Рассмотрим $G \in \tau_Y$, $x \in f^{-1}(G)$.
Тогда из непрерывности найдётся окрестность $U(x) \subset X$, такая что $f(U) \subset G$, или же $U \subset f^{-1}(G)$.
Беря объединение таких $U$ по всем $x$, получаем в точности $f^{-1}(G)$.
Так как мы объединяли открытые множества, получилось тоже открытое.

$2 \Rightarrow 1$. Рассмотрим $x \in X$, возьмём какую--нибудь окрестность $V$ точки $f(x)$.
Тогда $f^{-1}(V)$ содержит $x$ и является открытым.

$2 \iff 3$ очевидно.

\QED

\textbf{Определение.} Пусть $X, Y$ --- топологические пространства.
Они называются \textit{гомеоморфными}, если существует $f: X \to Y$, такое что $f$ --- биекция и $f, f^{-1}$ непрерывны.

\textbf{Теорема.} (Брауэра, б/д) Если $h: C \to C$ непрерывно, где $C$ --- круг, то у него есть стационарная точка.

\textbf{Следствие.} Если $S$ гомеоморфно кругу, $g: S \to S$ непрерывна, то у $g$ найдётся стационарная точка.

\textbf{Доказательство.} Положим $f: C \to S$ --- гомеоморфизм из круга в $S$.
Тогда $f^{-1} \circ g \circ f: C \to C$ подходит под теорему Брауэра, значит, существует стационарная точка $x$.
Иными словами, $f^{-1} \circ g \circ f(x) = x$, теперь применим $f$ к обеим частям: $g(f(x)) = f(x)$, получили стационарную точку $f(x)$ отображения $g$.

\QED

\section{Полные метрические пространства}
\epigraph{У нормальных людей пример неполного метрического пространства --- это $(0, 1)$}{Сергей Петрович}

\textbf{Определение.} Метрическое пространство $X$ называется \textit{полным}, если всякая фундаментальная последовательность сходится.

\textbf{Определение.} Пусть $X$ --- неполное метрическое пространство. 
Полное метрическое пространство $Y$ называется \textit{пополнением} $X$, если существует изометрия $\pi: X \to Y$ (отображение, сохраняющее расстояния), такая что $\overline{\pi(X)} = Y$.

\textbf{Теорема.} (Хаусдорфа, 2.4, б/д) Для любого неполного метрического пространства $X$ существует пополнение, определённое единственным образом с точностью до изометрии, если зафиксировать $x_0 \in X$, такую что $\pi(x_0) = x_0$, где $\pi$ берётся из определения пополнения.

\textbf{Определение.} Набор замкнутых шаров $\{\overline B_n(x_n, r_n)\}$ называется \textit{вложенным}, если $\overline B_{n+1} \subset \overline B_n$.

\textbf{Теорема.} (Принцип вложенных шаров) Пусть $X$ --- полное метрическое пространство, $\{\overline B_n\}$ --- последовательность вложенных шаров, $r_n \to 0$.
Тогда существует единственный $x$, такой что $x \in \bigcap_n \overline B_n$.

\textbf{Доказательство.} Рассмотрим последовательность центров шаров $\{x_n\}$.
Заметим, что она фундаментальна, ибо $\rho(x_{n+1}, x_n) \le r_n \to 0$.
Теперь из полноты у неё есть предел $x$.
Теперь рассмотрим неравенство $\rho(x_{n+p}, x_n) \le r_n$, перейдём к пределу при $p \to \infty$, тогда $\rho(x, x_n) \le r_n$ для всех $n$, то есть лежит во всех шарах.

Единственность. Рассмотрим две точки $x', x''$ в пересечении, тогда они обе лежат в шаре радиуса $r_n$ для всех $n$, то есть $\rho(x', x'') \le r_n$.
Переходя к пределу, получаем $\rho(x', x'') = 0$.

\QED

\textbf{Теорема.} (Бэр, 2.2) Пусть $X$ --- полное метрическое пространство, тогда $X$ является множеством второй категории, то есть оно не представимо в виде счётного объединения нигде не плотных множеств.

\textbf{Доказательство.} После долгой прелюдии про то, почему доказательства от противного --- это зло, докажем от противного.
Пусть $X = \bigcup_{n=1}^\infty M_n$, такие что все $\overline M_n$ не содержат шар.
Положим $\overline B_0(x_0, r_1 = 1)$.
Так как $M_1$ не плотно в $\overline B_0$, найдётся $\overline B_1 \subset \overline B_0$, такой что $\overline B_1 \cap M_1 = \varnothing$.
Построим такой же шарик $\overline B_2 \subset \overline B_1$ для множества $M_2$ и так далее.
Дополнительно сузим радиусы шаров так, чтобы $r_n \le \frac{1}{2^n}$.
Получили последовательность вложенных замкнутых шаров, поэтому их пересечение является точкой $x$.

Следовательно, $x$ не лежит ни в одном из $M_n$, однако он лежит в $X = \bigcup_{n=1}^\infty M_n$ --- противоречие.

\QED

\textbf{Теорема.} (Банах, 2.3) Пусть $X$ --- полное метрическое пространство, $f: X \to X$ --- сжимающее отображение.
Тогда у $f$ существует единственная неподвижная точка.

План доказательства (было на диффурах, даже есть в прошлогоднем конспекте): возьмём точку $x_0$ и рассмотрим последовательность $\{f^n(x_0)\}$ --- она окажется фундаментальной, а значит, сходящейся.

\section{Компактные топологические и метрические пространства}
\epigraph{
Отнесите, пожалуйста, доказательство теоремы Кантора на 17 этаж главного здания МГУ.
}{Сергей Петрович}

Компактные метрические пространства --- это подмножество полных, а компактные топологические --- просто топологических.

\textbf{Определение.} Топологическое пространство $X$ называется \textit{компактным}, если из всякого открытого покрытия $X$ можно выделить конечное подпокрытие.

\textbf{Определение.} Система подмножеств $\{M_\alpha\}_{\alpha \in A}$ множества $X$ называется \textit{центрированной}, если любая конечная подсистема $\bigcap_{k=1}^n M_{\alpha_k} \ne \varnothing$.

\textbf{Теорема.} (3.1) Топологическое пространство $X$ является компактным тогда и только тогда, когда любая центрированная система замкнутых множеств имеет непустое пересечение.
(В определении центрированности мы требуем непустоту пересечения лишь конечных кусков, а тут всей системы)

\textbf{Доказательство.} $\Rightarrow$. Пусть $\{F_\alpha\}$ --- ЦСЗМ, положим $G_\alpha = X \setminus F_\alpha$.
Заметим, что никакая конечная подсистема $G_\alpha$ не является покрытием $X$.
Действительно, рассмотрим конечную подсистему, $\{G_k\}$, тогда
\[
    \bigcup_{k=1}^n G_k = X \setminus \bigcap_{k=1}^n F_k \ne X,
\]
так как по определению центрированности пересечение непусто.

Теперь по условию $X$ --- компакт, следовательно, $G_\alpha$ не может являться его покрытием по доказанному выше, то есть объединение не даёт $X$, то есть пересечение $F_\alpha$ не даёт пустое множество.

$\Leftarrow$. Возьмём открытое покрытие $G_\alpha$, положим $F_\alpha = X \setminus G_\alpha$, тогда $\bigcap_{\alpha \in A} F_\alpha = \varnothing$.
По условию $\{F_\alpha\}$ не является центрированной, значит, найдётся конечная подсистема с пустым пересечением, значит, её дополнение является покрытием $X$.

\QED

\textbf{Упражнение.} Пусть $X$ --- компактное топологическое пространство, $Y$ --- просто топологическое пространство, $f: X \to Y$ непрерывна.
Тогда $f(X) \subset Y$ является компактом.

\textbf{Определение.} Множество $A$ метрического пространства $X$ называется \textit{вполне ограниченным}, если для любого $\varepsilon > 0$ найдётся конечное множество $\{b_1, \dots, b_n\}$, такое что $A \subset \bigcup_{k=1}^n B(b_k, \varepsilon)$.
Такое конечное множество называется \textit{$\varepsilon$--сетью}, и оно не обязательно является подмножеством $A$.

\textbf{Замечание.} Из вполне ограниченности следует ограниченность, а наоборот уже не следует.
Контрпример --- $l_2$ и подмножество базисных векторов с $\varepsilon = \frac{1}{2}$.

\textbf{Теорема.} (3.2) Пусть $X$ --- метрическое пространство. Следующие условия эквивалентны:
\begin{enumerate}
    \item $X$ компактно.
    \item $X$ полное и вполне ограничено.
    \item Из любой последовательности элементов $X$ можно выделить сходящуюся подпоследовательность.
    \item Всякое бесконечное множество в $X$ имеет предельную точку.
    \item $C(X, \mathbb R) \subset B(X, \mathbb R)$, то есть всякая непрерывная функция ограничена (\underline{B}ounded).
\end{enumerate}

\textbf{Доказательство.} $1 \Rightarrow 2$. 
Рассмотрим фундаментальную последовательность $\{x_n\}_{n=1}^\infty \subset X$.
Положим хвост последовательность $\overline A_n = \overline {\{x_n, x_{n+1}, \dots \} } \subset X$.
Тогда каждый следующий хвост вложен в предыдущий, а $\{\overline A_n\}$ является центрированной системой замкнутых множеств и по теореме 3.1 её пересечение непусто.
Рассмотрим точку $x_0$ из пересечения и докажем, что к ней последовательность и сходится.
Напишем определение фундаментальности:
\[
    \forall \varepsilon > 0~\exists N \in \mathbb N: \forall n, m \ge N~\rho(x_n, x_m) < \varepsilon.
\]
Возьмём произвольный $\varepsilon$, достанем $N$, тогда найдётся $n \ge N$, такое что $x_n = x_0$, ибо он лежит в пересечении всех хвостов.
Так как теперь для всех $m \ge N$ выполнено $\rho(x_0, x_m) < \varepsilon$, получаем определение предела.

Теперь докажем вполне ограниченность.
Зафиксируем $\varepsilon > 0$, рассмотрим шарики $B(x, \varepsilon)$.
Их объединение даёт $X$, из компатности существует конечное подпокрытие.

$2 \Rightarrow 3$. Зафиксируем $\varepsilon = \frac{1}{N}$.
Рассмотрим последовательность $\{x_n\} \subset X$ и $\varepsilon$--сеть.
Найдётся шарик $B_N$, в котором будет бесконечно элементов последовательности.
Построим сходящуюся: первый элемент $x_{n_1}$ берём из $B_1$, второй $x_{n_2}$ берём из $B_2$, и так далее.
Но нужно ещё брать шарики аккуратно, а именно, будем пересекать $B_{N+1}$ с $B_N$.

$3 \Rightarrow 2b$. Докажем вполне ограниченность от противного.
Зафиксируем $\varepsilon_0 > 0$, для которого нет $\varepsilon$--сети.
Возьмём $x_1 \in X$ произвольное.
Так как нет $\varepsilon_0$--сети, найдётся $x_2$, такой что $\rho(x_1, x_2) > \varepsilon_0$.
Далее возьмём $x_3$ и так далее.
По итогу найдётся последовательность, у которой все элементы находятся на расстоянии хотя бы $\varepsilon_0$, то есть, сходящейся подпоследовательности нет, противоречие.

$3 \Rightarrow 1$. Опять от противного, пусть существует открытое покрытие $\{G_\alpha\}$, из которого нельзя выделить конечное подпокрытие.
Раз уж доказали вполне ограниченность, то воспользуемся: берём $\varepsilon = \frac{1}{n}$, найдётся $\varepsilon$--сеть $\{\overline B(x_k, \varepsilon)\}$.
Тогда найдётся шарик, который нельзя покрыть конечным числом множеств.

Пусть $\{y_n\}_{n=1}^\infty$ --- последовательность центров плохих шаров.
Перейдём к сходящейся подпоследовательности, пусть $y$ --- её предел.
Он лежит в каком-то $G \in \{G_\alpha\}$, а раз оно открыто, то лежит с окрестностью $B(y)$.
Теперь заметим, что рано или поздно плохие шарики начнут попадать в $G$, ибо центры стремятся к $y$, а радиусы --- к нулю, противоречие.

$3 \Rightarrow 4$. Пусть $E$ бесконечно, возьмём $\{x_n\}_{n=1}^\infty \subset E$.
По условию можно выделить сходящуюся к $x_0$, это и есть предельная точка.

$3 \Leftarrow 4$. Возьмём $\{x_n\}$. Если у неё конечное число различных значений, то какое-то будет встречаться бесокнечно часто, его и возьмём.
Иначе по условию найдётся предельная точка.

$1 \Rightarrow 5$. По упражнению, ибо компакт ограничен.

$5 \Rightarrow 2$. От противного. Пусть у нас нет полноты $X$ или нет вполне ограниченности.

Итак, берём неполное пространство.
Возьмём его пополнение $Y$, пусть $\pi$ --- изометрия в него.
Пусть $x_0$ --- предельная точка, не лежащая в $X$.
Наконец, положим $f(x) = \frac{1}{\rho(\pi(x), y_0)}$, где $y_0$ --- предел $f(x_0)$.

Теперь случай не вполне ограниченного пространства.
Значит, существует $\varepsilon$, такой что в нём нет $2 \varepsilon$--сети.
Возьмём $x_1 \in X$, $x_2 \in X \setminus B_{\varepsilon}(x_1)$, и так далее, выкидываем шары и строим последовательность $B_n$ шаров и $x_n$ --- их центров.
Положим
\[
    f(x) =
    \begin{cases}
        n \cdot \left( 1 - \frac{\rho(x, x_n)}{\varepsilon} \right), & x \in \overline{B_n} \\
        0, & x \not\in \bigcup B_n
    \end{cases} .
\]
Эта функция непрерывна, и $f(x_n) = n$.

\QED

\textbf{Определение.} Пусть $X$ --- топологическое пространство, $M \subset X$.
$M$ называется \textit{предкомпактным} (относительно компактным), если замыкание компактно.
Пример: открытый круг на плоскости.

\textbf{Упражнение.} Пусть $X$ --- метрическое пространство, $M$ --- предкомпакт.
Тогда $M$ --- вполне ограниченное множество.
Более того, если $X$ ещё и полное, то вполне ограниченность эквивалентна предкомпактности.

Теперь мы хотим классифицировать вполне ограниченные множества в полном метрическом пространстве.
Например, одна из частей --- это ограниченные множества, но это далеко не всё.

\textbf{Теорема.} (Кантора) Пусть $X$ --- компактное метрическое пространство, $f \in C(X, \mathbb R)$.
Тогда $f$ равномерно непрерывна на $X$.

\textbf{Доказательство.} Первый способ, нормальный.
Зафиксируем $\varepsilon > 0$, $x \in X$, выберем $B(x)$ так, чтобы для $y \in B(x)$ было выполнено $|f(x) - f(y)| < \varepsilon$.
Положим $\tilde B(x)$ --- тот же шар, но с радиусом в два раза меньше.
Теперь $\bigcup_{x \in X} \tilde B(x)$ является открытым покрытием $X$, следовательно, существует конечное подпокрытие $\tilde B_1, \dots, \tilde B_n$.
Положим $\delta = \min_{k} \left( \frac{r_k}{2} \right)$.
Дальше классика: берём $x \in X$, находим, в каком шарике $\tilde B(x_k)$ он находится, засчёт подгона констант $\delta$--окрестность вложена в $B(x_k)$.

Второй способ, душно от противного.
Допустим, что нашлась непрерывная $f$ не равномерно непрерывная, то есть найдётся $\varepsilon_0 > 0$, такое что для всех $\delta > 0$ найдутся две точки $x, y$, такие что $|x - y| < \delta$ и $|f(x) - f(y)| \ge \varepsilon_0$.
Положим $\delta_n = \frac{1}{n}$, сопоставим им $x_n, y_n \in X$.
По эквивалентному определению №3 компактных метрических пространств выберем сходящуюся подпоследовательность $x_{n_k}$.
Тогда $x_{n_k}, y_{n_k} \to x_0$
Следовательно,
\[
    \varepsilon_0 \le |f(x_{n_k}) - f(y_{n_k})| \le |f(x_{n_k}) - f(x_0)| + |f(y_{n_k}) - f(x_0)| \to 0.
\]

\QED

\textbf{Теорема.} (3.3, Арцéла--Аскóли) Пусть $X$ --- компактное метрическое пространство, $M \subset C(X, \mathbb R)$. (или $C(X, \mathbb C)$).
Тогда $M$ является предкомпактным тогда и только тогда, когда выполнены следующие условия
\begin{enumerate}
    \item $M$ ограничено (или равномерно ограничено).
    \item $M$ \textit{равностепенно непрерывно}, то есть
        \[
            \forall \varepsilon > 0~\exists \delta > 0: \forall f \in M~(\rho(x_1, x_2) < \delta \Rightarrow |f(x_1) - f(x_2)| < \varepsilon).
        \]
        Иными словами, это равномерная непрерывность ``равномерно`` по множеству $M$.
\end{enumerate}

\textbf{Доказательство.} $\Rightarrow$. Возьмём $M$, покажем ограниченность.
По упражнению $M$ вполне ограничено, тогда существует $\varepsilon$--сеть $\{\phi_1, \dots, \phi_n\}$.
Значит, $M$ вложена в конечное объединение шаров, то есть ограничено.

Покажем равностепенную непрерывность, зафиксируем $\varepsilon > 0$.
Положим $\delta = \min_k(\delta_k)$, где $\delta_k$ берётся из определений равномерной непрерывности (по теореме Кантора) для функций $\phi_k$.
Теперь неравенство треугольника: рассмотрим $f \in M$, найдётся $\phi_k$, такая что $\|f - \phi_k\| < \varepsilon$.
Для всех $x, y \in X$, таких что $\rho(x, y) < \delta$, выполнено
\[
    |f(x) - f(y)| \le |f(x) - \phi_k(x)| + |\phi_k(x) - \phi_k(y)| + |\phi_k(y) - f(y)| < 3 \varepsilon.
\]

$\Leftarrow$. Из-за того, что лектор не умеет рисовать произвольные компакты, будем считать, что $X = [a, b]$.
По упражнению вновь достаточно доказать вполне ограниченность.

Зафиксируем $C$, такое что для всех $f \in M$ выполнено $\|f\| \le C$.
Возьмём $\varepsilon > 0$, достанем $\delta$ из равномерной непрерывности.
Теперь разобьём $[a, b]$ на подотрезки $a = x_1 < x_2 < \dots < x_k = b$ так, чтобы все $\Delta x_i$ были меньше $\delta$.

Построим $5 \varepsilon$--сеть.
А именно, отметим все точки $(x_i, y)$, где $y$ --- разбиение $[0, C]$ на отрезки длины $\varepsilon$, конструкция на рисунке 1.

\begin{figure}[ht]
    \centering
    \incfig{arcela-askoli}{0.5\linewidth}
    \caption{Построение ломаных}
\end{figure}

Теперь в качестве сети возьмём все ломаные, проходящие через эти точки.
Докажем, что она подходит: рассмотрим $f \in M$, подберём $\psi$ понятным образом.
Рассмотрим $x \in [a, b]$, пусть $x \in [x_i, x_{i+1}]$, теперь злостно применим неравенство треугольника:
\[
    |f(x) - \psi(x)| \le |f(x) - f(x_k)| + |f(x_k) - \psi(x_k)| + |\psi(x_k) - \psi(x)| \le
\]
\[
    \le \varepsilon + \varepsilon + |\psi(x_k) - \psi(x)|.
\]
Оценки получены по построению, но последний модуль чуть тяжелее.
Для этого воспользуемся линейностью $\psi$ на $[x_i, x_{i+1}]$.
\[
    |\psi(x) - \psi(x_i)| \le |\psi(x_{i+1}) - \psi(x_i)| \le
\]
\[
    \le |\psi(x_{i+1}) - f(x_{i+1})| + |f(x_{i+1}) - f(x_i)| + |f(x_i) - \psi(x_i)| \le 3 \varepsilon.
\]

\QED

\section{Линейные топологические и нормированные пространства}
\epigraph{
Чем отличаются большие кошки от маленьких? Вот, например, пума весит 120 кг и считается маленькой кошкой, а леопард весит 90 кг и считается большой.
Потому что большие кошки умеют рычать.
}{Сергей Петрович}

\textbf{Определение.} Пусть $E$ --- линейное пространство над полем $K \in \{\mathbb R, \mathbb C\}$.
Оно называется \textit{топологическим}, если на нём введена топология, в которой операции сложения и умножения на скаляр непрерывны.

\textbf{Определение.} Пусть $E$ --- линейное пространство, и функция $\|\cdot\|: E \to \mathbb R$ обладает следующими свойствами:
\begin{itemize}
    \item $\|x\| \ge 0$, $\|x\| = 0 \iff x = \ominus$ (нулевой элемент).
    \item $\|\alpha x \| = |\alpha| \cdot \|x\|$.
    \item $\|x + y\| \le \|x\| + \|y\|$.
\end{itemize}
Является подмножеством линейных топологических пространств.

\subsection{Маленький словарик}
\epigraph{Синус одного градуса весит больше одного грамма.}{Сергей Петрович}
\textbf{Определение.} Множество $E$ называется \textit{линейным многообразием}, если оно замкнутно относительно линейных операций.

\textbf{Определение.} Множество $E$ называется \textit{подпространством}, если оно замкнуто относительно линейных операций и топологически.

\textbf{Пример.} Пусть $X = C[a, b]$, $E \subset X$ --- множество многочленов.
Оно является линейным многообразием, но по теореме Вейерштрасса его замыканием является всё $X$, так что подпространством не является.

\textbf{Определение.} \textit{Линейная оболочка} множества $S$ --- это $[S]$, множество конечных линейных комбинаций элементов $S$.

\textbf{Определение.} \textit{Банахово пространство} --- линейное нормированное пространство, полное относительно своей метрики.

\textbf{Определение.} Множество $E$ называется \textit{выпуклым}, если любые две точки входят с отрезком.

\textbf{Определение.} Пусть $E$ --- линейное пространство, $\|\cdot\|_1$ и $\|\cdot\|_2$ --- две нормы на нём.
Норма $1$ \textit{не слабее} нормы $2$, если найдётся константа $C > 0$, такая что для всех $x \in E$ выполнено $\|x\|_1 \ge C \|x\|_2$.

Две нормы называются \textit{эквивалентными}, если они обе не слабее друг друга.

\textbf{Замечание.} Это определение является не совсем логичным, так как более сильная метрика порождает более слабую топологию, к этому вернёмся позже.

\textbf{Определение.} Система $E = \{e_n\}_{n=1}^{\infty}$ называется \textit{базисом} (Гамеля), если для любого $x \in X$ найдётся единственная последовательность $\{\alpha_n\} \subset K$, такая что $\|x - \sum_{n=1}^{N} \alpha_n e_n \| \to 0$.

\textit{Алгебраический базис}, или базис Шаудера, --- такой базис, что каждый элемент представляется конечной линейной комбинацией.

\textbf{Определение.} Система $S \subset E$ называется \textit{полной}, если $\overline{[S]} = E$.

\textbf{Замечание.} Система $\{x^k\}_{k=0}^\infty$ является полной в $C[-1, 1]$, но не является базисом.

\subsection{Конец словаря}
\epigraph{--- Как Сергея Львовича звали?\\
--- *Тишина*\\
--- Ну что вы боитесь, так и звали --- Сергей Львович.
}

\textbf{Утверждение.} Пусть $E$ --- линейное пространство, $\dim(E)$ конечно. Тогда в $E$ все нормы эквивалентны.

\textbf{Доказательство.} Выделим ортонормированный базис $\mathfrak E = \{e_1, \dots, e_n\}$.
Тогда $E \cong K^n$.
Введём норму $\|x\|_2 = \sqrt{\sum_{k=1}^{n} \alpha_k^2}$, где $x \leftrightarrow_{\mathfrak E} (\alpha_1, \dots, \alpha_n)$.
Теперь рассмотрим произвольную норму $\|\cdot\|$, докажем, что найдётся $C$, такое что $\|\cdot\| \le C\|\cdot\|_2$.
Распишем арифметику:
\[
    \|x\| = \left\| \sum_{k=1}^{n} \alpha_k e_k \right\| = \sum_{k=1}^{n} |\alpha_k| \cdot \|e_k\| \le \max_k \|e_k\| \cdot \sum_{k=1}^{n} |\alpha_k| \le
\]
\[
    \le \sqrt n \sqrt{ \sum_{k=1}^{n} \alpha_k^2 } = \sqrt n \cdot \|x\|_2.
\]

Теперь покажем обратное неравенство. Допустим, что нашлась последовательность $\{x_m\}$, такая что $\|x_m\|_2 > m \cdot \|x_m\|$.
Положим $y_m = \frac{x_m}{\|x_m\|_2} \in S(0, 1)_2$ (единичная сфера с центром в нуле в $2$--норме).
Но такая сфера является компактом в $K^n$ с нормой $\|\cdot\|_2$, следовательно, можно выделить сходящуюся, пусть $y$ --- её предел.
По пункту $1$ эта последовательность сходится и по норме $\|\cdot\|$.
Значит, $\|y_{m_k} - y\| \to 0$.
Но в то же время
\[
    \|y_{m_k}\| = \frac{\|x_{m_k}\|}{\|x_{m_k}\|_2} < \frac{1}{m_k} \to 0,
\]
то есть $y = \ominus$.
Остаётся вспомнить, что $y_m$ лежала на единичной сфере, так что $y$ лежит там же, что невозможно.

\QED

\textbf{Утверждение.} Пусть $E$ --- линейное пространство, $E \supset L = [e_1, \dots, e_n]$.
Тогда $L$ является подпространством. Очевидно.

\textbf{Лемма.} (О почти перпендикуляре) Пусть $E$ --- линейное нормированное пространство, $E_1 \subset E$ --- собственное подпространство, $\varepsilon > 0$.
Тогда найдётся $y \in E$, такое что $\|y\| = 1$ и $\rho(y, E_1) \ge 1 - \varepsilon$.

\textbf{Замечание.} $\varepsilon = 0$ брать нельзя. $E_1$ не обязательно замкнутое: можно взять $E = C[0, 1]$ и $E_1$ --- множество многочленов.

\textbf{Доказательство.} Так как $E_1$ --- собственное подпространство, найдётся $y_0 \in E \setminus E_1$.
Положим $d = \rho(y_0, E_1) > 0$, потому что подпространство замкнуто с точки зрения топологии, тогда по определению расстояния найдётся $z_0 \in E_1$, такой что $d \le \|y_0 - z_0\| < d(1 + \varepsilon)$.
Остаётся рассмотреть $y = \frac{y_0 - z_0}{\|y_0 - z_0\|}$.
Очевидно $\|y\| = 1$, остаётся доказать факт про расстояние.
Пусть $\alpha = \frac{1}{\|y_0 - z_0\|}$, $z \in E_1$, тогда
\[
    \|\alpha(y_0 - z_0) - z\| = |\alpha| \cdot \left\|y_0 - \left( z_0 + \frac{z}{\alpha} \right) \right\| \ge |\alpha| \cdot d,
\]
так как это расстояние от $y_0$ до элемента $E_1$.
Продолжая оценку,
\[
    |\alpha| \cdot d = \frac{d}{\|y_0 - z_0\|} > \frac{1}{1 + \varepsilon} > 1 - \varepsilon.
\]
Первое неравенство по построению $z_0$.

\QED

\textbf{Теорема.} (Рисса, 4.1) Пусть $E$ --- линейное нормированное пространство, $\dim(E) = \infty$.
Тогда единичная сфера в $E$ не является компактном (и даже не вполне ограниченным).

\textbf{Пример.} В $l_2$ можно рассмотреть множество $\{e^{n}\}$ базисных векторов.
Они все лежат на единичной сфере, и расстояние между всеми парами равно $\sqrt 2$.
Получается растопырка, и её не покрыть $\varepsilon$--сетью, так как каждому базисному вектору нужно своё множество.

\textbf{Доказательство.} Итак, аналогично примеру будем использовать лемму о почти перпендикуляре для построения растопырки.
Берём $\varepsilon > 0$ и $x_1 \in E$, такой что $\|x_1\| = 1$, положим $E_1 = [x_1]$.
По лемме о почти перпендикуляре найдётся $x_2$, такое что $\|x_2\| = 1$ и $\|x_2 - x_1\| > 1 - \varepsilon$.
Дальше берём $x_3$, почти перпендикулярное $E_2 = [x_1, x_2]$, и так далее.

Получили растопырку аналогичную примеру в $l_2$, аналогично и здесь не будет вполне ограниченности.

\QED

\textbf{Замечание.} А если $\dim(E) < \infty$, то уже будет компактом. Вместе с теоремой Рисса получаем

\textbf{Следствие.} Единичная сфера в $E$ является компактом тогда и только тогда, когда $\dim(E) < \infty$.

\textbf{Примеры.}
\begin{itemize}
    \item Единичная сфера в $C[0, 1]$ --- не компакт.
    \item Единичная сфера в $C^1[0, 1]$ --- не компакт.
    \item Единичная сфера пространства $C^1[0, 1]$ в пространстве $C[0, 1]$ --- предкомпакт, по теореме Арцела--Асколи.
\end{itemize}

\subsection{Евклидовы пространства}
\epigraph{
--- Я аннулятор пола.\\
--- Это не совсем корректно, вы лежите в аннуляторе.\\
--- *молчание* Давайте подумаем.
}

\textbf{Определение.} \textit{Евклидово пространство} --- это пространство $E$ со скалярным произведением $(\cdot, \cdot)$, обладающим следующими четырмя свойствами:
\begin{enumerate}
    \item $(x, x) \ge 0$, $(x, x) = 0 \iff x = \ominus$.
    \item $(x, y) = \overline{(y, x)}$.
    \item $(\alpha x, y) = \alpha (x, y)$.
    \item $(x + y, z) = (x, z) + (y, z)$.
\end{enumerate}
Норма вводится, как $\|x\| = \sqrt{(x, x)}$.

\textbf{Определение.} Если убрать условие $(x, x) = 0 \iff x = \ominus$, то скалярное произведение называется \textit{полускалярным}.

\textbf{Упражнение.} Доказать неравенство КБШ для полускалярного произведения.

\textbf{Напоминание.} (Неравенство Бесселя) Пусть $\{e_n\}$ --- ортонормированная система в евклидовом пространстве $E$.
Тогда для всех $N \in \mathbb N$
\[
    \sum_{n=1}^{N} |(x, e_n)|^2 \le \|x\|^2.
\]
Теперь неравенство Коши--Буняковского--Шварца --- это неравенство Бесселя для $N = 1$, если взять $x$ произвольное и $e_1 = \frac{y}{\|y\|}$.

\textbf{Определение.} \textit{Банахово пространство} --- полное нормированное пространство.

\textit{Гильбертово пространство} --- полное евклидово пространство. \\
\underline{Сепарабельность не требуется}, хотя и в некоторых книгах её включают в определение.
Мы так делаем для того, чтобы нельзя было читерить выделением базиса и чисто алгебраической работой с рядами.
Компенсировать это мы будем равенством параллелограмма
\[
    \|x - y\|^2 + \|x + y\|^2 = 2\|x\|^2 + 2\|y\|^2.
\]

\textbf{Теорема.} (Фреше, фон Нойман и другие, 4.2, б/д)
Норма в линейном нормированном пространстве порождается скалярным произведением тогда и только тогда, когда выполнено равенство параллелограмма.
Скалярное произведение определяется через норму формулой
\[
    (x, y) = \frac{\|x + y\|^2 - \|x - y\|^2}{4}.
\]

\textbf{Определение.} Пусть $E$ --- евклидово пространство, $S \subset E$.
Его \textit{аннулятором} называется 
\[
    S^\bot = \{y \in E~|~\forall s \in S~(s, y) = 0\}.
\]
Иными словами, множество элементов, перпендикулярных всему множеству $S$.

\textbf{Упражнение.} $S^\bot = [S]^\bot = \left(\overline{[S]} \right)^\bot$, а также аннулятор является подпространством.

\textbf{Подарок.} Рассмотрим $f \in C[0, 1]$. Мы хотим построить формульно последовательность многочленов из теоремы Вейерштрасса, которая будет сходиться к нему.
С этим помогает \textit{теорема Бернштейна}:
\[
    P_n(x) = \sum_{k=0}^{n} C_n^k x^k (1 - x)^{n-k} f \left( \frac{k}{n} \right).
\]

\textbf{Определение.} Пусть $E$ --- линейное нормированное пространство, $M \subset E$.
Элемент $y \in M$ называется \textit{элементом наилучшего приближения} для $x \in E$, если $\rho(x, y) = \rho(x, M)$.

\textbf{Лемма.} Пусть $H$ --- гильбертово пространство, $M \subset H$ --- подпространство.
Тогда для любого $h \in H$ найдётся единственный $x \in M$, являющийся элементом наилучшего приближения.

\textbf{Доказательство.} Неожиданно начнём с доказательства единственности: пусть $x_1, x_2$ --- два элемента наилучшего приближения.
Вспоминаем мантру: в гильбертовом пространстве тождество параллелограмма заменяет базис.
Рассмотрим $h \in H$, положим $a = h - x_1$, $b = h - x_2$ и запишем для них великое тождество:
\[
    \|2h - x_1 - x_2 \|^2 + \|x_1 - x_2\|^2 = 2\|a\|^2 + 2\|b\|^2.
\]
Пусть $d = \rho(h, M)$, тогда равенство переписывается в виде
\[
    4 \left\| h - \frac{x_1 + x_2}{2} \right\|^2 + \|x_1 - x_2\|^2 = 4d^2.
\]
В силу выпуклости $\frac{x_1 + x_2}{2} \in M$, поэтому расстояние от $h$ до него --- хотя бы $d$, что даёт неравенство
\[
    4d^2 + \|x_1 - x_2\|^2 \le 4d^2,
\]
то есть $\|x_1 - x_2\| = 0$.

Теперь докажем существование: пусть вновь $d = \inf_{x \in M} \|h - x\|$, тогда по определению инфимума найдётся последовательность $\{x_n\} \subset M$, такая что $\|h - x_n\| \to d$.
Докажем, что эта последовательность сходится к элементу наилучшего приближения, для этого покажем фундаментальность.
Зафиксируем $n, m$, тогда по тождество параллелограмма
\[
    \|2h - x_n - x_m\|^2 + \|x_n - x_m\|^2 = 2\|h - x_n\|^2 + 2\|h - x_m\|^2.
\]
Беря $n$ и $m$ достаточно большими, получаем $\|h - x_n\|^2, \|h - x_m\|^2 \le (d + \varepsilon)^2$ из фундаментальности.
Вместе с неравенством
\[
    \|2h - x_n - x_m\|^2 = 4 \left\| h - \frac{x_n + x_m}{2} \right\|^2 \ge 4d^2
\]
в силу того, что $\frac{x_n + x_m}{2} \in M$, это оценивает $\|x_n - x_m\|$ чем-то маленьким.

\QED

\textbf{Замечание.} (Задача 5.2) В качестве $M$ можно брать вместо подпространства просто любое непустое замкнутое выпуклое ограниченное множество.

\textbf{Теорема.} (О проекции, 4.3) Пусть $H$ --- гильбертово пространство, $M \subset H$ --- собственное подпространство.
Тогда $H$ представляется в виде $M \oplus M^\bot$.

\textbf{Доказательство.} Идея на рисунке 2.

\begin{figure}[ht]
    \centering
    \incfig{projection-theorem}{0.5\linewidth}
    \caption{Теорема о приближении}
\end{figure}

Берём произвольное $h \in H \setminus M$, тогда по лемме об элементе наилучшего приближения найдётся единственный $x \in M$, такой что $\|h - x\| = \rho(h, M)$.
Рассмотрим $y = h - x$, докажем, что $y \in M^\bot$.
Для этого возьмём произвольный $z \in M$ и $\alpha \in \mathbb R$.
Так как $x$ --- это наилучшее приближение, можно написать неравенство
\[
    d^2 = \|h - x\|^2 \le \|h - (x + \alpha z)\|^2 = d^2 - 2\alpha(h - x, z) + \alpha^2 \|z\|^2.
\]
Сокращая $d^2$, получаем
\[
    2\alpha (h - x, z) \le \alpha^2 \|z\|^2.
\]
Зафиксируем $\varepsilon > 0$, подставим $\alpha = \varepsilon$:
\[
    2(h - x, z) \le \varepsilon \|z\|^2 \to 0 + 0.
\]
Теперь подставим $\alpha = -\varepsilon$:
\[
    2(h - x, z) \ge -\varepsilon \|z\|^2 \to 0 - 0.
\]
Следовательно, $(h - x, z) = 0$, то есть мы разложили произвольный $h \in H$ на $x \in M$ и $y \in M^\bot$, что мы и хотели.

\QED

\subsection{Геометрия сепарабельного гильбертового пространства}
\epigraph{Прямая сумма пола и линейной оболочки меня --- это всё пространство.}{Сегрей Петрович}
Напоминание:

\textbf{Лемма.} (Тождество Бесселя)
\[
    \left\|x - \sum_{n=1}^{N} (x, e_n)e_n \right\|^2 = \|x\|^2 - \sum_{n=1}^{N} |(x, e_n)|^2.
\]

\textbf{Лемма.} (Минимальное свойство коэффициентов Фурье)
Для любой последовательности $\{\alpha_n\}$ выполнено
\[
    \left\| x - \sum_{n=1}^{N} \alpha_n e_n \right\|^2 \ge \left\|x - \sum_{n=1}^{N} (x, e_n) e_n \right\|^2.
\]

\textbf{Теорема.} (4.4)
Пусть $H$ --- сепарабельное гильбертово пространство, $\{e_n\}_{n=1}^\infty$ --- ортонормированная система в $H$.
Следующие утверждение эквивалентны:
\begin{enumerate}
    \item $\{e_n\}$ --- базис.
    \item $\{e_n\}$ --- полная система.
    \item Для всех $x \in H$ выполнено равенство Парсевалля: $\|x\|^2 = \sum_{n=1}^{\infty} |(x, e_n)|^2$.
    \item $\{e_n\}^\bot = \{0\}$.
\end{enumerate}

\textbf{Доказательство.} 
$1 \iff 2 \iff 3$ очевидно/было на гармоническом анализе.
Эквивалентность с четвёртым по теореме о проекции.

\QED

\textbf{Теорема.} (Рис, Фишер) Все сепарабельные гильбертовы пространства изоморфны.
Действительно, возьмём базис $\{e_n\}$.
Теперь для любой $\{\alpha_n\}$ ряд $\sum_{n=1}^{\infty} \alpha_n e_n$ сходится тогда и только тогда когда $\sum_{n=1}^{\infty} |\alpha_n|^2 < \infty$.
То есть существует биекция между $H$ и $\ell_2$.

\textbf{Доказательство.} $\Rightarrow$: рассмотрим частичные суммы $S_n$, тогда 
\[
    |S_{n+p} - S_n|^2 = \sum_{i=n+1}^{n+p} |\alpha_i|^2 \to 0.
\]
$\Leftarrow$: пусть $x = \sum_{n=1}^{\infty} \alpha_j e_n$. Тогда $(x, e_m) = \alpha_m$, так что ряд $x$ является рядом Фурье своей суммы.
По неравенству Бесселя частичные суммы ограничены.

\QED

\subsection{Линейные топологические пространства}
\epigraph{В Долгопрудном есть аудитория 415, в которой сидят студенты ФИВТа. В Тамбове есть такая же аудитория с такими же студентами.
В этом и есть смысл линейных топологических пространств --- в каждой точке происходит то же самое.}{Сергей Петрович}

\textbf{Определение.} $E$ называется \textit{линейным топологическим пространством}, если операции сложения и умножения непрерывны.

\textbf{Пример.} 
Первый пример из квантовой механики 1920--ых годов.
Возьмём $H$ --- гильбертово пространство и введём на нём \textit{слабую топологию}.
Определяется она через сходимость: $x_n$ сходится \textit{слабо} к $x$, если для любого $y \in H$ выполнено $(x_n, y) \to (x, y)$.
По теореме Фона--Неймана если $\dim(H) = \infty$, то слабая топология неметризуема.

\textbf{Пример.} Пространство $D$ бесконечно дифференцируемых функций, у которых носитель вкладывается в компакт.
Например, на прямой
\[
    \phi_a(x) =
    \begin{cases}
        \exp \left( - \frac{a^2}{a^2 - x^2} \right), & |x| < a \\
        0, & |x| \ge a
    \end{cases} .
\]
Определим на нём сходимость следующим образом: $\phi_n \to^D \phi$, если для всех $k$ выполнено $\phi_n^{(k)} \rightrightarrows \phi^{(k)}$.
Здесь тоже метрику ввести не получится.

Итог: мы получили два примера нетривиальных линейных топологических пространств, не являющихся нормированными.

\section{Линейные ограниченные (непрерывные) операторы (в ЛНП)}
\epigraph{Говоря современным языком, норма оператора --- это степень его крутизны.}{Сергей Петрович}

\textbf{Определение.} Пусть $E_1, E_2$ --- линейные нормированные пространства.
Оператор $A: E_1 \to E_2$ называется \textit{ограниченным}, если для любого ограниченного множества $M$ множество $A(M)$ ограничено.

\textbf{Утверждение.} Если $A$ --- линейный оператор, то ограниченность $A$ эквивалентна тому, что найдётся $K$, такой что для всех $x$ выполнено $\|Ax\| \le K \|x\|$.
Очевидно.

\textbf{Определение.} Для линейного оператора $A$ \textit{нормой} называется $\|A\| = \inf(K)$, где $K$ берётся из предыдущего утверждения.

\textbf{Утверждение 2.}
\[
    \|A\| = \sup_{\|x\| \le 1} \|Ax\| = \sup_{\|x\| = 1} \|Ax\| = \sup_{x \ne 0} \frac{\|Ax\|}{\|x\|}.
\]
Теперь для нахождения нормы можно использовать простой алгоритм: сначала находим оценку $\|Ax\| \le K\|x\|$, а потом строим $x$ (или последовательность $x_n$), на котором достигается равенство.
И ничего лучше математики так и не придумали.

\textbf{Утверждение 3.} Пусть $A$ --- линейный оператор.
Если $A$ непрерывен в какой-то точке $x_0$, то он непрерывен в всех точках.
Очевидно по линейности.

Будем обозначать через $\mathcal L(E_1, E_2)$ пространство ограниченных линейных операторов $E_1 \to E_2$.
В зарубежной литературе обычно обозначается через $B(E_1, E_2)$.

\textbf{Теорема.} (5.1) Если $E_1, E_2$ --- нормированные пространства (линейность здесь и дале опускается), $A: E_1 \to E_2$ --- линейный оператор.
Его ограниченность эквивалентна непрерывности.

\textbf{Доказательство.} $\Rightarrow$. Пусть $x_n \to x$.
Тогда
\[
    \|Ax_n - Ax\| = \|A(x_n - x)\| \le \|A\| \cdot \|x_n - x\| \to 0.
\]

$\Leftarrow$. В эту сторону сложнее, так как приходится от противного.
Пусть для любого $n \in \mathbb N$ найдётся $x_n$, такое что $\|A x_n \| \ge n \|x_n\|$.
Положим $y_n = \frac{x_n}{\|x_n\|} \cdot \frac{1}{n} \to 0$.
Тогда
\[
    \|A y_n\| = \frac{\|A x_n\|}{\|x_n\|} \cdot \frac{1}{n} > 1.
\]
Противоречие с непрерывностью.

\QED

\textbf{Теорема.} (5.2) Пусть $E_1, E_2$ --- нормированные пространства.
Тогда $\mathcal L(E_1, E_2)$ --- это линейное нормированное пространство с операторной нормой.
Более того, если $E_2$ является банаховым, то $\mathcal L(E_1, E_2)$ тоже банахово.

\textbf{Доказательство.} Докажем неравенство треугольника: пусть $A_1, A_2$ --- два линейных оператора.
Тогда
\[
    \sup_x \|(A_1 + A_2) x \| \le \sup_x \|A_1 x\| + \sup_x \|A_2 x\|,
\]
что доказывает первую часть.

Пусть $\{A_n\}$ --- фундаментальная последовательность в $\mathcal L(E_1, E_2)$, то есть для любого $\varepsilon > 0$ найдётся $N \in \mathbb N$, такое что для $n, m \ge N$ выполнено $\|A_n - A_m\| < \varepsilon$.
Рассмотрим $x \in E_1$, рассмотрим последовательность $\{A_n x\}$:
\[
    \|A_n x - A_m x\| \le \|A_n - A_m \| \cdot \|x\| < \varepsilon \|x\|,
\]
то есть $\{A_n x\}$ --- фундаментальная последовательность в $E_2$.
Так как оно полное, у неё существует предел $\lim_{n \to \infty}(A_n x) = Ax$.

Очевидным образом $A$ является линейным оператором, теперь остаётся доказать ограниченность.
Но это выводится из фундаментальности: найдётся $M > 0$, такое что все $\|A_n\| \le M$, то есть для всех $x \in E_1$ выполнено $\|Ax\| \le M\|x\|$.

Но мы пока лишь доказали поточечную сходимость $A_n$, докажем, что $\|A_n - A\| \to 0$.
Берём произвольный $x \in \overline B(\ominus, 1)$.
Из доказанного $\|A_n x - A_m x\| \le \varepsilon \|x\|$.
Устремляем $m$ в бесконечность, получаем $\|A_m - Ax\| \le \varepsilon$.
Беря супремум по всем $x$, получаем $\|A_m - A\| \le \varepsilon$.

\QED

\textbf{Следствие.} Если $E$ банахово, то $\mathcal L(E)$ банахово.
А также $E^* = \mathcal L(E_1, \mathbb R (\mathbb C))$ банахово.

\textbf{Теорема.} (5.3) Пусть $E_1$ --- нормированное пространство, $E_2$ банахово.
Пусть $D(A) \subset E_1$ --- линейное многообразие, $\overline{D(A)} = E_1$, $A$ --- линейный оператор, отображающий $D(A)$ в $E_2$ ($D(A)$ --- это просто обозначение области определения оператора $A$).

Тогда существует единственный $\widetilde A \in \mathcal L(E_1, E_2)$, такой что $\widetilde A|_{D(A)} \equiv A$ и $\|\widetilde A\| = \|A\|$.

\textbf{Доказательство.} Докажем единственность: пусть $\widehat A$, $\widetilde A$ --- два кандидата.
Рассмотим $x \in E_1$, тогда найдётся $x_n \to x$, $x_n \in D(A)$, такое что $\lim_{n \to \infty}(Ax_n) \to \widehat Ax = \widetilde Ax$.
Следовательно, они всюду совпадают.

Докажем существование.
Аналогично возьмём $x \in E_1$ и положим $\widetilde Ax = \lim_{n \to \infty}(A x_n)$.
Так как $\{x_n\}$ фундаментальна, $\{A x_n\}$ тоже фундаментальна, значит, у неё есть предел.
Ещё для корректности нужно доказать, что определение не зависит от выбора последвательности.
Действительно, если $x'_n \to x$, то
\[
    \|Ax_n - Ax'_n\| \le \|A\| \cdot \|x_n - x'_n\| \to 0,
\]
так как $x_n - x'_n \to 0$.

Проверим, что $\widetilde A$ линеен --- очевидно.
Что ограничен --- 
\[
    \|\widetilde A x\| = \lim_{n \to \infty} \|Ax_n\| \le \lim_{n \to \infty} \left( \|A\| \cdot \|x_n\| \right) = \|A\| \cdot \|x\|.
\]
Сохранение нормы относительно очевидно.

\QED

\textbf{Теорема.} (Банаха--Штейнгауза, 1927, 5.4)
Пусть $E_1$ --- банахово пространство, $E_2$ --- нормированное пространство, $A_n \in \mathcal L(E_1, E_2)$.
Если для всех $x \in E_1$ выполнено $\sup_n \|A_n x\| < \infty$, то $\sup_n\|A_n\| < \infty$.

Эквивалентная формулировка --- если $\sup_n \|A_n\| = \infty$, то найдётся $x \in E_1$, такой что $\sup_n\|A_n x\| = \infty$.

Иными словами, если существует крыша $M(x) = \sup_n\|A_n x\|$, то можно её сделать конической: $M'(x) = M \|x\|$.

\textbf{Доказательство.} Будем доказывать вторую формулировку, то есть найдём плохой элемент $x$.

Первый шаг, также известный, как принцип равномерной ограниченности.
Если $\{\|A_n x\|\}$ равномерно ограничена на некотором шаре, то и $\{\|A_n\|\}$ ограничена.
Итак, пусть у нас есть шар $B(x_0, r)$, и мы знаем, что для всех $n$
\[
    \sup_{x \in \overline B}\|A_n x\| \le M.
\]
Рассмотрим $x \in E_1$, положим $y - x_0 = r \cdot \frac{x}{\|x\|}$.
Тогда
\[
    x = \frac{\|x\|}{r} (y - x_0) \Rightarrow \|A_n x\| \le \frac{\|x\|}{2} (\|A_n y\| + \|A_n x_0\|) \le \frac{\|x\|}{2} \cdot 2M,
\]
что доказывает ограниченность $\{\|A_n\|\}$.

Второй шаг --- непосредственно найдём плохой $x$.
Возьмём шарик $\overline B_0(x_0, r_0)$.
По лемме найдётся $x_1 \in \overline B(x_0, \frac{r_0}{2})$, и номер $n_1$, такие что $\|A_{n_1} x_1\| > 1 \cdot \|x_1\|$.
В силу непрерывности $A_{n_1}$ найдётся окрестность $x_1$, в которой неравенство выполняется, обозначим её за $\overline B_1(x_1, r_1)$.
И так далее.

На шаре $\overline B_k(x_k, r_k)$ выполнено $\|A_{n_k} x\| > k \|x\|$, причём $r_k < \frac{1}{2} r_{k-1}$.
Как следствие, по теореме 2.1 существует единственная точка $x$, лежащая в пересечении, то есть для всех $k$ выполнено $\|A_{n_k} x\| > k \cdot \|x\|$.
Следовательно,
\[
    \sup_n \|A_n x\| \ge \sup_k \|A_{n_k} x\| = \infty.
\]

\QED

\textbf{Замечание.} Эту теорему можно проще доказать по теореме Бэра, но теорема Бэра --- это зло с доказательством от противного.

А ещё для запоминания формулировки можно заметить, что $E_1$ --- банахово, и Банах стоит на первом месте в названии.

Философия теоремы заключается в том, что линейный оператор над $E_1$ характеризуется элементами этого пространства, что бывает очень удобно во многих ситуациях.

\textbf{Теорема.} (5.5) Пусть $E_1, E_2$ --- банаховы пространства, $A_n \in \mathcal L(E_1, E_2)$, а также для всех $x \in E_1$ последовательность $\{A_n x\}$ фундаментальна.

Тогда существует $A \in \mathcal L(E_1, E_2)$, такой что $A_n x \to Ax$ для всех $x$ поточечно.

\textbf{Доказательство.} Положим $Ax = \lim_{n \to \infty}(A_n x)$, определение корректно, так как $\{A_n x\}$ фундаментальна, а пространство банахово.
Линейность очевидна, проверим ограниченность.
Так как $\{A_n x\}$ сходится, она ограничена, откуда по теореме Банаха--Штейнгауза $\sup_n \|A_n\| = M < \infty$.
Итак, для всех $x$ имеем $\|A_n x \| \le M \|x\|$ --- устремляя $n$ в бесконечность, получаем $\|A x\| \le M \|x\|$.

\QED

\textbf{Теорема.} (Критерий поточечной сходимости оператора из $\mathcal L(E_1, E_2)$, 5.6)
Пусть $E_1$ --- банахово пространство, $E_2$ --- нормированное пространство, $A_n, A \in \mathcal L(E_1, E_2)$.
$A_n \to A$ поточечно тогда и только тогда, когда $\{\|A_n\|\}$ ограничена и $A_n y \to Ay$ для всех $y$ из пространства $Y$, такого что $\overline{[Y]} = E_1$.

\textbf{Доказательство.} $\Rightarrow$ первое условие следует из теоремы Банаха--Штейнгауза, второе очевидно.

$\Leftarrow$. Зафиксируем $M$, такое что для всех $n$ выполнено $\|A_n\| \le M$.
Возьмём $x \in E_1$, $z \in [Y]$, такое что $\|x - z\| < \varepsilon$.
Теперь
\[
    \|A_n x - Ax \| \le \|A_n x - A_n z\| + \|A_n z - Az \| + \|Az - Ax\| < 3 \varepsilon.
\]
Все конкретные оценки, думаю, и так понятны.

\QED

\subsection{Следствия теоремы Банаха--Штейнгауза}
\epigraph{Теорема Банаха-Штейнгауза говорит, что прямо ехать --- живым не бывать, или, говоря на математическом языке, существует непрерывная периодическая функция, у которой ряд Фурье сходится неравномерно.}{Сегрей Петрович}

\textbf{Теорема.} (Ландау о резонансе) Пусть $x \in l_2$.
Тогда для любого $y \in l_2$ ряд $\sum_{n=1}^{\infty} |x_n y_n|$ сходится.

\textbf{Теорема.} Существует $f \in CP[-\pi, \pi]$, такая что ряд Фурье сходится неравномерно.

\textbf{Лемма.} Пусть
\[
    (Af)(x) = \int_{a}^b K(x, t) f(t) \dif t,
\]
где $K$ --- непрерывная на $[a, b]^2$ функция.
Тогда
\[
    \|A\| = \max_x \int_a^b |K(x, t)| \dif t.
\]

\textbf{Доказательство.} Проверим определение нормы
\[
    \|(Af)(x)\| \le \max_x \int_a^b |K(x, t)| \cdot |f(t)| \dif t \le \|f\| \max_x \int_a^b |K(x, t)| \dif t.
\]
Докажем неулучшаемость.
Пусть $x_0$ --- точка, в которой достигается максимум интеграла.
Теперь хотелось бы взять $f(t) \sign(K(x_0, t))$ и закончить доказательство, но эта функция разрывна.
Классическим образом это решается взятием $f_{\varepsilon}(\cdot)$ --- непрерывных приближений сигнума.

\QED

Теперь применим эту лемму к $S$ --- ряду Фурье, где $D_n$ --- $n$--ое ядро Дирихле.
\[
    \|S\| = \frac{1}{\pi} \max_x \int_{-\pi}^\pi |D_n(x - t)| \dif t \ge \frac{2}{\pi} \int_0^\pi |D_n(t)| \dif t =,
\]
при $x = 0$,
\[
    = \frac{2}{\pi} \int_0^\pi \left| \frac{\sin(n + \frac{1}{2})t}{2\sin(\frac{t}{2})} \right| \dif t \ge \frac{1}{\pi} \int_0^\pi \frac{|\sin(n + \frac{1}{2})| t}{t/2} \dif t =
\]
(Делая замену $s = (n + \frac{1}{2}) t$)
\[
    = \frac{2}{\pi} \int_0^{(n + \frac{1}{2}) \pi} \frac{\sin(s)}{s} \dif s = \infty.
\]

Итак, норма оператора суммы ряда Фурье бесконечна, значит, все не могут сходиться равномерно.

\section{Сопряжённое пространство, теорема Рисса--Фреше, Хана--Банаха, обобщённые функции}
\epigraph{Когда ты в джунглях встречаешь другого человека, что нужно ему говорить?
--- ``Мы с тобой одной крови``.
}{Сегрей Петрович}

Первый вопрос параграфа: всегда ли существуют нетривиальные линейные непрерывные операторы?
Иными словами, если $E$ --- линейное топологическое пространство, то правда ли что $E^*$ нетривиально?

\textbf{Теорема.} (6.1, Рисса--Фреше) Пусть $H$ --- гильбертово пространство, $f \in H^*$, тогда существует единственный $y_0 \in H$, такой что $f(x) = (x,y_0)$.
Как следствие, $\|f\|_{H^*} = \|y_0\|_H$.

\textbf{Замечание.} Теорема даёт ``почти изоморфизм`` между $H$ и $H^*$.
Это не полноценный изоморфизм, так как в случае, когда пространство построено над полем комплексных чисел, умножение не работает из-за сопряжений.

\textbf{Доказательство.} (В общем случае)
Докажем существование: если $f = 0$, то можно взять $y_0 = \ominus$ и закончить.
Иначе мы знаем, что $\ker(f) \ne H$, значит, по теореме 4.3 о проекции $\ker(f) \oplus (\ker(f))^\bot = H$, докажем, что размерность второго пространства равна единице.
Для этого рассмотрим $x_0 \in (\ker(f))^\bot$ и покажем, что для любого $x \in H$ существует разложение $x = z + \alpha x_0$, где $z \in \ker(f)$.
Действительно, $x - \alpha x_0 \in \ker(f)$, значит, $f(x) - \alpha \cdot f(x_0) = 0$, откуда $\alpha = \frac{f(x)}{f(x_0)}$.

Теперь умножим скалярно на $x_0$ тождество $x = z + \frac{f(x)}{f(x_0)} \cdot x_0$: получается $(x, x_0) = 0 + \frac{f(x)}{f(x_0)} \|x_0\|^2$.
Таким образом,
\[
    f(x) = \left(x, \frac{\overline{f(x_0)}}{\|x_0\|^2} x_0 \right).
\]

Докажем единственность: пусть $f(x) = (x, y_0') = (x, y_0'')$.
Значит, $(x, y_0' - y_0'') \equiv 0$.
Подставим $x = y_0' - y_0''$, получаем $\|y_0' - y_0''\| = 0$.

\QED

\textbf{Доказательство.} (В предположении, что $H$ сепарабельно)
Теперь мы знаем, что в $H$ есть ОНБ $\{e_n\}_{n=1}^{\infty}$.
Возьмём $x \in H$, тогда он раскладывается в ряд Фурье $x = \sum_{n=1}^{\infty} (x, e_n) e_n$.
Тогда
\[
    f(x_n) = \sum_{n=1}^{\infty} (x, e_n) f(e_n) = \sum_{n=1}^{\infty} (x, \overline{f(e_n)} e_n) = \left(x, \sum_{n=1}^{\infty} \overline{f(e_n)}, e_n \right).
\]
Но сходится ли сумма во втором аргументе скалярного произведения?
По условию сходимости ряда нужно показать, что $\sum_{n=1}^{\infty} |\overline f(e_n)|^2 < \infty$.
Элементы ряда --- это неотрицательные числа, засим нужно показать ограниченность частичных сумм $S_N = \sum_{n=1}^{N}|\overline f(e_n)|^2$.
Делается арифметикой:
\[
    S_N = \sum_{n=1}^{N} |\overline f(e_n)|^2 = \sum_{n=1}^{N} f(e_n) \overline{f(e_n)} = f \left( \sum_{n=1}^{N} \overline{f(e_n)} e_n \right) \le 
\]
\[
    \le \|f\| \cdot \left\| \sum_{n=1}^{N} \overline{f(e_n)} e_n \right\| \le \|f\| \sqrt{ \sum_{n=1}^{N} |\overline f(e_n)|^2 } = \|f\| \sqrt{S_N}.
\]
Делим на $\sqrt{S_N}$, получаем $\sqrt{S_N} \le \|f\|$.

\QED

\textbf{Теорема.} (6.2, Хана--Банаха) Пусть $E$ --- нормированное пространство над $\mathbb K \in \{\mathbb R, \mathbb C\}$, $M \subset E$ --- линейное многообразие, $f$ --- линейный ограниченный функционал, определённый на $M$.
Тогда существует функционал $\widetilde f \in E^*$, такой что $\widetilde f|_M = f$ и $\|\widetilde f\| = \|f\|$.

\textbf{Доказательство.} (В предположении, что $E$ вещественное и сепарабельное)
Пусть $M_0 = M$, $f_0 = f$.
Если $M_0 \ne E$, то найдётся $x_0 \not\in M_0$, рассмотрим $M_1 = M_0 \oplus [x_0]$ --- продолжим $f_0$ на него.
Пусть $y = x + \alpha \cdot x_0 \in M_1$, тогда $f_1(y) = f_1(x) + \alpha f_1(x_0) = f_0(x) + \alpha \cdot a$.

Теперь нужно подобрать $a$ так, чтобы $\|f_1\| \le \|f\|$ (обратное неравенство выполняется автоматически).
Иными словами, хотим $|f_1(y)| \le \|f\| \cdot \|y\|$.
Или же
\[
    |f(x) + \alpha a| \le \|f\| \cdot \|x + \alpha x_0\|.
\]
Если $\alpha = 0$, то всё понятно.
Иначе разделим:
\[
    \left| f \left(\frac{x}{\alpha} \right) + a \right| \le \|f\| \cdot \left\| \frac{x}{\alpha} + x_0 \right\|.
\]
Положим $z = \frac{x}{\alpha}$.
Мы знаем, что
\[
    -\|f\| \cdot \|z + x_0\| \le f(z) + a \le \|f\| \cdot \|z + x_0\|.
\]
Вычтем $f(z)$:
\[
    -f(z) -\|f\| \cdot \|z + x_0\| \le a \le -f(z) + \|f\| \cdot \|z + x_0\|.
\]
Остаётся показать, что найдётся $a$, такой что данное неравенство выполняется для всех $z$.
Для этого покажем, что супремум левой части не превосходит инфимум правой: берём $z_1$, $z_2$, тогда
\[
    -f(z_1) -\|f\| \cdot \|z_1 + x_0\| \le -f(z_2) + \|f\| \cdot \|z_2 + x_0\| \iff
\]
\[
    \iff (f(z_2) - f(z_1)) \le \|f\| \left( \|z_1 - x_0\| + \|z_2 + x_0\| \right).
\]
Докажем более сильное утверждение, когда слева стоит модуль.
\[
    |f(z_2 - z_1)| \le \|f\| \cdot (\|z_2 - z_1\|) = \|f\| \cdot \|z_2 - z_2 + x_0 - x_0\|.
\]
Остаётся применить неравенство треугольника, что даст искомое неравенство.

Пусть $X = \{x_n\}_{n=0}^\infty$ --- всюду плотное множество в $E$, то есть $\overline X = E$.
Будем брать $x_j$ и строить по нему $M_j$ процедурой выше.
По итогу мы получим последовательность $M_0, M_1, M_2, \dots$.
Положим $M_\infty = \bigcup_{k=0}^\infty = M_k$ --- всюду плотное линейное многообразие, $f_\infty$ --- объединение всех функций (так можно, ибо они согласованы)
По теореме 5.3 функцию $f_\infty$ можно доопределить единственным образом до всего $E$.

\QED

\textbf{Замечание.} Доказательство без сепарабельности требует трансфинитную индукцию, а в комплексном случае просто дополнительную арифметику.

\subsection{Следствия из теоремы Хана--Банаха}
\epigraph{Комплекснозначное зверьё --- это круто.}{Сергей Петрович}

Во всех следствиях одна прелюдия: пусть $E$ --- нормированное пространство, $M \subset E$ --- линейное многообразие.
Схема доказательства --- $1 \Rightarrow 2 \Rightarrow \{3, 4\}$.

\textbf{Следствие 1.} Если $\overline M \ne E$, $x_0 \not \in \overline M$, то найдётся $f \in E^*$, такой что
\begin{itemize}
    \item $M \subset \ker(f)$.
    \item $f(x_0) = 1$.
    \item $\|f\| = \frac{1}{d}$, где $d = \rho(x_0, M)$.
\end{itemize}

\textbf{Доказательство.} Положим $M_1 = M \oplus [x_0]$, определим на нём $f_1(y + \alpha x_0) = 0 + \alpha$, где $y \in M$, тогда он равен нулю на $M$, и $f(x_0) = 1$.

Остаётся доказать, что $\|f_1\| = \frac{1}{d}$, после чего теорема Хана--Банаха завершит доказательство.
\[
    \|f_1\| = \sup_{\|y + \alpha x_0\| = 1} |f(y + \alpha x_0)| = \sup_{\|y + \alpha x_0\| \ne 0} \left( \frac{|\alpha|}{\|y + \alpha x_0\|} \right) =
\]
\[
    = \sup_{\|\frac{y}{\alpha} + x_0\| \ne 0} \left( \frac{1}{\|\frac{y}{\alpha} + x_0\|} \right) = \frac{1}{\inf \|\frac{y}{\alpha} + x_0\|} = \frac{1}{d},
\]
так как $\frac{y}{\alpha}$ пробегает все элементы $M$.

\QED

\textbf{Следствие 2.} Если $x \ne 0$, то найдётся $f \in E^*$, такой что $\|f\| = 1$ и $f(x) = \|x\|$.

\textbf{Доказательство.} Положим $M = \{0\}$, $x_0 = \frac{x}{\|x\|}$.
По следствию 1 найдётся $f \in E^*$, такой что $f|_M = 0$ и $f(x_0) = 1 = \frac{f(x)}{\|x\|}$.

\QED

\textbf{Следствие 3.} Если $f(x) = f(y)$ для всех $f \in E^*$, то $x = y$.

\textbf{Доказательство.} Положим $z = x - y$, тогда для всех $f \in E^*$ выполнено $f(z) = 0$.
От противного: допустим, что $z \ne 0$, тогда по следствию 2 найдётся $f \in E^*$, такой что $\|f\| = 1$ и $f(z) = \|z\| \ne 0$.
Но по условию $f(z) = 0$ --- противоречие.

\QED

\textbf{Следствие 4.}
Для всех $x$
\[
    \|x\| = \sup_{\|f\|_{E^*} = 1} |f(x)|.
\]

\textbf{Доказательство.} Очевидно: оценка снизу по следствию 2, сверху --- по определению нормы $f$.

\QED

Зачем они нужны: следствие 1 --- это просто промежуточный шаг к остальным следствиям.
Следствие 2 говорит, что если пространство $E$ нетривиально, то и $E^*$ нетривиально.
Более того, если пространство над полем $\mathbb R$, то оно позволяет построить гиперплоскость, являющуюся касательной к сфере: достаточно просто взять точку $x_0$ на ней и применить следствие, гиперплоскосью будет $\{x \in E~|~ f(x) = f(x_0)\}$.

Следствие 3 говорит, что всегда можно найти функционал, различающий два неравных элемента.
И, наконец, следствие 4 гласит, что элементу $x \in E$ можно сопоставить $F_x: E^* \to \mathbb R(\mathbb C)$, $F_x(f) = f(x)$, и отображение $\pi: x \mapsto F_x$ является изометрией.

\section{Обобщённые функции (распределения)}
\epigraph{
Проблема борелевской меры --- вот ты берёшь кусок колбасы, взвешиваешь его, получаешь 0 грамм.
Потом отрезаешь от него кусочек, взвешиваешь, а весы взрываются.
}{Сергей Петрович}

Глобально мы берём линейное топологическое пространство $E$, называемое \textit{пробными} (основными) функциями, и рассматриваем линейные непрерывные функционалы над ними $E^*$.

\textbf{Определение.} \textit{Обобщённые функции} --- это пространство $D$ функций из $C_0^\infty(\mathbb R)$ в $\mathbb R$.

\textbf{Определение.} Последовательность функций $\phi_n$ \textit{сходится} к $\phi$ в $D$, если существует компакт $K$, такой что все носители $\phi_n$ в нём лежат, и для любого $k$ производные $\phi_n^{(k)}$ сходятся равномерно к $\phi^{(k)}$ в $K$.

\textbf{Утверждение.} $C_0^\infty$ непусто.
Пример функции --- пенёк или шапочка --- имеет вид
\[
    \omega(x, a) =
    \begin{cases}
        \exp \left( - \frac{a^2}{a^2 - x^2} \right), & |x| < a \\
        0, & x \ge a
    \end{cases} .
\]

Явно опишем топологию в $D$.
Зафиксируем $m \in \mathbb Z_{\ge 0}$ и набор непрерывных положительных функций $\gamma_0(\cdot), \dots, \gamma_m(\cdot)$.
Тогда окрестностью нуля будут всевозможные множества
\[
    \left\{ \phi \in C_0^\infty~\bigg|~|\phi^{(k)}(x)| < \gamma_k(x), k = 0, 1, \dots, m \right\}.
\]
Выглядит страшно, засим мы не будем про это думать, а вместо этого остановимся на сходимости.

\textbf{Определение.} \textit{Регулярные функции} --- функции $f$, для которых сходится интеграл
\[
    (f, \phi) = \int_{-\infty}^{+\infty} f(x) \phi(x) \dif x,
\]
для всех $\phi \in C_0^\infty$.
В противном случае функция называется \textit{сингулярной}.

\textbf{Определение.} \textit{Дельта--функцией} называется $\delta \in D$, такая что $(\delta, \phi) = \phi(0)$.

\textbf{Утверждение.} Дельта--функция не является регулярной.

\textbf{Доказательство.} Возьмём
\[
    \omega(x, a) =
    \begin{cases}
        \exp \left( -\frac{a^2}{a^2 - x^2} \right), & |x| < a \\
        0, & |x| \ge a
    \end{cases} .
\]
Тогда при $a \to 0$ имеем
\[
    \int_{-\infty}^{+\infty} \delta(x) \omega(x, a) \dif x \to 0
\]
по теореме Лебега,
но $\omega(0, a) = \frac{1}{e}$.


\subsection{Действия над обобщёными функциями}
\epigraph{Ры --- это рысь, Ра --- это бог солнца.}{Сегрей Петрович}

Во-первых, можно умножать на бесконечно дифференцируемую функцию: для регулярной функции $f$
\[
    (\psi f, \phi) = \int \psi f \phi \dif x = (f, \psi \phi).
\]
Во-вторых, можно рассматривать поточечную сходимость в $D'$: $f_n \to f$, если для любой $\phi \in D$ выполнено $(f_n, \phi) \to (f, \phi)$.

\textbf{Задача.} $D'$ полно относительно поточечной сходимости.

В-третьих, свёртка:
\[
    (f * g)(x) = \int f(x) g(x - t) \dif t.
\]
В частности, $f * \delta = f$.

Раньше у нас была проблема, что мы берём нормальную функцию, а у неё нет производной.
Раньше математики решали это тем, что брали $N$ раз дифференцируемые функции, а теперь обобщённые функции позволяют дифференцировать любые функции.

В-четвёртых, производная:
\[
    (f', \phi) = \int f'(x) \phi(x) \dif x = (f\phi)|_{-\infty}^{+\infty} -\int f \phi' \dif x = -(f, \phi').
\]
Отсюда мы и получаем классическое определение.

\textbf{Пример.} Решим обыкновенное дифференциальное уравнение $y' = 0$.
Первокурсник бы здесь использовал теорему Лагранжа.
Второкурсник скажет, что это очевидно константа.
А мы воспользуемся задачей из задавальника: если $E$ --- линейное пространство, $f \ne 0$ --- линейный функционал, то коразмерность ядра $f$ равна единице.

Итак, имеем уравнение $(y', \phi) = 0$ для всех $\phi$.
Это то же самое, что $(y, \phi') = 0$.
Иными словами, $(y, \psi) = 0$ для всех $\psi = - \phi'$.
А какие функции подходят в качестве $\psi$?
Просто интеграл
\[
    \int \psi \dif x = \int \phi' \dif x = \phi_{-\infty}^{+\infty} = 0.
\]
Пусть $K$ --- пространство таких функций $\psi$, тогда мы сейчас доказали, что для всех $\psi \in K$ интеграл равен нулю.
Более того, это условие является достаточным: если $\psi \in K$, то можно рассмотреть функцию
\[
    \phi(x) = \int_{-\infty}^x \psi(t) \dif t.
\]
Из геометрических соображений можно понять, что эта функция тоже финитна.

Итак, мы нашли, чему равно $K$, теперь представим это в более приятном виде.
Заметим, что $K = \ker(\kappa)$, где
\[
    \kappa(f) = \int_{-\infty}^{+\infty} f(x) \dif x.
\]
По задаче любой функционал $\phi$ представим в виде $\phi = \phi_1 + \alpha \phi_0$, где $\phi_1 \in K$, $\phi_0 \not\in K$.
Возьмём $\phi_0$ так, чтобы $\kappa(\phi_0) = 1$.
Пусть $(y, \phi_0) = C$, тогда $(y, \phi) = 0 + \alpha (y, \phi_0) = \alpha C$.
А это то же самое, что и $(\alpha C, \phi_0)$ или же $(C, \phi)$.

\textbf{Утверждение.} Топология в $D$ неметризеума.

\textbf{Доказательство.} В $D$ мы знаем лишь одного представителя --- шапочку $\omega(x, a)$.
Рассмотрим функции $\frac{1}{n} \omega(x, m)$.
Для каждого $m$ последовательность функций $\frac{1}{n} \omega(x, m)$ сходится к нулю.

Допустим, что нашлась метрика.
Посмотрим на это, как на таблицу $m \times n$ для $m, n \to \infty$.
Эта метрика нам позволяет взять $\frac{1}{n_1} \omega(x, 1)$, такую что $\|\frac{1}{n_1} \omega(x, 1)\| < \frac{1}{1}$, потом $\frac{1}{n_2} \omega(1, 2)$, такую что $\|\frac{1}{n_2} \omega(x, 2)\| < \frac{1}{2}$, и так далее.

Следовательно, эта последовательность сходится к нулю, то есть у неё должен найтись общий носитель, но его нет.

\QED
