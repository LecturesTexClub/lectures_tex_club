\section{Основные определения}

\begin{note}
	Далее $(\Omega, \F)$ и $(E, \cE)$ --- измеримые пространства.
\end{note}

\begin{definition}
	Отображение $\xi \colon \Omega \to E$ называется \textit{случайным элементом}, если оно $\F/\cE$-измеримо:
	\[
		\forall B \in \cE\ \ \xi^{-1}(B) \in \F
	\]
\end{definition}

\begin{definition}
	Случайный элемент $\xi$ называется \textit{случайным вектором}, если $(E, \cE) = (\R^m, \B(\R^m))$.
\end{definition}

\begin{definition}
	Случайный вектор $\xi$ называется \textit{случайной величиной}, если $(E, \cE) = (\R, \B(\R))$.
\end{definition}

\begin{note}
	Далее мы считаем, что $(\Omega, \F, P)$ --- вероятностное пространство
\end{note}

\begin{definition}
	Пусть $\xi$ --- случайный элемент. Тогда его \textit{распределением} называется мера $P_\xi$ на $(E, \cE)$, заданная следующим образом:
	\[
		\forall B \in \cE\ \ P_\xi(B) := P(\xi \in B)
	\]
\end{definition}

\begin{note}
	Статистика, в отличие от теории вероятностей, напрямую связана с проводимыми экспериментами и их результатами. Так, пусть наблюдается некоторый эксперимент, который можно описать $m$-мерным случайным вектором с распределением $P$. Мы должны построить вероятностное пространство и измеримый на нём случайный элемент, соответствующий этому вектору. 
\end{note}

\begin{definition}
	Рассмотрим вероятностное пространство $(\cX, \B(\cX), P)$, определённое следующим образом:
	\begin{itemize}
		\item $\cX$ --- выборочное множество всевозможных исходов одного эксперимента (если рассматриваем $m$-мерный вектор, то можно взять $\R^m$). Требуем, что $\cX$ является топологическим пространством.
		
		\item $\B(\cX)$ --- борелевская $\sigma$-алгебра подмножеств $\cX$
		
		\item $P$ --- мера, соответствующая распределению исходов одного эксперимента
	\end{itemize}
	Такое пространство называется \textit{вероятностно-статистической моделью}.
\end{definition}

\begin{note}
	В самом деле, если мы рассмотрим тождественный случайный элемент $\chi$ в таком пространстве, то $P_\chi = P$. Таким образом, этот случайный элемент соответствует проводимому эксперименту.
\end{note}

\begin{definition} \textcolor{red}{(Не по лектору)}
	Любой случайный элемент в вероятностно-статистической модели, действующий в $\cX$, называется \textit{наблюдением}.
\end{definition}

\begin{note}
	Покажем, как построить математическую модель для $n$ независимых повторений одного и того же эксперимента:
	\begin{itemize}
		\item $\cX^n = \underbrace{\cX \times \ldots \times \cX}_n$
		
		\item $\B^n(\cX) = \B(\cX^n) = \sigma(\{B_1 \times \ldots \times B_n, B_i \in \B(\cX)\})$
		
		\item $P^n = \underbrace{P \otimes \ldots \otimes P}_n$ --- тензорное произведение $n$ мер
	\end{itemize}
	Несложно показать, что $P^n$ --- действительно вероятностная мера на $(\cX^n, \B(\cX^n))$.
\end{note}

\begin{proposition}
	Рассмотрим пространство $(\cX^n, \B(\cX^n), P^n)$ и тождественный случайный вектор $\chi \colon \cX^n \to \cX^n$. Тогда любая компонента этого вектора будет случайной величиной того же пространства, причём всем компоненты независимы в совокупности и имеют распределение $P$.
\end{proposition}

\begin{proof}
	Доказывать измеримость компоненты мы не будем, это тривиально из теории вероятностей. Распишем распределение $\chi_i$ по определению:
	\begin{multline*}
		\forall B_i \in \B(\cX)\ \ P^n(\chi_i \in B_i) = P^n\{(x_1, \ldots, x_n) \colon x_i \in B_i\} =
		\\
		P^n(\cX \times \ldots \times \cX \times B_i \times X \times \ldots \times \cX) = P(B_i)
	\end{multline*}
	Независимость тоже проверяем через эквивалентное определение:
	\begin{multline*}
		\forall B_i \in \B(\cX)\ \ P^n(\chi_1 \in B_1 \wedge \ldots \wedge \chi_n \in B_n) =
		\\
		P^n\{(x_1, \ldots, x_n) \colon x_1 \in B_1 \wedge \ldots \wedge x_n \in B_n\} =
		\\
		P^n(B_1 \times \ldots \times B_n) = P(B_1) \cdot \ldots \cdot P(B_n) = P^n(\chi_1 \in B_1) \cdot \ldots \cdot P^n(\chi_n \in B_n)
	\end{multline*}
\end{proof}

\begin{definition}
	Совокупность $X = (X_1, \ldots, X_n)$ независимых одинаково распределённых наблюдений с распределением $P$ называется \textit{выборкой размера $n$ из распределения $P$}. Число $n$ также называют \textit{объёмом выборки}.
\end{definition}

\begin{note}
	Для счётного числа экспериментов модель строится аналогичным образом. Там уже происходит работа с вероятностным пространством $(\cX^\infty, \B(\cX^\infty), P^\infty)$. Тот факт, что $P^\infty$ существует и согласовано со всеми возникающими $P^n$, гарантируется теоремой Колмогорова.
\end{note}

\begin{note}
	Всюду далее для простоты опускаются индексы пространств $(\cX^n, \B(\cX^n), P^n)$ и $(\cX^\infty, \B(\cX^\infty), P^\infty)$, пишем просто $(\cX, \B(\cX), P)$ и фиксируем это обозначение для \\ вероятностно-статистической модели
\end{note}

\begin{definition}
	Значение выборки на конкретном исходе называется \textit{реализацией выборки}.
\end{definition}

\begin{note}
	Основная задача статистики --- это сделать вывод о неизвестном распределении выборки по её реализации.
\end{note}

\begin{definition}
	Пусть $X_1, \ldots, X_n$ --- выборка из распределения $P_X$ на пространстве $(\R^m, \B(\R^m))$, а также $B \in \B(\R^m)$. Тогда $P_n^*(B) = \frac{1}{n}\ps{\sum_{i = 1}^n \chi\{X_i \in B\}}$ называется \textit{эмпирическим распределением, построенным по выборке $X_1, \ldots, X_n$}.
\end{definition}

\begin{note}
	Внимательный читатель заметит, что это эмпирическое распределение является \textit{случайным}, то есть неявно зависит от $\omega \in \Omega$. Более того, это распределение является \textit{случайной величиной при фиксированном $B \in \B(\R^m$)}, ну а при фиксированном $\omega \in \Omega$ это действительно является распределением (вероятностной мерой на $(\R^m, \B(\R^m))$).
\end{note}

\begin{proposition}
	Пусть $X_1, \ldots, X_n$ --- выборка на вероятностном пространстве \\ $(\Omega, \F, P)$ из распределения $P_X$ на пространстве $(\R^m, \B(\R^m))$. Имеет место сходимость:
	\[
		\forall B \in \B(\R^m)\ \ P_n^*(B) \xrightarrow[n \to \infty]{P\text{ п.н.}} P_X(B)
	\]
\end{proposition}

\begin{proof}
	Доказательство опирается на УЗБЧ. Действительно, раз $\{X_i\}_{i = 1}^n$ --- это выборка, то $\chi\{X_i \in B\}$ при фиксированном $B$ являются тоже независимыми одинаково распределёнными случайными величинами (ибо выражены из выборки через борелевские функции). Стало быть, верна сходимость:
	\[
		\forall B \in \B(\R^m)\ \ \frac{\sum_{i = 1}^n \chi\{X_i \in B\}}{n} = P_n^*(B) \xrightarrow[n \to \infty]{P\text{ п.н.}} \E\chi\{X_1 \in B\} = P_X(B)
	\]
\end{proof}

\begin{note}
	До конца главы мы считаем, что $m = 1$ и $X_1, \ldots, X_n$ --- случайные величины, образующие выборку из распределения $P_X$ в пространстве $(\R^m, \B(\R^m))$
\end{note}

\begin{definition}
	Функция $F_n^*(x) = \frac{1}{n}\ps{\sum_{i = 1}^n \chi\{X_i \le x\}}$ называется \textit{эмпирической функцией распределения}.
\end{definition}

\begin{note}
	Опять же, отметим, что эмпирическая функция распределения зависит неявно от $\omega \in \Omega$. При каждом отдельном $\omega$ эта функция действительно является функцией распределения
\end{note}

\begin{corollary}
	Имеет место сходимость:
	\[
		\forall x \in \R\ \ F_n^*(x) \xrightarrow[n \to \infty]{P\text{ п.н.}} F(x)
	\]
\end{corollary}

\begin{proof}
	Следует из того, что $F_n^*(x) = P_n^*\rsi{-\infty; x}$
\end{proof}

\begin{theorem} (Гливенко-Кантелли)
	Пусть $\{X_i\}_{i = 1}^\infty$ --- независимые одинаково распределённые случайные величины с функцией распределения $F(x)$. Тогда имеет место сходимость:
	\[
		D_n := \sup_{x \in \R} |F_n^*(x) - F(x)| \xrightarrow[n \to \infty]{P\text{ п.н.}} 0
	\]
\end{theorem}

\begin{note}
	То есть $P$-почти наверное имеет место равномерная сходимость $F_n^*(x)$ к $F(x)$.
\end{note}

\begin{proof}
	Для начала нужно установить, что $D_n$ является случайной величиной. Это действительно так, ведь $F$ просто непрерывна справа, а при любом $\omega \in \Omega$ и $F_n^*$ тоже непрерывна справа. Значит, в силу всюду плотности $\Q$ в $\R$ можем записать супремум следующим образом:
	\[
		D_n(\omega) = \sup_{x \in \R} |F_n^*(\omega, x) - F(x)| = \sup_{x \in \Q} |F_n^*(\omega, x) - F(x)|
	\]
	Стало быть, $D_n$ является случайной величиной, коль скоро это супремум по счётной совокупности случайных величин. Чтобы показать стремление $D_n$ к нулю, сделаем зажимающие оценки сверху и снизу. Нам потребуется <<порезать>> $F(x)$ на дробные кусочки. Эти кусочки мы зададим соответствующими точками:
	\[
		\forall N \in \N\ \forall K \in \range{1}{N - 1}\ x_{N, K} = \inf \set{x \in \R \colon F(x) \ge \frac{K}{N}}
	\]
	Естественно определим $x_{N, 0} = -\infty$ и $x_{N, N} = +\infty$. Теперь мы готовы писать оценки:
	\begin{itemize}
		\item[$\le$] Пусть $x \in \lsi{x_{N, K}; x_{N, K + 1}}$. Тогда верна такая цепочка неравенств:
		\begin{multline*}
			F_n^*(x) - F(x) \le F_n^*(x_{N, K + 1} - 0) - F(x_{N, K}) =
			\\
			(F_n^*(x_{N, K + 1} - 0) - F(x_{N, K + 1} - 0)) + (F(X_{N, K + 1} - 0) - F(x_{N, K})) \le
			\\
			F_n^*(x_{N, K + 1} - 0) - F(x_{N, K + 1} - 0) + \frac{1}{N}
		\end{multline*}
		
		\item[$\ge$] Аналогичным методом получаем оценку $F_n^*(x) - F(x) \ge F_n^*(x_{N, K}) - F(x_{N, K}) - \frac{1}{N}$
	\end{itemize}
	Стало быть, верна такая общая оценка на модуль:
	\[
		|F_n^*(x) - F(x)| \le \max\big(|F_n^*(x_{N, K + 1} - 0) - F(x_{N, K + 1} - 0)|, |F_n^*(x_{N, K}) - F(x_{N, K})|\big) + \frac{1}{N}
	\]
	Отсюда же получаем оценку на супремум:
	\begin{multline*}
		\sup_{x \in \R} |F_n^*(x) - F(x)| \le
		\\
		\max_{0 \le k \le N - 1}\big(|F_n^*(x_{N, K + 1} - 0) - F(x_{N, K + 1} - 0)|, |F_n^*(x_{N, K}) - F(x_{N, K})|\big) + \frac{1}{N}
	\end{multline*}
	Понятно, что при стремлении $n \to \infty$ второй элемент максимума точно устремится к нулю, но вот с первым это не совсем ясно. Проясним этот момент через определения и уже известные факты:
	\[
		\forall y \in \R\ \ F_n^*(y - 0) = P_n^*(-\infty; y) \xrightarrow[n \to \infty]{P\text{ п.н.}} P_X(-\infty; y) = F(y - 0)
	\]
	Стало быть, и первый элемент стремится к нулю. Чтобы не оставить пробелов в формализме, зафиксируем $\eps > 0$ и выберем такое $N$, что $\frac{1}{N} < \eps$. В силу всего сказанного выше, остаётся сказать следующее:
	\[
		\varlimsup_{n \to \infty} \sup_{x \in \R} |F_n^*(x) - F(x)| <^{P\text{ п.н.}} \eps
	\]
	Ну а это уже напрямую означает, что $D_n \to^{P\text{ п.н.}} 0$
\end{proof}