\subsubsection{Среднеквадратический подход}

\begin{note}
	Среднеквадратический подход --- это особый случай равномерного подхода, когда
	\begin{itemize}
		\item $\cK$ --- класс несмещённых оценок для $\tau(\theta)$
		
		\item $g$ --- квадратичная функция потерь
	\end{itemize}
\end{note}

\begin{proposition}
	Если $T_1, T_2 \in \cK$, причём обе оценки не хуже любых из того же класса $\cK$, то они эквивалентны в определённом ранее смысле:
	\[
		\forall \theta \in \Theta\ \ T_1(X) =^{P_\theta\text{ п.н.}} T_2(X)
	\]
\end{proposition}

\begin{proof}
	Если оценки являются не хуже любых из того же класса, то выполнены равенства:
	\[
		E_\theta (T_i - \tau(\theta))^2 = \inf_{T \in \cK} \E_\theta (T - \tau(\theta))^2
	\]
	Более того, мы рассматриваем класс несмещённых оценок. Стало быть:
	\[
		\E_\theta (T - \tau(\theta))^2 = \E_\theta T^2 - 2\tau(\theta)\E_\theta T + \tau^2(\theta) = \E_\theta T^2 - \tau^2(\theta)
	\]
	Таким образом, для $T_1$ и $T_2$ мы имеем равенство $\E_\theta T_1^2 = \E_\theta T_2^2$. Воспользуемся следующим алгебраическим свойством:
	\[
		\ps{\frac{a + b}{2}}^2 + \ps{\frac{a - b}{2}}^2 = \frac{a^2 + b^2}{2}
	\]
	Положим $a = T_1$ и $b = T_2$ и обозначим $T_3 = \frac{T_1 + T_2}{2}$. Если взять матожидание с обеих сторон равенства, то получим:
	\[
		\E_\theta T_3^2 + \frac{1}{4}\E_\theta (T_1 - T_2)^2 = \E_\theta T_1^2
	\]
	Если вычесть с каждой стороны по $\tau^2(\theta)$, то придём к функциям риска $T_3$ и $T_1$. Так как $T_1$ не хуже любой оценки из класса $\cK$ и $T_3 \in \cK$, то верно неравенство:
	\[
		\frac{1}{4}\E_\theta (T_1 - T_2)^2 = \E_\theta T_1^2 - \E_\theta T_3^2 = \E_\theta (T_3 - \tau(\theta))^2 - \E_\theta (T_1 - \tau(\theta))^2 \le 0
	\]
	Так как слева написана неотрицательная величина, то $\E_\theta (T_1 - T_2)^2 = 0$ абсолютно точно. А такое возможно тогда и только тогда, когда $(T_1 - T_2)^2 =^{P_\theta\text{ п.н.}} 0$, что эквивалентно $T_1 =^{P_\theta\text{ п.н.}} T_2$
\end{proof}

\begin{definition}
	Если семейству распределений $\cP = \{P_\theta, \theta \in \Theta\}$ соответствует семейство плотностей $\{p_\theta, \theta \in \Theta\}$ по одной и той же мере $\mu$, то говорят, что \textit{$\cP$ доминируемо относительно $\mu$}
\end{definition}

\begin{note}
	Далее мы считаем, что $X$ --- наблюдение с распределением из $\cP$, причём семейство доминируемо относительно $\mu$, и рассматриваем $\Theta \subseteq \R$
\end{note}

\begin{note}
	Выборка (как вектор или кортеж) тоже является наблюдением. Её распределение согласовано с распределениями каждой компоненты, а отсюда плотность является тензорным произведением плотностей компонент.
\end{note}

\begin{definition}
	Пусть $X$ --- наблюдение. Случайная величина $U_\theta(X) = \pd{}{\theta}\ln p_\theta(X)$ называется \textit{вкладом наблюдения $X$}
\end{definition}

%\begin{note}
%	Интуиция за вкладом наблюдения раскроется, если расписать производную:
%	\[
%	U_\theta(x) = \frac{1}{p_\theta(x)} \cdot  \pd{}{\theta} p_\theta(x)
%	\]
%	С учётом того, что $p_\theta \ge 0$, вклад наблюдения по сути является относительной скоростью $p_\theta(x)$ по $\theta$.
%\end{note}

\begin{definition}
	Пусть $X$ --- наблюдение. Функция $I_X(\theta) = \E_\theta U_\theta^2(X)$ называется \textit{количеством информации о параметра $\theta$, содержащемся в $X$ (информация по Фишеру)}
\end{definition}

\begin{proposition}
	Если $X = (X_1, \ldots, X_n)$ --- выборка из $n$ наблюдений, то, при обозначении информации Фишера одного наблюдения $i(\theta)$ верно равенство $I_X(\theta) = ni(\theta)$.
\end{proposition}

\begin{proof}
	Как уже было упомянуто ранее, верно равенство:
	\[
		p_\theta(x) = p_\theta(x_1, \ldots, x_n) = p_{\theta, 1}(x_1) \cdot \ldots \cdot p_{\theta, n}(x_n)
	\]
	Стало быть, $U_\theta(x) = \pd{}{\theta} \ps{\sum_{i = 1}^n \ln p_{\theta, i}(x_i)} = \sum_{i = 1}^n U_{\theta, i}(x_i)$. С учётом того, что $X_i$ независимы, получаем равенство:
	\[
		I_X(\theta) = \E_\theta \ps{\sum_{i = 1}^n U_{\theta, i}(X_i)}^2 = \sum_{i = 1}^n \E_\theta U_{\theta, i}^2(X_i) = ni(\theta)
	\]
\end{proof}

\begin{definition}
	\textit{Условиями регулярности} для вероятностно-статистической модели $(\cX, \B(\cX), \cP)$ и наблюдения $X$ называют следующие требования:
	\begin{enumerate}
		\item $\Theta \subseteq \R$ --- открытый интервал (может быть бесконечным)
		
		\item $A = \{x \in \cX \colon p_\theta(x) > 0\}$ не зависит от $\theta$. Это множество называют \textit{носителем}
		
		\item Для любой статистики $S(X)$ с условием $\forall \theta \in \Theta\ \E_\theta S^2(X) < \infty$ выполнено равенство:
		\[
			\pd{}{\theta} \E_\theta S(X) = \pd{}{\theta} \int_A S(x)p_\theta(x)d\mu(x) = \int_A S(x)\pd{}{\theta} p_\theta(x)d\mu(x) = \E_\theta \ps{S(X)\pd{}{\theta}\ln p_\theta(X)}
		\]
		В частности требуем, чтобы для любого $\theta \in \Theta\ \pd{}{\theta}\ln p_\theta(x)$ существовало на $A$ и было конечно
		
		\item $\forall \theta \in \Theta\ \ 0 < I_X(\theta) < \infty$
	\end{enumerate}
\end{definition}

\begin{theorem} (Неравенство Рао-Крам\'{е}ра)
	Пусть выполнены условия регулярности, а также оценка $\hat{\theta}(X) \in \cK$ обладает условием $\forall \theta \in \Theta\ \E_\theta \hat{\theta}^2(X) < \infty$. Тогда верно следующее неравенство:
	\[
		D_\theta \hat{\theta}(X) \ge \frac{(\tau'(\theta))^2}{I_X(\theta)}
	\]
\end{theorem}

\begin{proof}
	Подставим $S(X) = 1$ в третье условие регулярности:
	\[
		\pd{}{\theta} \E_\theta S(X) = \pd{}{\theta} 1 = 0 = \E_\theta \pd{}{\theta} \ln p_\theta(X) = \E_\theta U_\theta(X)
	\]
	Получили $E_\theta U_\theta(X) = 0$. Теперь, в то же условие подставим $\hat{\theta}(X)$:
	\[
		\pd{}{\theta} \E_\theta \hat{\theta}(X) = \pd{}{\theta} \tau(\theta) = \tau'(\theta) = \E_\theta \hat{\theta}(X)U_\theta(X)
	\]
	Умножим равенство $\E_\theta U_\theta(X) = 0$ на $\tau(\theta)$ и вычтем его из последнего равенства. Получим следующее:
	\[
		\tau'(\theta) = \E_\theta (\hat{\theta}(X) - \tau(\theta))U_\theta(X)
	\]
	Остаётся воспользоваться интегральным неравенством КБШ:
	\[
		(\tau'(\theta))^2 \le \Big(\E_\theta (\hat{\theta} - \tau(\theta))^2\Big)\Big(\E_\theta U_\theta^2(X)\Big) = D_\theta \hat{\theta}(X) \cdot I_X(\theta)
	\]
\end{proof}

\begin{corollary}
	Если $\tau(\theta) = \theta$, то в условиях теоремы имеем неравенство $D_\theta \hat{\theta}(X) \ge \frac{1}{I_X(\theta)}$
\end{corollary}

\begin{anote}
	Далее $\wh{\cK}$ --- это подмножество класса $\cK$, у оценок которого существует конечная дисперсия (то есть выполнено условие $\forall \theta \in \Theta\ \E_\theta \hat{\theta}^2(X) < \infty$)
\end{anote}

\begin{definition}
	Если в неравенстве Рао-Крамера для оценки $\hat{\theta}(X) \in \wh{\cK}$ достигается равенство, то оценка $\hat{\theta}(X)$ называется \textit{эффективной оценкой $\tau(\theta)$}.
\end{definition}

\begin{theorem} (Критерий эффективности)
	Пусть выполнены условия регулярности. Тогда оценка $\hat{\theta}(X)$ принадлежит классу $\wh{\cK}$ и является эффективной оценкой $\tau(\theta)$ тогда и только тогда, когда выполнено равенство
	\[
		\hat{\theta}(X) - \tau(\theta) = c(\theta) \cdot U_\theta(X)
	\]
	где $c(\theta) = \frac{\tau'(\theta)}{I_X(\theta)}$.
\end{theorem}

\begin{proof}~
	\begin{itemize}
		\item[$\Ra$] Если $\hat{\theta}$ --- эффективная оценка, то в доказательстве неравенства Рао-Крамера у нас достигается равенство в КБШ. Это происходит тогда и только тогда, когда $\eta = \hat{\theta}(X) - \tau(\theta)$ и $\xi = U_\theta(X)$ являются линейно зависимыми случайными величинами, то есть $\alpha(\theta) + \beta(\theta)\xi + \gamma(\theta)\eta =^{P_\theta\text{ п.н.}} 0$. Заметим, что $\E_\theta \xi = \E_\theta \eta = 0$. Стало быть, если применить матожидание к имеющемуся равенству, то $\alpha(\theta) =^{P_\theta\text{ п.н.}} 0$. Этот факт также позволяет заявить, что $\gamma(\theta) \neq^{P_\theta\text{ п.н.}} 0$ (иначе линейная комбинация тривиальна). Таким образом:
		\[
			\eta =^{P_\theta\text{ п.н.}} -\frac{\beta(\theta)}{\gamma(\theta)}\xi;\ \ \hat{\theta}(X) - \tau(\theta) = c(\theta) \cdot U_\theta(X)
		\]
		
		\item[$\La$] Выразим оценку из равенства:
		\[
			\hat{\theta}(X) = \tau(\theta) + c(\theta) \cdot U_\theta(X)
		\]
		Так как $E_\theta U_\theta(X) = 0$, то $\hat{\theta} \in \cK$. Перенесём $\tau(\theta)$ влево, возведём равенство в квадрат и возьмём матожидание:
		\[
			D_\theta \hat{\theta}(X) = \E_\theta (\hat{\theta}(X) - \tau(\theta))^2 = c^2(\theta)\E_\theta U_\theta^2(X) = c^2(\theta)I_X(\theta)
		\]
		Так как $0 < I_X(\theta) < +\infty$, то $\hat{\theta} \in \wh{\cK}$. Теперь домножим исходное равенство на $U_\theta(X)$ и возьмём матожидание:
		\[
			\E_\theta (\hat{\theta} - \tau(\theta))U_\theta(X) = c(\theta) \E_\theta U_\theta^2(X) = c(\theta) I_X(\theta)
		\]
		При этом, из доказательства неравенства Рао-Крамера мы знаем, что левая часть равна $\tau'(\theta)$. Отсюда выражение для $c(\theta)$. В силу линейной зависимости $\eta = \hat{\theta} - \tau(\theta)$ и $\xi = U_\theta(X)$, $\hat{\theta}$ является эффективной оценкой.
	\end{itemize}
\end{proof}

\begin{corollary}
	Если $\theta^*(X) \in \wh{\cK}$ не хуже эффективной оценки $\hat{\theta} \in \wh{\cK}$, то $\theta^* =^{P_\theta\text{ п.н.}} \hat{\theta}$.
\end{corollary}

\begin{note}
	Если есть эффективная оценка $\tau(\theta)$, то она наилучшая оценка $\tau(\theta)$ в классе $\wh{\cK}$. Обратное, при этом, неверно.
\end{note}

\begin{theorem}
	Если в условиях регулярности существует эффективная оценка для $\tau(\theta)$, $\tau \neq const$, то множество функций, для которых существует эффективная оценка, может быть выражено как $\{a\tau(\theta) + b \colon a, b \in \R\}$.
\end{theorem}

\begin{anote}
	Теорема работает так же, как и неопределённый интеграл: для выражения множества всех первообразных достаточно знать одну.
\end{anote}

\begin{proposition}
	Пусть $A = \{x \in \cX \colon p_\theta(x) > 0\}$ не зависит от $\theta$. Тогда имеет место эквивалентность:
	\[
		\forall B \in \B(\cX)\quad \Big(\exists \theta_0 \in \Theta\ P_{\theta_0}(X \in B) = 1\Big) \Lra \Big(\forall \theta \in \Theta\ P_\theta(X \in B) = 1\Big)
	\]
\end{proposition}

\begin{proof}
	Доказывать утверждение нужно только в одну сторону, ибо в другую очевидно. Итак, $P_{\theta_0}(X \in B) = 1$. Вспомним, что всегда выполнено равенство:
	\[
		\forall C, D \in \B(\cX)\ \ P_\theta(C) + P_\theta(D) = P_\theta(C \cup D) + P_\theta(C \cap D)
	\]
	При всех $\theta$ верно, что $P_\theta(X \in A) = P_\theta(X \in A \cup B) = 1$. Стало быть, $P_\theta(X \in B) = P_\theta(X \in A \cap B)$.
	\textcolor{red}{Появится ближе к сессии. Идейно надо от противного}
\end{proof}

\begin{proof}
	Доказывать равенство множеств будем двумя вложениями:
	\begin{itemize}
		\item[$\subseteq$] Пусть $\hat{\tau}(X)$ и $\hat{v}(X)$ --- эффективные оценки для $\tau(\theta)$ и $v(\theta)$ соответственно. По критерию эффективности мы знаем, что:
		\[
			\forall \theta \in \Theta\ \System{
				&{\hat{\tau}(X) =^{P_\theta\text{ п.н.}} \tau(\theta) + c(\theta)U_\theta(X)}
				\\
				&{\hat{v}(X) =^{P_\theta\text{ п.н.}} v(\theta) + d(\theta)U_\theta(X)}
			}
		\]
		В условиях регулярности $\Theta$ является интервалом, причём мы знаем, что $\tau \neq const$. Стало быть, существует $\theta_0$ такой, что $\tau'(\theta_0) \neq 0$. Тогда и $c(\theta_0) \neq 0$, а потому при $\theta_0$ из первого равенства можно выразить $U_\theta(X)$ и подставить во второе:
		\[
			\hat{v}(X) = v(\theta_0) + d(\theta_0)\ps{\frac{\hat{\tau}(X) - \tau(\theta_0)}{c(\theta_0)}} = a(\theta_0)\hat{\tau}(X) + b(\theta_0)
		\]
		
		\item[$\supseteq$] Итак, $\hat{\tau}(X)$ --- эффективная оценка для $\tau(\theta)$. Нужно проверить, что и $\hat{v}(X) = a\hat{\tau}(X) + b$ является эффективной оценкой для $v(\theta) = a\tau(\theta) + b$. Простыми манипуляциями преобразуем уже имеющийся критерий эффективности для $\hat{\tau}(X)$ так, чтобы он соответствовал $\hat{v}(X)$:
		\[
			\hat{\tau}(X) = \tau(\theta) + c(\theta)U_\theta(X) \Ra (a\hat{\tau(X)} + b) = (a\tau(\theta) + b) + (ac(\theta))U_\theta(X)
		\]
		Получили верный критерий эффективности для $\hat{v}(X)$, что завершает доказательство.
	\end{itemize}
\end{proof}

