\subsection{Критерий согласия Пирсона}

\begin{problem}
	Рассмотрим дискретную модель $X_k \sim (a_i, p_i)_{i = 1}^m$. Необходимо проверить гипотезу $H_0 \colon p = p_0$ (где $p = (p_1, \ldots, p_m)^T$ и $p_0$ аналогично) на уровне значимости $\eps$.
\end{problem}

\begin{definition}
	$V_i(X)$ --- статистика количества исходов, равных $a_i$:
	\[
		V_i(X) := \sum_{k = 1}^n I\{X_k = a_i\}
	\]
\end{definition}

\begin{definition}
	\textit{Статистикой хи-квадрат (Пирсона)} называется статистика следующего вида:
	\[
		\wh{\chi}_n^2 = \sum_{i = 1}^m \frac{(V_i - np_{0, i})^2}{np_{0, i}}
	\]
\end{definition}

\begin{theorem} (Пирсона)
	Если гипотеза $H_0$ верна, то имеет место сходимость
	\[
		\wh{\chi}_n^2 \xrightarrow[n \to \infty]{d} \chi_{m - 1}^2
	\]
\end{theorem}

\begin{proof}
	Определим $Y_k$ как бинарную маску для $X_k$, где единица стоит в $i$-й позиции и означает, что $X_k = a_i$:
	\[
		Y_k = (I\{X_k = a_1\}, \ldots, I\{X_k = a_m\})^T
	\]
	Отсюда сразу $\E Y_k = p_0$, $\sum_{k = 1}^n Y_k = V(X) := (V_1, \ldots, V_m)^T$. Также $Y_k$ являются независимыми одинаково распределёнными векторами, коль скоро это функции от $X_k$. Выясним, какой ковариационной матрицей обладает $Y_k$:
	\[
		\cov(I\{X_k = a_i\}, I\{X_k = a_j\}) = \E I\{X_k = a_i \wedge X_k = a_j\} - p_{0, i}p_{0, j} = \System{
			&{p_{0, i} - p_{0, i}^2,\ i = j}
			\\
			&{-p_{0, i}p_{0, j},\ i \neq j}
		}
	\]
	Таким образом, если $B = \diag(p_{0, 1}, \ldots, p_{0, m})$, то $D Y_k = B - p_0p_0^T$. Заметим, что статистика хи-квадрат является квадратом нормы такого вектора:
	\[
		\xi_n := (\sqrt{B})^{-1}\sqrt{n}\ps{\frac{V}{n} - p_0} = (\sqrt{B})^{-1}\sqrt{n}\ps{\frac{\sum_{k = 1}^n Y_k}{n} - p_0}
	\]
	Последнее выражение крайне сильно намекает на ЦПТ и теорему о наследовании. Воспользуемся ими:
	\[
		\sqrt{n}(\ol{Y} - p_0) \xrightarrow[n \to \infty]{d} N(0, B - p_0p_0^T) \Ra \xi_n \xrightarrow[n \to \infty]{d} N(0, (\sqrt{B})^{-1}(B - p_0p_0^T)(\sqrt{B})^{-T})
	\]
	Так как $B$ симметрична, то $\sqrt{B}$ тоже. Перепишем дисперсию в более приятном виде:
	\[
		(\sqrt{B})^{-1}(B - p_0p_0^T)(\sqrt{B})^{-1} = E_m - \big((\sqrt{B})^{-1}p_0\big)\big((\sqrt{B})^{-1}p_0\big)^T
	\]
	Обозначим $z = (\sqrt{B})^{-1}p_0 = (\sqrt{p_{0, 1}}, \ldots, \sqrt{p_{0, m}})$. Тогда дисперсия --- просто $E_m - zz^T$. Пусть $M$ --- произвольная ортогональная матрица, такая, что её первая строка равна $z^T$. По теореме о наследовании сходимости получим сходимость для $M\xi_n$. Дисперсия в этом случае равна следующему:
	\[
		Vz = (1, 0, \ldots, 0)^T \Ra V(E_m - zz^T)V^T = E_m - (Vz)(Vz)^T = \diag(0, 1, \ldots, 1) = \wt{E}_m
	\]
	Стало быть:
	\[
		\|\xi_n\|^2 = \|M\xi_n\|^2 \xrightarrow[n \to \infty]{d} \|N(0, \wt{E}_m)\|^2 = \chi_{m - 1}^2
	\]
\end{proof}

\begin{solution} (Критерий $\chi^2$ Пирсона)
	Пусть $u_{1 - \eps}$ --- квантиль $\chi_{m - 1}^2$. Тогда критерий имеет вид $\{x \colon \wh{\chi}_n^2(x) > u_{1 - \eps}\}$.
\end{solution}

\begin{note}
	Важно отметить, что данный критерий является \textit{асимптотическим}. Его мощность уменьшается с ростом $n$, поэтому при малых $n$ он не применим. В статистике принято использовать этот критерий, только если $np_{0, i} \ge 5$ для всех $i \in \range{1}{m}$.
\end{note}

\begin{lemma} (о состоятельности критерия Пирсона против альтернативы)
	Пусть альтернатива имеет вид $H_1 \colon p \neq p_0$. Тогда критерий Пирсона состоятелен для проверки гипотезы $H_0$ против $H_1$.
\end{lemma}

\begin{proof}
	Запишем статистику хи-квадрат в следующем виде:
	\[
		\wh{\chi}_n^2 = n\sum_{i = 1}^m \ps{\frac{V_i}{n} - p_{0, i}}^2 \frac{1}{p_{0, i}}
	\]
	Согласно УЗБЧ:
	\[
		\forall i \in \range{1}{m}\ \ \frac{V_i}{n} = \frac{1}{n}\sum_{k = 1}^n I\{X_k = a_i\} \xrightarrow{\aal{P}} P(X_1 = a_i) = p_i
	\]
	Стало быть, по теореме о наследовании сходимости:
	\[
		\sum_{i = 1}^m \ps{\frac{V_i}{n} - p_{0, i}}^2 \frac{1}{p_{0, i}} \xrightarrow{\aal{P}} \sum_{i = 1}^m (p_i - p_{0, i})^2 \frac{1}{p_{0, i}}
	\]
	Если верна альтернатива, то величина справа --- это какое-то положительное число. Стало быть, при верности гипотезы $H_1$ есть сходимость $\wh{\chi}_n^2 \xrightarrow{\aal{P}} +\infty$, а это сразу даёт соответствие пределу $\lim_{n \to \infty} P(\wh{\chi}_n^2 > u_{1 - \eps}) = 1$, что и требовалось доказать.
\end{proof}

\subsection{Критерий согласия Колмогорова}

\begin{problem}
	Рассмотрим непрерывную модель $X_k \sim P$, где $P$ обладает непрерывной функцией распределения $F$ на $\R$. Мы снова хотим проверить гипотезу $H_0 \colon F = F_0$ на уровне значимости $\eps$.
\end{problem}

\begin{theorem} (без доказательства)
	Если функция $F$ распределения выборки непрерывна, то
	\begin{enumerate}
		\item Распределение $D_n(x) = \sup_{x \in \R} |F_n^*(x) - F(x)|$ не зависит от вида $F$
		
		\item Имеет место сходимость $\sqrt{n}D_n \xrightarrow{d} \xi$, где $\xi$ обладает \textit{распределением Колмогорова}:
		\[
			\xi \sim K \Lra P(\xi \le z) = I\{z > 0\}\sum_{j \in \Z} (-1)^je^{-2j^2z^2}
		\]
	\end{enumerate}
\end{theorem}

\begin{solution} (Критерий Колмогорова)
	Пусть $K_{1 - \eps}$ --- квантиль распределения Колмогорова. Тогда критерием будет $\{\sqrt{n}D_n(x) > K_{1 - \alpha}\}$.
\end{solution}

\begin{note}
	Как и критерий Пирсона, критерий Колмогорова тоже является асимптотическим. Его применяют при $n \ge 20$.
\end{note}

\subsection{Критерий согласия Смирнова-Фон Мизеса}

\begin{definition}
	\textit{Статистикой $\omega^2$} называется статистика следующего вида:
	\[
		\omega^2(X) = n\int_\R (F_n^*(x) - F_0(x))dF_0(x)
	\]
\end{definition}

\begin{anote}
	Зависимость $\omega^2$ от $X$ скрыта в эмпирическом распределении $F_n^*$.
\end{anote}

\begin{theorem} (без доказательства)
	Если проверяется гипотеза $H_0 \colon F = F_0$, где $F_0$ непрерывна, то распределение $\omega^2$ не зависит от вида $F_0$.
\end{theorem}

\begin{anote}
	Распределение, к которому сходится статистика $\omega^2$, достаточно сложно описать математическими терминами, поэтому его обычно опускают. Чтобы найти квантиль этого распределения, используют статистическую таблицу.
\end{anote}

\begin{definition}
	Критерии, которые проверяют гипотезу вида $H_0 \colon P = P_0$, называются \textit{критериями согласия}.
\end{definition}

\begin{note}
	Критерии Пирсона, Колмогорова, Смирнова-Фон Мизеса как раз относятся к критериям согласия.
\end{note}

\section{Байесовские оценки}

\begin{note}
	Здесь мы возвращаемся к байесовскому подходу в моделировании эксперимента: $\theta \in \Theta \subseteq \R^d$ не просто параметризация семейства $\cP$, а случайная величина с распределением $Q$. Напомним, что вероятностное пространство такой модели имеет вид $(\Theta \times \cX, \B(\Theta) \times \B(\cX), \wt{P})$, где $\wt{P}$ порождается через условную вероятность $P(B|\theta)$ и меру $Q$:
	\[
		P(A \times B) = \int_A P(B|t)dQ(t)
	\]
	Мы будем считать, что семейство $\cP$ и вероятностная мера $Q$ доминируемы относительно меры $\mu$, то есть определены плотности $p_\theta(x)$ (относительная плотность $x$ по $\theta$) и $q(\theta)$. Тогда:
	\[
		P(A \times B) = \int_A P(B|t)q(t)d\mu(t) = \int_{A \times B} p_t(x)q(t)d\mu(t)
	\]
	Стало быть, плотность $\wt{P}$ имеет вид $f(t, x) = p_t(x)q(t)$.
	
	Эксперимент в этой модели, как и ранее, описывается через тождественные случайные величины $X(x) = x$ и $\theta(t) = t$.
\end{note}

\begin{definition}
	\textit{Априорной плотностью параметра $\theta$} называется функция $q(t)$.
\end{definition}

\begin{definition}
	\textit{Апостериорной плотностью параметра $\theta$} называется условная плотность $\theta$ по $X$:
	\[
		q(t|X) = \frac{q(t)p_t(X)}{\int_\Theta p_u(x)q(u)d\mu(u)} = \frac{f(t, X)}{\int_\Theta p_u(x)q(u)d\mu(u)}
	\]
\end{definition}

\begin{definition}
	\textit{Байесовской оценкой параметра $\theta$} называется оценка следующего вида:
	\[
		\wh{\theta}(X) = \int_\Theta tq(t|X)dt = \E_{\wt{P}}(\theta|X)
	\]
\end{definition}