\section{Параметрическая модель}

\begin{definition}
	Вероятностно-статистическая модель $(\cX, \B(\cX), \cP)$ называется \textit{параметрической}, если $\cP$ является не мерой, а параметризованным множеством мер на \\ $(\cX, \B(\cX))$, то есть
	\[
		\cP = \{P_\theta \colon \theta \in \Theta\}
	\]
\end{definition}

\begin{note}
	В силу свойств множества, мы считаем верной импликацию $(\theta_1 \neq \theta_2) \Ra (P_{\theta_1} \neq P_{\theta_2})$
\end{note}

\begin{note}
	Как правило, $\Theta \subseteq \R^k$. Можно вспомнить базовые примеры, такие как $Exp(\theta)$, $Cauchy(\sigma)$, $N(a, \sigma^2)$
\end{note}

\begin{anote}
	Далее, когда встречаются понятия, такие как наблюдение, в рамках параметрической модели, то их нужно понимать в вероятностно-статистической модели $(\cX, \B(\cX), P_\theta)$ для некоторого $\theta \in \Theta$
\end{anote}

\subsection{Статистики и оценки}

\begin{definition}
	Пусть $(\cX, \B(\cX), \cP)$ --- вероятностно-статистическая модель, $X$ --- наблюдение, $(E, \cE)$ --- измеримое пространство, $S \colon \cX \to E$ --- измеримое отображение. Тогда $S(X)$ называют \textit{статистикой}.
\end{definition}

\begin{note}
	Как правило, $(E, \cE) = (\R, \B(\R))$ или $(\R^2, \B(\R^2))$.
\end{note}

\begin{example}
	Пусть $\cX = \R^m$. Тогда для выборки $X = (X_1, \ldots, X_m)$ можно рассмотреть статистику среднего значения:
	\[
		S(X) = \ol{X} := \frac{\sum_{i = 1}^n X_i}{n}
	\]
\end{example}\

\begin{definition}
	Пусть $X$ --- наблюдение в параметрической модели $(\cX, \B(\cX), \cP)$, а $S(X)$ --- оценка, чьё множество значений вложено в $\Theta$. Тогда $S(X)$ называется \textit{оценкой неизвестного параметра $\theta$}.
\end{definition}

\begin{note}
	Аналогичное определение можно ввести и для $\tau(\theta)$, где $\tau$ --- некоторая произвольная функция.
\end{note}

\begin{example}
	Приведём несколько базовых классов статистик. Далее $X = (X_1, \ldots, X_n)$ --- это выборка из некоторого распределения в $(\R^m, \B(\R^m))$:
	\begin{enumerate}
		\item Если $g \colon \R^m \to \R^n$ --- борелевская функция, то \textit{выборочной характеристикой функции $g(x)$} называется следующая статистика:
		\[
			\ol{g(X)} = \frac{1}{n} \sum_{i = 1}^n g(X_i)
		\]
		\begin{itemize}
			\item $\ol{X} = \frac{1}{n} \sum_{i = 1}^n X_i$ --- выборочное среднее
			
			\item ($m = 1$) $\ol{X^k} = \frac{1}{n} \sum_{i = 1}^n X_i^k$ --- выборочный момент $k$-го порядка
		\end{itemize}
		
		\item Если $h$ --- борелевская функция, то \textit{функцией от выборочных квантилей} называется статистика следующего вида:
		\[
			S(X) = h(\ol{g_1(X)}, \ldots, \ol{g_k(X)})
		\]
		\begin{itemize}
			\item ($m = 1$) $S^2 = \ol{X^2} - \ol{X}^2$ --- выборочная дисперсия. Здесь $h(x, y) = x - y^2$
			
			\item $M_k = \frac{1}{n} \sum_{i = 1}^n (X_i - \ol{X})^k$ --- выборочный центральный момент $k$-го порядка
		\end{itemize}
	
		\item ($m = 1$) \textit{$k$-й порядковой статистикой} называется $k$-й по величине снизу элемент.
		\begin{itemize}
			\item $X_{(1)} = \min(X_1, \ldots, X_n)$
			
			\item $X_{(2)} =$ второй по порядку возрастания элемент выборки
			
			\item $X_{(n)} = \max(X_1, \ldots, X_n)$
		\end{itemize}
		При этом вектор $(X_{(1)}, \ldots, X_{(n)})$ называется \textit{вариационным рядом}
	\end{enumerate}
\end{example}

\begin{exercise}
	Доказать, что $S^2 = \frac{1}{n} \sum_{i = 1}^n (X_i - \ol{X})^2 = M_2$
\end{exercise}

\begin{note}
	Отметим, что до этого мы смогли ввести понятие эмпирической функции распределения. Для неё верен факт $F_n^*(x) \xrightarrow{P_\theta\text{ п.н.}} F(x)$. Стало быть, имеет место и слабая сходимость этих функций, отсюда получаем связь между статистиками и их матожиданиями (в довольно частном случае, но его очевидно можно обобщать):
	\begin{multline*}
		\forall g \colon \R \to \R \text{ --- непрерывная ограниченная}
		\\
		\ol{g(X)} = \frac{1}{n}\sum_{i = 1}^n g(X_i) = \int_\R g(x)dF_n^*(x) \xrightarrow[n \to \infty]{} \int_\R g(x)dF(x) = \E g(X_i)
	\end{multline*}
\end{note}

\begin{proposition}
	Для выборочной дисперсии $S^2$ выборки $X$ верна формула:
	\[
		\E S^2 = \frac{n - 1}{n} DX_1
	\]
\end{proposition}

\begin{proof}
	Просто раскроем среднее и пересоберём формулу:
	\begin{multline*}
		\E S^2 = \frac{1}{n} \sum_{i = 1}^n \E\ps{X_i - \frac{1}{n}\sum_{j = 1}^n X_j}^2 = \frac{1}{n} \sum_{i = 1}^n \E\ps{X_i^2 - \frac{2}{n}\sum_{j = 1}^n X_iX_j + \frac{1}{n^2}\sum_{j = 1}^n \sum_{k = 1}^n X_jX_k} =
		\\
		\frac{1}{n} \sum_{i = 1}^n \E\ps{X_i^2 - \frac{2}{n}X_i^2 - \frac{2}{n} \sum_{j \neq i} X_iX_j + \frac{1}{n^2} \sum_{j = 1}^n \sum_{k \neq j} X_jX_k + \frac{1}{n^2} \sum_{j = 1}^n X_j^2}
	\end{multline*}
	Внесём в отдельные части матожидание и посчитаем их:
	\begin{itemize}
		\item $\E\ps{\frac{n - 2}{n}X_i^2} = \frac{n - 2}{n}\E X_i^2$
		
		\item $\E\ps{\frac{2}{n}\sum_{j \neq i} X_iX_j} = \frac{2}{n}\sum_{j \neq i} (\E X_i)\E X_j = \frac{2}{n} \E X_i \sum_{j \neq i} \E X_j = \frac{2(n - 1)}{n} (\E X_1)^2$
		
		\item $\E\ps{\frac{1}{n^2} \sum_{j = 1}^n \sum_{k \neq j} X_jX_k} = \frac{1}{n^2} \sum_{j = 1}^n \sum_{k \neq j} (\E X_j)\E X_k = \frac{n - 1}{n^2} \sum_{j = 1}^n (\E X_1)^2 = \frac{n - 1}{n} (\E X_1)^2$
		
		\item $\E\ps{\frac{1}{n^2} \sum_{j = 1}^n X_j^2} = \frac{1}{n} \E X_1^2$
	\end{itemize}
	Итого:
	\begin{multline*}
		\E S^2 = \frac{1}{n} \sum_{i = 1}^n \ps{\frac{n - 1}n \E X_1^2 - \frac{2(n - 1)}{n}(\E X_1)^2 + \frac{n - 1}{n}(\E X_1)^2} =
		\\
		\frac{n - 1}{n} (\E X_1^2 - (\E X_1)^2) = \frac{n - 1}{n} DX_1
	\end{multline*}
\end{proof}

\begin{note}
	Собственно, наблюдается неожиданная проблема: наша статистика немного смещена от реальной дисперсии, но это очень просто поправить!
	\[
		S_0^2 = \frac{n}{n - 1} S^2 = \frac{1}{n - 1} \sum_{i = 1}^n (X_i - \ol{X})^2 \text{ --- \textit{исправленная выборочная дисперсия}}
	\]
\end{note}

\subsection{Свойства оценок}

\begin{note}
	Далее, если не оговорено явно иного, мы живём в параметрической модели $(\cX, \B(\cX), \cP)$ и рассматриваем выборку $X = (X_1, \ldots, X_n)$ с неизвестным распределением $P = P_\theta \in \cP$
\end{note}

\begin{definition}
	Оценка $\theta^*(X)$ называется \textit{несмещённой оценкой параметра} $\theta$, если выполнено условие:
	\[
		\forall \theta \in \Theta\ \ \E_\theta \theta^*(X) = \theta
	\]
	где $\E_\theta$ --- интеграл Лебега (матожидание) по мере $P_\theta$
\end{definition}

\begin{anote}
	При работе с вероятностно-статистическим пространством с параметром $\theta$ принято писать не $\E$, а $\E_\theta$, чтобы подчеркнуть зависимость интеграла от $\theta$.
\end{anote}

\begin{example}
	Рассмотрим $\cP = \{N(\theta, 1) \colon \theta \in \R\}$. Тогда оценки $X_1$ и $\ol{X}$ --- несмещённые
\end{example}

\begin{anote}
	Последовательность оценок $\{\theta_n^*(X)\}_{n = 1}^\infty$ мы тоже будем называть оценкой далее.
\end{anote}

\begin{definition}
	Пусть $\{X_k\}_{k = 1}^\infty$ --- выборка из неизвестного распределения $P_\theta \in \cP$. Тогда оценка $\theta_n^*(X)$ называется \textit{состоятельной оценкой $\theta$}, если имеется сходимость:
	\[
		\forall \theta \in \Theta\ \ \theta_n^*(X) \xrightarrow[n \to \infty]{P_\theta} \theta
	\]
\end{definition}

\begin{note}
	Определение выше можно расписать так:
	\[
		\forall \theta \in \Theta\ \forall \eps > 0\ \ P_\theta(\|\theta_n^*(X) - \theta\|_2 \ge \eps) \to 0,\ n \to \infty
	\]
\end{note}

\begin{definition}
	Пусть $\{X_k\}_{k = 1}^\infty$ --- выборка из неизвестного распределения $P_\theta \in \cP$. Тогда оценка $\theta_n^*(X)$ называется \textit{сильно состоятельной оценкой $\theta$}, если имеется сходимость:
	\[
		\forall \theta \in \Theta\ \ \theta_n^*(X) \xrightarrow[n \to \infty]{P_\theta\text{ п.н.}} \theta
	\]
\end{definition}

\begin{example}
	Для $\cP = \{N(\theta, 1)\}$ верно, что $\ol{X}$ --- сильно состоятельная оценка (в силу УЗБЧ)
\end{example}

\begin{definition}
	Пусть $\{X_k\}_{k = 1}^\infty$ --- выборка из случайных величин с распределением $P_\theta$. Тогда оценка $\theta_n^*(X)$ называется \textit{асимптотически нормальной оценкой $\theta$}, если выполнена сходимость:
	\[
		\forall \theta \in \Theta\ \ \sqrt{n}(\theta_n^*(X) - \theta) \xrightarrow[n \to \infty]{d_\theta} N(0, \sigma^2(\theta))
	\]
	где $d_\theta$ --- сходимость по распределению $P_\theta$. Величину $\sigma^2(\theta)$ называют \textit{асипмтотической дисперсией оценки $\theta_n^*$}
\end{definition}

\begin{anote}
	По сути определение мотивировано одним из видов ЦПТ:
	\[
		\sqrt{n}\ps{\frac{S_n}{n} - \E X_1} \xrightarrow[n \to \infty]{d} N(0, DX_1)
	\]
\end{anote}

\begin{note}
	Иногда в определении выше требуют, что $\sigma(\theta) > 0$
\end{note}

\begin{example}
	Рассмотрим $\cP = \{N(\theta, 1)\}$. Тогда $\ol{X}$ является асимптотически нормальной оценкой $\theta$ в силу ЦПТ
\end{example}

\begin{note}
	Введённые определения состоятельной, сильно состоятельной и асимптотически нормальной выборки являются \textit{асимптотическими свойствами}, то есть имеют смысл только в том случае, когда число элементов выборки растёт.
\end{note}

\begin{note}
	Мы можем оценивать не только $\theta$, но и $\tau(\theta)$ для некоторой функции $\tau$. Определения меняются соответствующим образом.
\end{note}

\begin{proposition}
	Если $\tau_n^*(X)$ --- асимптотически нормальная оценка для $\tau(\theta)$, то $\tau_n^*(X)$ --- состоятельная оценка для $\tau(\theta)$
\end{proposition}

\begin{proof}
	Зафиксируем $\theta \in \Theta$. Тогда, в силу определения асимптотической нормальности и леммы Слуцкого, верна сходимость:
	\[
		\frac{1}{\sqrt{n}} \cdot \sqrt{n}(\tau_n^*(X) - \tau(\theta)) \xrightarrow[n \to \infty]{d_\theta} 0
	\]
	Итак, $\tau_n^* - \tau(\theta) \to^{d_\theta} 0$. Так как $\tau(\theta)$ --- константа, то имеет место эквивалентная сходимость $\tau_n^* \to^{P_\theta} \tau(\theta)$, что и требовалось доказать.
\end{proof}

\begin{proposition} (Наследование состоятельности и сильно состоятельности)
	Пусть $\theta_n^*(X)$ --- (сильно) состоятельная оценка $\theta \in \Theta \subseteq \R^k$, при этом $\tau \colon \R^k \to \R^s$ --- непрерывная на $\Theta$ функция. Тогда $\tau(\theta_n^*(X))$ --- (сильно) состоятельная оценка $\tau(\theta)$
\end{proposition}

\begin{proof}
	Фактически, это просто теорема о наследовании сходимости для случайных векторов.
\end{proof}