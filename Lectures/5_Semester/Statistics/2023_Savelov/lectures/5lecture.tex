\begin{note}
	Так как $z_p$ --- константа, то при помощи леммы Слуцкого несложно показать, что есть сходимость $z_{n, p} \to^P z_p$.
\end{note}

\begin{definition}
	\textit{Медианой распределения $P$} называется его $\frac{1}{2}$-квантиль.
\end{definition}

\begin{definition}
	Пусть $\{X_i\}_{i = 1}^n$ --- выборка. \textit{Выборочной медианой} $\hat{\mu}$ называется следующая оценка:
	\[
		\hat{\mu} = \System{
			&{X_{(k + 1)},\ n = 2k + 1}
			\\
			&{\frac{X_{(k)} + X_{(k + 1)}}{2},\ n = 2k}
		} = \frac{X_{(\floor{\frac{n}{2}})} + X_{(\ceil{\frac{n}{2}})}}{2}
	\]
\end{definition}

\begin{theorem} (о выборочной медиане, без доказательства)
	Пусть $\{X_i\}_{i = 1}^n$ --- выборка из распределения с плотностью $f(x)$. Пусть $z_{1 / 2}$ --- это $\frac{1}{2}$-квантиль этого распределения, причём $f$ непрерывно дифференцируема в некоторой окрестности $z_{1 / 2}$ и $f(z_{1 / 2}) > 0$. Тогда выполнена сходимость:
	\[
		\sqrt{n}(\hat{\mu}(X) - z_{1 / 2}) \xrightarrow[n \to \infty]{d} N\ps{0, \frac{1}{4f^2(z_{1 / 2})}}
	\]
\end{theorem}

\begin{note}
	Как и в случае теоремы о выборочных квантилях, имеет место сходимость $\hat{\mu} \to^P z_{1 / 2}$
\end{note}

\subsection{Сравнение оценок}

\begin{definition}
	\textit{Функцией потерь} называется любая борелевская неотрицательная функция $g \colon \R^n \times \R^n \to \R_+$.
\end{definition}

\begin{definition}
	Пусть $\Theta \subseteq \R^m$ и $g$ --- функция потерь из $\R^m \times \R^m$. Тогда, если $\theta_n^*$ --- оценка параметра $\theta$, то функция $g(\theta^*(X), \theta)$ называется \textit{величиной потерь}.
\end{definition}

\begin{example}
	Покажем несколько стандартных функций потерь:
	\begin{enumerate}
		\item $g(x, y) = |x - y|$
		
		\item $g(x, y) = (x - y)^2$ --- \textit{квадратичная функция потерь}
		
		\item Пусть $A \in \R^{k \times k}$, $A \succeq 0$ и $\theta \in \R^k$. Тогда $g(x, y) = \tbr{A(x - y), x - y}$
	\end{enumerate}
\end{example}

\begin{definition}
	Пусть $g$ --- функция потерь из $\Theta \times \Theta$. Тогда \textit{функцией риска оценки $\theta^*$} называется следующий объект:
	\[
		R(\theta^*, \theta) := \E_\theta g(\theta^*, \theta)
	\]
\end{definition}

\subsubsection{Равномерный подход к сравнению оценок}

\begin{definition}
	Пусть $\hat{\theta}(X)$ и $\theta^*(X)$ --- оценки параметра $\theta$. Тогда \textit{$\hat{\theta}(X)$ лучше оценки $\theta^*(X)$ в равномерном подходе с функцией потерь $g$}, если выполнены утверждения:
	\begin{itemize}
		\item $\forall \theta \in \Theta\ \ R(\hat{\theta}(X), \theta) \le R(\theta^*(X), \theta)$
		
		\item Для некоторого $\theta_0 \in \Theta$ неравенство выше является строгим
	\end{itemize}
\end{definition}

\begin{note}
	Далее мы отождествляем оценки $T_1$ и $T_2$, если выполнено утверждение:
	\[
		\forall \theta \in \Theta\ \ T_1(X) =^{P_{\theta}\text{ п.н.}} T_2(X)
	\]
\end{note}

\begin{definition}
	Оценка $\hat{\theta}$ называется \textit{наилучшей в классе оценок $\cK$}, если она лучше любой другой оценки $\theta^* \in \cK$.
\end{definition}

\begin{note}
	Наилучшая оценка не всегда существует.
\end{note}

\textcolor{red}{Доказать, что в классе всех оценок нет наилучшей}

\subsubsection{Минимаксный подход}

\begin{definition}
	Оценка $\theta^*(X)$ называется \textit{наилучшей в минимаксном подходе}, если выполнено равенство:
	\[
		\sup_{\theta \in \Theta} R(\theta^*(X), \theta) = \inf_{\hat{\theta} \in \cK} \sup_{\theta \in \Theta} R(\hat{\theta}, \theta)
	\]
\end{definition}

\begin{note}
	Иными словами, $\theta^*(X)$ наилучшая в минимаксном подходе, если она обладает наименьшим максимумом функции риска.
\end{note}

\subsubsection{Байесовский подход}

\begin{note}
	До этого момента мы искали $\theta$ без какого-либо предположения, просто считая, что оно фиксировано. Альтернативный подход состоит в том, чтобы считать $\theta$ \textit{случайной величиной с распределением $Q$}. Тогда задание распределения выборки есть задание условного распределения $P(x|\theta)$.
\end{note}

\begin{anote}
	А как выглядит вероятностное пространство в таком случае, почему это всё работает? Тройка будет иметь вид $(\Theta \times \cX, \B(\B(\F)\times \B(\cX)), P)$. Самым интересным, пожалуй, здесь будет определение $P$. А его можно задать следующим образом:
	\[
		P(A \times B) = \int_A P(B | \theta)dQ(\theta)
	\]
	По сути это формула полной вероятности в форме интеграла.
\end{anote}

\begin{definition}
	Пусть $\hat{\theta}(X)$ --- оценка $\theta$, $R(\hat{\theta}, \theta)$ --- её функция риска. Тогда \textit{риском оценки} \textcolor{red}{не факт, что лекторское определение} назовём следующую величину:
	\[
		R(\hat{\theta}(X)) = \E_Q R(\hat{\theta}(X), \theta) = \int_\Theta R(\hat{\theta}(X), t) dQ(t)
	\]
\end{definition}

\begin{definition}
	Оценка $\theta^*(X)$ называется \textit{наилучшей в байесовском подходе}, если выполнено равенство:
	\[
		R(\theta^*(X)) = \min_{\hat{\theta} \in \cK} R(\hat{\theta}(X))
	\]
\end{definition}

\subsubsection{Асимптотический подход}

\begin{definition}
	Пусть $\hat{\theta}_1$ и $\hat{\theta}_2$ --- асимптотически нормальные оценки параметра $\theta$ с асимптотическими дисперсиями $\sigma_1^2(\theta)$ и $\sigma_2^2(\theta)$ соответственно. Тогда \textit{оценка $\hat{\theta}_1$ лучше оценки $\hat{\theta}_2$ в асимптотическом подходе}, если выполнено утверждение:
	\[
		\forall \theta \in \Theta\ \ \sigma_1^2(\theta) \le \sigma_2^2(\theta)
	\]
\end{definition}

\begin{example}
	Рассмотрим выборку из $N(\theta, 1)$, а также оценки $\hat{\mu}$ и $\ol{X}$.
	\begin{itemize}
		\item С одной стороны, по ЦПТ $\sqrt{n}(\ol{X} - \theta) \xrightarrow{d_\theta} N(0, 1)$
		
		\item С другой стороны, по теореме о выборочной медиане $\sqrt{n}(\hat{\mu} - \theta) \xrightarrow{d_{\theta}} N(0, \pi / 2)$
	\end{itemize}
	Сразу видно, что оценка средним лучше оценки медианой в асимптотическом подходе для семейства $N(\theta, 1)$.
\end{example}

\begin{definition}
	Оценка $\hat{\theta}(X)$ называется \textit{наилучшей в классе оценок $\cK$ в асимптотически нормальном подходе}, если она лучше любой другой оценки.
\end{definition}

\begin{note}
	\textit{Плотностью дискретного распределения $P$ из $(\R^n, \B(\R^n))$} мы называем плотность этого распределения по считающей мере $\mu$ множества $\Z^n$. Понятно, что $p(x) = P(\{x\})$ для любого $x \in \Z^n$ (остальные точки доопределяются, например, нулём)
\end{note}

\begin{example}
	Если $\xi \sim Bin(n, p)$, то $P(\xi = k) = C_n^kp^kq^{n - k}\chi\{k \in \range{0}{n}\} = p_\xi(k)$
\end{example}

\begin{reminder}
	Если $\xi$ --- дискретная случайная величина с плотностью $p(x)$, то по теореме о вычислении интеграла Лебега для меры с плотностью верна формула:
	\[
		\E g(\xi) = \sum_{k \in \Z} g(k)P(\xi = k) = \int_\R g(x)p(x)d\mu(x)
	\]
\end{reminder}

\begin{note}
	Всюду далее, когда говорим о плотности распределения, мы считаем, что либо это обычная плотность абсолютно непрерывного распределения, либо плотность дискретного распределения по считающей мере на $\Z^n$.
\end{note}