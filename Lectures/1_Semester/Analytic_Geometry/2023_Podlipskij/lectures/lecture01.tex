\section{Матрицы}

\subsection{Матрицы. Специальные виды матриц}

\begin{definition}
	$\textit{Матрицей m$\times$n}$ называется упорядоченный набор из $m \cdot n$ чисел, записанных в таблицу, состоящую из m строк и n столбцов.
\end{definition}



\textbf{Обозначения:}
\begin{itemize}
	\item A, B - матрицы
	\item (...), ||...|| - матрицы
	\item $a_{ij}$ - элемент матрица, расположенный в i-той строке j-того столбца
\end{itemize}


\textbf{Специальные виды матриц}
\begin{itemize}
	\item $\textit{строка}$ $-$ матрица, состоящая из 1 строки и n столбцов
	\item $\textit{столбец}$ $-$ матрица, состоящая из n строк и 1 столбца
	\item $\textit{квадратная}$ $-$ матрица, в которой количество строк равняется количеству столбцов
	\item $\textit{единичная}$ $-$ матрица, элементы главной диагонали которой являются единицами, а остальные $-$ нулями, обозначается буквой E
	\item $\textit{треугольная}$ $-$ матрица, у которой элементы над (нижняя треугольная) или под главной диагональю (верхняя треугольная) являются нулями
	\item $\textit{диагональная}$ $-$ матрица, у которой все элементы кроме элементов главной диагонали являются нулями, обозначается diag
	\item $\textit{симметрическая}$ $-$ матрица, элементы которой симметричны относительно главной диагонали
	\item $\textit{кососимметрическая}$ $-$ матрица, элементы которой симметричны относительно главной диагонали, но противоположны по знаку, элементы главной диагонали $-$ нули
	\item $\textit{нулевая}$ $-$ матрица, полностью состоящая из нулей
\end{itemize}

При этом к квадратным матрицам относятся единичные, треугольные, диагональные, симметрические и кососимметрические.



\subsection{Операции над матрицами}

\begin{enumerate}
	\item A = B, если матрицы имеют одинаковые размеры и равны поэлементно
	\item Сложение $C_{m \times n} = A_{m \times n} + B_{m \times n}$ определено для матриц одного размера, при чём $c_{ij} = a_{ij} + b_{ij}$
	\item Умножение матрицы A на число $\alpha \in \R$ $B = \alpha A, b_{ij} = \alpha a_{ij}$
	\item Транспонирование матрицы $A_{m \times n}^T = B_{n \times m}$, где $b_{ij} = a_{ji}$
	
\end{enumerate}

\textit{Свойства операций:}
\begin{itemize}
	\item A + B = B + A
	\item A + (B + C) = (A + B) + C
	\item $\alpha$(A + B) = $\alpha$A + $\alpha$B
	\item ($\alpha\beta$)A = $\alpha(\beta A)$
	\item ($\alpha + \beta$)A = $\alpha A + \beta A$
	\item $A^T = A$ для симметрической матрицы
	\item $A^T = -A$ для кососимметрической матрицы
	\item $(A^T)^T = A$
	\item $(A + B)^T = A^T + B^T$
	\item $(\alpha A)^T = \alpha A^T$
\end{itemize}


\subsection{Определитель(детерминант) матрицы}

\begin{definition}
	$\textit{Определитель(детерминант) матрицы}$ $-$ функция или числовая характеристика квадратной матрицы. Обозначается как det A, |A|.
\end{definition}
	
	Определитель n-мерной матрицы вычисляется как
	\begin{enumerate}
		\item |$a_{11}$| = $a_{11}$, при n = 1
		\item 
		$\begin{vmatrix}
			a_{11} & a_{12}\\
			a_{21} & a_{22}\\
		\end{vmatrix}$ = $a_{11}a_{22}$ - $a_{12}a_{21}$, при n = 2
		\item 
		$\begin{vmatrix}
			a_{11} & a_{12} & a_{13}\\
			a_{21} & a_{22} & a_{23}\\
			a_{31} & a_{32} & a_{33}\\
		\end{vmatrix}$ = $a_{11}
		\begin{vmatrix}
			a_{22} & a_{23}\\
			a_{32} & a_{33}\\
		\end{vmatrix}$ - $a_{12}
		\begin{vmatrix}
			a_{21} & a_{23}\\
			a_{31} & a_{33}\\
		\end{vmatrix}$ + $a_{13}
		\begin{vmatrix}
			a_{21} & a_{22}\\
			a_{31} & a_{32}\\
		\end{vmatrix}$ = $a_{11}a_{22}a_{33} + a_{12}a_{23}a_{31} + a_{13}a_{21}a_{32}$ - $a_{11}a_{23}a_{32}$ - $a_{12}a_{21}a_{33}$ - $a_{13}a_{22}a_{31}$, при n = 3
	\end{enumerate}
	
	\subsection{Решение систем линейных уравнений}
	
	\begin{definition}
		$\textit{Система линейных уравнений}$ $-$ система уравнений вида\\
		$\begin{cases}
			a_{11}x_1 + ... + a_{1n}x_n = b_1\\
			...\\
			a_{m1}x_1 + ... + a_{mn}x_n = b_m\\
		\end{cases}$
	\end{definition}
	
	A = 
	$\begin{pmatrix}
		a_{11} & ... & a_{1n}\\
		... & ... & ...\\
		a_{m1} & ... & a_{mn}\\
	\end{pmatrix}$ $-$ матрица системы\\
	\newline
	b = 
	$\begin{pmatrix}
		b_{1}\\
		...\\
		b_{m}\\
	\end{pmatrix}$ $-$ столбец свободных членов\\
	\newline
	(A | b) = 
	$\begin{pmatrix}
		a_{11} & ... & a_{1n} & | b_1\\
		... & ... & ... & | ...\\
		a_{m1} & ... & a_{mn} & | b_{m}\\
	\end{pmatrix}$ $-$ расширенная матрица системы\\
	
	\newpage
	
	$\textit{Совместная}$ система имеет хотя бы одно решение, иначе система считается $\textit{несовместной}$.
	
	Система называется $\textit{однородной}$, если 
	$\begin{pmatrix}
		b_1\\
		...\\
		b_m\\
	\end{pmatrix}$ = 
	$\begin{pmatrix}
		0\\
		...\\
		0\\
	\end{pmatrix}$, иначе $\textit{неоднородной}$.
	
	\begin{theorem}
		Однородная система всегда совместна.
	\end{theorem}
	\begin{proof}
		Если $x_1 = x_2 = ... = x_n = 0$, то система имеет решение.
	\end{proof}
	
	$\textbf{Правило Крамера (для двухмерной матрицы)}$. Система
	$\begin{cases}
		a_{11}x_1 + a_{12}x_2 = b_1\\
		a_{21}x_1 + a_{22}x_2 = b_2\\
	\end{cases}$ имеет единстывенное решение $\longleftrightarrow$ det
	$\begin{pmatrix}
		a_{11} & a_{12}\\
		a_{21} & a_{22}\\
	\end{pmatrix}$ $\ne$ 0.\\
	
	Решения могут быть найдены по $\textit{формуле Крамера}$:
	
	\begin{center}
	$	x_1 =\frac{\Delta_1}{\Delta}$, $x_2 = \frac{\Delta_2}{\Delta}$, где
	\end{center}
	$\Delta$ $-$ определитель матрицы системы;\\
	$\Delta_1$ = 
	$\begin{vmatrix}
		b_1 & a_{12}\\
		b_2 & a_{22}\\
	\end{vmatrix}$;
	$\Delta_2$ = 
	$\begin{vmatrix}
		a_{11} & b_1\\
		a_{21} & b_2\\
	\end{vmatrix}$.\\
	
	\newline
	
	\textit{Свойства детерминанта:}
	\begin{itemize}
		\item det $A^T$ = det A
		\item определитель треугольной (и диагональной) матрицы равен производонию диагональных элементов
		\item det E = 1
		\item если поменять местами две строки, то детерминант умножится на -1
		\item если в матрице есть нулевая строка, то det A = 0
	\end{itemize}
		
	\subsection{Умножение матриц}
	
	Умножение определено только для матриц с количеством столбцов в первой, равным количеству строк во второй.
	
	\begin{center}
		$\begin{pmatrix}
			a_1 & ... & a_n\\
	    \end{pmatrix}\cdot 
		\begin{pmatrix*}
			b_1\\
			...\\
			b_n\\
		\end{pmatrix*}$ = 
		$\begin{pmatrix*}
			a_1 b_1 + ... + a_n b_n\\
		\end{pmatrix*}$
	\end{center}
	
	Матрица C, являющаяся результатом умножения матрицы $A_{n \times m}$ на матрицу $B_{m \times k}$, имеет размеры $n \times k$, причём $c_{ij} = \sum_{S = 1}^{m} a_{is}b_{sj}$.\\
	
	Если AB = BA, то такие матрицы A и B называются $\textit{перестановочными}$. Так, единичная матрица является перестановочной с любой другой матрицей подходящего размера.
	
	\begin{theorem}
		Если определено А(ВС), то определено и (AB)C, а результаты этих операций равны.
	\end{theorem}
	\begin{proof}
		Пусть матрицы А, В и С имеют размеры соответственно $m \times n, n \times p$ и $p \times q$. Тогда умножение для них определено и выполняется\\
		
		$\begin{cases}
			A \cdot (B \cdot C) = A \cdot (BC)_{n \times q} = (ABC)_{m \times q}\\
			(A \cdot B) \cdot C = (AB)_{m \times p} \cdot C = (ABC)_{m \times q}\\
		\end{cases}$ $\longrightarrow$ мы доказали равенство размеров.\\
		\newline
		Докажем равенство элементов.\\
		
		\newline
		
		$(AB)_{ij} = \sum_{k = 1}^{n} a_{ik}b_{kj}$, $(AB)C_{il} = \sum_{j = 1}^{p} (AB)_{ij} \cdot c_{jl} = \sum_{l = 1}^{p}(\sum_{k = 1}^{n} a_{ik}b_{k_j}) \cdot c_{jl} = \sum_{k = 1}^{n} a_{ik} \sum_{j = 1}^{p} b_{kj}c_{jl} = \sum_{k = 1}^{n} a_{ik} \cdot (BC)_{kl} = A(BC)_{il}$
	\end{proof}
