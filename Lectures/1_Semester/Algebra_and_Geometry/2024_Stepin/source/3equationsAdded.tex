\begin{definition}
	Пусть $\vec{a}, \vec{b}, \vec{c}\in V_3$, тогда двойное векторное произвденение:
	\([\vec{a}, [\vec{b}, \vec{c}]]\)
\end{definition}
\begin{theorem}
	(Тождество БАЦ-ЦАБ). \([\vec a, [\vec b, \vec c]] = \vec b(\vec a, \vec c) - \vec c(\vec a, \vec b)\)
\end{theorem}

\begin{proof}
	Выделим правый ОНБ так, что \(\vec a || \vec e_1, \vec e_2: (\vec a, \vec b, \vec e_2)\) - компаларная система, \(\vec e_3 = [\vec e_1, \vec e_2]\), тогда \(\vec a\underset{\mathfrak{E}}{\Longleftrightarrow} \begin{pmatrix}
	\alpha \\ 0 \\ 0
	\end{pmatrix}, 
	b\underset{\mathfrak{E}}{\Longleftrightarrow} \begin{pmatrix}
		\beta_1 \\ \beta_2 \\ 0
	\end{pmatrix}, 
	c\underset{\mathfrak{E}}{\Longleftrightarrow} \begin{pmatrix}
		\gamma_1 \\ \gamma_2 \\ \gamma_3
	\end{pmatrix}
	\). \newline Тогда \([\vec a, [\vec b, \vec c]] = \begin{pmatrix}
		0 \\ -\alpha(\beta_1\gamma_2-\beta_2\gamma_1) \\ -\alpha\beta_2\gamma_3
	\end{pmatrix}\). Также убеждаемся, что \(\vec b(\vec a, \vec c) - \vec c(\vec a, \vec b) = \begin{pmatrix}
	0 \\ -\alpha(\beta_1\gamma_2-\beta_2\gamma_1) \\ -\alpha\beta_2\gamma_3
\end{pmatrix}\). Тогда получаем, что левая и правая части дают один и тот же вектор. А так как вектора не зависят от выбранного базиса, то получаем верное утверждение для всех троек векторов.
\end{proof}
\begin{corollary}
	\([\vec a, [\vec b, \vec c]] + [\vec b, [\vec a, \vec c]] + [\vec c, [\vec a, \vec b]] = 0\)
\end{corollary}
\section{Понятие уравнения множества. Задание прямой на плоскости}
\begin{definition}
	Уравнение множества \(M\subset V_i\) называется высказывание, верное $\forall x\in M$ и неверно, что $\forall x\in V_i \\ M$ 
\end{definition}
\(V_2\) с фиксированным ДСК. \(l || \vec a\), \(\vec a\) - направляющий вектор прямой. \(\vec{OX} = \vec{OX_0}+\vec{X_0X} = \vec{OX_0} + \lambda \vec{a}\). Тогда \(\vec{r} = \vec{r_0} + t\cdot\vec{a}, t\in \mathbb{R}\) - векторное уравнение прямой.(в физике наызвается уравнение равномерного прямолинейного движения).
\(\vec r \longleftrightarrow \begin{pmatrix}
	x \\ y
\end{pmatrix}, 
\vec{r_0}\longleftrightarrow \begin{pmatrix}
	x_0 \\ y_0
\end{pmatrix},
\vec a\longleftrightarrow \begin{pmatrix}
	\alpha_1 \\ \alpha_2
\end{pmatrix} \Longleftrightarrow \left\{\begin{gathered}
x = x_0 + \alpha_1t \\
y = y_0 + \alpha_2t
\end{gathered}\right.
\) - параметрическое уравнение прямой в плоскоти. \newline
Если выразить t в данной системе, то получим каноническое уравение прямой:\newline \(\dfrac{x - x_0}{\alpha_1} = \dfrac{y - y_0}{\alpha_2}\). Аналогично каноническое уравнение прямой в пространстве\newline\(\dfrac{x - x_0}{\alpha_1} = \dfrac{y - y_0}{\alpha_2} = \dfrac{z - z_0}{\alpha_3}\)
\begin{note}
	Если $\alpha_2 = 0$, то считаем, что числитель приравнивается к нулю.
\end{note}
$\alpha_2(x - x_0) - \alpha_1(y - y_0) = 0 \Longleftrightarrow Ax+By+C = 0, A = \alpha_2, B = - \alpha_1$ - общее уравнение прямой, где \(\vec a = \begin{pmatrix}
	-B \\ A
\end{pmatrix}\) - направляющий вектор.
\begin{proposition}
	Пусть l задано общим уравнением \(X_0\begin{pmatrix}
		x_0 \\ y_0
	\end{pmatrix} \in l, X_1\begin{pmatrix}
	x_1 \\ y_1
	\end{pmatrix} \in l\Longleftrightarrow A(x_1-x_0) + B(y_1-y_0) = 0\)
\end{proposition}
\begin{proof}
	\(\left\{
	\begin{gathered}
		Ax_0 + By_0+C = 0 \\
		Ax_1 + By_1 + C = 0
	\end{gathered}
	\right. \Longrightarrow A(x_1 - x_0) + B(y_1 - y_0 ) = 0\). Наоборот аналогично верно.
\end{proof}
\begin{corollary}
	Вектор \(\vec b\begin{pmatrix}
		\beta_1 \\ \beta_2
	\end{pmatrix}\) является направляющим вектором прямой \(Ax + By + C = 0\), тогда и только тогда, когда \(A\beta_1+B\beta_2 = 0\), это вытекает из того, что если отложить вектор $\vec{b}$ от точки \(X_0 \in l\), то конец \(X_1\) также лежит на прямой.
\end{corollary}
\begin{theorem}
	Пусть \(l: Ax+By+C = 0\), тогда любой направляющий вектор коллинеарен вектору \(\begin{pmatrix}
		-B \\ A
	\end{pmatrix}\), а в качестве начальной точки можно взять \(X_0\begin{pmatrix}
	-\dfrac{AC}{A^2+B^2} \\ -\dfrac{BC}{A^2+B^2}
	\end{pmatrix}\)
\end{theorem}
\begin{proof}
	\begin{enumerate}
		\item \(\lambda \begin{pmatrix}
			-B \\ A
		\end{pmatrix}\) - направляющий вектор \(\Longleftrightarrow A(-\lambda B)+B(\lambda A) \equiv 0\)
		\item \(\dfrac{-A^2C}{A^2+B^2} - \dfrac{B^2C}{A^2+B^2}+C = \dfrac{-(A^2+B^2) + (A^2+B^2)C}{A^2+B^2} = 0\)
	\end{enumerate}
\end{proof}
\begin{corollary}
	Все рассмотренные свойства задания прямой эквивалентны.
\end{corollary}
Рассмотрим случай ПДСК с \((O, \mathfrak{E})\)
\begin{proposition}
	\(l: Ax+By+C = 0\). Тогда вектор \(\vec b\begin{pmatrix}
		A \\ B
	\end{pmatrix}\perp l \Longrightarrow \vec{a}\begin{pmatrix}
	-B \\ A
	\end{pmatrix}: (\vec n, \vec a) = -AB + AB = 0\Longleftrightarrow \vec a\perp \vec n\)
\end{proposition}
\begin{definition}
	Вектор \(\vec n \begin{pmatrix}
		A \\ B
		\end{pmatrix}\) называется вектором нормали к прямой
\end{definition}
\subsection{Уравнение прямой с угловым коэффициентом}
Пусть \(l \not|| Oy X \begin{pmatrix}
	x \\ y
\end{pmatrix}\) лежит на прямой \(\Longleftrightarrow \tg \phi = \dfrac{y - b}{x}\Longleftrightarrow y = \underbrace{\tg\phi}_{k} x + b\) \newline
\(Ax+By+C = 0 \overset{B\ne0}{\Longleftrightarrow} y = -\dfrac{A}{B}x - \dfrac{C}{B}\) 
\begin{proposition}
	\(y_1, y_2 \) - прямые с угловым коэффициентами \(k_1, k_2\) параллельны тогда и только тогда, когда \(k_1 = k_2\) \newline
	прямые \(l_1, l_2\) - прямые, заданные общим уравнением прямой \(A_1x+B_1y+C_1 = 0, A_2x+B_2y+C_2 = 0\) соответственно параллельны \(\Longleftrightarrow \begin{vmatrix}
		A_1 & B_1 \\ A_2 & B_2
	\end{vmatrix} = 0\)
\end{proposition}
\begin{proof}
	\begin{enumerate}
		\item Прямые параллельны \(\Longleftrightarrow\) образуют одинаковые углы с прямой OX, т. е. \(k_1 = \tg\phi = k_2\)
		\item возьмем векторы нормали \(\vec n_1, \vec n_2, l_1 || l_2 \Longleftrightarrow \vec n_1 || \vec n_2 \Longleftrightarrow S(\vec n_1, \vec n_2) = 0 = \begin{vmatrix}
			A_1 & B_1 \\
			A_2 & B_2
		\end{vmatrix}\)
	\end{enumerate}
\end{proof}
\begin{proposition}
	\begin{enumerate}
		\item \(\left\{
		\begin{gathered}
			y = k_1x+b \\
			y = k_2x+b_2
		\end{gathered}
		\right.: l_1 \perp l_2 \Longleftrightarrow k_1k_2 = -1 \Longleftrightarrow A_1A_2 + B_1B_2 = 0\)
		\item \(l_1 \perp l_2 \Longleftrightarrow A_1A_2 + B_1B_2 = 0\)
	\end{enumerate}
\end{proposition}
\begin{proof}
	\begin{enumerate}
		\item \(\phi_1 = \phi_2 + \dfrac{\pi}{2}, \tg\phi_1 = \tg(\phi_2 + \dfrac{\pi}{2}) = -\ctg(\phi_2)\Longleftrightarrow k_1 = -\dfrac{1}{k_2}\Longleftrightarrow k_1k_2 = -1\)
		\item \(l_1 \perp l_2 \Longleftrightarrow \vec n_1 \perp \vec n_2, \vec n_1\begin{pmatrix}
			A_1 \\ B_1
		\end{pmatrix}, \vec n_2 \begin{pmatrix}
		A_2 \\ B_2
		\end{pmatrix}. (\vec n_1, \vec n_2) = A_1A_+B_1B_2 = 0 \Longleftrightarrow \vec n_1 \perp \vec n_2\)
	\end{enumerate}
\end{proof}
\begin{proposition}
	\(\left\{
	\begin{gathered}
		l_1: A_1x+B_1y+C_1 = 0 \\
		l_2: A_2x + B_2y + C_2 = 0
	\end{gathered}
	\right.\)
	\newline
	\begin{enumerate}
		\item Пересекаются в одной точке $\Longleftrightarrow \begin{vmatrix}
			A_1 & B_1 \\ A_2 & B_2
		\end{vmatrix} \ne 0$
		\item Параллельны $\begin{vmatrix}
			A_1 & B_1 \\ A_2 & B_2
		\end{vmatrix} = 0$
		\item совпадают, когда уравнения пропорциональны
	\end{enumerate}
\end{proposition}
\begin{proof}
	\begin{enumerate}
		\item Получается система из двух уравнений, то есть имеет единственное решение(прямые пересекаются) $\Longleftrightarrow \begin{vmatrix}
			A_1 & B_1 \\ A_2 & B_2
		\end{vmatrix} \ne 0$
		\item Аналогично первому получаем, что \(\begin{vmatrix}
			A_1 & B_1 \\ A_2 & B_2
		\end{vmatrix} = 0\)
		\item Очевидно, что прямые совпадают, так как одно можно получить умножением другого на коэффициент.
	\end{enumerate}
\end{proof}
\begin{definition}
	Полуплоскостью, определяемой прямой \(l\) и нормалью \(\vec n\) называется множества точек X плоскости, так что вектор \(\vec{X_0X}\) образует с вектором $\vec{n}$ угол \(\phi\le \dfrac{\pi}{2}\), то есть \((\vec{X_0X}, \vec n)\ge0\Longleftrightarrow A(x-x_0)+B(y-y_0)\ge 0 \Longleftrightarrow Ax+By+C\ge 0\) 
\end{definition}
\subsection{Пучок прямых на плоскости}
\(V_2\) в фиксированной ДСК
\begin{definition}
	Пучком пересекающихся прямых на плоскости называется множество всех прямых на плоскости, проходящих через фиксированную точку
\end{definition}
\begin{definition}
	Пучком параллельных прямых на плоскости называется множество всех прямых, параллельных некоторой фиксированной прямой
\end{definition}
\begin{theorem}
	Пусть данные 2 различные прямые на плоскости \(l_1, l_2\) \(\left\{\begin{gathered}
		l_1: A_1x+B_1y+C_1 = f_1(x,y) = 0 \\
		l_2: A_2x+B_2y+C_2 = f_2(x,y) = 0
	\end{gathered}\right.\), тогда пучок прямых, задаваемый(порожденный) прямыми \(l_1, l_2\) состоит из тех и только тех прямых, координаты которых удовлетворяют уравнению \[\alpha f_1(x,y)+\beta f_2(x,y) = 0, \alpha, \beta\in R, \alpha^2+\beta^2\ne0\]
\end{theorem}
\begin{note}
	Данное уравнение не тождественно равно 0
\end{note}
\begin{proof}
	\begin{enumerate}
		\item Пусть \(l_1 \cap l_2 = X_0\). Тогда \(f_1(x_0,y_0) = 0, f_2(x_0, y_0) = 0\), то есть $\forall l$, удовлетворяющих уравнению \(\alpha f_1(x,y)+\beta f_2(x,y) = 0\) проходят через точку \(X_0\). \newline
		Теперь наоборот если l такова, что принадлежит пучку, порожденному \(l_1, l_2\), покажем, что \(\exists \alpha, \beta, \alpha^2+\beta^2\ne 0\) и \(\alpha f_1(x,y)+\beta f_2(x,y) = 0\). Пусть \(X \in l, X\ne X_0\), тогда возьмем \(\alpha = f_2(X), \beta = -f_1(X)\). Заметим, что если подставить X, то получим верное равенство, при этом \(\alpha^2+\beta^2\ne0\), иначе прямые \(l_1, l_2\) проходили бы ещё и через точку \(X\ne X_0\). То есть прямая \(l'\) удовлетворяющая этому уравнению проходит через \(X_0, X\), то есть совпадает с l.
		\item Пусть \(l_1 || l_2\Longleftrightarrow \vec n_1 || \vec n_2\). Пусть прямая l удовлетворяет уравнению \(\alpha f_1(x,y)+\beta f_2(x,y) = 0.\) Тогда \(\vec n_l = \alpha\vec n_1 + \beta\vec n_2 || \vec n_1 || \vec n_2\), то есть прямая l параллельна порождающим пучок прямым. \newline
		Наоборот, пусть l принадлежит пучку. Пусть \(X\in l:\alpha = f_2(X), \beta = -f_1(X): \alpha f_1(x,y)+\beta f_2(x,y) = 0\). Тогда подставив \(X\) в уравнение получаем, что l проходит через X и параллельна прямым \(l_1, l_2\). 
	\end{enumerate}
\end{proof}
\begin{note}
	Уравнение \(\alpha f_1(x,y)+\beta f_2(x,y) = 0\) называется уравнением пучка прямых, порожденных прямыми \(l_1, l_2\).
\end{note}
\section{Приложение к планиметрии}
\begin{enumerate}
	\item Расстояние от точки до прямой(\(\rho(X, l)\) - обозначение)
	\begin{proposition}
		Пусть прямая l в ПДСК задана общим уравнением \(Ax+By+C = 0\) и \(X\underset{(O, \mathfrak{E})}{\longleftrightarrow}\begin{pmatrix}
			x \\ y
		\end{pmatrix}\), тогда \(\rho(X, l) = \dfrac{|Ax+By+C|}{\sqrt{A^2+B^2}}\)
	\end{proposition}
	\begin{proof}
		Пусть вектор \(\vec a\begin{pmatrix}
			-B \\ A
		\end{pmatrix}\) направляющий вектор, тогда \(\rho(X, l) = h = \dfrac{S(\vec a, \vec{X_0X})}{|a|} = \dfrac{{A(x-x_0)+B(y-y_0)}}{\sqrt{A^2+B^2}} = \dfrac{|Ax+By+C|}{\sqrt{A^2+B^2}}\)
	\end{proof}
	\item Угол между пересекающимися прямыми
	\begin{definition}
		Углом между 2 пересекающимися прямыми называется наименьший из двух смежных углов, порожденных двумя прямыми \(l_i: A_ix+B_iy+C_i = 0, i= 1,2\)
	\end{definition}
	\begin{proposition}
		Угол между этими двумя пересекающимися прямыми может быть вычислен как \(\dfrac{|A_1A_2+B_1B_2|}{\sqrt{A_1^2+B_1^2}\sqrt{A_2^2+B_2^2}}\)
	\end{proposition}
	Возьмем нормали \(\vec n_1\begin{pmatrix}
		A_1 \\ B_1
	\end{pmatrix}, \vec n_2\begin{pmatrix}
		A_2 \\ B_2
	\end{pmatrix},
	 \psi = \angle(\vec n_1  \vec n_2), \phi + \psi + \pi = 2\pi \Longrightarrow \phi = \pi - \psi\). 
	 Тогда получаем, что \(\cos\phi = |\cos\phi| = |\cos\psi| = \dfrac{|(\vec n_1, \vec n_2)|}{|\vec n_1||\vec n_2|} =
	  \dfrac{|A_1A_2+B_1B_2|}{\sqrt{A_1^2+B_1^2}\sqrt{A_2^2+B_2^2}}\)
	  \begin{corollary}
	  	\(l_1 \perp l_2\Longleftrightarrow A_1A_2+B_1B_2 = 0\)
	  \end{corollary}
\end{enumerate}
\section{Плоскость в пространстве}
\(V_3\) В ДСК \((O, \mathfrak{E})\). Пусть есть точка \(X_0\begin{pmatrix}
	x_0 \\ y_0 \\ z_0
\end{pmatrix}\) и направляющие векторы плоскости \(\vec a, \vec b\) - то есть векторы, коллинеарные плоскости, но не коллинеарные между собой. Тогда уравнение плоскости \((\vec r - \vec r_0, \vec a, \vec b) = 0 \Longleftrightarrow \vec r - \vec r_0 - \overline{X_0X}\) - лежит в плоскости. Тогда \newline\(\overline{X_0X} = s\cdot\vec a + t\cdot\vec b, s, t \in \mathbb{R}\), \newline\(\vec r = \vec r_0 + s\cdot\vec a + t\cdot\vec b, s, t \in \mathbb{R}\) - векторное параметрическое уравнение. \newline
 \(\vec a \underset{(O, \mathfrak{E})}{\longleftrightarrow}\begin{pmatrix}
 	a_1 \\ a_2 \\ a_3
 \end{pmatrix},
 \vec b \underset{(O, \mathfrak{E})}{\longleftrightarrow}\begin{pmatrix}
 	b_1 \\ b_2 \\ b_3
 \end{pmatrix},
 \vec X \underset{(O, \mathfrak{E})}{\longleftrightarrow}\begin{pmatrix}
 	x \\ y \\ z
 \end{pmatrix} \Longrightarrow \left\{
 \begin{gathered}
 	x = x_0 +sa_1 + tb_1 \\
 	y = y_0 +sa_2+tb_2 \\
 	z = z_0 +sa_3+tb_3
 \end{gathered}
 \right.\)
\newline
\((\vec r - \vec r_0, \vec a, \vec b) = 0 \Longleftrightarrow \begin{vmatrix}
	x-x_0 & y-y_0 & z-z_0 \\
	a_1 & a_2 & a_3 \\
	b_1 & b_2 & b_3
\end{vmatrix} = 0.\) Если раскрыть определитель по формуле, то получится уравнение вида \(Ax+By+Cz+D = 0\) - общее уравнение плоскости.
\begin{proposition}
	Пусть \(X_0\begin{pmatrix}
		x_0 \\ y_0 \\ z_0
	\end{pmatrix}\in \alpha\) - плоскости: \(Ax+By+Cz+D=0\) и \(X_1\begin{pmatrix}
	x_1 \\ y_1 \\ z_1
	\end{pmatrix}\in \alpha\), тогда \(A(x_1-x_0)+B(y_1-y_0)+C(z_1-z_0) = 0\)
\end{proposition}
\begin{proof}
	Аналогично прямой
\end{proof}
\begin{corollary}
	\(\vec c \underset{(O, \mathfrak{E})}{\longleftrightarrow}
	\begin{pmatrix}
		\alpha \\ \beta \\ \gamma
	\end{pmatrix} || \alpha \Longleftrightarrow A\alpha+B\beta+C\gamma = 0\)
\end{corollary}
\begin{proof}
	Пусть \(X_0 \) - начало \(\vec c\), а \(X_1 \) - конец \(\vec c\), тогда \(\vec c || \alpha \Longleftrightarrow \overline{X_0X}\in \alpha \Longrightarrow A(x_1-x_0)+B(y_1-y_0)+C(z_1-z_0) = 0 \Longrightarrow A\alpha+B\beta+C\gamma = 0\)
\end{proof}
\begin{corollary}
	Пусть для определенности в \(Ax+By+Cz+D=0, A\ne0 \Longrightarrow \text{ вектора }\begin{pmatrix}
		-B \\ A \\ 0
	\end{pmatrix}, \begin{pmatrix}
	-C \\ 0 \\ A
	\end{pmatrix}\) - ненулевые, неколлинеарные и параллельные \(\alpha\). То есть они могут быть выбраны в качестве направляющих векторов, а как начальную можно взять \(X_0\begin{pmatrix}
	-\dfrac{D}{A} \\ 0 \\ 0
	\end{pmatrix}\)
\end{corollary}
\begin{corollary}
	Все приведенные выше способы задания плоскости эквивалентны
\end{corollary}
Пусть теперь мы находимся в ПДСК. \newline
\(A\alpha+B\beta+C\gamma = 0 \Longleftrightarrow (\alpha, \beta, \gamma)\begin{pmatrix}
	A \\ B \\ C
\end{pmatrix} = (\vec c, \vec n), \text{ где } \vec n \underset{(O, \mathfrak{E})}{\longleftrightarrow}\begin{pmatrix}
A \\ B \\ C
\end{pmatrix}\)
\begin{definition}
	Вектор $\vec{n}$ называется вектором нормали к плоскости $\alpha$
\end{definition}
\begin{proposition}
	В ПДСК вектор нормали $\vec{n}$ к $\alpha$ ортогонален любому вектору || $\alpha$ \newline
	Пусть теерь мы находим в ДСК. Тогда векторс координатами $\begin{pmatrix}
		A \\ B\\ C
	\end{pmatrix}$ называется сопутствующим вектором плоскости.
\end{proposition}
\begin{proposition}
	Плоскости \(\alpha_i: A_ix+B_iy+C_iz+D = 0, i= 1,2;\) с сопутствующими векторами \(\vec{n_1}, \vec{n_2}\) соответственно параллельны, если их сопутствующие векторы параллельны. Плоскости совпадают, если уравнения пропорциональны
\end{proposition}
\begin{proof}
	\begin{enumerate}
		\item Если уравнения пропорциональны, то множество решений одного совпадает с множеством решений другого.
		\item Пусть $\vec{n_1} || \vec{n_2}$ и уравнения непропорциональны, тогда \(\lambda = \dfrac{A_1}{B_1} = \dfrac{A_2}{B_2} = \dfrac{C_1}{C_2} \ne \dfrac{D_1}{D_2}\). Получаем \newline
		\(\lambda(A_2x+B_2y+C_2z+D_2) = A_1x+B_1y+C_1z+D_1 + (\lambda D_2 - D_1), \not\exists x_0: x_0\in \alpha_1\cap\alpha_2\Longrightarrow \alpha_1 || \alpha_2\)
		\item Пусть \(\vec n_1 \not || \vec n_2\). Докажем, что плоскости пересекаются. $\left\{
		\begin{gathered}
			A_1x+B_1y+C_1z+D_1 = 0 \\
			A_2x+B_2y+C_2z+D_2 = 0 \\ 
			z= z_0
		\end{gathered}
		\right.\Longrightarrow \left\{
		\begin{gathered}
			A_1x+B_1y = -(D_1 + C_1z_0) \\
			A_2x+B_2y = -(D_2 + C_2z_0)
		\end{gathered}
		\right.\Longrightarrow \begin{vmatrix}
			A_1 & B_1 \\
			A_2 & B_2
		\end{vmatrix}\ne 0,$ так как $\vec{n_1}\not||\vec{n_2}$. А значит существует единственная точка \(x_0, y_0, z_0\) удовлетворяющая системе и плоскости при этом не совпадают. Если же плоскости совпадают, то будут одинаковые сечения плоскостей \(z = 0\)
	\end{enumerate}
\end{proof}
%Added From Here
\begin{proposition}
	(ДСК). \(\pi_i: A_ix+B_iy+C_iz+D_i = 0, i = 1,2;\quad \vec n_i = 
	\begin{pmatrix}
		A_i \\ B_i \\ C_i
	\end{pmatrix}\) - сопутствующие вектора. Пусть \(\pi_1 \cap \pi_2 = l.\) Тогда за направляющий вектор прямой l можно взять вектор \(\vec u \underset{\mathfrak{E}}{\longleftrightarrow} (\begin{vmatrix}
	B_1 & C_1 \\ B_2 & C_2
	\end{vmatrix}, 
	\begin{vmatrix}
		C_1 & A_1 \\ C_2 & A_2
	\end{vmatrix}, 
	\begin{vmatrix}
		A_1 & B_1 \\ A_2 & B_2
	\end{vmatrix})\)
\end{proposition}
\begin{proof}
	Вектор \(\vec u \ne \vec 0\) по утверждению из предыдущей лекции \(\vec n_1 \ne||\vec n_2\).(\(\vec n_1 || \vec n_2\Longleftrightarrow \begin{pmatrix}
		A_1 \\ B_1 \\ C_1
	\end{pmatrix}= \lambda \begin{pmatrix}
	A_2 \\ B_2 \\ C_2
	\end{pmatrix}\)). Покажем, что \(\vec u || \vec \pi_i, \forall i = 1, 2;\)\newline
	\(A_i\begin{vmatrix}
		B_1 & C_1 \\ B_2 & C_2
	\end{vmatrix}+ B_i
	\begin{vmatrix}
		C_1 & A_1 \\ C_2 & A_2
	\end{vmatrix}+C_i\begin{vmatrix}
	A_1 & B_1 \\ A_2 & B_2
	\end{vmatrix}\overset{?}{=} 0 \Longleftrightarrow \begin{vmatrix}
	A_i & B_i & C_i \\ A_1 & B_1 & C_1 \\ A_2 & B_2 & C_2
	\end{vmatrix} = 0\Longleftrightarrow V(\vec n_1, \vec n_2, \vec n_3) = 0.\) Верно, то есть вектор подходит.
\end{proof}
\begin{note}
	В ПДСК \(\vec u = [\vec n_1, \vec n_2]\)
\end{note}

\begin{definition}
	Пучком пересекающихся плоскостей в пространстве называется множество плоскостей в пространстве, проходящее через фиксированную прямую в пространстве.
\end{definition}
\begin{definition}
	Пучком параллельных плоскостей в пространстве называется множество всех плоскостей в пространстве параллельно фиксированной плоскости.
\end{definition}
\begin{theorem}
	Пусть 2 различные плоскости \(pi_i\) заданы своими общими уравнениями\newline
	\(\begin{gathered}
		\pi_1: f_1(x,y,z) = A_1x+B_1y+C_1z+D_1 = 0\\
		\pi_2: f_2(x,y,z) = A_2x+B_2y+C_2z+D_2 = 0
	\end{gathered}\Longrightarrow\) пучок, порожденный \(\pi_1, \pi_2\) состоит из тех и только тех плоскостей, координаты точек которых удовлетворяют уравнению
	\begin{equation}\label{1}
		\alpha f_1(x,y,z)+\beta f_2(x,y,z) = 0, \alpha, \beta \in \mathbb{R}, \alpha^2+\beta^2\ne0
	\end{equation}
\end{theorem}


\begin{proof}
	\begin{enumerate}
		\item Пусть $\pi$ задается уравнением \ref{1} с \(\alpha^2+\beta^2\ne0\). \newline Пусть плоскости пересекаются по прямой l. Заметим, что \(f_1(l) = f_2(l) = 0\). Тогда $\alpha f_1(l) + \beta f_2(l) = 0$, то есть $\pi$ принадлежит пучку плоскостей. \newline
		Пусть плоскости параллельны, тогда \(\vec n_\pi = \alpha\vec n_1 + \beta\vec n_2 || n_1 || n_2\Longleftrightarrow \pi || \pi_1 || \pi_2\), то есть $\pi$ принадлежит пучку параллельных плоскостей.
		\item Пусть $\pi$ принадлежит пучку плоскостей, порожденному $\pi_1, \pi_2$. Тогда покажем, что $\pi$ можно задать через уравнение \ref{1}. Пусть \(X\ne\in l\in \pi, \not\in \pi_1,\not\in\pi_2\). Возьмем \(\alpha=f_2(X), \beta = -f_1(X)\). Тогда покажем, что \(f_2(X)f_1(x,y,z) -f_1(X)f_2(x,y,z) = 0\) - уравнение плоскости $\pi$. Действительно, если подставить X и l, то мы получим верное равенство. При этом если есть $\pi'$, удовлетворяющее этому равенству, то оно тоже проходит через X и l, то есть совпадает с $\pi$.
 	\end{enumerate}
\end{proof}
\subsection{Связка плоскостей}
\begin{definition}
	Множество всех плоскостей в пространстве \(V_3\), проходящих через фиксированную точку наызвается связкой плоскостей, а сама фиксированная точка - центром связки.\newline
	Связку можно задать с помощью фиксированой точки или через 3 плоскости в пространстве \(V_3\), не находящиеся в одном пучке.
\end{definition}
\begin{theorem}
	Пусть связка плоскостей в пространстве задается 3 плоскостями \(\pi_i: A_ix+B_iy+C_iz+D_i = f_i(x,y,z) = 0, i = 1,2,3\), пересекающиеся в одной точке \(X_0\). Тогда связка плоскостей состоит из тех и только тех плоскостей, координаты точек которых удовлетворяют уравнению \(\alpha f_1(x,y,z) + \beta f_2(x,y,z)+\gamma f_3(x,y,z) = 0\).
\end{theorem}
\begin{proof}
	Здесь будет приведена идея доказательства, так как данная тема не входит в программу курса. Составим систему из 3 уравнений плоскости, которая имеет единственное решение(так как они пересекаются в одной точке). Тогда определитель \(\begin{vmatrix}
	A_1 & A_2 & A_3 \\ B_1 & B_2 & B_3 \\ C_1 & C_2 & C_3
	\end{vmatrix} \ne 0 \Longleftrightarrow (\vec n_1, \vec n_2, \vec n_3) \ne 0\longrightarrow\) система этих векторов некомпланарна в пространстве \(V_3\) образует базис в нем. Тогда \(\vec n = \alpha \vec n_1 + \beta \vec n_2 + \gamma \vec n_3\Longrightarrow \alpha f_1(x,y,z) + \beta f_2(x,y,z) + \gamma f_3(x,y,z) = 0\). Дальше можно довести самостоятельно.
\end{proof}

\begin{enumerate}
	\item (Формула расстояния от точки до плоскости в ПДСК). \((\vec r - \vec r_0, \vec n) = 0. \rho(X,\pi) = |\pr_{\vec n}(\overline{X_0X})| = |\dfrac{(\overline{X_0X}, \vec n)}{(\vec n, \vec n)}\vec n| = \dfrac{(|\overline{X_0X}, \vec n)|}{|\vec{n}|} = \dfrac{|(\vec r_x - \vec r_0, n)|}{|\vec n|}\)
	\item Пусть \(\pi: Ax+By+Cz+D = 0\), тогда \(X_0\underset{(O,\mathfrak{E})}{\longleftrightarrow}\begin{pmatrix}
	x_0 \\ y_0 \\ z_0
	\end{pmatrix}, 
	X\underset{(O,\mathfrak{E})}{\longleftrightarrow}\begin{pmatrix}
		x \\ y \\ z
	\end{pmatrix}\). \(\vec r_x - \vec r_0 \begin{pmatrix}
	x- x_0 \\ y - y_0 \\ z-z_0
	\end{pmatrix}: \rho = \dfrac{|A(x-x_0)+B(y-y_0)+C(z-z_0)|}{\sqrt{A^2+B^2+C^2}} = \dfrac{||Ax+By+Cz+D}{\sqrt{A^2+B^2+C^2}}\)
	\item Формула угла между двумя плоскостями
	\(\pi_i: A_ix+B_iy+C_iz+D_i = 0, i = 1, 2; \vec n_i\begin{pmatrix}
		A_i \\ B_i \\ C_i
	\end{pmatrix}\)
	\begin{definition}
		За угол между двумя плоскостями принимается линейный угол между двумя прямыми, который образуется при пересечении $\pi_1, \pi_2$ плоскостью \(\pi\), перпендикулярной прямой пересечения \(\pi_1\cap \pi_2 = l\). Тогда \(\cos \phi = |\cos\angle(\vec n_1, \vec n_2)| = \dfrac{|(\vec n_1, \vec n_2)|}{|\vec n_1||\vec n_2|}\)
	\end{definition}
\end{enumerate}
\section{Прямая в пространстве}
\(\vec a\) - направляющий вектор(\(\vec a\begin{pmatrix}
	\alpha_1 \\ \alpha_2 \\ \alpha_3
\end{pmatrix}\)), если $\vec{a}\ne0, \vec a || l, \overline{X_0X} = \vec at \Longrightarrow \vec r = \vec r_0 + \vec at$.\newline В ДСК \(\left\{
\begin{gathered}
	x = x_0 + \alpha_1t \\
	y = y_0 + \alpha_2t\\
	z = z_0 + \alpha_3t
\end{gathered}
\right.\) - координатное параметрическое уравнение.
Исключим t и получим \(t = \dfrac{x-x_0}{\alpha_1} = \dfrac{y - y_0}{\alpha_2} = \dfrac{z - z_0}{\alpha_3}\) - каноническое уравнение прямой в пространстве.
\begin{note}
	Если \(\alpha_i\) = 0, то соответствующий ей числитель тоже должен быть равен 0.
\end{note}
\begin{proposition}
	Прямая \(\dfrac{x-x_0}{\alpha_1} = \dfrac{y - y_0}{\alpha_2} = \dfrac{z - z_0}{\alpha_3}\) лежит в плоскости $\pi: Ax+By+Cz+D = 0\Longleftrightarrow\left\{\begin{gathered}
		Ax_0+By_0+Cz_0+D = 0\qquad (1)\\
		A\alpha_1 + B\alpha_2 + C\alpha_3 = 0 \qquad (2)
	\end{gathered}\right.$
\end{proposition}
\begin{proof}
	\begin{enumerate}
		\item Пусть \(l\subset \pi \Longrightarrow (1)\), так как \(x_0\in \pi, \vec a||\pi \Longrightarrow (2)	\)
		\item Пусть верна система, тогда из условия (1) следует, что \(X_0\in\pi\). (2) \(\Longrightarrow \vec a || \pi \Longrightarrow l\subset \pi\).
	\end{enumerate}
\end{proof}
\begin{proposition}
	Прямые \(\l_i: \vec r = \vec r_0 + \vec a_it, i = 1, 2\) лежат в одной плоскости тогда и только тогда, когда векторы \(\vec a_1, \vec a_2, \vec r_2 - \vec r_1\) - компланарны
\end{proposition}
\begin{proof}
	\(\Longrightarrow\) Очевидно, что \(\vec r_2-\vec r_1 \subset \pi\).\newline
	\(\Longleftarrow\) Пусть \(\vec a_1, \vec a_2, \vec r_2 - \vec r_1\) компланарны. Если \(\vec a_1 || \vec a_2\), то очевидно, что \(l_1, l_2 \subset \pi\). \newline
	Пусть \(\vec a_1 \ne || \vec a_2\). Построим \(\pi\), проходящую через \(X_1\) с направляющими векторами \(\vec a_1, \vec a_2\). Так как они компланарны с \(\vec r_2 - \vec r_1\), то \(X_2\in\pi\Longrightarrow \overline{X_1X_2}\subset \pi \Longrightarrow l_1, l_2\subset \pi\).
\end{proof}
\begin{corollary}
	Прямые \(\vec r = \vec r_i + \vec a_it, i = 1, 2\) лежат в одной плоскости \(\Longleftrightarrow (\vec a_1, \vec a_2, \vec r_2 - \vec r_1) = 0\)
\end{corollary}
\begin{corollary}
	Прямые \(l_1, l_2\) пересекаются в одной точке \(\Longleftrightarrow \left\{\begin{gathered}
		(\vec a_1, \vec a_2, \vec r_2 - \vec r_1) = 0 \\
		\vec a_1 \not||\vec a_2
	\end{gathered}\right.\Longleftrightarrow\newline \left\{\begin{gathered}
	(\vec a_1, \vec a_2, \vec r_2 - \vec r_1) = 0 \\
	[\vec a_1, \vec a-2]\ne0
	\end{gathered}\right.\)
\end{corollary}
\begin{corollary}
	Ппямые \(l_1||l_2 \Longleftrightarrow \vec a_1 || \vec a_2 \Longleftrightarrow  [\vec a_1, \vec a_2] = 0\).
\end{corollary}
\begin{corollary}
	Прямые \(l_1 = l_2 \Longleftrightarrow \vec a_1 || \vec a_2 || \vec r_2 - \vec r_1\).
\end{corollary}

\begin{enumerate}
	\item Формула угла между прямыми
	\begin{definition}
		Углом между \(l_1, l_2\) называется наименьший из смежных углов, образованный \(\Tilde{l_1}, \Tilde{l_2}\) 
	\end{definition}
	\(\cos \phi = |\cos(\vec a_1, \vec a_2)| = \dfrac{|(\vec a_1, \vec a_2)}{|\vec a_1||\vec a_2|}\)
	\item Расстояние от точки до прямой в пространстве. \(X(\vec r_X), l: \vec r = \vec r_0 + \vec at\). \(\rho(X, l) = h = \dfrac{S_{(\vec a, \overline{X_0X})}}{|\vec a|} = \dfrac{|[\vec r - \vec r_0, \vec a]|}{|\vec a|}\)
	\item Формула расстояния между двумя скрещивающимися прямыми. \(\vec r = \vec r_i +\vec a_it, t\in\mathbb{R}, i = 1, 2; \vec a_1 \ne || \vec a_2\). В этом случае всегда существуют \(\pi_1, \pi_2: \pi_1||\pi_2, l_1\subset\pi_1, l_2\subset\pi_2, \rho(l_1, l_2) \overset{def}{\equiv} \rho(\pi_1, \pi_2) = h = \dfrac{|V(\vec a_1, \vec a_2, \overline{X_0X})|}{|S(\vec a_1, \vec a_2)|} = \dfrac{|(\vec r- \vec r_0, \vec a_1, \vec a_2)|}{|[\vec a_1, \vec a_2]|}\).
\end{enumerate}

