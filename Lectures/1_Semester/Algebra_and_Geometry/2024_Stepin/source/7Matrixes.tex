\subsection{Метод Гаусса}
\(AX = b, \Tilde A = (A|b)\) - расширенная матрица.\newline
Приведем матрицу \(\Tilde A\) к ступенчатому виду \(\Tilde A_{\text{ступ}}\)
\begin{enumerate}
	\item В \(\Tilde A_{\text{ступ}}\) есть лидер в столбце свободных членов. В таком случае очевидно, что система несовместная, так как ЛК нулей не может давать ненулевой член.
	\item В \(\Tilde A_{\text{ступ}}\) такого лидера нет. В этом случае СЛУ совместна. Пусть лидеры в \(\Tilde A_{\text{ступ}}\): \(a_{1k_1}, \ldots, a_{rk_r}\).
	\begin{definition}
		Неизвестные \(x_{k1}, x_{k_2}, \ldots, x_{k_n}\) назовем главными(базисными), а остальные - свободными(параметрическими).
	\end{definition}
	\(1\le k_1<k_2<\ldots<k_r\le n\).
	\begin{enumerate}
		\item Все неизвестные главные(свободных нет). Если все неизвестные главные, значит \(n = r\), тогда \(k_i = i\), лидеры будут \(a_{11}, a_{22}, \ldots, a_{nn}\). Тогда идя снизу наверх находя неизвестные, можно находить все корни уравнения однозначно, значит система совместная определенная.
		\item Есть хотя бы одна свободная неизвестная. Совместная неопределенная. Пусть главные неизвестные \(x_{j_1}, \ldots, x_{j_r}\) и лидеры \(a_{1k_1},\ldots, a_{rk_r}\). Восстановим систему, при этом все свободные неизвестные перенесем в правую часть. \newline 
		\begin{align*}
			&a'_{1k_1}x_{k1}+\ldots + a'_{1k_r}x_{k_r} = b_1 - L_1 \\
			&a'_{2k_1}x_{k2}+\ldots + a'_{rk_r}x_{k_r} = b_2 - L_2 \\
			&\ldots \\
			&a_{rk_R}x_{rk} = b_r'-L_r
		\end{align*}. 
		
		Так как система* имеет треугольный вид относительно главных неизвестных, то какие бы значения из поля F мы бы не придали свободным неизвестным, мы находим решение системы. В данном случае система совместная неопределенная.
	\end{enumerate}
\end{enumerate}
Чтобы записать общее решение в данном случае пусть Б.О.О \(x_1, \ldots, x_r\) - главные, \(x_{r+1},\ldots, x_n\) - свободные. Тогда \(\left\{\begin{gathered}
	x_1 = \alpha_1 + \sum_{j = r+1}^{n}\beta_{1j}x_j \\
	x_2 = \alpha_2 + \sum_{j = r+1}^{n}\beta_{2j}x_j\\
	\ldots \\
	x_r = \alpha_r + \sum_{j = r+1}^{n}\beta_{rj}x_j
\end{gathered}\right.\Longrightarrow X = \underset{X_0}{\begin{pmatrix}
\alpha_1 \\ \alpha_2 \\ \ldots \\ \alpha_r \\ 0 \\ 0 \\ \ldots \\ 0
\end{pmatrix}}+ \underbrace{\overset{\vec{\beta_{r+1}}}{\begin{pmatrix}
\beta_{1(r+1)} \\ \beta_{2(r+1)} \\ \ldots \\ \beta_{r(r+1)} \\ 1 \\ 0 \\ \ldots \\ 0
\end{pmatrix}} + \ldots + \overset{\vec{\beta_{r+1}}}{\begin{pmatrix}
\beta_{1n} \\ \beta_{2n} \\ \ldots \\ \beta_{rn} \\ 0 \\ 0 \\ \ldots \\ 1
\end{pmatrix}}}_{V_0 - \text{общее решение AX = 0}}\) - Запись общего решения системы. Тогда \(V_0\) - базис в пространстве решений однородной системы - ФСР.
\begin{corollary}
	\(\dim V_0 = n - r,\) где n - число неизвестных в системе, а r - число главных неизвестных в системе.
\end{corollary}


\section{Ранг системы векторов и ранг матрицы}
Пусть V - конечномерное линейное пространство над F, S - система векторов пространства V.
\begin{definition}
	Рангом системы векторов S называется размерность её линейной оболочки(\(\dim<S>\)). Пишут \(\rk S = \dim <S>\). 
\end{definition}
\begin{definition}
	Базисом системы векторов S называется её максимальная ЛнЗ подсистема.
\end{definition}
\begin{definition}
	Системы векторов (\(a_1, \ldots, a_m\)) и \((b_1,\ldots, b_n)\) называют эквивалентными, 
	если каждый вектор \(\forall j:b_j = \sum_{i=0}^{m} \beta_ia_i \quad \text{и} \quad \forall i: a_i = \sum_{j=0}^{n} \alpha_jb_j\)
\end{definition}
\begin{proposition}
	Эквивалентные системы векторов всегда имеют одинаковые ранги.
\end{proposition}
\begin{proof}

	 \((a_1, \ldots, a_m)\sim(b_1,\ldots, b_n), b_j\in <a_1, \ldots, a_m>\Longrightarrow \sum_{j=0}^{m} \beta_jb_j \in <a_1, \ldots, a_m> \\ 
	 \Longrightarrow <b_1, \ldots, b_n>\subset <a_1,\ldots, a_m>\). 
	 
	 Очевидно, что обратное тоже верно, то есть \(<a_1, \ldots, a_n> = <b_1, \ldots, b_n>\Longrightarrow \)ранги равны
\end{proof}
\begin{proposition}
	Совершение над матрицей \(A\in M_{m\times n}(F)\) э.п. строк не изменяет ранга её системы строк.
\end{proposition}
\begin{proof}
	\((\overline{a_1}, \ldots, \overline{a_m})\underset{\overline{a_i}\to \overline{a_i}+\lambda \overline{a_j}, i\ne j}{\longrightarrow}(\overline{a_1}, \ldots, \overline{a_i}+\lambda \overline{a_j}, \ldots, \overline{a_j},\ldots, \overline{a_n}) = S'\). 
	Очевидна эквивалентность систем(возможно за ислключением \(\overline {a_i} = (\overline{a_i}+\lambda\overline{a_j}+(-\lambda)\overline{a_j})\)), 
	то есть \(\rk S = \rk S'\). 
	Остальные преобразования очевидно также не меняют оболочку.
\end{proof}
\begin{definition}
	Пространством строк матрицы \(A_{m\times n}\) называется линейная оболочка её строк. \(rowsp(A)\).
\end{definition}
\begin{definition}
	Пространством столбцов матрицы \(A_{m\times n}\) называется линейная оболочка её столбцов. \(colsp(A)\).
\end{definition}
\begin{definition}
	Рангом матрицы по строкам называется размерность её пространства строк \(\rk_r(A) = \dim(rowsp(A))\)
\end{definition}
\begin{definition}
	Рангом матрицы по столбцам называаетя размерность её пространства столбцов. \(\rk_c(A)=\dim(colsp(A))\)
\end{definition}
\begin{proposition}
	Ранг \(rk_rA\) ступенчатой матрицы равен числу её ненулевых строк(числу лидеров строк)
\end{proposition}
\begin{proof}
	\(a_{1k_1}, \ldots, a_{rk_r}\) - лидеры в ступенчатой матрице A. \newline
	
	\[\begin{pmatrix}
		a_{1k_1} & 0 &0 & 0 & \ldots & 0 & 0 \\
		0 & 0 & a_{2k_2} & 0 & 0 &\ldots & 0 \\
		\ldots &(0) \\
		0& 0 & 0 & 0 & \ldots & 0 & 0
	\end{pmatrix}\]

	Достаточно показать, что  \((\vec a_1, \ldots, \vec a_r)\) - ЛнЗ. Пусть \(\lambda_1\vec a_1 +\ldots + \lambda_r\vec{a_r} = 0\). На месте \(k_1: \lambda_1\vec a_{1k_1}\Longrightarrow \lambda_1 = 0\). На месте \(k_2: \lambda_1\vec{a1k_2}+\lambda_2\vec{a_2k_2} = 0\). Поскольку $\lambda_1=0: \lambda_2 = 0$, аналогично для остальных $\lambda$ покажем, что система ЛнЗ.
\end{proof}
\begin{corollary}
	Чтобы найти ранг матрицы по строкам достаточно привести к ступечатому виду и посчитать число ненулевых строк
\end{corollary}
\begin{proposition}
	Совершение над матрицей \(A_{m\times n}\) э.п. столбцов сохраняет все зависимости между её строками
\end{proposition}
\begin{proof}
	Если над А сделать э.п. столбцов, то \(A\to AQ,\) где \(Q\) - элементарная матрица. Пусть есть А с зависимыми \(\alpha_1, \ldots, \alpha_m\) между строками A: \((\alpha_1 \quad \ldots \quad \alpha_n)A = \vec 0\). Тогда \((\alpha_1 \quad \ldots \quad \alpha_n)AQ = ((\alpha_1 \quad \ldots \quad \alpha_n)A)\cdot Q = 0\). То есть сохраняется также ЛЗ строк. То есть линейная зависимость сохраняется.
\end{proof}
\begin{corollary}
	Если над A совершить э.п. столбцов, то \(\rk_rA\) сохраняется.
\end{corollary}
\begin{proof}
	Пусть в A: \(\rk_rA =r\Longrightarrow \exists r\) - ЛнЗ строк в А и а всякая надсистема - ЛЗ. Так как r+1 строк A - ЛЗ, то \(r+1\) строк в A' - ЛЗ. То есть \(\rk_r A'\le r = \rk_r A\). Если же сделать обратное преобразование, то получим, что \(\rk_r A\le \rk_r A'\Longrightarrow \rk_r A' = \rk_r A\). 
\end{proof}
\begin{note}
	При э.п. столбцов сохраняется \(\dim(rowsp(A))\), но не само пространство. 
	
	\(A = \begin{pmatrix}
		1 & 0 & 0 \\
		0 & 1 & 0 \\ 
		0 & 0 & 0
	\end{pmatrix}, A' = \begin{pmatrix}
	0 & 0 & 1 \\
	0 & 1 & 0 \\
	0 & 0 & 0
	\end{pmatrix},\)  
	
	\( rowsp(A) = \{(x,y,0), x,y\in F\}, rowsp(A')=\{(0,y,x),x,y\in F\}\)
\end{note}
\begin{note}
	К какому наиболее простому виду можно привести строку э.п.?
\end{note}
\begin{definition}
	Матрица \(A_{m\times n}\) имеет единично диагональной вид, если в левом верхнем углу A содержится \(E_r\), а все остальные элементы - нулевые
\end{definition}
\begin{example}
	$\begin{pmatrix}
		1 & 0 & 0 & 0 & 0 \\
		0 & 1 & 0 & 0 & 0 \\
		0 & 0 & 1 & 0 & 0 \\
		0 & 0 & 0 & 0 & 0 \\
		0 & 0 & 0 & 0 & 0 \\
	\end{pmatrix}, \begin{pmatrix}
	1 & 0 & 0 & \ldots & 0 \\
	0 & 1 & 0 & \ldots & 0
	\end{pmatrix}$
\end{example}
\begin{proposition}
	Всякую ненулевую матрицу матрицу \(A_{m\times n}\) можно привести к единичному диагональному виду, применяя э.п. строк и столбцов.
\end{proposition}
\begin{proof}
	Приведем A к упрощенному виду и ненулевые элементы могут стоять только справа от единицы упрощенной матрицы. Тогда возьмем столбец \(k_1\) и аннулируем все элементы справа от это 1 э.п. столбцов. А теперь после э.п. столбцов нужно переставить столбцы так, чтобы единицы встали в левый верхний угол.
\end{proof}
\begin{corollary}
	(теорема о ранге матрицы). Ранг матрицы по строкам равен рангу матрицы по столбцам и равен r - порядок единичного минора единично диагонального вида матрицы A. \(\rk_r A= \rk_c A = r\)
\end{corollary}
\begin{proof}
	Ранг матрицы после приведения к единично диагональному виду ранги по строкам и столбцам не меняются, при этом очевидно, что \(rowsp(A), colsp(A)\) порождаются первыми r строками и столбцами.
\end{proof}
\begin{theorem}
	Пусть V - Линейное пространство над F, \(\dim_FV = n\) и пусть \((a_1, \ldots, a_n)\) - система векторов пространства V. \(\rk(a_1, \ldots, a_n) = \rk A\), где А содержит координатные столбцы векторов \(a_1, \ldots, a_k\) относительно произвольного базиса \(\mathfrak{E}\) в V.
\end{theorem}
\begin{proof}
	A = \((\alpha_1 | \ldots | \alpha_R)\). Существует изоморфзим \(\phi: V\to F^n, \phi(a) = \alpha\underset{\mathfrak{E}}{\longleftrightarrow}a\). \(\phi(<a_1, \ldots, a_k>)=colsp(A), \rk(a_1, \ldots, a_k) = \rk_cA = \rk A\).
\end{proof}
\begin{definition}
	Минором k-ого порядка матрицы \(A_{m\times n}\) называется квадратная подматрица, которая содержится на пересечении выбранных k строк и выбранных k столбцов. Пишут \(M^{i_1, \ldots, i_k\text{ - номера строк}}_{j_1, \ldots, j_k\text{ - номера столбцов}}\)
\end{definition}
\begin{definition}
	Минор k-ого порядка матрицы A называется невырожденным, если он имеет максимально возможный ранг, то есть \(\rk M = k\).
\end{definition}
\begin{note}
	Если ранг матрицы A равен r, то в матрице не может быть невырожденного минора \(M^{i_1, \ldots, i_k}_{j_1, \ldots, j_k}\) при k > r.
\end{note}
\begin{proof}
	От противного. Пусть есть такой минор. При этом строки минора M ЛнЗ. Тогда соответствующие строки A тоже ЛнЗ, иначе строки минора M тоже были бы ЛЗ. Противоречие с тем, что \(\rk A = r\).
\end{proof}
\begin{definition}
	Пусть для ненулевой A \(\rk A = r\). Базисным минором называется в матрице A называется произвольный невырожденный минор порядка r.
\end{definition}
\begin{theorem}
	(О базисном миноре). Пусть A - невырожденная матрица \(m\times n\). Тогда \(\rk A = r\)
	\begin{enumerate}
		\item Базисный минор существует
		\item Каждая строка матрицы А является ЛК базисных строк, то есть строк, проходящих через базисный минор.
		\item Каждый столбец в А является ЛК базисных столбцов.
	\end{enumerate}
\end{theorem}
\begin{proof}
	Раз \(\rk A = r\), то найдется r ЛнЗ строк \(a_{i_1}, \ldots, a_{i_r}\) и столбцов \(b_{j_1}, \ldots, b_{j_r}\). Покажем, что минор \(M^{i_1, \ldots, i_r}_{j_1,\ldots, j_r}\) - невырожденный. Если мы добавим любую строку \(a\), не содерждащую минор, то система окажется ЛЗ, так как \(\rk_r A = r\), то есть \(a = \lambda_1a_{i_1}+\ldots + \lambda_r a_{i_r} \). Тогда с помощью э.п. можно аннулировать эту строку a. Проделав так со всеми строками, не проходящими через M, а также столбцами, получаем матрицу, в которой ненулевой минор с окружающими его нулями. Тогда очевидно, что M - невырожденный, \(\rk M = r\) - М базисный минор, а пукты 2, 3 теоремы очевидны. 
\end{proof}