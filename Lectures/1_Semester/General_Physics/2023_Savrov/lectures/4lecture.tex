\section{Момент импульса}
 Упругий удар - когда полная кинематическая энергия системы сохраняется, неупругий - наоборот
Рассмотрим парное упругое столкновение, выполняется ЗСИ.
\[(\Vec{p_1}, \Vec{p_2}) \rightarrow (\Vec{p_1'}, \Vec{p_2'})\]
Удобно рассмотреть это взаимодействие с точки зрения ЦМ (в СО ЦМ).
\[V_{cm} = \frac{\Vec{p_1} + \Vec{p_2}}{m_1 + m_2}\]
\[\Vec{p_{1cm}} = \Vec{p_1} - m_1\Vec{V_{cm}}\]
\[\Vec{p_{2cm}} = \Vec{p_1} - m_2\Vec{V_{cm}}\]
ЗСИ:
\[\Vec{p_{1cm}} + \Vec{p_{2cm}} = \Vec{p_{1cm}'} + \Vec{p_{2cm}'} = \Vec{0}\]
ЗСЭ через импульсы:
\[\frac{\Vec{p_{1cm}^2}}{2m_1} + \frac{\Vec{p_{2cm}^2}}{2m_2} = \frac{\Vec{p_{1cm}^2'}}{2m_1} + \frac{\Vec{p_{1cm}^2'}}{2m_2}\]
Из этого получаем, что импульсы упругого столкновения остаются одинаковыми по модулю.
\[ |\Vec{p_{1cm}}| = |\Vec{p_{1cm}'}| = |\Vec{p_{2cm}}| = |\Vec{p_{2cm}'}|\]
То есть происходит только поворот на некоторой угол $\theta$, но модули импульсов не меняются.
\newline Очень часто задача звучит так: есть частица, которая движется и частица-мишень.
\newline То есть $\Vec{p_2} = \Vec{0}$
\newline Тогда угол, между вектором $\Vec{p_1}$ и $\Vec{p_1'}$ - это угол $\theta$ после взаимодействия с некой частицей-мишенью.
Заметим, что так как мишень неподвижна, то:
\[\Vec{p_1} = m_1\Vec{V_{cm}} + m_2\Vec{V_{cm}}\]
\[\Vec{p_1} = \Vec{p_1'} + \Vec{p_2'}\]
Далее, мы можем найти спокойно $\Vec{p_{2cm}'}$
\newline Неупругие удары, пороговая энергия
\newline Рассмотрим следующую реакцию:
\[ {}^4He + {}^{14}Ne = {}^{17}O + p - Q\] 
Q = 1.13 МэВ
\newline 1 эВ = 1e x 1В = $1.6 * 10^{-19}$ Кл * 1 В = $1.6 * 10^{-19}$ Дж
\newline Допустим, что вылетающее ядро гелия обладает необходимой энергией, чтоб её осуществить.
Тогда пороговая энергия (то есть минимальная энергия, необходимая на осуществление реакции).
\[K = \frac{(\Vec{p_o} + \Vec{p_p})^2}{2(m_o + m_p)} + Q = \frac{\Vec{p_{He}^2}}{2(m_o + m_p)} + Q\]
\[K = \frac{\Vec{p_{He}^2}}{2m_{He}} * \frac{m_{He}}{m_o + m_p} + Q\]
\[K = K * \frac{m_{He}}{m_o + m_p} + Q \rightarrow K = \frac{m_o + m_p}{m_o + m_p - m_{He}}Q\]
Момент импульса
Пусть частица движется со скоростью $\Vec{v}$, в данный момент находится в точке $\Vec{r}$, тогда можно записать новый радиус-вектор через время, как
\[\Vec{r} + \Vec{v}dt\]
Запишем площадь треугольника, образованного векторами $vdt, r, r + vdt$, тогда:
\[dS = \frac{1}{2}rvsin\alpha dt\]
\[\frac{dS}{dt} = const\]
Это называется секториальная скорость.
\[\dot S = \frac{1}{2}|[\Vec{r} * \Vec{v}]| = const\]
Момент импульса, это вектор:
\[\Vec{L} = [\Vec{r} * \Vec{p}]\]
Посчитаем его производную:
\[\dot L = [\dot \Vec{r} * \Vec{p}] + [\Vec{r} * \dot \Vec{p}] = [\Vec{r} * \dot \Vec{p}] = 0\]
Первое произведение 0 в силу того, что вектора $v, p$ - коллинеарны.
Следующее уравнение называется уравнением моментов.
\[\dot L = [\Vec{r} * \dot \Vec{p}] = [\Vec{r} * \Vec{F}] = \Vec{M}\]
Перейдём к закону сохранения момента импульса.
\newline Рассмотрим момент импульса замкнутой системы.
\[\frac{d\Vec{L}}{dt} = \sum_i [\Vec{r_i} * \dot \Vec{p_i}] = \sum_i [\Vec{r_i} * \sum_k \Vec{f_{ki}}]\]
\[\frac{d\Vec{L}}{dt} = \frac{1}{2}\sum_{ik}[\Vec{r_i} * \Vec{f_{ki}}] + \frac{1}{2}\sum_{ik}[\Vec{r_k} * \Vec{f_{ik}}] = \frac{1}{2}\sum_{ik} [(\Vec{r_i} - \Vec{r_k}) * \Vec{f_{ki}}] = 0\]
\newline То есть его производная равна 0, тогда момент импульса сохраняется.
\newline Законы сохранения связаны с симметрией пространства-времени.
\newline Закон сохранения моментов импульса связан с инвариантностью пространства Минковского относительно вращения, а импульсов - относительно перемещения.
\newline Запишем момент импульса относительно ЦМ.
\[\dot \Vec{L} = [\Vec{r} * \dot \Vec{p}] + [\Vec{v_0}t * \dot \Vec{p}] = \dot \Vec{L}'  + [\Vec{v_0}t * \dot \Vec{p}] = [\Vec{r} * \Vec{F}]\]
\[L = \sum_i [\Vec{r_i} * \Vec{p_i}] = \sum_i [(\Vec{r_i}' + \Vec{r_{cm})} * (\Vec{p_i}' + m_i\Vec{V_{cm}}) = ... = \Vec{L_{cm}} + [\Vec{r_{cm}} * \Vec{p}]\]
То есть мы получили, что момент импульса системы - это момент импульса центра масс + момент импульса относительно центра масс.
Первое слагаемое называют спином, а второе - орбитальным моментом.
Например, вращение Земли относительно себя самой - спин, а вокруг Солнца - орбительный момент.
В силу сохранения момента импульса мы можем считать, что сумма спина и орбитального момента сохраняется.