\section{Работа и энергия}
Работа по определению:
\[A_{12} = \int_1^2 F dr = \int F * v dt\]
Связь кинетической энергии и работы:
\[A_{21} = \int \frac{dp}{dt} v dt = \int v dp = m \int v dv = \frac{mv_2^2}{2} - \frac{mv_1^2}{2} = K_2 - K_1\]
Закон сложения скоростей и его связь с кинетической энергией.
\[\Vec{v} = \Vec{v'} + \Vec{v_0}\]
    \[K = \frac{m\Vec{v^2}}{2} = \frac{m}{2}(\Vec{v'} + \Vec{v_0)^2} = \frac{m\Vec{v^2}}{2} + m\Vec{v'}\Vec{v_0} + \frac{m\Vec{v_0^2}}{2} = K' + \beta'\Vec{v_0} + \frac{m\Vec{v_0^2}}{2} \]
    Все силы можно разделить на консервативные и неконсервативные.
    Консервативные силы (их работа не зависит от пути), все остальные - неконсервативные.
    Примеры
    \newline Сила тяжести:
    \[A_{21} = m \int_1^2 \Vec{g} d\Vec{r} = -mg \int_1^2 dh = -mg(h_2 - h_1)\]
    Вот так можно посчитать работу силы тяжести.
    \[A_{21} = \int_1^2 f(r) \frac{\Vec{r}}{r} d\Vec{r} = \int_1^2 f(r) dr = - [u(r_2) - u(r_1)]\]
    Неконсервативные силы, примеры:
    Сила трения
    \[ \Vec{F} = -\gamma(v)\Vec{v} \]
    В общем виде нельзя найти её первообразную, поэтому она неконсервативная.
    Для консервативной энергии вводится потенциальная энергия. 
    \[A_{21} = -[u(\Vec{r_2}) - u(\Vec{r_1})]\]
    Для консервативных сил работа, совершаемая ими равна разнице потенциальных энергий.
    Потенциальная энергия вводится с точностью до константы.
    Мы можем определить силу в любой точке, если знаем её потенциальную энергию и координаты.
    \[\Vec{F}d\Vec{r} = - du \]
    В случае если у нас фиксированны другие координаты, то мы можем посчитать консервативную силу так.
    \[F_x = - (\frac{du}{dx})_{yz}\]
    Рассмотрим случай шарика на пружинке:
    \newline Изменение происходит только вверх-вниз, то другие координаты не влияют:
    \[u = \frac{k\Vec{r^2}}{2} => F_x = -\frac{d}{dx}(\frac{kx^2}{2}) = -kx => \Vec{F} = -k\Vec{r}\]
    Мы смогли определить силу, что и хотели посмотреть.
    \newline Посмотрим, что происходит для гравитационной силы.
    \[u = -G \frac{Mm}{r}\]
    \[F_x = -GMm \frac{d}{dx} \frac{1}{r} = -GMm \frac{d}{dx} \frac{1}{\sqrt{x^2 + y^2 + z^2}} = -GMm \frac{2x}{(x^2+y^2+z^2)^{\frac{3}{2}}}\]
    \[-GMm \frac{2x}{(x^2+y^2+z^2)^{\frac{3}{2}}} = \frac{GMmx}{r^3} = \frac{GMm}{r^2} \frac{x}{r}\]
    Получаем силу ($\frac{GMm}{r^2}$) и направление ($\frac{x}{r}$).
    Закон сохранения энергии:
    \[A_{21} = K_2 - K_1 = - (u_2 - u_1) => K + U = E = const\]
    Вводится только для консервативных сил (работу которых можно посчитать).
    Верно для системы тел:
    \[E = \sum_i K_i + u(\Vec{x_1}, \Vec{x_2}, ... , \Vec{x_N})\]
    Например, для гравитационной силы и множества тел:
    \[-\sum_{i<j} \frac{Gm_im_j}{|r_i - r_j|}\]
    Запишем дифференциал ЗСЭ:
    \[dE = \sum_i[\Vec{v_i} * d\Vec{p_i} + d\Vec{r_i} * \frac{du}{d\vec{r_i}}] = dt \sum_i \Vec{v_i}[\dot \Vec{p_i} + \frac{du}{d\Vec{r_i}}] = 0\]
    Таким образом, мы получаем, что в замкнутой системе, где действуют только консервативные силы, выполняется закон сохранения энергии.
    \newline Центр масс
    \newline Для системы точек, центр масс будет определяться, как:
    \[ \Vec{r_{cm}} = \frac{1}{M} \sum_i m_i\vec{r_i} \]
    Если продифференцируем, то получим скорость ЦМ.
    \[\Vec{v_{cm}} = \frac{1}{M} \sum_i m_i \Vec{v_i} = \frac{1}{M} \sum_i \Vec{p_i}\]
    Запишем 2ЗН для центра масс:
    \[M\dot\vec{v_{cm}} = \sum_i \vec{F_i} = \sum_i (\sum_k '\Vec{f_{ki}} + \Vec{f_i}) = \sum_i \Vec{f_i} = \Vec{F} \]
    Получаем закон движения центра масс (то есть, что ускорение ЦМ на суммарную массу = сумме всех сил)
    \newline Теорема Кёнига
    \newline Сопутствующая система - инерциальная система, которая в данный момент движется с такой же скоростью, как и центр масс.
    Посмотрим на скорость i-го тела в такой системе:
    \[K_i = K_{icm} + \frac{m_i\Vec{v_{cm}^2}}{2} + \Vec{p_{icm} \Vec{v_{cm}}}\]
    Просуммируем по всем i:
    \[K = \sum_i K_i = K_{cm} + \frac{M\Vec{v_{cm}^2}}{2}\]
    Кинетическая тела в лабораторной системе = кинетическая энергия движения тела относительно центра масс + кинетическая энергия центра масс - это и есть теорема Кёнига.
    \newline
    Задача двух тел
    \[m_1\Vec{v_1} = \Vec{F_{21}}\]
    \[m_2\Vec{v_2} = \Vec{F_{12}}\]
    \[\frac{d}{dt}(m_1\Vec{v_1} + m_2\Vec{v_2}) = 0 => \Vec{v_{cm}} = \frac{m_1\Vec{v_1} + m_2\Vec{v_2}}{m_1 + m_2}\]
    \[\frac{d}{dt} (\Vec{v_1} - \Vec{v_2}) = (\frac{1}{m_1} + \frac{1}{m_2})\Vec{F_{21}} = \frac{m_1 + m_2}{m_1m_2}\vec{F_{21}}\]
    \[\Vec{V'} = \Vec{v_1} - \Vec{v_2}\]
    \[\mu = \frac{m_1m_2}{m_1 + m_2}\]
    \[\mu \dot \Vec{v'} = \vec{F_{21}}\]