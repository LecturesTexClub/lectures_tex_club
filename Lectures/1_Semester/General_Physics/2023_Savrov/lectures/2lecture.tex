\section{Закон сохранения импульса, закон сложения скоростей, уравнение Мещерского, уравнение Циолковского}
Если вы умеете измерять положение тела как функцию времени, то вот вам и система отсчёта. Не все они эквивалентны, существуют ИСО и неИСО.
 \newline Первый закон Ньютона: существуют ИСО, где тело движется равномерно прямолинейно в отсутсутствии действия других сил (или их действие скомпенсировано). Если существует одна система отсчёта, которая ИСО, тогда поворотом со скоростью $v_0$ можем получить множество других.
 \newline Пусть есть тело, тогда время абсолютно (если находимся в пределах Земли) (то есть нерелятивистская механика). Пока скорость света можно считать бесконечной, время можно считать абсолютным ($t = t'$)
 Пусть есть система К и К', пусть система K' движется со скоростью $v_0$ 
 тогда:
 \[ \Vec{r'}= \Vec{r} - \Vec{v_0}t\]
 \[\Vec{v'} = \Vec{v} - \Vec{v_0} \rightarrow \Vec{v} = \Vec{v_0} + \Vec{v'}\]
 $v_0$ - скорость системы отсчёта K' относительно лабораторной СО, $v$ - скорость тела относительно лабораторной СО, $v'$ - скорость тела относительно системы отсчёта K'.
 \newline Получили закон сложения скоростей (принцип относительности Галиллея).
 \newline Полностью реализовать ИСО - нельзя, поскольку даже аудитория находится на Земле, которое вращается вокруг своей оси, а она вокруг Солнца, а Солнце вокруг всего остального.
 Их нельзя назвать ИСО, но поскольку влияние чрезвычайно мало, то мы можем считать их ИСО. 
 ИСО - идеализация
 \newline Переходим к динамике:
 Импульс - количество движения. 
 \[\vec{p} = m\vec{v}, \ (v << c)\]
 Чем больше масса, тем труднее изменить состояние движения тела. При выстреле из ружья: ружьё тяжелое, а пуля легкая, поэтому пуля летит очень быстро)
 \newline $\Vec{p'} = \Vec{p} - m\Vec{v_0}$ (при изменении системы отсчёта).
 \newline Понятие силы:
 2 закон Ньютона:
 \[\Vec{F} = \frac{d\Vec{p}}{dt}\]
 $\Vec{F} = \frac{d\Vec{p}}{dt} = \frac{d\Vec{p'}}{dt}$ (от системы отсчёта не зависит). Силы не отличаются в разных ИСО.
 \newline Примеры сил:
 \newline Сила упругости: $\Vec{F}_l = -k\Vec{x}$ - сила возвращающая
 \newline Сила трения: $|F_{тр}| <= \mu \vec{N}$
 \newline $\Vec{F_v} = -\gamma * \Vec{v}$ - сила сопротивления.
 \newline $\Vec{F} = m\Vec{g}$ - сила тяжести.
 \newline Равнодействующая сила:
 \[\Vec{F} = \sum\limits_{i=1}^n \Vec{F_i}\]
 \newline Третий закон Ньютона:
 \[\Vec{F_{12}} + \Vec{F_{21}} = \Vec{0}\]
 Это 2 силы, но действуют они по прямой линии. Силы всегда появляются парами! 
 \newline Из 3 закона Ньютона можно получить закон сохранения импульса.
 \newline Рассмотрим замкнутую систему, то есть все внешние силы (действующие снаружи на эту систему) либо скомпенсированы, либо незначительны. 
 \newline Пусть тело $i$ действует на другие с силой $\vec{F_{ki}}$, тогда:
 \[ \frac{d\vec{p_i}}{dt} = \sum\limits_{k = 1}^n \Vec{F_{ki}}\]
 \[ \frac{d\sum\limits_{i = 1}^n \Vec{p_i}}{dt} = \sum\limits_{i = 1}^n\sum\limits_{k = 1}^n \Vec{'F_{ki}} = \frac{1}{2}\sum\limits_{i = 1}^n\sum\limits_{k = 1}^n (\Vec{F_{ik}} + \Vec{F_{ki}}) = \Vec{0}, \ i \neq k\] \[ \sum\limits_{i = 1}^n\sum\limits_{k = 1}^n \Vec{'F_{ki}} = \sum\limits_{k = 1}^n\sum\limits_{i = 1}^n = \Vec{'F_{ik}} = \sum\limits_{i = 1}^n\sum\limits_{k = 1}^n \Vec{'F_{ik}}\]
 \[\frac{d\sum\limits_{i = 1}^n \Vec{p_i}}{dt} = \Vec{0} \rightarrow \sum\limits_{i=1}^n \vec{p_i} = const\]
 Это и есть закон сохранения импульса (ЗСИ).
 Все тела взаимодействуют друг с другом, но их силы компенсируются по третьему закону Ньютона. 
 Закон сохранения импульса более общий, чем механика Ньютона. 
 В момент времени t, мы знаем координаты и импульс тел.
 \newline 
 Основная задача механики:
 \[(x_i, p_i)_{t = 0} \rightarrow (x_i, p_i)_t\]
 где t - любое.
 \newline В общем случае мы хотим решать следующую систему:
 \begin{equation}
     \begin{cases}
         \dot \vec{p} = F_i(x_1, x_2, ..., x_n; p_1, ..., p_n, t)
         \\
         \dot x_i = \frac{p_i}{m_i}
     \end{cases}
 \end{equation}
 \newline Рассмотрим колебательную сис-му.
 Тело на пружине, вклад пружины - возврат тела $\vec{F} = -k * \vec{x}$
 \[\frac{d\vec{p}}{dt} = -k\vec{x}, \ \frac{dx}{dt} = \frac{\vec{p}}{m} \rightarrow \frac{d^2x}{dt^2} = -\frac{k}{m} * x = -\omega^2 * x\]
 Решив эту уравнение, получим: $x(t) = Acos\omega t + Bsin\omega t$
 \newline Найти константы A, B можно: 
 \[x(0) = A = x_0, \ \frac{dx}{dt} = -\omega Asin\omega t + \omega Bcos\omega t, \ \dot x(0) = \omega B = v_0, \ B = v_0/\omega\]
 Итого получаем:
 \[x(t) = x_0cos\omega t + \frac{V_0}{\omega}sin\omega t\]
 Выведем уравнение движения ракеты.
 \newline Рассмотрим ракету, она летит со скоростью $\vec{v}$, пусть из неё вылетело часть вещества $-dm$ (приращение массы - отрицательное число), вылетает со скоростью $\vec{u}$ относительно ракеты, введём ось x вдоль скорости движения ракеты:
 \[d(m\vec{v}) + (-dm)(\vec{v} + \vec{u}) = \vec{F}dt\]
 \[dm\vec{v} + md\vec{v} - dm\vec{v} - dm\vec{u} = \vec{F}dt\]
 Получаем уравнение Мещерского:
 \[m\frac{d\vec{v}}{dt} = \vec{F} + \frac{dm}{dt}\vec{u}\]
 Где F:
 \[\vec{F} = m\vec{g} + \vec{F_c}\]
 После преобразований:
 \[md\vec{v} = -dm\vec{u}\]
 \[\frac{dm}{m} = - \frac{d\vec{v}}{\vec{u}} \rightarrow ln(m) = -\frac{\vec{v}}{\vec{u}} + const\]
 Из этого получаем уравнение Циолковского:
 \[m = m_0 * e^{-\frac{\vec{v} + \vec{g}t}{\vec{u}}}\]