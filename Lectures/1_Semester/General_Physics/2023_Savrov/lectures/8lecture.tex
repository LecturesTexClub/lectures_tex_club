\section{СТО, кинематика}
Рассмотрим следующую систему: зеркало и какое-то расстояние h, тогда время, проходимое светом туда и обратно можно определить как:
\[t = \frac{2h}{c}\]
Пусть есть 2 СК - СК часов, которая движется со скоростью $v$ и лабораторная
Запишем ЗСС (теоретически) (то есть свет попадает на часы, отражается и приходит обратно, но часы проходят путь $vt$, тогда можно составить следующее уравнение (относительно ЛСО):
\[(\frac{cT}{2})^2 = h^2 + (\frac{vT}{2})^2\]
Отсюда:
\[T = \frac{2h}{c\sqrt{1 - \frac{v^2}{c^2}}} = \frac{T'}{\sqrt{1 - \frac{v^2}{c^2}}} >= T'\]
То есть относительно ЛСО получилось другое время.
Рассмотрим неон: обычно он образуется на радиусе 15 км от Земли, но он попадаёт на неё, несмотря на то, что он живёт только:
\[t = 2,2 * 10^{-6} с\]
Если бы он даже двигался со скоростью света, то он прошёл бы:
\[L = ct = 700 м\]
Как это возможно? По уже выведенному, что время увеличится, если СК движется со скоростью, сравнимой со скоростью света. Таким образом он может добраться до Земли.
Рассмотрим мысленно следующий эксперимент:
Пусть у нас имеется стержень, который движется относительно ЛСО со скоростью $v$, будем измерять следующим образом: в системе K (стержня) будет стоять наблюдатель, который отмечает время прохождения двумя концами стержня какой-то отметки, тогда:
\[t = t_L - t_R\]
Это собственное время
Отсюда длина:
\[L = vt\]
Далее пусть у нас уже метка движется со скоростью $-v$ и мы измеряем так же конец прохождения, но уже наблюдатель сидит на стержне (то есть стержень покоится, относительно наблюдателя, движется лишь метка).
Тогда
\[t' = t'_L - t'_R = \frac{t}{\sqrt{1 - \frac{v^2}{c^2}}}\]
\[L' = v * t' = \frac{L}{\sqrt{1 - \frac{v^2}{c^2}}} >= L\]
Ещё один эксперимент:
Стержень, длиной 1 метр, скорость 0,85c, пролетает трубу, шириной 0.5 м
В силу уже увиденного эффекта сокращения длины, получим, что если сесть на стержень, то уже длина трубы сократится в 2 раза.
А если сесть в систему покоя трубы, то уже стержень уменьшится в 2 раза. Тогда мы получаем, что время, проходимое стержнем трубы - разное.
Вывод преобразований Лоренца (обобщение преобразования Галилея, относительно постулата, что скорость света не зависит от условий).
Есть 2 СО: K (лабораторная), K' (движется со скоростью v), тогда:
\[\Vec{r'} = \Vec{r} - \Vec{v}t\]
Начнём с более простого:
K и K', оси параллельны, ось x - совпадает, K' движется со скоростью v.
Поместим в начало K - фонарик, K' - зеркальце.
Пусть в момент t1 > 0 - световой импульс от фонарика покидает начало координат.
В момент t2 - он возвращется обратно, t12 - время, когда он достигает зеркальце.
Примем c = 1;
нарисуем следующую СК x(t)
в момент времени t12 - достигает зеркальца и отражается, в t2 - попадает обратно, координаты зеркальца в системе K описываются уравнением $x = vt$, свет везде отражается под 45 градусов.
Нарисуем уже эту ситуацию в К', тогда у нас координаты зеркальца 0, а $x = -vt$ - координаты начала координат, аналогично, но свет уже был испущен из начала (то есть от этой прямой), также отразился под 45 градусов и вернулся на эту прямную, времена уже со штрихами. Можно заметить, что по сути, картинка примерно симметрична, то есть эти треугольники подобны, тогда:
\[t'_{12} = \alpha t_1\]
\[t_2 = \alphat t'_{12}\]
Отсюда:
\[t_2 = \alpha^2t_1\]
\[t_{12} = \frac{1 + \alpha^2}{2}t_1\]
Проведём высоту в СК K (от точки отражения до оси t), тогда она равна:
\[vt_{12} = c(t_{12} - t_1)\]
P.S: На самом деле - твёрдость ограничена скоростью звука, поэтому если наши скорости сравнимы со скоростью света, то теряется твёрдость.
Из этих трёх уравнений, можем выразить $\alpha$:
\[\alpha = \sqrt{\frac{1 + \beta}{1 - \beta}}\]
\[\beta = \frac{v}{c}\]
Теперь рассмотрим более общий случай:
СО О' движется относительно O со скоростью $v$.
t1' - сигнал проходит через O', t1 - через O, далее от отражается и аналогично t2', t2
\[t_1 = t - \frac{v}{c}\]
\[t_1' = t' - \frac{x'}{c}\]
\[t_2 = t + \frac{x}{c}\]
\[t_2' = t' + \frac{x'}{c}\]
Далее, уже аналогично:
\[t_1' = \alpha t_1, \ t_2 = \alpha t_2'\]
Из этих уравнений находим связь t' и t:
\[t' = \gamma(t - \beta \frac{x}{c})\]
\[x' = \gamma(x - vt)\]
\[y' = y\]
\[z' = z\]
Где:
\[\beta = \frac{v}{c}\]
\[\gamma = \frac{1}{\sqrt{1 - \beta^2}}\]
Эти уравнения называется преобразованием Лоренца.
При переходе из одной в другую СК у нас сохраняется следующая величина (она инвариант):
\[c^2(t_2 - t_1)^2 - (x_2 - x_1)^2 = inv\]
Где x1, x2 - пространственные координаты в пронстранстве Минковского, там координаты выражаются:
\[(ct, x)\]
Обобщим закон сложения скоростей: возьмём уравнения Лоренца и продифференцируем:
\[dt' = \gamma(dt - u \frac{dx}{c^2}\]
\[dx' = \gamma(dx - udt)\]
\[dy' = dy, \ dz' = dz\]
Тогда скорость (+ раскроем беты):
\[v_x' = \frac{dx'}{dt'} = \frac{v_x - u}{1 - \frac{uv_x}{c^2}}\]
\[v_y' = \frac{dy'}{dt'} = \frac{v_y}{\gamma(1 - \frac{uv_x}{c^2}}\]
\[v_z' = \frac{dz'}{dt'} = \frac{v_z}{\gamma(1 - \frac{uv_x}{c^2}}\]
Проверим, сохраняется ли скорость света:
пусть $v_x = c, \ v_y = v_z = 0$
\[v_x' = \frac{c - u}{1 - \frac{u}{c}} = c\]
Эффект Допплера.
Частота звука источаемого источника зависит от его скорости, хотим её найти.
Волна:
\[\Vec{E}(\omega t - \Vec{k}\Vec{x})\]
k - волновой вектор, $\omega$ - частота, фаза (второй множитель) не зависит от системы координат, то есть она одинакова в СК K и в СК K'
\[\omega t - \Vec{k}\Vec{x} = \omega' t' - \Vec{k}'\Vec{x}'\]
\[\omega^2 = c^2 \Vec{k}^2\]
Подставим преобразвание Лоренца во вторую часть уравнения, приравнием коэффициенты при t и t':
\[\omega = \gamma(\omega' + uk_x')\]
В этом уравнении и заключается эффект Допплера (связь частот и скорости u K' относительно K).
\[k_x = \gamma (k_x' + \frac{u\omega'}{c^2})\]
\[k_y = k_y', \ k_z = k_z'\]
Из этих уравнений:
\[\omega = \frac{\omega'}{\gamma(1 - \frac{u}{c}cos\theta)}\]
Где $\theta$:
\[k_x = \frac{\omega}{c}cos\theta\]
Источник приближается - частота увеличивается, пусть $cos \theta = 1$ (то есть система приближается к наблюдателю))
\[\omega = \omega'\sqrt{\frac{1 + \frac{u}{c}}{1 - \frac{u}{c}}} > \omega'\]
Если отдаляется, тогда $cos \theta = -1$ (для простоты считаем, что наблюдатель и движение на одной прямой,тогда если система удаляется - косинус = 1, иначе -1)
\[\omega = \omega'\sqrt{\frac{1 - \frac{u}{c}}{1 - \frac{u}{c}}} < \omega'\]
Действительно, частота уменьшается.
То есть эффект Допплера доказан.