%08.02.23

\section{Математические определения,
утверждения и доказательства.}

\subsection{Определения.}

\textit{Определения} описывают объекты и понятия. Если определение записано логической формулой, то оно имеет вид предиката $D(x),$ который истиннен тогда и только тогда, когда $x,$ удовлетворяет определению.

Определения, данные словами ничуть не хуже определений, данных формулами. На первом курсе последние встречаются чаще, чтобы научить студентов изложению в кванторах. Так, определение предела можно переформулировать словами: <<число $a$ -- предел последовательности $\{x_n\}$, если любая окрестность числа $a$ содержит все элементы последовательности, начиная с некоторого номера>>.

\subsection{Математические утверждения.}

\begin{definition}
    \textit{Математические утверждения} -- это утверждения, которые либо, истинны либо ложны. В отличие от определений, они не зависят от параметров. Если вы встретили утверждение вида <<если последовательность $x_n$ сходится, то она ограничена>>, то в силу вступает математическое соглашение о том, что в случае отсутствия в утверждении квантора по параметру, нужно поставить квантор всеобщности.
\end{definition}

Среди математических утверждений выделяют \textit{теоремы} -- истинные утверждения. Как правило, теоремами называют значимые математические утверждения. Вспомогательные истинные математических утверждения называют \textit{леммами}, предложениями и просто утверждениями. Истинное утверждение называют критерием, если оно имеет вид $$\forall x (A(x) \leftrightarrow B(x)).$$ Критерии устанавливают необходимое и достаточное условие $B(x)$ для выполнения условия $A(x)$ или, что то же самое, устанавливает эквивалентность определений $A$ и $B.$

\begin{definition}
    Рассмотрим утверждения вида
    $$\forall x \ (A(x) \rightarrow B(x)) $$
    Условие $B(x)$ является \textit{необходимым} для выполнения $A(x),$ а условие $A(x)$ является \textit{достаточным} для выполнения $B(x).$ Условие $A(x)$ считается более \textit{сильным}, чем $B(x),$ а $B(x)$ считается более \textit{слабым}, чем $A(x).$
\end{definition}

\subsection{Доказательства.}

\begin{definition}
    \textit{Доказательство} -- это логическое рассуждение, которое убеждает в верности математического утверждения любого непредвзятого слушателя (читателя). 
\end{definition}

У доказательств есть формальное определение в математической логике, но оно требует введение формальных систем и фактически такие доказательства непроверяемы человеком. Математики любят пользоваться приведённым описанием доказательства, но в утилитарном смысле оно слабо годится.

\subsubsection{Логические выводы.}

Представьте, что известна истинность утверждений $A$ и $A \rightarrow B.$ Из этого можно сразу заключить истинность утверждения $B,$ ведь если $B$ ложно, а $A$ истинно, то импликация $A \rightarrow B$ ложна. В формальной логике у этого правила есть специальная запись:
$$\frac{A, \ A \rightarrow B}{B}$$
Запись интерпретируется так: если доказано то, что выше черты, то доказано и то, что ниже черты. По аналогии с импликацией, то что выше черты называют посылкой, а то что ниже -- заключением. Мы не будем уделять внимание разным правилам вывода и акцентировать внимание на этой записи -- мы привели их здесь, чтобы описать общие идеи.
Первая состоит в том, что если известна истинность какого-то сложного (составного) логического высказывания, то используя преобразования формул или логические рассуждения можно доказать истинность или ложность частей этого высказывания, вплоть до элементарных высказываний (логических переменных). Например, пусть известно, что следующее высказывание истинно
$\overline{A} \wedge (A \vee B) $
Отсюда сразу следует, что истинны операнды конъюнкции: $\overline{A}$ и $A \vee B.$ Из истинности первого операнда следует, что $A = 0,$ а из этого факта и истинности второго
операнда следует $B = 1.$ В процессе решения задач и доказательства теорем, не обязательно известна истинность сложного высказывания, как правило известна истинность нескольких фактов, например $\overline{A}$ и $A \vee B,$ из которых можно составить сложное высказывание и с помощью логических преобразований получить результаты, которые мы получили, но в то же время можно и не составлять формулу, а использовать логическое рассуждение. Поэтому в логике, полученное нами правило вывода записали бы так:
$$\frac{\overline{A}, \ A \vee B}{B}$$

\subsubsection{Контрапозиция.}

\textit{Закон контрапозиции} гласит, что утверждение $A \rightarrow B$ равносильно (контрапозитивному) утверждению $\overline{B} \rightarrow \overline{A},$ поэтому если требуется доказать первое, вместо него достаточно доказать последнее.

\subsubsection{Индукция.}

\begin{definition}
    Схемой доказательства по \textit{индукции} называют схему вида:
    $$\frac{A(0), \ \forall n(A(n) \rightarrow A(n + 1))}{\forall n \ A(n)}.$$
    Первая посылка называется \textit{базой}, а вторая -- \textit{шагом} индукции или \textit{переходом.}
\end{definition}

\subsubsection{От противного.}

Мы полагаем, что если утверждение $B$ истинно, то оно не может быть одновременно ложным. Если предположить, что утверждение $A$ ложно и с помощью него доказать, что ложно утверждение $B,$ то есть доказать истинность $\overline{A} \rightarrow \overline{B},$ то в случае, если утверждение $B$ истинно, утверждение $A$ не может быть ложным -- иначе бы мы получили истинность $B$ и $\overline{B}.$ Отсюда вытекает способ \textbf{доказательства от противного.}

\begin{definition}
    \textit{Доказательством от противного} называют способ рассуждений, который можно описать так:
    $$\frac{\overline{A} \rightarrow \overline{B}, \ B}{A}$$
\end{definition}

\subsubsection{Примеры и контрпримеры.}

В случае если утверждение имеет вид $\exists x : A(x),$ его можно доказать, приведя пример (и доказав справедливость этого примера). А для опровержения утверждения с квантором всеобщности $\forall x : A(x)$ достаточно привести контрпример, т. е. пример $x,$ для которого $A(x) = 0.$

\subsubsection{Неконструктивные доказательства.}

Утверждение вида $\exists x : A(x)$ не обязательно доказывать приводя пример, хотя это очень желательно, если таковой имеется -- наличие примера или контрпримера лучше всего убеждает в справедливости утверждения. Бывает так, что само утверждение доказать проще, чем найти пример и мы приведём здесь такое доказательство.

\begin{example}
    Существуют иррациональные $a, b,$ такие что $a ^ b$ рационально. 
\end{example}

\begin{proof}
    Пусть $a = b = \sqrt{2}.$ Если число $(\sqrt{2})^{\sqrt{2}}$ рационально, то утверждение доказано, если нет, то возьмем $a = (\sqrt{2})^{\sqrt{2}}, b = \sqrt{2}:$
    $$(\sqrt{2})^{\sqrt{2} \cdot \sqrt{2}} = 2.$$
    То есть, либо подходит одна пара чисел, либо другая, а какая из -- мы не знаем.
\end{proof}