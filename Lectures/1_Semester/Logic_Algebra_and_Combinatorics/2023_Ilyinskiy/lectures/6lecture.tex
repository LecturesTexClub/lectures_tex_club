%16.02.23


\subsection{Эйлеровы графы, Двудольные графы.}

\begin{definition}
    Граф $G$ называется \textit{эйлеровым,} если существует замкнутый маршрут, проходящий по всем рёбрам только один раз.
\end{definition}
\begin{theorem}(Критерий Эйлерова графа.)
Связный граф эйлеров $\Lra$ степень каждой вершины четно.
\end{theorem}

\begin{proof}
    Рассмотрим замкнутый маршрут, проходящий по ребрам не более одного раза максимальной длины. 1)Хотя бы один такой есть, так как степень всех вершин хотя бы 2. 2) Пусть ребро $e$ нет в маршруте. Тогда можно считать, что один из концов $e$ -- это какой-то $v_j$ (так как есть связность). в графе $G \backslash {e_1, e_2, \dots, e_n}$ все степени вершин четно, значит, в нем есть замкнутый маршрут $v_0e_1\dots v_n.$ Значит, мы получили ейлеров граф.
\end{proof}

\begin{lemma}
    Хроматическое число графа $G$ не больше 2, тогда и только тогда, когда в $G$ все циклы четной длины.
\end{lemma}

\begin{definition}
    Граф $G = (V, E)$ называется двудольным, если существует такое разбиение вершин, что $V = A \cup B, A \cap B = \varnothing$ и $\forall \ e \in E$ один конец лежит в $A,$ другой -- в $B.$ Он называется полным, если проведены все ребра между множествами $A, B.$
\end{definition}

\subsection{Паросочетания, функции.}

\begin{definition}
    \textit{Паросочетание} в $G$ - это набор попарно непересекающихся ребер. $$S \subseteq E: \forall \ e, f \in S \Longrightarrow e \cup f = \varnothing.$$ 
    Паросочетание называется \textit{совершенным,} если $$\forall \ v \in V  \ \exists e \in S \ v \in e.$$
\end{definition}

\begin{definition}
    \textit{Ориентированный граф} $\ G = (V, E),$ где $V -$ множество, а $E = \{ (a, b) | a, b \in V, a  - \text{ первый конец, } b - \text{ второй конец.}$ Входящей степень вершины $v$ называется $$d_{m} = \{ e \in E | e = (..,v) \}.$$ А выходящей степенью $$d_{int} = \{ e \in E | e = (v, ..) \}.$$
\end{definition}

\begin{definition}
    Пусть $A, B$  -- множества. Тогда функция из $A$ в $B$ -- это ориентрованный двудольный граф с долями $A$ и $B.$ Причем $$\forall e \in E, \ e = (a, \ b), \ a \in A, b \in B$$
    $$\forall a \in A \ |d_{m}(v)| \leq 1.$$
    Пишут, что $f(a) = b, $ если $(a, \ b) \in E,$ а также говорят, что $b$ является образом $a,$ $a$ лежит в прообразе $B.$
    
    Областью определения называется, $Dom(f) = \{ a \in A | d_{m}(a) = 1 \}$
    
    Множеством значений называется, $Range(f) = \{ b \in B | d_{int}(b) \geq 1 \}.$

    Пусть задано $X \subseteq A,$ тогда 
    $$f(X) = \{ b \in B | \exists a \in X \ f(a) = b \}.$$

    Пусть задано $Y \subseteq B,$ тогда 
    $$f(X) = \{ a \in A | \exists b \in Y \ f(a) = b \}.$$

\end{definition}

\begin{definition}
    Всюду определенной функцией или отображение -- это функция $f$ с $Dom(f) = A.$
\end{definition}

\begin{definition}
    Отображение $f$ называется \textit{сюръекцией,} если $Range(f) = B.$ А если $\forall a \neq c \in A \Longrightarrow \ f(a) \neq f(c),$ то оно называется \textit{инъекцией.} Если это отображение и инъекция, и сюръекция, то его называют биекцией.
\end{definition}

\begin{definition}
    Пусть есть паросочетание $S.$ Тогда \textit{чередующаяся цепь} -- это путь, который начинается к непокрытой вершине $a \in A$ и поочередно идет по ребрам из $S$ и по ребрам не из $S.$
\end{definition}

\begin{definition}
    \textit{Увеличивающая чередующая цепь} -- чередующаяся цепь нечетной длины.
\end{definition}

\begin{lemma}
    Если для паросочетания $S$ найдена увеличивающася  чередующася цепь, то она не максимальна
\end{lemma}
\begin{proof}
    Просто заменяем нечетные ребра на четные.
\end{proof}

\begin{theorem}
    (Теорема Холла 1935 г.) В двудольном графе $G = (A \cup B, C)$ есть совершенное паросочетание тогда и только тогда, когда 
    $$\forall M \subseteq A \ |E(M)| \geq |M|.$$
    Где $E(M) = \{ b \in B| \exists a \in M, e \in E, e = \{a, b\}.$
\end{theorem}

\begin{proof}
    Необходимость очевидна. Докажем достаточность. Рассмотрим максимальное паросочетание $S,$ пусть оно не совершенно, то есть вершина $a$ не покрыта $S.$ Рассмотрим всевозможные чередующиеся цепи с началом в $a.$ Пусть все они заканчиваются в доле $A.$ Обозначим за $X \subseteq A- $ концы всех чередующихся цепей из $a,$ $Y -$ предпоследние вершины на этих цепях. 
    $$M = (X \cup \{ a \} ) \geq X \cup \{ a \}  \geq Y.$$
\end{proof}

