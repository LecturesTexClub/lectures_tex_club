%22.03.23
\section{Производящие функции -- 1.}

\begin{definition}
    Пусть дана последовательность $\{a_1, \ldots, a_n\}$ можно представлять в виде многочлена $a_0 + a_1x + \ldots + a_nx^n.$
\end{definition}

\begin{example}
   $\binom{n}{0}, \binom{n}{1}, \ldots, \binom{n}{n}$ представляется в виде $\binom{n}{0}x + \binom{n}{1} x + \ldots + \binom{n}{n} x ^ n = (1 + x) ^ n.$
\end{example}

А что же будет происходить с бесконечной последовательностью ?
Есть два подхода:
\begin{enumerate}
    \item \underline{Аналитический подход.} $f(x)$ бесконечно дифференцируема в $0.$ Тогда по формуле Тейлора:
    $$f(x) = f(0) + \frac{f^{(1)}(0)}{1!} + \frac{f^{(2)}(0)}{2!} + \ldots + \frac{f^{(n)}(0)}{n!} + \ldots$$
    \item \underline{Алгебраический подход.} Тогда производящей функцией будет являться \textit{формальным степенным рядом.} То есть это будет просто последовательность чисел с введенными операциями сложения и умножения.
\end{enumerate}

\begin{example}
    Есть два формальных степенных ряда $A = \sum \frac{1}{n!}x^n, B = \sum \frac{(-1)^n}{n!}x^n.$ Найдем $C = A \cdot B.$
    $C = \frac{1}{n!} \cdot \sum \binom{n}{i} (-1)^i.$
    Но можно $A = e^x, B = e^{-x},$ тогда их произведение равно $-1.$
\end{example}

\begin{example}
    Для степенного ряда $1 + x + x^2 + \ldots + x^n = \frac{1}{1 - x}$ это верно из формулы Тейлора, а также это можно показать формально перемножив его с $1 - x.$ 
\end{example}

\subsection{Формула Муавра с ограничениями.}

\begin{problem}
    \textbf{Формула Муавра.} Рассмотрим уравнение $x_1 + \ldots + x_n = n, \ x_i \in \N_{0}.$ Требуется найти количество решений такого уравнения.
\end{problem}

\begin{solution}
    Пусть $a_n$ -- количество решений этого уравнения. Запишем для этой последовательности производящую функцию.
    $$a_0 + a_1 x + a_2 x^2 + \ldots + a_n x^{n} + \ldots$$
    Если $k = 1, $ то производящая функция равна $\frac{1}{1 - x}.$
    Если $k = 2,$ то производящая функция равна
    $$(1 + x + x^2 + \ldots) \cdot (1 + x + x^2 + \ldots) = \frac{1}{(1 - x) ^ 2}.$$
     Итак, для произвольного $k:$
     $$\frac{1}{(1 - x) ^ k}.$$
     Проверим:
     По формуле Тейлора получим:
     $$(1 - x) ^ {-k} = \sum_{n = 0}^{\infty} \binom{-k}{n}(-x)^n = \sum_{n = 0}^{\infty} \binom{k + n - 1}{n} \cdot x^n.$$
\end{solution}

Из этого решения явно следует способ решения модификации этой задачи, когда есть ограничения на значения переменных. 

\begin{problem}
    Задача о счастливых билетах.
\end{problem}

\begin{solution}
    Рассмотрим уравнение в целых числах, не более $9.$
    $$x_1 + x_2 + x_3 = n.$$
    Пусть $b_n$ -- это количество решений такого уравнения. Тогда количество счастливых билетов равно $b_n \cdot b_n,$ то есть оно равно:
    $$(1 + x + \ldots + x^9)^3 = (1 - x^{10})^3 \sum_{n = 0}^{\infty} \binom{n + 2}{n} x^n.$$
\end{solution}

\subsection{Разложение $n$ в сумму натуральных слагаемых.}

В этой части для нас не важен порядок слагаемых разбиения.
Из диаграмм Юнга следуeт утвреждение.

\begin{lemma}
    Количество разложений $n$ в сумму слагаемых, где максимальное равно $k$ равно количество разложений $n$ в сумму ровно $k$ слагаемых
\end{lemma}

\begin{lemma}
Количество разложений $n$ на попарно различные слагаемые равно количеству разложений $n$ на нечетные слагаемые.
\end{lemma}

Производящая функция для количества разложений. Разложение перепишем так: $n = 1 \cdot x_{1} + 2 \cdot x_{2} + \ldots x_n \cdot n, x_j \geq 0.$ Тогда 
$$(1 + x + x^2 + \ldots) (1 + x ^ 2 + \ldots + x^{2k} + \ldots) (1 + x ^ 3 + \ldots + x^{3k} + \ldots) \ldots (1 + x ^ n + \ldots + x^{kn} + \ldots) $$