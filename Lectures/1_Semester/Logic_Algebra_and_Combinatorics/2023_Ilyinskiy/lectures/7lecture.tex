%22.02.23

\section{Комбинаторика.}

\subsection{Правила сложения и правило умножения.}

Правило сложения основано на следующей лемме.
\begin{lemma}
    Пусть есть два конечных непересекающихся мн-ва $A, B.$ Тогда $|A \cup B| = |A| + |B|.$
\end{lemma}

Правило сложения основано на другой лемме.

\begin{lemma}
    Пусть $A, B - $ конечные множества, тогда $|A \times B| = |A| \cdot |B|.$
\end{lemma}

\begin{proof}
    Выпишем таблицу, по столбцам которой расположены элементы множества $A,$ а по строкам -- элементы множества $B.$
\end{proof}

\begin{example}
    Сколько есть шестизначных чисел, у которых соседние цифры имеют разную четность.
\end{example}

\begin{solution}
    Разбиваем на два непересекающихся множества и используем лемму. Тогда из правил произведения и суммы ответ $5 ^ 6 + 4 \cdot 5^5.$
\end{solution}

\begin{example}
    Сколькими способами из $20$ студентов можно выбрать старосту и его заместителя?
\end{example}

\begin{solution}
    Тут уже нельзя применить сразу правило умножения. Можно выписать дерево возможных выборов, получим $20 \cdot 19.$ 
\end{solution}

Правило биекций или разбиение на пары основано на следующей лемме.

\begin{lemma}
    Пусть $A, B$ -- конечные множвества. Тогда $|A| \geq |B| \Longleftrightarrow$ есть сюрьекция из $A$ в $B;$ $|A| \leq |B| \Longleftrightarrow$ есть инъекция из $A$ в $B;$ $|A| = |B| \Longleftrightarrow$ есть биекция из $A$ в $B.$
\end{lemma}

\begin{definition}
    Пусть есть множество $A.$ Тогда множество всех подмножеств множества $A$ обозначается $2 ^ A.$
\end{definition}

\begin{lemma}
    Если $S -$  множество и есть биекция из $S$ в $A \times B.$ Тогда $|S| = |A \times B|.$
\end{lemma}

\begin{definition}
    Пусть есть множество $A = \{ a_1, \dots, a_n \}.$ Тогда выберем из него $k$ элементов.
    \begin{enumerate}
        \item $k-$Размещение без повторений $A_{n}^{k} = \frac{n!}{(n - k)!}.$ Это такой выбор, при котором важен порядок и нельзя выбирать несколько раз один и тот же элемент.

        \item $k-$Размещение с повторениями $\overline{A_{n}^{k}} = n ^ k.$ Это такой выбор, при котором важен порядок и можно выбирать несколько раз один и тот же элемент.

        \item Сочетание без повторений $\overline{C_{n} ^ {k}} = C_{m + n - 1} ^ {m}.$ Это такой выбор, при котором не важен порядок и нельзя выбирать несколько раз один и тот же элемент.

        \item Сочетание с повторениями -- это число $k$ элементных подмножеств множества $A.\binom{n}{k} = C_{n} ^ {k}$ 
    \end{enumerate}
\end{definition}

\begin{definition}
    Перестановка $n-$элементного множества -- это $n-$размещение без повторений $A, |A| = n.$
\end{definition}

