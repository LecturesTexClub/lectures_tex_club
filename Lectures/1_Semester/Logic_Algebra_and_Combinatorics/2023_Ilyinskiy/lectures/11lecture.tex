\subsection{Отношения порядка.}

\begin{definition}
    \textit{Антисимметричность} отношения $R$ означает что
    $$aRb, \ bRa \Longrightarrow a = b.$$
    \textit{Асимметричность} отношения $R$ означает что
    $$aRb \Longrightarrow \overline{bRa}.$$
    \textit{Антирефлексивность} отношения $R$ означает что
    $$\overline{aRa}.$$
\end{definition}

\begin{definition}
    Пусть $A-$ множество, $\geq_{A} -$ \textit{(нестрогое) отношение частичного порядка} на множестве $A,$ если оно рефлексивно, антисимметрично и транзитивно. А \textit{строгое отношение частичного порядка} $>_{A},$ если оно антирефлексивно, асимметрично и транзитивно. 
\end{definition}

\begin{example}
    $a \geq b$ на $\N, \Z, \Q, \R.$
\end{example}

\begin{example}
    $\subseteq$ на $2^{A}.$
\end{example}

\begin{example}
     Делимость на $\N.$
\end{example}

\begin{example}
    Лексикографический порядок на множестве слов русского языка.
\end{example}

\subsection{Связность в ориентированном графе.}

\begin{proposition}
    В ориентированном графе:
    \begin{enumerate}
        \item Если есть маршрут из $v$ в $w$ $\Longrightarrow$ $\exists$ путь из $v$ в $w.$
        \item Если есть замкнутый маршрут $\Longrightarrow$ $\exists$ цикл.
    \end{enumerate}
\end{proposition}

Утверждение доказывается аналогично как для неориентированного графа.

Рассмотрим отношение $v ~ w \Longleftrightarrow \exists$ путь из $v$ в $w.$

Заметим, что такое отношение не является отношением эквивалентности, так как может быть не выполняться симметричность. Тогда введем другое отношение.

\begin{definition}
     Вершины $v, w$ ориентированного графа находятся в отношении \textit{сильной связности}, если есть путь из $v$ в $w$ и если есть путь из $w$ в $v.$ \\
     Вершины $v, w$ ориентированного графа находятся в отношении \textit{слабой связности}, если есть путь из $v$ в $w$ или если есть путь из $w$ в $v.$ 
\end{definition}

Вот такие отношения являются отношениями эквивалентности, то есть они разбивают граф на компоненты.

\begin{definition}
    \textit{Конденсат} графа $G = (V, E)$ это граф, вершины которого -- компоненты сильной связности, а ребра проводятся, если есть ребро между вершинами, входящими в соответсвующие компоненты.
\end{definition}

\begin{lemma}
    $u, v \in V$ лежат в одной компоненте сильной связности $\Longleftrightarrow$ $u, v$ лежат на замкнутом маршруте.
\end{lemma}


\begin{proposition}
    В конденсате нет циклов.
\end{proposition}

\begin{lemma}
    В ацикличном графе существует \textit{сток,} то есть вершина, из которой не выходят ребра.
\end{lemma}

\begin{proof}
    Рассмотрим самый длинный маршрут, он не замкнут (так как нет циклов). Из последней вершины этого маршрута не выходит ребро, так как мы выбрали самый длинный маршрут.
\end{proof}


\begin{theorem}
    Для ориентированного графа следующие условия эквивалентны:
    \begin{enumerate}
        \item В графе нет циклов.

        \item  Любая компонента сильной связности состоит из одного элемента.

        \item Можно занумеровать вершины так, что ребро идет от меньшего числа к большему.
    \end{enumerate}
\end{theorem}

\begin{proof}
    $1 \Longleftrightarrow 2:$


    $3 \Longleftarrow 1:$


    $1 \Longrightarrow 3:$ Это доказывается так называемой топологической сортировкой.
    
\end{proof}

\subsection{Отношения строгого частичного порядка.}

Изобразим это отношение с помощью ориентированного графа. Заметим, что такой граф ацикличен. Пусть он имеет цикл.

А вот по графу нельзя восстановить отношение, так как оно может быть не транзитивным.


\subsubsection{Диаграмма Хассе.}

Проводим ребро из $a$ в $b$ $\Longleftrightarrow a <_{A} b \wedge \neg \exists c: \ a <_A c <_A b.$ 


\subsection{Максимальный элемент.}

\begin{definition}
    Пусть задано $(A, >_{A}).$ Тогда элемент $x$ называется \textit{максимальным,} если $\notexists y \in A: \ y >_{A} x.$ Аналогично определяется \textif{минимальный.} А также он называется \textit{наибольшим,} если $\forall y \in A \ x \geq_{A} y.$ Аналогично определяется \textit{наименьший} элемент.
\end{definition}

\begin{proposition}
    Если $x$ -- наибольший, то $x$ -- единственный максимальный.
\end{proposition}

\begin{proof}
    $x-$ наибольший, следовательно по определению он и максимальный. Если есть еще максимальный $z,$ то $x \geq_{A} z,$ противоречие.
\end{proof}

\begin{note}
    Однако не всегда верно, что если $x$ -- единственный максимальный, то $x$ -- наибольший.
\end{note} 