% 13.09.2023

\section{Наивная теория множеств.}
\subsection{Определения.}

В этой теории понятия множества, элемента являются первоначальными, поэтому нет строгого их определения. Но интуитивно их определить можно.

\begin{definition}
Множество -- это совокупность объектов, объединенных в одно условие.
\end{definition}

Множества будем обозначать большими латинскими буквами, а их элементы -- маленькими. Будем записывать принадлежность элемента $x$ множеству $X$ так: $$x \in X$$

\subsection{Задание множества.}

Множество можно задать так:
\begin{enumerate}
    \item Перечисление элементов. То есть запись вида:
    $$\{x_1, x_2, \dots, x_n\}.$$
    Но этот способ плохо работает для бесконечных множеств, например, у множества $\{0, 1, 2, \dots\}$ казалось бы, что следующий элемент - это $3,$ но и $720!$ тоже может быть, ведь $1 = 1!, 2 = 2!!, 3!!!= 120!.$
    \item По характерному свойству, то есть записать множество в виде:
    $$\{ x | <<\text{условие на } x >> \}.$$
    Например, множество четных натуральных чисел можно записать так:
    $$\{ x \in \N_0 | \exists k, x = 2k \}.$$
\end{enumerate}

\subsection{Парадоксы.}

На самом деле наша теория имеет в себе парадоксы, рассмотрим два из них. 

Для начала рассмотрим \textbf{парадокс Рассела.}

Назовем $A$ \textit{неправильным}, если $A \in A.$ В противном случае, оно \textit{правильное.}

Пусть $M$ -- множество всех правильных множеств. $M$ -- правильное?

\begin{enumerate}
    \item Если $M$ -- правильное, то $M \in M($так как оно содержит все правильные множества). То есть оно еще и неправильное, противоречие.
    \item Если $M$ -- неправильное, то $M \in M.$ Тогда $M$ -- еще и правильное множество, так как содержит только все правильные множества. Противоречие.
\end{enumerate}

Теперь рассмотрим второй парадокс. Пусть 
$$A = \{ x \in \N | x \text{ можно описать не более чем 20 словами} \}$$

Ясно, что оно конечно. Теперь рассмотрим наименьшее число $n,$ которое не входит в это множество. То есть его можно описать так: $$n - \text{наименьшее натуральное число, которое нельзя описать $20$ словами}.$$
То есть оно описывается не более $20$ словами. Тогда оно должно входить в это множество, получим противоречие.

Оба парадокса показывают, что нам нужно ввести ограничения на характерное условие множества.

\subsection{Отношения между множествами.}

\begin{definition}
    Множество $A$ является подмножеством  множества $B,$ если любой элемент $A$ лежит во множестве $B.$ Записывается это так:
    $$A \subseteq B.$$
\end{definition}

Считается, что пустое множество является подмножеством всех множеств.

\begin{definition}
    $$A = B \Longleftrightarrow
    \left\{\begin{array}{l}
    A \subseteq B\\
    B \subseteq A
\end{array}\right.
$$
\end{definition}

Выпишем свойства этого отношения:

\begin{enumerate}
    \item $A \subseteq A \text{ (рефлексивность)} $\\
    \item $A \subseteq B, B \subseteq C \Longrightarrow A \subseteq C \text{ (транзитивность)}$ \\
    \item $A \subseteq B, B \subseteq A \Longrightarrow A = B$ \text{ (антисимметричность)}
\end{enumerate}

\subsection{Операции с множествами.}
\begin{definition} 
    \begin{enumerate}
        \item $A \cup B = \{x | x \in A \vee x \in B\}$
        \item $A \cap B = \{x | x \in A \wedge x \in B\}$
        \item $A \backslash B = \{ x | x \in A \wedge \overline{x \in B} \}$
        \item $A \triangle B = A \backslash B \vee B \backslash A.$
        \item $\overline{A} = \{x | x \notin A \}$
    \end{enumerate}
\end{definition}

\subsection{Предикаты.}

\begin{definition}
    \textit{Предикатом} назовем выражения, зависящие от переменных, являющиеся истинными или ложными при каждом значении переменных. Количество переменных назовем \textit{арностью.}
\end{definition}

\begin{problem}
Доказать, что для всех множеств $A_i$ верно:
$$\overline{A_1 \cap \dots A_n} = \overline{A_1} \cup \dots \cup \overline{A_n}$$
\end{problem}

\begin{proof}
    Пусть $B$ --- множество индексов, тогда перепишем требуемое:
    $$\underset{\alpha \in B}{\cap} A_{\alpha} = \overline{\underset{\alpha \in B}{\cup} \overline{A_{\alpha}}}$$
    Докажем, что левое множество вложено в правое. Пусть $x$ --- элемент левого множества. Тогда $x \in A_{\alpha}$ для всех $\alpha \in B \Longrightarrow x \notin \overline{A_i} \Longrightarrow  x \in \overline{\underset{\alpha \in B}{\cup} \overline{A_{\alpha}}}$
    Аналогично доказывается, что правое множество вложено в левое.
\end{proof}

\subsection{Кванторы.}

$$\forall x \text{  } P(x) := \underset{x}{\wedge} P(x)$$
$$\exists x \text{  } P(x) := \underset{x}{\vee} P(x)$$