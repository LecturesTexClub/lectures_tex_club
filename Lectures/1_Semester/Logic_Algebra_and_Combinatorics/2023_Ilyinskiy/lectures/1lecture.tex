% 01.02.23

\section{Алгебра логики.}

\begin{definition}
    \textit{Высказывание} -- суждение, которое истинно или ложно.
\end{definition}

\begin{example}
    <<Вчера был вторник>>; <<3 > 5>> -- суждения.
\end{example}

Будем здесь и далее обозначать за $1$ истину, а за $0$ -- ложь.

\subsection{Логические связки.}

\begin{definition}
    Пусть $A, B$ -- высказывания. Тогда:
    \begin{enumerate}
        \item назовем высказывание <<$A$ и $B$>> \textit{конъюнкцией.} При этом пишут $$A \wedge B.$$
        
        \item назовем высказывание <<$A$ или $B$>> \textit{дизъюнкцией.} При этом пишут $$A \vee B.$$
        
        \item назовем высказывание <<либо $A$, либо $B$>> \textit{исключающим или.} При этом пишут $$A \oplus B.$$

        \item назовем высказывание <<не $A$>> \textit{отрицанием.} При этом пишут $$\overline{A}.$$

        \item назовем высказывание <<если $A$, то $B$>> \textit{импликацией.} При этом пишут $$A \rightarrow B.$$

        \item назовем высказывание <<$A$ равносильно $B$>> \textit{эквивалентностью.} При этом пишут $$ A \equiv B.$$
    \end{enumerate}
\end{definition}

Приоритет у этих операций такой: отрицание, конъюнкция, или (иключающее или), импликация, эквивалентность.

\subsection{Булевы функции.}

\begin{definition}
    \textit{Булевой функцией} назовем функцию вида $f(x_1, x_2, \dots x_n).$ 
    Причем
    $\forall i \ x_i = 0 \text{ или } 1.$ А также область значений у этой функции равна $\{ 0, 1 \}.$
\end{definition}

\begin{definition}
    Переменная $x_i$ булевой функции $f(x_1, \dots, x_n)$ называется \textit{фиктивной,} если для любых значений $(x_1, x_2, \dots, x_{i - 1}, x_{i + 1}, \dots, x_n):$
    $$f(x_1, x_2, \dots, x_{i - 1}, 1, x_{i + 1}, \dots, x_n) = f(x_1, x_2, \dots, x_{i - 1}, 0, x_{i + 1}, \dots, x_n). $$ 
    Если же это не выполняется, то $x_i$ является \textit{существенной} переменной.
\end{definition}

\begin{definition}
    Булевы функции $f, g$ равны, если у них совпадают существенные переменные $x_1, \dots x_k$ и для любых наборах $(x_1, \dots, x_k):$
    $$f(x_1, \dots, x_k) = g(x_1, \dots, x_k)$$
\end{definition}

\begin{definition}
    \textit{Тавтологией} называется булева функция, которая всегда равна $1.$
\end{definition}

\begin{definition}
    Булевы функции $f, g$ равны, если булева функция $f \equiv g$ -- тавтология
\end{definition}

Из определения $1.4$ можно вывести следующие тождества:
$$A \rightarrow B = \overline{A} \vee B$$
$$\overline{A \wedge B} = \overline{A} \vee \overline{B}$$
$$\overline{A \vee B} = \overline{A} \wedge \overline{B}$$