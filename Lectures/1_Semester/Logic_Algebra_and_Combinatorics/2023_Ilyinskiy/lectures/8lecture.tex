\subsection{Шары и перегородки.}

Наша цель -- посчитать количество $k-$ сочетаний из $n$ элементов из $\{1, \dots, n\}$ с повторами. Заметим, что эта задача эквивалентна следующей. 

(если положить $x_i = \text{ сколько раз $i$ входит в сочетание})$

\begin{problem}
    (Муавр) Найти количество неотрицательных решений уравнения $$x_1 + x_2 + \ldots + x_n = k, k \in \N$$
\end{problem}

\begin{solution}
    Представим $k$ как последовательные $k$ единиц. Тогда при выборе мы должны из <<отделить>> $n - 1$ скобками. Отсюда ответ $\binom{n - 1 + k}{k}$
\end{solution}

\subsection{Свойства $\binom{n}{k}.$}

Для начала рассмотрим следующую задачу.

\begin{problem}
    (Линейный город) Сколько способов добраться из $A(0, 0)$ до $B(m,n) m \in \N, n \in \N ,$ если разрешено только двигаться вправо и вверх.
\end{problem}

\begin{solution}
    Мы должны пройти $m \in \N$ шагов направо, $n \in \N$ шагов вверх. Тогда надо посчитать число последовательностей из $m$ нулей и $n$ единиц, а оно равно $\binom{m + n}{m}.$
\end{solution}

Из этой задачи вытекает важное свойство (если посмотрим на предпоследний шаг до точки $B)$:
$$\binom{n}{k} = \binom{n - 1}{k} + \binom{n - 1}{k - 1}.$$ 

Это тождество можно также показать, непосредственно пользуясь формулой для $\binom{n}{k}.$

Эта задача также доказывает корректность так называемого 
\textit{треугольника Паскаля.}

\begin{problem}
Доказать, что верно
    $$\binom{n}{k} = \binom{n}{n - k}$$
\end{problem}

\begin{solution}
    Можно это показать, пользуясь формулой для чисел сочетаний. Это также несложно следует из того, что выбор $k$ элементов однозначное определяет выбор $n - k$ элементов.
\end{solution}

Это свойство показывает симметричность треугольника Паскаля.

\begin{theorem} 
Доказать, что
    $$(a + b) ^ n = \sum_{k = 1} ^ {n} \binom{n}{k} a ^{n - k} b^{k}$$
\end{theorem}

\begin{proof}
    Можно это показать по индукции. Покажем это комбинаторно. Заметим, что коэффициент при каждом слагаемом $a^{n - k} b ^ {k}$ равен $\binom{n}{k},$ что доказывает это утверждение.
\end{proof}

\begin{corollary}
    $$\sum_{k = 1} ^ {n} \binom{n}{k} = 2 ^ n.$$
  $$\binom{n}{0} - \binom{n}{1} + \binom{n}{2} - \ldots + (-1) ^ n \binom{n}{n} = 0, n > 0.$$
\end{corollary}

\begin{proof}
    Доказывается рассмотрением 
    $$(1 + 1) ^ n = 2 ^ n$$
    $$(1 - 1) ^ n = 0 ^ n = 0.$$
\end{proof}

\begin{proposition}
$$n \cdot 2^{n-1} = \sum_{k = 0}^{n}k\binom{n}{k}$$
$$\binom{2n}{n} = \sum_{k = 0} ^ {n} \binom{n}{k} ^ 2$$
\end{proposition}

\begin{proof}
Для первого тождества воспользуемся этим тождеством:
$$k\binom{n}{k} = n \binom{n - 1}{k - 1}.$$
Или можно выражение в правой сложить с таким же выражением, но расположенном в обратном порядке. Для второго тождества разобьем множество из $2n$ элементов на два множества из $n$ элементов, тогда оно следует из определения числа сочетаний.
\end{proof}

\begin{lemma}
    $a_0 + a_1 + \ldots + a_n = 2 ^ n \Longrightarrow max \{a_j \} \geq \frac{2^n}{n + 1}$
\end{lemma}

\begin{proof}
    Это верно, ведь в противном случае сумма этих чисел меньше $2^n.$
\end{proof}

\begin{proposition}
    Числа сочетаний $\binom{n}{k}$ возрастает при $k = 0, 1, \ldots, [\frac{n}{2}]$ и убывает на остальных $k.$ А также $$\binom{2n}{n} \geq \frac{2^{2n}}{n + 1}$$
\end{proposition}

\begin{proof}
    Докажем пользуясь формулой:
    $$\binom{n}{k} > \binom{n}{k - 1} \Longleftrightarrow n - k + 1 > k.$$ 
    Второе утверждение следует из леммы.
\end{proof}

\subsection{Числа Фибоначчи.}

\begin{definition}
    Количества последовательностей из $0$ и $1$ длины $n,$ в которых нет подпоследовательности $11$ называется $n-$тым \textit{числом Фибоначчи.}
\end{definition}

Обозначим последовательность Фибоначчи за $F_{n}.$ Тогда, если последовательность заканчивается на $0,$ то оставшаяся часть определяется $F_{n - 1},$ если заканчивается на $1,$ то перед ним стоит $0$ и получим $F_{n - 2}.$ Из правила суммы получаем:
$$F_{n} = F_{n - 1} + F_{n - 2}.$$

\subsection{Числа Каталана.}

\begin{definition}
    Рассмотрим последовательность из $2n$ чисел, которые лежат в $\{ -1, 1\},$ такую, что
    $$\sum_{j = 1} ^ {2n} a_j = 0$$
    $$\forall \ k \ \sum_{j = 1}^{k} a_j \geq 0.$$
    Число таких последовательностей называется $n-$ым \textit{числом Каталана.}
\end{definition}

Обозначим числа Каталана за $\{ C_n \}.$