%15.02.23

\subsection{Маршруты, пути, циклы, связность. Деревья, расстояния и диаметр, раскраски, Эйлеровы графы.}

\begin{definition}
        \textit{Маршрутом} в графе $G$ называется последовательность вершин $v_0, v_1, . . . , v_n,$ такая что $n \geq 0$ и $\{v_i, v_{i+1}\} \in E(G)$ для $ 0 \leq i \leq n - 1.$ \textit{Длина маршрута} -- это число рёбер, соединяющих вершины маршрута; оно совпадает с $n.$ Маршрут называется замкнутым, если $v_0 = v_n.$ Будем говорить, что ребро $\{x, y\} \in E(G)$ лежит на маршруте, если для некоторого $i$ выполняется $\{vi, vi+1\} = \{x, y\}.$
\end{definition}

\begin{lemma}
    Если в графе $G$ есть маршрут между $U$ и $V,$ то есть путь между $U$ и $V.$ Причем, если есть два различных пути, то есть цикл. А также $\forall v \ d(v) \geq 2 \text{ есть цикл.}$
\end{lemma}
\begin{proof}
    Рассмотрим маршрут минимальной длины между $U$ и $V.$ Допустим, в нем повторяются вершины.
    $$U=v_0e_1V_1e_2\dots v_n, v_0, \dots, v_n = V$$
    Отсюда видно, что мы можем сократить путь, если в нем повторяются вершины. Тогда $V_i \neq V_j \  i \neq j.$
    Тогда мы доказали существование пути. Пусть есть два различных пути. 
    $$U=v_0e_1V_1e_2\dots v_n = V$$
    $$U=w_0f_1w_1\dots w_l = V$$
    Возьмем минимальное $i: \ v_i \neq w_i$ и рассмотрим минимальное $j \geq i: \ v_j \in \{ w_i, \dots, w_l \}.$ Тогда есть цикл. Если у нас у каждой вершины степень хотя бы $2,$ то берем произвольную вершину и будем из нее ходить.
\end{proof}


\begin{lemma}
    Пусть в графе $G$ не циклов. Тогда между любыми вершинами существует не более одного пути между ними и есть вершина степени не более $1.$ 
\end{lemma}

\begin{definition}
    Граф $G$ называется \textit{ацикличным,} если в нем нет циклов.
\end{definition}

\begin{theorem}
    Для простого графа $G$ следующие утверждения эквивалентны:
    \begin{enumerate}
        \item $G$ -- минимально связный граф (т. е. при удалении любого ребра граф становится несвязным).

        \item $G$ -- cвязный граф, в котором $|E| = |V| - 1.$

        \item $G$ -- ациклический связный граф.

        \item $G$ -- граф, любая пара вершин которого связана единственным путём.
    \end{enumerate}
\end{theorem}

\begin{proof}
    Докажем эквивалентность, установив импликации по цепочке:
    $$(2) \Longrightarrow (1) \Longrightarrow (3) \Longrightarrow (4) \Longrightarrow (2).$$ 
    Импликация $(2) \Longrightarrow (1)$ верна, так как из связного графа с $|V| - 1$ ребром нельзя удалить ребро без нарушения связности. 
    
    Установим импликацию $(1) \Longrightarrow (3),$ воспользовавшись контрапозицией, т. е. докажем $\neg (3) \Longrightarrow \neg(1).$ Отрицание условия $(3)$ означает, что граф несвязен или имеет цикл, а условия $(1),$ что граф или несвязен или связен, но не минимиально. Если граф несвязен, то импликация $\neg (3) \Longrightarrow \neg (1)$ выполняется, поэтому сосредоточимся на случае связного графа, который содержит цикл. В лемме мы установили, что при удалении ребра из цикла в связном графе, граф остаётся связным, т. е. граф до удаления ребра был не минимально связным. 
    
    Также докажем $(3) \Longrightarrow (4),$ доказав контрапозицию $\neg (4) \Longrightarrow \neg (3).$ Если выполнено условие $\neg (4)$ и между какой-то парой вершин нет ни одного пути, то граф несвязный и справедливо условие $\neg (3).$

    Осталось доказать импликацию $(4) \Longrightarrow (2).$ Проведём доказательство индукцией по числу вершин в графе. База: при $|V| = 1$ в графе нет рёбер и в вершину в себя есть единственный путь длины $0.$ Шаг. Пусть утверждение верно для всех графов на $n$ вершинах и пусть $G$ -- произвольный граф, удовлетворяющий условию $(4),$ в котором $V(G) = n + 1.$ Выберем в $G$ самый длинный путь $P,$ конец которого обозначим через $z.$ Докажем от противного, что вершина $z$ имеет степень $1.$ Допустим, что у вершины $z$ есть ещё сосед $x,$ кроме предшествующей ей вершины $y$ на пути $P.$ Если вершина $x$ не лежит на пути $P,$ то к пути $P$ можно добавить ребро $zx$ и сделать его длиннее -- противоречие с выбором $P.$ Если же $x$ лежит на пути $P,$ то, поскольку $x \neq y,$ в графе есть два пути, соединяющие вершины $x$ и $z: xP z$ и ребро $xz,$ что противоречит условию $(4).$ Удалив вершину $z$ из графа $G$ получим связный граф $G'$ на $n$ вершинах,для которого справедливо предложение индукции: $|E(G′)| = n - 1,$ поскольку между любой парой вершин $G'$ существует единственный путь. Вернув $z$ на место, получаем, что мы увеличили на единицу и число вершин и число рёбер графа $G′,$ а потому доказали, что $|E(G)| = |V(G)| - 1;$ шаг индукции доказан.
\end{proof}


\begin{definition}
    Зафиксируем в роли цветов числа от $1$ до $k.$ \textit{Раскраска графа} -- это функция $f,$ которая ставит в соответствие каждой вершине графа некоторый цвет, т. е. $f(u) \in \{1, \dots , k\}.$ Раскраска $f$ называется \textit{правильной}, если концы всех рёбер покрашены в разные цвета, т. е. для каждого ребра $\{u, v\}$ справедливо $f(u) \neq f(v).$ Минимальное число цветов, в которое можно правильно раскрасить граф $G$ называется \textit{хроматическим числом} и обозначается через $\chi(G).$
\end{definition}