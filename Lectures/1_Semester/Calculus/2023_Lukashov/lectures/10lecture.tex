\begin{definition}
	Пусть $f(x)$ определена на $(a, b)\ |\ a, b \in \R; a < b$
	
	\textit{Левосторонним пределом} в точке $b$ называется $B \in \bar{\R} \cup \{\infty\}$ такое, что
	\begin{enumerate}
		\item ($\forall \eps > 0)(\exists \delta > 0)
		\left(\forall x,\ b - \delta < x < b\right)
		\ f(x) \in U_{\eps}(B)$
		
		\item $\left(\forall \{x_n\}_{n = 1}^\infty \subset
		(a; b),\ \liml_{n \to \infty} x_n = b\right)
	 	\ \liml_{n \to \infty} f(x_n) = B$
	\end{enumerate}
	Обозначается как
	\[
		f(b - 0) := \liml_{x \to b-0} f(x) = B
	\]

	\textit{Правосторонним пределом} в точке $a$ называется $A \in \bar{\R} \cup \{\infty\}$ такое, что
	\begin{enumerate}
		\item $(\forall \eps > 0)(\exists \delta > 0)
		\left(\forall x,\ a < x < a + \delta\right)
		\ f(x) \in U_{\eps}(A)$
		
		\item $\left(\forall \{x_n\}_{n = 1}^\infty \subset (a; b),\ \liml_{n \to \infty} x_n = a\right)\ \liml_{n \to \infty} f(x_n) = A$
	\end{enumerate}
	Обозначается как
	\[
	f(a + 0) := \liml_{x \to a+0} f(x) = A
	\]	
\end{definition}

\begin{definition}
	$(b - \delta; b)$ называется \textit{левосторонней} окрестностью точки $b$.
	
	$(a; a + \delta)$ называется \textit{правосторонней} окрестностью точки $a$.
\end{definition}

\begin{theorem} (Связь предела и односторонних пределов)

	Пусть $f(x)$ определена в некоторой
 	$\mc{U}_{\delta}(a)$, $a \in \R$. Тогда
	\[
		\exists \liml_{x \to a} f(x) = A \in \overline{\R}
		\cup \left\{\infty\right\} \lra \exists \liml_{x \to a-0}
		f(x) = \liml_{x \to a+0} f(x) = A
	\]
	Для бесконечностей возможны варианты, например:

	\[ 
		\liml_{x \to a + 0} f(x) = +\infty, 
		\liml_{x \to a - 0} f(x) = -\infty \Ra
		\liml_{x \to a} f(x) = \infty
	\]

\end{theorem}

\begin{proof}
\begin{enumerate}
	\item Пусть $\exists \liml_{x \to a} f(x) = A$, тогда
	\begin{align*}
		(\forall \eps > 0)(\exists \delta > 0)
		\left(\forall x \in \mc{U}_{\delta} (a)\right)\ 
		f(x) \in U_{\eps}(A) \Ra
		\\
		(\forall \eps > 0)(\exists \delta > 0)
		\left(\forall x \in (a, a + \delta)\right)\ 
		f(x) \in U_{\eps}(A)
		\\
		(\forall \eps > 0)(\exists \delta > 0)
		\left(\forall x \in (a - \delta, a)\right)\ 
		f(x) \in U_{\eps}(A)
	\end{align*}
	
	
	\item Пусть $\exists \liml_{x \to a-0} f(x) = \liml_{x \to a+0} f(x) = A$. Тогда
	\begin{align*}
		(\forall \eps > 0)(\exists \delta_1 > 0)
		\left( \forall x,\ a - \delta_1 < x < a\right)\ 
		f(x) \in U_{\eps}(A)
		\\
		(\forall \eps > 0)(\exists \delta_2 > 0)
		\left(\forall x,\ a < x < a + \delta_2\right)\ 
		f(x) \in U_{\eps}(A)
	\end{align*}
	Выберем $\delta := \min(\delta_1, \delta_2)$, получим
	\begin{align*}
		\delta_1 \ge \delta \Ra a - \delta_1 \le a - \delta
		\\
		\delta_2 \ge \delta \Ra a + \delta_2 \ge a + \delta
	\end{align*}
	Рассмотрим $\forall x \in \mc{U}_\delta(a)$:
	\begin{align*}
		a < x < a + \delta \Ra a < x < a + \delta_2
		\\
		a - \delta < x < a \Ra a - \delta_1 < x < a
	\end{align*}
	Любой из этих случаев ведёт к тому, что $f(x) \in U_{\eps}(A)$. А значит
	\[
		(\forall \eps > 0)(\exists \delta > 0)
		\left(\forall x,\ x \in \mc{U}_{\delta}(a)\right)
		\ f(x) \in U_{\eps}(A)
	\]
	Что равносильно левой стороне утверждения.
\end{enumerate}

\end{proof}

\begin{definition}
	Функция $f(x)$ называется
	\begin{itemize}
		\item \textit{неубывающей} на $X$, если 
			$(\forall x_1, x_2 \in X, x_1 < x_2)
			\Ra f(x_1) \le f(x_2)$
		\item \textit{невозрастающей} на $X$, если 
			$(\forall x_1, x_2 \in X, x_1 < x_2)
			\Ra f(x_1) \ge f(x_2)$
		\item \textit{убывающей} на $X$, если 
			$(\forall x_1, x_2 \in X, x_1 < x_2)
			\Ra f(x_1) > f(x_2)$
		\item \textit{возрастающей} на $X$, если 
			$(\forall x_1, x_2 \in X, x_1 < x_2)
			\Ra f(x_1) < f(x_2)$
	\end{itemize}
	В любом из этих случаев $f$ монотонна на $X$, в 2х последних $f$ строго монотонна на $X$.
\end{definition}

\begin{definition}
	$\sup f(x) := \sup \left\{f(x) : x \in X\right\}$ 
\end{definition}

\begin{theorem} (Существование односторонних пределов монотонной функции)
	
	\begin{itemize}
		\item Если $f$ невозрастающая на $(a, b), -\infty \le a
			< b < +\infty$, то $f(b - 0) = \inf f(x); \ x \in (a, b)$
		\item Если $f$ неубывающая на $(a, b), -\infty \le a
			< b < +\infty$, то $f(b - 0) = \sup f(x); \ x \in (a, b)$
		\item Если $f$ невозрастающая на $(a, b), -\infty < a
			< b \le +\infty$, то $f(a + 0) = \sup f(x); \ x \in (a, b)$
		\item Если $f$ неубывающая на $(a, b), -\infty < a
			< b \le +\infty$, то $f(a + 0) = \inf f(x); \ x \in (a, b)$
	\end{itemize}
\end{theorem}

\begin{proof}
	Пусть $f$ неубывающая. Положим $\sup\limits_{x \in (a; b)} f(x) := M$
	\begin{enumerate}
		\item $M = +\infty$. Тогда
		\[
			(\forall \eps > 0)(\exists x_0 \in (a; b))\ f(x_0)
			 > \frac{\dse 1}{\dse \eps}
		\]
		Отсюда $(\exists \delta := b - x_0 > 0)
		(\forall x, b - \delta < x < b) \Ra
		\frac{\dse 1}{\dse \eps} < f(x_0) \le f(x)$,
		то есть $f(x) \in U_{\eps}(+\infty)$. В итоге
		\[
			(\forall \eps > 0)(\exists \delta > 0)
			(\forall x,\ b - \delta < x < b)\ f(x)
			 \in U_{\eps}(M) \lra \liml_{x \to b-0} f(x) = +\infty = M
		\]
		
		\item $M \in \R$. Тогда
		\[
			(\forall \eps > 0)(\exists x_0 \in (a; b))
			\ f(x_0) \in (M - \eps; M]
		\]
		Отсюда уже аналогично получим, что
		\[
			(\forall \eps > 0)(\exists \delta > 0)
			(\forall x,\ b - \delta < x < b)\ f(x)
			 \in U_{\eps}(M) \lra \liml_{x \to b-0} f(x) = M
		\]
		Если $a = -\infty$, то $\liml_{x \to -\infty}
		f(x)$ вместо $\liml_{x \to a+0} f(x)$
		
		Если $b = +\infty$, то $\liml_{x \to +\infty}
		f(x)$ вместо $\liml_{x \to b-0} f(x)$
		
	\end{enumerate}
\end{proof}

\subsection{Непрерывность}

\subsubsection*{Непрерывность в точке}

\begin{definition}
	Пусть $f$ определена в некоторой окрестности
	точки $x_0 \in \R$. Тогда функция $f$ называется
	\textit{непрерывной} в точке $x_0$, если
	$\liml_{x \to x_0} f(x) = f(x_0)$.
\end{definition}

\begin{definition}
	Пусть $f$ определена на полуинтервале $(a, x_0],\ a < x_0$.
	Тогда $f$ \textit{непрерывна слева} в т. $x_0$, если
	$\liml_{x \to x_0 - 0} f(x) = f(x_0) \ \ (f(x_0 - 0) = f(x_0))$
\end{definition}

\begin{definition}
	Пусть $f$ определена на полуинтервале $[x_0, b),\ b > x_0$.
	Тогда $f$ \textit{непрерывна справа} в т. $x_0$, если
	$\liml_{x \to x_0 + 0} f(x) = f(x_0) \ \ (f(x_0 + 0) = f(x_0))$
\end{definition}

\begin{theorem}
	Пусть $f$ определена в некоторой окрестности
	точки $x_0 \in \R$. Тогда, следующие утверждения эквивалентны:
	\begin{enumerate}
		\item $f$ непрерывна в точке $x_0$
		\item $(\forall \eps > 0)(\exists \delta > 0)\left(\forall x,\ |x - x_0| < \delta\right)
	    	\ |f(x) - f(x_0)| < \eps$
		\item $\left(\forall \{x_n\}_{n = 1}^\infty \subset D(f),\ \liml_{n \to \infty} x_n = x_0\right) \liml_{n \to \infty} f(x_n) = f(x_0)$
	\end{enumerate}
\end{theorem}

\subsubsection*{Точки разрыва}

\begin{definition}
	Пусть $f$ определена в проколотой окрестности точки $x_0$.
	Если $\liml_{x \to x_0} f(x) \neq f(x_0)$,
	то $x_0$ называется \textit{точкой разрыва} функции $f(x)$.
\end{definition}

\begin{note}
	Неравенство полагается верным также и в тех случаях, когда хоть одна из частей не определена.
\end{note}

\begin{definition}
	Если $\exists \liml_{x \to x_0-0} f(x),\ \liml_{x \to x_0+0} f(x) \in \R$, то точка разрыва называется \textit{точкой разрыва первого рода}.
	
	В противном случае \textit{точкой разрыва второго рода}.
\end{definition}

\begin{definition}
	Если $\liml_{x \to x_0-0} f(x) = \liml_{x \to x_0+0} f(x) \in \R$ и $\neq f(x_0)$, то $x_0$ называется \textit{точкой устранимого разрыва}.
\end{definition}

\begin{definition}
	Если хотя бы 1 из односторонних пределов бесконечен, то $x_0$ называется \textit{точкой бесконечного разрыва}.
\end{definition}

\begin{definition}
	Величину $\liml_{x \to x_0+0} f(x) - \liml_{x \to x_0-0} f(x)$ называется \textit{скачком функции} в точке $x_0$.
\end{definition}

\begin{example}
	$f(x) = \sgn x = \System{&{1,\ x > 0} \\ &{0,\ x = 0} \\ &{-1,\ x < 0}}
	\Ra x_0 = 0$ --- неустранимая точка разрыва (I рода)
	\begin{align*}
		&\liml_{x \to 0 + 0} f(x) = 1
		\\
		&\liml_{x \to 0 - 0} f(x) = -1
	\end{align*}
\end{example}

\begin{example}
	$f(x) = \sgn^2 x \Ra $  $x_0 = 0$ --- точка устранимого разрыва
	$$
		\liml_{x \to 0} \sgn^2 x = 1 \neq \sgn^2 0 = 0
	$$
\end{example}

\begin{example}
	$f(x) = \frac{1}{x} \Ra x_0 = 0$ --- точка разрыва второго рода 
	\begin{align*}
		\liml_{x \to -0} \frac{\dse 1}{\dse x} = -\infty
		\\
		\liml_{x \to +0} \frac{\dse 1}{\dse x} = +\infty
	\end{align*}
\end{example}

\begin{example}
	$f(x) = \sin \frac{\dse 1}{\dse x}$
	
	Рассмотрим
	$$
		\System{
		&{x'_n = \frac{1}{\frac{\pi}{2} + 2\pi n},\ n \in \N}
		\\
		&{x''_n = \frac{1}{-\frac{\pi}{2} + 2\pi n}}
		}
		\Ra
		\System{
		&{\liml_{n \to \infty} f(x'_n) = 1}
		\\
		&{\liml_{n \to \infty} f(x''_n) = -1}
		}
	$$
\end{example}

\subsubsection*{Следствия свойств предела функции}

\begin{enumerate}
	\item (Ограниченность непрерывной функции) Если $f$ непрерывна в $x_0$, то она ограничена в некоторой окрестности точки $x_0$.
	
	\item (Отделимость от нуля и сохранение знака непрерывной функции)
	Если $f$ непрерывна в $x_0$ и $f(x_0) \neq 0$, то 
	$(\exists c > 0)(\exists \delta > 0)(\forall x \in U_\delta (x_0))\ |f(x)| > c,
	\ \sgn f(x) = \sgn f(x_0)$

	\item (Арифметические операции с непрерывными функциями) Если $f$ и $g$ непрерывны в $x_0$, то $f \pm g$, $f \cdot g$ и (если $g(x_0) \neq 0$) $\frac{f}{g}$ непрерывны в $x_0$.
\end{enumerate}

\begin{definition}
	Композицией функций $f$ и $g$ называется 
	$$
		(g \circ f)(x) := g(f(x))
	$$
\end{definition}

\begin{theorem} (Переход к пределу под знаком непрерывной функции) \\
	Если $\liml_{x \to a(\pm 0)} f(x) = b$ и $g(y)$ непрерывна в точке
	$b \in \R$, то $\liml_{x \to a(\pm 0)} (g \circ f)(x) = g(b)$
\end{theorem}

\begin{proof}
	Рассмотрим $\left(\forall \{x_n\}_{n = 1}^\infty \subset D(f),
	\ x_n \underset{\left(\underset{<}{>}\right)}{\neq} a,\ \liml_{n \to \infty} x_n = a\right)\ \liml_{n \to \infty} f(x_n) = b$
	
	Положим $y_n := f(x_n)$
	$$
		\left(\{y_n\}_{n = 1}^\infty \subset D(g),
		\ \liml_{n \to \infty} y_n = b\right) \Ra
		\liml_{n \to \infty} g(y_n) = g(b) 
	$$
\end{proof}

\begin{note} (Предел сложной функции)
	Для того, чтобы из $\liml_{x \to a} f(x) = b$ и
	$\liml_{y \to b} g(y) = l$ следовало $\liml_{x \to a}
	(g \circ f)(x) = l$, достаточно потребовать, чтобы
	$f(x) \neq b$ в ни в одной точке некоторой проколотой
	окрестности точки $a$.
\end{note}

\begin{addition} (Следствие теоремы выше. Непрерывность сложной функции)
	Если $f$ непрерывна в $a$, $g$ непрерывна в $f(a)$, то $g \circ f$ непрерывна в $a$.
\end{addition}

\begin{theorem} (О точках разрыва монотонной функции)
	Если $f(x)$ монотонна на $(a; b),\ -\infty 
	\le a < b \le +\infty$, то она может иметь на $(a; b)$ не более
	чем счетное множество точек разрыва, причем все эти точки - 1го
	рода, но не устранимые разрывы.
\end{theorem}

\begin{proof}
	$(\forall x_0 \in (a; b))\ \exists \ f(x_0 - 0),\ f(x_0 + 0) \in \R$
	
	Пусть $f$ невозрастающая, тогда

	$f(x_0 - 0) = \inf f(x),\ x \in (a, x_0);\ f(x_0 + 0) = \sup f(x),\ x\in (x_0, b)$

	$x \in (a, x_0) \Ra f(x) \ge f(x_0);\ x \in (x_0, b) \Ra f(x) \le f(x_0)$

	$f(x_0 - 0) \ge f(x_0) \ge f(x_0 + 0)$ --- Чтобы точка разрыва существовала,
	должно быть хотя бы одно строгое неравенство, значит, разрыв 1го рода,
	причем не устранимый $(f(x_0 - 0) \neq f(x_0 + 0))$

	Счётность: Пусть $x_1 < x_2$ --- точки разрыва, тогда
	
	$f(x_1 - 0)
	\ge f(x_1) \ge f(x_1 + 0) \ge f(x_2 - 0) \ge f(x_2) \ge
	f(x_2 + 0)$

	Интервал $(f(x_1 + 0), f(x_1 - 0))$ не пересекается с
	$(f(x_2 + 0), f(x_2 - 0))$. Поставим в соответствие интервалам
	рациональные числа внутри них, следовательно, таких интервалов
	не более чем счетное количество, а значит, и множество точек разрыва
	не более чем счётно.

\end{proof}

\begin{example} (Функция Римана)
	\[
		f(x) = \System{&{\frac{1}{n}, \text{ если } x = \frac{m}{n}} \\ &{0, \text{ если } x \in \R \bs \Q}}
	\]
	
	Докажем, что $f$ непрерывна в $x_0 \in \R \bs \Q$: зафиксируем произвольный $\eps > 0$ и рассмотрим множество
	\[
		M = \{x \such f(x) \ge \eps\}
	\]
	Так как $\eps > 0$ и $f(x) = 0\ \forall x \in \R \bs \Q$, то любой элемент $M$ - рациональное число, имеющее вид в несократимой дроби $\frac{m}{n}, m \in \Z, n \in \N$.
	\[
		f(x) = \frac{1}{n} \ge \eps \Ra n \le \frac{1}{\eps}
	\]
	То есть число таких $n$ конечно. Это значит, что число рациональных точек, попавших в $U_\delta(x_0) \cap M$, конечно (в самом деле, бесконечность может достигаться только за счёт $m$, а это мы ограничили пересечением). Ну а раз так, то найдётся $\delta > 0$ такое, что $U_\delta(x_0) \cap M = \emptyset$. Иными словами,
	\[
		(\forall \eps > 0)(\exists \delta > 0)(\forall x \in U_\delta(x_0))\ f(x) < \eps
	\]
	это означает непрерывность функции Римана в любой иррациональной точке.
	
	Теперь докажем, что $f(x)$ разрывна во всех рациональных точках. Пусть $x_0 \in \Q$ и мы снова зафиксировали $\eps > 0$. Какую $\delta$-окрестность точки $x_0$ ни взять, там найдётся иррациональное число, для которого $f(x) = 0 \Ra$ получим разрывность.
	
	Таким образом, функция Римана непрерывна $\forall x \in \R \bs \Q$ и разрывна $\forall x \in \Q$.
\end{example}

\subsubsection*{Непрерывность на множестве}

\begin{definition}
	Функция называется \textit{непрерывной на множестве} $X$, если 
	\[
		(\forall x_0 \in X)\left(\forall \{x_n\}
		\subset X, \liml_{n \to \infty} x_n = x_0 \Ra
		\liml_{n \to \infty} f(x_n) = f(x_0)\right)
	\]
	или по Коши
	\[
		(\forall x_0 \in X)(\forall \eps > 0)(\exists \delta > 0)
		(\forall x \in X,\ |x - x_0| < \delta)\ \left|f(x) - f(x_0)\right| < \eps
	\]
\end{definition}

\begin{note}
	Не стоит думать, что непрерывность на множестве - это непрерывность в каждой точке этого множества. Это не так. Как минимум потому, что мы не требуем определённость функции в некоторой окрестности точки из $X$.
\end{note}