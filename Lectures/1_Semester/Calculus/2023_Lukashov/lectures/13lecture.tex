\subsection{Сравнение функций}

\begin{definition}
	Пусть $f(x) = \lambda(x) \cdot g(x)$ справедливо в некоторой
	проколотой окрестности точки $x_0$.
	\begin{enumerate}
		\item Если $\lambda(x)$ ограничена в
		этой проколотой окрестности, то
		$f(x) = O(g(x))$ при $x \to x_0$
		
		\item Если $\liml_{x \to x_0} \lambda(x) = 0$,
		то $f(x) = o(g(x))$ при $x \to x_0$
		
		\item Если $\liml_{x \to x_0} \lambda(x) = 1$,
		то $f(x) \sim g(x)$ при $x \to x_0$. ($f$ эквивалентно $g$)
	\end{enumerate}
\end{definition}

\begin{theorem}
	Другое определение. Если $g(x) \neq 0$ в некоторой проколотой
	окрестности точки $x_0$, то
	\begin{enumerate}
		\item $\frac{f(x)}{g(x)}$ ограничена в
		некоторой проколотой окрестности $x_0$, то
		$f(x) = O(g(x))$ при $x \to x_0$
		
		\item $\liml_{x \to x_0} \frac{f(x)}{g(x)} = 0$, то
		$f(x) = o(g(x))$ при $x \to x_0$
		
		\item $\liml_{x \to x_0} \frac{f(x)}{g(x)} = 1$, то
		$f(x) \sim g(x)$ при $x \to x_0$
	\end{enumerate}
\end{theorem}

\begin{proof}
	Положим $\lambda(x) := \frac{f(x)}{g(x)}$. А дальше всё уже следует из сказанного выше.
\end{proof}

\begin{examples}~
	\begin{enumerate}
		\item $\sin x \cdot \sin \frac{1}{x} =
		O(\sin \frac{1}{x})$ при $x \to 0$.

		\item $x \cdot \sin \frac{1}{x}
		= o(\sin \frac{1}{x})$ при $x \to 0$.

		\item $\sin x \cdot \sin \frac{1}{x}
		\sim x \cdot \sin \frac{1}{x}$ при $x \to 0$.
	\end{enumerate}
\end{examples}

\begin{theorem} (Связь эквивалентных и б.м.)
	$f(x) \sim g(x)$ при $x \to x_0 \lra f(x) = g(x)
	+ o(g(x))$ при $x \to x_0$.
\end{theorem}

\begin{proof}
	Необходимость $(\Ra)$. Положим $f(x) \sim g(x)$ при $x \to x_0$. Тогда
	\[
		f(x) = \lambda(x) \cdot g(x),\ \liml_{x \to x_0} \lambda(x) = 1 \Ra
	\]
	\[
		f(x) - g(x) = (\lambda(x) - 1) \cdot g(x),\ \liml_{x \to x_0} (\lambda(x) - 1) = 0 \Ra f(x) - g(x) = o(g(x))
	\]
	Достаточность доказывается аналогично. 
\end{proof}

\begin{theorem} (Использование эквивалентных при вычислении пределов)
	
	Если $f_1(x) \sim f_2(x)$ и $g_1(x) \sim 
	g_2(x)$ при $x \to x_0$ , то
	\[
		\liml_{x \to x_0} f_1(x) \cdot g_1(x) =
		\liml_{x \to x_0} f_2(x) \cdot g_2(x)
	\]
	А также
	\[
		\liml_{x \to x_0} \frac{g_1(x)}{f_1(x)} =
		\liml_{x \to x_0} \frac{g_2(x)}{f_2(x)}
	\]
	при условии, что хотя бы один из пределов в
	каждом равенстве существует
\end{theorem}

\begin{proof}~

	Из условия следует: $f_1(x) = \lambda_1(x) \cdot f_2(x)$ и
	$g_1(x) = \lambda_2(x) \cdot g_2(x)$

	\[
		f_1(x) \cdot g_1(x) = \lambda_1(x) \cdot \lambda_2(x) \cdot 
		g_2(x) \cdot f_2(x) \Ra
	\]
	\[
		\liml_{x \to x_0} \left(f_1(x) \cdot g_1(x)\right) =
		\liml_{x \to x_0} \left(\lambda_1(x) \cdot \lambda_2(x) \cdot 
		g_2(x) \cdot f_2(x)\right) = \liml_{x \to x_0}
		\left(f_2(x) \cdot g_2(x)\right)
	\]
	
	Для дробей доказательство аналогично.
\end{proof}

\begin{proposition}
	\[
		\sin x \sim \tg x \sim (e^x - 1) \sim \ln(1 + x)
		\sim \sh x \sim \th x \sim \arcsin x \sim \arctg x
		\sim x,\ x \to 0
	\]
	\[
		\sh x := \frac{e^x - e^{-x}}{2};\ 
		\ch x := \frac{e^x + e^{-x}}{2};\ 
		\th x := \frac{\sh x}{\ch x};\ 
		\cth x := \frac{\ch x}{\sh x}
	\]
	\[
		\ch^2 x - \sh^2 x = 1;\ \sh(2x) = 2 \sh x \ch x;
		\ \ch(2x) = \ch^2 x + \sh^2 x
	\]
\end{proposition}

%\begin{note}
%	Мы можем писать $o(f) = O(f)$ и подобное, подразумевая, что на самом деле мы рассматриваем некоторое $g = o(f)$
%\end{note}


\begin{definition}
	Если $f = O(g)$ и $g = O(f)$ при $x \to x_0$,
	то говорят, что $f \asymp g$
\end{definition}


\section{Дифференциальное исчисление функций одной переменной}

\subsection{Производная}

\begin{definition}
	Пусть $y = f(x)$ определена в некоторой окрестности
	точки $x_0 \in \R$.
	
	\textit{Приращением} $\Delta y$ этой функции в
	точке $x_0$, отвечающим приращению аргумента
	$\Delta x$, называется $\Delta y = f(x_0 + \Delta x) - f(x_0)$.
	
	\textit{Производной} функции $y = f(x)$ в точке $x_0$ называется предел (если он существует и конечен)
	\[
		\liml_{\Delta x \to 0} \frac{\Delta y}{\Delta x} = \liml_{x \to x_0} \frac{f(x) - f(x_0)}{x - x_0} =: f'(x_0)
	\]
\end{definition}

\begin{theorem} \label{diff_function_to_cont}
	Если функция имеет производную в точке $x_0$,
	то она непрерывна в этой точке.
\end{theorem}

\begin{proof}
	По условию,
	\[
		\liml_{x \to x_0} \frac{f(x) - f(x_0)}{x - x_0} =
		f'(x_0) + \alpha(x),\ \liml_{x \to x_0} \alpha(x) = 0
	\]
	Следовательно,
	\[
		f(x) - f(x_0) = f'(x_0)(x - x_0) + \alpha(x)(x - x_0)
	\]
	при $x \to x_0$. Перенесём $f(x_0)$ в другую сторону
	и в предельном переходе получим, что
	\[
		\liml_{x \to x_0} f(x) = f(x_0) + 0 + 0 = f(x_0)
	\]
\end{proof}

\begin{theorem} (Арифметические операции и производные)
	Если $\exists f'(x_0)$ и $g'(x_0)$, то
	\begin{enumerate}
		\item $\exists (f \pm g)'(x_0) = f'(x_0) \pm g'(x_0)$
		
		\item $\exists (f \cdot g)'(x_0) =
		f'(x_0) \cdot g(x_0) + f(x_0) \cdot g'(x_0)$
		
		\item Если $g(x_0) \neq 0$, то
		$\exists \left(\frac{f}{g}\right)'(x_0) =
		\frac{f'(x_0) \cdot g(x_0) - f(x_0) \cdot g'(x_0)}{g^2(x_0)}$
	\end{enumerate}
\end{theorem}

\begin{proof}~
	\begin{enumerate}
		\item 
		\[
			f'(x_0) = \liml_{x \to x_0} \frac{f(x) - f(x_0)}{x - x_0},\ 
			g'(x_0) = \liml_{x \to x_0} \frac{g(x) - g(x_0)}{x - x_0}
		\]
		Отсюда
		\[
			\liml_{\Delta x \to 0}
			\frac{\Delta (f(x) \pm g(x))}{\Delta x} = 
			\liml_{x \to x_0} \frac{(f(x) \pm g(x)) -
			(f(x_0) \pm g(x_0))}{x - x_0} = \liml_{x \to x_0}
			\frac{f(x) - f(x_0)}{x - x_0} \pm \liml_{x \to x_0}
			\frac{g(x) - g(x_0)}{x - x_0}
		\]
		
		\item Аналогично первому,
		\begin{multline*}
			\liml_{x \to x_0} \frac{(f \cdot g)(x) - (f \cdot g)(x_0)}{x - x_0} =
			\liml_{x \to x_0} \frac{f(x)g(x) - f(x_0)g(x_0) + f(x_0)g(x) - f(x_0)g(x)}{x - x_0} =\\
			= \liml_{x \to x_0} \frac{f(x)g(x) - f(x_0)g(x)}{x - x_0} +
			\liml_{x \to x_0} \frac{f(x_0)g(x) - f(x_0)g(x_0)}{x - x_0} = \\
			= \liml_{x \to x_0} \frac{f(x) - f(x_0)}{x - x_0} \cdot g(x) + f(x_0)
			\cdot \liml_{x \to x_0} \frac{g(x) - g(x_0)}{x - x_0}
		\end{multline*}
		Что в предельном переходе даёт
		\[
			f'(x_0) \cdot g(x_0) + f(x_0) \cdot g'(x_0).
		\]
		
		\item Аналогично первому и второму,
		\begin{multline*}
			\liml_{x \to x_0} \frac{(\frac{f}{g})(x) -
			(\frac{f}{g})(x_0)}{x - x_0} =
			\liml_{x \to x_0} \frac{f(x)g(x_0) - f(x_0)g(x)}{g(x)g(x_0)(x - x_0)} =\\
			= \liml_{x \to x_0} \frac{f(x)g(x_0) - f(x_0)g(x_0)}{g(x)g(x_0)(x - x_0)} +
			\liml_{x \to x_0} \frac{f(x_0)g(x_0) - f(x_0)g(x)}{g(x)g(x_0)(x - x_0)}
		\end{multline*}
		В предельном переходе получим
		\[
			\frac{g(x_0)f'(x_0)}{g^2(x_0)} - \frac{f(x_0)g'(x_0)}{g^2(x_0)}
		\]
	\end{enumerate}
\end{proof}

\begin{theorem} (Производные основных элементарных функций)
	Для всех $x_0$ из областей определения соответствующих
	функций справедливы равенства:
	\begin{enumerate}
		\item $(\sin x)' \big|_{x = x_0} = \cos x_0$
		\item $(\cos x)' \big|_{x = x_0} = -\sin x_0$
		\item $(\tg x)' \big|_{x = x_0} = \frac{1}{\cos^2 x_0}$
		\item $(\ctg x)' \big|_{x = x_0} = -\frac{1}{\sin^2 x_0}$
		\item $(x^b)' \big|_{x = x_0} = b \cdot x_0^{b - 1}\ (x > 0)$
		\item $(b^x)' \big|_{x = x_0} = b^{x_0} \ln b$
		\item $(\sh x)' \big|_{x = x_0} = \ch x_0$
		\item $(\ch x)' \big|_{x = x_0} = \sh x_0$
		\item $(\th x)' \big|_{x = x_0} = \frac{1}{\ch^2 x_0}$
		\item $(\cth x)' \big|_{x = x_0} = -\frac{1}{\sh^2 x_0}$
	\end{enumerate}
\end{theorem}

\begin{proof}
	Рутинно
	\begin{enumerate}
		\item 
			$\liml_{x \to x_0} \frac{\sin x - \sin x_0}{x - x_0} =
			\liml_{x \to x_0} \frac{2 \sin \frac{x - x_0}{2}
			\cos \frac{x + x_0}{2}}{x - x_0} =
			\left[\sin \frac{x - x_0}{2} \sim \frac{x - x_0}{2},\ x \to x_0\right]
			= \liml_{x \to x_0}	\cos \frac{x + x_0}{2} = \cos x_0$
			
		\item 
			$\liml_{x \to x_0} \frac{\cos x - \cos x_0}{x - x_0} =
			\liml_{x \to x_0} \frac{-2 \sin \frac{x - x_0}{2}
			\sin \frac{x + x_0}{2}}{x - x_0} =
			\liml_{x \to x_0} (-\sin \frac{x + x_0}{2}) = -\sin x_0$
		
		\item Аналогично
		\item Аналогично

		\item 
		\begin{multline*}
			\liml_{x \to x_0} \frac{x^b - x_0^b}{x - x_0} = 
			x_0^b \liml_{x \to x_0} \frac{\left(\frac{x}{x_0}\right)^b - 1}{x - x_0} = \\
			= x_0^b \liml_{x \to x_0} \frac{e^{b \ln \frac{x}{x_0}} - 1}{x - x_0} =
			\left[e^{b \ln \frac{x}{x_0}} - 1 \sim
			b \ln \frac{x}{x_0},\ x \to x_0\right] =
			x_0^b \liml_{x \to x_0} \frac{b \ln \frac{x}{x_0}}{x - x_0} = \\
		    = x_0^b \cdot b \cdot \liml_{x \to x_0}
			\frac{\ln(1 + \frac{x - x_0}{x_0})}{x - x_0} =
			\left[\ln \left(1 + \frac{x - x_0}{x_0}\right) \sim
			\frac{x - x_0}{x_0},\ x \to x_0\right] =
			x_0^b \cdot b \cdot \liml_{x \to x_0} \frac{x - x_0}{x_0(x - x_0)}
			= b \cdot x_0^{b - 1}
		\end{multline*}
		
		\item
		\[
			\liml_{x \to x_0} \frac{b^x - b^{x_0}}{x - x_0} =
			b^{x_0} \liml_{x \to x_0} \frac{e^{(x - x_0)\ln b } - 1}{x - x_0} =
			\left[ e^{(x - x_0)\ln b} - 1 \sim  (x - x_0)\ln b,\ x \to x_0 \right] =
			b^{x_0} \cdot \ln b
		\]
		
		\item
		\[
			(\sh x)' \big|_{x = x_0} =
			\left(\frac{e^x - e^{-x}}{2}\right)' =
			\frac{1}{2}\left(e^{x_0} -
			\left(\frac{1}{e^x}\right)'\right) =
			\frac{1}{2}\left(e^{x_0} - \frac{-e^{x_0}}{e^{2a}}\right) =
			\frac{e^{x_0} + e^{-x_0}}{2} = \ch x_0
		\]
		
		\item Аналогично
		\item Аналогично

		\item
			$(\cth x)' \big|_{x = x_0} =
			\left(\frac{\ch x}{\sh x}\right)' =
			\frac{(\ch x)' \cdot \sh x_0 - \ch x_0 \cdot
			(\sh x)'}{\sh^2 x_0} =\frac{\sh^2 x_0 - \ch^2 x_0}{\sh^2 x_0} =
			-\frac{1}{\sh^2 x_0}$
	\end{enumerate}
\end{proof}

\begin{theorem} \label{inverse_function_derivative}
	(Производная обратной функции)
	Если $f(x)$ непрерывна и строго монотонна на
	$[x_0 - \delta,\ x_0 + \delta],\ \delta > 0$ и $\exists f'(x_0) \neq 0$,
	то обратная функция $f^{-1}$ имеет производную в точке
	$f(x_0)$, равную
	\[
		(f^{-1})'(f(x_0)) = \frac{1}{f'(x_0)}
	\]
\end{theorem}

\begin{proof}
	Во первых, обратная функция определена, непрерывна и
	строго монотонна на интервале $f([x_0 - \delta,\ x_0 + \delta])$ по
	теореме об обратной функции \ref{inverse_function}.
	Для краткости обозначим $\varphi = f^{-1}$. Для определённости
	будем считать $f$ - возрастающей функцией. Тогда $\varphi$
	определена на $y \in [f(x_0 - \delta); f(x_0 + \delta)],\ y_0 = f(x_0)$.
	По определению производной, нам надо найти предел
	\[
		\liml_{\Delta y \to 0} \frac{\Delta \varphi(y_0)}{\Delta y}
	\]
	$\varphi$ непрерывна в т. $y_0 \Ra
	\liml_{\Delta y \to 0} \Delta \varphi(y_0) = 0$
	по теореме о непрерывности функции, имеющей производную в точке
	\ref{diff_function_to_cont}. $\varphi$ возрастает и
	$\Delta y \neq 0 \Ra \Delta \varphi(y_0) \neq 0$. Тогда пусть
	\[
		\Delta x := \Delta \varphi(y_0) =
		\varphi(y_0 + \Delta y) - \varphi(y_0)
	\]
	Тогда
	\[
		\varphi(y_0 + \Delta y) = \varphi(y_0) + \Delta x =
		x_0 + \Delta x \Ra y_0 + \Delta y = f(x_0 + \Delta x)
	\]
	То есть
	\[
		f(x_0 + \Delta x) - y_0 = y_0 + \Delta y - y_0 = \Delta y
	\]
	Отсюда исходный предел выражается как
	\[
		\frac{\varphi(y_0 + \Delta y) -
		\varphi(y_0)}{\Delta y} =
		\frac{\Delta x}{\Delta y} =
		\frac{1}{\frac{\Delta y}{\Delta x}} =
		\underset{\Delta x \to 0}{\to}		
		\frac{1}{f'(x_0)}
	\]
	\[
		\liml_{\Delta y \to 0} \frac{\Delta \varphi(y_0)}{\Delta x}=
		\liml_{\Delta x \to 0} \frac{1}{\frac{\Delta y}{\Delta x}} =
		\frac{1}{f'(x_0)}
	\]
\end{proof}

\begin{anote}
	Возможно, корректность подмены
	$\Delta y$ на $\Delta x$ в последнем переходе
	неочевидна для читателя. В таком случае обратим внимание
	на то, что справедливость этого преобразования следует
	из определения предела по Гейне:
	\begin{align*}
		& \left(\forall \{\Delta x_n\}
		\subset [- \delta,\ \delta] \bs \{0\}\ \liml_{n \to \infty} \Delta x_n = 0\right)
		\ \liml_{n \to \infty} f(\Delta x_n) = A\\
		& \left(\forall \{\Delta y_n\}
		\subset f([- \delta,\ \delta]) \bs \{0\}\ \liml_{n \to \infty} \Delta y_n = 0\right)
		\ \liml_{n \to \infty} f(\Delta y_n) = B
	\end{align*}
	Из определения $\Delta x$ в доказательстве ясно, что
	$\liml_{\Delta y \to 0} \Delta x = 0$ (это верно в силу непрерывности
	$\varphi$ и $f$). Тогда скажем, что $\Delta x$ --- это функция от
	$\Delta y$.
	\begin{align*}
		& \left(\forall \{\Delta y_n\}
		\subset f([- \delta,\ \delta]) \bs \{0\}\ 
		\liml_{n \to \infty} \Delta y_n = 0\right)
		\ \liml_{n \to \infty} \Delta x(\Delta y_n) = 0
	\end{align*}
	Осталось только заменить в изначальном определении $\Delta y_n$
	на $\Delta x_n$, где $\Delta x_n = \Delta x(\Delta y_n)$.
	Выполнив данное действие, мы увидим, что $A = B$, то есть мы
	совершенно безнаказанно заменили $\Delta y$ на $\Delta x$
	внутри предела (читатель может убедиться собственноручно,
	что так же справедлива замена $\Delta x$ на $\Delta y$).
\end{anote}