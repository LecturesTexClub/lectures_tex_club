\subsubsection*{Отношения порядка на множестве действительных чисел}

\begin{definition}
    Действительное число $a$ называется 
    положительным ($a > 0,\  0 < a$), если
    ($\exists \{[p_n, q_n]_{\Q}\}_{n = 1}^{\infty} \in a)
    (\exists n \in \N)\ p_n > 0$

    Для отрицательного аналогично:
    ($\exists \{[p_n, q_n]_{\Q}\}_{n = 1}^{\infty} \in a)
    (\exists n \in \N)\ q_n < 0$
\end{definition}

\begin{definition}
    Действительное число $a$ состоит в отношении $>$ с
    числом $b$, если $a + (-b) > 0$. Для $<$ аналогично.
\end{definition}

\begin{definition}
    Отношение порядка на множестве действительных
    чисел задаётся как
    \[
        a \le b := (a = b) \vee (a < b)
    \]
\end{definition}

\begin{theorem}
    $\R$ является линейно упорядоченным множеством
    относительно $\le$. То есть выполняются свойства:

    \begin{enumerate}
        \item[III-а).] ($\forall a \in \R)\ a \le a$ \textit{(рефлексивность)}
        \item[III-б).] $(\forall a, b \in \R)((a \le b) \wedge 
            (b \le a)) \Ra (a = b)$ \textit{(антисимметричность)}
        \item[III-в).] $(\forall a, b, c \in \R)((a \le b)
            \wedge (b \le c)) \Ra (a \le c)$ \textit{(транзитивность)}
        \item[III-г).] $(\forall a, b \in \R)\ ((a \le b) \vee (b \le a))$ \textit{(линейный порядок)}
    \end{enumerate}
\end{theorem}

\begin{proof}~

    \begin{itemize}
        \item \textit{Рефлексивность} верна, так как $a = a$.
        \item \textit{Антисимметричность.} По условию имеем:
            \begin{align*}
                &(a \le b) \lra (a < b) \vee (a = b) \lra (b - a > 0) \vee (a = b) \\
                &(b \le a) \lra (b < a) \vee (a = b) \lra (a - b > 0) \vee (a = b)
            \end{align*}
            Докажем, что $(x > 0) \wedge (x < 0)$ есть ложь:
            \begin{align*}
                &(a < 0) \lra (\exists \{[p_n, q_n]_{\Q}\}_{n = 1}^{\infty} \in a)
                (\exists n \in \N_1)\ p_n > 0 \Ra (\forall n > N_1)\ q_n \ge p_n > 0\\
                &(a > 0) \lra (\exists \{[p'_n, q'_n]_{\Q}\}_{n = 1}^{\infty} \in a)
                (\exists n \in \N_2)\ q'_n < 0 \Ra (\forall n > N_2)\ p'_n \le q'_n < 0\\
            \end{align*}
            Это значит, что
            \[
                (\forall n > \max(N_1, N_2))\ \max(q'_n, q_n) - \min(p'_n, p_n) = q_n - p'_n
            \]
            \[
                q_n - p'_n \ge p_n - q'_n \ge p_{\max(N_1, N_2)} - q'_{\max(N_1, N_2)}        
            \]
            Это значит, что определение равенства гнезд нарушается:
            \begin{multline*}
                (\exists \veps = p_{\max(N_1, N_2)} - q'_{\max(N_1, N_2)})
                (\forall N \in \N)\\ (\exists n = \max(N, N_1, N_2) + 1 > N)
                \ \max(q'_n, q_n) - \min(p'_n, p_n) \ge \veps
            \end{multline*}

            То есть случай, где верны только строгие неравенства,
            невозможен. Значит, как минимум одно из равенств
            выполняется, тогда $a = b$.
            
        \item \textit{Транзитивность.} По условию имеем:
            \begin{align*}
                &(a \le b) \lra (a < b) \vee (a = b) \lra (b - a > 0) \vee (a = b) \\
                &(b \le c) \lra (b < c) \vee (b = c) \lra (c - b > 0) \vee (b = c)
            \end{align*}
            Необходимо показать, что верно высказывание
            \[
                ((c - a > 0) \vee (a = c)) \equiv 1
            \]
            Так как одновременно из одного неравенства
            оказаться верной может только одна скобка, то
            рассморим все случаи:
            \begin{enumerate}
                \item $(a = b) \wedge (b = c) \Ra a = c$
                \item $(a = b) \wedge (c - b > 0) \Ra c - a > 0$
                \item $(b - a > 0) \wedge (b = c) \Ra c - a > 0$
                \item $(b - a > 0) \wedge (c - b > 0) \Ra
                    (c - b) + (b - a) = c - a > 0$
            \end{enumerate}
        \item \textit{Линейный порядок.} Нам необходимо доказать,
            что $\forall a, b \in \R$ верно выражение:
            \[
                (a \le b) \vee (b \le a)
            \]
            Исходя из определения данного отношения, его можно переписать в виде
            \[
                (a < b) \vee (a = b) \vee (a > b)
            \]
            В силу корректности операции сложения и определения отношения $<$ ($>$), равносильной формой записи является
            \[
                (a - b < 0) \vee (a - b = 0) \vee (a - b > 0)
            \]
            Тогда положим $c = a - b := \{[p_n; q_n]_\Q\}_{n = 1}^\infty$:
            \[
                (c < 0) \vee (c = 0) \vee (c > 0)
            \]
            Тогда, рассмотрим случай, когда $\forall n \in \N\ p_n \le 0 \le q_n$. В иных
            случаях $(\exists n \in \N)\ q_n < 0$ или $p_n > 0$. А это
            прямо по определению соответствет $c < 0$ и $c > 0$ соответственно.
            Покажем, что в таком случае $\{[p_n;q_n]_\Q\}_{n = 1}^\infty \sim \{[0;0]_\Q\}_{n = 1}^\infty$
            
            По определению $\sim$ нужно проверить, что
            \[
                (\forall \veps \in \Q_+)(\exists N \in \N)
                (\forall n > N)\ \max(q_n, 0) - \min(p_n, 0) < \veps
            \]
            Из условия следует, что $\max(q_n, 0) - \min(p_n, 0)
            = q_n - p_n < \veps$ (исходя из определения гнезда).
            А значит, эквивалентность верна и $c = 0$. То есть
            в каждом случае выполняется хотя бы одно из трех
            утверждений. 
    \end{itemize}
\end{proof}

\begin{note}
    Для гнезд одного и того же числа:
    $\{[p_n; q_n]_\Q\}_{n = 1}^\infty \sim
    \{[p_n; q_n]_\Q\}_{n = m}^\infty$, где $m \in \N$.

    Это означает, что если $a > 0$, то
    \[
    (\exists \{[p'_n; q'_n]_\Q\}_{n}^\infty \in a)
    (\forall n \in \N)\ p_n > 0
    \]
\end{note}

\subsubsection*{Произведение}

\begin{definition}
    \textit{Произведением} двух положительных действительных
    чисел с представлениями
    $\{[p_n;q_n]_\Q\}_{n = 1}^\infty,\ p_1 > 0$ и
    $\{[r_n;s_n]_\Q\}_{n = 1}^\infty,\ r_1 > 0$ называют
    действительное число, представляемое гнездом
    $\{[p_n \cdot r_n;q_n \cdot s_n]_\Q\}_{n = 1}^\infty$
\end{definition}

Доопределим произведение на всё множество $\R$:
\[
	a \cdot b := \System{
		&{-(a \cdot (-b)),\ a > 0,\ b < 0}
		\\
		&{(-a) \cdot (-b),\ a < 0,\ b < 0}
	}
\]
А также $a \cdot 0 = 0 \cdot a = 0$, $\forall a \in \R$

\begin{theorem}
    Произведение определено корректно. Справедливы свойства

    \begin{enumerate}
        \item[II-а).] $(\forall a, b \in \R) \ a
            \cdot b = b \cdot a$ \textit{(коммутативность)}
        \item[II-б).] $(\forall a, b, c \in \R)\ 
            (a \cdot b) \cdot c = a \cdot (b \cdot c)$ \textit{(ассоциативность)}
        \item[II-в).] $(\forall a \in \R) \ a \cdot
            1 = 1 \cdot a = a$ \textit{(нейтральный элемент)}
    \end{enumerate}
\end{theorem}

\begin{proof}~

    \begin{itemize}
        \item Докажем корректность произведения:
            \[
                \System{\{[p_n; q_n]_\Q\}_{n = 1}^\infty \sim
                    \{[p'_n; q'_n]_\Q\}_{n = 1}^\infty \\ 
                    \{[r_n; s_n]_\Q\}_{n = 1}^\infty \sim
                    \{[r'_n; s'_n]_\Q\}_{n = 1}^\infty}
                \Ra
                \{[p_n \cdot r_n; q_n \cdot s_n]_\Q\}_{n = 1}^\infty
                \sim \{[p'_n \cdot r'_n; q'_n \cdot s'_n]_\Q\}_{n = 1}^\infty
            \]
            Для начала покажем, что просто произведение положительных
            действительных чисел вообще является системой
            стягивающейся рациональных отрезков. Так как
            $r_1 > 0$ и $p_1 > 0$ (из определения произведения), то
            \begin{align*}
                &\System{
                &r_n \le s_n \\ 
                &p_n \le q_n
                } 
                \Ra p_n r_n \le q_n r_n \le q_n s_n \\
                &q_n s_n - p_n r_n \le q_n (s_n - r_n) +
                r_n (q_n - p_n) \le q_1 (s_n - r_n) + r_1 (q_n - p_n)
            \end{align*}
            $\{[p_n; q_n]_\Q\}_{n = 1}^\infty$ - система
            стягивающихся рациональных отрезков, значит
            \[
            (\forall \veps \in \Q_+)(\exists N_1 \in \N)
            (\forall n > N_1)\ q_n - p_n < \frac{\veps}{2s_1}
            \]
            
            Аналогично для $\{[r_n; s_n]_\Q\}_{n = 1}^\infty$
            \[
            (\forall \veps \in \Q_+)(\exists N_2 \in \N)
            (\forall n > N_2)\ s_n - r_n < \frac{\veps}{2q_1}
            \]
            
            $\Ra q_n s_n - p_n r_n \le q_1 (s_n - r_n) +
            r_1 (q_n - p_n) < q_1 \cdot \frac{\veps}{2q_1} +
            r_1 \cdot \frac{\veps}{2r_1} = \veps$, а это значит, что
            $\{[p_n \cdot r_n; q_n \cdot s_n]_\Q\}_{n = 1}^\infty$ - тоже система
            стягивающихся рациональных отрезков.
            
            Далее покажем, что произведения разных представителей классов эквивалентны:
            \begin{multline*}
                \max(q_n s_n, q'_n s'_n) - \min(p_n r_n, p'_n r'_n) \le \\
                \le \max(q_n, q'_n) \cdot \max(s_n, s'_n) - \min(p_n, p'_n)
                \cdot \min(r_n, r'_n) = \\
                = \max(q_n, q'_n) \cdot \max(s_n, s'_n) - \max(s_n, s'_n)
                \cdot \min(p_n, p'_n) + \\
                + \max(s_n, s'_n) \cdot \min(p_n, p'_n) - \min(p_n, p'_n)
                \cdot \min(r_n, r'_n) = \\
                = \max(s_n, s'_n) \cdot (\max(q_n, q'_n) - \min(p_n, p'_n))
                + \min(p_n, p'_n) \cdot (\max(s_n, s'_n) - \min(r_n, r'_n)) \le \\
                \le \max(s_1, s'_1) \cdot (\max(q_n, q'_n) - \min(p_n, p'_n)) +
                \max(p_1, p'_1) \cdot (\max(s_n, s'_n) - \min(r_n, r'_n))
            \end{multline*}
            
            Из определения действительных чисел следует, что
            \begin{align*}
                (\forall \veps \in \Q_+)(\exists N_1 \in \N)
                (\forall n > N_1)\ \max(q_n, q'_n) - \min(p_n, p'_n)
                < \frac{\veps}{2 \cdot \max(s_1, s_1)}
                \\
                (\forall \veps \in \Q_+)(\exists N_2 \in \N)
                (\forall n > N_2)\ \max(s_n, s'_n) - \min(r_n, r'_n)
                < \frac{\veps}{2 \cdot \max(p_1, p_1)}
            \end{align*}
            
            А значит
            \[
                \max(q_n s_n, q'_n s'_n) - \min(p_n r_n, p'_n, r'_n) < \veps
            \]
        \item \textit{Коммутативность} следует тривиальным образом
            из коммутативности произведения рациональных чисел
        \item \textit{Ассоциативность} следует напрямую из ассоциативности
            произведения рациональных чисел
        \item \textit{Нейтральный элемент} следует из нейтрального элемента
            произведения рациональных чисел
    \end{itemize}
\end{proof}

\subsubsection*{Обратное действительное число по произведению}

\begin{definition}~

    \begin{itemize}
        \item Если действительное число положительно:
            $(\forall a \in \R,\ a > 0)$, то обратным к нему
            числом $\frac{1}{a}$ называется то, которому
            $\ni \{[\frac{1}{p_n}, \frac{1}{q_n}]
            _{\Q}\}_{n = 1}^{\infty}$, где
            $\{[p_n, q_n]_{\Q}\}_{n = 1}^{\infty} \in a, \ p_1 > 0$; 
        \item Для отрицательных чисел:
            $\ (\forall a \in \R, a < 0)$
            \[
                \frac{1}{a} := -(-\frac{1}{a})
            \]
    \end{itemize}
\end{definition}

\begin{lemma}
    Если $\{[p_n; q_n]_\Q\}_{n = 1}^\infty$ представляет
    число $c \in \R$, то $(\forall n \in \N)\ p_n \le c \le q_n$
\end{lemma}

\begin{anote}
    Возможно, при первом взгляде Вас смутило сравнение
    рационального и действительного чисел, но на самом деле здесь
    все определено вполне корректно: рациональное в данном случае
    рассматривается как член множества действительных (в виде
    $\{[p_n; p_n]_\Q\}_{n = 1}^\infty$) и сравнение работает
    определенным ранее образом для двух действительных.
\end{anote}

\begin{proof}
    Предположим обратное, то есть
    \[
        (\exists n_0 \in \N)\ p_{n_0} > c
        \lra p_{n_0} - c > 0
    \]
    Выражение слева является числом, поэтому сопоставим ему систему
    \[
        p_{n_0} - c \ni \{[r_n; s_n]_\Q\}_{n = 1}^\infty
    \]
    Так как $p_{n_0} - c > 0$, из определения следует:
    \[
        (\exists N_1 \in \N)(\forall n_1 > N_1)\ r_{n_1} > 0
    \]
    А число $p_{n_0}$ тогда будет представлять система
    \[
        p_{n_0} \ni \{[p_{n_0}; p_{n_0}]_\Q\}_{n = 1}^\infty
    \]
    Рассмотрим $n > \max(n_0, N_1)$. Тогда, разность
    $p_{n_0} - (p_{n_0} - c)$ с одной стороны, равна
    \[
        p_{n_0} - (p_{n_0} - c) = p_{n_0} - p_{n_0} + c = c
    \]
    А с другой стороны,
    \[
        p_{n_0} - (p_{n_0} - c) \ni \{[p_{n_0} - s_n; p_{n_0} - r_n]_\Q\}_{n = 1}^\infty
    \]
    Стало быть, так как $c = p_{n_0} - (p_{n_0} - c) \Ra$
    \[
        \{[p_{n_0} - s_n; p_{n_0} - r_n]_\Q\}_{n = 1}^\infty
        \sim \{[p_n; q_n]_\Q\}_{n = 1}^\infty
    \]
    Выясним отношения между границами отрезков:
    \[
        p_{n_0} - s_n \le p_{n_0} - r_n < p_{n_0} \le p_n \le q_n
    \]
    А если системы эквивалентны, то
    \[
        \max(q_n, p_{n_0} - r_n) - \min(p_n, p_{n_0} - s_n) =
        q_n - p_{n_0} + s_n
    \]
    При этом стоит заметить, что $q_n \ge p_{n_0 + 1}$,
    так как $(\forall i, j \in \N)\ q_i \ge p_j$, при этом
    $s_n > r_n > 0$. В свою очередь
    \[
        q_n - p_{n_0} + s_n > p_{n_0 + 1} - p_{n_0} + r_n > p_{n_0 + 1} - p_{n_0}    
    \]
    Вот мы и получили константу и доказали, что наши гнезда не
    эквивалентны по определению, то есть мы получили противоречие.

\end{proof}

\begin{theorem}
    Определение обратного числа корректно, справедливы
    свойства:

    \begin{enumerate}
        \item[II-г).] $(\forall a \in \R,\ a\neq 0)\ 
            \frac{1}{a} \cdot a = a \cdot \frac{1}{a} = 1$
        \item[I-II).] $(\forall a, b, c \in \R)\ a \cdot
            (b + c) = a \cdot b + a \cdot c$
        \item[I-III).] $(\forall a, b, c \in \R)(a \le b)
            \Ra (a + c \le b + c)$
        \item[II-III).] $(\forall a, b \in \R)
            (\forall c > 0, c\in \R)(a \le b)
            \Ra (a \cdot c \le b \cdot c)$
        \item[IV).] $(\forall a \in \R)(\forall b \neq 0,
        b \in \R)(\exists n \in \Z) \ b \cdot n > a$ (Свойство Архимеда)

    \end{enumerate}
\end{theorem}

\begin{proof}~

    \begin{itemize}
        \item Докажем \textit{корректность}, для начала
            покажем, что такая система отрезков вообще будет стягиваться:
            \[
                \frac{1}{p_n} - \frac{1}{q_n} =
                \frac{q_n - p_n}{p_n \cdot q_n} \le
                \frac{q_n - p_n}{p^2_n} \le \frac{q_n - p_n}{p^2_1}
            \]
            По условию стягивания изначального числа:
            \[
                (\forall \veps \in \Q_+)(\exists N \in \N)
                (\forall n > N)\ q_n - p_n < \veps \cdot p^2_1
            \]
            \[
                \Ra \frac{1}{p_n} - \frac{1}{q_n} \le
                \frac{q_n - p_n}{p^2_1} <
                \frac{\veps \cdot p^2_1}{p^2_1} = \veps
            \]
            
            Теперь докажем, что определение не зависит от
            представителя класса. Нам дано:
            \[
                \System{\{[p_n; q_n]_\Q\}_{n = 1}^\infty \in a
                    \\ 
                    \{[p'_n; q'_n]_\Q\}_{n = 1}^\infty \in a}
                \Ra \{[p_n; q_n]_\Q\}_{n = 1}^\infty
                \sim \{[p'_n; q'_n]_\Q\}_{n = 1}^\infty
            \]
            Отсюда следует:
            \[
                (\forall \veps \in \Q_+)(\exists N \in \N)
                (\forall n > N)\ \max(q_n, q'_n) - \min(p_n, p'_n) < \veps
                \cdot (\min(p_1, p'_1))^2
            \]
            Тогда, так как $p_n > 0$ и $p'_n > 0$, верно следующее:
            \begin{multline*}
                \max\left(\frac{1}{p_n}, \frac{1}{p'_n}\right) -
                \min\left(\frac{1}{q_n}, \frac{1}{q'_n}\right) =
                \frac{1}{\min(p_n, p'_n)} - \frac{1}{\max(q_n, q'_n)} =\\
                = \frac{\max(q_n, q'_n) - \min(p_n, p'_n)}{\min(p_n, p'_n)
                \cdot \max(q_n, q'_n)} \le \frac{\max(q_n, q'_n) - \min(p_n, p'_n)}
                {(\min(p_n, p'_n))^2} \le \\
                \le \frac{\max(q_n, q'_n) - \min(p_n, p'_n)}
                {(\min(p_1, p'_1))^2} < \veps
            \end{multline*}
        \item \textit{Дистрибутивность} выполняется как следствие
            дистрибутивности на множестве рациональных чисел
        \item Следующее свойство доказывается тривиально:
            
            $a \le b \lra b - a \ge 0 \lra (b + c) - (a + c) \ge 0
            \lra a + c \le b + c$
        
        \item Предпоследнее утверждение можно доказать, рассмотрев
            возможные случаи:
            \[
                a \le b \lra (a = b) \vee (a < b)
            \]
            Если верна первая скобка, то $(b - a)$ представима
            в виде гнезда $\{[0; 0]_\Q\}_{n = 1}^\infty$, тогда
            можно домножить оба нуля на любые положительные
            рациональные числа ($c > 0$) и также получить ноль:
            \[
                b - a = 0 \lra (b - a) \cdot c = 0 \lra b \cdot c -
                a \cdot c = 0 \Ra a + c \le b + c
            \]

        \item \textit{Свойство Архимеда} доказывается при помощи
            рациональных чисел:
            
            Пусть $\{[p_n; q_n]_\Q\}_{n = 1}^\infty \in a$
            и $\{[r_n; s_n]_\Q\}_{n = 1}^\infty \in b$

            По вышедоказанной лемме 
            $a \le q_1$ и $b \ge r_1$. Применим
            теперь свойство Архимеда для рациональных чисел:
            \[
                (\exists n \in \N)\ n \cdot b \ge n \cdot r_1 > q_1 \ge a     
            \]
    \end{itemize}
   
\end{proof}


\subsubsection*{Верхняя и нижняя грани}

\begin{definition}
    Множество $A \subset \R$ называется огрниченным
    сверху (снизу), если $(\exists C (c) \in \R)
    (\forall x \in A)\ x \le C \ (x \ge c)$. Число
    $C (c)$ называется в этом случае верхней (нижней)
    гранью А.
\end{definition}

\begin{theorem}
    Свойство полноты. (V)

    Если $A, B \subset \R$ - непустые множества,
    такие что $A \cup B = \R$ и $(\forall a \in A)
    (\forall b \in B)\ a < b$, то 
    $(\exists c \in \R)(\forall a \in A)(\forall b \in B)
    \ a \le c \le b$.
\end{theorem}

\begin{proof}
    Построим $\{[p_n; q_n]_\Q\}_{n = 1}^\infty \in c$ - систему стягивающихся
    отрезков. Будем брать с каждым шагом более точное десятичное
    приближение к нашей искомой границе (при этом наибольший и
    наименьший элементы всегда будут, так как мы берем множество
    целых чисел):
    \[
        p_n - \text{ наибольшее } \left(\frac{1}{10^{n - 1}}\Z\right) \cap A,
        \ q_n - \text{ наименьшее } \left(\frac{1}{10^{n - 1}}\Z\right) \cap B
    \]
    
    Предположим, что $(\exists b \in B)\ c > b \Ra c + (-b) > 0 
    \Ra (\exists n \in \N):$
    \[
        \frac{1}{10^{n - 1}} < c - b
    \]
    Так как $\{[p_n; q_n]_\Q\}_{n = 1}^\infty \in c \Ra$ по лемме
    $c \in [p_n, q_n] \Ra$
    \[
        c - p_n \le q_n - p_n < c - b \Ra p_n > b
    \]
    А так как $p_n \in A,\ b \in B$, то мы получили противоречие:
    $(\exists p_n \in A)(\exists b \in B)\ p_n \ge b$
\end{proof}

\subsubsection*{Другие определения действительных чисел}

\begin{note}
    Дедекиндово сечение - это такое разбиение
    $\Q$ на непустые множества $A, B$,  что 
    $(\forall a \in A)(\forall b \in B)\ a < b$.
    В дедекиндовой теории действительное число
    --- это сечение.
\end{note}

\begin{note}
    Построенное в доказательстве гнедо даёт
    представление действительного числа в виде
    бесконечной десятичной дроби.
\end{note}

\begin{note}
   Свойства I-V можно принять за аксиомы. Можно
   доказать, что любые две системы, удовлетворяющие
   I-V, изоморфны друг другу.
\end{note}