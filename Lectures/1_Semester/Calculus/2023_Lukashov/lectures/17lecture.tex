\begin{theorem} (Кантора о равномерной непрерывности)
	Если $f$ непрерывна на $[a; b]$, то она равномерно
	непрерывна на нём.
\end{theorem}

\begin{idea}
	Предположим противное: f не равномерно непрерывна,
	тогда найдутся какие-то последовательности точек, что
	будут сколь угодно близки по аргументу, но по значению
	будут отличаться не менее, чем на константу. Выделим из
	этих последовательностей сходящиеся подпоследовательности
	(тут нам поможет то, что мы на отрезке), тогда со одной
	стороны они сходятся к одному числу, а значит по
	непрерывности их образы должны сходится к одну числу,
	с другой стороны их образы отличаются не менее, чем на
	константу. Противоречие получено
\end{idea}

\begin{proof}
	От противного. Пусть $f$ неравномерно непрерывна на $[a; b]$:
	\[
		(\exists \eps_0 > 0)(\forall \delta > 0)
		(\exists x, y \in [a; b],\ |x - y| < \delta)
		\ \ |f(x) - f(y)| \ge \eps_0
	\]
	Построим $\{x_n\}$ и $\{y_n\}$, последовательно выбирая
	$\delta = 1, \frac{1}{2}, \ldots, \frac{1}{n}, \ldots$.
	То есть:
	\begin{align*}
		&(\delta := 1)(\exists x_1,\ y_1 \in [a; b])
		(|x_1 - y_1| < 1)\ \ |f(x_1) - f(y_1)| \ge \eps_0\\
		&(\delta := \frac{1}{2})(\exists x_2,\ y_2 \in [a; b])
		(|x_2 - y_2| < \frac{1}{2})\ \ |f(x_2) - f(y_2)| \ge \eps_0\\
		&(\delta := \frac{1}{3})(\exists x_3,\ y_3 \in [a; b])
		(|x_3 - y_3| < \frac{1}{3})\ \ |f(x_3) - f(y_3)| \ge \eps_0\\
		&\ldots\\
		&(\delta := \frac{1}{n})(\exists x_n,\ y_n \in [a; b])
		(|x_n - y_n| < \frac{1}{n})\ \ |f(x_n) - f(y_n)|
		\ge \eps_0
	\end{align*}
	\[
		(\exists \{x_n\},\ \{y_n\} \subset [a; b])
	    (|x_n - y_n| < \frac{1}{n})\ \ |f(x_n) - f(y_n)|
		\ge \eps_0
	\]
	Так как последовательность $\{x_n\}$ ограничена,
	то по теореме Больцано-Вейерштрасса из неё можно
	выделить сходящуюся подпоследовательность:
	\[
		(\exists \{x_{n_k}\} \subset \{x_n\})\ 
		\liml_{k \to \infty} x_{n_k} = x_0 \in [a; b]
	\]
	Тогда
	\[
		|x_{n_k} - y_{n_k}| < \frac{1}{n_k} \lra x_{n_k} -
		\frac{1}{n_k} < y_{n_k} < x_{n_k} + \frac{1}{n_k}
	\]
	Это нам даёт, что
	\[
		\liml_{k \to \infty} y_{n_k} = x_0
	\]
	А раз функция непрерывна, то верны равенства
	\begin{align*}
		&\liml_{k \to \infty} f(x_{n_k}) = f(x_0)
		\\
		&\liml_{k \to \infty} f(y_{n_k}) = f(x_0)
	\end{align*}
	Получили противоречие с условием, что $|f(x_n) - f(y_n)|
	\ge \eps_0$
\end{proof}

\begin{theorem} (Признак равномерной непрерывности)
	Если $f$ дифференцируема на промежутке $I$ и имеет
	ограниченную производную на этом промежутке, то она
	равномерно непрерывна 
\end{theorem}

\begin{idea}
	Из ограниченности производной следует ограниченность наклона
	секущей по теореме Лагранжа, а из этого следует
	равномерная непрерывность
\end{idea}

\begin{proof}
	Нужно доказать, что
	\[
		(\forall \eps > 0)(\exists \delta > 0)
		(\forall x_1, x_2 \in I,\ |x_1 - x_2| < \delta)
		\ \ |f(x_1) - f(x_2)| < \eps
	\]
	По теореме Лагранжа
	\[
		(\exists \xi \in (x_1; x_2))\ \ 
		f(x_2) - f(x_1) = f'(\xi)(x_2 - x_1)
	\]
	Так как производная ограничена, то
	\[
		(\exists M > 0)(\forall x \in I)
		\ \ |f'(x)| \le M
	\]
	Поэтому положим $\delta := \frac{\eps}{M}$ и
	тогда следует, что
	\[
		|f(x_2) - f(x_1)| \le M \cdot |x_2 - x_1| < \eps
	\]
\end{proof}

% \subsubsection*{Геометрический смысл равномерной непрерывности}

% Нарисовать. Последняя минута 14й лекции Лукашова 2019го года

\subsection{Формула Тейлора}


%%% Его стоило дать ещё в 4м параграфе, так Лукашов сказал
\begin{definition}
	Функция $f$ называется $n$ раз дифференцируемой в точке
	$x_0$, если её производные $f', f'', \ldots, f^{(n - 1)}$
	определены в некоторой окрестности точки $x_0$ и $f^{(n - 1)}$
	дифференцируема в точке $x_0$.
\end{definition}

\begin{lemma}
	Для любой функции $f$, $n$ раз дифференцируемой в точке
	$x_0$, существует единственный многочлен $P_n(f, x)$
	степени не выше $n$ такой, что $P_n^{(k)}(f, x_0) =
	f^{(k)}(x_0)$ для $k = 0, 1, \ldots, n$. При этом
	\[
		P_n(f, x) = f(x_0) + f'(x_0)(x - x_0) +
		\frac{f''(x_0)}{2!}(x - x_0)^2 + \ldots +
		\frac{f^{(n)}(x_0)}{n!}(x - x_0)^n
 	\]
 	Этот многочлен называется \textit{многочленом Тейлора
	функции $f$ в точке $x_0$ степени $n$}.
\end{lemma}

\begin{proof}
	Докажем, что приведённый многочлен удовлетворяет
	всем условиям, сказанным в лемме. То есть докажем,
	что существует многочлен, подходящий лемме. Сразу из
	определения следует, что
	\[
		P_n(f, x_0) = f(x_0)
	\]
	Теперь возьмём $k$-ю производную данного многочлена.
	Несложно понять, что слагаемые $(x - x_0)^j$, где
	$j < k$, сократятся полностью. Для остальных имеем
	\[
		\left((x - x_0)^j\right)^{(k)} =
		j(j - 1) \ldots (j - k + 1)(x - x_0)^{j - k}
	\]
	То есть
	\[
		P_n^{(k)}(f, x) = \suml_{j = k}^n
		\frac{f^{(j)}(x_0)}{j!} \cdot j(j - 1) \ldots
		(j - k + 1)(x - x_0)^{j - k} = \suml_{j = k}^n
		\frac{f^{(j)}(x_0)}{(j - k)!} (x - x_0)^{j - k}
	\]
	В точке $x = x_0$ это нам даёт
	\[
		P_n^{(k)}(f, x_0) = \frac{f^{(k)}(x_0)}{0!} =
		f^{(k)}(x_0)
	\]
	
	Теперь докажем единственность: пусть даны 2
	различных многочлена $P$ и $Q$ степени не выше $n$,
	удовлетворяющие условиям леммы. Тогда
	\[
		(P - Q)^{(n)}(x_0) = 0,\ k = 0, \ldots, n
	\]
	При этом разность многочленов - тоже многочлен вида
	\[
		(P - Q)(x) = a_0 + a_1(x - x_0) + \ldots +
		a_n(x - x_0)^n
	\]
	Последовательно рассматривая все $k$-е производные
	получим, что
	\[
		a_0 = a_1 = \ldots = a_n = 0
	\]
	То есть $P(x) = Q(x)$
\end{proof}

\begin{note}
	При подстановке $x = x_0$ можно заметить, что полное
	слагаемое имеет вид
	\[
		\frac{f^{(j)}(x_0)}{(j - k)!}(x_0 - x_0)^{j - k}
	\]
	Казалось бы, при $j = k$ мы имеем дело с
	неопределённостью. Но помним, что производная -
	это предел при $x \to x_0$, который мы вначале считаем,
	а потом уже делаем подстановку $x = x_0$. Здесь точно 
	такая же ситуация - мы вначале должны
	\textbf{полностью досчитать производную}, а потом
	подставлять $x = x_0$. То есть вначале будет
	$(x - x_0)^0 = 1$, и только потом подстановка $x$
	(которая с единицей уже ничего не сделает).
\end{note}

\begin{lemma} \label{lemTaylor}
	Пусть $\varphi$ и $\psi$ $n + 1$ раз дифференцируемы в
	окрестности точки $x_0$, а также
	\begin{align*}
		&\varphi(x_0) = \varphi'(x_0) = \ldots =
		\varphi^{(n)}(x_0) = \psi(x_0) = \psi'(x_0) =
		\ldots = \psi^{(n)}(x_0) = 0
		\\
		&\psi', \psi'', \ldots, \psi^{(n + 1)} \neq 0
		\text{ в } \mc{U}_{\delta}(x_0)
	\end{align*}
	Тогда $\forall x \in \mc{U}_{\delta}(x_0)$ существует
	$\xi$ между $x_0$ и $x$ такое, что
	\[
		\frac{\varphi(x)}{\psi(x)} =
		\frac{\varphi^{(n + 1)}(\xi)}{\psi^{(n + 1)}(\xi)}
	\]
\end{lemma}

\begin{proof}
	По теореме Коши
	\[
		\exists \xi_1 \such \frac{\varphi(x)}{\psi(x)} =
		\frac{\varphi(x) - \varphi(x_0)}{\psi(x) - \psi(x_0)} =
		\frac{\varphi'(\xi_1)}{\psi'(\xi_1)}
	\]
	В силу того, что $\varphi'(x_0) = \psi'(x_0) = 0$, а также
	$\varphi$ и $\psi$ снова удовлетворяют условиям теоремы
	Коши, получим
	\[
		\exists \xi_2 \such \frac{\varphi'(\xi_1) -
		\varphi'(x_0)}{\psi'(\xi_1) - \psi(x_0)} =
		\frac{\varphi''(\xi_2)}{\psi''(\xi_2)} =
		\frac{\varphi(x)}{\psi(x)}
	\]
	И так продолжаем до $\xi = \xi_{n + 1}$:
	\[
		\frac{\varphi^{(n)}(\xi_n) - \varphi^{(n)}(x_0)}
		{\psi^{(n)}(\xi_n) - \psi^{(n)}(x_0)} =
		\frac{\varphi^{(n + 1)}(\xi)}{\psi^{(n + 1)}(\xi)}
	\]
\end{proof}

\begin{theorem} (Формула Тейлора с остаточным членом в
	форме Лагранжа)
	Если $f$ дифференцируема $n + 1$ раз в окрестности
	$U_\delta(x_0)$, то $\forall x \in U_\delta(x_0)$
	существует $\xi$ между $x_0$ и $x$ такое, что
	\begin{multline*}
		f(x) = P_n(f, x) +
		\frac{f^{(n + 1)}(\xi)}{(n + 1)!}(x - x_0)^{n + 1} = \\
		f(x_0) + f'(x_0)(x - x_0) + \ldots +
		\frac{f^{(n)}(x_0)}{n!}(x - x_0)^n +
		\frac{f^{(n + 1)} (\xi)}{(n + 1)!}(x - x_0)^{n + 1}
	\end{multline*}
\end{theorem}

\begin{proof}
	Рассмотрим функции
	\begin{align*}
		&\varphi(x) := f(x) - P_n(f, x)
		\\
		&\psi(x) := (x - x_0)^{n + 1}
	\end{align*}
	Заметим, что данные функции удовлетворяют условиям
	леммы \ref{lemTaylor}. То есть
	\[
		\frac{f(x) - P_n(f, x)}{(x - x_0)^{n + 1}} =
		\frac{f^{(n + 1)}(\xi)}{(n + 1)!}
	\]
	Следовательно
	\[
		f(x) = P_n(f, x) + \frac{f^{(n + 1)}(\xi)}
		{(n + 1)!}(x - x_0)^{n + 1}
	\]
\end{proof}

\begin{theorem} (Формула Тейлора с остаточным членом в
	форме Пеано)
	Если $f$ $n$ раз дифференцируема в точке $x_0$, то
	\[
		f(x) = f(x_0) + f'(x_0)(x - x_0) + \ldots +
		\frac{f^{(n)}(x_0)}{n!}(x - x_0)^n +
		o\left((x - x_0)^n\right),\ x \to x_0
	\]
\end{theorem}

\begin{proof}
	Положим
	\begin{align*}
		&\varphi(x) = f(x) - P_n(f, x)
		\\
		&\psi(x) = (x - x_0)^n
	\end{align*}
	Отсюда
	\[
		\varphi(x_0) = \ldots = \varphi^{(n - 2)}(x_0) =
		\psi(x_0) = \ldots = \psi^{(n - 2)}(x_0) = 0
	\]
	То есть по лемме \ref{lemTaylor} $\exists \xi$ между
	$x$ и $x_0$ такое, что
	\[
		\frac{f(x) - P_n(f, x)}{(x - x_0)^n} =
		\frac{f^{(n - 1)}(\xi) - P_n^{(n - 1)}(f, \xi)}
		{n! \cdot (\xi - x_0)} = \frac{f^{(n - 1)}
		(\xi) - f^{(n - 1)}(x_0) - f^{(n)}(x_0)(\xi - x_0)}
		{n! \cdot (\xi - x_0)}
	\]
	Посчитаем предел(при $x \to x_0$ $\xi \to x_0$, так как
	$\xi$  между $x$ и $x_0$)
	\begin{multline*}
		\liml_{x \to x_0} \frac{\phi(x)}{\psi(x)} =
		\liml_{\xi \to x_0} \frac{f^{(n - 1)}(\xi) -
		f^{(n - 1)}(x_0) - f^{(n)}(x_0)(\xi - x_0)}
		{n! \cdot (\xi - x_0)} = \liml_{\xi \to x_0}
		\frac{f^{(n - 1)}(\xi) - f^{(n - 1)}(x_0)}
		{n! \cdot (\xi - x_0)} - \frac{f^{(n)}(x_0)}{n!} = \\ 
		\frac{1}{n!}(f^{(n)}(x_0) - f^{(n)}(x_0)) = 0
	\end{multline*}
	Следовательно
	\[
		f(x) = P_n(f, x) + o((x - x_0)^n),\ x \to x_0
	\]
\end{proof}

\begin{theorem} (Единственность разложения по формуле Тейлора)
	Если 
	\[
		f(x) = a_0 + a_1(x - x_0) + \ldots +
		a_n(x - x_0)^n + o\left((x - x_0)^n\right),
		\ x \to x_0
    \]
    и 
    \[
    	f(x) = b_0 + b_1(x - x_0) + \ldots +
		b_n(x - x_0)^n + o\left((x - x_0)^n\right),
		\ x \to x_0
    \] то
	\[
		a_k = b_k,\ k = 0, 1, \ldots, n
	\]
\end{theorem}

\begin{proof}
	Рассмотрим разность этих многочленов:
	\[
		f(x) - f(x) = (a_0 - b_0) + (a_1 - b_1)(x - x_0) +
		\ldots + (a_n - b_n)(x - x_0)^n + 
		o\left((x - x_0)^n\right) = 0,\ x \to x_0
	\]
	В предельном переходе получим
	\[
		\liml_{x \to x_0} (a_0 - b_0) + (a_1 - b_1)(x - x_0) +
		\ldots + (a_n - b_n)(x - x_0)^n +
		o\left((x - x_0)^n\right) = a_0 - b_0 = 0
	\]
	Отсюда $a_0 = b_0$. Теперь разность имеет вид:
	\[
		f(x) - f(x) = (a_1 - b_1)(x - x_0) + \ldots +
		(a_n - b_n)(x - x_0)^n + o((x - x_0)^n) = 0,
		\ x \to x_0
	\]
	При этом $x - x_0 \neq 0$. Значит, можно разделить
	уравнение на $(x - x_0)$ и снова взять предел. В
	этот раз получим $a_1 = b_1$. Делая так $n + 1$ раз,
	придём к нужному утверждению
	\[
		a_k = b_k,\ k = 0, 1, \ldots, n
	\]
\end{proof}

\begin{corollary}
	Если $f(x)$ $n$ раз дифференцируема в точке $x_0$ и
	\[
		f(x) = a_0 + a_1(x - x_0) + \ldots +
		a_n(x - x_0)^n + o\left((x - x_0)^n\right),
		\ x \to x_0
	\]
	то
	\[
		a_k = \frac{f^{(k)}(x_0)}{k!},\ k = 0, 1,\ldots, n
	\]
	Причём формула существенна только для $n > 1$. Для
	$n = 1$ - разложение равносильно дифференцируемости в
	точке $x_0$, а для $n = 0$ - непрерывности
\end{corollary}

\begin{example}
	Данная функция имеет асимптотическое разложение, но не
	дважды дифференцируема в нуле (то есть коэффициенты не
	совпадут с теми, что есть в формуле Тейлора)
	\[
		f(x) = \System{
			&x^3 \sin \frac{1}{x},\ x \neq 0
			\\
			&0,\ x = 0
		}
	\]
	Из определения сразу видно, что есть разложение
	\[
		f(x) = a_0 + a_1(x - 0) + a_2(x - 0)^2 + o((x - 0)^2)
	\]
	Первая производная имеет вид
	\[
		f'(x) = \System{
			&3x^2 \sin \frac{1}{x} - x \cos \frac{1}{x},\ x \neq 0
			\\
			&0,\ x = 0
		}
	\]
	Посчитаем $f''(0)$:
	\begin{multline*}
		f''(0) = \liml_{\Delta x \to 0}
		\frac{f'(0 + \Delta x) - f'(0)}{0 + \Delta x - 0} =
		\liml_{\Delta x \to 0} \frac{3\Delta x^2 \sin
		\frac{1}{\Delta x} - \Delta x \cos
		\frac{1}{\Delta x}}{\Delta x} = \\
		\liml_{\Delta x \to 0} \left(3 \Delta x \sin
		\frac{1}{\Delta x} - \cos \frac{1}{\Delta x}\right)
		\text{ - расходится}
	\end{multline*}
\end{example}

\begin{definition}
	Если $x_0 = 0$, то формулы Тейлора называются также
	\textit{формулами Маклорена}
\end{definition}

\subsubsection*{Формулы Маклорена основных элементарных функций}

\begin{enumerate}
	\item $e^x:$ $\left(e^x\right)^{(n)} = e^x \Ra
		e^x = 1 + x + \frac{x^2}{2!} + \ldots +
		\frac{x^n}{n!} + o(x^n),\ x \to 0$
	
	\item $\sin x:$ $\left(\sin x\right)^{(n)} =
		\sin(x + \frac{n\pi}{2})$. То есть
		\[
			\left(\sin x\right)^{(n)}(0) = \sin
			\frac{n\pi}{2} =
			\System{
					&{0,\ n = 2k}
					\\
					&{(-1)^{k - 1},\ n = 2k - 1}
				},\ k \in \N
		\]
		Так как синус имеет любую производную,
		то для любого $n$, если $k$ - это частное от
		деления на 2, формулу Маклорена можно записать так:
		\[
			\sin x = x - \frac{x^3}{3!} + \frac{x^5}{5!} -
			\ldots + (-1)^{k - 1} \frac{x^{2k - 1}}{(2k - 1)!}
			+ o(x^{2k}),\ x \to 0
		\]
		Формула Маклорена в виде Лагранжа также имеет вид:
		\[
			\sin x = x - \frac{x^3}{3!} + \frac{x^5}{5!} -
			\ldots + (-1)^{k} \frac{x^{2k + 1}}{(2k + 1)!}
			\sin \left(\xi + \frac{2k + 1}{2}\pi\right)
		\]
	
	\item $\cos x:$ $\left(\cos x\right)^{(n)} =
		\cos(x + \frac{n\pi}{2})$. То есть
		\[
			\left(\cos x\right)^{(n)}(0) =
			\cos \frac{n\pi}{2} =
			\System{
				&{0,\ n = 2k - 1}
				\\
				&{(-1)^k,\ n = 2k}
			}
		\]
		Отсюда формула Маклорена для косинуса имеет вид:
		\[
			\cos x = 1 - \frac{x^2}{2!} + \frac{x^4}{4!} - \ldots + \frac{(-1)^k}{(2k)!}x^{2k} + o(x^{2k}),\ x \to 0
		\]
	
	\item $(1 + x)^\alpha:$ $\left((1 + x)^\alpha\right)^{(n)} =
		\alpha(\alpha - 1)\ldots(\alpha - n + 1)(1 + x)^{\alpha - n}
		= n! C_{\alpha}^n (1 + x)^{\alpha - n}$, где
		$\alpha \notin \N$
		\[
			(1 + x)^\alpha = 1 + \alpha x +
			\frac{\alpha (\alpha - 1)}{2!}x^2 + \ldots +
			\underbrace{\frac{\alpha (\alpha - 1) \ldots
			(\alpha - n + 1)}{n!}}_{C_\alpha^n}x^n + o(x^n),
			\ x \to 0
		\]
		Если $\alpha \in \N$, то на каком-то шаге производная
		станет нулём и будет таковой дальше.
		(То есть можно будет получить точное разложение,
		бином Ньютона)
	
	\item $\ln (1 + x):$ $\left(\ln (1 + x)\right)^{(n)} =
		\left((1 + x)^{-1}\right)^{(n - 1)}$
		То есть
		\[
			\left(\ln (1 + x)\right)^{(n)}= (-1) \cdot
			(-2) \cdot \ldots \cdot (-1 - (n - 2)) \cdot
			(1 + x)^{-1 - (n - 1)} = (-1)^{n - 1} \cdot
			(n - 1)! \cdot (1 + x)^{-n}
		\]
		Формула Маклорена для логарифма имеет вид:
		\[
			\ln(1 + x) = x - \frac{x^2}{2} +
			\frac{x^3}{3} - \ldots + (-1)^{n - 1}
			\frac{x^n}{n} + o(x^n)
		\]
\end{enumerate}

\begin{note}
	Если $f$ - чётная функция, то в формуле Маклорена все
	нечётные степени имеют нулевые коэффициенты. Если $f$ -
	нечётная, то четные степени имеют нулевые коэффициенты.
\end{note}

\begin{proof}
	Пусть $f$ - чётная функция. Посчитаем $f'(-x)$ по определению:
	\[
		f'(-x) = \liml_{\Delta x \to 0}
		\frac{f(-x + \Delta x) - f(-x)}{\Delta x} =
		\liml_{\Delta x \to 0} \frac{f(x - \Delta x) -
		f(x)}{\Delta x} = \liml_{t \to 0} \frac{f(x + t) -
		f(x)}{-t} = -f'(x)
	\]
	То есть $f'(0) = -f'(0)$. Значит, $f'(0) = 0$.
	
	Пусть $f$ - нечётная функция. Аналогично:
	\[
		f'(-x) = \liml_{\Delta x \to 0} \frac{f(-x + \Delta x) -
		f(-x)}{\Delta x} = \liml_{\Delta x \to 0}
		-\frac{f(x - \Delta x) - f(x)}{\Delta x} =
		\liml_{t \to 0} -\frac{f(x + t) - f(x)}{-t} = f'(x)
	\]
\end{proof}

\begin{addition}
	Рассмотрим функцию
	\[
		f(x) = \System{
		&{\frac{\sin x}{x},\ x \neq 0}
		\\
		&{1,\ x = 0}
		}
	\]
	Она непрерывна на $\R$. Раз мы знаем разложение
	синуса, то можно записать выражение
	\[
		f(x) = 1 - \frac{x^2}{3!} + \frac{x^4}{5!} -
		\ldots + (-1)^{k - 1} \frac{x^{2k - 2}}{(2k - 1)!}
		+ o(x^{2k - 1}),\ x \to 0
	\]
	Является ли оно формулой Тейлора? Как оказывается, да. 
	Но для доказательства нужно показать, что $f(x)$
	дифференцируема $n$ раз в нуле, что сделать крайне трудно.
\end{addition}