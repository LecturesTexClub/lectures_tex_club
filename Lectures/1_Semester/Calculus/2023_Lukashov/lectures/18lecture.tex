\subsubsection*{Некоторые приёмы разложения функций по формуле Тейлора}

\begin{enumerate}
	\item Если $f$ дифференцируема $n + 1$ раз в точке $x_0$ и 
	\[
	f'(x - x_0) = \suml_{k = 0}^n b_k (x - x_0)^k + o((x - x_0)^n),\ x \to x_0
	\]
	то
	\[
		f(x) = f(0) + \suml_{k = 0}^n \frac{b_k}{k + 1} (x - x_0)^{k + 1} + o((x - x_0)^{n + 1}),\ x \to x_0
	\]
	\begin{proof}
		Разложим $f(x)$ по формуле Тейлора:
		\[
		f(x) = f(0) + \suml_{k = 0}^n a_k (x - x_0)^{k + 1} + o((x - x_0)^{n + 1}),\ x \to x_0
		\]
		При этом $a_k = \frac{f^{(k + 1)}(x_0)}{(k + 1)!} = \frac{(f')^{(k)}(x_0)}{k!} \cdot \frac{1}{k + 1} = \frac{b_k}{k + 1}$
	\end{proof}
	\begin{example}
		\[
			(\arcctg x)' = -\frac{1}{1 + x^2}
		\]
		Разложим производную $\arcctg x$ в ряд Маклорена:
		\[
			-\frac{1}{1 + x^2} = -(1 + x^2)^{-1} = \suml_{k = 0}^n C_{-1}^k \cdot k! \cdot x^{2k} + o(x^{2n + 1}) = \suml_{k = 0}^n (-1)^{k + 1} x^{2k} + o(x^{2n + 1})
		\]
		Отсюда получаем, что
		\[
			\arcctg x = \frac{\pi}{2} + \suml_{k = 0}^n \frac{(-1)^{k + 1}}{2k + 1} x^{2k + 1} + o(x^{2n + 2}),\ x \to 0
		\]
	\end{example}

	\item Метод неопределённых коэффициентов
	\begin{example}
		Пусть $f$ имеет вид
		\[
			f(x) = \System{
				&{x \ctg x,\ x \neq 0}
				\\
				&{1,\ x = 0}
			}
		\]
		Тогда $f(x)$ при $x \neq 0$ имеет ещё вид
		\begin{align*}
			&f(x) = \frac{\cos x}{\frac{\sin x}{x}} = \frac{1 - \frac{x^2}{2!} + \frac{x^4}{4!} + o(x^5)}{1 - \frac{x^2}{3!} + \frac{x^4}{5!} + o(x^5)} = a_0 + a_2x^2 + a_4x^4 + o(x^5),\ x \to 0
			\\
			&\left(1 - \frac{x^2}{2} + \frac{x^4}{24} + o(x^5)\right) = \left(a_0 + a_2x^2 + a_4x^4 + o(x^5)\right) \cdot \left(1 - \frac{x^2}{6} + \frac{x^4}{120} + o(x^5)\right),\ x \to 0
		\end{align*}
		Чтобы равенство выполнялось, должны быть равны коэффициенты при одинаковых степенях у приведённых многочленов. Отсюда имеем
		\begin{align*}
			&{1 = a_0}
			\\
			&{-\frac{1}{2} = a_2 - \frac{a_0}{6} \Ra a_2 = -\frac{1}{3}}
			\\
			&{\frac{1}{24} = a_4 - \frac{a_2}{6} + \frac{a_0}{120} \Ra a_4 = -\frac{1}{45}}
		\end{align*}
		То есть
		\[
			f(x) = 1 - \frac{x^2}{3} - \frac{x^4}{45} + o(x^5)
		\]
		Но опять же, нужно доказать, что это формула Тейлора. Иначе это просто асимптотическое разложение
	\end{example}

	\item Применение формулы Тейлора для подсчёта пределов
	\begin{example}
		Вычислим следующий предел:
		\[
			\liml_{x \to 0} \left(e^{x^2 \ctg x} + \ln(1 - x)\right)^{1 / \left(\arcctg (\sh x) + \sin x - \frac{\pi}{2}\right)}
		\]
		Заметим, что он представим в виде
		\[
			\liml_{x \to 0} \left(1 + u(x)\right)^{1 / v(x)} = \liml_{x \to 0} e^{ \frac{\ln (1 + u(x))}{v(x)}}
		\]
		где $u, v = o(1),\ x \to 0$. Тогда, в силу эквивлентности
		\[
			\liml_{x \to 0} e^{ \frac{\ln (1 + u(x))}{v(x)}} = \liml_{x \to 0} e^{\frac{u(x)}{v(x)}}
		\]
		Распишем $u(x)$:
		\begin{multline*}
			u(x) = e^{x\left(1 - \frac{1}{3}x^2 - \frac{1}{45}x^4 + o(x^5)\right)} - x - \frac{x^2}{2} - \frac{x^3}{3} - \frac{x^4}{4} - \frac{x^5}{5} + o(x^5) - 1 =
			\\
			1 + x\left(1 - \frac{1}{3}x^2 + o(x^3)\right) + \frac{x^2\left(1 - \frac{1}{3}x^2 + o(x^3)\right)^2}{2!} + \frac{x^3\left(1 - \frac{1}{3}x^2 + o(x^3)\right)^3}{3!} + 
			\\
			o\left(x^3\left(1 - \frac{1}{3}x^2 + o(x^3)\right)^3\right) - 1 - x - \frac{x^2}{2} - \frac{x^3}{3} + o(x^3) = \left(-\frac{1}{3} + \frac{1}{6} - \frac{1}{3}\right)x^3 + o(x^3) = 
			\\
			-\frac{1}{2}x^3 + o(x^3)
		\end{multline*}
		Теперь $v(x)$:
		\[
			\sh x = \frac{e^x - e^{-x}}{2} = x + \frac{x^3}{3!} + \frac{x^5}{5!} + o(x^6)
		\]
		\begin{multline*}
			\arcctg(\sh x) = \frac{\pi}{2} - \left(x + \frac{x^3}{3!} + \frac{x^5}{5!} + o(x^6)\right) + \frac{1}{3}\left(x + \frac{x^3}{3!} + \frac{x^5}{5!} + o(x^6)\right)^3 -
			\\
			\frac{1}{5}\left(x + \frac{x^3}{3!} + \frac{x^5}{5!} + o(x^6)\right)^5 + o\left(\left(x + \frac{x^3}{3!} + \frac{x^5}{5!} + o(x^6)\right)^6\right) =
			\\
			\frac{\pi}{2} - x + \left(\frac{1}{3} - \frac{1}{6}\right)x^3 + \left(\frac{1}{6} - \frac{1}{120} - \frac{1}{5}\right)x^5 + o(x^6),\ x \to 0
		\end{multline*}
		В итоге имеем
		\begin{multline*}
			v(x) = \frac{\pi}{2} - x + \frac{1}{6}x^3 - \frac{1}{24}x^5 + o(x^6) + x - \frac{1}{6}x^3 + \frac{1}{120}x^5 + o(x^6) - \frac{\pi}{2} =
			\\
			-\frac{1}{30}x^5 + o(x^6),\ x \to 0
		\end{multline*}
		Отсюда предел получает вид
		\[
			\liml_{x \to 0} e^{\frac{u(x)}{v(x)}} = \liml_{x \to 0} e^{\frac{-\frac{1}{2}x^3 + o(x^3)}{-\frac{1}{30}x^5 + o(x^6)}} = +\infty
		\]
	\end{example}
\end{enumerate}

\subsection{Исследование функции с помощью производной}

\begin{theorem} \label{monoF} (Необходимое и достаточное условия монотонности функции)
	Если $f$ дифференцируема на $(a; b)$, то
	\begin{enumerate}
		\item $\forall x \in (a; b)\  f'(x) \ge 0 \lra f(x)$ неубывающая на $(a; b)$
		
		\item $\forall x \in (a; b)\ f'(x) \le 0 \lra f(x)$ невозрастающая на $(a; b)$
		
		\item $\forall x \in (a; b)\  f'(x) > 0 \Ra f(x)$ возрастающая на $(a; b)$
		
		\item $\forall x \in (a; b)\ f'(x) < 0 \Ra f(x)$ убывающая на $(a; b)$
	\end{enumerate}
\end{theorem}

\begin{proof}
	Докажем первый случай. Начнём с утверждения $\Ra$:
	
	Рассмотрим $\forall a < x_1 < x_2 < b$. Тогда, по теореме Лагранжа
	\[
		\exists c \in (x_1; x_2) \such \frac{f(x_2) - f(x_1)}{x_2 - x_1} = f'(c)
	\]
	Отсюда
	\[
		f(x_2) - f(x_1) = f'(c) (x_2 - x_1) \ge 0
	\]
	
	Теперь докажем $\La$: посчитаем производную в некоторой точке $x_0 \in (a; b)$:
	\[
		f'(x_0) = f'_+(x_0) = \liml_{\Delta x \to 0+} \frac{f(x_0 + \Delta x) - f(x_0)}{\Delta x}
	\]
	Так как $x_0 + \Delta x > x_0 \Ra f(x_0 + \Delta x) \ge f(x_0)$. Отсюда
	\[
		f'(x_0) = f'_+(x_0) \ge 0
	\]
\end{proof}

\begin{note}
	В случаях 3 и 4 утверждение верно в одну сторону. Контрпример: 
	\[
		y = \pm x^3,\ x \in (-1; 1)
	\]
\end{note}

\begin{note}
	Если дополнительно потребовать непрерывности $f$ на $[a; b]$, то в теорема будет верна на $[a; b]$.
\end{note}

\begin{theorem} (Первое достаточное условие локального экстремума)
	Пусть $f$ непрерывна в $U_\delta(x_0)$, дифференцируема в $\mc{U}_\delta(x_0)$. Тогда
	\begin{enumerate}
		\item Если $\forall x \in (x_0 - \delta, x_0)\ f'(x) < 0$ и $\forall x \in (x_0, x_0 + \delta)\ f'(x) > 0$, то $x_0$ является точкой строгого локального минимума.
		
		\item Если $\forall x \in (x_0 - \delta, x_0)\ f'(x) > 0$ и $\forall x \in (x_0, x_0 + \delta)\ f'(x) < 0$, то $x_0$ является точкой строгого локального максимума.
	\end{enumerate}
\end{theorem}

\begin{proof}
	Докажем первый случай. Выберем $x_1 \in (x_0 - \delta; x_0)$. Тогда, $f$ непрерывна на $[x_1; x_0]$ и $\forall x \in (x_1; x_0)\ f'(x) < 0$. То есть, $f$ убывает на $[x_1; x_0]$ по теореме \ref{monoF}. Аналогично доказывается, что $f$ возрастает на $[x_0; x_2]$. Значит
	\[
		\exists \delta' = \min(x_0 - x_1, x_2 - x_0) \such \forall x \in \mc{U}_{\delta'}(x_0)\ f(x) > f(x_0)
	\]
	$x_0$ - локальный минимум.
\end{proof}

\begin{theorem} (Второе достаточное условие локального экстремума)
	Пусть $f^{(n)}(x_0) \neq 0$, а $f'(x_0) = \ldots = f^{(n - 1)}(x_0) = 0$. Тогда
	\begin{enumerate}
		\item При $n$ - чётном 
		\[
			\System{
				&{f^{(n)}(x_0) > 0 \Ra x_0 \text{ - точка строгого локального минимума}}
				\\
				&{f^{(n)}(x_0) < 0 \Ra x_0 \text{ - точка строгого локального максимума}}
			}
		\]
		
		\item При $n$ - нечётном $x_0$ не является точкой локального экстремума
	\end{enumerate} 
\end{theorem}

\begin{proof}
	По формуле Тейлора с остаточным членом в форме Пеано можно записать разложение:
	\[
		f(x) = f(x_0) + \frac{f^{(n)}(x_0)}{n!}(x - x_0)^n + o((x - x_0)^n),\ x \to x_0
	\]
	Перепишем это выражение в другом виде:
	\[
		n! \cdot \frac{f(x) - f(x_0)}{f^{(n)}(x_0)} = (x - x_0)^n + o((x - x_0)^n),\ x \to x_0
	\]
	
	Если $n$ чётно, то справа стоит положительное число. То есть
	\[
		\frac{f(x) - f(x_0)}{f^{(n)}(x_0)} > 0,\ x \to x_0
	\]
	Если $f^{(n)}(x_0) > 0$, то и $f(x) > f(x_0),\ \forall x \in \mc{U}_\delta(x_0)$. Следовательно $x_0$ - точка строгого локального минимума. Аналогично при $f^{(n)}(x_0) < 0$ $x_0$ - точка строгого локального максимума.
	
	Если $n$ нечётно, то выражение справа положительно при $x \to x_0+0$ и отрицательно при $x \to x_0-0$. Это значит, что какой бы знак мы не выбрали для $f^{(n)}(x_0)$, разность $f(x) - f(x_0)$ принимает разные знаки по разные стороны от $x_0$, то есть $x_0$ не является точкой локального экстремума.
\end{proof}

\begin{definition}
	Функция $f$ называется \textit{выпуклой вниз(вогнутой вверх)} на $(a; b)$, если её график лежит не выше любой хорды, стягивающей две точки графика.
\end{definition}

\begin{definition}
	Функция $f$ называется \textit{выпуклой вверх(вогнутой вниз)} на $(a; b)$, если её график лежит не ниже любой хорды, стягивающей две точки графика.
\end{definition}

%\subsubsection*{Геометрический смысл выпуклости}

%% Нарисовать. 1:15:42 16я лекция 2021г

\subsubsection*{Аналитический смысл выпуклости}

Возьмём 2 точки $a < x_1 \le x_2 < b$. Координаты точки на хорде можно выразить параметрически:
\[
	\System{
		&x_0 = tx_1 + (1 - t)x_2
		\\
		&y_0 = tf(x_1) + (1 - t)f(x_2)
	}
	,\ t \in [0; 1]
\]
Выпуклость вниз по определению означает, что
\[
	f(tx_1 + (1 - t)x_2) \le tf(x_1) + (1 - t)f(x_2)
\]

\begin{note}
	Если неравенство - строгое $\forall t \in (0; 1),\ \forall x_1, x_2 \in (a; b)$, то $f$ \textit{строго выпукла вниз (вверх)}
\end{note}