\begin{theorem} (Дарбу)
	Пусть $f$ дифференцируема на $(a; b)$ и
	$\exists f'_+(a), f'_-(b) \in \R$. Тогда для любого
	$c$ в интервале между $f'_+(a)$ и $f'_-(b)$ существует
	$\xi \in (a; b)$ такое, что $f'(\xi) = c$
\end{theorem}

\begin{proof}~
	\begin{enumerate}
		\item $c = 0 \Ra f'_+(a) \cdot f'_-(b) < 0$
		
		Так как $f$ дифференцируема на $(a; b)$ и
		односторонние производные конечны, то $f$
		непрерывна на $[a; b]$. А значит по теореме Вейерштрасса
		\[
			\exists \xi \in [a; b] :\ f(\xi) =
			\max\limits_{x \in [a; b]} f(x)
		\]
		Докажем, что $\xi \neq a$. По определению
		односторонней производной
		\[
			f'_+(a) = \liml_{\Delta x \to +0}
			\frac{f(a + \Delta x) - f(a)}{\Delta x}
		\]
		Если предположить, что $f'_+(a) > 0$, то из равенства выше следует
		\[
			(\exists \delta > 0)(\forall x :\ 0 < x < \delta)\ 
			f(a + \Delta x) - f(a) > 0 \Ra f(a + \Delta x) > f(a)
		\]
		Поэтому точка $f(a)$
		не есть максимум на отрезке. Аналогично получим, что
		$\xi \neq b$, и в данном предположении $f'_-(b) < 0$.
		Таким же образом рассматривается случай, когда
		$f'_+(a) < 0 \Ra f'_-(b) > 0$ (в таком случае нужно
		взять не $\max$, а $\min$). Отсюда
		\[
			\exists \xi \in (a; b) :\ f(\xi) = \max\limits_{x \in [a; b]} f(x)
		\]
		и по теореме Ферма получаем, что
		\[
			f'(\xi) = 0 = c
		\]
		
		\item $c \neq 0$
		
		Сведём случай к доказанному. Достаточно
		рассмотреть вспомогательную функцию
		\[
			F(x) := f(x) - cx
		\]
		Так как функция $x$ определена и дифференцируема
		на всей числовой прямой, то
		\[
			F'(x) = f'(x) - c
		\]
		для интервала $(a; b)$. При этом
		\[
		f'_+(a) > c > f'_-(b) \Ra
		F'_+(a) > 0,\ F'_-(b) < 0
		\]
		По уже доказанному, 
		\[
			\exists \xi \in (a; b) :\ F'(\xi) = f'(\xi) - c =
			0 \Ra f'(\xi) = c
		\]
	\end{enumerate}
\end{proof}

\begin{corollary}
	Если $f$ дифференцируема на $(a; b)$, то $f'$ не
	может иметь точек разрыва первого рода на $(a; b)$.
\end{corollary}

\begin{idea}
	Пусть такие разрывы есть.
	В случае разрыва-скачка посмотрим на односторонние пределы
	производной, возьмем у них малые $\eps$ окрестности, так,
	чтобы они не пересекались, тогда в некоторых $\delta$ окрестностях
	все точки лежат в заданных $\eps$. Тогда это противоречит теореме Дарбу,
	а именно не все значения между пределами принимаются на интервале из полученных
 	$\delta$ окрестностей (так как $\eps$ окрестности не пересекаются).
	Случай с устранимым разрывом разбирается аналогично.
\end{idea}

\begin{proof}
	Пусть для определенности $f'(c - 0) < f'(c + 0)$. Тогда
	$\eps_0 := \frac{f'(c + 0) - f'(c - 0)}{4}$. Из определения
	предела:
	\begin{align*}
		&(\exists \delta_1 > 0)(\forall x \in (c - \delta_1, c))
		\ \ f'(x) < f'(c - 0) + \frac{f'(c + 0) - f'(c - 0)}{4}\\
		&(\exists \delta_2 > 0)(\forall x \in (c, c + \delta_2))
		\ \ f'(x) > f'(c + 0) - \frac{f'(c + 0) - f'(c - 0)}{4}
	\end{align*}
	То есть:
	\begin{align*}
		&(\exists \delta_1 > 0)(\forall x \in (c - \delta_1, c))
		\ \ f'(x) < \frac{f'(c + 0) + 3f'(c - 0)}{4}\\
		&(\exists \delta_2 > 0)(\forall x \in (c, c + \delta_2))
		\ \ f'(x) > \frac{3f'(c + 0) + f'(c - 0)}{4}
	\end{align*}
	Пусть $x_1 < x_2 \in (a; b) \such 
	x_1 \in (c - \delta_1, c),\ x_2 \in (c, c + \delta_2)$. Тогда
	из $\eps$ окрестностей понятно, что $f'(x_1) < f'(x_2)$.
	При этом выполняются все условия теоремы Дарбу на $(x_1, x_2)$
	(т.к. $f$ дифференцируема на $(a; b)$ по условию), значит,
	\[
		(\forall \gamma \in (f'(x_1); f'(x_2)))(\exists \xi \in (x_1; x_2))
		(f'(\xi) = \gamma)
	\]
	При этом
	\[
		(\forall x \in (x_1, x_2))\left(f'(x) \in 
		\left(-\infty, \frac{f'(c + 0) + 3f'(c - 0)}{4}\right) \cup 
		\left(\frac{3f'(c + 0) + f'(c - 0)}{4}, +\infty\right)\right)	
	\]
	Если взять $\gamma = \frac{f'(c + 0) + f'(c - 0)}{2}$, то нет такого
	$x$, чтобы $f'(x) = \gamma$. Противоречие с теоремой Дарбу.
	Для случая с устранимым разрывом идея доказательства сохраняется.
\end{proof}

\begin{theorem} (Правило Лопиталя для случая $\frac{0}{0}$)

	Пусть $f$, $g$ дифференцируемы на $(a; b)$, при этом
	$\exists \liml_{x \to a+0} f(x) = \liml_{x \to a+0} g(x)
	= 0$ и $\exists \liml_{x \to a+0} \frac{f'(x)}{g'(x)} = C
	\in \bar{\R}$. Тогда
	\[
		\exists \liml_{x \to a+0} \frac{f(x)}{g(x)} = C
	\]
\end{theorem}

\begin{note}
	Аналогичное утверждение верно для предела $x \to b-0$,
	а также для любого предела $x \to x_0,\ x_0 \in (a; b)$.
\end{note}

\begin{proof}
	Доопределим $f(a) = g(a) = 0$. Из существование предела
	отношения производных следует, что
	\[
		(\exists \delta > 0)(\forall x
		\in (a; a + \delta))\ g'(x) \neq 0
	\]
	Следовательно, на отрезке
	$\left[a; a + \frac{\delta}{2}\right]$ для функции $g$
	выполнены все условия теоремы Коши о среднем \ref{Cauchy_mid}. Значит
	\[
		\left(\forall x \in \left(a; a + \frac{\delta}{2}\right)\right)
		\left(\exists \xi \in (a; x)\right)\ 
		\frac{f(x) - f(a)}{g(x) - g(a)} =
		\frac{f'(\xi)}{g'(\xi)} = \frac{f(x)}{g(x)}
	\]
	при этом понятно, что $\xi = \xi(x)$. Так как
	$a < \xi(x) < x$, то если устремить $x$ к $a + 0$,
	то в предельном переходе $a < \xi(x) \le a + 0 \Ra \xi(x) \to a + 0$.
	Отсюда получаем
	\[
		\liml_{x \to a + 0} \frac{f'(\xi(x))}{g'(\xi(x))} = \liml_{x \to a + 0} \frac{f'(x)}{g'(x)} = C = \liml_{x \to a + 0} \frac{f(x)}{g(x)}
	\]
\end{proof}

\begin{corollary} (Признак дифференцируемости)
	Если $f$ дифференцируема в $\mc{U}_\delta(x_0)$,
	непрерывна в $x_0$ и $\exists \liml_{x \to x_0} f'(x)
	\in \bar{\R}$, то
	\[
	\exists f'(x_0) = \liml_{x \to x_0} f'(x)
	\]
\end{corollary}

\begin{proof}
	Пусть $F(x) = f(x) - f(x_0)$, $g(x) = x - x_0$. Тогда,
	$\forall x \in \mc{U}_\delta(x_0)$
	\begin{align*}
	&F'(x) = f'(x)
	\\
	&g'(x) = 1
	\end{align*}
	Следовательно,
	\[
	\exists \liml_{x \to x_0} \frac{F'(x)}{g'(x)} =
	\liml_{x \to x_0} \frac{f'(x)}{g'(x)} \in \bar{\R}
	\]
	при этом $\liml_{x \to x_0} F(x) = \liml_{x \to x_0}
	g(x) = 0$. А значит по правилу Лопиталя
	\[
	\exists \liml_{x \to x_0} \frac{F(x)}{g(x)} =
	\liml_{x \to x_0} \frac{F'(x)}{g'(x)} =
	\liml_{x \to x_0} \frac{f'(x)}{g'(x)} =
	\liml_{x \to x_0} f'(x)
	\]
	С другой стороны
	\[
	\liml_{x \to x_0} \frac{F(x)}{g(x)} =
	\liml_{x \to x_0} \frac{f(x) - f(x_0)}{x - x_0} =
	f'(x_0) \in \bar{\R}
	\]
\end{proof}

\begin{theorem} (Правило Лопиталя для случая
	$\frac{\infty}{\infty}$)
	Пусть $f$, $g$ дифференцируемы на $(a; b)$,
	$\exists \liml_{x \to a + 0} g(x) = \pm \infty$ и
	$\exists \liml_{x \to a + 0} \frac{f'(x)}{g'(x)} =
	C \in \bar{\R}$. Тогда
	\[
		\exists \liml_{x \to a + 0} \frac{f(x)}{g(x)} = C
	\]
\end{theorem}

\begin{note}
	Аналогичное утверждение верно для предела $x \to b-0$,
	а также для любого предела $x \to x_0,\ x_0 \in (a; b)$.
\end{note}

\begin{proof}
	Докажем случай для $\liml_{x \to a + 0} g(x) = +\infty$:
	\begin{itemize}
		\item $C = -\infty$. Тогда, выберем $\forall p > q > C$.
		Теперь из пределов:
		\begin{align*}
			&\liml_{x \to a + 0} \frac{f'(x)}{g'(x)} = -\infty \Ra
			(\exists \delta_1 > 0)(\forall x \in
			(a; a + \delta_1))\ \frac{f'(x)}{g'(x)} < q
			\\
			&\liml_{x \to a + 0} g(x) = +\infty \Ra 
			(\exists \delta_2 \in (0; \delta_1))
			(\forall x \in (a; a + \delta_2))\ g(x) > 0
		\end{align*}
		Зафиксируем $y > x > a,\ y \in (a; a + \delta_2)$. Тогда из
		предела $\liml_{x \to a + 0} g(x) = +\infty \Ra$
		\[
			(\exists \delta_3 > 0)(\forall x \in
			(a; a + \delta_3) \subset (a; y))\ g(y) - g(x) < 0
		\]
		Заметим, что $f$, $g$ удовлетворяют условиям теоремы
		Коши о среднем \ref{Cauchy_mid} на отрезке $[x; y]$ ($g' \neq 0$
		следует из существования предела производных). То есть
		\[
			\left(\exists \xi \in (x; y) \subset (a; a + \delta_1)\right)
			\ \frac{f(y) - f(x)}{g(y) - g(x)} =
			\frac{f'(\xi)}{g'(\xi)} < q
		\]
		Если убрать равенство с $\xi$, то получим
		\[
			\frac{f(y) - f(x)}{g(y) - g(x)} < q
		\]
		Умножим обе части на $\frac{g(x) - g(y)}{g(x)} > 0$
		при $x \in (a; a + \delta_3)$. Отсюда имеем
		\begin{align*}
			&{\frac{f(x) - f(y)}{g(x)} < q \cdot
			\frac{g(x) - g(y)}{g(x)}}
			\\
			&{\Ra \frac{f(x)}{g(x)} < q - q \cdot
			\frac{g(y)}{g(x)} + \frac{f(y)}{g(x)}}
		\end{align*}
		Из последнего неравенства следует, что
		\[
			(\forall p > C)(\exists \delta_4 > 0)
			(\forall x \in (a; a + \delta_4))\ \frac{f(x)}{g(x)} < p
		\]
		Откуда согласно $C = -\infty$ имеем
		\[
			(\forall \eps > 0)(\exists \delta_4 > 0)
			(\forall x \in (a; a + \delta_4))\ 
			\frac{f(x)}{g(x)} < -\frac{1}{\eps} \lra 
			\liml_{x \to a + 0} \frac{f(x)}{g(x)} = -\infty =
			\liml_{x \to a + 0} \frac{f'(x)}{g'(x)}
		\]
		
		\item $C = +\infty$. Тогда аналогично случаю выше,
		получается утверждение
		\[
			(\forall r < C)(\exists \delta_5 > 0)
			(\forall x \in (a; a + \delta_5))\ 
			\frac{f(x)}{g(x)} > r
		\]
		
		\item $-\infty < C < +\infty$. Ключевые утверждения,
		полученные выше, будут верны и в этом случае,
		потому что $\exists p > C > r$. А значит
		\[
			(\forall \eps > 0)(\exists \delta :=
			\min(\delta_4, \delta_5))(\forall x \in (a; a + \delta))
			\ r < \frac{f(x)}{g(x)} < p
		\]
		выберем теперь $r := C - \eps,\ p := C + \eps$. То есть
		\[
			(\forall \eps > 0)(\exists \delta > 0)
			(\forall x \in (a; a + \delta))\ 
			\left|\frac{f(x)}{g(x)} - C\right| < \eps
			\lra \liml_{x \to a + 0} \frac{f(x)}{g(x)} = C =
			\liml_{x \to a + 0} \frac{f'(x)}{g'(x)}
		\]
	\end{itemize}
\end{proof}

\begin{note}
	Правило Лопиталя работает и в случаях, когда $x \to \pm \infty$
\end{note}

\begin{proof}
	Выберем такое $t = \frac{1}{x}$, тогда $t \to +0$ при
	$x \to +\infty$. Пусть $f_1(t) = f(\frac{1}{t})$ и 
	$g_1(t) = g(\frac{1}{t})$
	\[
		f_1'(t) = f'\left(\frac{1}{t}\right) \cdot
		\left(-\frac{1}{t^2}\right),\ g_1'(t) = g'\left(\frac{1}{t}\right) \cdot
		\left(-\frac{1}{t^2}\right)
	\]
	Применим правило Лопиталя для $t \to +0$:
	\[
		\liml_{x \to +\infty} \frac{f(x)}{g(x)} =
		\liml_{t \to +0} \frac{f_1(t)}{g_1(t)} =
		\liml_{t \to +0} \frac{f_1'(t)}{g_1'(t)} =
		\liml_{x \to +\infty} \frac{f'(x)}{g'(x)}
	\]
\end{proof}

\subsection{Равномерная непрерывность}

\begin{definition}
	Функция $f$ \textit{равномерно непрерывна} на
	множестве $X \subset \R$, если
	\[
		(\forall \eps > 0)(\exists \delta > 0)
		(\forall x, y \in X, |x - y| < \delta)\ 
		\left|f(x) - f(y)\right| < \eps
	\]
\end{definition}

\begin{note}
	Отличие от обычного определения заключается в том,
	что выбор $\delta$ не зависит от рассматриваемой точки $x$.
\end{note}

\begin{example}
	\[
		X = (0; 1),\ f(x) = \frac{1}{x}
		\text{ - неравномерно непрерывна}
	\]
	То есть нужно доказать утверждение
	\[
		(\exists \eps > 0)(\forall \delta > 0)
		(\exists x, y \in X, |x - y| < \delta)\ 
		\left|\frac{1}{x} - \frac{1}{y}\right| \ge \eps
	\]
	Для любого $0 < \delta < 1$ найдётся $n \in \N$ такое,
	что верно неравенство
	\[
		\frac{1}{n} \le \delta < \frac{1}{n - 1}
	\]
	Положим $x_n = \frac{1}{n}$, а $y_n = \frac{1}{3n}$. Тогда
	\begin{align*}
		&{|x_n - y_n| = \frac{2}{3n} < \frac{1}{n} \le \delta}
		\\
		&{\left|\frac{1}{x_n} - \frac{1}{y_n}\right| = 2n \ge 2}
	\end{align*}
	Отсюда наше утверждение выполнено $\forall \eps \le 2$,
	для $\delta \ge 0$ можно взять $n = 1$ и
	$\eps = 2$.
\end{example}