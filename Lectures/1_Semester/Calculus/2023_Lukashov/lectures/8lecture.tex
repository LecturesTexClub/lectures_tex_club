\begin{definition}
	\textit{Верхним пределом последовательности} $\{x_n\}_{n = 1}^\infty
	\subset R$ называется наибольший из её частичных пределов
	$\overline{\liml_{n \ra \infty}} x_n$.
\end{definition}

\begin{definition}
	\textit{Нижним пределом последовательности} $\{x_n\}_{n = 1}^\infty
	\subset R$ называется наименьший из её частичных пределов
	$\varliminf\limits_{n \ra \infty} x_n$.
\end{definition}

\begin{anote}
	Следует помнить, что частичный предел может быть бесконечным.
	Следовательно, верхний и нижний тоже.
\end{anote}

\begin{theorem} (3 определения верхнего и нижнего пределов)
	Для любой ограниченной последовательности $\{x_n\}_{n = 1}^\infty$
	существуют \underline{конечные} $L = \overline{\liml_{n \to \infty}}
	x_n$, $l = \varliminf\limits_{n \to \infty} x_n$.
	Для них справедливы следующие утверждения:
	\begin{enumerate}
		\item
		\begin{enumerate}
			\item верхний предел
			$\System{
			(\forall \eps > 0)(\exists N \in \N)
			(\forall n > N)\ x_n < L + \eps 
			\\
			(\forall \eps > 0)(\forall N \in \N)(\exists n > N)
			\ x_n > L - \eps}$ 
			
			\item нижний предел
			$\System{
			(\forall \eps > 0)(\exists N \in \N)(\forall n > N)
			\ x_n > l - \eps
			\\
			(\forall \eps > 0)
			(\forall N \in \N)(\exists n > N)\ x_n < l + \eps}$
		\end{enumerate}
		
		\item 
			$L = \liml_{n \to \infty} \sup \{x_n, x_{n + 1} \dots\}$
		
			$\ l = \liml_{n \to \infty} \inf \{x_n, x_{n + 1}, \dots\}$
	\end{enumerate}
	Причём определения равносильны (стандартное и эти 2 пункта).
\end{theorem}

\begin{proof}
	Доказательство приводится только для верхнего предела.
	Для нижнего оно аналогично.
	
	Рассмотрим последовательность $s_n := \sup \{x_n, x_{n + 1},
	\dots\} = \sup\limits_{m \ge n} x_m$. Мы можем это сделать,
	так как $\{x_n\}_{n = 1}^\infty$ ограничена по условию
	теоремы. Несложно заметить 2 утверждения из данного
	определения:
	\begin{align*}
		&s_n \ge s_{n + 1}
		\\
		&s_n \ge \inf \{x_n\}
	\end{align*}
	А значит по теореме Вейерштрасса, данная последовательность
	сходится и имеет предел $L := \liml_{n \to \infty} s_n = \inf
	\{s_n\}$. 
	
	Покажем, что для этой последовательности верен первый пункт.
	Тут есть два варианта доказательства:
	\begin{itemize}
		\item 
			По определению предела 
			\[
				(\forall \eps > 0)(\exists N \in \N)(\forall n > N)
				\ |s_n - L| < \eps
			\]
			Так как $s_n := \sup \{x_n, x_{n + 1}, \dots\}$, то
			$x_n \le s_n < L + \eps$ (доказано следствие первой части
			п.1. из п.2.).
			
			Рассмотрим $N \in \N$ и $s_{N + 1} = \sup \{x_{N + 1},
			x_{N + 2}, \dots\}$. Так как $L = \inf \{s_n\}$, то 
			\[
				s_{N + 1} \ge L
			\]
			А так как $s_n := \sup \{x_n, x_{n + 1}, \dots\}$,
			то ещё имеем
			\[
				(\forall \eps > 0)(\forall N \in \N)(\exists n > N)
				\ x_n > s_{N + 1} - \eps \ge L - \eps \Ra x_n > L -\eps
			\]
			(доказано следствие второй части п.1. из п.2.)
		\item
			Так же из определения предела:

			$(\forall \eps > 0)(\exists N \in \N)
			(\forall n > N)\ |s_n - L| < \eps \lra L - \eps <
			s_n < L + \eps$
			
			$\System{
				s_n < L + \eps \Ra (\forall m \ge n)\ x_m <
				L + \eps \Ra (\forall \eps > 0)(\exists N \in \N)
				(\forall m \ge N)\ x_m < L + \eps
				\\
				L - \eps < s_n \Ra (\exists m \ge n)\ x_m >
				L - \eps \Ra (\forall \eps > 0)(\forall N \in \N)
				(\exists m \ge N)\ x_m > L - \eps
			}$
	\end{itemize}
	
	
	Теперь докажем, что из пункта 1 $\Ra L$ - наибольший частичный
	предел $\{x_n\}$. Построим подпоследовательность:
	\begin{align*}
		&\eps := 1 & &\System{&(\exists N \in \N)(\forall n > N)
		\ x_n < L + 1
		\\ &(\forall N \in \N)(\exists n_1 > N)\ x_{n_1} > L - 1}
		\Ra |x_{n_1} - L| < 1
		\\
		&\eps := \frac{1}{2} & &\System{&(\exists N \in \N)
		(\forall n > N)\ x_n < L + \frac{1}{2}
		\\ &(\forall N \in \N)(\exists n_2 > \max{(n_1, N)})
		\ x_{n_2} > L - \frac{1}{2}} \Ra |x_{n_2} - L| < \frac{1}{2}
		\\
		&\dots& & \dots
		\\
		&\eps := \frac{1}{k} & &\System{&(\exists N \in \N)
		(\forall n > N)\ x_n < L + \frac{1}{k}
		\\ &(\forall N \in \N)(\exists n_k > \max{(n_{k - 1}, N)})
		\ x_{n_k} > L - \frac{1}{k}} \Ra \underbrace{0}_{\to 0} \le
		|x_{n_k} - L| < \underbrace{\frac{1}{k}}_{\to 0}\Ra \exists
		\liml_{k \to \infty} x_{n_k} = L
	\end{align*}



	Существование номера обусловлено тем, что мы вначале
	применяем первую часть пункта 1., а затем подставляем во
	вторую часть пункта 1 и находим такое $n_k$, что для него
	верны оба неравенства сразу.
	
	Получили $\{x_{n_k}\}_{k = 1}^\infty$ такую, что
	$\liml_{k \to \infty} x_{n_k} = L$. Осталось доказать, что
	этот частичный предел и есть наибольший:
	
	Рассмотрим произвольную $\{x_{m_i}\}_{i = 1}^\infty$ такую,
	что $\exists \liml_{i \to \infty} x_{m_i} = t$.
	Из уже доказанного пункта 1. следует, что
	\[
		(\forall \eps > 0)(\exists I \in \N)(\forall i > I)
		\ x_{m_i} < L + \eps
	\]
	
	Совершая предельный переход в неравенстве, получим
	\[
		(\forall \eps > 0)\ t \le L + \eps
	\]
	Отсюда понятно, что $t \le L$, то есть $L$ действительно наибольший
	частичный предел.
\end{proof}

\begin{anote}
	По моему мнению, ключевая идея выше в том, что мы всегда
	из-за ограниченности можем рассмотреть последовательность
	$s_n$ и доказать, что её предел либо удовлетворяет другому
	определению, либо свойству (которое можно принять за
	определение).
\end{anote}

\begin{definition}
	Последовательность $\{x_n\}_{n = 1}^\infty$ называется
	\textit{фундаментальной}, или же \textit{последовательностью
	Коши}, если $(\forall \eps > 0)(\exists N \in \N)
	(\forall n > N)(\forall p \in \N)\ |x_{n + p} - x_n| < \eps$
	Эквивалентная форма: $(\forall \eps > 0)(\exists N \in \N)
	(\forall m, n > N)\ |x_m - x_n| < \eps$
\end{definition}

\begin{theorem} (Критерий Коши)
	Последовательность $\{x_n\}_{n = 1}^\infty$ сходится тогда и
	только тогда, когда она фундаментальна. 
\end{theorem}

\begin{proof}:
	\begin{enumerate}
		\item (Необходимость) Сходимость $\Ra$ фундаментальность
		
		Пусть $\exists \liml_{n \ra \infty} x_n = l$, тогда
		\[
			(\forall \eps > 0)(\exists N \in \N)
			(\forall n > N)\ |x_n - l| < \frac{\eps}{2}
		\]
		
		Тогда $(\forall p \in \N)\ n + p > N \Ra |x_{n + p} - l|
		< \frac{\eps}{2}$
		
		$|x_{n + p} - x_n| = |x_{n + p} - l + l - x_n| \le
		|x_{n + p} - l| + |l - x_n| < \frac{\eps}{2} +
		\frac{\eps}{2} = \eps$
		
		\item (а) Фундаментальность $\Ra$ ограниченность
		
		Согласно свойству фундаментальности, положим
		$\eps := 1 \Ra n := N + 1$. Теперь
		\[
			(\forall p \in \N)\ |x_{N + 1 + p} - x_{N + 1}| < 1
			\Ra x_{N + 1} - 1 < x_{N + 1 + p} < x_{N + 1} + 1
		\]
		Отсюда для $(\forall m \in \N)$
		\[
			\min(x_1, \dots, x_{N + 1}) - 1 < x_m <
			\max(x_1, \dots, x_{N + 1}) + 1
 		\]
 		
 		\item (б) Фундаментальность $\Ra$ ограниченность $\Ra$
		сходимость. 
 		
		Так как последовательность ограничена, то по теореме
		Больцано-Вейерштрасса можно выделить сходящуюся
		подпоследовательность.
		\[
			(\exists \{x_{n_k}\}_{k = 1}^\infty)\ \liml_{k \to \infty} x_{n_k} = l
		\]
		По определению предела,
		\[
			(\forall \eps > 0)(\exists K \in \N)(\forall k > K)
			\ |x_{n_k} - l| < \frac{\eps}{2}
		\]
		При этом исходная последовательность фундаментальна. То есть
		\[
			(\forall \eps > 0)(\exists N \in \N)(\forall n > N)
			(p \in \N)\ |x_{n + p} - x_n| < \frac{\eps}{2}
		\]
		Рассмотрим $(\forall m > \max(N, n_{K + 1}))$, тогда
		\[
			|x_m - l| \le |x_m - x_{n_{K + 1}}| +
			|x_{n_{K + 1}} - l| < \frac{\eps}{2} + \frac{\eps}{2} = \eps
		\]
	\end{enumerate}
\end{proof}

\begin{theorem} (Число Эйлера)
	Последовательность $\{x_n = \left(1 +
	\frac{1}{n}\right)^n\}_{n = 1}^\infty$ сходится.
	Её предел называется числом $e$.
	\[
		e \approx 2,718281828459045\dots
	\]
\end{theorem}

\begin{proof}
	Рассмотрим последовательность $y_n :=
	\left(1 + \frac{1}{n}\right)^{n + 1}$.
	Докажем, что $y_n$ убывает. Используем неравенство Бернулли:
	$(\forall x > -1)(\forall n \in \N)\ (1 + x)^{n} \ge 1 + nx$ (*)
	\begin{multline*}
		\frac{y_{n - 1}}{y_n} = \frac{(1 + \frac{1}{n - 1})^n}
		{(1 + \frac{1}{n})^{n + 1}} = \left(\frac{\frac{n}{n - 1}}
		{\frac{n + 1}{n}}\right)^{n+1} \cdot \frac{1}{1 + \frac{1}
		{n - 1}} = \left(\frac{n^2}{n^2 - 1}\right)^{n+1}
		\cdot \frac{1}{1 + \frac{1}{n - 1}} = 
		\\
		=\left(1 + \frac{1}{n^2 - 1}\right)^{n+1} \cdot
		\frac{1}{1 + \frac{1}{n - 1}} \underset{(*)}{\ge}
		\left(1 + \frac{n + 1}
		{n^2 - 1}\right) \cdot \frac{1}{1 + \frac{1}{n - 1}} = 
		\\
		=\left(1 + \frac{1}{n - 1}\right) \cdot \frac{1}{1 + 
		\frac{1}{n - 1}} = 1,\  n > 1 \Ra \frac{y_{n - 1}}{y_n}
		\ge 1 \Ra y_n \le y_{n - 1}
	\end{multline*}
	При этом $\{y_n\}$ - ограниченная снизу последовательность,
	так как $(\forall n \in \N)\ y_n \ge 0$
	
	Следовательно, по теореме Вейерштрасса $\{y_n\}$ сходится.
	Её предел равен $e$.
	
	Покажем, что к тому же пределу сходится и $x_n$:
	\[
		\liml_{n \to \infty} x_n = 
		\frac{\liml_{n \to \infty} y_n }{\liml_{n \to \infty}
		\left(1 + \frac{1}{n}\right)} = \frac{e}{1 + 0} = e
	\]
\end{proof}