\begin{definition}
	Множество $X$ называется \textit{топологическим пространством},
	если в нём выделена совокупность подмножеств $\Tau$,
	называемых \textit{открытыми}, которая удовлетворяет
	свойствам из теоремы \ref{mainProp}.
	
	Множество $\Tau$ называется \textit{топологией} множества $X$.

	В таком случае любое множество из $\Tau$,
	содержащее точку $x \in X$ называют
	окрестностью точки $x$.
\end{definition}

\begin{definition} Предел последовательности:
	если $\{x_n\}_{n = 1}^\infty \subset X,\ x_0 \in X$
	\begin{enumerate}
		\item Топологическое пространство
			\[
				\liml_{n \to \infty} x_n = x_0 \lra
				(\forall G \in \Tau, x_0 \in G)(\exists N \in \N)
				(\forall n > N)\ x_n \in G
			\]
		\item Метрическое пространство: 
			\[
				\liml_{n \to \infty} x_n = x_0 \lra
				(\forall \eps > 0)(\exists N \in \N)
				(\forall n > N)\ \rho(x_n, x_0) < \eps
			\]
		\item Линейное нормированное пространство
			\[
				\liml_{n \to \infty} x_n = x_0 \lra
				(\forall \eps > 0)(\exists N \in \N)
				(\forall n > N)\ \|x_n - x_0\| < \eps
			\]
		\item Линейное действительное пространство $\R^n$	
			\[
				\liml_{n \to \infty} \vec{x}_n = \vec{x}_0 \lra
				(\forall \eps > 0)(\exists N \in \N)
				(\forall n > N)\ |\vec{x}_n - \vec{x}_0| < \eps
			\]
	\end{enumerate}
\end{definition}

\begin{example}
	Если не накладывать на топологию никаких
	дополнительных ограничений, то предел в топологическом
	пространстве не
	обязан даже быть единственным. Рассмотрим
	$X = \{a, b\}$ с топологией $\Tau =
	\{\emptyset, \{a\}, \{a, b\}\}$. Рассмотрим
	последовательность
	\[
		x_n = a,\ n \in \N
	\]
	Тогда понятно, что $\liml_{n \to \infty} x_n =
	a,\ \liml_{n \to \infty} x_n = b$.
\end{example}

\begin{anote}
	Чтобы обеспечить единственность предела,
	достаточно добавить свойство \textit{хаусдорфовости}:
	\[
		(\forall x, y \in X)\ \exists G_x, G_y \in
		\Tau \such (x \in G_x) \wedge (y \in G_y)
		\wedge (G_x \cap G_y = \emptyset)
	\]
	То есть для каждой точки должно существовать
	изолированное открытое множество.
	Такое топологическое пространство называется
	\textit{хаусдорфовым}.
\end{anote}

\subsection{Топология пространства $\R^n$ и непрерывные отображения}

\begin{theorem} (Основные свойства предела последовательности)
	\begin{enumerate}
		\item \underline{Единственность предела}
			
			В метрическом пространстве
			последовательность $\{x_n\}_{n = 1}^\infty$ не может
			иметь более одного предела
		
		\item \underline{Ограниченность сходящейся последовательности}
		
			Если $\{x_n\}_{n = 1}^\infty$ --- сходящаяся последовательность 
			в метрическом пространстве,
			то она ограничена, то есть существует открытый шар, содержащий
			все точки последовательности:
			\[
				(\exists x_0 \in X)(\exists R > 0)(\forall n \in \N)\ x_n \in U_R(x_0)
			\] 
		
		\item \underline{Отделимость от нуля}
		
			Если $\{x_n\}_{n = 1}^\infty \subset E$ ---
			последовательность ЛНП, сходящаяся
			к $x_0 \neq 0 \in E$, то она отделена от нуля. То есть
			\[
				(\exists c > 0)(\exists N \in \N)(\forall n > N)\ ||x_n|| > c
			\]
			
		\item \underline{Предел и арифметические операции}
		
			Если последовательности $\{x_n\}_{n = 1}^\infty$,
			$\{y_n\}_{n = 1}^\infty \subset E$ (ЛНП) - сходящиеся к
			$x_0, y_0 \in E$ соответственно,
			$\{\alpha_n\}_{n = 1}^\infty \subset \R(\Cm)$ сходится к
			$\alpha_0 \in \R(\Cm)$, то
			\begin{enumerate}
				\item $\liml_{n \to \infty} (x_n + y_n) = x_0 + y_0$
				
				\item $\liml_{n \to \infty} (\alpha_n \cdot x_n) =
					\alpha_0 \cdot x_0$
			\end{enumerate}
	
		\item \underline{Предел и скалярное произведение}
		
			Если $\{x_n\}_{n = 1}^\infty, \{y_n\}_{n = 1}^\infty \subset E$
			--- последовательности в евклидовом пространстве $E$,
			сходящиеся к $x_0, y_0$ соответственно, то
			\[
				\liml_{n \to \infty} \trbr{x_n, y_n} = \trbr{x_0, y_0}
			\]
		
		\item \underline{Предел и векторное произведение}
		
			Если последовательности
			$\{\vec{x}_n\}_{n = 1}^\infty,
			\{\vec{y}_n\}_{n = 1}^\infty \subset \R^3$ -
			сходящиеся к $\vec{x}_0, \vec{y}_0$ соответственно, то
			\[
				\liml_{n \to \infty} [\vec{x}_n, \vec{y}_n] =
				[\vec{x}_0, \vec{y}_0]
			\]
	\end{enumerate}
\end{theorem}

\begin{note}
	Так как $\R^n$ является метрическим, линейным нормированным,
	евклидовым пространством, то свойства $1-5$ справедливы для $\R^n$
\end{note}

\begin{idea}
	В целом все доказательства аналогичны доказательствам
	для последовательностей действительных чисел
	\begin{enumerate}
		\item От противного, показываем, что расстояние между
			двумя предполагаемыми пределами меньше, чем мы предположили
		\item Говорим, что все члены, начиная с какого-то лежат
			в определенном множестве, а дальше пользуемся фактом,
			что до этого есть только конечное число членов последовательности
		\item Эпсилон берем половиной от предела и пользуемся неравенстом
			треугольника
		\item Берем правильные эпсилоны (во втором случае пользуемся
			ограниченностью сходящейся последовательности) и складываем
			их по неравенству треугольника
		\item Аналогично 4 + неравенство Коши-Буняковского-Шварца +
			определение нормы через корень скалярного произведения
		\item Аналогично 4.
	\end{enumerate}
\end{idea}

\begin{proof}~
	\begin{enumerate}
		\item От противного. Пусть $\liml_{n \to \infty}
			x_n = x_0,\ \liml_{n \to \infty} x_n = y_0,\ 
			x_0 \neq y_0$. Из условия и свойств метрики сразу следует,
			что $\rho(x_0, y_0) > 0$. Рассмотрим
			$\eps := \frac{1}{2}\rho(x_0, y_0)$:
			\begin{align*}
				&(\exists N_1 \in \N)(\forall n > N_1)
				\ \rho(x_n, x_0) < \eps
				\\
				&(\exists N_2 \in \N)(\forall n > N_2)\ 
				\rho(x_n, y_0) < \eps
			\end{align*}
			Следовательно, если положить $N := \max(N_1, N_2)$, то
			\[
				(\forall n > N)\ \rho(x_0, y_0) \le \rho(x_0, x_n) + \rho(x_n, y_0) < 2\eps = \rho(x_0, y_0)
			\]
			Противоречие.
		
		\item Положим $\eps := 1$. Обозначим за $x_0$ точку,
			к которой сходится последовательность. Тогда из условия
			\[
				(\exists N \in \N)(\forall n > N)\ 
				\rho(x_n, x_0) < 1
			\]
			Обозначим за $R$ следующую величину:
			\[
				R := \max(\rho(x_1, x_0), \rho(x_2, x_0),
				\ldots, \rho(x_N, x_0)) + 1
			\]
			Из определения следует, что
			\[
				(\forall n \in \N)\ \rho(x_n, x_0) < R
			\]
			То есть все точки последовательности лежат
			в открытом шаре радиуса $R$ и с центром в точке $x_0$
		
		\item По определению
			\[
				(\forall \eps > 0)(\exists N \in \N)
				(\forall n > N)\ \|x_n - x_0\| < \eps
			\]
			Положим $\eps := \frac{\|x_0\|}{2}$. По
			неравенству треугольника имеем
			\[
				\|x_0\| = \|x_0 - x_n + x_n\| \le
				\|x_0 - x_n\| + \|x_n\| =
				\|x_n - x_0\| + \|x_n\| <
				\frac{\|x_0\|}{2} + \|x_n\|
			\]
			А уже отсюда
			\[
				\|x_n\| > \frac{\|x_0\|}{2}
			\]
		
		\item
		\begin{enumerate}
			\item Раз исходные последовательности сходятся,
			то справедливы утверждения
			\begin{align*}
				&(\forall \eps > 0)(\exists N_1 \in \N)
				(\forall n > N_1)\ \ \|x_n - x_0\| < \frac{\eps}{2}
				\\
				&(\forall \eps > 0)(\exists N_2 \in \N)
				(\forall n > N_2)\ \ \|y_n - y_0\| < \frac{\eps}{2}
			\end{align*}
			Ну и как обычно: $N := \max(N_1, N_2)$ и тогда
			$\forall n > N$ оба неравенства верны одновременно. Отсюда
			\[
				\|(x_n + y_n) - (x_0 + y_0)\| =
				\|(x_n - x_0) + (y_n - y_0)\| \le
				\|x_n - x_0\| + \|y_n - y_0\| < \eps
			\]
			
			\item По уже доказанному свойству,
				сходящаяся последовательность ограничена:
				\[
					(\exists C > 0)(\forall n \in \N)\ \ \|x_n\| < C
				\]
				Из условия можем также заключить два утверждения:
				\begin{align*}
					&(\forall \eps > 0)(\exists N_1 \in \N)
					(\forall n > N_1)\ \ |\alpha_0| \cdot
					\|x_n - x_0\| < \frac{\eps}{2}
					\\
					&(\forall \eps > 0)(\exists N_2 \in \N)
					(\forall n > N_2)\ \ |\alpha_n - \alpha_0|
					< \frac{\eps}{2C}
				\end{align*}
				Мы не стали переносить $|\alpha_0|$ в знаменатель,
				так как это число может быть нулем.
				В итоге имеем $N := \max(N_1, N_2)$ и $\forall n > N$:
				\begin{multline*}
					\|\alpha_n x_n - \alpha_0 x_0\| \le
					\|\alpha_n x_n - \alpha_0 x_n\| +
					\|\alpha_0 x_n - \alpha_0 x_0\| =
					\\
					|\alpha_n - \alpha_0| \cdot \|x_n\| +
					|\alpha_0| \cdot \|x_n - x_0\| <
					\frac{\eps}{2C} \cdot C + \frac{\eps}{2} = \eps
				\end{multline*}
		\end{enumerate}
	
		\item Аналогично предыдущему пункту
		\begin{align*}
			&(\exists C > 0)(\forall n \in \N)\ \ \|x_n\| < C
			\\
			&(\forall \eps > 0)(\exists N_1 \in \N)
			(\forall n > N_1)\ \ \|y_0\| \cdot
			\|x_n - x_0\| < \frac{\eps}{2}
			\\
			&(\forall \eps > 0)(\exists N_2 \in \N)
			(\forall n > N_2)\ \ \|y_n - y_0\| < \frac{\eps}{2C}
		\end{align*}
		Теперь $N := \max(N_1, N_2)$ и тогда $\forall n > N$, используя
		неравенство Коши-Буняковского-Шварца (\ref{Cauchy–Schwarz}) и определение
		нормы через корень скалярного произведения:
		\begin{multline*}
			|\trbr{x_n, y_n} - \trbr{x_0, y_0}| \le
			|\trbr{x_n, y_n} - \trbr{x_n, y_0}| +
			|\trbr{x_n, y_0} - \trbr{x_0, y_0}| =
			\\
			= |\trbr{x_n, y_n - y_0}| + |\trbr{x_n - x_0, y_0}|
			\le \|x_n\| \cdot \|y_n - y_0\| +
			\|x_n - x_0\| \cdot \|y_0\| <
			\\
			< C \cdot \frac{\eps}{2C} + \frac{\eps}{2} = \eps
		\end{multline*}
		
		\item Снова аналогично пункту про скалярное произведение
		\begin{align*}
			&(\exists C > 0)(\forall n \in \N)\ \ |x_n| < C
			\\
			&(\forall \eps > 0)(\exists N_1 \in \N)
			(\forall n > N_1)\ \ |y_0| \cdot |x_n - x_0|
			< \frac{\eps}{2}
			\\
			&(\forall \eps > 0)(\exists N_2 \in \N)
			(\forall n > N_2)\ \ |y_n - y_0| < \frac{\eps}{2C}
		\end{align*}
		Положим $N := \max(N_1, N_2)$ и рассмотрим $\forall n > N$,
		тогда по неравенству треугольника для нормы:
		\[
			|[x_n, y_n] - [x_0, y_0]| \le |[x_n, y_n] -
			[x_n, y_0]| + |[x_n, y_0] - [x_0, y_0]| \le
			|x_n| \cdot |y_n - y_0| + |y_0| \cdot |x_n - x_0| < \eps
		\]
		Предпоследний переход получен из тех соображений, что
		\[
			|[a, b]| = |a| \cdot |b| \cdot \sin \angle(a, b)
			\le |a| \cdot |b|
		\]
	\end{enumerate}
\end{proof}

\begin{lemma} (Критерий сходимости последовательности в $\R^n$) \label{limCoordinates}
	Последовательность $\{\vec{x}_m = (x_m^{(1)},
	\ldots, x_m^{(n)})\}_{m = 1}^\infty$ сходится
	к $\vec{x}_0 = (x_0^{(1)}, \ldots, x_0^{(n)})$
	тогда и только тогда, когда $\forall j \in
	\{1, \ldots, n\}$ последовательность
	$\{x_m^{(j)}\}_{m = 1}^\infty$ сходится к $x_0^{(j)}$.
\end{lemma}

\begin{idea}
	Доказательство несложно запомнить с помощью этого неравенства:
	\[
		|x_m^{(j)} - x_0^{(j)}| \le
		\underbrace{\sqrt{(x_m^{(1)} - x_0^{(1)})^2 +
		\ldots + (x_m^{(n)} - x_0^{(n)})^2}}_{|\vec{x}_m - \vec{x}_0|}
		\le \left(\max\limits_{j \in \{1, \ldots, n\}}
		|x_m^{(j)} - x_0^{(j)}|\right) \cdot \sqrt{n} 
	\]
\end{idea}

\begin{proof}
	Докажем необходимость $(\Ra)$. По условию
	\[
		(\forall \eps > 0)(\exists N \in \N)
		(\forall m > N)\ \ |\vec{x}_m - \vec{x}_0| < \eps
	\]
	При этом
	\[
		|\vec{x}_m - \vec{x}_0| =
		\sqrt{(x_m^{(1)} - x_0^{(1)})^2 +
		\ldots + (x_m^{(n)} - x_0^{(n)})^2}
	\]
	Отсюда
	\[
		\forall j \in \{1, \ldots, n\}\ \ 
		|x_m^{(j)} - x_0^{(j)}| \le
		\sqrt{(x_m^{(1)} - x_0^{(1)})^2 +
		\ldots + (x_m^{(n)} - x_0^{(n)})^2} =
		|\vec{x}_m - \vec{x}_0| < \eps
	\]
	
	Теперь докажем достаточность. Из условия
	\[
		(\forall j \in \range{n})(\eps_j := \frac{\eps}{\sqrt{n}})
		(\exists N_j \in \N)(\forall n > N_j)\ \ 
		|x_m^{(j)} - x_0^{(j)}| < \frac{\eps}{\sqrt{n}}
	\]
	Снова распишем метрику:
	\[
		|\vec{x}_m - \vec{x}_0| =
		\sqrt{(x_m^{(1)} - x_0^{(1)})^2 + \ldots
		+ (x_m^{(n)} - x_0^{(n)})^2} \le
		\left(\max\limits_{j \in \{1, \ldots, n\}}
		|x_m^{(j)} - x_0^{(j)}|\right) \cdot \sqrt{n} <
		\frac{\eps}{\sqrt{n}} \cdot \sqrt{n} = \eps
	\]
	Это верно, так как всего под корнем $n$ слагаемых, то есть
	корень общей суммы не превышает произведение корня максимального
	элемента этой суммы на $\sqrt{n}$.
\end{proof}

\begin{theorem} (Больцано-Верейштрасса в $\R^n$)
	Из каждой ограниченной последовательности в $\R^n$
	можно выделить сходящуюся подпоследовательность.
\end{theorem}

\begin{idea}
	Так как последовательность ограничена, то и все координаты
	ограничены. Тогда будем выделять последовательно из координатных
	последовательностей сходящиеся подпоследовательности
	по привычной теореме Больцано-Вейерштрасса
	(\ref{Bolzano–Weierstrass}). Важно выделять каждую
	следующую подпоследовательность из предыдущей, чтобы
	она сходилась по всем предыдущим координатам.
\end{idea}

\begin{proof}
	Пусть $\{\vec{x}_m\}_{m = 1}^\infty$ - ограниченная
	последовательность. Это означает, что
	\[
		(\exists C > 0)(\forall m \in \N)\ \ 
		|\vec{x}_m| < C \Ra (\forall j \in \range{n})\ 
		|x_m^{(j)}| < C
	\]
	Рассмотрим $j = 1$.
	Тогда последовательность $\{x_m^{(1)}\}_{m = 1}^\infty$
	- ограниченная, а значит, по теореме
	Больцано-Вейерштрасса, существует
	$\{x_{m_k}^{(1)}\}_{k = 1}^\infty$ -
	сходящаяся подпоследовательность. Теперь из получившейся
	подпоследовательности выделим сходящуюся подпоследовательность,
	но уже по второй координате, то есть получится
	$\{x_{m_{k_i}}^{(2)}\}_{i = 1}^\infty$:
	\[
		\liml_{i \to \infty} x_{m_{k_i}}^{(2)} = x_0^{(2)}
	\]
	Тогда для каждой
	координаты будем выделять подпоследовательность
	из предыдущей подпоследовательности и получим $x_0 \in \R^n$
	с координатами:
	\[
		(\forall j \in \range{n})\ 
		x_0^{(j)} = \liml_{k \to \infty}
		x_{m_{k_{i_{\dots_{s}}}}}^{(j)}
	\]
	Применим доказанную выше лемму и получим, что наша
	построенная последовательность
	$\{\vec{x}_{m_{k_{i_{\dots_{s}}}}}\}_{s = 1}^\infty$
	сходится к $\vec{x}_0$
\end{proof}

\begin{definition}
	\textit{Фундаментальной последовательностью в
	метрическом пространстве} $(X, \rho)$
	называется такая последовательность
	$\{x_n\}_{n = 1}^\infty \subset X$, что
	\[
		(\forall \eps > 0)(\exists N \in \N)
		(\forall n > N)(\forall p \in \N)\ \ 
		\rho(x_n, x_{n + p}) < \eps
	\]
\end{definition}

\begin{theorem} (Критерий Коши в $\R^n$)
	Последовательность
	$\{\vec{x}_m\}_{m = 1}^\infty \subset \R^n$ сходится
	тогда и только тогда, когда
	$\{\vec{x}_m\}_{m = 1}^\infty$ фундаментальная
\end{theorem}

\begin{idea}
	Необходимость доказывается аналогично Коши для $\R$, то
	есть выбираем правильный эпсилон и раскрываем по неравенству
	треугольника. Достаточность доказываем покоординатно:
	фундаментальность в $\R^n$ означает
	фундаментальность для каждой координаты, тогда
	по критерию Коши каждая координата сходится,
	а значит, и последовательность в $\R^n$ тоже 
	сходится (по критерию выше).
\end{idea}

\begin{proof}~
	\begin{itemize}
		\item Сходимость $\Ra$ Фундаментальность 
			(это верно в \textbf{любом} метрическом
			пространстве)
			
			По условию
			\[
				(\forall \eps > 0)(\exists M \in \N)(\forall m > M)
				\ \ \rho(x_m, x_0) < \frac{\eps}{2}
			\]
			Оценим $\rho(x_m, x_{m + p})$ для
			$\forall p \in \N$ при уже зафиксированных
			$\eps$ и $M$:
			\[
				\rho(x_m, x_{m + p}) \le \rho(x_m, x_0) +
				\rho(x_0, x_{m + p}) < \frac{\eps}{2} +
				\frac{\eps}{2} = \eps
			\]
			
		\item Фундаментальность $\Ra$ Сходимость
			(эта часть верна \textbf{только для} $R^n$).
			
			По определению фундаментальности и метрики в $\R^n$:
			\[
				(\forall \eps > 0)(\exists M \in \N)
				(\forall m > M)(p \in \N)\ \ 
				|\vec{x}_m - \vec{x}_{m + p}| < \eps
			\]
			Следовательно, для $\forall j \in \range{n}$
			верно неравенство $|x_m^{(j)} - x_{m + p}^{(j)}| < \eps$.
			Воспользовавшись критерием Коши из $\R$ получим, что
			\[
				(\forall j \in \range{n})\ \ 
				\exists \liml_{m \to \infty} x_m^{(j)} = x_0^{(j)}
			\]
			Отсюда по критерию сходимости в $\R^n$ (\ref{limCoordinates})
			уже получаем, что
			\[
				\exists \liml_{m \to \infty} \vec{x}_m = \vec{x}_0
			\]
	\end{itemize}
\end{proof}

\begin{definition}
	Метрическое пространство, в котором каждая
	фундаментальная последовательность сходится,
	называется \textit{полным метрическим пространством}.
	
	Полное линейное нормированное пространство
	называется \textbf{банаховым}, в честь Стефана Банаха.
	
	Полное евклидово пространство называется
	\textit{гильбертовым} (не конечномерное), в честь Гильберта.
\end{definition}

\begin{proposition} Пусть $F \subset X$, $(X, \rho)$ ---
	метрическое пространство. Тогда
	\[
		x_0 \in \cl F \lra (\exists \{x_k\}_{n = 1}^\infty \subset
		F)\ \liml_{k \to \infty} x_k = x_0
	\]
\end{proposition}

\begin{proof}~
	\begin{itemize}
		\item Необходимость $(\Ra)$.
			По определению точки прикосновения:
			\[
				(\forall \eps > 0)\ U_\eps(x_0) \cap F \neq \emptyset	
			\]
			Выберем поочереди $\eps := 1, \frac{1}{2}, \dots$
			и для каждого $\eps$ существует $x_k \in U_\eps(x_0) \cap F$,
			раз это пересечение не пустое. Тогда
			получившася последовательность:
			\[
				(\exists \{x_k\}_{n = 1}^\infty \subset
				F)(\forall k \in \N)\ 0 \le \rho(x_k, x_0) < \frac{1}{k}
			\]
			При этом при $k \to \infty$ правая и левая части стремятся
			к 0. Значит, по теореме о милиционерах:
			\[
				\liml_{k \to \infty} x_k = x_0
			\]
			Вот мы и получили искомую последовательность
		\item Достаточность $(\La)$

			У нас есть из условия:
			\[
				x_0 \in \cl F \lra (\exists \{x_k\}_{n = 1}^\infty \subset
				F)\ \liml_{k \to \infty} x_k = x_0
			\]
			Значит, из определения предела:
			\[
				(\forall \eps > 0)(\exists N \in \N)
				(\forall k > N)\ \rho(x_k, x_0) < \eps
			\]
			Значит, для любого открытого шара с центром в точке
			$x_0$ есть точка, которая лежит в нём (и в $F$), тогда:
			\[
				(\forall \eps > 0)\ U_\eps(x_0) \cap F \neq
				\emptyset \Ra x_0 \in \cl F	
			\]
	\end{itemize}
\end{proof}

\begin{theorem} (Критерий замкнутости множества)
	Множество $F$ в метрическом пространстве является
	\textit{замкнутым} тогда и только тогда, когда
	\[
		\left(\forall \{x_n\}_{n = 1}^\infty \subset F,
		\ \liml_{n \to \infty} x_n = x_0\right)\ x_0 \in F
	\]
\end{theorem}

\begin{proof}~
	\begin{itemize}
		\item Докажем необходимость $(\Ra)$.

			Рассмотрим произвольную последовательность,
			сходящуюся к какой-то непонятной точке $x_0$:
			\[
				\forall \{x_n\}_{n = 1}^\infty \subset F,
				\ \liml_{n \to \infty} x_n = x_0
			\]
			Из доказанного только что утверждения мы знаем:
			\[
				(\exists \{x_k\}_{n = 1}^\infty \subset
				F)\ \liml_{k \to \infty} x_k = x_0 \Ra x_0 \in \cl F
			\]
			Наша последовательность подходит под это условие, значит
			эта самая непонятная точка $x_0 \in \cl F$. Мы знаем, что
			$\cl F \subset F$ (из определения замкнутого множества).
			Это значит, что $x_0 \in F$. Выходит, что
			\[
				\left(\forall \{x_n\}_{n = 1}^\infty \subset F,
				\ \liml_{n \to \infty} x_n = x_0\right)\ x_0 \in F
			\]
		\item Докажем достаточность $(\La)$.
	
			Рассмотрим произвольную точку $x \in \cl F$. Тогда
			по вышедоказанному утверждению:
			\[
				x_0 \in \cl F \Ra (\exists \{x_k\}_{n = 1}^\infty \subset
				F)\ \liml_{k \to \infty} x_k = x_0
			\]
			При этом из условия мы знаем, что
			\[
				\left(\forall \{x_n\}_{n = 1}^\infty \subset F,
				\ \liml_{n \to \infty} x_n = x_0\right)\ x_0 \in F
			\]
			Значит, $x_0 \in F$. Так как мы выбирали произвольную
			точку из замыкания, то $\cl F \subset F$, то есть
			само множество $F$ замкнуто по определению.
	\end{itemize}
\end{proof}