\begin{theorem}
    Каждое рациональное число
    представимо в виде периодической десятичной
    дроби и наоборот.
     \[
     	r \in \Q \lra (r = \alpha_0,\alpha_1 \dots \alpha_k (\beta_1 \dots \beta_t)), \System{
     		&{\alpha_0 \in \N \cup \{0\}}
     		\\
     		&{(\forall i > 0)\ \alpha_i \in \{0, 1, \dots, 9\}}
     		\\
     		&{\forall \beta_i \in \{0, 1, \dots, 9\}}
     	}
     \]
\end{theorem}

\begin{proof}
    Пусть есть число $r \in \Q_+$ (не умаляя общности).
    Покажем, что оно представимо в виде периодической
    десятичной дроби:
    
    Пусть $[r]$ - целая часть числа $r$. Тогда понятно,
    что $[r] =: \alpha_0$.
    
    Рассмотрим $r - [r] = \frac{m}{n}, 0 \le m < n$
    (несложно доказать, что это верно и единственно)
    
    Далее возможно только 2 случая:
    \[
        \System{&{m = 0 \Ra r = \alpha_0,(0) \Ra
            \text{ периодическая десятичная дробь}} \\ &m \neq 0}
    \]
    Продолжим рассуждения для второго случая.
    Согласно свойству Архимеда,
    \[
        (\forall n \in \N)\ 10^n > n \Ra
    \]
    \[
        (\exists p \in \N)\ 10^p \cdot m \ge n
    \]
    Если взять $p = p_{\min}$, то будет также выполнено
    \[
        10^{p - 1} \cdot m < n
    \]

    Теперь будем строить десятичную дробь поциферно.
    По сути степень $p$ обозначает позицию первой следующей
    значащей цифры:
    \[
        \frac{m}{n} = \frac{10^p m}{10^p n} = \frac{k}{10^p}
        + \frac{m_1}{10^p n},\ \ k = \Big[\frac{10^p m}{n}\Big]
    \]
    То есть $k$ --- $p$-ая цифра после запятой, при этом $m_1$ либо
    равно 0, тогда мы до конца разложили $\frac{m}{n}$, так как
    все последующие цифры - нули, либо $0 < m_1 < n$. 
    
    Теперь вынесем за скобку $\frac{1}{10^p}$ и рассмотрим
    $\frac{m_1}{n}$ как такую же дробь и снова разложим по
    такому же алгоритму. Тогда мы получим следующую значащую
    цифру (степени 10 будут только увеличиваться). Так мы
    получим $m_2$, причем $m_2 = 0$ --- завершаем работу
    или $0 < m_2 < n$ - продолжаем выполнение алгоритма.

    Стоит заметить, что мы либо получим на каком-то шаге $0$,
    либо попадём в цикл (на $n$-ом шаге в худшем случае), так
    как $0 < m_i < n \Ra m_i$ может принять не более $n - 1$
    разных значений, а так как знаменатель всегда остается
    равным $n$, то при совпадении $m_i = m_j$ все
    последующие вычисления повторятся, и алгоритм зациклится.
    То есть мы получим период в таком случае.

    Теперь покажем, что если есть периодическая десятичная
    дробь $\alpha_0, \alpha_1 \dots \alpha_k (\beta_1 \dots
    \beta_t)$, то она представима в виде
    $\frac{m}{n},\ m \in \Z, n \in \N$:
    
    Обозначим $r = \alpha_0, \alpha_1 \dots \alpha_k
    (\beta_1 \dots \beta_t)$, тогда
    \begin{align*}
        &r \cdot 10^{k + t}= \alpha_0 \alpha_1
            \dots \alpha_k \beta_1 \dots \beta_t,
            (\beta_1 \dots \beta_t) \\
        &r \cdot 10^k= \alpha_0 \alpha_1 \dots
            \alpha_k,(\beta_1 \dots \beta_t) \\
        &r \cdot 10^k \cdot (10^t - 1) = \alpha_0
            \alpha_1 \dots \alpha_k \beta_1 \dots \beta_t - \alpha_0 \alpha_1 \dots \alpha_k = [r \cdot 10^{k + t}] - [r \cdot 10^k] \\
        &\Ra r = \frac{[r \cdot 10^{k + t}] -
            [r \cdot 10^k]}{10^k \cdot (10^t - 1)}
    \end{align*}
    
    Числитель целое число, а знаменатель - натуральное
    $\Ra$ периодическая десятичная дробь представима как
    $\frac{m}{n}$, что и требовалось показать.
\end{proof}

\subsection{Действительные числа}


\subsection*{Определение действительных чисел}

\begin{definition}
    Действительные числа определяются 4 способами.

    \begin{enumerate}
        \item Бесконечные десятичные дроби (Стевин)
        \item Дедекиндовы сечения (Дедекинд)
        \item Классы эквивалентных фундаментальных
            последовательностей рациональных
            чисел (Кантор)
        \item Стягивающиеся рациональные отрезки (Бахман)
    \end{enumerate}
\end{definition}

\begin{definition}
    Рациональным отрезком называется
    \[[p, q]_{\Q} = \{r \in \Q :\ p \le r \le q\}
    \ (p \le q \in \Q)\]
\end{definition}

\begin{definition}
    Система вложенных рациональных отрезков есть
    $\{[p_n, q_n]_{\Q}\}_{n = 1}^{\infty}$, такое что
    $(\forall n \in \N)\ [p_n, q_n]_{\Q} \supset
    [p_{n + 1}, q_{n + 1}]_{\Q}$
\end{definition}

\begin{definition}
    Последовательность рациональных чисел ---
    отображение 
    
    $f :\ \N \rightarrow \Q\ \ f(n)
    =: f_n \ \{f_n\}_{n = 1}^{\infty}$
\end{definition}

\begin{definition}
    Система вложенных рациональных отрезков
    $\{[p_n, q_n]_{\Q}\}_{n = 1}^{\infty}$ 
    называется стягивающейся (гнездом), если
    $(\forall \epsilon \in \Q_+)(\exists N \in \N)
    (\forall n > N)\ 0 \le q_n - p_n < \epsilon$
\end{definition}

\subsubsection*{Отношение эквивалентности на множестве
систем стягивающихся отрезков}

\begin{definition}
    Два гнезда $\{[p_n, q_n]_{\Q}\}_{n = 1}^{\infty}$ 
    и $\{[p_n', q_n']_{\Q}\}_{n = 1}^{\infty}$
    называются эквивалентными, если 
    $\{[\min (p_n, p_n'), \max (q_n, q_n')]
    _{\Q}\}_{n = 1}^{\infty}$ --- гнездо.
\end{definition}

\begin{proposition}
    Определение эквивалентности систем стягивающихся
    рациональных отрезков удовлетворяет всем свойствам
    отношения эквивалентности.
\end{proposition}

\begin{proof}~
    \begin{itemize}
        \item \textit{Рефлексивность} очевидна, так как
            $\min(p_n, p_n) = p_n$ и $\max(q_n, q_n) = q_n$,
            что обозначает данное изначально гнездо.
        \item \textit{Симметричность} тоже очевидна, так
            как минимум и максимум - инвариантны
            относительно порядка аргументов.
        \item \textit{Транзитивность:} нужно доказать, что
            \[
            \System{\{[p_n; q_n]_\Q\}_{n = 1}^\infty \sim
                \{[p'_n; q'_n]_\Q\}_{n = 1}^\infty \\
                \{[p'_n; q'_n]_\Q\}_{n = 1}^\infty \sim
                \{[p''_n; q''_n]_\Q\}_{n = 1}^\infty}
                \Ra
                \{[p_n; q_n]_\Q\}_{n = 1}^\infty \sim
                \{[p''_n; q''_n]_\Q\}_{n = 1}^\infty
            \]
            Из условия следует, что
            \begin{align*}
                \{[\min(p_n, p'_n);
                \max(q_n, q'_n)]_\Q\}_{n = 1}^\infty
                \text{ - стягивающаяся, то есть } \\
                (\forall \veps \in \Q)(\exists N_1 \in \N)(\forall n > N_1)
                \ \max(q_n, q'_n) - \min(p_n, p'_n) < \frac{\veps}{2} \\
                \{[\min(p'_n, p''_n); \max(q'_n, q''_n)]_\Q\}_{n = 1}^\infty
                \text{ - стягивающаяся, то есть } \\
                (\forall \veps \in \Q)(\exists N_2 \in \N)(\forall n > N_2)
                \ \max(q'_n, q''_n) - \min(p'_n, p''_n) < \frac{\veps}{2}
            \end{align*}
            Тогда для $\forall n > \max(N_1, N_2)$ оба неравенства
            будут выполняться. Тогда, нужно доказать следующее:
            \[
                \max(q_n, q''_n) - \min(p_n, p''_n) < \veps
            \]
            Для этого рассмотрим по отдельности 4 случая:
            \begin{enumerate}
                \item $\max(q_n, q''_n) - \min(p_n, p''_n) = q_n - p_n$ \\ 
                    Тогда $\ q_n - p_n \le \max(q_n, q'_n) - p_n
                    \le \max(q_n, q'_n) - \min(p_n, p'_n) < \veps$.

                    Следовательно $\max(q_n, q''_n) - \min(p_n, p''_n)
                    = q_n - p_n < \veps$
                \item $\max(q_n, q''_n) - \min(p_n, p''_n) = q''_n - p''_n$
                    - аналогично 1му случаю
                \item $\max(q_n, q''_n) - \min(p_n, p''_n) = q_n - p''_n$ \\
                    В таком случае: $\ q_n - p''_n = (q_n - p'_n) + (p'_n - p''_n)$
                    
                    При этом $q_n - p'_n \le \max(q_n, q'_n) - p'_n \le
                    \max(q_n, q'_n) - \min(p_n, p'_n) < \frac{\veps}{2}$.
                    
                    А также $p'_n - p''_n \le q'_n - p''_n \le \max(q'_n, q''_n) - p''_n
                    \le \max(q'_n, q''_n) - \min(p'_n, p''_n) < \frac{\veps}{2}$.
                
                    Сложим оба выражения:
                    $\ \max(q_n, q''_n) - \min(p_n, p''_n) = q_n - p''_n
                    < \frac{\veps}{2} + \frac{\veps}{2} = \veps$
                \item $\max(q_n, q''_n) - \min(p_n, p''_n) = q''_n - p_n$
                    - аналогично 3му случаю
            \end{enumerate}
            Таким образом, $\{[p_n; q_n]_\Q\}_{n = 1}^\infty
            \sim \{[p''_n; q''_n]_\Q\}_{n = 1}^\infty \Ra$
            транзитивность верна.
    \end{itemize}
\end{proof}

\begin{definition}
    \textit{Действительными числами} называются
    классы эквивалентности гнёзд. Множество действительных
    чисел обозначается $\R$.
\end{definition}

\begin{proposition}
    Множество рациональных чисел вложено в множество
    действительных $\Q \subset \R$.
\end{proposition}

\begin{proof}
    Действительно, $r \in \Q \Ra \{[r;r]_\Q\}_{n = 1}^\infty$
    - система стягивающихся рациональных отрезков.
\end{proof}

\subsubsection*{Операция сложения действительных чисел}

\begin{definition}
    \textit{Суммой двух действительных чисел}, представляемых
    гнездами $\{[p_n, q_n]_{\Q}\}_{n = 1}^{\infty}$ и
    $\{[r_n, s_n]_{\Q}\}_{n = 1}^{\infty}$ называется
    действительное число, представляемое гнездом
    \[
        \{[p_n + r_n, q_n + s_n]_{\Q}\}_{n = 1}^{\infty}
    \] 
\end{definition}

\begin{definition}
    Противоположным к действительному числу,
    представляемому гнездом $\{[p_n, q_n]_{\Q}\}_{n = 1}^{\infty}$,
    называется действительное число, представляемое гнездом
    $\{[-q_n, -p_n]_{\Q}\}_{n = 1}^{\infty}$ 
\end{definition}

\begin{theorem}
    Сумма и противоположное число определены
    корректно (то есть сложение не зависит от того,
    каких представителей классов эквивалентностей мы
    складываем). Справедливы свойства:

    \begin{enumerate}
        \item[I-а).]  $(\forall a \in \R)(\forall b \in \R)
            \ a + b = b + a$ (\textit{коммутативность})
        \item[I-б).]  $(\forall a, b, c \in \R)
            \ (a + b) + c = a +(b + c)$ (\textit{ассоциативность})
        \item[I-в).]  $(\forall a \in \R)
            \ a + 0 = 0 + a = a$ (\textit{нейтральный элемент} относительно сложения)
        \item[I-г).] $(\forall a \in \R)
            \ a + (-a) = (-a) + a = 0$ (\textit{обратный элемент} относительно сложения)
    \end{enumerate}
\end{theorem}

\begin{proof}~

    \begin{itemize}
        \item Для \textit{корректности сложения} нужно доказать:
            \[
            \System{\{[p_n; q_n]_\Q\}_{n = 1}^\infty \sim
                \{[p'_n; q'_n]_\Q\}_{n = 1}^\infty \\ 
                    \{[r_n; s_n]_\Q\}_{n = 1}^\infty \sim
                \{[r'_n; s'_n]_\Q\}_{n = 1}^\infty}
            \Ra
            \{[p_n + r_n; q_n + s_n]_\Q\}_{n = 1}^\infty \sim
            \{[p'_n + r'_n; q'_n + s'_n]_\Q\}_{n = 1}^\infty
            \]
            По условию:
            \begin{align*}
                (\forall \veps \in \Q_+)(\exists N_1 \in \N)
                (\forall n > N_1)\ \max(q_n, q'_n) - \min(p_n, p'_n)
                < \frac{\veps}{2} \\
                (\forall \veps \in \Q_+)(\exists N_2 \in \N)
                (\forall n > N_2)\ \max(s_n, s'_n) - \min(r_n, r'_n)
                < \frac{\veps}{2}
            \end{align*}
            А нам нужно проверить, что верно:
            \[
                (\forall \veps \in \Q_+)(\exists N = \max(N_1, N_2) \in \N)
                (\forall n > N)\ \max(q_n + s_n, q'_n + s'_n) -
                \min(p_n + r_n, p'_n + r'_n) < \veps
            \]
            Рассмотрим неравенства:
            \begin{multline*}
                \max(q_n + s_n, q'_n + s'_n) -
                \min(p_n + r_n, p'_n, + r'_n) \le \\
                \le (\max(q_n, q'_n) + \max(s_n, s'_n)) -
                (\min(p_n, p'_n) + \min(r_n, r'_n)) = \\
                = (\max(q_n, q'_n) - \min(p_n, p'_n)) +
                (\max(s_n, s'_n) - \min(r_n, r'_n)) <
                \frac{\veps}{2} + \frac{\veps}{2} = \veps
            \end{multline*}
            Следовательно, по определению:
            \[
                \{[p_n + r_n; q_n + s_n]_\Q\}_{n = 1}^\infty \sim \{[p'_n + r'_n; q'_n + s'_n]_\Q\}_{n = 1}^\infty
            \]
        \item \textit{Корректность противоположного числа} доказывается
            тривиально: мы имеем
            \[
                \{[p_n; q_n]_\Q\}_{n = 1}^\infty \sim
                \{[p'_n; q'_n]_\Q\}_{n = 1}^\infty
            \]
            То есть по определению:
            \[
                (\forall \veps \in \Q_+)(\exists N \in \N)
                (\forall n > N)\ \max(q_n, q'_n) - \min(p_n, p'_n)
                < \veps
            \]

            Осталось заметить, что $\max(q_n, q'_n) = - \min(-q_n, -q'_n)$
            и по аналогии $- \min(p_n, p'_n) = \max(-p_n, p'_n)$.
            Тогда предыдущее выражение принимает вид:

            \[
                (\forall \veps \in \Q_+)(\exists N \in \N)
                (\forall n > N)\ \max(-p_n, -p'_n) - \min(-q_n, q'_n)
                < \veps
            \]

            А это означает в точности следующее:
            \[
                \{[-q_n; -p_n]_\Q\}_{n = 1}^\infty \sim
                \{[-q'_n; -p'_n]_\Q\}_{n = 1}^\infty
            \]
        \item \textit{Коммутативность сложения} действительных чисел
            следует сразу из коммутативности сложения рациональных
            чисел:
            \[
                a + b \lra \{[p_n + r_n; q_n + s_n]_\Q\}_{n = 1}^\infty
                \sim
                \{[r_n + q_n; s_n + q_n]_\Q\}_{n = 1}^\infty \lra b + a
            \]
        \item \textit{Ассоциативность сложения} действительных
            чисел работает также вследствие свойств рациональных чисел
        \item Существование \textit{нейтрального элемента} доказывается
            очевидным образом и следует из свойств нуля в
            рациональных числах
        \item \textit{Обратный элемент} есть противоположное к
            данному действительное число. Доказательство
            следует сразу из определения сложения
    \end{itemize}
\end{proof}