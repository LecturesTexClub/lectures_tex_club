\begin{definition}
    Точка $x \in \mathds{R}$ называется \underline{предельной точкой} множества $E \in \mathds{R}$,\\
    если $\forall \epsilon > 0 \ (\mathring{E}_\epsilon(x) \cap E \neq \emptyset)$
\end{definition}

\begin{lemma}
    $x$ -- предельная точка множества $E \iff \exists \{x_{n}\} \subset E \ (x_n \to x)$
\end{lemma}

\begin{proof} $\Rightarrow$ \\
    Пусть $x \in E'$, где $E'$ -- множество всех предельных точек множества $E$.
    Тогда $\forall n \in \mathds{N} (\mathring{B}_{\frac{1}{n}}(x) \cap E \neq \emptyset)$.
    Выберем $x_n \in \mathring{B}_{\frac{1}{n}}(x) \cap E$.
    Имеем $\forall n \ (0 < |x_n - x| < \frac{1}{n}) \Rightarrow x_n \to x, x_n \neq x$
\end{proof}

\begin{proof} $\Leftarrow$ \\
    Пусть $x_{n} \to x$, $x_{n} \neq x$ Зафиксируем $\epsilon > 0$. Тогда $\exists N \ \forall n \geq N (|x_{n} - x| < \epsilon)$. Следовательно, $\mathring{B}_{\epsilon}(x) \cap E \neq \emptyset \ (\mathring{B}_{\epsilon}(x) \cap E \ni x_{N})$
\end{proof}

\begin{theorem} Критерий замкнутости.\\
    Пусть $E \subset \mathds{R}$, тогда следующие утверждения эквивалентны:
    \begin{enumerate}
        \item $E$ замкнуто
        \item $E$ содержит все свои граничные точки
        \item $E$ содержит все свои предельные точки
    \end{enumerate}
\end{theorem}

\begin{proof}
    \begin{enumerate}
        \item $1 \Rightarrow 2$\\
        $x \in \mathds{R} \setminus E$ (открытое) $\Rightarrow \exists B_{\epsilon}(x) \subset \mathds{R} \setminus E \Rightarrow x$ -- внешняя точка $E \Rightarrow x \neq \delta E \Rightarrow E \supset \delta E$.
        \item $2 \Rightarrow 3$\\
        Любая предельная точка -- (внутренняя/граничная). $intE \subset E, \delta E \subset E \Rightarrow E' \subset E$.
        \item $3 \Rightarrow 1$\\
        $x \in \mathds{R} \setminus E \Rightarrow x \notin E' \Rightarrow \exists \mathring{B}_{\epsilon}(x) \cap E = \emptyset \Rightarrow B_{\epsilon}(x) \subset \mathds{R} \setminus E \Rightarrow \mathds{R} \setminus E$ -- открыто $\Rightarrow E$ -- замкнуто.
    \end{enumerate}
\end{proof}

\begin{corollary}
    $E$ -- замкнуто $\iff \forall {x_{n}} \subset E (x_{n} \to x \Rightarrow x \in E)$
\end{corollary}

\begin{proof} $\Rightarrow$\\
    Пусть $\{x_{n}\} \subset E$, а $x \notin E$.\\
    $x_{n} \to x \Rightarrow x \in E' \Rightarrow E$ -- не замкнуто по теореме 1 пункту 3, так как $x \notin E$.
\end{proof}

\begin{proof} $\Leftarrow$\\
    Пусть задано условие на последовательности. Тогда $E \supset E'$ по лемме 2, следовательно $E$ -- замкнуто по теореме 1 пункту 3.
\end{proof}

\begin{example}
    Пусть $L$ -- множество частичных пределов числовой последовательности $\{a_{n}\}$.
    Покажем, что $L$ -- замкнуто.\\
    Пусть $x_{n} \to n$ и $x_{n} \in L$. Так как $x_{k}$ -- частичный предел $\{a_{n}\}$, то существует строго возрастащая последовательность номеров $\{n_{k}\} \subset \mathds{N}$, такая что $|x_{k} - a_{n_{k}}| < \frac{1}{k}$. Тогда $|a_{n_{k}} - x| \leq |a_{n_k} - x_k| + |x_k - x| < \frac{1}{k} + |x_k - x| \forall k \Rightarrow a_{n_k} \to x$, то есть $x \in L$.
\end{example}

\begin{definition}
    $\bar{E} = E \cup \delta E$ -- \underline{замыкание множества} $E$.
\end{definition}

\begin{lemma}
    Множество $\bar{E}$ -- замкнуто. Более чтого $\bar{E} = E \cup E'$.
\end{lemma}

\begin{proof}
    Пусть $x \in \mathds{R} \setminus \bar{E} \Rightarrow x \in extE \Rightarrow \exists B_{\epsilon}(x) \subset \mathds{R} \setminus E$.
    Кроме того, $B_{\epsilon}(x) \subset \mathds{R} \setminus \bar{E}$, иначе $B_{\epsilon}(x) \cap \delta E \neq \emptyset$, но тогда $B_{\epsilon}(x) \cap E \neq \emptyset$.
    Следовательно $\mathds{R} \setminus \bar{E}$ - открыто.\\
    2 утверждение вытекает из 2 наблюдений:
    \begin{enumerate}
        \item любая предельная точка либо внутренняя, либо граничная ($E \cup E' \subset E \cup \delta E$)
        \item Любая граничная точка, не принадлежащая множеству $E$ является предельной ($E \cup \delta E \subset E \cup E'$)
    \end{enumerate}
\end{proof}

\begin{problem}
    $a \in \bar{E} \iff \exists\{x_{n}\} \subset E \ (x_{n} \to a)$
\end{problem}

\begin{definition}
    Семейство $\{G_{\lambda}\}_{\lambda \in \Lambda}$ называется \underline{покрытием множества} $E$, если $E \subset \cup_{\lambda \in \Lambda} G_{\lambda}$. Если все множества $G_{\lambda}$ открыты, то покрытие называется \underline{открытым}.
\end{definition}

\begin{example}
    $\{(\frac{1}{n}, 1)\}$ -- открытое покрытие $(0, 1)$, так как $\cup_{n = 1}^{\infty}(\frac{1}{n}, 1) = (0, 1)$.
\end{example}

\begin{theorem} Гейне-Борель\\
    Если $\{G_{\lambda}\}_{\lambda \in \Lambda}$ -- открытое покрытие $[a, b]$, то $\exists\lambda_1, \lambda_2, ..., \lambda_n \in \Lambda \  ([a, b] \subset G_{\lambda_1} \cup G_{\lambda_2} \cup ... \cup G_{\lambda_n})$
\end{theorem}

\begin{proof}
    Предположим, $[a, b]$ не покрывается никаким конечным набором $G_{\lambda}$.
    Разделим $[a, b]$ пополам и обозначим $[a_1, b_1]$ ту половину, которая не покрывается конечным набором $G_{\lambda}$.
    Разделим пополам $[a_1, b_1]$ и т.д.
    По индукции будет построена $\{[a_{n}, b_{n}]\}$ -- стягивающаяся ($b_n-a_n = \frac{b-a}{2^{n}} \to 0$),
    каждый из её отрезков не покрывается конечным набором $G_{\lambda}$. По теореме Кантора $\exists c \in \cap_{n = 1}^{\infty}[a_n, b_n]$.
    $c \in [a, b] \subset \cup_{\lambda \in \Lambda} G_{\lambda} \Rightarrow \exists \lambda_0 \in \Lambda \ (c \in G_{\lambda_0})$. $G_{\lambda_0}$ -- открыто $\Rightarrow \exists B_{\epsilon}(c) \subset G_{\lambda_0}$.\\
    $a_{n} \to c, b_n \to c \Rightarrow \exists k : c - a_{k} < \epsilon, b_{k} - c < \epsilon \Rightarrow [a_k, b_k] \subset B_{\epsilon}(c) \subset G_{\lambda_0}$!!! (с выбором $[a_k, b_k]$).
\end{proof}

\begin{corollary}
    Если $F$ -- замкнутое ограниченное множество и $\{G_{\lambda}\}_{\lambda \in \Lambda}$ -- открытое покрытие $F$, то $\exists \lambda_1, \lambda_2, ..., \lambda_n \in \Lambda (F \subset G_{\lambda_1} \cup G_{\lambda_2} \cup ... \cup G_{\lambda_n})$ (найдется такое конечное множество $\lambda$).
\end{corollary}

\begin{proof}
    Пусть $F$ -- замкнутое ограниченное множество, $\{G_{\lambda}\}_{\lambda \in \Lambda}$ -- открытое покрытие $F$.\\
    Так как $F$ -- ограниченное, то $\exists [a, b] \supset F$.\\
    $\{G_{\lambda}\}_{\lambda \in \Lambda} \cup \{\mathds{R} \setminus F\}$ -- открытое покрытие $[a, b]$, так как $\cup_{\lambda \in \Lambda}G_{\lambda} \cup (\mathds{R} \setminus F) = \mathds{R}$.\\
    По теореме 2 $\exists \lambda_1, \lambda_2, ..., \lambda_n \in \Lambda \ (F \subset [a, b] \subset G_{\lambda_1} \cup G_{\lambda_2} \cup ... \cup G_{\lambda_n} \cup (\mathds{R} \setminus F))$.
    Следовательно, $F \subset G_{\lambda_1} \cup G_{\lambda_2} \cup ... \cup G_{\lambda_n}$.\\
\end{proof}

\begin{definition}
    Пусть $\epsilon > 0$.\\
    $B_{\epsilon}(+\infty) = (\frac{1}{\epsilon}, +\infty) \cup \{\infty\}$ -- \underline{$\epsilon$ -- окрестность $+\infty$}\\
    $\mathring{B}_{\epsilon}(+\infty) = (\frac{1}{\epsilon}, +\infty)$ -- \underline{проколотая $\epsilon$ -- окрестность $+\infty$}\\
    $B_{\epsilon}(-\infty) = \{-\infty\} \cup (-\infty, -\frac{1}{\epsilon})$ -- \underline{$\epsilon$ -- окрестность $-\infty$}\\
    $\mathring{B}_{\epsilon}(-\infty) = (-\infty, -\frac{1}{\epsilon})$ -- \underline{проколотая $\epsilon$ - окрестность $-\infty$}\\
\end{definition}

Поскольку все понятия этого параграфа вводились через окрестности, то определения без изменения переносятся на множество $\bar{\mathds{R}}$.
В частности, $+\infty(-\infty)$ -- предельная точка $E \subset \bar{\mathds{R}} \iff E \setminus \{\pm\infty\}$ -- неограничено сверху(снизу) в $\mathds{R}$.

В терминах окрестности можно дать общее определение предела числовой последовательности.

\begin{definition}
    Точка $a \in \bar{\mathds{R}}$ называется \underline{пределом последовательности} $\{a_{n}\}$, если $\forall \epsilon > 0 \ \exists N \in \mathds{N} \ \forall n \geq N \ (a_{n} \in B_{\epsilon}(a))$
\end{definition}
