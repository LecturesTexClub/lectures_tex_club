\subsection{Принцип вложенных отрезков}
    \begin{definition}
        Последовательность отрезков $\{[a_{n}, b_{n}]\}$ называется \underline{вложенной}, если $[a_{n}, b_{n}] \supset [a_{n+1}, b_{n+1}] \forall n \in \mathds{N}$.
        Если к тому же $\{b_{n}-a_{n}\} \to 0$, то $\{[a_{n}, b_{n}]\}$ называется \underline{стягивающейся}.
    \end{definition}

    \begin{theorem}
        Теорема Кантора\\
        Всякая последовательность вложенных отрезков имеет общую точку. Если последовательность стягивающейся, то такая точка единственная.
    \end{theorem}

    \begin{proof}
        Пусть $\{[a_{n}, b_{n}]\}$ - опследовательность вложенных отрезков.\\
        Поскольку $a_1 \leq a_{n} \leq a_{n+1} \leq b_{n+1} \leq b_{n} \leq b_1 \ \forall n$, следовательно \\
        $\{a_{n}\}$ - нестрого возрастает и ограничена сверху числом $b_1$,\\
        $\{b_{n}\}$ - нестрого убывает и ограничена снизу числом $a_1$.\\
        По теореме 7 обе последовательности сходятся $a_{n} \to \alpha$ и $b_{n} \to \beta$.

        Переходя в неравенстве $a_{n} \leq b_{n} \forall n$ к пределу при $n \to \infty$, получим $\alpha \leq \beta$. Ввиду монотонности $a_{n} \leq \alpha \leq \beta \leq b_{n} \forall n$, следовательно $\cap_{n = 1}^{\infty}[a_{n}, b_{n}] \supset [\alpha, \beta]$, и значит $\cap_{n = 1}^{\infty}[a_{n}, b_{n}] \neq \emptyset$\\

        Пусть $\{[a_{n}, b_{n}]\}$ - стягивающаяся, и $x, y \in \cap_{n = 1}^{\infty} [a_{n}, b_{n}]$. Так как $x, y \in [a_{n}, b_{n}] \forall n \Rightarrow |x-y| \leq b_{n} - a_{n} \forall n$. Переходя к пределу, получим $x = y$, то есть $\cap_{n = 1}^{\infty}[a_{n}, b_{n}] = \{x\}$, где $x = \alpha = \beta$.
    \end{proof}

    \begin{problem}
        Будет ли последовательность вложенной, если все отрезки заменить на интервалы?
    \end{problem}

\subsection{Подпоследовательности и частичные пределы}
    \begin{definition}
        Пусть $\{a_{n}\}$ - последовательность, $\{n_{k}\}$ - возрастающая последовательность натуральных чисел.\\
        Последовательность $\{b_{k}\}$, где $b_{k} = a_{n_{k}} \forall k$, называется \underline{подпоследовательностью} $\{a_{n}\}$ и обозначается $\{a_{n_{k}}\}$.
    \end{definition}

    \begin{note}
        \begin{enumerate}
            \item Подпоследовательность $\{a_{n_{k}}\}$ -- это композиция функций $\sigma: \mathds{N} \to \mathds{N}$, $\sigma(k) = n_{k}$ и самой последовательности $a: \mathds{N} \to \mathds{R}$, $a(n) = a_{n}$.
            \begin{example}
                $\{\frac{1}{2^{k}}\}$ - подпоследовательность $\{\frac{1}{2n}\}$, где $n_{k} = 2^{k-1}$.
            \end{example}
            \item Если $\{a_{n_{k}}\}$ - подпоследовательность, то $n_{k} \geq k$ для всех $k$.\\
                ММИ по $n$:\\
                \begin{enumerate}
                    \item $n = 1$: $n_1 = 1 \geq 1$ - верно.\\
                    \item 
                    $\begin{rcases}
                        n_{k} \geq k\\
                        n_{k+1} > n_{k}\\
                    \end{rcases}
                    \Rightarrow n_{k+1} > k \Rightarrow n_{k+1} \geq k+1$
                \end{enumerate}
        \end{enumerate}
    \end{note}

    \begin{lemma}
        Если последовательность имеет предел в $\overline{\mathds{R}}$, то любая её подпоследовательность имеет тот же предел.
    \end{lemma}

    \begin{proof}
        Пусть $\{a_{n_{k}}\}$ - подпоследовательность последовательности $\{a_{n}\}$.\\
        Пусть $a = \lim_{n \to \infty} \{a_{n}\}$, тогда\\
        $\exists N \in \mathds{N} \ \forall n \geq N (|a_{n}-a| < \epsilon) \xRightarrow{n_{k} \geq k}$\\
        $\exists N \in \mathds{N} \ \forall k \geq N (|a_{n_{k}}-a| < \epsilon)$\\
        Это означает, что $a = \lim_{k \to \infty} \{a_{n_{k}}\}$.
    \end{proof}

    \begin{theorem}
        Больцано - Вейерштрасса\\
        Всякая ограниченная последовательность имеет сходящуюся подпоследовательность.
    \end{theorem}

    \begin{proof}
        Пусть $\{a_{n}\}$ - ограничена, тогда $a_{n} \in [c, d] \ \forall n$.\\
        Положим $[c_1, d_1] = [c, d]$.\\
        Положим $y = \frac{c_{k}+d_{k}}{2}$, тогда:\\
        $[c_{k+1}, d_{k+1}] = \begin{cases}
            [c_{k}, y]$, если $\{m: a_{m} \in [c_{k}, y]\}$ - бесконечно$\\
            [y, d_{k}]$ - иначе$
        \end{cases}$

        По индукции будет построена последовательность вложенных отрезков $[c_{k}, d_{k}]$, каждый из которых содержит значения бесконечного множества членов $a_{n}$.\\
        Причем $d_{k} - c_{k} = \frac{c - d}{2^{k-1}} \to 0$. По теореме Кантера (о вложенных отрезках) существует общая точка $a = \lim_{k \to \infty} c_{k} = lim_{k \to \infty} d_{k}$.\\
        Построим строго возрастающую последовательность номеров $\{n_{k}\}$.\\
        Положим $n_1 = 1$, если номер $n_{k}$ найден, то выберем номер $n_{k+1} > n_{k}$ так, что\\
        $a_{n_{k+1}} \in [c_{k+1}, d_{k+1}]$.

        Так как по построению $c_{k} \leq a_{n_{k}} \leq d_{k} \ \forall k$, то по теореме о зажатой последовательности $\{a_{n_{k}}\} \to a$.
    \end{proof}

    \begin{definition}
        Точка $a \in \overline{\mathds{R}}$ называется \underline{частичным пределом} последовательности $\{a_{n}\}$, если $a$ - предел некоторой подпоследовательности $\{a_{n_{k}}\}$.
    \end{definition}

    Для последовательности $\{a_{n}\}$ определим $M_{k} = \sup_{n \geq k}\{a_{n}\}$, $m_{k} = inf_{n \geq k}\{a_{n}\}$. Так как при переходе к подмножеству, $\sup$ не увеличивается, а $\inf$ не уменьшается, то имеем следующую цепочку неравенств:\\
    $m_{k} \leq m_{k+1} \leq M_{k+1} \leq M_{k} \  \forall k$\\
    Cледовательно, $\{m_{k}\}$ нестрого возрастает, а $\{M_{k}\}$ нестрого убывает, и значит, эти последовательности имеют предел в $\overline{\mathds{R}}$.
    \begin{note}
        Если $\{a_{n}\}$ не ограничена сверху (снизу), то $M_{k} = +\infty \ (m_{k} = -\infty) \ \forall k$. Будем считать, что $\lim_{k \to \infty} M_{k} = +\infty \ (\lim_{k \to \infty} m_{k} = -\infty)$.
    \end{note}

    \begin{definition} \ \\
        $\overline{\lim_{n \to \infty}} a_{n} = \lim_{k \to \infty} \sup_{n \geq k}\{a_{n}\}$ называется \underline{верхним пределом} $\{a_{n}\}$\\
        $\underline{\lim_{n \to \infty}} a_{n} = \lim_{k \to \infty} \inf_{n \geq k}\{a_{n}\}$ называется \underline{нижним пределом} $\{a_{n}\}$\\
    \end{definition}

    \begin{theorem}
        Верхний (нижний) предел - это наибольший (наименьший) из частичных пределов последовательности в $\overline{\mathds{R}}$.
    \end{theorem}

    \begin{proof}
        $M = \overline{\lim_{n \to \infty}} a_{n}$, $m = \underline{\lim_{n \to \infty}} a_{n}$\\
        Нужно показать, что $m, M$ - частичные пределы и все частичные пределы лежат между $[m, M]$.\\
        Докажем, что $M$ - это частичный предел $\{a_{n}\}$:
        \begin{enumerate}
            \item 
            Пусть $M \in \mathds{R}$. Так как $M-1 < M_1 = \sup\{a_{n}\}_{n = 1}^{+\infty}$, то \\
            существует $n_1$ такой, что $M-1 < a_{n_1} \leq M_1$. Так как $M - \frac{1}{2} < M_{n_1 + 1} = \sup_{n \geq n_1+1}{a_{n}}$, то существует номер $n_2 > n_1$ такой, что $M - \frac{1}{2} < a_{n_2} \leq M_{n_2}$ и т.д.\\
            По индукции будет построена подпоследовательность $\{a_{n_{k}}\}$, такая что\\
            $M-\frac{1}{k} < a_{n_{k}} \leq M_{n_{k}} \ \forall k$.\\
            Поскольку $\lim_{k \to \infty} (M - \frac{1}{k}) = \lim_{k \to \infty} (M_{n_{k}}) = M$, то по теореме о зажатой последовательности $\{a_{n_{k}}\} \to M$
            \item 
            Пусть $M = +\infty \Rightarrow M_{k} = +\infty \ \forall k$\\
            Так как $\{a_{n}\}$ не ограничена сверху, то существует номер $n_1$, такой что $1 < a_{n_1}$.\\
            Так как $\{a_{n}\}$ не ограничена сверху, то существует $n_2$, такой что $2 < a_{n_2}$.\\
            По индукции будет построена $\{a_{n}\}$, такая что $k < a_{n_{k}}$. По пункту 1 теоремы 6, так как последовательность $\{k\}_{k = 1}^{\infty} \to +\infty$, то $a_{n_{k}} \to +\infty$.
            \item 
            Пусть $M = -\infty$. Так как $a_{k} \leq M_{k} \ \forall k$\\
            $M_{k} \to -\infty$, то по пункту 2 теоремы 6 $a_{k} \to -\infty$.
        \end{enumerate}
        В любом из случаев $M$ - частичный предел $\{a_{n}\}$.\\
        Доказательство для $m$ аналогично.\\
        Пусть $a$ -- частичный предел $\{a_{n}\}$, $a_{n_{k}} \rightarrow a$. Т.к. $n_{k} \geq k$, то 
            \[m_{k} \leq a_{n_{k}} \leq M_{k} \  \forall k\]
            Перейдем к пределу при $k \rightarrow \infty$. Получим $m \leq a \leq M$.
    \end{proof}
    