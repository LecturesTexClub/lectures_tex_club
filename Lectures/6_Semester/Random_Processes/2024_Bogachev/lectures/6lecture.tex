\begin{proposition}
	Пусть $\xi, \eta \in L_1(P)$ --- случайные величины. Тогда, условное среднее обладает следующими свойствами:
	\begin{enumerate}
		\item Если $\xi \ge 0$, то $\E^\cA\xi \ge 0$
		
		\item Если $\xi \le \eta$, то $\E^\cA\xi \le \E^\cA\eta$
		
		\item Если $\eta$ --- $\cA$-измеримая ограниченная функция, то
		\[
			\E^\cA(\eta\xi) = \eta\E^\cA\xi
		\]
		
		\item Если $\xi \indep \cA$, то $\E^\cA\xi = \E\xi$
	\end{enumerate}
\end{proposition}

\begin{proof}~
	\textcolor{red}{Смотреть конспект по теории вероятностей за 2023 год}
\end{proof}

\begin{proposition}
	Пусть $\xi$ --- случайная величина. Тогда $\xi$ независима с $\cA$ тогда и только тогда, когда
	\[
		\forall f \text{ --- ограниченная борелевская}\ \ \E^\cA[f(\xi)] = \E[f(\xi)]
	\]
\end{proposition}

\begin{proof}~
	\begin{itemize}
		\item[$\Ra$] Тривиально, ведь $f(\xi)$ остаётся $\F_\xi$-измеримой величиной, а значит независимой с $\cA$.
		
		\item[$\La$] Раз это выполнено для любых ограниченных борелевских функций, то выполнено и для $f(\xi) = \chi_{\xi \in B}$, $B \in \B(\R)$. Отсюда
		\[
			\forall A \in \cA\ \ \E(\chi_{\xi \in B}\chi_A) = \E((\E\chi_{\xi \in B})\chi_A) = (\E\chi_A) \cdot \E\chi_{\xi \in B}
		\]
		Слева $\E\chi_{\xi \in B}\chi_A = \E\chi_{\xi \in B \wedge A} = P(\xi \in B \wedge A)$, справа просто $P(A)P(\xi \in B)$.
	\end{itemize}
\end{proof}

\section{Фильтрации и мартингалы}

\begin{note}
	В этом параграфе $T \subseteq \R$ и задано вероятностное пространство $(\Omega, \F, P)$.
\end{note}

\begin{definition}
	Семейство $\sigma$-алгебр $\F_t \subseteq \F$ называется \textit{фильтрацией}, если выполнено условие
	\[
		\forall s, t \in T\ \ (s < t \Ra \F_s \subseteq \F_t)
	\]
\end{definition}

\begin{definition}
	Случайный процесс $\xi$ \textit{согласован с фильтрацией $(\F_t)_{t \in T}$}, если
	\[
		\forall t \in T\ \ \xi_t \text{ --- $\F_t$-измеримая случайная величина}
	\]
\end{definition}

\begin{note}
	Стоит отметить, что всякий процесс $\xi$ порождает фильтрацию $(\F_t)_{t \in T}$ следующего вида:
	\[
		\F_t = \sigma(\xi_s \colon s \le t)
	\]
	То есть $\F_t$ есть $\sigma$-алгебра, порождённая случайными величинами процесса до момента $t$. Очевидным образом, $\xi$ согласован с порождённой фильтрацией.
\end{note}

\begin{definition}
	Случайный процесс $\xi$ называется \textit{мартингалом относительно фильтрации $(\F_t)_{t \in T}$}, если
	\begin{enumerate}
		\item $\xi$ согласован с $(\F_t)_{t \in T}$
		
		\item $\forall s, t \in T, s \le t\ \ \E(\xi_t | \F_s) = \xi_s$
	\end{enumerate}
\end{definition}

\begin{note}
	Если поменять равенство во втором условии на неравенство $\le$, то получится определение \textit{супермартингала}. Если поставить $\ge$ --- \textit{субмартингала}.
\end{note}

\begin{definition}
	Случайный процесс называется \textit{мартингалом}, если он является мартингалом относительно порождённой собой фильтрации.
\end{definition}

\begin{proposition}
	Если случайный процесс $\xi$ --- это мартингал, то его среднее постоянно.
\end{proposition}

\begin{proof}
	Рассмотрим моменты времени $s < t$. Достаточно доказать, что $\E\xi_s = \E\xi_t$. Воспользуемся определением мартингала:
	\[
		\E\xi_s = \E(\E(\xi_t | \F_s)) = \E\xi_t
	\]
\end{proof}

\begin{example}
	Пусть $\xi_i \in L_1(P)$ --- независимые случайные величины, причём $\E\xi_i = 0$. Обозначим $S_n = \xi_1 \plusdots \xi_n$, $\F_n = \sigma(\xi_1, \ldots, \xi_n)$. Тогда $\{S_n\}_{n \in \N}$ --- мартингал относительно фильтрации $(\F_n)_{n \in \N}$. Чтобы доказать это, достаточно проверить второе свойство из его определения. Пусть $n < m$. Имеем:
	\[
		\E(S_m | \F_n) = \E(S_n + \xi_{n + 1} \plusdots \xi_m | \F_n) = S_n + \sum_{i = n + 1}^m \E(\xi_i | \F_n)
	\]
	Так как $\xi_i$ независимы, то при $i > n$ случайная величина $\xi_i$ будет независима с $\F_n$. Стало быть, $\E(\xi_i | \F_n) = \E\xi_i = 0$ и, следовательно
	\[
		\E(S_m | \F_n) = \E(S_n | \F_n) = S_n
	\]
\end{example}

\begin{example}
	Пусть $\xi \in L_1(P)$. $(\F_t)_{t \in T}$ --- фильтрация. Обозначим $\xi_t = \E(\xi | \F_t)$, тогда $(\xi_t)_{t \in T}$ --- мартингал.
\end{example}

\begin{proposition}
	Пусть $\xi$ --- мартингал, причём $\xi_t \in L_2(P)$. Тогда, это процесс с ортогональными приращениями, то есть
	\[
		\forall t_1 < t_2 < t_3 < t_4, t_i \in T\ \ \cov(\xi_{t_2} - \xi_{t_1}, \xi_{t_4} - \xi_{t_3}) = \E(\xi_{t_2} - \xi_{t_1})(\xi_{t_4} - \xi_{t_3}) = 0
	\]
\end{proposition}

\begin{proof}
	Распишем ковариацию целиком:
	\[
		\E(\xi_{t_2} - \xi_{t_1})(\xi_{t_4} - \xi_{t_3}) = \E\xi_{t_2}\xi_{t_4} + \E\xi_{t_1}\xi_{t_3} - \E\xi_{t_1}\xi_{t_4} - \E\xi_{t_2}\xi_{t_3}
	\]
	В силу мартигнальности:
	\[
		\E\xi_{t_1}\xi_{t_3} = \E(\xi_{t_1}\E(\xi_{t_3} | \F_{t_1})) = \E\xi_{t_1}^2 < \infty
	\]
	Аналогично поступаем с остальными слагаемыми. Тогда:
	\[
		\E\xi_{t_1}\xi_{t_3} = \E\xi_{t_2}^2 + \E\xi_{t_1}^2 - \E\xi_{t_1}^2 - \E\xi_{t_2}^2 = 0
	\]
\end{proof}

\begin{theorem}
	Пусть $\xi$ --- случайный процесс с независимыми приращениями, $\xi_t \in L_2(P)$, $\xi_0 = const$, $M_\xi = const$. Тогда $\xi$ --- мартингал.
\end{theorem}

\begin{proof}
	Рассмотрим $s, t \in T$, $s < t$. Снова нужно проверить только одно свойство мартингала. Так как приращения независимы, логично переписать $\xi_t$ через них:
	\[
		\E(\xi_t | \F_s) = \E(\xi_t - \xi_s + \xi_s | \F_s) = \E(\xi_t - \xi_s | \F_s) + \E(\xi_s | \F_s)
	\]
	Понятно, что $\E(\xi_s | \F_s) = \xi_s$ --- то, что нам нужно в левой части. Стало быть, должно выполняться равенство $\E(\xi_t - \xi_s | \F_s) = 0$. Это действительно так, ведь приращение $\xi_t - \xi_s$ не зависит от всех $\xi_u$, $u < s$, ибо они тоже могут быть записаны как борелевская функция от приращения: $\xi_u = (\xi_u - \xi_0) + \xi_0$.
\end{proof}

\begin{corollary}
	Винеровский процесс является мартингалом относительно своей фильтрации.
\end{corollary}

\begin{proposition}
	Пусть $T = \N$, $(\F_n)_{n \in \N}$ --- фильтрация, $\xi$ --- мартингал относительно этой фильтрации, причём $\E\xi_n^2 \le C$. Тогда, существует $\psi \in L_2(P)$, что $\xi_n = \E(\psi | \F_n)$.
\end{proposition}

\begin{proof}
	Из условия $\E\xi_n^2 \le C$ следует $\xi_n \in L_2(P)$. Стало быть, у $\xi$ приращения ортогональны. Построим $\psi$ как предел $\psi_n$ таких, что
	\[
		\forall m \le n\ \ \xi_m = \E(\psi_n | \F_m)
	\]
	Для начала, пусть $\xi_1 = 0$. Тогда, $\psi_n$ строится так:
	\begin{itemize}
		\item $n = 1$: $\psi_1 = \xi_1$
		
		\item $n > 1$: $\psi_n = \psi_{n - 1} + (\xi_n - \xi_{n - 1}) = \xi_n$. С одной стороны, $\psi_n = \xi_n$, поэтому требуемое равенство для $\xi_n$ выполнено. С другой стороны, любой элемент $L_2(\F_m, P)$, $m < n$ представим как борелевская функция от набора $\xi_1, \ldots, \xi_m$. Так как $\xi_1 = 0$, то $\xi_n - \xi_{n - 1}$ некоррелирует со всем этим набором. \textcolor{red}{Допридумать}
		
	\end{itemize}
\end{proof}

\begin{theorem} (Дуба, о сходимости мартингалов, без доказательства)
	Пусть $T = \N$, $\xi$ --- мартингал относительно фильтрации $(\F_n)_{n \in \N}$. Тогда
	\[
		\xi_n \text{ сходятся в $L_1(P)$} \Lra \exists \psi \in L_1(P) \colon \xi_n = \E(\psi | \F_n)
	\]
	При этом, если сходимость имеется, то $\xi_n \to\aal{P} \psi$.
\end{theorem}

\begin{note}
	Сходимость в $L_1(P)$ эквивалентна равномерной сходимости интегралов.
\end{note}

\begin{proposition}
	Пусть $\xi$ --- случайный процесс с независимыми приращениями, $\xi_0 = 0$, $\xi_t \in L_2(P)$ и $\E \xi_t = 0$. Тогда $(\xi_t^2 - \E\xi_t^2)_{t \in T}$ --- мартингал относительно фильтрации, порождённой $\xi$.
\end{proposition}

\begin{proof}
	Согласованность очевидна, снова нужно проверить только свойство сужения:
	\begin{multline*}
		\E(\xi_t^2 - \E\xi_t^2 | \F_s) = \E((\xi_t - \xi_s + \xi_s)^2 - \E(\xi_t - \xi_s + \xi_s)^2 | \F_s) =
		\\
		\underbrace{\E((\xi_t - \xi_s)^2 - \E(\xi_t - \xi_s)^2 | \F_s)}_{0} + \underbrace{\E(\xi_s^2 - \E\xi_s^2 | \F_s)}_{\xi_s^2 - \E\xi_s^2} + 2\E(\xi_s(\xi_t - \xi_s) - \E\xi_s(\xi_t - \xi_s) | \F_s)
	\end{multline*}
	Так как среднее слагаемое даёт ровно требуемое, последнее должно оказаться равным нулю. Покажем это явно:
	\[
		\E(\xi_s(\xi_t - \xi_s) - \E\xi_s(\xi_t - \xi_s) | \F_s) = \E(\xi_s(\xi_t - \xi_s) | \F_s) - \E\xi_s(\xi_t - \xi_s)
	\]
	Первое слагаемое можно переписать дальше: 
	\[
		\E(\xi_s(\xi_t - \xi_s) | \F_s) = \xi_s\E(\xi_t - \xi_s | \F_s) = \xi_s \E(\xi_t - \xi_s) = 0
	\]
	Аналогично и второе слагаемое:
	\[
		\E\xi_s(\xi_t - \xi_s) = \E(\E(\xi_s(\xi_t - \xi_s) | \F_s)) = \E(\xi_s\E(\xi_t - \xi_s | \F_s)) = 0
	\]
\end{proof}

\begin{example}
	Пусть $W$ --- винеровский процесс. Тогда, по последнему утверждению $(W_t^2 - t)_{t \in T}$ --- мартингал.
\end{example}