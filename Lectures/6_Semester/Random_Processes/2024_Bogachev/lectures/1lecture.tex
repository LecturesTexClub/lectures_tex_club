\section{Случайный процесс}

\begin{definition}
	Пусть $T \neq \emptyset$ --- произвольное множество. \textit{Случайным процессом значениями из $E$} называется кортеж $\xi = (\xi_t)_{t \in T}$ из случайных величин на вероятностном пространстве $(\Omega, \F, P)$, действующих в $(E, \cE)$.
\end{definition}

\begin{note}
	Хоть и в большинстве приложений $T$ играет роль \textit{времени} (то есть как минимум линейно упорядоченное множество), но мы не накладываем на него никаких требований кроме непустоты.
\end{note}

\begin{note}
	Получается, что $\xi_t(\omega) = \xi(t, \omega) \colon T \times \Omega \to E$ --- функция двух переменных, измеримая по второму из них. Если мы зафиксируем $\omega \in \Omega$, то получим (в рассмотрении $T$ как времени) некоторый \textit{детерминированный сценарий} процесса.
\end{note}

\begin{definition}
	Пусть $\xi$ --- случайный процесс. Тогда при каждом $\omega \in \Omega$ кортеж значений $(\xi_t(\omega))_{t \in T}$ называется \textit{траекторией (или реализацией) (случайного процесса)}.
\end{definition}

\begin{example}
	Самым простым случайным процессом будет такой процесс, что он образован произведением двух функций:
	\[
		\xi_t(\omega) = \phi(t)\psi(\omega)
	\]
	где $\psi$, естественно, является случайной величиной. Более общий вариант --- это разложение некоторого процесса в ряд:
	\[
		\xi_t(\omega) = \sum_{n = 1}^\infty \phi_n(t)\psi_n(\omega)
	\]
\end{example}

\subsection{Распределение на пространстве функций}

\begin{reminder}
	Пусть задано вероятностное пространство $(\Omega, \F, P)$ и случайная величина $\xi \colon \Omega \to \R$ на нём. Тогда \textit{распределением случайной величины} называется следующая мера на $(\R, \B(\R))$:
	\[
	\forall B \in \B(\R)\ \ P_\xi(B) = P(\xi \in B)
	\]
\end{reminder}

\begin{reminder}
	Пусть задано вероятностное пространство $(\Omega, \F, P)$ и случайная вектор $\xi \colon \Omega \to \R^n$ на нём. Тогда \textit{распределением случайной вектора} называется следующая мера на $(\R^n, \B(\R^n))$:
	\[
	\forall B \in \B(\R^n)\ \ P_\xi(B) = P((\xi_1, \ldots, \xi_n)^T \in B) = P(\xi \in B)
	\]
\end{reminder}

\begin{reminder}
	Пусть $\xi_1, \ldots, \xi_n$ --- случайные величины. Тогда они независимы в совокупности тогда и только тогда, когда распределение вектора, составленного из них, распадается на компоненты:
	\[
	\forall B_1 \times \ldots \times B_n = B \in \B(\R^n)\ \ P_{(\xi_1, \ldots, \xi_n)^T}(B) = P_{\xi_1}(B_1) \cdot \ldots \cdot P_{\xi_n}(B_n)
	\]
\end{reminder}

\begin{note}
	Есть более общее понятие, описывающее подобные меры. Пусть $(\Omega, \F, P)$ --- вероятностное пространство, $(E, \cE)$ --- измеримое пространство, $f \colon \Omega \to E$ --- измеримое отображение относительно $\F|\cE$. Тогда \textit{образом меры $P$ при отображении $f$} называется мера $P \circ f^{-1} \colon E \to [0; 1]$:
	\[
		P \circ f^{-1}(E) = P(f^{-1}(E))
	\]
\end{note}

\begin{note}
	Мы хотим определить аналогичные понятия случайных величин и векторов, но на уровне процессов. Когда мы говорим о распределении случайной величины, то подразумеваем вероятность попадания в некоторое подмножество множества значений. Аналогично и тут, только так как каждое $\omega \in \Omega$ делает из случайного процесса функцию $T \to E$, то мы должны рассматривать подмножества пространства траекторий $E^T$.
\end{note}

\begin{note}
	Далее, для простоты, мы изучаем $E = \R$ и, соответственно, пространство $\R^T$.
\end{note}

\begin{note}
	На пространстве $\R^T$ нужно определить $\sigma$-алгебру. Это можно сделать разными способами. В частности, можно рассмотреть измеримую функцию $\R^T \to \R$ и взять $\sigma$-алгебру, порождённую ей.
\end{note}

\begin{definition} \textcolor{red}{(не по лектору)}
	Пусть даны измеримые функции $f_a \colon X \to E$, $a \in A$ из пространства $(X, \F)$ в $(E, \cE)$. Тогда \textit{$\sigma$-алгеброй $\cA$, порождённой функциями $\{f_a\}_{a \in A}$}, называется следующая сигма-алгебра:
	\[
		\cA = \sigma\ps{\bigcup_{a \in A} \F_{f_a}}
	\]
\end{definition}

\begin{definition}
	$\sigma$-алгеброй пространства $\R^T$ называется $\sigma$-алгебра, порождённая следующими функциями на $\R^T$ (они же \textit{проекторы} $\R^T \to \R$):
	\[
		\forall t \in T\ \ \phi_t(x) = x(t)
	\]
\end{definition}

\begin{anote}
	Поймём, какую $\sigma$-алгебру порождает проектор. Её можно описать так:
	\[
		\sigma(\phi_t) = \sigma\big(\big\{\{x \in \R^T \colon x(t) \in B\}, B \in \B(\R)\big\}\big)
	\]
\end{anote}

\begin{definition} \textcolor{red}{(не по лектору)}
	\textit{Элементарным цилиндром в $\R^T$} называется множество следующего вида:
	\[
		C_{t, B} = \{x \in \R^T \colon x(t) \in B\}
	\]
\end{definition}

\begin{definition}
	\textit{Цилиндром} в $\R^T$ называется множество, описываемое следующим образом:
	\[
		\forall t_1, \ldots, t_n \in T, B \in \B(\R^n)\ \ C_{t_1, \ldots, t_n, B} = \{x \in \R^T \colon (x(t_1), \ldots, x(t_n))^T \in B\}
	\]
\end{definition}

\begin{anote}
	Мы разрешаем ситуации, когда $t_i = t_j$. Если вдруг нужно, чтобы все точки были различными, то такой цилиндр можно запросто переписать (при помощи пересечений условий на $t_i = t_j$).
\end{anote}

\begin{proposition} \textcolor{red}{(не по лектору)}
	$C_{t_1, \ldots, t_n, B} \in \sigma(\R^T)$
\end{proposition}

\begin{proof}
	Пусть $t_i \in T$, $B = B_1 \times \ldots \times B_n \in \B(\R^n)$. Внимательно сравним пересечение элементарных цилиндров с цилиндром:
	\[
		\bigcap_{i = 1}^n C_{t_i, B_i} = \{x \in \R^T \colon (x(t_1), \ldots, x(t_n)) \in B\} = C_{t_1, \ldots, t_n, B}
	\]
	Однако, не каждое множество из $\B(\R^n)$ может быть разложено в прямое произведение. Тем не менее, известно, что система таких произведений порождает $\B(\R^n)$. С учетом того, что операции над цилиндрами с одинаковым набором $\{t_i\}_{i = 1}^n$ дают цилиндр с тем же набором, имеем $C_{t_1, \ldots, t_n, B} \in \sigma(\R^T)$.
\end{proof}

\begin{proposition}
	Цилиндры в $\R^T$ образуют алгебру множеств.
\end{proposition}

\begin{proof}
	Для примера покажем, как из двух цилиндров получить объединённый:
	\[
		C_{t_1, \ldots, t_n, B} = \{x \in \R^T \colon (x(t_1), \ldots, x(t_n))^T \in B\},\ C_{s_1, \ldots, s_m, B'} = \{x \in \R^T \colon (x(s_1), \ldots, x(s_m)) \in B'\}
	\]
	Приведём оба цилиндра к общему виду. Общий набор точек будет $(t_k)_{k = 1}^n \sqcup (s_l)_{l = 1}^m$. Тогда в первом цилиндре нужно поменять $B$ на $B \times \R \times \ldots \times \R$, а во втором --- на $R \times \ldots \times \R \times B'$.
\end{proof}

\begin{note}
	Можно определить понятие \textit{канонического представления} цилиндра, когда число точек, задающих его \textit{основание}, минимально. Это единственным образом определяет и число этих точек, и соответствующее им множество (с точностью до перестановки элементов).
\end{note}

\begin{proposition}
	Имеет место следующее определение $\sigma(\R^T)$:
	\[
		E \in \sigma(\R^T) \Lra \exists \{t_j\}_{j = 1}^\infty \subseteq T, B \in \B(\R^\N)\ \ E = \{x \in \R^T \colon (x(t_1), x(t_2), \ldots) \in B\}
	\]
\end{proposition}

\begin{proof} \textcolor{red}{(не по лектору)}
	Покажем вложение в 2 стороны (пока что систему множеств справа обозначим за $\Tau$):
	\begin{itemize}
		\item[$\subseteq$] Очевидным образом, все элементарные цилиндры вложены в $\Tau$. Так как $\Tau$ очевидным образом является $\sigma$-алгеброй, то $\sigma(\R^T) \subseteq \Tau$.
		
		\item[$\supseteq$] Пусть $E \in \Tau$ и представляется следующим образом:
		\[
			E = \{x \in \R^T \colon (x(t_1), x(t_2), \ldots) \in B\},\ t_i \in T,\ B \in \B(\R^\N)
		\]
		Тогда, определим $B_n \in \B(\R^n)$ как проекцию (или срезку) $B$ на первые $n$ координат. Аналогичным образом введём множества $E_n$:
		\[
			E_n = C_{t_1, \ldots, t_n, B_n} = \{x \in \R^T \colon (x(t_1), \ldots, x(t_n)) \in B_n\} \in \sigma(\R^T)
		\]
		Отсюда $E = \bigcap_{n = 1}^\infty E_n$. Так как $\sigma$-алгебра всегда является и $\delta$-алгеброй, получаем $E \in \sigma(\R^T)$, а значит $\Tau \subseteq \sigma(\R^T)$
	\end{itemize}
\end{proof}

\begin{proposition}
	Пусть $T = [0; 1]$. Тогда
	\begin{itemize}
		\item $C[0; 1] \notin \sigma(\R^T)$
		
		\item $\{x \in \R^T \colon x \ge 0\} \notin \sigma(\R^T)$
		
		\item $\{x \in \R^T \colon x \text{ возрастает}\} \notin \sigma(\R^T)$
	\end{itemize}
\end{proposition}

\begin{proof}
	Всё дело в том, что в $\sigma$-алгебру попадают только такие множества, которые можно распознать по счётному числу точек. Ни один из примеров выше в эту категорию не попадает. Для примера, разберём пункт с непрерывными траекториями. Если $C[0; 1] \in \sigma(\R^T)$, то
	\[
		\exists \{t_j\}_{j = 1}^\infty \subseteq [0; 1], B \in \B(\R^\N)\ \ C[0; 1] = \{x \in \R^T \colon (x(t_1), x(t_2), \ldots) \in B\}
	\]
	Как нам известно, в отрезке континуум точек. Возьмём $s \in [0; 1] \bs \{t_j\}_{j = 1}^\infty$. Тогда, мы можем <<испортить>> (лишить непрерывности) любую функцию, сделав скачок в этой точке. Получили противоречие.
\end{proof}

\begin{note}
	Далее нам будет удобно обозначить $\B_T = \sigma(\R^T)$
\end{note}

\begin{definition}
	Пусть $\xi$ --- случайный процесс на пространстве $(\Omega, \F, P)$. Тогда \textit{распределением процесса} $P_\xi$ называется вероятностная мера на $(\R^T, \B_T)$, заданная следующим образом:
	\[
		P_\xi(C_{t_1, \ldots, t_n, B}) = P(\{\omega \in \Omega \colon (\xi_{t_1}(\omega), \ldots, \xi_{t_n}(\omega)) \in B\})
	\]
	В общем случае, можно написать так:
	\[
		\forall E \in \B_T\ \ P_\xi(E) = P(\{\omega \in \Omega \colon \xi(\omega) \in E\})
	\]
\end{definition}

\begin{note}
	Разумный вопрос, который нужно задать к определению выше: а почему оно корректно? Говоря явно, мы не показали, что имеем право считать меру $P$ от множества в скобках. Для этого покажем, что случайный процесс $\xi$ является $\F|\B_T$-измермым отображением.
\end{note}

\begin{reminder}
	Пусть $\xi \colon \Omega \to E$ --- отображение из $(\Omega, \F)$ в $(E, \cE)$. Если $\cM$ --- система множеств такая, что $\sigma(\cM) = \cE$, то $\xi$ является $\F|\cE$-измеримым отображением тогда и только тогда, когда оно является $\F|\cM$-измеримым.
\end{reminder}

\begin{proposition}
	Пусть $\xi$-случайный процесс. Тогда $\xi$ как отображение $\Omega \to \R^T$ является $\F|\B_T$ измеримым.
\end{proposition}

\begin{proof}
	Мы знаем, что $\B_T$ порождается алгеброй цилиндров, поэтому достаточно проверить измеримость на них:
	\[
		\forall \{t_i\}_{i = 1}^n \subseteq T, B \in \R^n\ \ \xi^{-1}(C_{t_1, \ldots, t_n, B}) = \{\omega \in \Omega \colon (\xi_{t_1}(\omega), \ldots, \xi_{t_n}(\omega)) \in B\} \in \F
	\]
	По определению случайного процесса, все $\xi_t$ являются измеримыми отображениями. А раз так, то и вектор, составленный из них, измерим. Требуемое доказано.
\end{proof}

\begin{definition}
	Класс множеств $\cK$ называется \textit{компактным}, если для любой последовательности $K_n \in \cK$ с условием $\bigcap_{l = 1}^n K_l \neq \emptyset$ следует, что  $\bigcap_{l = 1}^\infty K_l \neq \emptyset$.
\end{definition}

\begin{proposition} (без доказательства)
	Пусть $\mu \ge 0$ ---- аддитивная функция на алгебре $\cA$ и существует компактный класс $\cK \subseteq \cA$ такой, что
	\[
		\forall A \in \cA\ \ \mu(A) = \sup_{K \subseteq A, K \in \cK} \mu(K)
	\]
	Тогда $\mu$ --- счётно-аддитивная мера.
\end{proposition}

\begin{exercise}
	Класс цилиндров с компактными основаниями компактен.
\end{exercise}
