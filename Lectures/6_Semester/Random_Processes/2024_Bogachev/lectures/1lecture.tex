\section{Случайный процесс}

\begin{definition}
	Пусть $T \neq \emptyset$ --- произвольное множество. \textit{Случайным процессом} называется кортеж $\xi = (\xi_t)_{t \in T}$ из случайных величин на пространстве $(\Omega, \F, P)$
\end{definition}

\begin{note}
	Хоть и в большинстве приложений $T$ играет роль \textit{времени} (то есть как минимум линейно упорядоченное множество), но мы не накладываем на него никаких требований кроме непустоты.
\end{note}

\begin{note}
	Получается, что $\xi_t(\omega) = \xi(t, \omega) \colon T \times \Omega \to E$ --- функция двух переменных ($E$ --- просто обозначение множества значений), измеримая по второму из них. Если мы зафиксируем $\omega \in \Omega$, то получим (в рассмотрении $T$ как времени) некоторый \textit{детерминированный сценарий} процесса.
\end{note}

\begin{definition}
	Пусть $\xi$ --- случайный процесс. Тогда при каждом $\omega \in \Omega$ кортеж значений $(\xi_t(\omega))_{t \in T}$ называется \textit{траекторией (или реализацией) (случайного процесса)}.
\end{definition}

\begin{example}
	Самым простым случайным процессом будет такой процесс, что он образован произведением двух функций:
	\[
		\xi_t(\omega) = \phi(t)\psi(\omega)
	\]
	где $\psi$, естественно, является случайной величиной. Более общий вариант --- это разложение некоторого процесса в ряд:
	\[
		\xi_t(\omega) = \sum_{n = 1}^\infty \phi_n(t)\psi_n(\omega)
	\]
\end{example}

\begin{reminder}
	Пусть задано вероятностное пространство $(\Omega, \F, P)$ и случайная величина $\xi \colon \Omega \to \R$ на нём. Тогда \textit{распределением случайной величины} называется следующая мера на $(\R, \B(\R))$:
	\[
		\forall B \in \B(\R)\ \ P_\xi(B) = P(\xi \in B)
	\]
\end{reminder}

\begin{reminder}
	Пусть задано вероятностное пространство $(\Omega, \F, P)$ и случайная вектор $\xi \colon \Omega \to \R^n$ на нём. Тогда \textit{распределением случайной вектора} называется следующая мера на $(\R^n, \B(\R^n))$:
	\[
	\forall B \in \B(\R^n)\ \ P_\xi(B) = P((\xi_1, \ldots, \xi_n)^T \in B) = P(\xi \in B)
	\]
\end{reminder}

\begin{reminder}
	Пусть $\xi_1, \ldots, \xi_n$ --- случайные величины. Тогда они независимы в совокупности тогда и только тогда, когда распределение вектора, составленного из них, распадается на компоненты:
	\[
		\forall B_1 \times \ldots \times B_n = B \in \B(\R^n)\ \ P_{(\xi_1, \ldots, \xi_n)^T}(B) = P_{\xi_1}(B_1) \cdot \ldots \cdot P_{\xi_n}(B_n)
	\]
\end{reminder}

\subsection{Распределение на пространстве функций}

\begin{note}
	Мы хотим определить аналогичные понятия случайных величин и векторов, но на уровне процессов. Когда мы говорим о распределении случайной величины, то подразумеваем вероятность попадания в некоторое подмножество множества значений. Аналогично и тут, только так как каждое $\omega \in \Omega$ делает из случайного процесса функцию $T \to E$, то мы должны рассматривать подмножества пространства $E^T$.
\end{note}

\begin{note}
	Далее, для простоты, мы изучаем $E = \R$ и, соответственно, пространство $\R^T$.
\end{note}

\begin{note}
	На пространстве $\R^T$ хочется определить $\sigma$-алгебру. Это можно сделать разными способами. В частности, можно рассмотреть измеримую функцию $\R^T \to \R$ и взять $\sigma$-алгебру, порождённую ей.
\end{note}

\begin{definition} \textcolor{red}{не по лектору}
	Пусть даны измеримые функции $f_n \colon X \to E$ из пространства $(X, \F)$ в $(E, \cE)$. Тогда $\sigma$-алгеброй $\cA$, порождённой функциями $f_n$, называется следующая сигма-алгебра:
	\[
		\cA = \sigma\ps{\bigcup_{n = 1}^\infty \F_{f_n}}
	\]
\end{definition}

\begin{definition}
	$\sigma$-алгеброй пространства $\R^T$ называется $\sigma$-алгебра, порождённая следующими функциями на $\R^T$ (они же \textit{проекторы}):
	\[
		\forall t \in T\ \ \phi_t(x) = x(t)
	\]
\end{definition}

\begin{definition}
	\textit{Цилиндром} в $\R^T$ называется множество, описываемое следующим образом:
	\[
		\forall t_1, \ldots, t_n \in T, B \in \B(\R^n)\ \ C_{t_1, \ldots, t_n, B} = \{x \in \R^T \colon (x(t_1), \ldots, x(t_n))^T \in B\}
	\]
\end{definition}

\begin{note}
	Несложно увидеть, что $C_{t_1, \ldots, t_n, B} \in \sigma(\R^T)$. \textcolor{red}{Сложно. Вообще всё выше выглядит как какой-то обман}
\end{note}

\begin{proposition}
	Цилиндры в $\R^T$ образуют алгебру множеств.
\end{proposition}

\begin{proof}
	Для примера покажем, как из двух цилиндров получить объединенный:
	\[
		C_{t_1, \ldots, t_n, B} = \{x \in \R^T \colon (x(t_1), \ldots, x(t_n))^T \in B\},\ C_{s_1, \ldots, s_m, B'} = \{x \in \R^T \colon (x(s_1), \ldots, x(s_m)) \in B'\}
	\]
	Добавим к $(t_k)_{k = 1}^n$ в конец недостающие $s_l$. Тогда в описании первого цилиндра добавим $B \times \R \times \ldots \times \R$ нужное число раз (для новых переменных). Со вторым цилиндром поступим аналогично, только может потребоваться поменять некоторые переменные местами и добавить измерения не <<в конец>>, а где-то <<между>> измерениями $B'$.
\end{proof}

\begin{note}
	Можно определить понятие \textit{канонического представления} цилиндра, когда число точек, задающих его \textit{основание}, минимально. Это единственным образом определяет и число этих точек, и соответствующее им множество (с точностью до перестановки элементов).
\end{note}

\begin{proposition}
	Имеет место следующее определение $\sigma(\R^T)$:
	\[
		E \in \sigma(\R^\N) \Lra \exists \{t_j\}_{j = 1}^\infty \subseteq T, B \in \B(\R^\N)\ \ E = \{x \in \R^\N \colon (x(t_1), x(t_2), \ldots) \in B\}
	\]
\end{proposition}

\begin{proof}
	
\end{proof}

\begin{proposition}
	Пусть $T = [0; 1]$. Тогда
	\begin{itemize}
		\item $C[0; 1] \notin \sigma(\R^T)$
		
		\item $\{x \in \R^T \colon x \ge 0\} \notin \sigma(\R^T)$
		
		\item $\{x \in \R^T \colon x \text{ возрастает }\} \notin \sigma(\R^T)$
	\end{itemize}
\end{proposition}

\begin{proof}
	Всё дело в том, что в $\sigma$-алгебру попадают только такие множества, которые можно распознать по счётному числу точек. Ни один из примеров выше в эт(у категорию не попадает. Для примера, разберём пункт с непрерывными траекториями. Если $C[0; 1] \in \sigma(\R^T)$, то
	\[
		\exists \{t_j\}_{j = 1}^\infty \subseteq [0; 1], B \in \B(\R^\N)\ \ C[0; 1] = \{x \in \R^\N \colon (x(t_1), x(t_2), \ldots) \in B\}
	\]
	Как нам известно, в отрезке континуум точек. Возьмём $s \in [0; 1] \bs \{t_j\}_{j = 1}^\infty$. Тогда. мы можем <<испортить>> (лишить непрерывности) любую функцию, сделав скачок в этой точке. Получили противоречие.
\end{proof}

\begin{definition}
	Пусть $\xi$ --- случайный процесс на пространстве $(\Omega, \F, P)$. Тогда \textit{распределением процесса} $P_\xi$ называется вероятностная мера на $\sigma(\R^T)$, заданная следующим образом:
	\[
		P_\xi(C_{t_1, \ldots, t_n, B}) = P(\{\omega \in \Omega \colon (\xi_{t_1}(\omega), \ldots, \xi_{t_n}(\omega)) \in B\})
	\]
	В общем случае, можно написать так (здесь уже $B \in \B(\R^\N)$):
	\[
		\forall E \in \sigma(\R^T)\ \ P_\xi(E) = P(\{\omega \in \Omega \colon (\xi_{t_1}(\omega), \xi_{t_2}(\omega), \ldots) \in B\})
	\]
\end{definition}

\begin{note}
	Разумный вопрос, который нужно задать к определению выше: а почему оно корректно? Говоря явно, мы не показали, что имеем право считать меру $P$ от множества в скобках. Для этого покажем, что множество таких подмножеств $\cA$ образует $\sigma$-алгебру:
	\[
		\cA = \Big\{\{\omega \in \Omega \colon (\xi_{t_1}(\omega), \xi_{t_2}(\omega), \ldots) \in B\} \in \F\Big\}
	\]
	Рассмотрим даже не все множества из $\cA$, а только их подкласс, когда внутри написано $\xi_{t_1}(\omega) \in (a; b)$ (должно быть тривиально, что они тоже внутри $\cA$ есть в силу свойств $\xi_{t_1}$). Несложно также заметить, что даже этот класс будет образовывать $\sigma$-алгебру: дополнение до элемента тривиально есть, а вместо счётного объединения достаточно посмотреть на счётное пересечение, у которого множество значений $B$ останется интервалом, а значит будет лежать в подклассе. Стало быть, вся исходная $\sigma$-алгебра <<схлопывается>> до этого подкласса.
\end{note}

\begin{definition}
	Класс множеств $\cK$ называется \textit{компактным}, если для любой последовательности $K_n \in \cK$ с условием $\bigcup_{l = 1}^n K_l \neq \emptyset$ следует, что  $\bigcup_{l = 1}^\infty K_l \neq \emptyset$.
\end{definition}

\begin{proposition} (без доказательства)
	Пусть $\mu \ge 0$ ---- аддитивная функция на алгебре $\cA$ и существует компактный класс $\cK \subseteq \cA$ такой, что
	\[
		\forall A \in \cA\ \ \mu(A) = \sup_{K \subseteq A, K \in \cK} \mu(K)
	\]
	Тогда $\mu$ --- счётно-аддитивная мера.
\end{proposition}

\begin{exercise}
	Класс цилиндров с компактными основаниями компактен.
\end{exercise}
