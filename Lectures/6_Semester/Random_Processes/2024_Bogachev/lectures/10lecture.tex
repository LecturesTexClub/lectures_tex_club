\section{Ветвящиеся процессы Гальтона-Ватсона}

\begin{definition}
	\textit{Процессом Гальтона-Ватсона} называется однородная цепь Маркова $Z$ с фазовым пространством $E = \N_0$, начальным распределением $\pi_0$ и переходными вероятностями следующего вида:
	\[
		p_{k, j} = P(Z_{n + 1} = j | Z_n = k) = p_j^{*k}, k > 0
	\]
	 при этом $p_{0, 0} = 1$, $p_{0, j} = 0$ и
	 \[
	 	p_j^{*k} = \sum_{j_1 \plusdots j_k = j} p_{j_1} \cdot \ldots \cdot p_{j_k}
	 \]
	 где $(p_0, p_1, \ldots)$ --- вероятностное распределение на $\N_0$.
\end{definition}

\begin{note}
	$\pi_0$, вообще говоря, может быть любым, но далее мы предполагаем детерминрованный вариант $\pi_0(1) = 1$.
\end{note}

\begin{note}
	Процесс Гальтона-Ватсона можно записать и другим образом. Пусть $\{\xi_k\}_{k = 1}^\infty$ --- независимые случайные величины с общим распределением:
	\[
		P(\xi_k = m) = p_m
	\]
	Тогда $p_j^{*k} = P(\xi_1 \plusdots \xi_k = j)$.
\end{note}

\begin{note}
	Интерпретация рассматриваемого процесса Гальтона-Ватсона такова: наблюдается процесс смены поколений частиц. Частица живёт 1 единицу времени, и в начале была ровно 1 частица. Каждая частица может породить произвольное число потомков согласно вероятностному распределению $(p_0, p_1, \ldots)$. На моменте времени 2 уже, естественно, будет не одна частица, но все они могут породить потомков по этому же распределению, и так далее.
	
	Число частиц в $n$-м поколении есть $Z_n$. Если $\xi_k^{(n)}$ --- случайная величина, отвечающая количеству потомков от $k$-й частицы в $(n - 1)$-м поколении, то
	\[
		Z_n = \sum_{k = 1}^{Z_{n - 1}} \xi_k^{(n)}, n \ge 1;\ Z_0 = 1
	\]
\end{note}

\begin{problem}
	Найти вероятность вырождения процесса Гальтона-Ватсона, то есть вероятность того, что все частицы умрут:
	\[
		q = P(\exists n \in \N \colon Z_n = 0)
	\]
	Также должно быть понятно, что $q = \lim_{n \to \infty} q_n$, где $q_n := P(Z_n = 0)$.
\end{problem}

\begin{note}
	Отсюда и до конца параграфа, если не сказано явно другого, любая случайная величина рассматривается со значениями в $\N_0$.
\end{note}

\begin{definition} \textcolor{red}{(не по лектору)}
	\textit{Производящей функцией случайной величины $\xi$} называется производящая функция, у которой коэффициенты при степени $z^k$ соответствуют $P(\xi = k)$.
\end{definition}

\begin{proposition}
	Пусть $\xi$ --- случайная величина. Тогда её производящая функция может быть записана как
	\[
		\forall z \in \Cm, |z| \le 1\ \ \phi_\xi(z) = \E z^\xi = \sum_{k \ge 0} P(\xi = k)z^k
	\]
\end{proposition}

\begin{proof}
	Тривиально
\end{proof}

\begin{proposition}
	Пусть $\xi$ --- случайная величина. Тогда производящая функция $\phi_\xi(z) = \E z^\xi$ определена в замкнутом единичном круге, задаёт аналитическую функцию на внутренности и непрерывную на замыкании, причём $\phi_\xi(1) = 1$ и $\phi'_\xi(1) = \E\xi$
\end{proposition}

\begin{note}
	Если в производящей функции положить $z = e^{it}$, то получится преобразование Фурье.
\end{note}

\begin{lemma} (рекуррентность харфункции процесса Гальтона-Ватсона)
	Имеет место формула:
	\[
		\forall z \in \Cm, |z| \le 1\ \ \phi_{Z_n}(z) = \phi_{Z_{n - 1}}(\phi_\xi(z))
	\]
\end{lemma}

\begin{proof}
	Заметим, что $|\phi_\xi(z)| \le 1$ при $|z| \le 1$. Найдём вид для $\E(z^{Z_n} | Z_{n - 1})$:
	\begin{multline*}
		\E(z^{Z_n} | Z_{n - 1} = m) = \E\ps{z^{\sum_{k = 1}^{Z_{n - 1}} \xi_k^{(n)}} | Z_{n - 1} = m} = \E\ps{z^{\sum_{k = 1}^m \xi_k^{(n)}} | Z_{n - 1} = m} = [1] =
		\\
		\E\ps{z^{\sum_{k = 1}^m \xi_k^{(n)}}} = \E\ps{\prod_{k = 1}^m z^{\xi_k^{(n)}}} = \prod_{k = 1}^m \E(z^{\xi_k^{(n)}}) = \phi_\xi(z)^m
	\end{multline*}
	Пояснение 1: $Z_{n - 1} = \sum_{k = 1}^{Z_{n - 2}} \xi_k^{(n - 1)}$ по определению, то есть $Z_{n - 1}$ зависит только от $\xi_k^{(l)}$, где $l \le n - 1$. Значит, $Z_{n - 1}$ независимо со всеми $\xi_k^{(n)}$. Итого:
	\[
		\E(z^{Z_n} | Z_{n - 1}) = \phi_\xi(z)^{Z_{n - 1}}
	\]
	Применяя формулу полной вероятности в рамках УМО, получим
	\[
		\phi_{Z_n}(z) = \E z^{Z_n} = \E(\E(z^{Z_n} | Z_{n - 1})) = \E(\phi_\xi(z)^{Z_{n - 1}}) = \phi_{Z_{n - 1}}(\phi_\xi(z))
	\]
\end{proof}

\begin{corollary}
	Для $\phi_{Z_n}(z)$ имеет место формула
	\[
		\phi_{Z_n}(z) = \underbrace{\phi_\xi(\cdots \phi_\xi(\phi_\xi}_{n}(z)) \cdots)
	\]
	Отсюда же $\phi_{Z_n}(z) = \phi_\xi(\phi_{Z_{n - 1}}(z))$
\end{corollary}

\begin{proof}
	Проведём индукцию по $n$:
	\begin{itemize}
		\item База $n = 0$: $\phi_{Z_0}(z) = \E z^1 = z$
		
		\item Переход $n > 0$:
		\[
			\phi_{Z_n}(z) = [\text{лемма}] = \phi_{Z_{n - 1}}(\phi_\xi(z)) = \underbrace{\phi_\xi(\cdots \phi_\xi}_{n - 1}(\phi_\xi(z)) \cdots)
		\]
	\end{itemize}
\end{proof}

\begin{lemma}
	Вероятность вырождения $q$ является неподвижной точкой $\phi_\xi$:
	\[
		q = \phi_\xi(q)
	\]
\end{lemma}

\begin{proof}
	В силу рекуррентных формул для производящей функции $Z_n$:
	\[
		q_n := P(Z_n = 0) = \phi_{Z_n}(0) = \phi_\xi(\phi_{Z_{n - 1}}(0)) = \phi_\xi(q_{n - 1})
	\]
	Сделаем предельный переход с обеих сторон:
	\[
		\lim_{n \to \infty} q_n = q = \lim_{n \to \infty} \phi_\xi(q_{n - 1}) = \phi_\xi(\lim_{n \to \infty} q_{n - 1}) = \phi_\xi(q)
	\]
\end{proof}

\begin{note}
	Из вышесказанного совершенно не обязательно, что $q = 1$, ведь у $\phi_\xi$ может быть много неподвижных точек.
\end{note}

\begin{theorem}
	Пусть $P(\xi = 1) < 1$. Тогда
	\begin{itemize}
		\item Если $\E\xi \le 1$, то уравнение $z = \phi_\xi(z)$ имеет единственное решение на $[0; 1]$. (Следовательно, $q = 1$)
		
		\item Если $\centernot\exists \E\xi$ или $\E\xi > 1$, то уравнение $z = \phi_\xi(z)$ имеет единственное решение на $\lsi{0; 1}$. (Причём $q = z_0 \in \lsi{0; 1}$)
	\end{itemize}
\end{theorem}

\begin{note}
	Таким образом, $q$ всегда равно наименьшему корню уравнения $z = \phi_\xi(z)$ на $[0; 1]$.
\end{note}

\begin{proof}
	Для начала отметим, что функция $\phi_\xi$ выпукла вниз на $\lsi{0; 1}$:
	\[
		\forall x \in \lsi{0; 1}\ \ \phi'_\xi(x) = \sum_{k \ge 1} P(\xi = k)kx^{k - 1} \ge 0;\ \ \phi''_\xi(x) = \sum_{k \ge 2} P(\xi = k)k(k - 1)x^{k - 2} \ge 0
	\]
	Причём $\phi_\xi(0) = P(\xi = 0) \ge 0$, $\phi_\xi(1) = 1$. Произведём большой разбор случаев:
	\begin{itemize}
		\item $P(\xi = 0) = 1$. Тогда $\phi_\xi = 1$, $\E\xi = 0$ тривиальным образом
		
		\item $P(\xi = 0) < 1 \wedge \forall x \in \lsi{0; 1}\ \phi''_\xi(x) = 0$. Тогда, можно решить дифференциальное уравнение и найти $\phi_\xi(x) = P(\xi = 0) + xP(\xi = 1)$. Отсюда $\E \xi \le 1$ и $x = 1$ --- единственный корень уравнения $x = \phi_\xi(x)$
		
		\item $P(\xi = 0) < 1 \wedge \phi''_\xi|_{\lsi{0; 1}} \neq 0$. Из формы $\phi''_\xi$ в виде ряда видно, что \\ $\forall x \in (0; 1)\ \phi''_\xi(x) > 0$, поэтому кроме корня 1 может быть только ещё один корень $x_0 \in \lsi{0; 1}$.
		\begin{itemize}
			\item $\phi'_\xi(1) = \E\xi \le 1$. Тогда, второго корня нет, ибо $\phi_\xi(z) \ge z$ при $z \in [0; 1]$ из-за выпуклости вниз.
			
			\item $\E\xi > 1$ либо $\E\xi = +\infty$. Тогда, есть ровно один корень $x_0 \in \lsi{0; 1}$. Покажем, что $q = x_0$ в таком случае. Для этого достаточно проверить, что $q_n \le x_0$. Сделаем это по индукции:
			\begin{itemize}
				\item База $n = 0$: $q_0 = \phi_\xi(0) \le \phi_\xi(x_0) = x_0$.
				
				\item Переход $n > 0$: $\phi_\xi(q_{n - 1}) = q_n \ge q_{n - 1}$, поэтому $\phi_\xi(q_n) - q_n \ge 0$, а отсюда сразу $q_n \le z_0$ в силу известных свойств $\phi_\xi$.
			\end{itemize}
		\end{itemize}
	\end{itemize}
\end{proof}

\begin{example}
	Рассмотрим бинарное деление частиц: с вероятностью $p$ частица производит 2х потомков, либо с вероятностью $1 - p$ просто умирает. Тогда, вероятность вырождения --- это корень уравнения
	\[
		x = (1 - p) + px^2
	\]
	Отсюда $q = (1 - p) / p$ при $p > 1 / 2$ и $q = 1$ иначе.
\end{example}

\begin{example}
	В случае деления частиц с геометрическим распределением ($P(\xi = k) = (1 - p)p^k$, $p \in (0; 1)$), получится уравнение
	\[
		x = \frac{1 - p}{1 - xp}
	\]
	В результате $q$ принимает те же значения, что и в предыдущем примере. Величину $p$ можно однозначно определить по $\E\xi = p / (1 - p)$, поэтому если $\E\xi = 6.5$, то уже $q = 5 / 6 > 0.8$ (огромная вероятность вырождения).
\end{example}