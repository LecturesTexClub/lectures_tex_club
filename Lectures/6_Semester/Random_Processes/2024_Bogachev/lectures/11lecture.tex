\section{Модель страхования Крамера-Лундберга}

\begin{problem}
	Предположим, что мы --- страховая компания. Помимо желания заработать, нам также важно понимать, можем ли мы разориться, и если да, то когда. Пусть $T = \R_+$, $X_t$ --- размер капитала в момент времени $t \in T$, $c > 0$ --- скорость поступления страховых взносов, $\eta_0 > 0$ --- начальный капитал, а $\eta_k$ --- сумма выплат по $k$-му страховому случаю и $N_t$ означает число таких случае до момента $t$ включительно. Тогда, капитал можно описать следующей формулой:
	\[
		X_t = \eta_0 + ct - \sum_{k = 1}^{N_t} \eta_k
	\]
	Вполне логично ожидать, что $(N_t)_{t \in T}$ является пуассоновским процессом интенсивности $\lambda > 0$, $\{\eta_k\}_{k = 1}^\infty$ --- независимые одинаково распределённые неотрицательные величины, причём эта последовательность также независима с $N$. Момент разорения $\tau \colon \Omega \to T \cup \{+\infty\}$ можно задать так:
	\[
		\tau(\omega) = \inf \{t \in T \colon X_t < 0\}
	\]
	Соответствено, вся задача свелась к изучению $P(\tau < +\infty)$. Также мы потребуем, что $a = \E\eta_1 > 0$, $c - \lambda a > 0$ и $\psi(v) = \E e^{v\eta_1} < \infty$ для всех $v > 0$.
\end{problem}

\begin{proposition}
	У процесса $X$ независимые приращения и непрерывные справа траектории.
\end{proposition}

\begin{proof}~
	\begin{itemize}
		\item (Независимость приращений) Рассмотрим $0 \le t_1 < t_2 < t_3 < t_4$. Найдём вид $X_{t_2} - X_{t_1}$:
		\[
			X_{t_2} - X_{t_1} = c(t_2 - t_1) - \sum_{k = 1}^{N_{t_2}} \eta_k + \sum_{k = 1}^{N_{t_1}} \eta_k
		\]
		В силу того, что приращения пуассоновского процесса принимают неотрицательные значения, $N_{t_1} \le N_{t_2}$. Тогда
		\[
			X_{t_2} - X_{t_1} = c(t_2 - t_1) - \sum_{k = N_{t_1} + 1}^{N_{t_2}} \eta_k
		\]
		Отсюда видно, что $X_{t_2} - X_{t_1} = \phi(\eta_{N_{t_1} + 1}, \ldots \eta_{N_{t_2}})$. Этот набор не пересекается с набором для $X_{t_4} - X_{t_3}$, поэтому приращения обязаны быть независимыми. \textcolor{red}{Это не совсем правда, может быть ситуация $N_{t_2} = N_{t_3}$, да и даже так, наборы случайных величин не фиксированы}
		
		\item (Непрерывность справа траекторий) Зафиксируем $\omega \in \Omega$. Тогда
		\[
			X(t, \omega) - X(t_0, \omega) = c(t - t_0) - \sum_{k = N(t_0, \omega)}^{N(t, \omega)}\eta_k(\omega)
		\]
		Часть $c(t - t_0)$ тривиально будет стремится к нулю при $t \to t_0+$. Так как слагаемые под знаком суммы не зависят от $t$, вопрос становится лишь в том, почему $N(t, \omega) - N(t_0, \omega)$ будет стремится к нулю при $t \to t_0+$ для почти всех $\omega \in \Omega$, а это известный факт: у пуассоновского процесса $P$-почти наверное траектории непрерывны справа.
	\end{itemize}
\end{proof}

\begin{lemma} (без доказательства)
	Пусть $s < t$. Тогда
	\[
		\E\exp(-v(X_t - X_s)) = \exp((t - s)g(v)),\ g(v) = \lambda(\psi(v) - 1) - vc
	\]
\end{lemma}

\begin{proof}
	Подставим процесс $X$:
	\[
		X_t - X_s = c(t - s) - \sum_{k = N_s + 1}^{N_t} \eta_k
	\]
	Тогда
	\[
		\E\exp(-v(X_t - X_s)) = e^{-vc(t - s)}\E\exp\ps{v\sum_{k = N_s + 1}^{N_t} \eta_k} = e^{-vc(t - s)}\E\ps{\prod_{k = N_s + 1}^{N_t} e^{v\eta_k}}
	\]
	Последнее математическое ожидание можно посчитать через разбиение $\Omega$ на подмножества $B_{m_s, m_t} = \{N_s = m_s \le m_t = N_t\}$. Тогда
	\begin{multline*}
		\E\ps{\prod_{k = N_s + 1}^{N_t} e^{v\eta_k} \cdot \chi_{B_{m_s, m_t}}} = \E\ps{\prod_{k = m_s + 1}^{m_t} e^{v\eta_k}\chi_{B_{m_s, m_t}}} = \prod_{k = m_s + 1}^{m_t} \E(e^{v\eta_k}\chi_{B_{m_s, m_t}}) =
		\\
		[\text{$\eta_k$ одинаково распр.}] = \prod_{k = m_s + 1}^{m_t} \E(e^{v\eta_1}\chi_{B_{m_s, m_t}}) = \E(e^{v\eta_1}\chi_{B_{m_s, m_t}})^{m_t - m_s}
	\end{multline*}
	Отсюда
	\[
		\E\ps{\prod_{k = N_s + 1}^{N_t} e^{v\eta_k}} = \sum_{m_s \le m_t} \E(e^{v\eta_1}\chi_{B_{m_s, m_t}})^{m_t - m_s}
	\]
	\textcolor{red}{Допридумать}
\end{proof}

\begin{proposition}
	Случайный процесс $M_t = \exp(-vX_t - tg(v))$ является мартингалом относительно фильтрации $(\F_t)_{t \in T}$, порождённой $X$.
\end{proposition}

\begin{proof}
	Согласованность очевидна. Свойство сужения проверяется классическим образом, через выделение приращения:
	\begin{multline*}
		\E(M_t | \F_s) = \E\big(\exp(-v(X_t - X_s) - vX_s -tg(v)) | \F_s\big) =
		\\
		\exp(-vX_s - tg(v))\E(\exp(-v(X_t - X_s)) | \F_s) =
		\\
		[\text{приращения незав.}] = \exp(-vX_s - tg(v))\E\exp(-v(X_t - X_s)) =
		\\
		[\text{лемма}] = \exp(-vX_s - tg(v) + (t - s)g(v)) = \exp(-vX_s - sg(v)) = M_s
	\end{multline*}
\end{proof}

\begin{solution}
	Итак, мы бы снова хотели пользоваться теоремой об остановке, но для $\tau$ ничего не знаем. Зато для всякого $t > 0$ величина $\tau_t = \min\{t, \tau\}$ является ограниченным моментом остановки. Отсюда
	\begin{multline*}
		e^{-v\eta_0} = M_0 = \E M_0 = [\text{т. об остановке}] = \E M_{\tau_t} \ge \E M_{\tau_t}\chi_{\tau \le t} = \E M_\tau\chi_{\tau \le t} =
		\\
		\E\exp(-vX_\tau - \tau g(v))\chi_{\tau \le t} \ge [\text{$X_\tau \le 0$ в силу непр. справа}] \ge
		\\
		\E \exp(-\tau g(v))\chi_{\tau \le t} \ge P(\tau \le t)\min_{s \in [0; t]} \exp(-sg(v))
	\end{multline*}
	Таким образом
	\[
		P(\tau \le t) \le e^{-v\eta_0}\max_{s \in [0; t]} \exp(sg(v))
	\]
	Нужно происследовать выражение справа на минимум.
	\begin{itemize}
		\item $g(v) \le 0$. Тогда $s = 0$ и $\max = 1$. Из всех таких $v$ нужно брать максимальное, чтобы сомножитель $e^{-v\eta_0}$ становился как можно меньше.
		
		\item $g(v) \ge 0$. Тогда $s = t$ и $\max = e^{tg(v)} \ge 1$. Оценка получается $P(\tau \le t) \le \exp(tg(v) - v\eta_0)$. Требуется исследовать на минимум $g_1(v) = tg(v) - v\eta_0$
	\end{itemize}
	Для начала поймём поведение $g$. Посчитаем первую и вторую производные:
	\[
		g'(v) = \lambda\psi'(v) - c;\ \ g''(v) = \lambda\psi''(v)
	\]
	Так как $\psi(v) = \E e^{v\eta_1}$, то $\psi'(v) = \E \eta_1 e^{v\eta_1}$ и $\psi''(v) = \E \eta_1^2e^{v\eta_1}$. Значит, $g''(v) \ge 0$ при $v \ge 0$, то есть $g$ выпукла вниз. При этом $g(0) = 0$ и $g'(0) = \lambda a - c < 0$, а значит у $g$ существует единственный корень $v_0 > 0$ уравнения $g(v) = 0$. Этот корень даёт оценку
	\[
		P(\tau \le t) \le e^{-v_0\eta_0}
	\]
	\textcolor{red}{Поиск минимума $g_1(v)$ просто не производим, либо можно как-то показать, что $v_0$ будет лучшим вариантом и для него.}
\end{solution}