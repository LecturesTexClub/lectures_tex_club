\begin{proposition}
	Случайный процесс $\xi$ является марковским тогда и только тогда, когда
	\[
	\E(fg | \F_{= t}) = \E(f | \F_{= t}) \cdot \E(g | \F_{ = t})
	\]
	где $f$ --- $\F_{\le t}$-измеримая ограниченная функция и $g$ --- $\F_{\ge t}$-измеримая ограниченная функция.
\end{proposition}

\begin{proof}
	По определению условного распределения, марковость процесса означает
	\[
		\forall t \in T\ \forall A \in \F_{\le t}, B \in \F_{\ge t}\ \ \E(\chi_{A \cap B} | \F_{= t}) = \E(\chi_A | \F_{= t}) \cdot \E(\chi_B | \F_{= t})
	\]
	При этом $\chi_{A \cap B} = \chi_A \cdot \chi_B$. Отсюда эквивалентность утверждений становится очевидной.
\end{proof}

\begin{proposition}
	Случайный процесс $\xi$ является марковским тогда и только тогда, когда
	\[
		\forall B \in \F_{\ge t}\ \ P(B | \F_{\le t}) = P(B | \F_{= t})
	\]
\end{proposition}

\begin{note}
	Как и в предыдущем утверждении, есть эквивалентная формулировка
	\[
		\forall f \text{ --- $\F_{\ge t}$-измеримая ограниченная}\ \ \E(f | \F_{\le t}) = \E(f | \F_{= t})
	\]
	И эту эквивалентность можно записать в ещё более слабой форме:
	\[
		\forall B \in \F_{\ge t}\ \ \E(\chi_B | \F_{\le t}) = \E(\chi_B | \F_{= t})
	\]
\end{note}

\begin{proof}~
	\begin{itemize}
		\item[$\Ra$] С учётом эквивалентных утверждений, нужно проверить следующее:
		\[
			\forall B \in \F_{\ge t}, A \in \F_{\le t}\ \ \E(\chi_B\chi_A) = \E(\E(\chi_B | \F_{= t})\chi_A)
		\]
		Действительно:
		\[
			\E(\chi_B\chi_A) = \E(\chi_{A \cap B}) = \E(\E(\chi_{A \cap B} | \F_{= t})) = \E(\E(\chi_A | \F_{= t}) \E(\chi_B | \F_{ = t})) = \E(\chi_A \E(\chi_B | \F_{= t}))
		\]
		Последний переход верен, коль скоро $\E(\chi_B | \F_{= t})$ является $\F_{= t}$-измеримым по определению.
		
		\item[$\La$] Теперь выполнено тождество из утверждения. Пусть $A \in \F_{\le t}$. Необходимо проверить выполнение равенства:
		\[
			\forall C \in \F_{= t}\ \ \E(P(A \cap B | \F_{= t})\chi_C) =^? \E(P(A | \F_{= t})P(B | \F_{= t})\chi_C)
		\]
		Итак, $C \in \F_{= t}$. Распишем правую часть:
		\begin{multline*}
			\E(P(A \cap B | \F_{= t})\chi_C) = \E(\E(\chi_{A \cap B} | \F_{= t})\chi_C) =
			\\
			\E(\E(\chi_{A \cap B}\chi_C | \F_{= t})) = \E(P(A \cap B \cap C | \F_{= t})) = P(A \cap B \cap C)
		\end{multline*}
		Аналогично можно привести левую часть:
		\begin{multline*}
			\E(P(A | \F_{= t})P(B | \F_{= t})\chi_C) = \E(P(A | \F_{= t})P(B \cap C | \F_{= t})) =
			\\
			[\text{условие}] = \E(P(A | \F_{\le t})P(B \cap C | \F_{\le t})) =
			\\
			\E(\chi_A P(B \cap C | \F_{\le t})) = \E(P(A \cap B \cap C | \F_{\le t})) = P(A \cap B \cap C)
		\end{multline*}
	\end{itemize}
\end{proof}

\begin{theorem}
	Случайный процесс $\xi$ со значениями в $E$ является марковским тогда и только тогда, когда
	\[
		\forall g \text{ --- $\cE$-измеримая ограниченная}\ \forall s, t \in T, s \le t\ \ \E(g(\xi_t) | \F_{\le s}) = \E(g(\xi_t) | \F_{= s})
	\]
\end{theorem}

\begin{note}
	Эквивалентная формулировка теоремы выглядит так:
	\[
		\forall B \in \cE\ \forall s, t \in T, s \le t\ \ P(\xi_t \in E | \F_{\le s}) = P(\xi_t \in E | \F_{= s})
	\]
\end{note}

\begin{proof}
	Доказываем не равносильность марковости, а эквивалентному утверждению:
	\[
		\forall f \text{ --- $\F_{\ge t}$-измеримая ограниченная}\ \ \E(f | \F_{\le t}) = \E(f | \F_{= t})
	\]
	\begin{itemize}
		\item[$\Ra$] Если процесс марковский, то $f = g(\xi_t)$ является также и $\F_{\ge t}$-измеримой ограниченной функцией.
		
		\item[$\La$] Проведём доказательство в несколько этапов
		\begin{enumerate}
			\item $f = f_1(\xi_{t_1}) \cdot \ldots \cdot f_n(\xi_{t_n})$ --- функция $f$ имеет выражение через $f_i$ --- $\cE$-измеримые ограниченные функции и $t \le t_1 < \ldots < t_n$. Проверим равенство индукцией по $n \in \N$:
			\begin{enumerate}
				\item База $n = 1$: тривиальное совпадение с теоремой
				
				\item Переход $n > 1$: перепишем левую часть доказываемого равенства так:
				\begin{multline*}
					\E(f_1(\xi_{t_1}) \cdot \ldots \cdot f_n(\xi_{t_n}) | \F_{\le t}) =
					\\
					\E\ps{\E(f_1(\xi_{t_1}) \cdot \ldots \cdot f_n(\xi_{t_n}) | \F_{\le t_{n - 1}}) \Big| \F_{\le t}} =
					\\
					\E\ps{f_1(\xi_{t_1}) \cdot \ldots \cdot f_{n - 1}(\xi_{t_{n - 1}})\E(f_n(\xi_{t_n}) | \F_{\le t_{n - 1}}) \Big| \F_{\le t}}
				\end{multline*}
				А теперь можно воспользоваться условием. Тогда
				\[
					\E(f_n(\xi_{t_n}) | \F_{\le t_{n - 1}}) = \E(f_n(\xi_{t_n}) | \F_{= t_{n - 1}})
				\]
				Причём справа записана величина, измеримая относительно $\F_{= t_{n - 1}}|\cE$. Значит, она может быть записана в виде $\phi(\xi_{t_{n - 1}})$, где $\phi \colon E \to \R$ --- $\cE$-измеримая функция. Итого:
				\begin{multline*}
					\E(f | \F_{\le t}) = \E(f_1(\xi_{t_1}) \cdot \ldots \cdot f_n(\xi_{t_n}) | \F_{\le t}) =
					\\
					\E(f_1(\xi_{t_1}) \cdot \ldots \cdot f_{n - 1}(\xi_{t_{n - 1}})\phi(\xi_{t_{n - 1}}) | \F_{\le t}) = [\text{предположение индукции}] =
					\\
					\E(f_1(\xi_{t_1}) \cdot \ldots \cdot f_{n - 1}(\xi_{t_{n - 1}})\phi(\xi_{t_{n - 1}}) | \F_{= t}) = \E(f | \F_{= t})
				\end{multline*}
				Последний переход работает в силу аналогичной цепочке равенств для $\F_{= t}$, что мы написали в начале разбора этого случая.
			\end{enumerate}
			
			\item $f$ --- произвольная $\F_{\ge t}$-измеримая ограниченная функция. Такая $f$ принадлежит $L_2(P)$ и, более того, представима в виде предела функций $f_j \in L_2(P)$, имеющие рассмотренный выше вид. Тогда, соответствующие условные средние из теоремы тоже сходятся в $L_2(P)$ и равенство установлено для $f$. Естественный вопрос, который тут надо задать: а почему подобные $f_j$ найдутся? Рассмотрим все $\{t_k\}_{k = 1}^\infty \subseteq T$, $t \le t_k \le t_{k + 1}$. Тогда, имеются $\sigma$-алгебры вида $\sigma(\xi_{t_1}, \ldots, \xi_{t_n})$. Их объединение является алгеброй, порождающей $\F_{\ge t}$. Тогда, $f \in \F_{\ge t}$ может быть приближено как линейная комбинация индикаторов множеств из этой алгебры.
		\end{enumerate}
	\end{itemize}
\end{proof}

\begin{corollary}
	Пусть $\xi$ --- марковский процесс со значениями в $E$, $\psi \colon E \to E$ --- изоморфизм. Тогда $\psi(\xi) := (\psi(\xi_t))_{t \in T}$ --- тоже марковский процесс.
\end{corollary}

\begin{problem}
	Пусть $W$ --- винеровский процесс, $f \colon \R \to \R$ --- борелевская функция. Найти выражение для $s, t \in T, s \le t\ \E(f(W_t) | \F_{= s})$ и доказать, что $W$ --- марковский процесс.
\end{problem}

\begin{solution}
	\textcolor{red}{Решить самому}
\end{solution}

\begin{note}
	Этот же результат может быть получен как применение следующей теоремы.
\end{note}

\begin{theorem} (без доказательства)
	Пусть $\xi$ --- процесс с независимыми приращениями. Тогда $\xi$ --- марковский процесс.
\end{theorem}

\begin{problem}
	Пусть $\xi$ --- марковский процесс со значениями в $\R$, $\xi_t \in L_1(P)$. Доказать, что $\xi$ является мартингалом относительно порождённой $\xi$ фильтрации тогда и только тогда, когда выполнено равенство
	\[
		\forall s, t \in T, s \le t\ \ \E(\xi_t | \F_{= s}) = \xi_s
	\]
\end{problem}

\begin{note}
	То есть для марковского процесса, в отличие от мартингала, достаточно заменить $\F_{\le s}$ на $\F_{= s}$.
\end{note}

\begin{solution}~
	\begin{itemize}
		\item[$\Ra$] По определению мартингала:
		\[
			\forall s, t \in T, s \le t\ \ \E(\xi_t | \F_{\le s}) = \xi_s
		\]
		Заметим, что $\xi_t$ является $\F_{\ge t} \subseteq \F_{\ge s}$-измеримой величиной. Значит, можно использовать эквивалентное свойство марковости:
		\[
			\forall s, t \in T, s \le t\ \ \E(\xi_t | \F_{= s}) = \E(\xi_t | \F_{\le s}) = \xi_s
		\]
		
		\item[$\La$] Из определения марковости и условия задачи имеем:
		\[
			\forall s, t \in T, s \le t\ \ \E(\xi_t | \F_{= s}) = \xi_s = \E(\xi_t | \F_{\le s})
		\]
		Получили мартингал по определению.
	\end{itemize}
\end{solution}

\begin{theorem} (без доказательства)
	Пусть $S \subseteq \R^n$ --- борелевское множество с $\sigma$-алгеброй $\cS$ на нём. Также $\xi$ --- марковский процесс со значениями в $S$, причём $T \in \{\lsi{0; +\infty}, \R, \N, \Z\}$. Тогда $\xi$ обладает \textit{переходной функцией} $P \colon T \times S \times T \times \cA$, $\cA \subseteq \cS \to [0; 1]$. Она обладает следующими свойствами:
	\begin{itemize}
		\item Функция $\mu \colon B \mapsto P(s, x, t, B)$ для фиксированных $s, x, t$ --- вероятностная мера. Причём, если $s = t$, то получится мера Дирака $\delta_x$.
		
		\item Функция $f \colon x \mapsto P(s, x, t, B)$ --- борелевская
		
		\item Выполнено \textit{уравнение Чепмена-Колмогорова}
		\[
			\forall s \le u \le t\ \ P(s, x, t, B) = \int_S P(u, y, t, B)P(s, x, u, dy)
		\]
		
		\item Для фиксированных $s, t, B$, $s \le t$ имеет место равенство
		\[
			P(s, \xi_s, t, B) =\aal{P} P(\xi_t \in B | \F_{= s})
		\]
		где $P$ у равенства --- вероятностная мера исходного пространства
	\end{itemize}
\end{theorem}

\begin{note}
	Переходная функцию можно понимать как вероятность процесса попасть в множество $B$ в момент времени $t$ при нахождении в точке $x$ в момент времени $s$.
\end{note}

\begin{proposition} (Связь конечномерных распределений с переходной функцией, без доказательства)
	Пусть $\xi$ --- случайный процесс, обладающий переходной функцией. Тогда
	\begin{multline*}
		\forall C \in \R^n\ \ P((\xi_{t_1}, \ldots, \xi_{t_n}) \in C) =
		\\
		\int_E \cdots \int_E \chi_C(x_1, \ldots, x_n) P(t_{n - 1}, x_{n - 1}, t_n, dx_n) \cdot \ldots \cdot P(t_1, x_1, t_2, dx_2)dP_{t_1}(x_1)
	\end{multline*}
\end{proposition}