\begin{note}
	Далее нам будет удобно обозначить $\B_T = \sigma(\R^T)$
\end{note}

\begin{note}
	Разумный вопрос, на который был дан однозначный ответ в рамках теории вероятностей: <<Если я задал некоторую функцию распределения, то существует ли случайный вектор, соответствующий ей?>> Так же и тут, но с заданным распределением $P$ на пространстве $(\R^T, \sigma(\R^T))$. Соответствующий процесс должен тоже быть в каком-то роде тождественным, то есть для каждого исхода (оно же траектория) выдавать его самого:
	\[
	\forall \omega \in \R^T\ \ \xi_t(\omega) = \omega(t)
	\]
	Покажем, что процесс $\xi$, заданный таким образом, действительно обладает нужной мерой. Действительно, меру цилиндра (достаточно проверить только на алгебре) можно переписать так:
	\begin{multline*}
		P_\xi(C_{t_1, \ldots, t_n, B}) = P\{\omega \in \Omega \colon (\xi_{t_1}(\omega), \ldots, \xi_{t_n}(\omega)) \in B\} =
		\\
		P\{\omega \in \Omega \colon (\omega(t_1), \ldots, \omega(t_n)) \in B\} = P(C_{t_1, \ldots, t_n, B})
	\end{multline*}
\end{note}

\begin{example}
	Рассмотрим нулевой процесс: $\xi_t(\omega) = 0$, $T = [0; 1]$. Тогда его распределение --- это \textit{мера Дирака} $\delta_0$:
	\[
	\forall E \in \sigma(\R^T)\ \ \delta_0(E) = \System{
		&{1, 0 \in E}
		\\
		&{0, 0 \notin E}
	}
	\]
	Казалось бы, тогда $\delta_0(\{0\}) = 1$, но есть нюанс: $\{0\} \notin \sigma(\R^T)$! В самом деле, как это уже обсуждалось ранее, для проверки свойства мы можем использовать не более чем счётное число точек, а для проверки константной траектории этого явно недостаточно. Более того, даже если мы произведём \textit{пополнение $\sigma(\R^T)$ алгебры по мере $\delta_0$} (то есть добавим в неё все множества, измеримые по $\delta_0$, причём их мера нулевая), то всё равно нулевая траектория не будет ему принадлежать (наше множество состоит из одного элемента, поэтому его максимум можно разбить на пустое и само себя, что бессмысленно).
\end{example}

\begin{exercise}
	Найдите пополнение $\sigma(\R^T)$ по мере $\delta_0$
\end{exercise}

\begin{definition}
	Пусть $\xi$ --- случайный процесс. \textit{Конечномерным распределением} $\xi$ на $\R^n$ называется мера, заданная следующим образом:
	\[
	\forall B \in \B(\R^n)\ \ P_{t_1, \ldots, t_n}(B) = P(\omega \in \Omega \colon (\xi_{t_1}(\omega), \ldots, \xi_{t_n}(\omega)) \in B)
	\]
\end{definition}

\begin{proposition}
	Конечномерные распределения обладают следующими свойствами:
	\begin{enumerate}
		\item Есть согласованность:
		\[
		P_{t_1, \ldots, t_n}(B) = P_{t_1, \ldots, t_n, t_{n + 1}}(B \times \R)
		\]
		
		\item Пусть $\sigma \in S_n$ --- перестановка $n$ индексов. Тогда
		\[
		P_{t_{\sigma(1)}, \ldots, t_{\sigma(n)}}(B) = P_{t_1, \ldots, t_n}(B')
		\]
		где $B'$ --- это $B$, полученное соответствующей перестановкой координат
	\end{enumerate}
\end{proposition}

\begin{proof}
	Очевидно из определения
\end{proof}

\begin{note}
	Эти свойства --- необходимые для того, чтобы набор мер на конечных пространствах соответствовал мере на пространстве траекторий. А можно ли что-то сказать в обратную сторону, то есть какие условия на набор достаточны?
\end{note}

\begin{theorem} (Колмогорова о согласованных конечномерных распределениях, без доказательства)
	Пусть $\{t_k\}_{k = 1}^\infty \subseteq T$. Если набор мер $\{P_{t_1, \ldots, t_n}\}$ обладает свойствами
	\begin{enumerate}
		\item Согласованность
		\[
		P_{t_1, \ldots, t_n}(B) = P_{t_1, \ldots, t_n, t_{n + 1}}(B \times \R)
		\]
		
		\item Если $\sigma \in S_n$ --- перестановка $n$ индексов, то
		\[
		P_{t_{\sigma(1)}, \ldots, t_{\sigma(n)}}(B) = P_{t_1, \ldots, t_n}(B')
		\]
		где $B'$ --- это $B$, полученное соответствующей перестановкой координат
	\end{enumerate}
	Тогда, существует мера $P$ на пространстве $(\R^T, \B_T)$, соответствующая этому набору мер:
	\[
	\forall n \in \N\ \forall B \in \B(\R^n)\ \ P_{t_1, \ldots, t_n}(B) = P\{x \in \R^T \colon (x(t_1), \ldots, x(t_n)) \in B\}
	\]
\end{theorem}

\begin{proof}
	Условия на набор мер позволяют корректно определить меру $P$ по последнему равенству. Проблема связана с тем, что цилиндр может быть представим разными способами, и если у нас нет условий теоремы, то гарантировать мы ничего не можем. Дальше остаётся проверить, что заданная равенством $P$ действительно является счётно-аддитивной мерой на алгебре цилиндров. Но чтобы доказать именно счётно-аддитивность, нужно использовать упражение про компактный класс.
\end{proof}

\begin{note}
	\textcolor{red}{Сюда нужно какую-то затравку}
\end{note}

\begin{definition}
	Пусть $\xi, \eta$ --- случайные процессы на одном и том же пространстве $(\Omega, \F, P)$. Тогда $\eta$ называют \textit{версией (или модификацией)} $\xi$, если
	\[
	\forall t \in T\ \ \xi_t =^{P\text{ п.н.}} \eta_t
	\]
\end{definition}

\begin{example}
	Рассмотрим пространство $([0; 1], \B[0; 1], \mu)$, $T = [0; 1]$ и процесс $\xi = 0$ и $\eta$, заданный следующим образом:
	\[
		\eta_t(\omega) = \System{
			&{0, \omega \neq t}
			\\
			&{1, \omega = t}
		}
	\]
	Понятно, что второй процесс не имеет нулевых траекторий, потому что всегда будет скачок при $t = \omega$. Однако, если мы посмотрим на процесс при фиксированном $t$, то заметим, что $\eta_t =^{\text{п.в.}} = 0$. Таким образом, $\eta$ является версией нулевого процесса $\xi$.
\end{example}

\begin{note}
	Из определения очевидно, что версии имеют одинаковые распределения. Однако, обратное неверно (точно так же, как и в случае случайных величин)!
\end{note}

\begin{theorem} (Колмогорова о непрерывных траекториях, без доказательства)
	Пусть $\xi$ --- это случайный процесс с $T = [a; b]$, причём выполнено условие
	\[
		\exists \alpha, \eps > 0, C \ge 0 \such \forall t, s \in T\ \ \E |\xi_t - \xi_s|^\alpha \le C|t - s|^{1 + \eps}
	\]
	Тогда у $\xi$ существует такая непрерывная версия $\eta$, что $\eta_t$ --- непрерывная траектория при почти всех $\omega \in \Omega$.
\end{theorem}

\begin{proof}
	\textcolor{red}{Конструктивные рассуждения}
\end{proof}

\begin{note}
	На самом деле, эта теорема позволяет построить версию с чисто непрерывными траекториями. Действительно, так как у нас остались <<плохие>> исходы меры ноль, то положим $\eta_t = 0$ в них. Этим мы не испортим версию, ибо затронули при любом фиксированном $t \in T$ только множество нулевой меры.
\end{note}

\begin{lemma} (о сужении)
	Пусть $\mu$ --- вероятностная мера на пространстве $(X, \F)$ и $X_0 \subseteq X$ --- множество с единичной верхней мерой $\mu^*(X_0) = 1$ (то есть в $X \bs X_0$ нет множеств из $\F$ положительной меры). Тогда $\mu$ можно сузить на $X_0$ в следующем смысле:
	\begin{enumerate}
		\item В $X_0$ есть $\sigma$-алгебра $\F_0 = \{X_0 \cap A, A \in \F\}$
		
		\item Можно корректно определить вероятностную меру $\wt{\mu}$ на $(X_0, \F_0)$:
		\[
			\wt{\mu}(X_0 \cap A) = \mu(A)
		\]
	\end{enumerate}
\end{lemma}

\begin{proof}
	Лемма очевидна по модулю корректности, её-то и надо проверить. То есть, почему нет проблем, если $X_0 \cap A = X_0 \cap B$? Если это так, то
	\[
		X_0 \cap A = X_0 \cap B = X_0 \cap (A \cap B)
	\]
	Отсюда получаем, что $A \Delta B \subseteq X \bs X_0$, а значит $\mu(A \Delta B) = 0$ и требуемое установлено.
\end{proof}