\begin{note}
	Рассмотрим процессы, у которых при почти всех исходах имеется непрерывная траектория и $T = [a; b]$. Доказанная лемма позволяет строить распределения таких процессов не в исходном пространстве $(\R^T, \B_T)$, а в пространстве непрерывных траекторий $(C(T), \F)$, где $\F = \B_T \cap C[a; b]$ (это неформальная запись, на деле же каждый элемент из системы $\B_T$ надо пересечь с $C[a; b]$).
	
	Действительно, если при почти всех исходах траектории $\xi_t$ непрерывны, то $\mu^*_\xi(C[a; b]) = 1$ (если $E \subseteq \R^T \bs C[a; b]$, то $\mu_\xi(E) = P\{\omega \in \Omega \colon \xi(\omega) \in E\} = 0$).
	
	Важно отметить, что подобное сужение процесса $\xi$ не будет являться его версией! В определении версии мы не разрешали как-либо менять исходное пространство.
\end{note}

\begin{exercise}
	Доказать, что $C[a; b]$ --- полное сепарабельное линейное нормированное пространство.
\end{exercise}

\begin{corollary}
	Существует $\B(C[a; b])$ --- борелевская $\sigma$-алгебра на пространстве непрерывных функций.
\end{corollary}

\begin{exercise}
	Доказать, что $\B(C[a; b]) = \B_T \cap C[a; b]$. То есть, борелевская $\sigma$-алгебра на пространстве непрерывных функций порождается функциями-проекторами.
\end{exercise}

\section{Процессы с независимыми приращениями}

\begin{note}
	В теории вероятностей крайне важна концепция последовательности независимых случайных величин. Аналогичным вопрос можно рассмотреть и для процессов (потребовать $t \neq s \Ra \xi_t \indep \xi_s$), но такой подход ни к чему не приведёт. Как оказалось, удачной формулировкой будет требовать независимость не значений в разных точках, а приращений между ними.
\end{note}

\begin{definition}
	Пусть $T \subseteq \R$. Случайный процесс $\xi$ имеет \textit{независимые приращения}, если
	\[
		\forall \{t_i\}_{i = 1}^n \subseteq T,\ t_i < t_{i + 1}\ \ (\xi_{t_2} - \xi_{t_1}) \indep \ldots \indep (\xi_{t_n} - \xi_{t_{n - 1}}) \text{ --- независимы в совокупности}
	\]
\end{definition}

\begin{definition}
	\textit{Пуассоновским процессом с интенсивностью $\lambda$} называется процесс с независимыми приращениями $(N_t)_{t \in T}$ такой, что $N_0 = 0$ и приращения имеют распределения
	\[
		\forall t > s\ \ N_t - N_s \sim Poiss(\lambda(t - s))
	\]
\end{definition}

\begin{note}
	Иначе говоря:
	\[
		\forall t > s, k \in \N_0\ \ P(N_t - N_s = k) = \frac{(\lambda(t - s))^k}{k!}e^{-\lambda(t - s)}
	\]
\end{note}

\begin{definition}
	\textit{Винеровский процессом} называется процесс $(W_t)_{t \in T}$ с независимыми приращениями такой, что $W_0 = 0$ и приращения имеют распределения
	\[
		\forall t > s\ \ W_t - W_s \sim N(0, t - s)
	\]
\end{definition}

\begin{note}
	Винеровский процесс изначально был построен как абстракное описание \textit{броуновского движения}.
\end{note}

\begin{proposition}
	$\forall t \neq s \in T\ \ \E W_tW_s = \min(t, s)$
\end{proposition}

\begin{proof}
	Пусть $t > s$. Распишем дисперсию приращения по линейности:
	\[
		\E(W_t - W_s)^2 = t - s = \E W_t^2 + \E W_s^2 - 2\E W_tW_s
	\]
	Так как $W_0 = 0$, то $\E W_t^2 = \E (W_t - W_0)^2 = t$. Отсюда
	\[
		\E W_t W_s = \frac{1}{2} (t + s - (t - s)) = s = \min(t, s)
	\]
	Случай $t \le s$ аналогичен.
\end{proof}

\begin{note}
	Определения это, конечно, хорошо, но ключевой вопрос здесь --- а существуют ли такие процессы? Естественно да, и есть даже критерий существования. Чтобы его сформулировать, придётся вспомнить некоторые вещи о характеристических функциях.
\end{note}

\begin{reminder}
	\textit{Характеристической функцией (преобразованием Фурье)} для меры $\mu$ на $(\R^n, \B(\R^n))$ называется мера, заданная следующим образом:
	\[
		\forall y \in \R^n\ \ \wt{\mu}(y) = \int_{\R^n} e^{i(y, x)}d\mu(x)
	\]
	Таким образом, $\wt{\mu} \colon \R^n \to \Cm$.
\end{reminder}

\begin{reminder}
	Если $\wt{\mu} = \wt{\nu}$, то $\mu = \nu$.
\end{reminder}