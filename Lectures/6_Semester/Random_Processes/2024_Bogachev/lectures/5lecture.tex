\begin{proposition}
	Гауссовский процесс однозначно определяется своим средним и ковариационной функцией.
\end{proposition}

\begin{proof}
	Конечномерные распределения однозначно определяют гауссовский процесс. Так как по среднему и ковариационной функции можно однозначно восстановить параметры конечномерных распределений и наоборот, исходное утверждение очевидно.
\end{proof}

\begin{theorem} (Колмогорова о существовании гауссовского случайного процесса)
	Пусть $M \colon T \to \R$ --- произвольная функция, $K \colon T^2 \to \R_+$ --- положительно полуопределённая симметричная функция. Тогда, существует гауссовский случайный процесс со средним $M(t)$ и ковариацией $K(t, s)$.
\end{theorem}

\begin{proof}
	Проверим условия теоремы Колмогорова о существовании процесса в терминах характеристических функций. Запишем характеристическую функцию конечного распределения:
	\[
	\phi_{t_1, \ldots, t_n}(y) = \exp\ps{i\sum_{j = 1}^n M(t_j)y_j - \frac{1}{2}\sum_{j, k \le n} K(t_j, t_k)y_jy_k}
	\]
	Из вида очевидно, что если положить $y_m = 0$, справа получится выражение, соответствующее харфункции конечного распределения без $t_m$.
\end{proof}

\begin{note}
	Может ли быть гауссовский процесс с независимыми приращениями? На этот вопрос можно дать ответ, воспользовавшись уже доказанным критерием. Нужно проверить выполнение свойства:
	\[
	\forall 0 < s < u < t\ \ \phi_{s, t} = \phi_{s, u} \cdot \phi_{u, t}
	\]
	Для начала, разберёмся со случаем $M = 0$ (среднее процесса нулевое. Если $M \neq 0$, то к процессу с нулевым средним можно прибавить функцию $M$ и получить требуемое). Так как $\xi_t - \xi_s$ является линейной комбинацией гауссовских, то эта случайная величина обязана быть гауссовской. Стало быть
	\[
	\phi_{s, t}(y) = \exp\ps{-\frac{1}{2}\sigma_{s, t}^2y^2}
	\]
	где $\sigma_{s, t}^2 = \E (\xi_t - \xi_s)^2 - 0 = K(t, t) + K(s, s) - 2K(s, t)$. Более того, используя вид характеристической функции гауссовской случайной величины, можно переписать критерий в следующем виде:
	\[
	\forall 0 < s < u < t\ \ \sigma_{s, t}^2 = \sigma_{s, u}^2 + \sigma_{u, t}^2
	\]
	Если мы дополнительно потребуем, что $\sigma_{s, t}^2 = \psi(t - s)$ для $s \le t$, то у критерия появляется ещё один вид:
	\[
	\forall 0 < s < u < t\ \ \psi(t - s) = \psi(u - s) + \psi(t - u)
	\]
	Если обозначить $v = u - s \ge 0$ и $w = t - u \ge 0$, то получим следующее:
	\[
	\forall v, w \ge 0\ \ \psi(v + w) = \psi(v) + \psi(w) \text{ --- аддитивность $\psi$}
	\]
	В классе произвольных функций, подобных $\psi$ существует крайне много. Однако, если потребовать от $\psi$ непрерывности, то вид $\psi$ однозначен --- произвольная линейная функция $\psi(v) = kv$.
	
	Интересный факт состоит в том, что при $k = 1$ получается винеровский процесс!
\end{note}

\begin{proposition}
	Рассмотрим гауссовский процесс $\xi$ со средним $M = 0$ и ковариационной функцией вида
	\[
		K(s, t) = \chi\{s = t\}
	\]
	Тогда $\xi$ --- процесс с \textit{независимыми значениями}, то есть $\xi_{t_1}, \ldots \xi_{t_n}$ независимы в совокупности для любых $\{t_i\}_{i = 1}^n \subseteq T$.
\end{proposition}

\begin{proof}
	По условию $K(s, t) = \chi\{s = t\} = \cov(\xi_s, \xi_t)$. Так как $\xi_s$ является гауссовской случайной величиной, некоррелированность равносильна независимости случайных величин.
\end{proof}

\section{Дополнительные свойства винеровского процесса}

\begin{theorem} (без доказательства)
	Пусть $\{\xi_n\}_{n = 1}^\infty$ --- независимые стандартные гауссовские случайные величины, $\{e_n\}_{n = 1}^\infty$ --- ортонормированный базис в $L_2[0; 1]$. Тогда ряд
	\[
		W_t(\omega) = \sum_{n = 1}^\infty \xi_n(\omega)\phi_n(t)
	\]
	где $\phi_n(t) = \int_0^t e_n(s)d\mu(s)$, сходится равномерно при почти всех $\omega \in \Omega$, причём его сумма --- винеровский процесс.
\end{theorem}

\begin{theorem} (<<Липшицевость>> винеровского процесса, без доказательства)
	Пусть $0 < \alpha < \frac{1}{2}$, $W$ --- винеровский процесс. Тогда, существует случайная величина $C_\alpha(\omega)$ такая, что
	\[
		\forall s < t\ \ |W_t(\omega) - W_s(\omega)| \le\aal{P} C_\alpha(\omega)|t - s|^\alpha
	\]
\end{theorem}

\begin{theorem} (без доказательства)
	Пусть $W$ --- винеровский процесс. Тогда почти всюду на $T$ траектории $W$ не дифференцируемы. Более того, они $P$-почти наверное обладают неограниченной вариацией на всяком отрезке.
\end{theorem}

\begin{theorem} (Закон повторного логарифма, без доказательства)
	С вероятностью 1 выполнены следующие пределы:
	\[
		\varlimsup_{t \to \infty} \frac{|W_t(\omega)|}{\sqrt{2t\ln\ln t}} = 1,\quad \varlimsup_{t \to 0} \frac{|W_t(\omega)|}{\sqrt{2t\ln|\ln t|}} = 1
	\]
\end{theorem}

\begin{note}
	Эти пределы объясняют асимптотическое поведение винеровского процесса в нуле и на бесконечности.
\end{note}

\begin{problem} (Колмогорова)
	Пусть $H \in \rsi{0; 1}$ --- \textit{индекс Хёрста}. Тогда, определим функцию $K_H \colon \R_+^2 \to \R$ следующим образом
	\[
		\forall s, t \in \R_+\ \ K_H(s, t) = \frac{1}{2}(|s|^{2H} + |t|^{2H} - |s - t|^{2H})
	\]
	Нужно доказать, что эта функция является ковариационной функцией для гауссовского процесса $B_t^H$ с нулевым средним. Такой процесс называется \textit{дробным броуновским движением с параметром $H$}. При $H = 1 / 2$ получается винеровский процесс.
\end{problem}

\begin{solution}
	Идея состоит в том, чтобы подобрать кривую $x(t)$ в гильбертовом пространстве со следующим свойством:
	\[
		\|x(t) - x(s)\|^2 = |t - s|^{2H}
	\]
	Тогда, из факта существования этой функции можно извлечь неотрицательность $K_H$.
\end{solution}

\begin{note}
	Дробное броуновское движение, как оказалось после Колмогорова, играет важную роль в финансовой математике.
\end{note}

\section{Условные вероятности и условные математические ожидания}

\begin{note}
	Далее $(\Omega, \F, P)$ --- вероятностное пространство. Мы будем обозначать классы интегрируемых функций $L_q(\mu)$, определённых как в курсе математического анализа, но с рассмотрением интеграла Лебега по некоторой мере $\mu$ на пространстве $(\Omega, \F)$. Также $\cA \subseteq \F$ --- обозначение некоторой под-$\sigma$-алгебры, $L_q(A, \mu) \subseteq L_q(\mu)$ --- подпространство интегрируемых функций, измеримых относительно $\cA$.
	
	Важно помнить, что элементы этого пространства всё же не конкретные функции, а классы эквивалентности по равенству $\mu$-почти наверное.
\end{note}

\begin{definition}
	Пусть $\xi \in L_1(P)$ --- случайная величина. Тогда \textit{условное математическое ожидание (среднее) $\xi$} на $\cA$ есть $\cA$-измеримая случайная величина $\E^\cA\xi \in L_1(P)$, для которой верно тождество:
	\[
		\forall \eta \text{ --- $\cA$-измеримая ограниченная случайная величина}\ \ \E(\xi\eta) = \E((\E^\cA\xi)\eta)
	\]
\end{definition}

\begin{note}
	$\E^\cA\xi = \E(\xi | \cA)$ --- есть второе обозначение условного среднего.
\end{note}

\begin{note}
	Почему условное среднее существует? Поясним этот факт
\end{note}

\begin{proposition}
	Пусть $\xi \in L_2(P)$. Тогда $\E^\cA \xi$ есть ортогональная проекция $\xi$ на $L_2(\cA, P)$.
\end{proposition}

\begin{proof}
	Перепишем определение условного среднего в следущем виде:
	\[
		\E(\xi\eta) - \E((\E^\cA\xi)\eta) = \int_\Omega (\xi - \E^\cA\xi)\eta dP = (\xi - \E^\cA\xi, \eta)_{L_2(P)} = 0
	\]
	Равенство выше выполнено для любых $\cA$-измеримых ограниченных случайных величин. В силу построения интеграла Лебега, подмножество этих функций всюду плотно в $\cL_2(\cA, P)$. Стало быть, скалярное произведение с любой $\eta \in L_2(\cA, P)$ будет равно нулю, а значит $\xi - \E^\cA \xi$ ортогонально подпространству $L_2(\cA, P)$, поэтому $\E^\cA \xi$ --- проекция на это подпространство.
\end{proof}

\begin{corollary}
	Если $\xi \in L_2(P)$, то $\E^\cA\xi$ существует.
\end{corollary}

\begin{proof}
	$L_2(P)$ --- гильбертово сепарабельное пространство. По теореме о проекции, существует и единственна (здесь в терминах классов эквивалентности) проекция $\xi$ на $L_2(\cA, P)$.
\end{proof}

\begin{note}
	В случае, когда известно лишь $\xi \in L_1(P)$, можно воспользоваться теоремой Радона-Никодима. \textcolor{red}{Здесь лектор поясняет, как именно использовать, но в классическом курсе ОВиТМа этой теоремы не рассказывалось}
\end{note}

