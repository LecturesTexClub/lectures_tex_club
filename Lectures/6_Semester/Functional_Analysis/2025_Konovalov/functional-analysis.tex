\setcounter{section}{-1}
\section{Напоминание}
\epigraph{Когда вы произносите японские слова в русском языке, вам нужно куда-то ставить ударение, а не просто губами шуршать.}{Сергей Петрович}

Данные теоремы из прошлого семестра могут в дальнейшем упоминаться, поэтому для удобства чтения они включены и сюда.

\label{th:4-1} \textbf{Теорема.} (Рисса, 4.1) Пусть $E$ --- линейное нормированное пространство, $\dim(E) = \infty$.
Тогда единичная сфера в $E$ не является компактном (и даже не вполне ограниченным).

\textbf{Следствие.} Единичная сфера в $E$ является компактом тогда и только тогда, когда $\dim(E) < \infty$.

\label{th:4-3} \textbf{Теорема.} (О проекции, 4.3) Пусть $H$ --- гильбертово пространство, $M \subset H$ --- собственное подпространство.

\label{th:5-3} \textbf{Теорема.} (5.3) Пусть $E_1$ --- нормированное пространство, $E_2$ банахово.
Пусть $D(A) \subset E_1$ --- линейное многообразие, $\overline{D(A)} = E_1$, $A$ --- линейный оператор, отображающий $D(A)$ в $E_2$ ($D(A)$ --- это просто обозначение области определения оператора $A$).

Тогда существует единственный $\widetilde A \in \mathcal L(E_1, E_2)$, такой что $\widetilde A|_{D(A)} \equiv A$ и $\|\widetilde A\| = \|A\|$.
Тогда $H$ представляется в виде $M \oplus M^\bot$.

\label{th:5-6} \textbf{Теорема.} (Критерий поточечной сходимости оператора из $\mathcal L(E_1, E_2)$, 5.6)
Пусть $E_1$ --- банахово пространство, $E_2$ --- нормированное пространство, $A_n, A \in \mathcal L(E_1, E_2)$.
$A_n \to A$ поточечно тогда и только тогда, когда $\{\|A_n\|\}$ ограничена и $A_n y \to Ay$ для всех $y$ из пространства $Y$, такого что $\overline{[Y]} = E_1$.

\textbf{The Теорема.} (6.2, Хана--Банаха) Пусть $E$ --- нормированное пространство над $\mathbb K \in \{\mathbb R, \mathbb C\}$, $M \subset E$ --- линейное многообразие, $f$ --- линейный ограниченный функционал, определённый на $M$.
Тогда существует функционал $\widetilde f \in E^*$, такой что $\widetilde f|_M = f$ и $\|\widetilde f\| = \|f\|$.

\label{th:hahn-banach-coll-2} \textbf{Следствие 2.} Если $x \ne 0$, то найдётся $f \in E^*$, такой что $\|f\| = 1$ и $f(x) = \|x\|$.

\label{th:hahn-banach-coll-3} \textbf{Следствие 3.} Если $f(x) = f(y)$ для всех $f \in E^*$, то $x = y$.

\label{th:hahn-banach-coll-4} \textbf{Следствие 4.}
Для всех $x$
\[
    \|x\| = \sup_{\|f\|_{E^*} = 1} |f(x)|.
\]

\setcounter{section}{6}
\section{Слабая сходимость}
\epigraph{
    Сэндвич --- это два ломтя хлеба, соединённые начинкой.
}{Сергей Петрович}
Глобально линейные нормированные пространства можно изучать двумя способами: при помощи нормы и при помощи функционалов.
Как понятно из названия параграфа, сейчас мы будем рассматривать второй способ.
Основные понятия данной темы изложены в следующей таблице:

\begin{tabularx}{0.9\textwidth}{|X|X|c|}
    \hline
    Имя & Смысл & Теорема \\
    \hline
    Слабая сходимость & $\forall f \in E^*~ f(x_n) \to f(x)$ & 7.1 \\
    Слабое секвенциальное замыкание, связь с выпуклостью & &\\
    \hline
    Слабое непрерывное отображение & Если $x_n \xrightarrow{w} x$, а $A$ --- линейный оператор, то правда ли, что $Ax_n \xrightarrow{w} x$? & 7.2 \\
    \hline
    Слабая ограниченность & $\forall f \in E^*$ множество $f(S)$ --- это ограниченное числовое множество. & 7.3 \\
    \hline
    Слабая секвенциальная компактность & Из любой $\{x_n\} \subset S$ можно выделить сходящуюся. & 7.4 \\
    \hline
    Слабая полнота & Если $\{x_n\}$ слабо фундаментальна, то слабо сходится. & \\
    \hline
\end{tabularx}

\subsection{Введение}
\textbf{Определение.} Пусть $E$ --- линейное нормированное пространство.
Последовательность $\{x_n\}$ \textit{слабо сходится} к $x$, если для любого $f \in E^*$ выполнено $f(x_n) \to f(x)$.

\textbf{Замечание.} По \hyperref[th:hahn-banach-coll-3]{следствию 3 теоремы Хана--Банаха} определение корректно.

\textbf{Пример.} В гильбертовом пространстве по теореме Рисса--Фреше любой функционал представляется в виде $f(x) = (x, y)$.
Из этого следует, что $x_n \to x \iff (x_n, y) \to (x, y)~\forall y \in H$.

\textbf{Пример.} Если мы в конечномерном пространстве, то слабая сходимость эквивалентна сходимости по норме.

\textbf{Утверждение.} Если $\|x_n - x\| \to 0$, то $x_n \xrightarrow{w} x$.
Следует из определения нормы: $|f(x_n) - f(x)| \le \|f\| \cdot \|x_n - x\|$.

\textbf{Замечание.} Обратное следствие неверно: рассмотрим $l_2$ и базисные векторы.

\textbf{Утверждение.} Пусть $E$ --- ЛНП, $x_n \xrightarrow{w} x$ и все $\|x_n\| \le R \in \mathbb R$.
Тогда $\|x\| \le R$.

\textbf{Доказательство.} Рассмотрим $f \in E^*$, тогда $|f(x_n)| \to |f(x)|$.
Возьмём волшебный функционал из \hyperref[th:hahn-banach-coll-2]{следствия 2 теоремы Хана--Банаха}: $|f(x)| = \|x\|$, а $|f(x_n)| \le \|f\| \cdot R = R$.

\QED

\textbf{Упражнение.} Пусть $H$ --- гильбертово пространство, $x_n \xrightarrow{w} x$ и $\|x_n\| \to \|x\|$.
Тогда $\|x_n - x\| \to 0$.

\label{th:7-1} \textbf{Теорема.} (7.1, критерий слабой сходимости) Пусть $E$ --- линейное нормированное пространство, $\{x_n\} \subset E$.
Тогда $x_n \xrightarrow{w} x$ тогда и только тогда, когда $\{\|x_n\|\}$ ограничена и для любой $f \in S \subset E^*$, такого что $\overline{[S]} = E^*$, выполнено $f(x_n) \to f(x)$.

\textbf{Доказательство.} Условие похоже на \hyperref[th:5-6]{теорему 5.6} (критерий поточечной сходимости), к ней и сведём.
Положим $\pi: E \to E^{**}$, $\pi(x) = F_x$, где $F_x(f) = f(x)$.
Тогда $\|x\| = \sup_{\|f\| = 1} |f(x)|$ по \hyperref[th:hahn-banach-coll-4]{следствию 4 теоремы Хана--Банаха}.
А это равно $\sup_{\|f\| = 1} |F_x(f)|$, что, в свою очередь, равно $\|F_x\|$.
Следовательно, $\|x\| = \|F_x\|$.

Теперь возьмём $E^*$ в качестве банахова пространства $E_1$, $\mathbb K$ --- в качестве нормированного пространства $E_2$.
Теперь заметим, что $x_n \xrightarrow{w} x$ тогда и только тогда, когда $F_{x_n} \to F_x$ поточечно, что завершает сведение к теореме 5.6.

\QED

\textbf{Следствие.} Если $E = l_p$, то $x_n \xrightarrow{w} x$ тогда и только тогда, когда $\{\|x_n\|\}$ ограничено и для всех координат $k$ выполнено $x_n^k \to x^k$ при $n \to \infty$.

\textbf{Следствие.} Если $E = C[0, 1]$, то $f_n \xrightarrow{w} f$ тогда и только тогда, когда $\{\|f_n\|\}$ ограничена и имеет место поточечная сходимость.

\label{th:7-2} \textbf{Теорема.} (7.2) Пусть $E_1, E_2$ --- линейные нормированные пространства, $A \in \mathcal L(E_1, E_2)$.
Если $x_n \xrightarrow{w} x$, то $Ax_n \xrightarrow{w} Ax$.

\textbf{Доказательство.} Рассмотрим $g \in E_2^*$, положим $f = g \circ A$.
Тогда $f(x_n) \to x$ из слабой сходимости $x_n \xrightarrow{w} x$, или, что эквивалентно, $g(Ax_n) \to g(Ax)$, что завершает доказательство.

\QED

\subsection{Слабая ограниченность и компактность}
\label{th:7-3} \textbf{Теорема.} (7.3, Хана) Пусть $E$ --- ЛНП, $S \subset E$.
Тогда $S$ ограничено тогда и только тогда, когда $S$ слабо ограничено.

\textbf{Доказательство.} $\Rightarrow$ очевидно. $\Leftarrow$ от противного: для любого $n \in \mathbb N$ существует $x_n$, такое что $\|x_n\| > n^2$.
Рассмотрим последовательность $y_n = \frac{x_n}{n}$.
Для любого $f \in E^*$ выполнено $|f(y_n)| = \frac{|f(x_n)|}{n}$.
Из слабой ограниченности $x_n$ получаем, что $|f(x_n)|$ ограничена, значит, $|f(y_n)| \to 0$.
Следовательно, $y_n \xrightarrow{w} \ominus$, значит, $\{\|y_n\|\}$ ограничена по \hyperref[th:7-1]{теореме 7.1} --- противоречие.

\QED

\label{th:7-4} \textbf{Теорема.} (7.4, Банаха) Замкнутый шар $\overline B(\ominus, R)$ в гильбертовом пространстве слабо секвенциально компактен.

\textbf{Доказательство.} Возьмём последовательность $\{x_n\} \subset \overline B$ и попробуем найти слабо сходящуюся подпоследовательность $\{y_n\}$.
Вспомним критерий слабой сходимости \hyperref[th:7-1]{7.1}: нам нужна ограниченность $\{\|y_n\|\}$ --- выполнено автоматически --- и для всех $f \in S$, таком что $\overline{[S]} = H^*$, выполнено $f(y_n) \to f(y)$.

Откуда бы взять такое $S$?
Поступим хитро: возьмём $S = \{x_n\}$.
Но тогда $\overline{[S]} \ne H^*$, но это легко чинится: пусть $L = \overline{[S]}$, тогда по теореме о проекции $H = L \oplus L^\bot$.
Если мы докажем, что $y_n \xrightarrow{w} y$ в $L$, то далее для всех $h \in H^*$ имеет место представление $h = l + l^\bot$, откуда $h(y_n) = l(y_n) + 0 \to l(y)$ из ортогональности.

Итак, остаётся выделить $\{y_n\}$ в $L$, для этого подгоним под диагональный метод Кантора.
Рассмотрим последовательность $\{(x_1, x_m)\}_{m=1}^\infty$.
Она ограничена, значит, есть сходящаяся $\{(x_1, x_m^1)\}$ по теореме Больцана--Вейерштрасса.
Теперь рассмотрим последовательность $\{(x_2, x_m^1)\}$, аналогично выделяется сходящаяся $\{(x_2, x_m^2)\}$.
Продолжаем так делать для всех элементов, после чего возьмём последовательность $\{x_n^n\}$ --- для всех $m$ последовательность $\{(x_m, x_n^n)\}$ сходящаяся, значит, она слабо сходится в $L$.

Дополнение от автора: казалось бы, стоит явно предъявить $x \in \overline B$, такой что $x_n^n \xrightarrow{w} x$.
Для этого можно на $L$ рассмотреть линейный функционал $f(y) = \lim_{n \to \infty} (y, x_n^n)$, тогда по теореме Рисса--Фреше найдётся $x \in L$, такой что $(y, x) = \lim_{n \to \infty} (y, x_n^n)$ на $L$.
Теперь, так как все $\|x_n^n\| \le 1$, получаем $\|x\| \le 1$.

\QED

\textbf{Замечание.} Если $E$ лишь банахово, то нужно ещё потребовать сепарабельность и рефлексивность ($E = E^{**}$).

А если только сепарабельно, то шар в $E^*$ секвенциально слабо компактен.

\textbf{Упражнение.} Пусть $E$ --- нормированное пространство, $x_n \xrightarrow{w} x$.
Тогда $x \in \overline{[\{x_n\}]}$.

\textbf{Упражнение.} Пусть $H$ --- гильбертово пространство, $x_n \xrightarrow{w} x$ и $\|x_n\| \to \|x\|$.
Тогда $\|x_n - x\| \to 0$.

\textbf{Вывод.} Опишем для классических пространств $E$ вид $E^*$, вид $f \in E^*$ и вид слабой сходимости:
\begin{itemize}
    \item $l_p$ для $p > 1$: $E^* = l_q$, $f \in E^* \leftrightarrow \sum_{i=1}^{\infty} x_n y_n$, слабая сходимость эквивалентна покоординатной.
    \item $l_1$: $E^* = l_\infty$, $E^*$ аналогично $l_p$, слабая сходимость эквивалентна сходимости по норме, ибо $l_\infty$ слишком большое.
    \item $L_k[a, b]$, $E^* = L_q$, $E^*$ состоит из интегралов $\int_a^b f(x) g(x) \dif x$, слабая сходимость имеет вид $\int_a^\tau f_n(t) \dif t \to \int_a^\tau f(t) \dif t$.
    \item $C[a, b]$, $E^*$ состоит из функций ограниченной вариации $BV[a, b]$, функционалы имеют вид $\int_a^b f(t) \dif g(t)$, слабая сходимость эквивалентна поточечной.
\end{itemize}

\textbf{Замечание.} Обычно слабая сходимость нужна для того, чтобы найти кандидата на сильную сходимость, ибо работать с функионалами проще, чем с нормой, а потом проверить сильную сходимость.

\section{Обратный оператор}
\epigraph{Надо провести эксперимент: если красные тюльпаны посадить в зелёной траве, то будут ли они казаться белыми?}{Сергей Петрович}
Пусть $E_1, E_2$ --- нормированные пространства, $A \in \mathcal L(E_1, E_2)$.

\textbf{Определение.} Если на $\im A$ уравнение $Ax = y$ имеет единственное решение, то $A$ называется \textit{обратимым}.

Вопросы параграфа: когда существует и когда он ограничен и непрерывен.

\textbf{Пример.} Пусть $E_1 = E_2 = E = C[0, 1]$, рассмотрим оператор \textit{Вольтéра} $(Vf)(x) = \int_0^x f(t) \dif t$.
Образом будет $\{g \in C^1~|~g(0) = 0\}$, а обратным оператором будет дифференцирование, не являющееся ограниченным (пример --- $\frac{\sin(nx)}{n}$).

\label{th:8-1} \textbf{Теорема.} (8.1) Пусть $E_1, E_2$ --- нормированные пространства, $A \in \mathcal L(E_1, E_2)$.
Оператор $A^{-1}$ существует и ограничен на $\im A$ тогда и только тогда, когда найдётся $m > 0$, такой что для всех $x \in E_1$ выполнено $\|Ax\| \ge m\|x\|$.

\textbf{Доказательство.} $\Rightarrow$. Пусть $x \in E_1$.
Тогда
\[
    \|x\| = \|A^{-1} A x\| \le \|A^{-1}\| \cdot \|Ax\| \iff \|Ax\| \ge \frac{1}{\|A^{-1}\|}\|x\|.
\]

$\Leftarrow$. Заметим, что ядро тривиально, следовательно, $A$ --- биекция.
Теперь докажем ограниченность обратного: пусть $y = Ax$, тогда
\[
    \|A^{-1} y\| = \|x\| \le \frac{1}{m} \|Ax\| = \frac{1}{m} \|y\|.
\]

\QED

\label{th:8-2} \textbf{Теорема.} (8.2) Пусть $E$ --- банахово пространство, $A \in \mathcal L(E)$, $\|A\| < 1$.
Тогда $(I + A)^{-1} \in \mathcal L(E)$.

\textbf{Доказательство.} Идея: если смотреть на $I$ и $A$, как на числа, то получится
\[
    \frac{1}{I + A} = \frac{1}{1 + A} = \sum_{k=0}^{\infty} (-A)^k.
\]
Так и сделаем: рассмотрим ряд $\sum_{k=0}^{\infty} (-A)^k$, пусть $S_n$ --- его конечная сумма.
Тогда 
\[
    S_n(I + A) = (I + A) S_n = I + (-1)^n A^{n+1}.
\]
Но правая часть стремится к $I$ засчёт $\|A\| < 1$, то есть $S_n \to (I + A)^{-1}$.

\QED

\textbf{Пример.} Рассмотрим матрицу $A$, такую что для всех $i$ сумма $\sum_j |a_{ij}| < 1$, тогда матрица $(I + A)$ обратима.
Действительно, это просто $\infty$--норма.

\textbf{Пример.} Рассмотрим оператор в $C[0, 1]$, сопоставляющий $f \mapsto f(x) + \int_0^x f(t) \dif t$.
Доказать, что он обратим.
К сожалению, норма оператора равна единице, поэтому не получится.

Но для этого существует усиление теоремы: вместо $\|A\| < 1$ потребуем, что существует $n_0$, такое что $\|A^{n_0}\| < 1$, тогда сработает.

\label{th:8-3} \textbf{Теорема.} (8.3) Пусть $E$ --- банахово пространство, $A, A^{-1}, \Delta A \in \mathcal L(E)$.
Если $\|\Delta A\| < \frac{1}{\|A^{-1}\|}$, то $(A + \Delta A)^{-1} \in \mathcal L(E)$.
Очевидно из теоремы 8.2:
\[
    A + \Delta A = A (I + A^{-1} \Delta A).
\]

\label{th:8-4} \textbf{Теорема.} (8.4, Банаха об обратном операторе) Пусть $E_1, E_2$ --- банаховы пространства, $A \in \mathcal L(E_1, E_2)$ --- биекция.
Тогда $A^{-1}$ линеен и непрерывен.

\textbf{Доказательство.} Общий случай доказывается через теорему Бэра, но это духота.
Будем доказывать для $E_1 = E_2 = H$ гильбертовых.
В данном доказательстве мы неожиданно будем ссылаться на \hyperref[th:10-2]{теорему 10.2}, так что перед прочтением доказательства рекомендуется сначала дочитать до туда.

Итак, докажем, что $A^*$ обратим на $H$.
Мы знаем, что $\overline{\im A} \oplus \Ker A^* = H$, или же $\overline{\im A^*} \oplus \Ker A = H$.
Так как $\im A = H$, $\Ker A^* = \{0\}$, то есть $\overline{\im A^*} = H$.
Докажем, что $A^*$ ограничен снизу для \hyperref[th:8-1]{теоремы 8.1}, а потом докажем замкнутость образа, чтобы получить как раз $\im A^* = \overline{\im A^*} = H$.

Так как $\Ker A^*$ тривиально, для любого $x \ne 0$ найдётся $\alpha$, такое что $\|A^*(\alpha x)\| = 1$.
Засим введём $S = \{x: \|A^* x\| = 1\}$ и докажем его ограниченность.
Но по \hyperref[th:7-3]{теореме 7.3 Хана} достаточно показать слабую ограниченность, то есть для любого $y \in H$ существует $K_y > 0$, такое что $|(x, y)| \le K_y$ для всех $x$.
Вспомним, что $A$ у нас --- биекция: пусть $Az = y$, тогда по неравенству КБШ
\[
    |(x, Az)| = |(A^* x, z)| \le \|A^* x\| \cdot \|z\| = \|z\| =: K_y.
\]
Итак, $S$ ограничено, то есть существует $m > 0$, такое что для всех $x \in S$ выполнено $\|x\| \le m$ или же $\|x\| \le m \cdot \|A^* x\|$, что эквивалентно $\|A^* x\| \ge \frac{1}{m} \|x\|$.
Остаётся для любого $y \in H$ найти $x = \alpha y \in S$, тогда
\[
    \|A^* y\| = |\alpha| \cdot \|A^* x\| \ge \frac{|\alpha|}{m} \cdot \|x\| = \frac{1}{m} \|y\|.
\]

Докажем замкнутость $\im A^*$.
Рассмотрим $\{y_n\}$, $y_n \to y$, пусть $A^* z_n = y_n$.
Тогда
\[
    \|A^* z_{n + p} - A^* z_n\| \ge m \|z_{n + p} - z_n\|,
\]
то есть $\{z_n\}$ фундаментальна, значит, сходится к какому-то $z$, что даёт $A^* z_n \to A^*z$, но в то же время $A^* z_n = y_n \to y$.

\QED

\section{Спектр. Резольвента}
\epigraph{Всё, стираю Сергея Петровича.}{Сергей Петрович}
Слово спектр имеет корень ``спек``, что означает нечто видимое, а резольвета --- от ``resolve``, то есть решать.

\textbf{Напоминание.} Аналог данной темы был на алгеме: для оператора $A$ мы рассматриваем решения уравнения $(A - \lambda I) x = 0$ и $(A - \lambda I) x = y$.
Иными словами, исследуем ядро оператора $(A - \lambda I)$ в первом случае, а во втором --- его обратимость.

В данном параграфе мы будем активно пользоваться основной теоремой алгебры, поэтому везде будем считать, что $E$ --- банахово пространство над полем $\mathbb C$.
Также для оператора $A \in \mathcal L(E)$ и $\lambda \in \mathbb C$ мы будем обозначать $A_\lambda = A - \lambda I$.

\textbf{Определение.} Пусть $A \in \mathcal L(E)$.
\begin{itemize}
    \item \textit{Резольвентным множеством} называется $\rho(A) = \{\lambda \in \mathbb C~|~A_\lambda^{-1} \in \mathcal L(E)\}$, то есть ``хорошее`` множество, на котором всё обратимо.
    \item \textit{Спектром} называется $\sigma(A) = \mathbb C \setminus \rho(A)$, то есть ``плохое`` множество.
    \item \textit{Резольвентой} для $\lambda \in \rho(A)$ называется оператор $R_\lambda := A_\lambda^{-1}$.
    \item \textit{Собственными значениями} называется $\sigma_p(A)$ --- множество таких $\lambda \in \mathbb C$, что уравнение $(A - \lambda I) x = 0$ имеет нетривиальное решение.
    \item \textit{Непрерывным спектром} называется $\sigma_C(A)$ --- множество таких $\lambda \in \mathbb C$, что $\Ker A_\lambda = \{\ominus\}$, но $\im A_\lambda \ne E$ и $\overline{\im A_\lambda} = E$.
    \item \textit{Остаточным спектром} называется $\sigma_R(A) = \sigma(A) \setminus (\sigma_p(A) \cup \sigma_C(A))$.
\end{itemize}
Как и в ТФКП, нас будут интересовать именно плохие точки оператора, то есть спектр, ибо именно он определяет интересующие нас свойства.

\textbf{Замечание.} При $\dim E < \infty$ выполнено $\sigma(A) = \sigma_p(A)$, а иначе --- не обязательно.
Пример: для $E = l_2$ оператор сдвига $Ax = (0, x_1, x_2, \dots)$.
Собственных значений нет, но $\sigma(A) = \{|\lambda| \le 1\}$.

\label{th:9-1} \textbf{Теорема.} (9.1) Пусть $A \in \mathcal L(E)$.
Тогда
\begin{enumerate}
    \item $\sigma(A)$ --- непустое замкнутое множество.
    \item Пусть $r(A) := \sup_{\lambda \in \sigma} |\lambda|$, тогда $r(A) = \lim_{n \to \infty} \sqrt[n]{\|A^n\|}$.
\end{enumerate}

\textbf{Доказательство.} Заметим, что $R_\lambda$ можно рассматривать, как оператор $R: \rho(A) \to \mathcal L(E)$.
По факту, $\mathcal L(E)$ --- это некоммутативная алгебра, поэтому в ней можно ввести аналоги топологии, производной, интеграла, интегральную формулу Коши и так далее.

Далее доказательство разбивается на 9 пунктов:
\begin{enumerate}
    \setcounter{enumi}{-2}
    \item Если $|\lambda| > \|A\|$, то $\lambda \in \rho(A)$.
        Следствие из \hyperref[th:8-2]{теоремы 8.2}.

    \item $\rho(A)$ --- открытое множество.
        Следствие из \hyperref[th:8-2]{теоремы 8.3}, что доказывает замкнутость спектра.

    \item $\lambda \mapsto R_\lambda$ непрерывно на $\rho(A)$.
        Аналогично доказательству теоремы 8.2 рассмотрим $B = A_{\lambda_0}$, $\Delta B = -\lambda I$, тогда
        \[
            (B + \Delta B)^{-1} = B^{-1}(I + B^{-1} \Delta B)^{-1} = \sum_{k=0}^{\infty} (-1)^k B^{-1} (B^{-1} \Delta B)^k.
        \]
        Теперь
        \[
            (B + \Delta B)^{-1} - B^{-1} = \sum_{k=1}^{\infty} (-1)^k B^{-1} (B^{-1} \Delta B)^k,
        \]
        значит,
        \[
            \|(B + \Delta B)^{-1} - B^{-1}\| \le \sum_{k=1}^{\infty} \|B^{-1}\|^{1 + k} \cdot \|\Delta B\|^k = \frac{\|B^{-1}\|^2 \cdot \|\Delta B\|}{1 - \|B^{-1}\| \cdot \|\Delta B\|}.
        \]
        Обратно к $A_\lambda$:
        \[
            \|R_{\lambda_0 + \lambda} - R_{\lambda_0}\| \le \frac{\|R_{\lambda_0}\|^2 \cdot |\lambda|}{1 - \|R_{\lambda_0}\| \cdot |\lambda|}.
        \]
        При $\lambda \to 0$ это стремится к нулю, что доказывает непрерывность.

    \item Равенство Гильберта: $R_\lambda - R_{\lambda_0} = (\lambda - \lambda_0) R_\lambda R_{\lambda_0}$.
        Заметим, что
        \[
            A_{\lambda} (R_\lambda - R_{\lambda_0}) A_{\lambda_0} = A_{\lambda_0} - A_{\lambda} = (\lambda - \lambda_0) I.
        \]
        Остаётся домножить на $R_\lambda$ слева и на $R_{\lambda_0}$ --- справа.

    \item $\lambda \mapsto R_\lambda$ дифференцируема на $\rho(A)$.
        Действительно,
        \[
            \frac{R_{\lambda} - R_{\lambda_0}}{\lambda - \lambda_0} = R_{\lambda} R_{\lambda_0}
        \]
        по равенству Гильберта.

    \item $r(A)$ --- это радиус сходимости \textit{ряда Неймана} (Не Фона Неймана, а Карла Неймана):
        \[
            R_{\lambda} = -\frac{1}{\lambda} \sum_{n=0}^\infty \frac{1}{\lambda^n} A^n.
        \]
        Мы знаем, что равенство верно при $|\lambda| > \|A\|$, а также, что $R$ дифференцируема на $\rho(A)$.
        Рассмотрим случаи:
        \begin{itemize}
            \item $|\lambda_0| > r(A)$, докажем сходимость.
                Действительно, здесь $R$ дифференцируема, а значит, раскладывается в ряд Лорана, причём единственным образом, и разложение мы уже знаем.

            \item $|\lambda_0| < r(A)$, докажем расходимость.
                От противного, тогда ряд бы сходился и при всех бóльших по модулю $\lambda$, ибо это ряд Лорана в бесконечности.
                Но из-за этого получается, что $r(A) \le |\lambda_0|$, ибо в точках спектра ряд сходиться не может.
        \end{itemize}

    \item $\sigma(A) \ne \varnothing$.
        От противного: $\rho(A) = \mathbb C$, тогда $R_\lambda$ целая.
        Отсюда при $|\lambda| > \|A\|$ выполнено
        \[
            \|R_\lambda\| \le \frac{1}{|\lambda|} \cdot \frac{1}{1 - \frac{\|A\|}{|\lambda|}},
        \]
        что стремится к нулю при $\lambda \to \infty$, значит, $R_\lambda = \ominus$ по теореме Лиувилля из ТФКП.
        Противоречие: нет обратного оператора.

    \item Если $\lambda \in \sigma(A)$, то $\lambda^n \in \sigma(A^n)$.
        (Аналог факта из линейной алгебры про собственные значения матрицы)
        (Более того, верно и обратное)

        От противного: допустим, что $\lambda^n \in \rho(A^n)$, то есть $(A^n - \lambda^n I)^{-1} \in \mathcal L(E)$.
        Распишем:
        \[
            A^n - \lambda^n I = (A - \lambda I)(A^{n-1} + \lambda A^{n-2} + \dots + \lambda^{n-1} I).
        \]
        Применим справа к обеим частям оператор $(A^n - \lambda^n I)^{-1}$, тогда слева будет $I$, а справа --- композиция $(A - \lambda I)$ и какого-то оператора, то есть правого обратного.
        Но, так как степенные операторы перестановочны, можно аналогично получить, что есть и левый обратный, откуда $(A - \lambda I)$ обратим --- противоречие.

    \item $r(A) = \lim_{n \to \infty} \sqrt[n]{\|A^n\|}$.
        В пункте 4 мы уже доказали, что $r(A) = \overline{\lim}_{n \to \infty} \sqrt[n]{\|A^n\|}$, осталось убрать верхний предел.
        Заметим, что по пункту 6
        \[
            r(A) \le \sqrt[n]{r(A^n)} \le \sqrt[n]{\|A^n\|} =: \alpha_n.
        \]
        Значит, $r(A) \le \alpha_n$ для всех $n$, то есть $\underline{\lim}_{n \to \infty} \alpha_n \ge r(A) = \overline{\lim}_{n \to \infty} \alpha_n$, то есть верхний и нижний предел совпадают, и они равны $r(A)$.
\end{enumerate}

\QED

\textbf{Упражнение.} Пусть $\{\lambda_n\}$ --- числовая последовательность, $Ax = (\lambda_1 x_1, \lambda_2 x_2, \dots)$.
Тогда $\sigma(A) = \overline{\{\lambda_n\}}$.

\section{Сопряжённый оператор}
\epigraph{В русском языке буква ё всегда ударная. Например, в слове ёж.}{Сергей Петрович}
\subsection{В банаховых пространствах}
\textbf{Определение.} Пусть $E_1, E_2$ --- два нормированных пространства над полем $\mathbb K$, $A \in \mathcal L(E_1, E_2)$.
Если $g \in E_2^*$, то $x \mapsto g(Ax)$ --- это линейный функционал, действующий из $E_1$ в $\mathbb K$.
В связи с этим можно определить $A^*: E_2^* \to E_1^*$, $(A^* g) x = g(Ax)$.

\label{th:10-1} \textbf{Теорема.} (10.1) $A^* \in \mathcal L(E_2^*, E_1^*)$, а также $\|A^*\| = \|A\|$.

\textbf{Доказательство.} Линейность очевидна.
Ограниченность:
\[
    |(A^* g)(x)| = |g(Ax)| \le \|g\| \cdot \|A\| \cdot \|x\|,
\]
значит,
\[
    \|A^* g\| \le \|A\| \cdot \|g\|,
\]
то есть $\|A^*\| \le \|A\|$.
Теперь докажем обратное неравенство: по \hyperref[th:hahn-banach-coll-4]{следствию 4 теоремы Хана--Банаха}
\[
    \|Ax\| = \sup_{\|g\| = 1} |g(Ax)| = \sup_{\|g\| = 1} |(A^* g) x| \le \sup_{\|g\| = 1} \|A^*\| \cdot \|g\| \cdot \|x\| = \|A^*\| \cdot \|x\|,
\]
то есть $\|A\| \le \|A^*\|$.

\QED

\subsection{В гильбертовых пространствах}
Далее мы будем рассматривать только гильбертовы пространства, ибо здесь всё гораздо проще.

\textbf{Определение.} Пусть $H_1, H_2$ --- гильбертовы пространства, $A \in \mathcal L(H_1, H_2)$.
\textit{Эрмитово сопряжённым оператором} $A^*: H_2 \to H_1$ называется такой оператор, что для всех $x \in H_1$, $y \in H_2$ выполнено $(Ax, y) = (x, A^* y)$.
Слово ``эрмитово`` далее будет опускаться.

\textbf{Определение.} Пусть $A \in \mathcal L(H)$.
Он называется \textit{самосопряжённым}, если $A = A^*$.

\textbf{Упражнение.} Доказать непосредственно существование сопряжённого оператора (через теорему Рисса--Фреше).

\textbf{Упражнение.}
\begin{itemize}
    \item $(\alpha A + \beta B)^* = \overline \alpha A^* + \overline \beta B^*$.
    \item $(AB)^* = B^* A^*$.
    \item $(A^*)^* = A$.
    \item $I^* = I$.
    \item Если $A$ обратим, то $(A^*)^{-1} = (A^{-1})^*$.
\end{itemize}

\label{th:10-2} \textbf{Теорема.} (10.2) Пусть $H$ гильбертово над $\mathbb C$, $A \in \mathcal L(H)$.
Тогда $\overline{\im A} \oplus \Ker A^* = H$.

\textbf{Доказательство.} Докажем, что $(\im A)^\bot = \Ker A^*$.
Возьмём $y \in (\im A)^\bot$, тогда для всех $x \in H$ выполнено $(Ax, y) = 0$, то есть $(x, A^* y) = 0$, что может быть верно только при $A^* y = 0$, то есть при $y \in \Ker A^*$.
Теперь по теореме 4.3 выполнено $\overline{\im A} \oplus \overline{\im A}^\bot = H$, остаётся заметить, что $(\im A)^\bot = (\overline{\im A})^\bot$.

\QED

\section{Самосопряжённые операторы}
\epigraph{Если вы повесите данное утверждение над своей кроватью, то через некоторое время вы не захотите больше просыпаться.}{Сергей Петрович}
В данном параграфе мы будем работать с гильбертовым пространством $H$ над полем $\mathbb C$ по умолчанию.

\subsection{Введение}
\textbf{Утверждение 1.} Если $A$ --- ССО, $M$ --- инвариантное относительно $A$ подпространство, то $M^\bot$ тоже инвариантно относительно $A$.

\textbf{Доказательство.} Если $y \in M^\bot$, то для всех $x \in M$ выполнено $(Ax, y) = 0$ или же $(x, Ay) = 0$, то есть $Ay \in M^\bot$.

\QED

Как проверить, является ли оператор $A$ самосопряжённым?
Один способ --- подставить в скалярное произведение, но это не всегда легко делается.
Ещё один --- рассмотреть квадратичную форму $K_A: H \to \mathbb C$, $K_H(x) = (Ax, x)$.

\textbf{Утверждение 2.} Если $H$ построено над полем $\mathbb C$, то из $K_A = K_B$ следует $A = B$.
Но если над $\mathbb R$, то это уже не верно.

\textbf{Доказательство.} Рассмотрим $C = A - B$, тогда для всех $x$ выполнено $(Cx, x) = 0$.
Тогда для всех $x, y \in H$ выполнено
\[
    0 = (C(x + y), x + y) = (Cx, x) + (Cx, y)+ (Cy, x) + (Cy, y) = (Cx, y)+ (Cy, x).
\]
Теперь
\[
    (C(x + iy), x + iy) = (Cx, iy) + (Ciy, x) = -i(Cx, y) + i(Cy, x).
\]
Итого получили, что $(Cx, y) + (Cy, x) = (Cx, y) - (Cy, x) = 0$.

\QED

\label{th:11-1} \textbf{Теорема.} (11.1, о свойствах ССО) Пусть $A$ --- ССО в $H$.
Тогда
\begin{enumerate}
    \item Для всех $x \in H$ выполнено $K_A(x) \in \mathbb R$.
        Причём это критерий самосопряжённости.
    \item Если $\lambda$ --- собственное значение $A$, то $\lambda \in \mathbb R$.
    \item Если $e_1, e_2$ --- два собственных вектора, соответствующих разным собственным значениям, то $(e_1, e_2) = 0$.
\end{enumerate}

\textbf{Доказательство.}
\begin{enumerate}
    \item Заметим, что 
        \[
            K_A(x) = (Ax, x) = \overline{(x, Ax)} = (x, Ax).
        \]
        Обратно: по утверждению 2 достаточно проверить, что $K_{A^*} = K_A$.
        Заметим, что $(A^* x, x) = (x, Ax)$, а $(Ax, x) = \overline{(x, Ax)}$.
        Но $(x, Ax) \in \mathbb R$, поэтому они равны.

    \item Пусть $Ae = \lambda e$, тогда, если применить скалярное произведение справа, то $(Ae, e) = \lambda (e, e)$.
        А если слева, --- то $(e, Ae) = \overline \lambda (e, e)$.
        Правые части равны, значит, и левые равны.

    \item Рассмотрим $(Ae_1, e_2)$.
        Это равно $(e_1, Ae_2)$ по самосопряжённости, $\lambda_1(e_1, e_2)$, так как собственный вектор $e_1$, и $\lambda_2(e_1, e_2)$, так как собственный вектор $e_2$.
        Итого $\lambda_1(e_1, e_2) = \lambda_2 (e_1, e_2)$, что может быть верно только при $(e_1, e_2) = 0$.
\end{enumerate}

\QED

\label{th:11-2} \textbf{Теорема.} (11.2) Если $A$ --- ССО, то справедлива формула $\overline{\im A_\lambda} \oplus \Ker A_\lambda = H$.

\textbf{Доказательство.} Это не так просто следует из \hyperref[th:10-2]{теоремы 10.2}, она нам даёт только $\overline{\im A_\lambda} \oplus \Ker A_\lambda^* = H$.
Засим рассмотрим два случая:
\begin{itemize}
    \item $\lambda \in \mathbb R$. Так как $\lambda = \overline \lambda$, $A^*_\lambda = A_\lambda$.
    \item $\lambda \not\in \mathbb R$.
        Из теоремы 11.1 получаем, что $\lambda$ и $\overline \lambda$ --- не собственные значения, тогда $\Ker A_\lambda = \Ker A^*_\lambda = \{0\}$, так как $A^*_\lambda = A_{\overline \lambda}$.
\end{itemize}

\QED

\subsection{Теоремы о спектре}
\label{th:11-3} \textbf{Теорема.} (11.3) Пусть $A$ --- ССО.
Тогда:
\begin{enumerate}
    \item $\lambda \in \rho(A)$ тогда и только тогда, когда найдётся $m > 0$, такое что $\|A_\lambda x\| \ge m \|x\|$ для всех $x$.
    \item $\lambda \in \sigma(A)$ тогда и только тогда, когда существует $\{x_n\}$, такое что все $\|x_n\| = 1$ и $\|A_\lambda x_n\| \to 0$, то есть ``почти собственное значение``.
\end{enumerate}

\textbf{Доказательство.} 
\begin{enumerate}
    \item Вспонимая \hyperref[th:8-1]{теорему 8.1}, получаем, что слева направо очевидно.
        Справа налево: есть ограниченность снизу, нужно $\lambda \in \rho(A)$.
        У нас есть $\overline{\im A_\lambda} \oplus \Ker A_\lambda = H$, причём засчёт ограниченности снизу образ замкнут (доказательство в \hyperref[th:8-4]{теореме 8.4}).
        Значит, $\Ker A_\lambda = \{0\}$ засчёт ограниченности снизу и $\im A_\lambda = H$.
        Наконец, по теореме 8.4 доказательство завершается.

    \item По контрапозиции.
\end{enumerate}

\QED

\label{th:11-4} \textbf{Теорема.} (11.4) Если $A$ --- ССО в $H$, то $\sigma(A) \subset \mathbb R$.

\textbf{Доказательство.} Пусть $\lambda = \mu + i \nu \in \sigma(A)$.
Докажем, что $\|A_\lambda x\| \ge |\nu| \cdot \|x\|$:
\[
    \|A_\lambda x\|^2 = \|A_\mu x - i \nu x\|^2 = 
\]
\[
    = \|A_\mu x\|^2 + (A_\mu x, - i\nu x) + (-i \nu x, A_\mu x) + |\nu|^2 \|x\|^2.
\]
Итак, $\|A_\mu x\|^2 \ge 0$, а второе и третье слагаемые сокращаются засчёт самосопряжённости $A_\mu$.

Следовательно, по теореме 11.3 получаем, что $\nu = 0$.

\QED

\textbf{Лемма.} Если $A$ --- ССО, то $\|A^n\| = \|A\|^n$.

\textbf{Доказательство.} Проверим для $n = 2^k$.
При $n = 2$:
\[
    \|A\|^2 = \sup_{\|x\| = 1} (Ax, Ax) = \sup_{\|x\| = 1} (x, A^2 x) \le \sup_{\|x\| = 1} \|x\| \cdot \|A^2 x\|,
\]
откуда $\|A\|^2 \le \|A^2\|$.
Обратное неравенство тривиально.

Аналогично для остальных степеней двойки по индукции.
Теперь сделаем переход от $n$ к $n - 1$:
\[
    \|A\|^n = \|A^n\| = \|A \cdot A^{n-1}\| \le \|A\| \cdot \|A^{n-1}\| \le \|A\| \cdot \|A\|^{n-1} = \|A\|^n.
\]
Получаем, что $\|A\| \cdot \|A^{n-1}\| = \|A\|^n$, после чего остаётся сократить на $\|A\|$.

\QED

\label{th:11-5} \textbf{Теорема.} (11.5) Пусть $H$ комплексное, $A$ --- ССО.
Тогда:
\begin{itemize}
    \item $\sigma(A) \subset [m_-, m_+]$, где $m_- = \inf (Ax, x)$, $m_+ = \sup (Ax, x)$, где супремум и инфимум берутся по $\|x\| = 1$, причём $m_-, m_+ \in \sigma$.
    \item $r(A) := \max(|m_-|, |m_+|)$, $r(A) = \|A\|$.
\end{itemize}

\textbf{Следствие.} $\|A\| = \sup_{\|x\| = 1} |(Ax, x)|$.

\textbf{Доказательство.} 
Второе --- это просто спектральный радиус, и для него в \hyperref[th:9-1]{теореме 9.1} была выведена формула $\lim_{n \to \infty} \sqrt[n]{\|A^n\|}$.
Засим остаётся применить предыдущую лемму.

Зафиксируем $\lambda > m_+$, хотим показать, что $\lambda \in \rho(A)$.
По неравенству КБШ
\[
    \|A_\lambda x\| \cdot \|x\| \ge |(A_\lambda x, x)| = |(Ax, x) - \lambda (x, x)| =
\]
Так как $(Ax, x) \le m_+\|x\|^2$, модуль раскрывается с понятным знаком:
\[
    = \lambda(x, x) - (Ax, x) \ge \lambda \|x\|^2 - m_+\|x\|^2.
\]
Итак, найдётся $m > 0$, такое что $\|A_\lambda x\| \ge m \|x\|$, что завершает доказательство по \hyperref[th:11-3]{теореме 11.3}.

Факт про $m_-$ аналогичен, докажем, что $m_+ \in \sigma(A)$.
Вновь для теоремы 11.3 найдём последовательность $\{x_n\}$, такую что все $\|x_n\| = 1$ и $A_{m_+} x_n \to 0$.
По определению супремума найдётся $\{x_n\}$, такая что все $\|x_n\| = 1$ и $(Ax_n, x_n) \to m_+ \|x_n\|^2$ --- почти то, что нужно, ведь $((A - m_+ I) x_n, x_n) \to 0$.

Положим $B = m_+ I - A$ --- положительно полуопределённая форма.
Она нам позволяет ввести ``скалярное произведение`` $\left<x, y \right> = (Bx, y)$.
Написано в кавычках, ибо это полускалярное произведение --- свойство $\left<x, x \right> = 0 \iff x = \ominus$ отсутствует.
Однако мы всё ещё можем применить неравенство КБШ.

Докажем, что $Bx_n \to 0$.
Для этого неожиданно докажем, что $\|Bx_n\|^4 \to 0$.
\[
    \|Bx_n\|^4 = (Bx_n, Bx_n)^2 = \left<x_n, Bx_n \right>^2 \le \left<x_n, x_n \right> \cdot \left<Bx_n, Bx_n \right>.
\]
Второй сомножитель ограничен, поэтому достаточно доказать, что $\left<x_n, x_n \right> \to 0$.
Но 
\[
    \left<x_n, x_n \right> = (Bx_n, x_n) = ((A - m_+I)x_n, x_n) \to 0
\]
из доказанного.
Для $m_-$ аналогично.

\QED

\textbf{Замечание.} Доказательство неравенства КБШ для полускалярного произведения отличается только в случае, если $|\left<x, y \right>|^2 \le |\left<x, x \right>| \cdot |\left<y, y \right>|$, но $\left<x, x \right> = 0$ или $\left<y, y \right> = 0$.
В этом случае проще всего доказать по отдельности, что действительная и комплексная части $\left<x, y \right>$ равны нулю.

\section{Компактные операторы}
\epigraph{Лучше есть вещественный кусок хлеба, чем мнимый.}{Сергей Петрович}

\subsection{Введение и основные свойства}

\textbf{Определение.} Пусть $E_1, E_2$ --- нормированные пространства, $A \in \mathcal L(E_1, E_2)$.
Оператор $A$ называется \textit{компактным}, если образ ограниченного множества в $E_1$ является предкомпактом в $E_2$.
Множество компактных операторов обозначается через $\mathcal K(E_1, E_2)$.

\textbf{Упражнение.} Компактность линейного оператора эквивалентна предкомпактности единичного шара.

\textbf{Замечание.} Если же $A$ не является линейным, то для компактности дополнительно нужна непрерывность.
Далее под компактными операторами мы будем подразумевать только линейные.

\textbf{Утверждение 1.} Если $\dim(E_1) < \infty$ или $\dim(E_2) < \infty$, то любой $A \in \mathcal L(E_1, E_2)$ компактен.

\textbf{Доказательство.} Если $E_1$ конечномерно, то образ любого множества конечномерен, значит, предкомпактен.
Если $E_2$ конечномерен, то аналогично.

\QED

\textbf{Утверждение 2.} Если размерность $E$ бесконечна, то $I \not\in \mathcal K(E)$.

\textbf{Доказательство.} По \hyperref[th:4-1]{теореме 4.1} единичный шар компактен тогда и только когда, когда $E$ конечномерно.

\QED

\textbf{Утверждение 3.} $\mathcal K(E)$ --- двусторонний идеал в $\mathcal L(E)$.

\textbf{Доказательство.} Двусторонний идеал означает, что если $A \in \mathcal K(E)$ и $B \in \mathcal L(E)$, то $AB, BA \in \mathcal K(E)$.

Докажем для $AB$: пусть $S \subset E$ ограничено, тогда $B(S)$ ограничено, значит, $A(B(S))$ --- предкомпакт.

Для $BA$: $A(S)$ --- предкомпакт, то есть вполне ограничено, значит, можно выделить $\varepsilon$--сеть.
И под действием $B$ эта $\varepsilon$--сеть перейдёт в $(\|B\| \varepsilon)$--сеть.

\QED

\textbf{Утверждение 4.} Если $\dim(E) = \infty$, $A \in \mathcal K(E)$, то $A^{-1}$ не может быть элементом $\mathcal L(E)$.
Иными словами, компактные операторы, действующие из бесконечномерных пространств, ``теряют информацию``.

\textbf{Доказательство.} От противного: $A^{-1} A = I$.
Так как по утверждению 3 множество $\mathcal K(E)$ является двусторонним идеалом, произведение лежит в $\mathcal K(E)$ --- противоречие с утверждением 2.

\QED

\textbf{Утверждение 5.} (б/д) Если $A \in \mathcal K(E)$, $x_n \xrightarrow{w} x$, то $\|Ax_n - Ax\| \to 0$.
Более того, если $E$ банахово, то верно и обратное.

\textbf{Утверждение 6.} (б/д) Пусть $E$ --- банахово пространство (ещё может потребоваться рефлексивность), $A \in \mathcal K(E)$, тогда сопряжённый оператор $A^* \in \mathcal K(E^*)$.

\label{th:12-1} \textbf{Теорема.} (12.1) Пусть $E_1$ --- нормированное пространство, $E_2$ --- банахово пространство (для эквивалентности предкомпактности и вполне ограниченности), $A_n$ --- последовательность компактных операторов, такая что $\|A_n - A\| \to 0$.
Тогда $A$ компактен.

\textbf{Доказательство.} 
По определению сходимости по норме для любого $\varepsilon > 0$ найдётся $N$, такое что для $n \ge N$ выполнено $\|A_n - A\| < \varepsilon$.
Значит, при $n \ge N$ выполнено $\sup_{\|x\| = 1} \|A_n x - Ax \| < \varepsilon$.
Далее будем искать $\varepsilon$--сеть во множестве $A (\overline B(0, 1))$.

Из компактности получаем, что $A_n \overline B$ является предкомпактным множеством и, что эквивалентно, вполне ограниченным.
Значит, найдётся $\varepsilon$--сеть $\{y_1^n, \dots, y_m^n\}$ для $A_n \overline B$.
Получается, что для любого $x \in \overline B$ найдётся $y_k^n$, такой что $\|A_n x - y_k^n\| < \varepsilon$.
Итак,
\[
    \min_{1 \le k \le m} \|Ax - y_k^n\| \le \min_k \left( \|Ax - A_n x \| + \|A_n x - y_k^n \| \right).
\]
Первое меньше $\varepsilon$ засчёт сходимости по норме, второе --- из существования $\varepsilon$--сети.

\QED

\textbf{Замечание 1.} Здесь можно провести аналогию с тем, что предел непрерывных функций непрерывен.

\textbf{Замечание 2.} Поточечной сходимости $A$ недостаточно, и контрпример придумать несложно.

\textbf{Замечание 3.} Данная теорема и определение --- это два самых популярных способа доказательства компактности.
Например, пусть $(\lambda_n)$ --- ограниченная последовательность, тогда оператор $A: l_2(\mathbb C) \to \mathbb C^*$ компактен тогда и только тогда, когда $\lambda_n \to 0$.

Слева направо --- простое упражнение, а справа налево можно рассмотреть операторы $P_n$, действующие, как $A$, но с нулями после $n$--ой координаты.
Каждый из этих операторов будет компактен, и при $\lambda_n \to 0$ они стремятся к $A$.

\textbf{Определение.} Нормированное пространство $F$ \textit{обладает свойством аппроксимации}, если для любого банахова пространства $E$ любой оператор $A \in \mathcal K(E, F)$ можно представить, как предел операторов конечного ранга.

\textbf{Задача.}$^*$ Если в банаховом сепарабельном пространстве $F$ существует счётный базис (базис Шаудера), то $F$ обладает свойством аппроксимации.

\textbf{Задача.} Гильбертовы пространства обладают свойством аппроксимации.

\textbf{Гипотеза.} (Банаха, 1972) В любой сепарабельном банаховом пространстве существует базис Шаудера.

\textbf{Теорема.} (Per Enflo, б/д) Существует банахово пространство, не обладающее свойством аппроксимации.
Как следствие, гипотеза Банаха неверна.

\subsection{Теорема Фредгольма}

\label{th:12-2} \textbf{Теорема.} (12.2) Пусть $E(\mathbb C)$ --- банахово пространство, $A \in \mathcal K(E)$, $\lambda \in \mathbb C$, $\lambda \ne 0$.
Тогда $\dim(\Ker A_\lambda) < \infty$.

\textbf{Замечание.} $\lambda = 0$ не работает из-за того, что тогда ядром $A_\lambda$ будет просто ядро $A$, и про него ничего не понятно.

\textbf{Доказательство.} Докажем вполне ограниченность $S(0, 1)$, ибо это эквивалентно по следствию из теоремы Рисса--Фишера.
Что эквивалентно, предкомпактность сферы, то есть из любой $\{x_n\} \subset S(0, 1)$ можно выделить сходящуюся подпоследовательность.

Так как $\{Ax_n\}$ --- предкомпакт, из неё можно выделить сходящуюся $Ax_{n_k} \to y$.
Теперь вспомним, что мы живём в ядре $A_\lambda$, значит, все $A_\lambda x_n = 0$.
Итак, $Ax_{n_k} = \lambda x_{n_k} \to y$, засим остаётся поделить на $\lambda \ne 0$ и получить $x_{n_k} \to \frac{1}{\lambda} y$.

\QED

\label{th:12-3} \textbf{Теорема.} (12.3) Пусть $E(\mathbb C)$ --- банахово пространство, $A \in \mathcal K(E)$.
Тогда для любого $\delta > 0$ вне круга $\{|\lambda| \le \delta\}$ может находиться лишь конечное число собственных значений оператора $A$.

\textbf{Доказательство.} Будем доказывать в случае гильбертова $E = H$ и самосопряжённого $A$.
От противного: нашлось $\delta_0$, такое что вне круга собственных значений бесконечно много.
Выделим последовательность $\{\lambda_n\}$, тогда каждому из них можно сопоставить единичный собственный вектор $e_n$.
Так как это собственные векторы, для всех $i \ne j$ векторы $e_i$ и $e_j$ ортогональны.

Итак, рассмотрим последовательность $\{Ae_n\}$ --- предкомпакт.
Для любых $i \ne j$:
\[
    \|Ae_i - Ae_j\|^2 = \|Ae_i\|^2 + \|Ae_j\|^2 = |\lambda_i|^2 + |\lambda_j|^2 \ge 2\delta_0^2.
\]
Следовательно, в $\{Ae_n\}$ не удастся выделить $\varepsilon$--сеть --- противоречие.

\QED

\textbf{Следствие 1.} Для любого $\delta > 0$ выполнено $\sum_{|\lambda| > \delta} \dim(\Ker A_\lambda)) < \infty$.

\textbf{Следствие 2.} Если $A$ компактен, то точечный спектр $\sigma_p(A)$ не более, чем счётный.

\label{th:12-4} \textbf{Теорема.} (Фрéдгольма, 12.4, частный случай)
Пусть $H(\mathbb C)$ --- гильбертово пространство, $A: H \to H$ --- компактный самосопряжённый оператор, $\lambda \in \mathbb C \setminus \{0\}$.
Тогда $\im(A_\lambda) \oplus \Ker(A_\lambda) = H$.

\textbf{Лемма 1.} В условиях теоремы 12.4 если $\lambda \in \sigma(A)$ и $\lambda \ne 0$, то $\lambda \in \sigma_p(A)$.

\textbf{Доказательство.} По \hyperref[th:11-3]{теореме 11.3} $\lambda \in \sigma(A)$ тогда и только тогда, когда найдётся последовательность $\{x_n\}$, такая что $\|x_n\| = 1$ и $A_\lambda x_n \to 0$.
Это значит, что $Ax_n - \lambda x_n \to 0$.
Так как $\{x_n\}$ ограничена и $A$ компактен, можно выделить сходящуюся $A x_{n_k} \to y$.

Вместе с тем, что $Ax_{n_k} - \lambda x_{n_k} \to 0$, получаем, что $\lambda x_{n_k} \to y$.
Значит, $x_{n_k} \to \frac{1}{\lambda} y$.
Возьмём образ $A$: $Ax_{n_k} \to \frac{1}{\lambda} A y$, и в то же время $Ax_{n_k} \to y$ по построению.
Следовательно, $y = \frac{1}{\lambda} Ay$, то есть $y$ --- собственный вектор с собственным значением $\lambda$.

\QED

\textbf{Лемма 2.} В условиях теоремы 12.4 выполнено $\overline{\im A_\lambda} = \im A$.

\textbf{Лемма 3.} (Также известное, как утверждение 1 параграфа 11) Если $A$ --- самосопряжённый оператор, $M$ инвариантно относительно $A$, то $M^\bot$ тоже инвариантно относительно $A$.

\textbf{Доказательство.} (Леммы 2) Зафиксируем $\lambda$.
По \hyperref[th:11-2]{теореме 11.2} выполнено $H = \overline{\im A_\lambda} \oplus \Ker A_\lambda$.
Так как ядро $A_\lambda$ инвариантно относительно $A$, его можно взять в качестве $M$ и применить лемму 3, откуда $\overline{\im A_\lambda}$ инвариантно относительно $A$.

Положим $\tilde A = A|_{\overline{\im A_\lambda}}$ --- компактный самосопряжённый оператор.
Заметим, что $\lambda$ не является собственным значением $\tilde A$, ведь все возможные собственные векторы лежат в $\Ker A_\lambda$.
Следовательно, $\lambda \in \rho(\tilde A)$ и для любого $y \in \overline{\im A_\lambda}$ уравнение $\tilde A_\lambda x = y$ имеет решение.

И на этом доказательство завершается, ибо последнее предложение означает вложение $\overline{\im A_\lambda} \subset \im A_\lambda$.

\QED

\textbf{Доказательство.} (Теоремы Фредгольма) По \hyperref[th:11-2]{теореме 11.2} получаем $H = \overline{\im A_\lambda} \oplus \Ker A_\lambda$.
И по лемме 2 имеем $\overline{\im A_\lambda} = \im A_\lambda$.

\QED

\textbf{Теорема'.} (Альтернатива Фредгольма) Рассмотрим уравнения $A_\lambda x = y$ (1) и $A_\lambda z = 0$ (2).
Тогда либо:
\begin{itemize}
    \item Уравнение (1) имеет решение для всех $y$.
    \item Уравнение (2) имеет нетривиальное решение.
\end{itemize}
Во втором случае уравнение (1) разрешимо тогда и только тогда, когда $y$ ортогонально всем решениям уравнения (2).

\textbf{Доказательство.} Рассмотрим $\lambda$.
Если оно не является собственным значением, то $\Ker A_\lambda = \varnothing$, значит, $H = \im A_\lambda$.

Если же является собственным значением, то ядро непусто, а решения уравнения (1) лежат в ортогональном дополнении к решениям уравнения (2).

\QED

\subsection{Теорема Гильберта--Шмидта}

\label{th:12-5} \textbf{Теорема.} (12.5, Гильберта--Шмидта) Пусть $H(\mathbb C)$ --- сепарабельное гильбертово пространство, $A$ --- компактный самосопряжённый оператор.
Тогда в $H$ найдётся ОНБ, состоящий из собственных векторов оператора $A$.

Эта теорема является обобщением теоремы из линейной алгебры про то, что самосопряжённый оператор можно диагонализовать.

\textbf{Лемма 4.} В условиях теоремы и если $A$ не нулевой, то у $A$ найдётся ненулевое собственное значение.

\textbf{Доказательство.} По \hyperref[th:11-5]{теореме 11.5} имеем $0 \ne \|A\| = \max(|m_-|, |m_+|)$.
Дальше вспоминаем, что $m_-$ и $m_+$ лежат в спектре, поэтому по теореме Фредгольма доказано.

\QED

\textbf{Доказательство.} (Теоремы) По \hyperref[th:12-3]{теореме 12.3} можно упорядочить ненулевые собственные значения $|\lambda_1| \ge |\lambda_2| \ge \dots$ и сопоставить им собственные векторы $e_1 \bot e_2 \bot \dots$.
Получаем ортонормированную систему, но это ещё не всё, ибо мы добавили векторы только для ненулевых собственных значений.
Засим рассмотрим базис $\Ker A$ (найдётся, так как пространство сепарабельно) и добавим его к $\{e_n\}$.
Теперь уже докажем, что получился базис, а для этого достаточно доказать, что $\overline {[\{e_n\}]} = H$, обозначим левую часть за $M$.

По \hyperref[th:4-3]{теореме 4.3} выполнено $H = M \oplus M^\bot$, засим покажем, что $M^\bot = \{0\}$.
Заметим, что $M$ инвариантно относительно $A$ (без замыкания очевидно, а потом рассмотрим элемент замыкания и последовательность, сходящуюся к нему).
Следовательно, по лемме 3 пространство $M^\bot$ инвариантно относительно $A$.

Рассмотрим $\tilde A = A|_{M^\bot}$ --- компактный самосопрядённый оператор.
Допустим, что $M^\bot$ нетривиально.
Далее имеются два случая: $\tilde A$ нулевой и ненулевой.

При $\tilde A = \ominus$: для всех $x \in M^\bot$ выполнено $\tilde Ax = 0 \cdot x$, то есть $x$ --- собственный вектор, но они все должны лежать в $M$ --- противоречие.

При $\tilde A \ne \ominus$: по лемме 4 у него найдётся ненулевое собственное значение, а значит, и собственный вектор --- противоречие.

\QED

\textbf{Следствие.}
Рассмотрим уравнение $A_\lambda x = y$, $\lambda \in \rho(A)$, $A$ --- КССО из теоремы 12.5.
Тогда можно рассмотреть ОНБ из собственных векторов $\{e_n\}$ и разложить $x = \sum (x, e_n) e_n$ и $y = \sum (y, e_n) e_n$.
Значит, в силу ортогональности $\{e_n\}$ можно записать для всех $n$
\[
    \lambda_n (x, e_n) - \lambda (x, e_n) = (y, e_n) \Rightarrow (x, e_n) = \frac{(y, e_n)}{\lambda_n - \lambda}.
\]

\textbf{Следствие.} Пусть $A$ имеет конечное число ненулевых собственных значений $\{\lambda_n\}$.
Тогда $\sigma(A) = \sigma_p(A)$.
То же самое верно, если собственных значений бесконечно много, и ноль является таковым.

\section{Элементы нелинейного анализа}
\epigraph{Слово ``дифференцируема`` нужно сокращать до ``диф.``, а не ``дифф.`` по ГОСТу 7.0.12--2011.}{Сергей Петрович}
\subsection{Дифференциал}
Пусть $E_1, E_2$ --- банаховы пространства, $D \subset E_1$ открыто, $x_0 \in D$, $F: D \to E_2$.

\textbf{Определение.} $F$ \textit{дифференцируема по Фреше} в точке $x_0$, если найдётся $A \in \mathcal L(E_1, E_2)$, такой что
\[
    F(x_0 + h) - F(x_0) = Ah + o(\|h\|).
\]
$A$ называется \textit{производной (Фреше)} отображения $F$ в точке $x_0$ и обозначается $A = F'(x_0)$.

\textit{Дифференциалом} называется оператор, действующий следующим образом:
\[
    \dif F(x_0, h) = F'(x_0) h = Ah.
\]

\textbf{Утверждение.} 
\begin{itemize}
    \item Если $F = const$, то $F' \equiv 0$.
    \item Если $F \in \mathcal L(E_1, E_2)$, то $F = F'$.
    \item Линейность.
\end{itemize}

\textbf{Определение.} \textit{Производной по направлению} называется 
\[
    \Dif F(x_0, h) = \frac{\dif}{\dif t} F(x_0 + th)\bigg|_{t = 0}.
\]
Если существует $A \in \mathcal L(E_1, E_2)$, такой что $\Dif F(x_0, h) = Ah$, то $F$ называется \textit{дифференцируемой по Гато} (Gateaux).

\textbf{Теорема.} (13.1, О дифференцируемости композиции) Пусть $X, Y, Z$ --- линейные нормированные пространства, $F: X \to Y$ дифференцируема в $x_0 \in X$, $G: Y \to Z$ --- в точке $y_0 = F(x_0) \in Y$.
Тогда $H := G \circ F$ дифференцируема в точке $x_0$ и $H'(x_0) = G'(F(x_0)) \cdot F'(x_0)$.

\textbf{Доказательство.} Заметим, что $F$ и $G$ непрерывны в $x_0$ и $y_0$ соответственно.
Теперь напишем определения дифференцируемости, начиная расписывать, как и полагается в матрёшках, с внешней.
\[
    \Delta z = G'(y_0) \Delta y + \varepsilon_1(\Delta y) \|\Delta y\|,
\]
где $\varepsilon_1 \to 0$ при $\Delta y \to 0$.
\[
    \Delta y = F'(x_0) \Delta x + \varepsilon_2(\Delta x) \|\Delta x\|,
\]
где $\varepsilon_2 \to 0$ при $\Delta x \to 0$.

Теперь мы хотим представить $\Delta z$ в виде $H'(x_0) \Delta x + o(\Delta x)$.
Для этого подставим $\Delta y$ в формулу для $\Delta z$:
\[
    \Delta z = G'(y_0) \big( F'(x_0) \Delta x + \underbrace{\varepsilon_2(\Delta x) \|\Delta x\|}_{(1)} \big) + \underbrace{\varepsilon_1(\Delta y) \|\Delta y\|}_{(2)}.
\]
Заметим, что $(1)$ --- это $o(\Delta x)$, ибо $G'(y_0)$ ограничена.
Теперь $(2)$: разделим на $\|\Delta x\|$ и возьмём норму:
\[
    \frac{\|\varepsilon_1(\Delta y) \| \cdot \|\Delta y\|}{\|\Delta x\|} = \|\varepsilon_1(\Delta y) \| \cdot \frac{\bigg\|F'(x_0) \Delta x + \varepsilon_2(\Delta x) \|\Delta x\| \bigg\|}{\|\Delta x\|} \le
\]
\[
    \le \|\varepsilon_1(\Delta y) \| \cdot \frac{\|F'(x_0) \| \cdot \|\Delta x\| + \|\varepsilon_2(\Delta x) \| \cdot \|\Delta x\|}{\|\Delta x\|}.
\]
Итак, произведение бесконечно малой $\|\varepsilon_1\|$ на ограниченную --- это бесконечно малая, что завершает доказательство.

\QED

\textbf{Определение.} \textit{Областью} называется открытое \underline{выпуклое} множество.

\textbf{Теорема.} (13.2, О среднем) Пусть $E_1$, $E_2$ --- вещественные банаховы пространства, $D \subset E_1$ --- область, $x_0, x_1 \in D$.
Положим $x(t) = x_0 + t(x_1 - x_0)$ для $t \in [0, 1]$.
Пусть $F: D \to E_2$ дифференцируема.
Тогда
\[
    \|F(x_1) - F(x_0)\| \le \sup_{t \in (0, 1)} \|F'(x(t))\| \cdot \|x_1 - x_0\|.
\]

\textbf{Доказательство.} Положим $\phi(t) = f(F(x(t)))$ для $f \in E_2^*$, который определим позже, тогда по теореме о композиции оно дифференцируемо, и $\phi'(t) = f(F'(x_0 + t \Delta x) \Delta x)$, где $\Delta x = x_1 - x_0$.

Применим к $\phi$ теорему Лагранжа:
\[
    |\phi(1) - \phi(0)| \le \|f\| \cdot \sup_{t \in (0, 1)} \|F'(x(t))\| \cdot \|\Delta x\|.
\]
Слева получаем $|f(F(x_1) - F(x_0))|$, поэтому можно по следствию 2 из теоремы Хана--Банаха взять функционал $f$, такой что $\|f\| = 1$ и $f(F(x_1) - F(x_0)) = \|F(x_1) - F(x_0)\|$.
Заметим, что мы получили ровно то, что нужно.

\QED

\textbf{Замечание.} Банаховостью мы нигде не пользовались, её потребовали только для простоты формулировки.

\subsection{Теоремы о стационарных точках}
\textbf{Теорема.} (Брауэра, б/д) Пусть $\overline B \subset \mathbb R^n$ --- замкнутый шар, $F: \overline B \to \overline B$ непрерывна.
Тогда найдётся $x \in B$, такая что $F(x) = x$.

\textbf{Пример.} (Задача Перрона--Фробениуса)
Имеется матрица $A$, все элементы которой больше нуля.
Доказать, что существует собственный вектор, все элементы которого положительные.

План решения:
\begin{itemize}
    \item Вводим множество $K = \{x \in \mathbb R^n~|~\text{все $x_i > 0$}\}$.
    \item $A(K \setminus \{0\}) \subset \operatorname{Int} K$.
    \item Положим $P = \{x~|~\sum_{i} x_i = 1\}$.
    \item Вводим $D = K \cap P$.
        Тогда на $D$ по теореме Брауэра отображение $F$ на нём имеет неподвижную точку.
\end{itemize}

\section{Преобразование Фурье и свёртка в $L_1(\mathbb R)$ и $L_2(\mathbb R)$}
\epigraph{
--- Как по-английски будет свёртка?

--- Convolution.

--- А какое у этого слова значение?

--- Свёртка.
}{}
В данном параграфе мы не будем различать классы эквивалентности функций в $L_1, L_2$ и их представителей.
На всякий случай: $\hat f'$ --- производная преобразования Фурье, $\widehat{f'}$ --- преобразование Фурье производной.

\subsection{Преобразование Фурье в $L_1(\mathbb R)$}
\textbf{Определение.} Пусть $f \in L_1$.
\textit{Преобразованием Фурье} называется 
\[
    \hat f(y) = (Ff)(y) = \int_{-\infty}^{+\infty} f(x) e^{-ixy} \dif x.
\]

\textbf{Воспоминания.} В гармоническом анализе доказывалось, что:
\begin{itemize}
    \item Если $f, f' \in L_1$, то $\widehat{f'}(y) = iy \cdot \hat f(y)$.
    \item Если $f, (x \mapsto xf(x)) \in L_1$, то $\hat f'(y) = -i \cdot \widehat{x f(x)}(y)$.
    \item Существует пространство Шварца $S$ функций из $C^\infty(\mathbb R)$, у которых все степени производных убывают на бесконечности быстрее любого многочлена.
        Также здесь преобразование Фурье $F: S \to S$ является изоморфизмом.
\end{itemize}
Далее мы будем исследовать свойства оператора $F: L_1(\mathbb R) \to~??$.

\textbf{Утверждение.} ($L_1$Ф.2)
Если $f_n \to f$ в $L_1$, то $\|\hat f_n - \hat f\|_{B(\mathbb R)} \to 0$, то есть $\hat f_n \rightrightarrows \hat f$.
Здесь $\|f\|_{B(\mathbb R)} = \sup_x |f(x)|$.

\textbf{Доказательство.} Нетрудно заметить, что $\|\hat f\|_B \le \|f\|_{L_1}$.
Следовательно, $F$ является линейным ограниченным оператором.

\QED

\textbf{Утверждение.} ($L_1$Ф.1)
Пусть $f \in L_1$.
Тогда $\hat f \in C(\mathbb R)$ и $\hat f(y) \to 0$ при $y \to \pm \infty$.

\textbf{Доказательство.} Вспоминая построение интеграла Лебега, докажем для ступенчатых функций.
Пусть $f_a(x) = a I\{|x| \le R\}$, тогда $\hat f_a(y) = \frac{2a \sin (Ry)}{y}$ --- свойства выполнены.
Значит, выполнено для линейной комбинации, и, по утверждению $L_1$Ф.2, выполнено при предельном переходе.

\QED

\textbf{Утверждение.} ($L_1$Ф.3, формула умножения) Пусть $f, g \in L_1$, тогда
\[
    \int_{-\infty}^{+\infty} \hat f(x) g(x) \dif x = \int_{-\infty}^{+\infty} f(x) \hat g(x) \dif x.
\]

\textbf{Доказательство.} Так как $\|\hat f \cdot g\|_{L_1} \le \|f\| \cdot \|g\|$, можно применить теорему Фубини.

\QED

Итак, мы получили, что преобразование Фурье действует в $C_0(\mathbb R) \subset B(\mathbb R)$.
Однако $C_0$ и $L_1$ не вложены друг в друга, что не очень удобно.
Какие у него есть ещё свойства?
\begin{itemize}
    \item Образ --- непонятно.
    \item Ядро тривиально, но доказать сложно.
    \item Компактность отсутствует.
\end{itemize}

\subsection{Свёртка}

\textbf{Определение.} Пусть $f, g \in L_1(\mathbb R)$. \textit{Свёрткой} называется
\[
    (f * g)(x) = \int_{-\infty}^{+\infty} f(\xi) g(x - \xi) \dif \xi.
\]

\textbf{Утверждение.} ($L_1$Св.1) Для всех $x$ выполнено $(f * g)(x) = (g * f)(x)$.
Очевидно, теорема Фубини.

\textbf{Утверждение.} ($L_1$Св.2) $\widehat{f * g} = \hat f \cdot \hat g$.
Очевидно, теорема Фубини.

\textbf{Утверждение.} ($L_1$Св.3) Пусть $f \in C^n$ финитная, $g$ локально интегрируема.
Тогда $f * g \in C^n$ и $\Dif^n(f * g) = (\Dif^n f * g)$.
Иными словами, свёртка даёт настолько хорошую функцию, насколько хорошая лучшая из функций в аргументах.
Без доказательства, можно найти в Зориче.

\textbf{Замечание.} (О пользе свёртки) В гармоническом анализе была теорема Вейерштрасса о приближении непрерывной на отрезке функции многочленами, но традиционно она доказывается неестественно через теорему Фейера.

Можно посчитать, что $\hat \delta = 1$ и $f * \delta = f$.
Поэтому если мы найдём многочлены $f_n \to \delta$, то $f_n * f \to \delta * f = f$, то есть $f_n * f$ --- многочлен, стремящийся к $f$.
Примером таких многочленов является $\frac{(1 - x^2)^n}{I_n}$, где $I_n$ --- интеграл числителя.

\subsection{Преобразование Фурье в $L_2(\mathbb R)$}
Проблема $L_2(\mathbb R)$ заключается в том, что преобразование Фурье не обязано существовать.
Засим тут имеются разные способы определения:
\begin{enumerate}
    \item Как предел при $R \to +\infty$ интеграла
        \[
            \frac{1}{\sqrt{2\pi}} \int_{-R}^R f(x) e^{-ixy} \dif x.
        \]
        Однако он не очень идиоматичен с точки зрения функционального анализа.

    \item Заметим, что пространство Шварца $S$ является линейным многообразием в $L_2$, а также $\overline S = L_2$ (б/д).
        Теперь, пользуясь тем, что $FS = S$, по \hyperref[th:5-3]{теореме 5.3} оператор $F$ можно продолжить на $L_2$.
\end{enumerate}

Мы пойдём по второму пути, засим явно напишем все необходимые свойства преобразования Фурье на $S$ из гармонического анализа:

\textbf{Лемма 1.} $F: S \to S$ --- это биекция, причём обратный оператор имеет вид
\[
    (F^{-1} g)(x) = \frac{1}{\sqrt{2\pi}} \int_{-\infty}^{+\infty} g(y) e^{ixy} \dif y.
\]

\textbf{Доказательство.} Б/д, но надо знать из гармонического анализа.

\QED

\textbf{Лемма 2.} Преобразование Фурье является изометрией: для всех $f, g \in S$ выполнено $(f, g) = (\hat f, \hat g)$.

\textbf{Доказательство.} Просто распишем и воспользуемся формулой обращения:
\[
    (f, g) = \int_{-\infty}^{+\infty} f(x) \overline{g(x)} \dif x = \frac{1}{\sqrt{2\pi}} \int_{-\infty}^{+\infty} f(x) \dif x \overline{\int_{-\infty}^{+\infty} \hat g(x) e^{ixy} \dif y} =
\]
\[
    = \frac{1}{\sqrt{2\pi}} \int_{-\infty}^{+\infty} f(x) \dif x \int_{-\infty}^{+\infty} \overline{\hat g(x)} e^{-ixy} \dif y =
\]
(По теореме Фубини)
\[
    = \int_{-\infty}^{+\infty} \overline{\hat g(y)} \dif y \left( \frac{1}{\sqrt{2\pi}} \int_{-\infty}^{+\infty} f(x) e^{-ixy} \dif x \right) = (\hat f, \hat g).
\]

\QED

\textbf{Замечание.} Здесь можно начать задумываться над тем, что формально равенства тут почти всюду, но мы это опускаем.

\textbf{Теорема.} Преобразование Фурье в $L_2$ сохраняет скалярное произведение.

\textbf{Доказательство.} Довольно очевидно: берём $f_n \to f$ и $g_n \to g$ из $S$, тогда $(\hat f_n, \hat g_n) = (f_n, g_n)$, и далее по непрерывности скалярного произведения.

\QED

\textbf{Свойство 1.} $\im F = L_2$.

\textbf{Доказательство.} Пусть $g \in L_2$, тогда найдётся $\{g_n\} \subset S$, такая что $g_n \to^{L_2} g$.
У них есть прообразы, поэтому найдётся $\{f_n\} \subset S$, такая что $\hat f_n = g_n$.
Теперь для всех $n < m$ выполнено $\|f_n - f_m\|_{L_2} = \|\hat f_n - \hat f_m\|_{L_2}$ по лемме 2 --- по фундаментальности $\hat f_n$ последовательность $f_n$ тоже фундаментальна.
Следовательно, $f_n$ сходится к какой-то функции $f$, значит, $\hat f_n$ сходится к $\hat f$.

\QED

\textbf{Свойство 2.} $\Ker F = \{0\}$.

\textbf{Доказательство.} По теореме о проекции $\overline{\im F} \oplus \Ker F^* = L_2$.
Но что такое $F^*$?
Аккуратным расписыванием теоремы Фубини можно получить, что $F^* = F^{-1}$, поэтому, как следует из формулы, $\Ker F = \Ker F^{-1} = \{0\}$.

\QED

\textbf{Свойство 3.} $F$ не является самосопряжённым оператором.
Как написано выше, $F^* = F^{-1}$.

\textbf{Свойство 4.} $F$ не компактен.
Следует из следующего свойства.

\textbf{Свойство 5.} $F^4 = I$.

\textbf{Доказательство.} На $S$ можно посчитать, что $(F^2 \phi)(y) = \phi(-y)$.
Следовательно, $F^4 = I$ на $S$ и можно продолжить на $L_2$.

\QED

\textbf{Свойство 6.} $\sigma(F) = \{\pm 1, \pm i\}$.

\textbf{Доказательство.} Следует из того, что $\{H_n(x) e^{-\frac{x^2}{2}}\}$ --- это ОНБ в $L_2$, где $H_n$ --- многочлены Эрмита.
Подробнее в Колмогорове--Фомине.

\QED

\subsection{Преобразование Фурье на $S'$}
\textbf{Определение.} $S'$ --- класс линейных непрерывных функционалов на $S$.

Для $f \in S'$ и $\phi \in S$ выполнено
\[
    \hat f(\phi) = \int \phi(x) \hat f(x) \dif x = \int \hat \phi f \dif x
\]
по формуле умножения.
Поэтому можно определить $F$ на $S'$ естественным образом: для $(\hat f, \phi) = (f, \hat \phi)$.

\textbf{Замечание.} А можно ли так определить на $D$?
Нет, потому что преобразование Фурье может вывести за $D$.
Проблема в том, что для $\phi \in L_1$ у нас есть формула
\[
    \hat \phi(x) = \int_{-\infty}^{+\infty} \phi(x) e^{-ixz} \dif x,
\]
в которой подынтегральная функция $\phi(x) e^{-ixz}$ является аналитической, что не является обобщённой функцией.
Вспоминая ТФКП, мы можем написать интеграл по контуру
\[
    \int_{\Gamma} \phi e^{-ixz} \dif x = \int_{-R}^R \phi(x) e^{-ixz} \dif x = 0.
\]

\section{Не вошедшее ранее}
\epigraph{
--- Чтобы выросла яблоня, какое семечко нужно посадить?

--- Семечно яблони.

--- Спасибо! Вы мудры.
}{}
\subsection{Оператор Вольтера}
\textbf{Определение.} \textit{Оператором Вольтера} называется $V: L_2[0, 1] \to L_2[0, 1]$, 
\[
    (Vf)(x) = \int_0^x f(t) \dif t.
\]
На подынтегральное можно смотреть, как на произведение $f(t)$ и $K(x, t)$, где $K$ --- ядро оператора, индикатор $t \le x$.

\textbf{Определение.} Оператор $A$ называется \textit{оператором Гильберта--Шмидта}, если существует $K \in L_2([a, b]^2)$, такой что
\[
    (Af)(x) = \int_a^b K(x, t) f(t) \dif t.
\]

\textbf{Замечание.} Докажем корректность определения.
Заметим, для почти всех $x$ функция $t \mapsto K(x, t)$ лежит в $L_2$, поэтому по неравенству КБШ образ лежит в $L_2$.
Отсюда же следует и ограниченность.

\textbf{Теорема.} Все операторы Гильберта--Шмидта компактны.

\textbf{Доказательство.} Как известно, в $L_2([a, b])$ есть ОНБ $\{\phi_i\}$.
Более того, в $L_2([a, b]^2)$ есть ОНБ $\{\phi_i(x) \cdot \phi_j(t)\}$.
Разложим $K$ по нему: 
\[
    K(x, t) = \sum_{i,j} a_{i,j} \phi_i(x) \phi_j(t).
\]
Возьмём конечные суммы:
\[
    K_N(x, t) = \sum_{i,j=1}^N a_{i,j} \phi_i(x) \phi_j(t)
\]
--- компактные операторы, как конечномерные операторы.
Аналогично определим $A_N$, которые тоже компактны.

Тогда $\|A_N - A\| \le \|K_N - K\| \to 0$, следовательно, $A$ компактен, как предел компактных.

\QED

\textbf{Задача.} Пусть $H$ --- сепарабельное гильбертово пространство, $\{e_n\}$ --- ОНБ, $A \in \mathcal L(H)$.
Если $\sum_n \|Ae_n\|^2 < \infty$, то $A$ компактен.

\subsection{Задача 14.3}
Рассматривается краевая задача
\[
    \begin{cases}
        y'' + \lambda \sin y = f(x) \\
        y(0) = y(1) = 0
    \end{cases}
\]
для $x \in (0, 1)$, $\lambda \in \mathbb R$, $f \in C[0, 1]$.
Доказать, что решение существует.

Решается рассмотрением функции Грина и теоремой Брауэра.

\subsection{Теорема Пеано}
Рассматривается задача
\[
    \begin{cases}
        y' = f(t, y) \\
        y|_{t = t_0} = y_0
    \end{cases} .
\]
для $f \in C$.
Решение существует в достаточно малой окрестности $t_0$.

Для доказательства рассматриваются ломаные с вершиной в $t_0$, после этого достраиваем их, как отрезки с наклоном $y'(y_0) = f(t_0, y_0)$.
Далее множество ломаных оказывается предкомпактом по теореме Арцела--Асколи.

Более подробно в учебнике Сергеева ``Дифференциальные уравнения, курс лекций``.

\section{Древнегреческие музы и их связь с интегралом Лебега}
\epigraph{Я могу тратить время впустую. Вы --- нет.}{Сергей Петрович}

\textbf{Определение.} \textit{Музы} --- это богини в древнегреческой мифологии, выполняющие роль покровительнец исскуств и наук.
Всего их имеется девять штук:
\begin{itemize}
    \item Каллиопа --- эпическая поэзия.
    \item Эвтерпа --- лирическая поэзия.
    \item Мельпомена --- трагедия.
    \item Талия --- комедия.
    \item Эрато --- любовная поэзия.
    \item Полигимния --- гимны.
    \item Терпсихора --- танцы.
    \item Клио --- история.
    \item Урания --- астрономия.
\end{itemize}

Однако в таком виде запомнить их сложно, поэтому можно построить \textit{зиккурат} муз, упорядочив их по высотам аналогично построению интеграла Лебега:

\begin{figure}[ht]
    \centering
    \incfig{muses}{0.75\linewidth}
    \caption{Зиккурат муз}
\end{figure}
\end{document}
