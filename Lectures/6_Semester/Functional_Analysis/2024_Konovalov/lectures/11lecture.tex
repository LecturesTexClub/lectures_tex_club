\begin{proof} (теоремы)
	Построим нужный базис явным образом. Для этого упорядочим все собственные ненулевые значения оператора $A$ по модулю, причём включим в этот ряд копии этих значений столько раз, сколько соответствует размерности их собственного подпространства (в силу теоремы \ref{compact_finite_dim} это возможно). Получим ряд:
	\[
		|\lambda_1| \ge |\lambda_2| \ge |\lambda_3| \ge \ldots
	\]
	Пусть $v_n$ --- нормированный собственный вектор, соответствующий $\lambda_n$ (для $\lambda_n = \lambda_m$, очевидно, берём ортонормированные вектора базиса подпространства). Образуем ортонормированную систему $\{e_n\}_{n = 1}^\infty$, полученную перенумерованием (при необходимости) векторов $v_n$ и добавлением собственных векторов, соответствующих $\lambda = 0$ (в случае, если оно является собственным значением). Так как мы находимся в сепарабельном пространстве, то для того, чтобы эта система была базисом, достаточно доказать её полноту. Обозначим $M = \cl([\{e_n\}_{n = 0}^\infty])$. Коль скоро это подпространство, можно применить теорему о проекции:
	\[
		M \oplus M^\bot = H
	\]
	Стало быть, $M = H$ тогда и только тогда, когда $M^\bot = \{0\}$. Покажем, что $M^\bot$ инвариантен относительно $A$. В силу самосопряженности $A$, достаточно это доказать для просто $M$. Введём дополнительное обозначение $L = [\{e_n\}_{n = 0}^\infty]$. Тогда $AL \subseteq L$ тривиальным образом. При этом оператор $A$ компактен, а значит непрерывен, то есть $AM = A(\cl(L)) \subseteq \cl(AL) \subseteq \cl(L) = M$:
	\[
		\forall v_n \in L, \lim_{n \to \infty} v_n = v \in M \Lora \lim_{n \to \infty} Av_n = Av, Av_n \in AL \subseteq L \Lora Av \in M
	\]
	Исследуем $\wdt{A} = A|_{M^\bot}$. Возможно 2 случая:
	\begin{itemize}
		\item $\wdt{A} = 0$. Этот факт можно записать следующим образом:
		\[
			\forall x \in M^\bot\ \wdt{A}x = 0 \Ra x \in \ker \wdt{A}
		\]
		Стало быть, $M^\bot \subseteq \ker \wdt{A}$. Но так как мы рассмотрели сужение $A$ на $M^\bot$, то по определению $M$ мы оставили $\ker A \bs \{0\}$ за бортом, то есть $\ker \wdt{A} = \{0\} = M^\bot$.
		
		\item $\wdt{A} \neq 0$. Предположим противное: $M^\bot \neq \{0\}$ (это очень важный шаг! Иначе нельзя воспользоваться леммой в силу того, что мы рассматриваем оператор на тривиальном пространстве). По доказанной лемме, у $\wdt{A}$ существует ненулевое собственное значение $\lambda$. Обозначим за $e$ --- соответствующий нормированный собственный вектор, то есть $\wdt{A}e = \lambda e$, но ведь тогда и $Ae = \lambda e$. Получили противоречие с определением $M$.
	\end{itemize}
\end{proof}

\begin{note}
	Пусть $H$ --- гильбертово сепарабельное пространство над $\Cm$, $A \in K(H)$ --- компактный самосопряжённый оператор. Из всего доказанного следуют следующие факты:
	\begin{enumerate}
		\item $\forall x \in H\ \ Ax = \sum_{\lambda_n \neq 0} \lambda_n(x, e_n)e_n$
		
		\item $\forall \lambda \in \rho(A) \ps{A_\lambda x = y \Ra (x, e_n) = \frac{(y, e_n)}{\lambda_n - \lambda}}$. Если обозначить $P_n$ --- оператор проекции на $e_n$, то $x = R_\lambda y = \sum_{\lambda_n \neq 0} \frac{1}{\lambda_n - \lambda} P_ny$
		
		\item Если нужно решить уравнение $A_\lambda x = y$, $\lambda \in \sigma_p(A)$, то возможно 2 ситуации:
		\begin{itemize}
			\item $\lambda \neq 0$. Тогда $A_\lambda x = 0$ имеет конечный базис решений. Чтобы получить общее решение $A_\lambda x = y$, нужно найти частное решение и сложить его с комбинацией этой системы (формула координат из предыдущего пункта верна, но при $\lambda_n = \lambda$ просто имеется неоднозначность, которую закрывает базис решений однородной части).
			
			\item $\lambda = 0$
		\end{itemize}
		
		\item Структура $\sigma(A)$: $0 \in \sigma(A)$. Возможно только 2 случая:
		\begin{enumerate}
			\item $\{\lambda_n\}_{n = 1}^\infty \subseteq \sigma_p(A) \bs \{0\}$ --- бесконечное число собственных ненулевых значений. Тогда либо $0 \in \sigma_p(A)$, либо $0 \in \sigma_c(A)$
			
			\item $\{\lambda_n\}_{n = 1}^\infty \subseteq \sigma_p(A) \bs \{0\}$ --- конечное число собственных ненулевых значений. Тогда $0 \in \sigma_p(A)$, причём это собственное значение бесконечной кратности
		\end{enumerate}
	\end{enumerate}
\end{note}

\textcolor{red}{Возможно, добавить историческую справку про уравнение колебаний}