\begin{note}
	Условие теоремы эквивалентно тому, что
	\[
		\forall \lambda \in \Cm\ \ \im \lambda \neq 0 \Ra \lambda \in \rho(A)
	\]
\end{note}

\begin{proof}
	Воспользуемся эквивалентной формулировкой теоремы. Пусть $\lambda = \mu + i\nu$, $\nu \neq 0$.
	\begin{lemma}
		При всех условиях параграфа, имеет место неравенство
		\[
			\|A_\lambda x\|^2 \ge \|\nu\|^2 \|x\|^2
		\]
	\end{lemma}
	
	\begin{proof}
		Заметим, что $A_\lambda = A - \lambda I = A - (\mu + i\nu)I = A_\mu - i\nu I$. Так как речь идёт о квадрате нормы, то мы можем расписать её через скалярное произведение:
		\[
			\|A_\lambda x\|^2 = (A_\lambda x, A_\lambda x) = \|A_\mu x\|^2 - i\nu(x, A_\mu x) + i\nu(A_\mu x, x) + \|\nu\|^2\|x\|^2
		\]
		Так как $\mu \in \R$, то $A_\mu$ --- самосопряжённый оператор. Стало быть, мы можем сократить слагаемые в середине. Тогда:
		\[
			\|A_\lambda x\|^2 = \|A_\mu x\|^2 + \|\nu\|^2\|x\|^2 \ge \|\nu\|^2\|x\|^2
		\]
	\end{proof}
	По доказанной лемме, $A_\lambda$ ограничен снизу. В силу критерия принадлежности спектру, такое возможно лишь в том случае, когда $\lambda \in \rho(A)$.
\end{proof}

\begin{note}
	Для резольвенты можно оценить норму:
	\[
		\forall \lambda \in \rho(A)\ \ \|R_\lambda\| \le \frac{1}{\|\nu\|}
	\]
	Действительно, $\|A_\lambda x\| \ge \|\nu\| \cdot \|x\|$ по доказанному. Если мы сделаем замену $x = A_\lambda^{-1}y$, то получим $\|A_\lambda^{-1}y\| \le \frac{1}{\|\nu\|}\|y\|$, откуда $\|A_\lambda^{-1}\| \le \frac{1}{\|\nu\|}$, при этом $R_\lambda(A) = A_\lambda^{-1}$.
\end{note}

\begin{theorem} \label{sao_spectre_bound}
	Обозначим при всех условиях параграфа, $m_- = \inf_{\|x\| = 1} (Ax, x)$ и $m_+ = \sup_{\|x\| = 1} (Ax, x)$. Тогда
	\begin{enumerate}
		\item $\sigma(A) \subseteq [m_-; m_+]$, причём $m_-, m_+ \in \sigma(A)$
		
		\item $\|A\| = r(A) = \max(|m_-|, |m_+|)$
	\end{enumerate}
\end{theorem}

\begin{proof}~
	\begin{enumerate}
		\item Покажем, что если $\lambda > m_+$, то $\lambda \in \rho(A)$. Будем снова ограничивать $\|A_\lambda x\|$ снизу. С одной стороны, по КБШ:
		\[
			|(A_\lambda x, x)| \le \|A_\lambda x\| \cdot \|x\| \Ra \|A_\lambda x\| \ge \frac{1}{\|x\|}|(A_\lambda x, x)|
		\]
		С другой стороны, распишем скалярное произведение:
		\begin{multline*}
			|(A_\lambda x, x)| = |(Ax, x) - \lambda(x, x)| =
			\\
			\Big[|(Ax, x)| \le m_+\|x\|^2 \wedge \lambda(x, x) > m_+\|x\|^2\Big] =
			\\
			\lambda\|x\|^2 - (Ax, x) \ge (\lambda - m_+)\|x\|^2
		\end{multline*}
		Отсюда сразу $\lambda \in \rho(A)$. Теперь докажем, что $m_+ \in \sigma(A)$. Для этого воспользуемся критерием принадлежности спектру (предъявим последовательность). В силу определения $m_+$:
		\[
			\exists \{x_n\}_{n = 1}^\infty \subseteq H \colon \|x_n\| = 1 \wedge \lim_{n \to \infty} (Ax_n, x_n) = m_+
		\]
		Надо показать предел $\lim_{n \to \infty} \|A_{m_+}x_n\| = 0$. Так как норма векторов единична, то текущий предел можно переписать в следующем виде:
		\[
			\lim_{n \to \infty} (Ax_n, x_n) - m_+ = 0 = \lim_{n \to \infty} (Ax_n, x_n) - m_+(x_n, x_n) = \lim_{n \to \infty} ((A - m_+I)x_n, x_n) = 0
		\]
		Также из определения $m_+$ следует, что $A_{m_+}$ --- отрицательно полуопределенный оператор. Так как неравенство КБШ справедливо для скалярных произведений, порождённых положительными полуопределёнными операторами, то перейдём к $B = -A_{m_+}$. Чтобы получить требуемое, нам достаточно показать предел $\lim_{n \to \infty} Bx_n = 0$. Запишем четвёртую степень нормы следующим образом:
		\[
			\|Bx_n\|^4 = |(x_n, Bx_n)_B|^2 \le |(x_n, x_n)_B| \cdot |(Bx_n, Bx_n)_B| = |(Bx_n, x_n)| \cdot |(B^2x_n, Bx_n)|
		\]
		Первый сомножитель стремится к нулю, а второй ограничен:
		\[
			|(B^2x_n, Bx_n)| \le \|B^2x_n\| \cdot \|Bx_n\| \le \|B\|^3 \cdot \|x_n\|^2 = \|B\|^3
		\]
		Поэтому требуемый предел установлен. Доказательство для $m_-$ аналогично.
		
		\item Из формулы спектрального радиуса
		\[
			r(A) = \lim_{n \to \infty} \sqrt[n]{\|A^n\|}
		\]
		\begin{lemma}
			При всех условиях параграфа, для $n = 2^k$ верно равенство $\|A^n\| = \|A\|^n$
		\end{lemma}
		
		\begin{proof}
			Достаточно доказать, что $\|A^2\| = \|A\|^2$.
			\begin{itemize}
				\item[$\le$] Воспользуемся неравенством для ограниченных операторов:
				\[
					\|A^2x\| = \|A(Ax)\| \le \|A\| \cdot \|Ax\| \le \|A\|^2 \cdot \|x\| \Ra \|A^2\| \le \|A\|^2
				\]
				
				\item[$\ge$] Распишем квадрат нормы $\|Ax\|^2$:
				\[
					\|Ax\|^2 = (Ax, Ax) = (x, A^2x) \le \|x\| \cdot \|A^2\| \cdot \|x\|
				\]
				Осталось взять супремум от обеих частей неравенства:
				\[
					\|A\|^2 = \sup_{\|x\| = 1} \|Ax\|^2 \le \sup_{\|x\| = 1} \|A^2\| \cdot \|x\|^2 = \|A^2\|
				\]
			\end{itemize}
		\end{proof}
		Так как предел в формуле существует, то достаточно найти любой частичный предел, чтобы узнать итоговый. Лемма как раз объясняет, какую подпоследовтельность надо рассматривать.
	\end{enumerate}
\end{proof}

\section{Компактные операторы}

\begin{note}
	В этом параграфе, если не сказаного иного, $E_1, E_2$ --- линейные нормированные пространства, $A \in \cL(E_1, E_2)$.
\end{note}

\begin{definition}
	Оператор $A$ называется \textit{компактным}, если
	\[
		\forall M \subseteq E_1 \text{ --- ограниченное} \Ra A(M) \subseteq E_2 \text{ --- предкомпакт}
	\]
	Множество компактных операторов обозначается как $K(E_1, E_2)$
\end{definition}

\begin{proposition}
	Оператор $A$ компактен тогда и только тогда, когда $A(B(0, 1))$ --- предкомпакт.
\end{proposition}

\begin{proof}~
	\begin{itemize}
		\item[$\Ra$] Очевидно, коль скоро множество $B(0, 1)$ ограничено.
		
		\item[$\La$] Рассмотрим произвольное ограниченное множество $M \subseteq E_1$. Коль скоро $A$ линеен, то можно проверять предкомпактность не $A(M)$, а $A(M')$, где $M'$ получен сжатием $M$ (то есть $x \mapsto \alpha x$, $\alpha > 0$). Тогда, рассмотрим $M' \subseteq B(0, 1)$. Отсюда $A(M') \subseteq A(B(0, 1))$, а значит $M'$ предкомпактно (замкнутое подмножество $\cl A(M')$ компакта $\cl A(M)$ тоже компактно).
	\end{itemize}
\end{proof}

\begin{proposition}
	Имеют место следующие утверждения:
	\begin{enumerate}
		\item $\dim E_1 < \infty \Ra \cL(E_1, E_2) = K(E_1, E_2)$
		
		\item $\dim E_2 < \infty \Ra \cL(E_1, E_2) = K(E_1, E_2)$
	\end{enumerate}
\end{proposition}

\begin{proof}
	Достаточно понимать, что в конечномерном пространстве любое замкнутное ограниченное множество компактно.
	\begin{enumerate}
		\item Коль скоро $\dim \im A \le \dim E_1 < \infty$, то образ любого ограниченного множества оказывается ограниченным множеством в подпространстве $\im A$ конечной размерности.
		
		\item Сразу следует из исходного заявления в доказательстве.
	\end{enumerate}
\end{proof}

\begin{reminder}
	Пусть $R$ --- кольцо, $I$ --- подгруппа $(R, +)$. Тогда $I$ называется \textit{левосторонним идеалом}, если $I$ обладает свойством \textit{поглощения слева}:
	\[
		\forall r \in R\ \forall a \in I\ ra \in I
	\]
	Аналогично определяется правосторонний идеал. Ну и $I$ --- (двухсторонний) идеал, если он является и левосторонним, и правосторонним идеалом.
\end{reminder}

\begin{proposition}
	Пусть $E_1 = E_2 = E$. Тогда $K(E) \subseteq \cL(E)$ --- двухсторонний идеал.
\end{proposition}

\begin{proof}~
	\begin{enumerate}
		\item $K(E)$ является подгруппой по сложению. Всё тривиально, кроме сложения. Пусть $A, B \in K(E)$. Тогда $A + B \in K(E)$ тогда и только тогда, когда $(A + B)(B(0, 1))$ --- предкомпакт. Это эквивалентно тому, что из любой ограниченной последовательности в этом множестве можно выделить сходящуюся подпоследовательность (причём предел не обязательно в множестве). Действительно, рассмотрим ограниченную последовательность $\{y_n\}_{n = 1}^\infty \subseteq (A + B)(B(0, 1))$. В силу определения, её элементы распишутся так:
		\[
			\forall n \in \N\ \ y_n = Ax_n + Bx_n,\ x_n \in B(0, 1)
		\]
		Так как $A \in K(E)$, то из $Ax_n$ можно выделить сходящуюся подпоследовательность $Ax_{n_k}$. Аналогично, уже из $Bx_{n_k}$ можно выделять сходящуюся подпоследовательность $Bx_{n_{k_l}}$, причём предыдущая сходимость никуда не денется. Таким образом, $y_{n_{k_l}}$ обязаны быть искомой сходящейся подпоследовательностью, что и требовалось.
		
		\item $K(E)$ поглощает элементы $\cL(E)$. Пусть $A \in K(E)$ и $B \in \cL(E)$. Тогда $AB \in K(E)$ --- это тривиально, ибо $B(B(0, 1))$ тоже ограниченное множество. Для $BA$ же так не получится, но мы можем воспользоваться приёмом предыдущего пункта. Рассмотрим ограниченную последовательность $\{y_n\}_{n = 1}^\infty \subseteq BA(B(0, 1))$. Тогда $y_n = BAx_n$, $x_n \in B(0, 1)$. В силу компактности оператора $A$, можно из $Ax_n$ выделить сходящуюся подпоследовательность $Ax_{n_k}$. Так как оператор $B$ непрерывен, сходимость в образе сохранится, а значит нужная подпоследовательность $y_{n_k} = BAx_{n_k}$ найдена.
	\end{enumerate}
\end{proof}
