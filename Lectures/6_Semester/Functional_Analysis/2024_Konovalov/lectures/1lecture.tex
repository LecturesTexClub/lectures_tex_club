\setcounter{section}{6}
\section{Слабая сходимость}

\begin{definition}
	Пусть $E$ --- линейное нормированное пространство, $\{x_n\}_{n = 1}^\infty \subseteq E$. Последовательность $x_n$ \textit{слабо сходится к} $x$, если
	\[
		\forall f \in E^*\ \ f(x_n) \xrightarrow[n \to \infty]{} f(x)
	\]
	Обозначется как $x_n \wto x$.
\end{definition}

\begin{proposition}
	В гильбертовом пространстве $H$ есть эквивалентное определение:
	\[
	x_n \wto x \Lra \forall y \in H\ \ (x_n, y) \xrightarrow[n \to \infty]{} (x, y)
	\]
\end{proposition}

\begin{proof}
	Тривиально по теореме Рисса-Фреше.
\end{proof}

\begin{note}
	Далее, если не сказано обратного, мы находимся в линейном нормированном пространстве $E$.
\end{note}

\begin{proposition}
	Если последовательность $x_n$ слабо сходится, то предел единственен.
\end{proposition}

\begin{proof}
	Пусть $x', x''$ --- два предела по слабой сходимости последовательности $x_n$. По условию:
	\[
		\forall f \in E^*\ \ f(x') = \lim_{n \to \infty} f(x_n) = f(x'')
	\]
	По следствию 3 теоремы Хана-Банаха такое возможно только в том случае, когда $x' = x''$.
\end{proof}

\begin{proposition}
	Из сходимости $\lim_{n \to \infty} x_n = x$ следует слабая сходимость $x_n \wto x$. Обратный факт неверен только в том случае, когда $\dim E = \infty$.
\end{proposition}

\begin{proof}~
	\begin{itemize}
		\item[$\Ra$] Рассмотрим $f \in E^*$. Тогда:
		\[
			|f(x_n) - f(x)| = |f(x_n - x)| \le \|f\| \cdot \|x_n - x\| \xrightarrow[n \to \infty]{} 0
		\]
		
		\item[$\La$] Теперь $\dim E < \infty$. Тогда, существует базис $e$. Заметим, что функционалы координат (то есть такие функционалы, которые на вход $x$ выдают координату по соответствующему вектору базиса) лежат в $E^*$. Стало быть, есть покоординатная сходимость, а она соответствует норме $l_\infty$, которая эквивалентна (в силу конечной размерности) исходной норме $E$.
	\end{itemize}
\end{proof}

\begin{example}
	Покажем случай $\dim E = \infty$, когда утверждение выше не работает. Рассмотрим пространство $l_2$, за последовательность возьмём просто базис:
	\[
		e_n = (\underbrace{0, \ldots, 0}_{n}, 1, 0, \ldots)
	\]
	Тогда очевидно, что эта последовательность никуда не сходится: попарное расстояние между элементами всегда равно $\sqrt{2}$. При этом, воспользуемся тем фактом, что $l_2$ гильбертово, и посмотрим слабую сходимость:
	\[
		\forall y \in l_2\ \ (e_n, y) = y_n \xrightarrow[n \to \infty]{} 0 = (0, y)
	\]
	Предельный переход работает, коль скоро верно равенство Парсеваля. Стало быть, есть слабая сходимость к нулю, но нет сходимости по норме.
\end{example}

\begin{task}
	Пусть $x_n \in \ole{B}(0, R)$, причём $x_n \wto x$. Тогда $x \in \ole{B}(0, R)$.
\end{task}

\begin{proof}
	Разберём две разные ситуации
	\begin{itemize}
		\item $E = H$ --- гильбертово пространство. Тогда, в силу эквивалентного определения
		\[
			\forall y \in H\ \ \lim_{n \to \infty} (x_n, y) = (x, y) \Ra \lim_{n \to \infty} |(x_n, y)| = |(x, y)|
		\]
		Если мы сможем оценить $|(x_n, y)|$, то такая же оценка верна и для предела в силу предельного перехода:
		\[
			|(x_n, y)| \le \|x_n\| \cdot \|y\| \le R\|y\| \Ra |(x, y)| \le R\|y\|
		\]
		Так как неравенство КБШ достигается, то должно быть верно неравенство $|(x, y)| \le \|x\| \cdot \|y\| \le R\|y\|$, то есть $\|x\| \le R$.
		
		\item $E$ --- произвольное линейное нормированное пространство. Здесь у нас нет права пользоваться теоремой Рисса-Фреше, но, тем не менее, есть теорема Хана-Банаха и её следствия. Согласно второму следствию:
		\[
			\forall x \in E \bs \{0\}\ \exists f_0 \in E^* \such \|f_0\| = 1 \wedge f_0(x) = \|x\|
		\]
		В силу слабой сходимости:
		\[
			\forall f \in E^*\ \ \lim_{n \to \infty} f(x_n) = f(x) \Ra \lim_{n \to \infty} |f(x_n)| = |f(x)| \Ra \lim_{n \to \infty} |f_0(x_n)| = |f_0(x)| = \|x\|
		\]
		При этом $|f_0(x_n)| \le \|f_0\| \cdot \|x_n\| \le 1 \cdot R$, а отсюда уже тривиально по предельному переходу в неравенстве.
	\end{itemize}
\end{proof}

\begin{theorem} (Мазура, 1933г., без доказательства)
	Пусть $S \subseteq E$ --- выпуклое замкнутое множество, $x_n \wto x$, причём $x_n \in S$. Тогда $x \in S$.
\end{theorem}

\begin{proposition} (Связь сходимости по норме и слабой сходимости)
	Пусть $x_n, x \in E$. Тогда
	\begin{enumerate}
		\item Из сходимости $x_n \to x$ всегда следует, что $x_n \wto x$ и $\lim_{n \to \infty} \|x_n\| = \|x\|$
		
		\item Если $E$ --- гильбертово пространство, то из слабой сходимости $x_n \wto x$ и предела $\lim_{n \to \infty} \|x_n\| = \|x\|$ следует сходимость $x_n \to x$
	\end{enumerate}
\end{proposition}

\begin{anote}
	Таким образом, в случае гильбертова пространства у нас есть явный критерий сходимости через слабую сходимость.
\end{anote}

\begin{proof}~
	\begin{enumerate}
		\item Повторим уже известные рассуждения:
		\begin{itemize}
			\item Для слабой сходимости всё есть из того факта, что рассматриваются $f \in E^*$:
			\[
				\forall f \in E^*\ \ \|f(x_n) - f(x)\| = \|f(x_n - x)\| \le \|f\| \cdot \|x_n - x\| \xrightarrow[n \to \infty]{} 0
			\]
			
			\item Для сходимости норм вспомним неравенство:
			\[
				\big|\|x_n\| - \|x\|\big| \le \|x_n - x\| \xrightarrow[n \to \infty]{} 0
			\]
		\end{itemize}
		
		\item По определению, нужно показать следующее:
		\[
			\forall \eps > 0\ \exists N \in \N \such \forall n \ge N\ \ \|x_n - x\| < \eps
		\]
		Распишем $\|x_n - x\|^2$ через скалярное произведение:
		\[
			\|x_n - x\|^2 = \|x_n\|^2 + \|x\|^2 - (x_n, x) - (x, x_n)
		\]
		Так как оценка КБШ точная, а скалярное произведение $f(x_n) = (x_n, x)$ является линейной непрерывной функцией, то есть сходимость $\lim_{n \to \infty} (x_n, x) = \|x\|^2$. Стало быть
		\[
			\|x_n - x\|^2 \xrightarrow[n \to \infty]{} 2\|x\|^2 - 2\|x\|^2 = 0
		\]
	\end{enumerate}
\end{proof}

\begin{anote}
	В случае банахова пространства, вообще говоря, сходимость $x_n \to x$ может не следовать из $x_n \wto x$ и $\lim_{n \to \infty} \|x_n\| = \|x\|$. Рассмотрим пространство $c_0$ --- сходящиеся последовательности. Известно, что $c_0^* \simeq \ell_1$, причём для слабой сходимости есть эквивалентное свойство, использующее <<скалярное произведение>>:
	\[
		x_n \wto x \Lolra \forall y \in \ell_1\ \ \phi_y(x_n) = \sum_{k = 1}^\infty x_n^ky_k \xrightarrow[n \to \infty]{} \sum_{k = 1}^\infty x^ky_k = \phi_y(x)
	\]
	Рассмотрим базис с $e_0 = (1, 1, \ldots)$ и $e_n = (0, \ldots, 0, 1, 0, \ldots)$ (единица стоит в $n$-й позиции). Несложно понять, что $x_n = e_0 - e_n$ должен хотя бы слабо сходится к $x = e_0$, при этом $\|x_n\| = \|x\| = 1$, тем самым предел по норме $c_0$ уже имеется. Проверим сходимости:
	\begin{itemize}
		\item $|\phi_y(x) - \phi_y(x_n)| = |\phi_y(x - x_n)| = |\phi_y(e_n)| = |y_n| \xrightarrow[n \to \infty]{} 0$ --- слабая сходимость есть
		
		\item $\|x - x_n\| = \|e_n\| = 1$ --- сходимости по норме $c_0$ нет и быть не может
	\end{itemize}
\end{anote}

\textcolor{red}{Тут должна быть картинка с нормами и сходимостями, 1 лекция 1:04:00}

\begin{theorem} (фон Неймана, без доказательства)
	Пусть $H$ --- гильбертово пространство, причём $\dim H = \infty$. Рассмотрим сходимость $x_n \xrightarrow{h} x$ \textcolor{red}{(лектор не давал обозначения, просто я ввёл для формулировки)} следующего вида:
	\[
		x_n \xrightarrow[n \to \infty]{h} x \Lra \forall y \in H\ \ \lim_{n \to \infty} (x_n, y) = (x, y)
	\]
	Тогда эта сходимость не является метризуемой.
\end{theorem}

\begin{theorem}
	Между пространствами $E$ и $E^{**}$ существует изометрия $\pi$, чей вид явно записывается так:
	\[
		\forall x \in E, f \in E^*\ \ \pi(x)(f) := f(x)
	\]
\end{theorem}

\begin{anote}
	В курсе Алгебры и Геометрии мы установили, что в конечномерном случае отображение выше задаёт \textit{канонический изоморфизм}.
\end{anote}

\begin{proof}
	Определим элемент $\pi(x)$ поточечно, согласно утверждению теоремы. Тогда мы тривиально получим линейное отображение, для которого остаётся лишь установить равенство норм:
	\[
		\|\pi(x)\| = \sup_{\|f\| = 1} |\pi(x)(f)| = \sup_{\|f\| = 1} |f(x)| = \|x\|
	\]
	Последний переход сделан по следствию 4 теоремы Хана-Банаха
\end{proof}

\begin{note}
	Далее для обозначения $\pi(x)$ мы будем использовать более лаконичное $F_x$.
\end{note}

\begin{definition}
	Если $E = E^{**}$ (в смысле наличия изоморфизма), то пространство $E$ называется \textit{рефлексивным}.
\end{definition}