\subsection*{Собственные значения компактных операторов}

\begin{note}
	В этой главе мы рассматриваем банахово пространство $E$ над полем $\Cm$, $A \in K(E)$.
\end{note}

\begin{note}
	Одна из интересных особенностей компактных операторов состоит в том, что они наиболее похожи на хорошо известные читателю операторы из конечномерных пространств. В этом нас убедят следующие теоремы.
\end{note}

\begin{theorem} \label{compact_finite_dim}
	Пусть $\lambda \in \Cm \bs \{0\}$. Тогда $\dim \ker A_\lambda < \infty$.
\end{theorem}

\begin{proof}
	Утверждение теоремы эквивалентно тому, что единичная сфера в пространстве $\ker A_\lambda$ компактна. Это будет доказано, если мы покажем, как выделить из любой последовательности сходящуюся подпоследовательность. Пусть $x_n \in S(0, 1) \subseteq \ker A_\lambda$. Отсюда $\|x_n\| = 1$ и $Ax_n = \lambda x_n$. Более того, $\{x_n\}_{n = 1}^\infty$ --- ограниченное множество, а значит $\{Ax_n\}_{n = 1}^\infty$ --- предкомпакт. Стало быть, существует сходящаяся подпоследовательность $\lim_{k \to \infty} Ax_{n_k} = y$. В силу того, что мы можем раскрыть образ через $x_{n_k}$, получим следующее:
	\[
		\lim_{k \to \infty} \lambda x_{n_k} = y \Lra \lim_{k \to \infty} x_{n_k} = \frac{1}{\lambda}y
	\]
	Однако, это ещё не всё. Нам также нужно показать, что $y \in \ker A_\lambda$ --- принадлежит рассматриваемому подпространству. Для этого мы применим оператор $A$ к обеим частям предела (в его силу непрерывности):
	\[
		\lim_{k \to \infty} Ax_{n_k} = y = \frac{1}{\lambda} Ay \Lra Ay = \lambda y \Lra y \in \ker A_\lambda
	\]
\end{proof}

\begin{theorem}
	В условиях главы, для любого $\delta > 0$ вне любого круга $\{|\lambda| \le \delta\}$ может лежать лишь конечное число собственных значений оператора $A$.
\end{theorem}

\begin{proof}
	Проведём доказательство в частном случае $E = H$ --- гильбертово пространство, и $A$ --- компактный самосопряженный оператор. Предположим противное. Тогда, должно существовать $\delta_0 > 0$ и хотя бы счётное число $\{\lambda_n\}_{n = 1}^\infty$ собственных значений вне этого круга (то есть $|\lambda_n| > \delta_0$). Пусть $e_n$ --- нормированный собственный вектор для значения $\lambda_n$. Тогда $\{e_n\}_{n = 1}^\infty$ --- ограниченное множество, а значит $\{Ae_n\}_{n = 1}^\infty$ --- предкомпакт. Однако, в то же время верно неравенство (здесь мы используем ортогональность собственных векторов, это свойство самосопряжённого оператора):
	\[
		\forall n \neq m\ \ \|Ae_n - Ae_m\|^2 = \|\lambda_n e_n - \lambda_m e_m\|^2 = \lambda_n^2 + \lambda_m^2 > 2\delta_0^2
	\]
	Получили явное противоречие с вполне ограниченностью.
\end{proof}

\begin{corollary}
	Верно утверждение:
	\[
		\forall \delta > 0\ \ \sum_{|\lambda| > \delta} \dim \ker A_\lambda < \infty
	\]
\end{corollary}

\begin{proof}
	Тривиальное следствие последних двух теорем.
\end{proof}

\begin{theorem} (Фредгольма)
	Пусть $H$ --- гильбертово пространство над $\Cm$, $A$ --- компактный самосопряжённый оператор и $\lambda \in \Cm \bs \{0\}$. Тогда
	\[
		H = \im A_\lambda \oplus_{\bot} \ker A_\lambda
	\]
\end{theorem}

\begin{note}
	Отметим, что подобную теорему мы уже доказали для самосопряженных операторов --- это теорема \ref{sao_fred_th}. Теперь же, если добавить требование компактности $A$, то мы можем убрать замыкание образа $A_\lambda$
\end{note}

\begin{lemma} \label{simple_spectre}
	Если $\lambda \in \sigma(A) \bs \{0\}$, то $\lambda \in \sigma_p(A)$.
\end{lemma}

\begin{proof}
	По критерию принадлежности спектру, существует нормированная последовательность $x_n$, для которой есть предел $\lim_{n \to \infty} A_\lambda x_n = 0$. Так как $\{x_n\}_{n = 1}^\infty$ --- ограниченное множество, то в силу компактности $A$ можно выделить сходящуюся подпоследовательность $\lim_{k \to \infty} Ax_{n_k} = y$. Тогда, мы в то же время имеем равенство
	\[
		\lim_{k \to \infty} Ax_{n_k} = \lim_{k \to \infty} \lambda x_{n_k} = y
	\]
	Воспользуемся техникой, которой пользовались ранее: в силу непрерывности оператора $A$, его можно применить к последнему равенству:
	\[
		\lim_{k \to \infty} \lambda Ax_{n_k} = \lambda y = Ay \Lra y \in \ker A_\lambda
	\]
	Важно отметить, что $y \neq 0$. Это следует из упомянутого предела $\lim_{k \to \infty} \lambda x_{n_k} = y$. Стало быть, $\lambda \in \sigma_p(A)$.
\end{proof}

\begin{lemma}
	Пусть $M \subseteq H$ --- подпространство, инвариантное относительно самосопряжённого оператора $A$ (то есть $AM \subseteq M$). Тогда $M^\bot$ тоже инвариантен относительно $A$.
\end{lemma}

\begin{proof}
	Пусть $x \in M$. В силу условия, $Ax \in M$. Вопрос состоит в том, чтобы из  $y \in M^\bot$ показать верность $Ay \in M^\bot$. Проверим это явно:
	\[
		\forall x \in M\ (x, Ay) = (Ax, y) = 0 \Lora Ay \in M^\bot
	\]
\end{proof}

\begin{lemma}
	$\cl (\im A_\lambda) = \im A_\lambda$. Иначе говоря, образ $A_\lambda$ замкнут.
\end{lemma}

\begin{proof}
	Применим лемму об инвариантности. Заметим, что $M = \ker A_\lambda$ инвариантен относительно $A$ и $A_\lambda$, а значит и $M^\bot = \cl(\im A_\lambda)$ инвариантен относительно тех же операторов. Если мы докажем, что $A_\lambda|_{\cl(\im A_\lambda)}$ является сюръективным оператором, то всё будет доказано. Действительно, получим тогда $\cl(\im A_\lambda) = A_\lambda(\cl \im A_\lambda) \subseteq \im A_\lambda$. Обозначим $\tilde{A} = A|_{\cl(\im A_\lambda)}$ (<<понять это невозможно, можно только запомнить>>). Это тоже компактный самосопряжённый оператор, действующий из $\cl(\im A_\lambda)$ в само себя. Заметим, как связаны собственные значения $\wdt{A}$ с исходными:
	\[
		(\wdt{A})_\lambda = \wdt{A} - \lambda I = A|_{\cl(\im A_\lambda)} - \lambda I|_{\cl(\im A_\lambda)} = (A - \lambda I)|_{\cl(\im A_\lambda)} = \wdt{(A_\lambda)}
	\]
	А как мы знаем по теореме \ref{sao_fred_th}, все собственные вектора лежат в другой части прямого разложения. Раз так, то $\lambda \notin \{0\} \cup \sigma_p(\wdt{A})$. По доказанной лемме \ref{simple_spectre} может быть лишь верно $\lambda \in \rho(\wdt{A})$. Значит, оператор $(\wdt{A})_\lambda = \wdt{(A_\lambda)}$ биективен, что включает в себя его сюрьективность.
\end{proof}

\begin{proof} (теоремы)
	Очевидно.
\end{proof}

\begin{note}
	В разной литературе и доказанная теорема, и соответствующие леммы носят имя теорем Фредгольма.
\end{note}

\textcolor{red}{Сделать картинку теорем и лемм}

\begin{note}
	Теорема Фредгольма --- результат развития теории Фредгольма, посвящённой решению интегральных уравнений следующих видов:
	\begin{itemize}
		\item $A_\lambda x = y$
		
		\item $A_\lambda z = 0$
	\end{itemize}
	При $\lambda \neq 0$, возможно всего 2 ситуации:
	\begin{itemize}
		\item $\lambda$ --- не собственное значение. Тогда $\lambda \in \rho(A)$, а значит первое уравнение имеет единственное решение при любом $y$.
		
		\item $\lambda$ --- собственное значение. Это значит, что $\ker A_\lambda \neq \{0\}$ --- есть ненулевые собственные вектора. Тогда, в силу разложение $\im A_\lambda \oplus_\bot \ker A_\lambda = H$, первое уравнение будет иметь решение только в том случае, если $y$ ортогонален всем решениям второго уравнения.
	\end{itemize}
	Это утверждение называется \textit{альтернативой Фредгольма}. В его времена она формулировалась так (снова $\lambda \neq 0$). Возможно только 2 ситуации:
	\begin{itemize}
		\item Либо первое уравнение имеет единственное решение при любом $y$.
		
		\item Либо у него есть решение при некоторых $y$, но не единственное.
	\end{itemize}
\end{note}

\begin{theorem} (Гильберта --- Шмидта)
	Пусть $H$ --- сепарабельное гильбертово пространство над полем $\Cm$, $A$ --- компактный самосопряжённый оператор. Тогда в $H$ найдётся ортонормированный базис, состоящий из собственных векторов оператора $A$.
\end{theorem}

\begin{note}
	Случай $\dim H < \infty$ был доказан в курсе Алгебры и Геометрии, поэтому далее $\dim H = \infty$.
\end{note}

\begin{lemma}
	Если $A \neq 0$, то у этого оператора существует собственное значение $\lambda \neq 0$.
\end{lemma}

\begin{proof}
	Коль скоро $A \neq 0$ и мы рассматриваем компактный оператор, то $\|A\| \neq 0$. Коль скоро $A$ --- самосопряжённый оператор, то можно воспользоваться теоремой \ref{sao_spectre_bound}. По ней $\|A\| = \max(|m_-|, |m_+|)$. Так как $m_-, m_+ \in \sigma(A)$, то хотя бы одно из этих чисел ненулевое и является собственным значением, что и требовалось.
\end{proof}