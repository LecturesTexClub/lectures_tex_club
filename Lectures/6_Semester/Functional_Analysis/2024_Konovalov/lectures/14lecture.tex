\begin{proposition} (Связь свёртки и преобразования Фурье)
	Пусть $f, g \in L_1(\R)$. Тогда верна формула:
	\[
		\wdh{f * g} = \wdh{f} \cdot \wdh{g} 
	\]
\end{proposition}

\begin{proof}
	Распишем преобразование Фурье от свёртки:
	\[
		\wdh{f * g}(y) = \int_\R (f * g)(x)e^{-ixy}d\mu(x) = \int_\R \int_\R f(\xi)g(x - \xi)e^{-ixy}d\mu(\xi)d\mu(x)
	\]
	Первое, что хочется сделать с повторным интегралом --- это переставить пределы интегрирования. Нужно обосновать возможность этого действия, тут мы воспользуемся следствием теоремы Фубини. Несложно заметить, что $|f(\xi)g(x - \xi)e^{-ixy}| = |f(\xi)g(x - \xi)|$, а двойной интеграл с такой подынтегральной функцией (модулем) мы уже оценили при доказательстве замкнутости свёртки выше. Итак:
	\begin{multline*}
		\wdh{f * g}(y) = \int_\R f(\xi) \int_\R g(x - \xi)e^{-ixy}d\mu(x)d\mu(\xi) = [1 = e^{i\xi y} \cdot e^{-i\xi y}] =
		\\
		\int_\R f(\xi)e^{-i\xi y} \int_\R g(x - \xi)e^{-i(x - \xi)y}d\mu(x - \xi)d\mu(\xi) = \wdh{f}(y) \cdot \wdh{g}(y)
	\end{multline*}
	 
\end{proof}

\textcolor{red}{Здесь была долгая беседа с примерчиками}

\subsection{В пространстве $L_2(\R)$}

\begin{reminder}
	\textit{Пространством Шварца} $S \subset L_2(\R)$ называется множество бесконечно дифференциемых функций, которые вместе со всеми своими производными убывают на бесконечности быстрее любой степени:
	\[
		S = \{f \in C^\infty(\R) \colon \forall n \in \N_0, m \in \N\ \ f^{(n)}(x) = o(x^m),\ x \to \infty\}
	\]
\end{reminder}

\begin{reminder}
	Замыкание $S$ --- это пространство $L_2(\R)$ (естественно подразумевается замыкание по 2-норме):
	\[
		\cl S = L_2(\R)
	\]
\end{reminder}

\begin{reminder}
	Преобразование Фурье биективно на пространстве Шварца:
	\[
		FS = S 
	\]
\end{reminder}

\begin{proposition}
	Преобразование Фурье продолжается на $L_2$. Более того, $F[L_1] \subseteq S$.
\end{proposition}

\begin{proof}
	Как известно из предыдущего семестра, линейный ограниченный оператор, определённый на линейном многообразии, продолжается на его замыкание с сохранением нормы. Именно это тут и происходит.
\end{proof}

\begin{proposition}
	Преобразование Фурье на $S$ обладает следующими свойствами:
	\begin{enumerate}
		\item $F \colon S \to S$ --- биекция
		
		\item $F \colon S \to S$ --- изометрия (сохраняется скалярное произведение)
		
		\item $F^4 = I$ (Более того, $F^2f(x) = f(-x)$)
	\end{enumerate}
\end{proposition}

\begin{proof}~
	\begin{enumerate}
		\item Было в курсе математического анализа \textcolor{red}{Я пока морально не готов}
		
		\item Распишем скалярное произведение с использованием \textit{формулы обращения} (она идёт без доказательства):
		\begin{multline*}
			(f, g) = \int_\R f(x)\ole{g(x)}d\mu(x) = \int_\R f(x)\ole{\ps{\frac{1}{\sqrt{2\pi}}\int_\R \wdh{g}(y)e^{ixy}d\mu(y)}}d\mu(x) =
			\\
			\frac{1}{\sqrt{2\pi}} \int_\R \int_\R f(x)\ole{\wdh{g}(y)}e^{-ixy}d\mu(x)d\mu(y) = \frac{1}{\sqrt{2\pi}} \int_\R \ole{\wdh{g}(y)} \int_\R f(x)e^{-ixy}d\mu(x)d\mu(y) = (\wdh{f}, \wdh{g})
		\end{multline*}
		
		\item Докажем именно ту часть, что $F^2f(x) = f(-x)$. \textcolor{red}{Доказать?}
	\end{enumerate}
\end{proof}

\begin{proposition} (Теорема Планшереля, 1910г.)
	\begin{enumerate}
		\item Преобразовние Фурье осуществляет биекцию на $L_2(\R)$ и даже изометрию. В частности, $\|f\|_2^2 = \|\wdh{f}\|_2^2$
		
		\item Определим $g_N(y)$ как своего рода \textit{срезку преобразования Фурье}:
		\[
			g_N(y) = \frac{1}{\sqrt{2\pi}} \int_{-N}^N g(x)e^{-ixy}d\mu(x)
		\]
		Тогда $g_N \xrightarrow[]{L_2} \wdh{g}$
	\end{enumerate}
\end{proposition}

\begin{proof}~
	\begin{enumerate}
		\item Начнём с изометрии. Мы уже знаем, что $\cl S = L_2(\R)$, поэтому рассмотрим подходящие к $f, g \in L_2(\R)$ последовательности $\{f_n\}_{n = 1}^\infty, \{g_n\}_{n = 1}^\infty \subset S$ соответственно. Уже известно, что $(f_n, g_n) = (\wdh{f}_n, \wdh{g}_n)$. При этом, в силу непрерывности скалярного произведения, левая часть стремится к $(f, g)$. Вспомним, что мы продолжили преобразование Фурье на $L_2(\R)$ с сохранением нормы оператора, а значит $F$ остался линейным непрерывным оператором, то есть $\lim_{n \to \infty} \wdh{f}_n = \wdh{f}$ и $\lim_{n \to \infty} \wdh{g}_n = \wdh{g}$ и по тем же причинам есть вторая сходимость скалярных произведений:
		\[
			(f, g) = \lim_{n \to \infty} (f_n, g_n) = \lim_{n \to \infty} (\wdh{f}_n, \wdh{g}_n) = (\wdh{f}, \wdh{g})
		\]
		
		Теперь биективность. Инъективность обеспечена изометрией, поэтому остаётся разобраться с сюръективностью. Пусть $g \in L_2(\R)$. Тогда мы точно знаем, что существует последовательность $\{g_n\}_{n = 1}^\infty \subset S \colon \lim_{n \to \infty} g_n = g$, причём каждый её элемент (в силу биективности $FS = S$) является преобразованием Фурье другой функции $f_n \in S$. Так как есть сходимость $g_n$, то эта последовательность фундаментальна. Но тогда
		\[
			\forall \eps > 0\ \exists N \in \N \such \forall m \ge n > N\ \ \|f_n - f_m\|^2 = \|\wdh{f}_n - \wdh{f}_m\|^2 = \|g_n - g_m\|^2 < \eps^2
		\]
		То есть и $f_n$ образуют фундаментальную последовательность, а значит сходятся к $f \in L_2(\R)$. Так как преобразование Фурье непрерывно, получаем требуемое:
		\[
			Ff = F(\lim_{n \to \infty} f_n) = \lim_{n \to \infty} Ff_n = \lim_{n \to \infty} g_n = g
		\]
		
		\item \textcolor{red}{Без доказательства?}
	\end{enumerate}
\end{proof}