\begin{proposition} (Связь свёртки и преобразования Фурье)
	Пусть $f, g \in L_1(\R)$. Тогда верна формула:
	\[
		\wdh{f * g} = \wdh{f} \cdot \wdh{g} 
	\]
\end{proposition}

\begin{proof}
	Распишем преобразование Фурье от свёртки:
	\[
		\wdh{f * g}(y) = \int_\R (f * g)(x)e^{-ixy}d\mu(x) = \int_\R \int_\R f(\xi)g(x - \xi)e^{-ixy}d\mu(\xi)d\mu(x)
	\]
	Первое, что хочется сделать с повторным интегралом --- это переставить пределы интегрирования. Нужно обосновать возможность этого действия, тут мы воспользуемся следствием теоремы Фубини. Несложно заметить, что $|f(\xi)g(x - \xi)e^{-ixy}| = |f(\xi)g(x - \xi)|$, а двойной интеграл с такой подынтегральной функцией (модулем) мы уже оценили при доказательстве замкнутости свёртки выше. Итак:
	\begin{multline*}
		\wdh{f * g}(y) = \int_\R f(\xi) \int_\R g(x - \xi)e^{-ixy}d\mu(x)d\mu(\xi) = [1 = e^{i\xi y} \cdot e^{-i\xi y}] =
		\\
		\int_\R f(\xi)e^{-i\xi y} \int_\R g(x - \xi)e^{-i(x - \xi)y}d\mu(x - \xi)d\mu(\xi) = \wdh{f}(y) \cdot \wdh{g}(y)
	\end{multline*}
	 
\end{proof}

\textcolor{red}{Здесь была долгая беседа с примерчиками}

\subsection{В пространстве $L_2(\R)$}

\begin{note}
	До этого момента мы рассматривали преобразование Фурье в \textit{ассимметричной форме} (коэффициенты перед интегралами разные). Далее нам крайне важно перейти от этого варианта к симметричному, то есть теперь и в прямом, и в обратном преобразовании Фурье будет записан коэффициент $\frac{1}{\sqrt{2\pi}}$ перед интегралом.
\end{note}

\begin{reminder}
	\textit{Пространством Шварца} $S \subset L_1(\R) \cap L_2(\R)$ называется множество бесконечно дифференциемых функций, которые вместе со всеми своими производными убывают на бесконечности быстрее любой степени:
	\[
		S = \{f \in C^\infty(\R) \colon \forall n \in \N_0, m \in \N\ \ \lim_{x \to \infty} x^mf^{(n)}(x) = 0\}
	\]
\end{reminder}

\begin{proposition} \textcolor{red}{(не по лектору)}
	Пространство Шварца инвариантно относительно преобразования Фурье.
\end{proposition}

\begin{reminder} \textcolor{red}{(не по лектору)}
	Если $f(x), xf(x) \in L_1(\R)$, то преобразование Фурье $g = F[f]$ будет дифференцируемо на $\R$.
\end{reminder}

\begin{corollary} \textcolor{red}{(не по лектору)}
	Если $f(x), xf(x), \ldots, x^pf(x) \in L_1(\R)$, то преобразование Фурье $g = F[f]$ будет $p$ раз дифференцируемо на $\R$.
\end{corollary}

\begin{corollary} \textcolor{red}{(не по лектору)}
	Если $f(x) \in L_1(\R)$ и $\forall p \in \N\ x^pf(x) \in L_1(\R)$, то преобразование Фурье $g = F[f]$ дифференцируемо бесконечное число раз на $\R$.
\end{corollary}

\begin{proof} \textcolor{red}{(не по лектору)}
	Пусть $f \in S$. Функции пространства Шварца можно описать эквивалентным образом:
	\[
		\exists C_{n, m} \in \R_+ \such \forall x \in \R\ \ |x^mf^{(n)}(x)| \le C_{n, m}
	\]
	Покажем, что из этого факта следует $x^pf(x) \in L_1(\R)$ при любом $p \in \N$. Действительно, можно написать следующее:
	\[
		\forall m \in \N\ \ \exists C_{0, m + 2} \in \R_+ \such \forall x \in \R\ \ |x^mf(x)| \le \frac{C_{0, m + 2}}{x^2}
	\]
	Отсюда тривиально получаем абсолютную интегрируемость. Стало быть, преобразование Фурье $g = F[f]$ обладает всеми производными. Чтобы доказать, что они тоже являются функциями из пространства Шварца, воспользуемся следующим равенством:
	\[
		(iy)^qg^{(m)}(y) = (-i)^qF\big[(x^mf(x))^{(q)}\big](y)
	\]
	Так как $F$ действует на пространствах $L_1(\R) \to C_0(\R)$, то требуемое установлено.
\end{proof}

\begin{proposition}
	Преобразование Фурье на $S$ обладает следующими свойствами:
	\begin{enumerate}
		\item $F \colon S \to S$ --- биекция
		
		\item $F \colon S \to S$ --- изометрия (сохраняется скалярное произведение)
		
		\item $F^4 = I$ (Более того, $F^2f(x) = f(-x)$)
	\end{enumerate}
\end{proposition}

\begin{proof}~
	\begin{enumerate}
		\item \textcolor{red}{(по книге Колмогорова-Фомина)} В силу уже доказанного факта, достаточно показать, что для любой $g \in S$ найдётся прообраз по преобразованию Фурье. Для этого, наоборот, посмотрим на образ:
		\[
			f^*(x) = \frac{1}{\sqrt{2\pi}}\int_\R g(y)e^{-iyx}d\mu(y)
		\]
		Положим $f(x) = f^*(-x)$. Из уже доказанного, $f^* \in S$, а значит и $f \in S$. Осталось воспользоваться формулой обращения:
		\[
			g(y) = \frac{1}{\sqrt{2\pi}}\int_\R f^*(x)e^{ixy}d\mu(x) = \int_\R f(x)e^{-ixy}d\mu(x) = F[f](y)
		\]
		
		\item Распишем скалярное произведение с использованием \textit{формулы обращения} (она идёт без доказательства):
		\begin{multline*}
			(f, g) = \int_\R f(x)\ole{g(x)}d\mu(x) = \int_\R f(x)\ole{\ps{\frac{1}{\sqrt{2\pi}}\int_\R \wdh{g}(y)e^{ixy}d\mu(y)}}d\mu(x) =
			\\
			\frac{1}{\sqrt{2\pi}}\int_\R \int_\R f(x)\ole{\wdh{g}(y)}e^{-ixy}d\mu(x)d\mu(y) = \int_\R \ole{\wdh{g}(y)} \frac{1}{\sqrt{2\pi}}\int_\R f(x)e^{-ixy}d\mu(x)d\mu(y) = (\wdh{f}, \wdh{g})
		\end{multline*}
		
		\item \textcolor{red}{(от автора)} Докажем именно ту часть, что $F^2f(x) = f(-x)$. Заметим связь между прямым и обратным преобразованием Фурье:
		\[
			F[f]y = \wdh{f}(y) = \frac{1}{\sqrt{2\pi}}\int_\R f(x)e^{-ixy}d\mu(x) = \frac{1}{\sqrt{2\pi}}\int_\R f(x)e^{ix(-y)}d\mu(x) = \check{f}(-y)
		\]
		Так как преобразование Фурье биективно, можно применить его к полученному равенству и получить требуемое.
	\end{enumerate}
\end{proof}

\begin{reminder}
	Замыкание $S$ --- это пространство $L_2(\R)$ (естественно подразумевается замыкание по 2-норме):
	\[
	\cl S = L_2(\R)
	\]
\end{reminder}

\begin{proposition}
	Преобразование Фурье продолжается на $L_2(\R)$. Более того, \\ $F[L_2(\R)] \subseteq L_2(\R)$
\end{proposition}

\begin{proof}
	Как известно из предыдущего семестра, линейный ограниченный оператор, определённый на линейном многообразии, продолжается на его замыкание с сохранением нормы. Именно это тут и происходит.
\end{proof}

\begin{corollary}
	Имеет место 2 факта про преобразование Фурье
	\begin{enumerate}
		\item $F$ - не компактный оператор
		
		\item $F$ не является самосопряжённым оператором.
	\end{enumerate}
\end{corollary}

\begin{proof}
	Оба утверждения получаются из $F^4 = I$:
	\begin{enumerate}
		\item Тривиально в силу некомпактности тождественного оператора в рассматриваемом пространстве
		
		\item Если бы это было так, то у преобразования Фурье точечный спектр лежал бы на вещественной прямой. Однако из уравнения следует, что $\sigma_p(F) \supseteq \{\pm 1, \pm i\}$.
	\end{enumerate}
\end{proof}

\begin{proposition} (Теорема Планшереля, 1910г.)
	\begin{enumerate}
		\item Преобразование Фурье осуществляет биекцию на $L_2(\R)$ и даже изометрию. В частности, $\|f\|_2^2 = \|\wdh{f}\|_2^2$
		
		\item Определим $g_N(y)$ как своего рода \textit{срезку преобразования Фурье}:
		\[
			g_N(y) = \frac{1}{\sqrt{2\pi}}\int_{-N}^N g(x)e^{-ixy}d\mu(x)
		\]
		Тогда $g_N \xrightarrow[]{L_2} \wdh{g}$
	\end{enumerate}
\end{proposition}

\begin{proof}~
	\begin{enumerate}
		\item Начнём с изометрии. Мы уже знаем, что $\cl S = L_2(\R)$, поэтому рассмотрим подходящие к $f, g \in L_2(\R)$ последовательности $\{f_n\}_{n = 1}^\infty, \{g_n\}_{n = 1}^\infty \subset S$ соответственно. Уже известно, что $(f_n, g_n) = (\wdh{f}_n, \wdh{g}_n)$. При этом, в силу непрерывности скалярного произведения, левая часть стремится к $(f, g)$. Вспомним, что мы продолжили преобразование Фурье на $L_2(\R)$ с сохранением нормы оператора, а значит $F$ остался линейным непрерывным оператором, то есть $\lim_{n \to \infty} \wdh{f}_n = \wdh{f}$ и $\lim_{n \to \infty} \wdh{g}_n = \wdh{g}$ и по тем же причинам есть вторая сходимость скалярных произведений:
		\[
			(f, g) = \lim_{n \to \infty} (f_n, g_n) = \lim_{n \to \infty} (\wdh{f}_n, \wdh{g}_n) = (\wdh{f}, \wdh{g})
		\]
		
		Теперь биективность. Инъективность обеспечена изометрией, поэтому остаётся разобраться с сюръективностью. Пусть $g \in L_2(\R)$. Тогда мы точно знаем, что существует последовательность $\{g_n\}_{n = 1}^\infty \subset S \colon \lim_{n \to \infty} g_n = g$, причём каждый её элемент (в силу биективности $FS = S$) является преобразованием Фурье другой функции $f_n \in S$. Так как есть сходимость $g_n$, то эта последовательность фундаментальна. Но тогда
		\[
			\forall \eps > 0\ \exists N \in \N \such \forall m \ge n > N\ \ \|f_n - f_m\|^2 = \|\wdh{f}_n - \wdh{f}_m\|^2 = \|g_n - g_m\|^2 < \eps^2
		\]
		То есть и $f_n$ образуют фундаментальную последовательность, а значит сходятся к $f \in L_2(\R)$. Так как преобразование Фурье непрерывно, получаем требуемое:
		\[
			Ff = F(\lim_{n \to \infty} f_n) = \lim_{n \to \infty} Ff_n = \lim_{n \to \infty} g_n = g
		\]
		
		\item Без доказательства
	\end{enumerate}
\end{proof}