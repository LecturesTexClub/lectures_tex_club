\subsubsection*{Таблица пространств}

\begin{anote}
	В таблице, $E^*$, естественно, говорит не о самом сопряжённом пространстве, а о том, чему оно изоморфно. Точно так же и вид $f \in E^*$ --- это действие на $x$ эквивалентного $f$ элемента из изоморфного пространства. Оно должно быть точно таким же, как и $f(x)$.
\end{anote}

\begin{center}
\begin{tabular}{ | p{1.5cm} | p{2cm} | p{4.2cm} | p{7.8cm} | }
	\hline
	\begin{center}$E$\end{center} & \begin{center}$E^*$\end{center} & \begin{center}Вид $f \in E^*$\end{center} & Критерий слабой сходимости (дополнительное условие к ограниченности норм $\|x_n\|$) \\
	\hline
	$\ell_p$ \newline $p > 1$ & $\ell_q$ \newline $\frac{1}{p} + \frac{1}{q} = 1$ & \vspace{0.025ex}\hspace{2.5ex}$f(x) = \sum_{k = 1}^\infty x_ky_k$ & Координатная сходимость \newline $x_k^n \xrightarrow[k \to \infty]{} x_k$ \\
	\hline
	$\ell_1$ & $\ell_\infty$ & \hspace{2.5ex}$f(x) = \sum_{k = 1}^\infty x_ky_k$ & Эквивалентна сходимости по норме \\
	\hline
	$c_0$ & $\ell_1$ & \vspace{0.025ex}\hspace{2.5ex}$f(x) = \sum_{k = 1}^\infty x_ky_k$ & Координатная сходимость \newline $x_k^n \xrightarrow[k \to \infty]{} x_k$ \\
	\hline
	$C[a; b]$ & $BV[a; b]$ & \hspace{2.1ex}$f(x) = \int_a^b x(t)dy(t)$ & Поточечная сходимость \newline $\forall t \in [a; b]\ x_n(t) \xrightarrow[n \to \infty]{} x(t)$ \\
	\hline
	$L_p[a; b]$ \newline $p > 1$ & $L_q[a; b]$ \newline $\frac{1}{p} + \frac{1}{q} = 1$ & \vspace{2.5ex}$f(x) = \int_a^b x(t)y(t)d\mu(t)$ & Сходимость первообразных \newline \[\forall s \in [a; b]\ \int_a^s x_n(t)d\mu(t) \xrightarrow[n \to \infty]{} \int_a^s x(t)d\mu(t)\] \\
	\hline
\end{tabular}
\end{center}

\section{Обратный оператор}

\textcolor{red}{Картинка и некоторое предисловие}

\begin{note}
	Далее мы фиксируем обозначения $E_1, E_2$ для линейных нормированных пространств.
\end{note}

\begin{definition}
	Оператор $A \colon E_1 \to E_2$ называется \textit{обратимым} на $\im A$, если
	\[
		\forall y \in \im A\ \ \exists ! x \in E_1 \such Ax = y
	\]
\end{definition}

\begin{anote}
	Фактически, оператор обратим, если он осуществляет биекцию $E_1 \to \im A \subseteq E_2$.
\end{anote}

\begin{example}
	Самый простой пример необратимого оператора --- это $A = 0$. Также подойдёт любой оператор, чьё ядро нетривиально (в силу критерия инъективности).
\end{example}

\begin{example}
	Естественно, далеко не всегда обратный оператор ограничен. Рассмотрим $E_1 = C[0; 1]$ и определим оператор $A$:
	\[
		(Af)(x) = \int_0^x f(t)dt =: g(x)
	\]
	Тогда $E_2 = \{g \in C^1[0; 1] \colon g(0) = 0\}$. Понятно, что $A^{-1} = \frac{d}{dx}$, но, как уже было показано в 5 семестре, этот оператор неограничен.
\end{example}

\begin{note}
	Достаточно разумный вопрос: <<А когда мы можем гарантировать, что обратный оператор непрерывен?>> Ответ дал Банах своей теоремой, чей частный случай мы докажем позже.
\end{note}

\begin{theorem} (Банаха, об обратном операторе)
	Пусть $E_1, E_2$ --- банаховы пространства, $A \in \cL(E_1, E_2)$ --- биективный оператор. Тогда $A^{-1} \in \cL(E_2, E_1)$.
\end{theorem}