\begin{theorem} \label{reverse_op_crit}
	Пусть $A \in \cL(E)$ --- взаимно однозначный оператор $E \to \im A$. Тогда обратный оператор $A^{-1}$ будет ограничен тогда и только тогда, когда образы $A$ оцениваются снизу:
	\[
		\forall x \in E\ \ \|Ax\| \ge m\|x\|
	\]
\end{theorem}

\begin{proof}~
	\begin{itemize}
		\item[$\Ra$] В силу ограниченности оператора $A^{-1}$, можно записать следующее:
		\[
			\forall y = Ax\ \ \|x\| = \|A^{-1}y\| \le \|A^{-1}\| \cdot \|y\| = \|A^{-1}\| \cdot \|Ax\|
		\]
		Отсюда имеем $\|Ax\| \ge \frac{1}{\|A^{-1}\|}\|x\|$
		
		\item[$\La$] Раз $A$ --- биекция, то и $A^{-1}$ тоже. Поэтому, вместо $x$ можно подставить соответствующий ему $A^{-1}y$, $y \in \im A$:
		\[
			\forall y \in \im A\ \ \|AA^{-1}y\| \ge m\|A^{-1}y\| \Lra \|A^{-1}y\| \le \frac{1}{m}\|y\|
		\]
		Это в точности ограниченность оператора $A^{-1}$.
	\end{itemize}
\end{proof}

\begin{theorem} \label{simple_inverse_op_theorem}
	Пусть $E$ --- банахово пространство, $A \in \cL(E)$, причём $\|A\| < 1$. Тогда оператор $(I + A)^{-1} \in \cL(E)$ обратим. Более того, справедлива формула
	\[
		(I + A)^{-1} = \sum_{k = 0}^\infty (-1)^kA^k
	\]
\end{theorem}

\begin{note}
	Ряд, записанный справа, называется \textit{рядом Неймана}
\end{note}

\begin{exercise}
	Доказать теорему с применением теоремы Банаха о сжимающих отображениях
\end{exercise}

\begin{proof}
	Нужно доказать, что ряд справа действительно является обратным к оператору $(1 + A)$ (дробь обозначает именно это). Обозначим $S_n = \sum_{k = 0}^n (-1)^kA^k$.
	\begin{enumerate}
		\item Покажем, что $S_n$ сходятся к некоторому $S \in \cL(E)$. Во-первых, $S_n \in \cL(E)$ тривиальным образом, а в силу банаховости $\cL(E)$, достаточно проверить фундаментальность этой последовательности:
		\[
			\|S_{n + p} - S_n\| = \no{\sum_{k = n + 1}^{n + p} (-1)^k A^k} \le \sum_{k = n + 1}^{n + p} \|A\|^k < \eps \text{ (при $n > N$ для данного $\eps > 0$)}
		\]
		
		\item Так как многочлены от одного и того же оператора коммутируют, то если мы покажем предел $\lim_{n \to \infty} S_n(1 + A) = 1$, тогда $S(1 + A) = 1 = (1 + A)S$ и всё доказано. Раскроем выражение под пределом:
		\[
			S_n(1 + A) = S_n + S_nA = \sum_{k = 0}^n (-1)^kA^k + \sum_{k = 1}^{n + 1} (-1)^{k - 1}A^k = A^0 + (-1)^n A^{n + 1} = 1 + (-1)^nA^{n + 1}
		\]
		Оценим норму последнего слагаемого:
		\[
			\|(-1)^nA^{n + 1}\| \le \|A\|^{n + 1} \xrightarrow[n \to \infty]{} 0 \Lolra \lim_{n \to \infty} (-1)^nA^{n + 1} = 0
		\]
		Стало быть, $\lim_{n \to \infty} S_n(1 + A) = 1 + 0 = 1$, что и требовалось доказать.
	\end{enumerate}
\end{proof}

\begin{theorem} \label{extended_inverse_op_theorem}
	Пусть $E$ --- банахово пространство, $A \in \cL(E)$ и $A^{-1} \in \cL(E)$. Также пусть $\Delta A \in \cL(E)$, причём $\|\Delta A\| < \|A^{-1}\|^{-1}$. Тогда $(A + \Delta A)^{-1} \in \cL(E)$.
\end{theorem}

\begin{proof}
	Сведём теорему к предыдущей:
	\[
		A + \Delta A = A(I + A^{-1}\Delta A)
	\]
	Проверим, что норма оператора из скобки удовлетворяет условию на норму:
	\[
		\|A^{-1}\Delta A\| \le \|A^{-1}\| \cdot \|\Delta A\| < 1
	\]
\end{proof}

\begin{task}
	Оператор $(I + A)$ непрерывно обратим тогда и только тогда, когда у ряда $\sum_{k = 0}^\infty (-1)^kA^k$ можно найти номер $k_0 \in \N$ такое, что $\|A^{k_0}\| < 1$.
\end{task}

\section{Сопряжённые операторы}

\begin{note}
	Далее, если не сказано явно иного, $E_1, E_2$ --- линейные нормированные пространства.
\end{note}

\textcolor{red}{Сюда надо картинку добавить}

\begin{definition}
	Пусть $A \in \cL(E_1, E_2)$. Тогда \textit{сопряжённым оператором} $A^* \colon E_2^* \to E_1^*$ называются оператор, удовлетворяющий условию:
	\[
		\forall g \in E_2^*\ \forall x \in E_1\ (A^*g)x = g(Ax)
	\]
\end{definition}

\begin{anote}
	Иначе говоря, $A^*g = g \circ A$
\end{anote}

\begin{proposition}
	Пусть $A \colon E_1 \to E_2$ --- линейный оператор, тогда $A^*$ тоже линеен.
\end{proposition}

\begin{proof}
	В определении сопряжённого оператора $g$ взято из линейного пространства, причём элементы пространства тоже сами линейны.
\end{proof}

\begin{definition}
	Пусть $E_1 = H_1, E_2 = H_2$ --- гильбертовы пространства, $A \in \cL(H_1, H_2)$. Тогда \textit{эрмитово сопряжённым оператором} $A^* \in E_2^* \to E_1^*$ называется оператор, удовлетворяющий условию:
	\[
		\forall x \in E_1, y \in E_2\ \ (Ax, y)_{H_2} = (x, A^*y)_{H_1}
	\]
\end{definition}

\begin{definition}
	Пусть $E_1 = E_2 = H$ --- гильбертово пространство. Тогда, если $A \in \cL(H)$ и $A^* = A$, то оператор $A$ называется \textit{самосопряжённым}.
\end{definition}

\begin{exercise}
	Доказать, что для любого оператора $A \in \cL(H)$ из гильбертова пространства существует единственный сопряжённый оператор.
\end{exercise}

\begin{theorem}
	Пусть $A \in \cL(E_1, E_2)$. Тогда $A^* \in \cL(E_2^*, E_1^*)$, причём $\|A^*\| = \|A\|$.
\end{theorem}

\begin{proof}
	Покажем неравенства для норм в 2 стороны:
	\begin{itemize}
		\item[$\le$] Верна следующая оценка:
		\[
			\forall g \in E_2^*, x \in E_1\ \ |(A^*g)x| = |g(Ax)| \le \|g\| \cdot \|Ax\| \le (\|g\| \cdot \|A\|) \cdot \|x\|
		\]
		Из последнего имеем $\|A^*g\| \le \|A\| \cdot \|g\|$, что напрямую означает $\|A^*\| \le \|A\|$
		
		\item[$\ge$] Так как $A^* \in \cL(E_2^*, E_1^*)$, то можно воспользоваться следствием теоремы Хана-Банаха для нормы элемента $Ax$:
		\[
			\forall x \in E_1\ \ \|Ax\| = \sup_{\|g\| = 1} |g(Ax)| = \sup_{\|g\| = 1} |(A^*g)x|
		\]
		При этом $\|(A^*g)x\| \le \|A^*\| \cdot 1 \cdot \|x\|$, а значит $\|Ax\| \le \|A^*\| \cdot \|x\| \Ra \|A\| \le \|A^*\|$, что и требовалось
	\end{itemize}
\end{proof}

\begin{exercise}
	Рассмотрим $H(\Cm)$ --- гильбертово пространство над комплексным полем. Тогда есть следующие свойства:
	\begin{enumerate}
		\item $\forall A, B \in \cL(H)\ \forall \alpha, \beta \in \Cm\ \ (\alpha A + \beta B)^* = \ole{\alpha}A^* + \ole{\beta}B^*$
		
		\item $\forall A \in \cL(H)\ A^{**} = A$
		
		\item $\forall A, B \in \cL(H)\ (AB)^* = B^*A^*$
		
		\item $I^* = I$
	\end{enumerate}
\end{exercise}

\begin{exercise}
	Пусть $H$ --- гильбертово пространство, $A \in \cL(H)$ --- самосопряжённый оператор. Если $M \subseteq H$ --- инвариантное относительно $A$ подпространство, то $M^\bot$ --- тоже инвариантно относительно $A$.
\end{exercise}