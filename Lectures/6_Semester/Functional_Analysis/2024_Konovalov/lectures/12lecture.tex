\section{Элементы нелинейного анализа}

\begin{note}
	В курсе математического анализа мы изучили функции одной переменной $F \colon \R \to \R$ и доказали, что понятия
	\begin{enumerate}
		\item существования производной
		\[
			\exists \lim_{\Delta x \to 0} \frac{f(x) - f(x_0)}{\Delta x} =: f'(x_0)
		\] 
		
		\item дифференцируемости
		\[
			\exists A(x_0) \in \R \such \Delta F = A\Delta x + o(\Delta x), \Delta x \to 0
		\]
	\end{enumerate}
	являются эквивалентными понятиями. При изучении функций многих переменных мы обнаружили, что из второго следует первое, но не наоборот (только при добавлении дополнительных условий). Однако, подобный эффект наблюдается лишь при рассмотрении функций над $\R$. Стоило нам прийти к ТФКП, как обнаружилась эквивалентность этих понятий для $F \colon \Cm \to \Cm$.
\end{note}

\begin{note}
	Далее мы рассматриваем банаховы пространства $E, E_1, E_2$ над полем $\R$, если не сказано обратного.
\end{note}

\begin{definition}
	Пусть $D \subseteq E_1$ --- открытое подмножество, $F \colon D \to E_2$. Тогда говорят, что $F$ \textit{дифференцируема по Фреше в точке $x_0 \in D$}, если существует оператор $A \in \cL(E_1, E_2)$ такой, что приращение можно представить в следующем виде:
	\[
		\Delta F = F(x_0 + h) - F(x_0) = Ah + o(\|h\|), h \to 0
	\]
\end{definition}

\begin{anote}
	Конечно же, выше под $o(\|h\|)$ понимается некоторый элемент пространства $E_2$, чья норма в свою очередь является $o(\|h\|), h \to 0$.
\end{anote}

\begin{note}
	Далее мы рассматриваем функции $F \colon D \to E_2$, где $D \subseteq E_1$ --- открытое подмножество, не упоминая этот факт.
\end{note}

\begin{proposition}
	Дифференцируемость по Фреше в точке $x_0 \in D$ эквивалентна следующему утверждению:
	\[
		\exists A \in \cL(E_1, E_2) \colon \exists \lim_{h \to 0} \frac{\|F(x_0 + h) - F(x_0) - Ah\|}{\|h\|} = 0
	\]
\end{proposition}

\begin{proof} \textcolor{red}{(не по лектору)}
	Если в определении дифференцируемости перенести $Ah$ в левую часть, то получится требуемое (почти: по модулю того факта, что предел из доказываемого утверждения эквивалентен $o(\|h\|)$, но это уже тривиальный факт математического анализа).
\end{proof}

\begin{anote}
	Далее, для указания факта линейности оператора по своему аргументу, будет удобно использовать квадратные скобки: $Ah = A[h]$ (если $A$ линеен, естественно)
\end{anote}

\begin{definition}
	Пусть $F \colon D \to E_2$ дифференцируема по Фреше в точке $x_0 \in D$. Тогда соответствующий оператор $A \in \cL(E_1, E_2)$ назывется \textit{производной (Фреше) $F$ в точке $x_0 \in D$}:
	\[
		F'(x_0) := A
	\]
\end{definition}

\begin{definition}
	Пусть $F \colon D \to E_2$ дифференцируема по Фреше в точке $x_0 \in D$. Тогда значение $Ah$ назывется \textit{дифференциалом $F$ в точке $x_0$ по приращению $h$}:
	\[
		dF(x_0, h) := Ah = F'(x_0)h = F'(x_0)[h]
	\]
\end{definition}

\begin{anote}
	Не знаю, как этого не было упомянуто на лекции, но производная по Фреше, если существует, единственна. Этот факт доказывается оценкой нормы разности двух операторов-производных.
\end{anote}

\begin{proposition}
	Пусть $F = const$. Тогда $F' = 0$.
\end{proposition}

\begin{proof} \textcolor{red}{(не по лектору)}
	По определению дифференцируемости по Фреше, нам нужен такой оператор $A \in \cL(E_1, E_2)$, что выполнено равенство:
	\[
		F(x_0 + h) - F(x_0) = 0 = Ah + o(\|h\|),\ h \to 0
	\]
	В такой ситуации понятно, что $Ah = o(\|h\|),\ h \to 0$, и очевидным образом $A = 0$ подходит.
\end{proof}

\begin{proposition}
	Пусть $F \in \cL(E_1, E_2)$. Тогда
	\[
		\forall x \in E_1\ \ F'(x) = F
	\]
\end{proposition}

\begin{proof} \textcolor{red}{(не по лектору)}
	Достаточно расписать разность, чтобы увидеть, как $F$ появится в нужном слагаемом:
	\[
		F(x_0 + h) - F(x_0) = F(h)
	\]
\end{proof}

\begin{proposition}
	Операция дифференцирования по Фреше линейна:
	\[
		\forall \alpha, \beta \in \R, F_1, F_2 \colon D \to E_2\ \ \ (\alpha_1F_1 + \alpha_2F_2)' = \alpha_1F'_1 + \alpha_2F'_2
	\]
\end{proposition}

\begin{proof} \textcolor{red}{(не по лектору)}
	Докажем этот факт в 2 шага:
	\begin{itemize}
		\item Умножение на скаляр. Пусть $\alpha \in \R$. Покажем, что $(\alpha F)' = \alpha F'$:
		\[
			(\alpha F)(x_0 + h) - (\alpha F)(x_0) = \alpha(F(x_0 + h) - F(x_0)) = \alpha F'(x_0) h + o(\|h\|),\ h \to 0
		\]
		
		\item Сложение двух операторов. Также честно распишем приращение $F_1 + F_2$:
		\begin{multline*}
			(F_1 + F_2)(x_0 + h) - (F_1 + F_2)(x_0) =
			\\
			(F_1(x_0 + h) - F_1(x_0)) + (F_2(x_0 + h) - F_2(x_0)) =
			\\
			F'_1(x_0)h + F'_2(x_0)h + o(\|h\|) = (F'_1 + F'_2)(x_0)h + o(\|h\|),\ h \to 0
		\end{multline*}
	\end{itemize}
\end{proof}

\begin{theorem} (Правило дифференцирования сложной функции)
	Пусть $F \colon E_1 \to E_2$, $G \colon E_2 \to E_3$. Тогда для $H = G \circ F$ верна формула:
	\[
		H'(x_0) = G'(y_0) \circ F'(x_0),\ y_0 = F(x_0)
	\]
\end{theorem}

\textcolor{red}{Добавить картинку}

\begin{proposition} \textcolor{red}{(не по лектору)}
	Если $F$ дифференцируема по Фреше, то она непрерывна.
\end{proposition}

\begin{proof}
	Действительно, заметим, что предел правой части из определения равен нулю при стремлении $h \to 0$, что в точности означает непрерывность.
\end{proof}

\begin{proof} (теоремы)
	Для удобства, заменим обозначения: $(E_1, E_2, E_3) = (X, Y, Z)$. Воспользуемся фактом дифференцируемости $F$ в точке $x_0 \in X$:
	\[
		F(x_0 + h) - F(x_0) = \Delta F = F'(x_0)\Delta x + \underbrace{\eps_1(\Delta x)}_{\in Y}\|\Delta x\|,\ \lim_{\Delta x \to 0} \eps_1(\Delta x) = 0
	\]
	Аналогично запишем для $G$:
	\[
		G(y_0 + t) - G(y_0) = \Delta G = G'(y_0)\Delta y + \underbrace{\eps_2(\Delta y)}_{\in Z}\|\Delta y\|
	\]
	В силу непрерывности, мы можем рассмотреть $t = F(x_0 + h) - F(x_0)$. Тогда $t \to 0$ при $h \to 0$. Более того, мы можем подставить первую формулу во вторую:
	\[
		\Delta G = G'(y_0)\sbr{F'(x_0)\Delta x + \eps_1(\Delta x)\|\Delta x\|} + \eps_2(\Delta y)\|\Delta y\|, h \to 0
	\]
	Если мы покажем, что $G'(y_0)[\eps_1(\Delta x)\|\Delta x\|] + \eps_2(\Delta y)\|\Delta y\| = o(\|\Delta x\|)$, то всё будет доказано. А это действительно так, разберёмся с каждым слагаемым отдельно. Для первого квадратные скобки (конкретно тут) говорят нам о том, что оператор $G'(y_0)$ линеен, и даже непрерывен. Отсюда
	\[
		\Delta G = G'(y_0)F'(x_0)\Delta x + G'(y_0)[\eps_1(\Delta x)]\|\Delta x\| + \eps_2(\Delta y)\|\Delta y\|, h \to 0
	\]
	Для второго --- распишем $\|\Delta y\|$:
	\[
		\|\Delta y\| = \|F'(x_0)[\Delta x] + \eps_1(\Delta x)\|\Delta x\|\| \le \|F'(x_0)\| \cdot \|\Delta x\| + \|\eps_1(\Delta x)\| \cdot \|\Delta x\|
	\]
	Получили, что $\|\Delta y\| = O(\|\Delta x\|)$ при $\Delta x \to 0$. Так как $\eps_2(\Delta y) \to 0,\ \Delta x \to 0$, то $\eps_2(\Delta y)\|\Delta y\| = o(\|\Delta x\|), \Delta x \to 0$
\end{proof}

\begin{definition}
	\textit{Дифференциалом по Гато функции $F$ в точке $x_0 \in D$ по приращению $h$} называется следующее число:
	\[
		DF(x_0, h) := \frac{d}{dt}F(x_0 + th)|_{t = 0}
	\]
\end{definition}

\begin{definition}
	Если для дифференциала по Гато функции $F$ в точке $x_0$ существует оператор $A \in \cL(E_1, E_2)$ такой, что
	\[
		DF(x_0, h) = Ah
	\]
	то он называется \textit{производной по Гато}.
\end{definition}

\begin{note}
	Производную по Гато также называют \textit{слабой производной}. Соответственно, производная по Фреше --- \textit{сильная производная}.
\end{note}

\begin{exercise}
	Если $F$ дифференцируема по Фреше в точке $x_0 \in D$, то она дифференцируема по Гато в той же точке, причём производные Фреше и Гато совпадают.
\end{exercise}

\begin{exercise}
	Из дифференцируемости по Гато не следует дифференцируемость по Фреше.
\end{exercise}

\begin{solution}
	Рассмотрим плоскость $XY$. Проведём на ней параболу $y = x^2$. Положим функционал $F$ таким образом, что он равен 1 при $y = x^2$ и 0 иначе. Тогда производная Гато будет известна для любого направления и равна 0, тогда как производной Фреше существовать не будет.
\end{solution}

\begin{note}
	Для чего изучать нелинейный анализ?
	\begin{enumerate}
		\item Доказательство того, что $F$ является сжимающим отображением. Для этого было бы хорошо получить аналог теоремы Лагранжа в общем случае
		
		\item Поиск экстремумов --- по сути, теория оптимизации (оптимального управления)
		
		\item Решение уравнений (метод Ньютона)
	\end{enumerate}
\end{note}

\begin{theorem} (о среднем)
	Пусть $D \subseteq E_1$ --- линейно связанное открытое множество, $F$ --- дифференцируема по Фреше на $D$. Тогда верно неравенство:
	\[
		\forall x_0, x_1 \in D\ \ \|F(x_1) - F(x_0)\| \le \sup_{y \in s(x_0, x_1)} \|F'(y)\| \cdot \|x_1 - x_0\|
	\]
	где $s(x_0, x_1)$ --- интервал от $x_0$ до $x_1$.
\end{theorem}

\begin{proof}
	\textcolor{red}{Добавить картинку <<сэндвича>>}
	
	Вся идея в том, чтобы построить сквозное отображение $\phi \colon [0; 1] \to E_1 \to E_2 \to \R$ и применить к нему теорему Лагранжа. Итак, $x(t) = x_0 + t(x_1 - x_0)$, а $f \in E_2^*$ --- произвольный функционал. Определим $\phi$ следующим образом:
	\[
		\phi(t) = (f \circ F \circ x)(t)
	\]
	Каждая часть композиции является дифференцируемой по Фреше функцией. Стало быть, и их комбинация дифференцируема:
	\[
		\phi'(t) = f'(F(x(t))) \circ F'(x(t)) \circ x'(t) = f[F'(x(t))[x'(t)]] = f[F'(x(t))[x_1 - x_0]]
	\]
	Теперь, применим теорему Лагранжа для всего отрезка $[0; 1]$:
	\[
		|\phi(1) - \phi(0)| = |\phi'(\xi)| \cdot (1 - 0)
	\]
	Разберёмся с левой частью. Она переписывается следующим образом:
	\[
		|\phi(1) - \phi(0)| = |f[F(x_1)] - f[F(x_0)]| = |f[F(x_1) - F(x_0)]| \le \|f\| \cdot \|F(x_1) - F(x_0)\|
	\]
	В этот момент нужно вспомнить теорему Хана-Банаха. Одним из её следствий было то, что для произвольного ненулевого элемента можно подобрать функционал с единичной нормой, который на этом же элементе принимает значение --- норму этого элемента. Воспользуемся этим следствием, чтобы найти $f$ по точке $F(x_1) - F(x_0)$. Тогда неравенство выше превращается в равенство:
	\[
		\exists f \in E_2^* \colon |\phi(1) - \phi(0)| = |f[F(x_1) - F(x_0)]| = \|F(x_1) - F(x_0)\|
	\]
	Итак, соберём всё вместе:
	\begin{multline*}
		\|F(x_1) - F(x_0)\| = |f[F'(x(\xi))[x_1 - x_0]]| \le
		\\
		\|f\| \cdot \|F'(x(\xi))\| \cdot \|x_1 - x_0\| \le \sup_{y \in s(x_0, x_1)} \|F'(y)\| \cdot \|x_1 - x_0\|
	\end{multline*}
\end{proof}