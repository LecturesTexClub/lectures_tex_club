\begin{definition}
	\textit{Оператором Гильберта-Шмидта} называется частный случай оператора Фредгольма в $L_2[a; b]$:
	\[
		(Af)(x) = \int_a^b K(x, t)f(t)d\mu(t),\ K \in L_2([a; b]^2)
	\]
\end{definition}

\begin{proposition}
	Оператор Гильберта-Шмидта отображает в $L_2[a; b]$.
\end{proposition}

\begin{proof}
	Раз $K \in L_2([a; b]^2)$, то как функция по одному из своих аргументов $K$ тоже будет из $L_2[a; b]$. Коль скоро пространство $L_2[a; b]$ гильбертово, значение оператора Гильберта-Шмидта оценивается так:
	\begin{multline*}
		|(Af)(x)|^2 = \md{\int_a^b K(x, t)f(t)d\mu(t)}^2 = |(K(x, t), f(t))|^2 \le
		\\
		\int_a^b |K(x, t)|^2d\mu(t) \cdot \int_a^b |f(t)|^2d\mu(t) \le \|f\|^2 \cdot \int_a^b |K(x, t)|^2d\mu(t) < \infty
	\end{multline*}
	Отсюда также можно получить оценку на 2-норму для $Af$:
	\[
		\|Af\|^2 = \int_a^b |(Af)(x)|^2d\mu(x) \le \|f\|^2 \int_a^b \int_a^b |K(x, t)|d\mu(t)d\mu(x) < \infty
	\]
	Стало быть, $(Af) \in L_2[a; b]$.
\end{proof}

\begin{theorem}
	Оператор Гильберта-Шмидта является компактным оператором на $L_2[a; b]$.
\end{theorem}

\begin{reminder}
	Если $\{\phi_n\}_{n = 1}^\infty \subseteq L_2[a; b]$ --- ортонормированный базис, то \\ $\{\phi_n(x)\phi_m(y)\}_{n, m = 1}^\infty \subseteq L_2([a; b]^2)$ --- тоже ортонормированный базис.
\end{reminder}

\begin{proof}
	$L_2[a; b]$ --- сепарабельное гильбертово пространство, поэтому в нём точно есть ортонормированный базис $\{\phi_n\}_{n = 1}^\infty$. Идея состоит в том, чтобы найти последовательность компактных операторов ${A_N}_{N = 1}^\infty \subseteq K(L_2[a; b])$, которые сходятся по норме к $A$. Тогда всё доказано по теореме \ref{compact_approx_th}. Итак, можно разложить ядро $K$ по вышеупомянутому базису:
	\[
		K(x, t) = \sum_{n, m = 1}^\infty c_{n, m}\phi_n(x)\phi_m(t)
	\]
	Возьмём за отдельные ядра --- <<срезки>> от ряда выше:
	\[
		K_N(x, t) = \sum_{n, m = 1}^N c_{n, m}\phi_n(x)\phi_m(t)
	\]
	Тогда, тривиальным образом $A_Nf(x) = \int_a^b K_N(x, t)f(t)d\mu(t) \in K(L_2[a; b])$. Осталось вспомнить, что норма оператора Фредгольма оценивается сверху 2-нормой ядра, а значит:
	\[
		\|A - A_N\| \le \|K - K_N\| \xrightarrow[N \to \infty]{} 0
	\]
\end{proof}

\begin{definition}
	Пусть $H$ --- гильбертово пространство, $\{e_n\}_{n = 1}^\infty$ - его базис. \textit{Классом операторов Гильберта-Шмидта} называется следующее множество операторов $A \in \cL(H)$ таких, что
	\[
		\sum_{n = 1}^\infty \|Ae_n\|^2 < \infty
	\]
\end{definition}

\begin{proposition} (без доказательства)
	Класс операторов Гильберта-Шмидта вложен в класс компактных операторов.
\end{proposition}