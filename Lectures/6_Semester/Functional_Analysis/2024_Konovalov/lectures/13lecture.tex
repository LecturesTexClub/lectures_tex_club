\section{Преобразование Фурье и свёртка}

\begin{note}
	Всюду далее мы работаем с комплекснозначными функциями.
\end{note}

\subsection{В пространстве $L_1(\R)$}

\begin{definition}
	Пусть $f \in L_1(\R)$. Тогда \textit{преобразованием Фурье функции $f$} называется функция, заданная следующим образом:
	\[
		\wdh{f}(y) = F[f](y) := \int_{\R} f(x)e^{-ixy}d\mu(x)
	\]
\end{definition}

\begin{note}
	Несложно видеть, что $F$ --- линейный оператор.
\end{note}

\begin{definition} \textcolor{red}{(не по лектору)}
	\textit{Обратным преобразованием Фурье функции $g$} называется функция, заданная следующим образом:
	\[
		\wdt{g}(x) = F^{-1}[g](x) := \frac{1}{2\pi} \text{v.p.} \int_\R g(y)e^{ixy}d\mu(y)
	\]
\end{definition}

\begin{proposition}
	Преобразование Фурье отображает функции из $L_1(\R)$ в $B(\R)$ --- множество ограниченных функций.
\end{proposition}

\begin{proof}
	\[
		\md{F[f](y)} \le \int_{\R} |f(x)| \cdot 1 d\mu(x) = \|f\|_{1} \Ra \|F[f]\|_\infty \le \|f\|_1
	\]
\end{proof}

\begin{corollary}
	Преобразование Фурье --- непрерывный оператор, $\|F\| \le 1$.
\end{corollary}

\begin{proposition}
	Преобразование Фурье отображает функции из $L_1(\R)$ в $C_0(\R)$ -- множество непрерывных функций, стремящиеся к нулю на бесконечности.
\end{proposition}

\begin{proof}
	\[
		\wdh{I_{[a; b]}}(y) = \frac{e^{-iay} - e^{-iby}}{iy} \in C_0(\R)
	\]
	В силу того, что у индикаторов отличное преобразование Фурье, при этом любая функция из $L_1(\R)$ приближается (по определению) ступенчатыми, то во-первых, преобразование Фурье любой ступенчатой функции лежит в $C_0(\R)$ (ибо она представима как конечная сумма индикаторов с коэффициентами), а во-вторых, $F$ --- непрерывный оператор, и поэтому образы будут равномерно сходиться (что влечёт непрерывность предела).
\end{proof}

\begin{exercise}
	Преобразование Фурье $L_1(\R) \to C_0(\R)$ не является компактным.
\end{exercise}

\begin{proposition} (формула умножения)
	Пусть $f, g \in L_1(\R)$, тогда
	\[
		\int_\R f(y)\wdh{g}(y)d\mu(y) = \int_\R \wdh{f}(y)g(y)d\mu(y)
	\]
\end{proposition}

\begin{reminder} (следствие теоремы Фубини)
	Пусть $f \colon \R \to \R$ --- измеримая функция, причём конечен хотя бы один из интегралов
	\[
		\int_\R \int_\R |f(x, y)|d\mu(y)d\mu(x) < \infty \vee \int_\R \int_\R |f(x, y)|d\mu(x)d\mu(y) < \infty
	\]
	Тогда $f$ интегрируема и применима теорема Фубини.
\end{reminder}

\begin{proof}
	Распишем преобразование Фурье по определению:
	\[
		\int_\R f(y)\wdh{g}(y)d\mu(y) = \int_\R \int_\R f(y)g(x)e^{-ixy}d\mu(x)d\mu(y)
	\]
	Покажем, что мы можем переставить интегралы (по следствию теоремы Фубини). Для этого оценим повторный интеграл:
	\begin{multline*}
		\int_\R \int_\R |f(y)g(x)e^{-ixy}|d\mu(x)d\mu(y) = \int_\R \int_\R |f(y)g(x)|d\mu(x)d\mu(y) \le
		\\
		\|g\|_1\int_\R |f(y)|d\mu(y) \le \|f\|_1\|g\|_1
	\end{multline*}
	Дальнейшие рассуждения тривиальны --- собрать внутренний интеграл в преобразование Фурье $f$.
\end{proof}

\begin{definition}
	Пусть $f, g \in L_1(\R)$. Тогда \textit{свёрткой функций $f$ и $g$} называется следующая функция:
	\[
		(f * g)(x) = \int_\R f(y)g(x - y)d\mu(y)
	\]
\end{definition}

\begin{note}
	История свёртки тесно связана с умножением многочленов. Рассмотрим произведение многочленов:
	\[
		\sum_{k = 0}^K c_kz^k := \ps{\sum_{n = 0}^N a_nz^n}\ps{\sum_{m = 0}^M b_mz^m}
	\]
	Тогда известно, что коэффициенты результирующего многочлена можно вычислить по следующей формуле:
	\[
		c(k) := c_k = \sum_{m = 0}^k a_mb_{m - k} = \sum_{m = 0}^k a(m)b(m - k)
	\]
	Недостающие слагаемые многочленов просто заменяем нулём.
\end{note}

\begin{proposition}
	Свёртка функций $f, g \in L_1(\R)$ тоже лежит в пространстве $L_1(\R)$.
\end{proposition}

\begin{proof}
	Не будем доказывать существование свёртки напрямую. Вместо этого, покажем, что сразу ограничен интеграл от модуля свёртки. Тогда всё будет следовать по теореме Фубини (пояснение под формулами):
	\begin{multline*}
		\int_\R \int_\R |f(y)g(x - y)|d\mu(y)d\mu(x) = \int_\R \int_\R |f(y)g(x - y)|d\mu(x)d\mu(y) =
		\\
		\int_\R |f(y)| \int_\R |g(x - y)|d\mu(x)d\mu(y) = \int_\R |f(y)| \int_\R |g(x - y)|d\mu(x - y)d\mu(y) =
		\\
		\int_\R |f(y)|d\mu(y) \cdot \int_\R |g(t)|d\mu(t) \le \|f\|_1 \cdot \|g\|_1
	\end{multline*}
	На первый взгляд, даже первое равенство уже совершено необосновано, но на самом деле --- так и задумано. Мы смогли оценить второй интеграл первой строки, а значит для него переставлять интегралы можно! Следовательно, первое равенство обосновано. А в силу уже этой оценки, мы можем применить следствие теоремы Фубини к двойному интегралу без модуля, чем доказывается корректность свёртки и требуемое утверждение.
\end{proof}