\begin{theorem} \label{conj_decomp_th}
	Пусть $H$ --- гильбертово пространство, $A \in \cL(H)$. Тогда
	\[
		H = \cl(\im A) \oplus \ker A^*
	\]
\end{theorem}

\begin{proof}~
	\begin{enumerate}
		\item Покажем, что $(\im A)^\bot = \ker A^*$. Для этого рассмотрим произвольный элемент ортогонального дополнения:
		\[
			\forall y \in (\im A)^\bot\ \forall x \in H\ (Ax, y) = 0
		\]
		Стало быть, для любых $x, y$ выше будет $(x, A^*y) = 0$, а в силу гильбертовости пространства это означает, что $A^*y = 0$, при этом $A^*$ биективен. Это возможно тогда и только тогда, когда $y = 0$.
		
		\item Заметим, что $(\im A)^\bot = (\cl(\im A))^\bot$. Так как последнее является подпространством, то по теореме о проекции получаем требуемое разложение:
		\[
			H = \cl(\im A) \oplus (\cl(\im A))^\bot = \cl(\im A) \oplus \ker A^*
		\]
	\end{enumerate}
\end{proof}

\begin{theorem} (Банаха, об обратном операторе)
	Пусть $H$ --- гильбертово пространство над полем $\Cm$, $A \in \cL(H)$ и $A$ --- биекция. Тогда $A^{-1} \in \cL(H)$ --- непрерывный ограниченный оператор.
\end{theorem}

\begin{proof}
	Основная идея состоит в том, чтобы доказать утверждение теоремы не для $A$, а для $A^*$. Запишем 2 разложения пространства $H$:
	\begin{align*}
		&{\cl(\im A) \oplus \ker A^* = H}
		\\
		&{\cl(\im A^*) \oplus \ker A = H}
	\end{align*}
	Так как $A$ биективен, то $\ker A = \{0\}$ и мы сразу получаем $\cl(\im A^*) = H$. С другой стороны, $\cl(\im A) = \im A = H$, а потому $\ker A^* = \{0\}$.
	\begin{lemma} \label{conj_op_bot_bounded_lemma}
		Оператор $A^*$ ограничен снизу
	\end{lemma}
	
	\begin{proof}
		Напомним, что означает это свойство:
		\[
			\exists m > 0 \such \forall x \in H\ \ \|A^*x\| \ge m\|x\|
		\]
		Понятно, что это утверждение эквивалентно такому же, но где рассматриваются не абсолютно все элементы $H$, а просто для каждого направления, соответствующего $x \in H$, будет потребован хотя бы какой-то $x_0$ с соответствующим неравенством. Это работает в силу линейности, мы уже это видели в эквивалентных определениях нормы оператора. Итак, рассмотрим $S = \{x \in H \colon \|A^*x\| = 1\}$. Условие перепишется так:
		\[
			\exists m > 0 \such \forall x \in S\ \ \|x\| \le \frac{1}{m}
		\]
		По сути надо доказать ограниченность $S$. Сделаем это через доказательство, что $S$ слабо ограничено. Нужное свойство можно записать так:
		\[
			\forall y \in H\ \exists K_y \in \R_+ \such \forall x \in S\ \ |(x, y)| \le K_y
		\]
		Вспомним, что $A$ --- сюръекция, а значит для любого $y \in H$ мы найдём $z \in H$ такой, что $Az = y$. Отсюда:
		\[
			\forall z \in H\ \exists K_z \in \R_+ \such \forall x \in S\ \ |(x, Az)| = |(A^*x, z)| \le 1 \cdot \|z\| =: K_z
		\]
	\end{proof}
	
	
	\begin{lemma}
		Пусть $B \in \cL(H)$ и $B$ ограничен снизу. Тогда $\cl(\im B) = \im B$.
	\end{lemma}
	
	\begin{proof}
		Пусть $y_n \in \im B$ и $\lim_{n \to \infty} y_n = y$. Докажем, что $y \in \im B$. В силу сходимости есть и фундаментальность:
		\[
			\forall p \in \N\ \ \|y_{n + p} - y_n\| \xrightarrow[n \to \infty]{} 0
		\]
		Коль скоро $y_n \in \im B$, то можно переписать фундаментальность:
		\[
			\forall p \in \N\ \ \|y_{n + p} - y_n\| = \|Bz_{n + p} - Bz_n\| \ge m\|z_{n + p} - z_n\|
		\]
		Так как левая часть неравенства стремится к нулю, то и правая тоже. Стало быть, $z_n$ фундаментальна, а в силу полноты $H$ должен существовать предел $\lim_{n \to \infty} z_n = z$. Тогда $y = \lim_{n \to \infty} Bz_n = B( \lim_{n \to \infty} z_n) = Bz$.
	\end{proof}
	
	Итак, $\cl(\im A^*) = \im A^* = H$ и $\ker A^* = \{0\}$, то есть $A^*$ --- биекция. Значит, применима теорема о существовании обратного оператора $(A^*)^{-1}$, который сразу же непрерывен на $\im A^* = H$, то есть требуемое про $A^*$ мы доказали. Осталось сказать, что $A^{**} = (A^*)^* = A$, ну а сам $A^*$ все условия, которые были изначально наложены на $A$, сохранил, поэтому требуемое тоже доказано.
\end{proof}