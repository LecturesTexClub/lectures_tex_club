\begin{theorem} (Критерий слабой сходимости)
	Пусть $x_n, x \in E$. Тогда $x_n \wto x$ тогда и только тогда, когда выполнено 2 условия:
	\begin{itemize}
		\item Нормы последовательности $\|x_n\|$ ограниничены
		
		\item Существует множество $S \subseteq E^*$ такое, что $\cl [S] = E^*$ и есть поточечная сходимость $\forall f \in S\ f(x_n) \to f(x)$
	\end{itemize}
\end{theorem}

\begin{proof}
	Перейдём к рассмотрению операторов $F_{x_n}, F_x \in E^*$. Тогда слабая сходимость $x_n \wto x$ по определению является поточечной сходимостью $F_{x_n}(f) \to F_x(f)$. Из условия:
	\begin{itemize}
		\item Пространство $E^*$ полно
		
		\item Нормы $\|F_{x_n}\| = \|x_n\|$ ограничены
		
		\item $\exists S \subseteq E^* \such \cl [S] = E^* \wedge \forall f \in S\ \ F_{x_n}(f) \to F_x(f)$
	\end{itemize}
	Эти условия не могут не напомнить о критерии поточечной сходимости для линейных ограниченных операторов. А поточечная сходимость наших операторов во всём пространстве соответствует $x_n \wto x$.
\end{proof}

\begin{note}
	\textcolor{red}{В случае рефлексивности и банаховости пространства $E$} условие для слабой сходимости можно ослабить. Достаточно потребовать не $f(x_n) \to f(x)$ (и соответственно знания конкретного $x$), а существования предела $\lim_{n \to \infty} f(x_n)$. В предыдущем семестре мы доказали теорему о полноте относительно поточечной сходимости (её нужно применить к $F_{x_n}$ и $F_x$).
\end{note}

\begin{anote}
	Ради пущего убеждения, напишу здесь явно доказательство слабой версии, а точнее доказательство существования нужного $x$.
\end{anote}

\begin{theorem}
	Пусть $E_1, E_2$ --- нормированные пространства, $x_n, x \in E_1$, причём $x_n \wto x$, а также $A \in \cL(E_1, E_2)$. Тогда, есть слабая сходимость образов $Ax_n \wto Ax$.
\end{theorem}

\begin{proof}
	По определению слабой сходимости
	\[
		\forall f \in E_1^*\ \ f(x_n) \xrightarrow[n \to \infty]{} f(x)
	\]
	В частности, можно рассмотреть функционал $f = g \circ A$ для любого $g \in E_2^*$. Тогда
	\[
		\forall g \in E_2^*\ \ g(Ax_n) \xrightarrow[n \to \infty]{} g(Ax)
	\]
	Это утверждение в точности совпадает с определением слабой сходимости $Ax_n \wto Ax$.
\end{proof}

\begin{definition}
	Множество $S \subseteq E$ называется \textit{слабо ограниченным}, если
	\[
		\forall f \in E^*\ \ f(S) \text{ --- ограниченное множество в $\K$}
	\]
\end{definition}

\begin{proposition}
	Пусть $S \subseteq E$ --- ограниченное множество. Тогда $S$ слабо ограничено.
\end{proposition}

\begin{proof}
	По определению, если $f \in E^*$, то это линейный ограниченный функционал. Ограниченный функционал переводит ограниченные множества в ограниченные, по определению. Поэтому слабая ограниченность $S$ тривиальна.
\end{proof}

\begin{theorem} (Хана, 1922г.)
	Пусть $S \subseteq E$ --- слабо ограниченное множество. Тогда $S$ ограничено.
\end{theorem}

\begin{proof}
	Предположим противное, то есть $S$ неограничено. Тогда
	\[
		\forall n \in \N\ \ \exists x_n \in S \such \|x_n\| \ge n^2
	\]
	Рассмотрим последовательность $y_n = \frac{x_n}{n}$. В силу слабой ограниченности, мы можем сделать следующую оценку на образ $f(y_n)$, $f \in E^*$ (где $K_f$ --- константа, ограничивающая образ $f(S)$):
	\[
		\forall f \in E^*\ \ |f(y_n)| = \frac{|f(x_n)|}{n} \le \frac{K_f}{n} \xrightarrow[n \to \infty]{} 0
	\]
	Стало быть, $y_n \wto 0$. В силу критерия слабой сходимости, $\|y_n\| \le M$ --- есть ограниченность норм. Стало быть
	\[
		\forall n \in \N\ \ M \ge \|y_n\| = \frac{\|x_n\|}{n} \ge \frac{n^2}{n} = n
	\]
	Получили противоречие.
\end{proof}

\begin{definition}
	Последовательность $\{x_n\}_{n = 1}^\infty \subseteq E$ называется \textit{слабо фундаментальной}, если
	\[
		\forall f \in E^*\ \ \{f(x_n)\}_{n = 1}^\infty \subset \K \text{ --- фундаментальная последовательность}
	\]
\end{definition}

\begin{proposition}
	Слабо сходящаяся последовательность всегда является слабо фундаментальной
\end{proposition}

\begin{proof}
	Раз последовательность слабо сходящаяся, то
	\[
		\forall f \in E^*\ \exists \lim_{n \to \infty} f(x_n)
	\]
	При этом $f(x_n) \subseteq \K$ и поле $\K$ полно, а значит последовательность $\{f(x_n)\}_{n = 1}^\infty$ фундаментальна при любом $f$.
\end{proof}

\begin{definition}
	Пространство $E$ называется \textit{слабо полным}, если любая слабо фундамантальная последовательность в нём является слабо сходящейся.
\end{definition}

\begin{definition}
	Множество $S \subseteq E$ называется \textit{слабо секвенциально компактным (или секвенциально слабо компактным)}, если из любой ограниченной последовательности можно выделить слабо сходящуюся подпоследовательность.
\end{definition}

\begin{theorem} (Банаха, 1932г.)
	Пусть $H$ --- гильбертово пространство. Тогда $\ole{B}(0, R)$ --- слабо секвенциально компактное множество.
\end{theorem}

\begin{proof}
	Мы будем действовать согласно следующему плану:
	\begin{enumerate}
		\item Рассмотрим любую последовательность $\{x_n\}_{n = 1}^\infty \subseteq \ole{B}(0, R)$. Хотим показать, что в ней выделяется слабо сходящаяся подпоследовательность $\{x_{n_k}\}_{k = 1}^\infty$
		
		\item Рассмотрим $L = \cl [\{x_n\}_{n = 1}^\infty]$. В силу гильбертовости пространства $H$, мы можем воспользоваться теоремой о проекции. Тогда $H = L \oplus L^\bot$
		
		\item Выделить такую подпоследовательность $\{y_k\}_{k = 1}^\infty \subseteq \{x_n\}_{n = 1}^\infty$, что есть сходимость для любого скалярного произведения с $x_m$:
		\[
			\forall m \in \N\ \exists \lim_{k \to \infty} (x_m, y_k)
		\]
		Тогда, в силу критерия слабой сходимости, $y_k$ будет слабо сходящейся последовательностью в $L$
		
		\item Заметим, что из имеющейся сходимости следует слабая сходимость и во всём пространстве $H$:
		\[
			 H = L \oplus L^\bot \Ra \forall h = l + l^\bot\ \ (y_k, h) = (y_k, l) + (y_k, l^\bot) = (y_k, l)
		\]
		А $(y_k, l)$ сходится в силу результата предыдущего пункта.
	\end{enumerate}
	Единственное нестрогое место в плане --- пункт 3, выделение слабо сходящейся последовательности. Воспользуемся диагональным методом Кантора:
	\begin{enumerate}
		\item Зафиксируем $x_m$. Тогда $(x_m, x_n) \le R^2$ и, получается, $\{(x_m, x_n)\}_{n = 1}^\infty$ является ограниченной последовательностью чисел. По теореме Больцано-Вейерштрасса, из неё можно выделить сходящуюся подпоследовательность $x_{n_k}$
		
		\item Итерируемся по $m \in \N$ (с началом $m = 1$ и последовательностью $x_n$) и выделяем новую подпоследовательность из той, что была получена на предыдущем шаге. Обозначим их как $x_{m, n}$ ($x_{1, n} = x_n$)
		
		\item Получили искомую последовательность $y_k = x_{k, k}$
	\end{enumerate}
\end{proof}