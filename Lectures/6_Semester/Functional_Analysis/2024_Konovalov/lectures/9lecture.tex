\begin{proposition}
	Если $\dim E = \infty$, то тождественный оператор $I \notin K(E)$.
\end{proposition}

\begin{proof}
	Действительно, по теореме Рисса мы знаем, что замкнутый единичный шар в таком пространстве не компактен, а значит $B(0, 1) = I(B(0, 1))$ не может быть предкомпактом.
\end{proof}

\begin{corollary}
	Если $\dim E = \infty$, $A \in K(E)$, то $A^{-1} \notin \cL(E)$.
\end{corollary}

\begin{proof}
	Предположим противное. Тогда $I = AA^{-1} \in K(E)$, чего не может быть.
\end{proof}

\begin{proposition}
	Если $A \in K(E_1, E_2)$ и $x_n \wto x_0 \in E_1$, то $Ax_n \to Ax_0$.
\end{proposition}

\begin{proof}
	Пусть $x_n \wto x_0$. Тогда $\{x_n\}_{n = 1}^\infty$ --- ограниченная последовательность, а значит $\{Ax_n\}_{n = 1}^\infty$ --- предкомпакт. Более того, из слабой сходимости аргументов и непрерывности $A$ следует слабая сходимость $Ax_n \wto Ax_0$.
	\begin{lemma}
		Пусть $y_n \in E$ таково, что $\{y_n\}_{n = 1}^\infty$ --- предкомпакт и $y_n \wto y_0$. Тогда $y_n \to y_0$.
	\end{lemma}
	
	\begin{proof}
		Предположим противное. Тогда:
		\[
			\exists \eps_0 > 0, \{n_k\}_{k = 1}^\infty \such \|y_{n_k} - y_0\| \ge \eps_0
		\]
		Так как $\{y_n\}_{n = 1}^\infty$ --- предкомпакт, то и $\{y_{n_k}\}_{k = 1}^\infty$ --- тоже. Значит, можно выделить сходящуюся под-подпоследовательность:
		\[
			\exists \{y_{n_{k_l}}\}_{l = 1}^\infty \subseteq E \such y_{n_{k_l}} \xrightarrow[l \to \infty]{} y'_0
		\]
		Так как из сходимости по норме следует слабая сходимость, для которой предел единственнен по теореме Хана-Банаха, возможна лишь ситуация $y'_0 = y_0$. Получили противоречие с тем, что последовательность $y_n$ не могла приблизится к $y_0$.
	\end{proof}
	Для завершения доказательства утверждения осталось воспользоваться леммой.
\end{proof}

\begin{task} (трудная)
	Если $E$ --- рефлексивное и оператор $A \in \cL(E)$  таков, что
	\[
		\forall x_n \in E \colon x_n \wto x_0 \Ra Ax_n \xrightarrow[n \to \infty]{} Ax_0
	\]
	Тогда $A \in K(E)$.
\end{task}

\begin{theorem} \label{compact_approx_th}
	Пусть $E_2$ --- банахово пространство, $A_n \in K(E_1, E_2)$, $A \in \cL(E)$, причём $\lim_{n \to \infty} A_n = A$. Тогда $A \in K(E_1, E_2)$.
\end{theorem}

\begin{proof}
	Идея состоит в повторении доказательства уже встречавшегося в курсе математического анализа утверждения:
	\[
		f_n \in C[a; b] \wedge f_n \rra f \Lora f \in C[a; b]
	\]
	В силу банаховости $E_2$, для компактности оператора $A$ достаточно проверить, что $A(B(0, 1))$ является вполне ограниченным множеством. Идея состоит в том, чтобы взять достаточно близкий оператор $A_n$, взять соответствующую ему $\eps$-сеть и заявить, что она подойдёт к $A$:
	\begin{itemize}
		\item $\forall \eps > 0\ \exists n_0 \in \N \such \|A - A_{n_0}\| < \eps$
		
		\item $\forall \eps > 0\ \exists \{y_t\}_{t = 1}^T \subseteq E \such \forall x \in B(0, 1)\ \exists s \such \|A_{n_0}x - y_s\| < \eps$
	\end{itemize}
	Зафиксируем $\eps > 0$, $n_0 \in \N$ и $\{y_t\}_{t = 1}^T \subseteq E$ согласно утверждениям выше. Тогда (далее $y_s$ соответствует $A_{n_0}x$):
	\[
		\forall x \in B(0, 1)\ \exists y_s \such \|Ax - y_s\| \le \|Ax - A_{n_0}x\| + \|A_{n_0}x - y_s\| < 2\eps
	\]
	Стало быть, $\{y_t\}_{t = 1}^T$ --- это $2\eps$-сеть для $A(B(0, 1))$, то есть образ вполне ограничен.
\end{proof}

\begin{note}
	В своё время эту задачу в случае сепарабельных пространств изучали Банах и Мазур. Достаточно скоро стало понятно, что эта задача тесно связана с существованием базиса Шаудера в таких пространствах, а ответ в конечном итоге дал Пер \'{Э}нфло в 1972г. Он конструктивно показал пространство, в котором утверждение теоремы оказывается неверным.
\end{note}