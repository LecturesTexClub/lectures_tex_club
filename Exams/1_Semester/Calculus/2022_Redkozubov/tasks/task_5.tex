\section{5. Непрерывность функции в точке. Равносильные определения непрерывности. Теорема о непрерывности композиции. Точки разрыва, их классификация. Теорема о разрывах монотонной функции.}
    %\overset{o}{B_{\delta}}
    \begin{definition}
        Пусть $E \subset \R$, задана функция $f: E \to \R$ и $a \in E$. Функция $f$ называется \textit{непрерывной} в точке $a$, если
        \[\forall \epsilon > 0 \ \exists \delta > 0 \ \forall x \in E \ (x \in B_{\delta}(a) \Rightarrow f(x) \in B_{\epsilon}(f(a)))\]
    \end{definition}

    \begin{theorem}
        Пусть $f: E \to \R$, $a \in E$. Следующие условия эквивалентны:
        \begin{enumerate}
            \item функция $f$ непрерывна в точке $a$;
            \item $\forall \{x_{n}\}$, $x_{n} \in E \ (x_{n} \to a \Rightarrow f(x_{n}) \to f(a))$;
            \item $a$ -- изолированная точка множества $E$, или $a$ -- предельная точка $E$ и $\lim_{x \to a} f(x) = f(a)$.
        \end{enumerate}
    \end{theorem}

    \begin{proof} \ \\
        (1) $\Rightarrow$ (2) Рассмотрим произвольную последовательность точек $x_{n} \in E$, сходящуюся к $a$. Покажем, что $f(x_{n}) \to f(a)$. Зафиксируем $\epsilon > 0$. По определению непрерывности найдется такое $\delta > 0$, что $|f(x) - f(a)| < \epsilon$ для всех $x \in B_{\delta}(a) \cap E$. Так как $x_{n} \to a$, то существует такой номер $N$, что $x_{n} \in B_{\delta}(a) \cap E$ при всех $n \geq N$ и, значит, $|f(x_{n}) - f(a)| < \epsilon$ при всех $n \geq N$. \\
        (2) $\Rightarrow$ (3) Если $a$ -- предельная точка $E$, то в силу (2) $\lim_{x \to a} f(x) = f(a)$ по определению Гейне предела функции. Если $a$ не является предельной точкой $E$, то по определению $a$ -- изолированная точка $E$. \\
        (3) $\Rightarrow$ (1) Если $a$ изолирована, то $B_{\delta_{0}}(a) \cap E = \{a\}$ для некоторого $\delta_{0} > 0$. Тогда определение непрерывности в точке $a$ выполняется для $\delta = \delta_{0}$. Пусть $a$ предельная точка $E$. По определению Коши предела функции $\forall \epsilon > 0 \ \exists \delta > 0 \ \forall x \in E \ (0 < |x - a| < \delta \Rightarrow |f(x) - f(a)| < \epsilon)$. Но последняя импликация очевидно выполняется и для $x = a$. Значит, функция $f$ непрерывна в точке $a$.
    \end{proof}

    \begin{theorem}{О непрерывности композиции}\\
        Пусть $f: E \to \R$, и $g: D \to \R$, причем $f(E) \subset D$. Если функция $f$ непрерывна в точке $a$, функция $g$ непрерывна в точке $f(a)$, то композиция $g \circ f: E \to \R$ непрерывна в точке $a$.
    \end{theorem}

    \begin{proof}
        Пусть $x_{n} \in E$ и $x_{n} \to a$. Тогда $f(x_{n}) \to f(a)$ в силу непрерывности $f$ в точке $a$, и $g(f(x_{n})) \to g(f(a))$ в силу непрерывности $g$ в точке $f(a)$. По теореме об эквивалентных определениях непрерывности, функция $g\circ f$ непрерывна в точке $a$.
    \end{proof}

    \begin{definition}
        Пусть $f: E \to \R$, $a \in E$. Если $f$ не является непрерывной в точке $a$, то говорят, что $f$ \textit{разрывна} в точке $a$, а точку $a$ называют \textit{точкой разрыва} функции $f$. 
    \end{definition}

    Пусть функция задана в проколотой окрестности точки $a$. Если существуют конечные пределы $f(a - 0)$ и $f(a + 0)$, но не все числа $f(a - 0)$, $f(a + 0)$ и $f(a)$ равны между собой (случай, когда $f$ не определена в самой точке $a$ тоже допускается), то $a$ называется \textit{точкой разрыва ${\MakeUppercase{\romannumeral 1}}$ рода}. В противном случае, т.е. хотя бы один из односторонних пределов бесконечен или не существует, то $a$ называется \textit{точкой разрыва ${\MakeUppercase{\romannumeral 2}}$ рода} $f$.\\
    Если $a$ -- точка разрыва ${\MakeUppercase{\romannumeral 1}}$ рода и $f(a - 0) = f(a + 0)$, то $a$ называется \textit{точкой устранимого разрыва} функции $f$.

    \begin{theorem}{О разрывах монотонной функции}\\
        Если функция $f$ монотонна на конечном или бесконечном интервале $(a, b)$, то $f$ может иметь на $(a, b)$ разрывы только ${\MakeUppercase{\romannumeral 1}}$ рода, причем их не более чем счетное множество.
    \end{theorem}

    \begin{proof}
        Пусть для определенности $f$ нестрого возрастает. По следствию из теоремы об односторонних пределах монотонной функции, для любой точки $c \in (a, b)$ существуют конечные $f(c + 0)$ и $f(c - 0)$, причем $f(c - 0) \leq f(c) \leq f(c + 0)$. Поэтому, если $f$ разрывна в точке $c$, то $f(c - 0) < f(c + 0)$ и, значит, $c$ -- точка разрыва ${\MakeUppercase{\romannumeral 1}}$ рода.\\
        Пусть $c, d \in (a, b)$, причем $c < d$. Ввиду возрастания функции $\inf_{(c, b)}f(x) = \inf_{(c, d)}f(x)$ и $\sup_{(a, d)}f(x) = \sup_{(c, d)}f(x)$ и, значит,
        \[f(c + 0) = \inf_{x \in (c, d)} f(x) \leq \sup_{x \in (c, d)} f(x) = f(d - 0).\]
        Поэтому, если $c, d$ -- точки разрыва $f$, то интервалы $(f(c - 0), f(c + 0))$ и $(f(d - 0), f(d + 0))$ не пересекаются. Поставим в соответствие каждому такому интервалу рациональное число, содержащееся в нем. Тем самым установим биекцию между множеством таких интервалов и подмножеством $\Q$. Любое подмножество $\Q$ не более чем счетно, поэтому множество точек разрыва $f$ не более чем счетно.
    \end{proof}