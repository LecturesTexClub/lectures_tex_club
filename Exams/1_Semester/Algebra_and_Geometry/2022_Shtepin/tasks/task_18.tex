\subsection{Системы линейных уравнений. Элементарные преобразования строк и столбцов матрицы, элементарные матрицы, их свойства. Приведение матрицы к ступенчатому и упрощенному виду. Метод Гаусса решения систем линейных уравнений. Основная лемма о линейной зависимости. Фундаментальная система решений и общее решение однородной системы линейных уравнений. Общее решение неоднородной системы.}

    \begin{definition}
    	Пусть $A = (a_{ij}) \in M_{k \times n}(F)$, $b = (b_i) \in F^n$. \textit{Системой линейных уравнений} $Ax = b$ называется следующая система:
    	\[
    	\left\{
    	\begin{aligned}
    	&a_{11}x_1 + a_{12}x_2 + \dotsb + a_{1n}x_n = b_1\\
    	&a_{21}x_1 + a_{22}x_2 + \dotsb + a_{2n}x_n = b_2\\
    	&\dots\\
    	&a_{k1}x_1 + a_{k2}x_2 + \dotsb + a_{kn}x_n = b_k\\
    	\end{aligned}
    	\right.
    	\]
    	
    	Матрица $A$ называется \textit{матрицей системы}, матрица $(A|b)$ --- \textit{ расширенной матрицей системы}.
    \end{definition}
    
    \begin{definition}
    	Система линейных уравнений $Ax = b$ называется:
    	\begin{itemize}
    		\item \textit{Однородной}, если $b = 0$
    		\item \textit{Совместной}, если множество ее решений непусто
    	\end{itemize} 
    \end{definition}
    
    \begin{definition}
    	\textit{Элементарными преобразованиями} строк матрицы $A \in M_{n \times k}(F)$ называются следующие операции:
    	\begin{itemize}
    		\item Прибавление к $i$-й строке $j$-й строки, умноженной на скаляр $\alpha \in F$, $i, j \in \{1, \dotsc, n\}$, $i \ne j$
    		\item Умножение $i$-й строки на скаляр $\lambda \in F^*$, $i \in \nset{n}$
    		\item Перестановка $i$-й и $j$-й строк местами, $i, j \in \{1, \dotsc, n\}$, $i \ne j$
    	\end{itemize}
    \end{definition}
    
    \begin{definition}
    	\textit{Элементарными матрицами} порядка $n \in \N$ называются матрицы, умножение слева
    	на которые приводит к осуществлению соответствующего элементарного преобразования строк над матрицей с $n$ строками:
    	\begin{itemize}
    		\item $D_{ij}(\alpha) := E + \alpha E_{ij}$, $i, j \in \{1, \dotsc, n\}$, $i \ne j$
    		\item $T_{i}(\lambda) := E + (\lambda - 1) E_{ii}$, $i \in \{1, \dotsc, n\}$
    		\item $P_{ij} := E - (E_{ii} + E_{jj}) + (E_{ij} + E_{ji})$, $i, j \in \{1, \dotsc, n\}$, $i \ne j$
    	\end{itemize}
    \end{definition}
    
    \begin{definition}
    	Матрица $A \in M_{n \times k}(F)$ имеет \textit{ступенчатый вид}, если номера главных элементов ее строк строго возрастают. При этом если в матрице есть нулевые строки, то они расположены внизу матрицы.
    \end{definition}
    
    \begin{theorem}[метод Гаусса]
    	Любую матрицу $A \in M_{n \times k}(F)$ элементарными преобразованиями строк можно привести к ступенчатому виду.
    \end{theorem}
    
    \begin{proof}
    	Предъявим алгоритм приведения к ступенчатому виду:
    	\begin{enumerate}
    		\item Если $A = 0$, то она уже имеет ступенчатый вид, тогда завершим процедуру.
    		\item Пусть $j \in \{1, \dotsc, k\}$ "--- наименьший номер ненулевого столбца. Переставим строки так, чтобы $a_{1j}$ стал ненулевым.
    		\item Для всех $i \in \{2, \dots, n\}$ к $i$-й строке прибавим первую, умноженную на $-a_{ij}(a_{1j})^{-1}$. Тогда все элементы $a_{2j}, \dots, a_{nj}$ станут нулевыми.
    		\item Пусть матрица $A$ была приведена к виду $A'$. Повторим шаги $(1), \dotsc, (4)$ для подматрицы $B$, расположенной на пересечении строк с номерами $2, \dotsc, n$ и столбцом с номерами $j + 1, \dotsc, k$. Дальнейшие преобразования не изменят элементов за пределами этой подматрицы.\qedhere
    	\end{enumerate}
    \end{proof}
    
    \begin{theorem}[основная лемма о линейной зависимости]
    	Пусть $V$ "--- линейное пространство над полем $F$, и $V = \langle\overline{v_1}, \dots, \overline{v_k}\rangle$ для некоторых $\overline{v_1}, \dotsc, \overline{v_k} \in V$. Тогда для любых векторов $\overline{u_1}, \dots, \overline{u_n} \in V$, $n > k$, система $(\overline{u_1}, \dots, \overline{u_n})$ линейно зависима.
    \end{theorem}
    
    \begin{proof}
    	Векторы $\overline{u_1}, \dots, \overline{u_n}$ выражаются через $\overline{v_1}, \dots, \overline{v_k}$, поскольку лежат в их линейной оболочке $\langle\overline{v_1}, \dots, \overline{v_k}\rangle = V$. Следовательно, $(\overline{u_1}, \dots, \overline{u_n}) \hm{=} (\overline{v_1}, \dots, \overline{v_k})A$ для некоторой матрицы $A \in M_{k \times n}(F)$. Но $n > k$, поэтому существует такой ненулевой столбец $\gamma \hm{\in} F^n$, что $A\gamma = 0$, тогда $(\overline{u_1}, \dots, \overline{u_n})\gamma = (\overline{v_1}, \dots, \overline{v_k})A\gamma = \overline{0}$. Значит, система линейно зависима.
    \end{proof}
    
    \begin{definition}
    	Матрица $A \in M_{n \times k}(F)$ имеет \textit{упрощенный вид}, если она является ступенчатой, и всякий ее столбец, содержащий главный элемент, состоит из одной единицы, соответствующей главному элементу, и нулей.
    \end{definition}
    
    \begin{theorem}
    	Любую матрицу $A \in M_{n \times k}(F)$ элементарными преобразованиями строк можно привести к упрощенному виду.
    \end{theorem}
    
    \begin{proof}
    	Сначала приведем матрицу $A$ к ступенчатому виду, затем запустим следующий алгоритм:
    	\begin{itemize}
    		\item Если $A = 0$, она уже имеет упрощенный вид.
    		\item Пусть $i \in \{1, \dotsc, n\}$ "--- наибольший номер ненулевой строки, $a_{ik}$ "--- главный элемент в ней. Умножим $i$-ю строку на $(a_{ik})^{-1}$, чтобы коэффициент $a_{ik}$ стал равным $1$.
    		\item Для всех $j \in \{1, \dots, i - 1\}$ к $j$-й строке прибавим $i$-ю, умноженную на $-a_{jk}$. Тогда все элементы $a_{1k}, \dots, a_{(i-1)k}$ станут нулевыми.
    		\item Пусть матрица $A$ была приведена к виду $A'$. Повторим шаги $(1), \dotsc, (4)$ для подматрицы $B$, расположенной на пересечении строк $1, \dotsc, i - 1$ и столбцов $1,\dotsc, k - 1$. Дальнейшие преобразования не изменят элементов за пределами этой подматрицы.\qedhere
    	\end{itemize}
    \end{proof}
    
    \begin{definition}
    	\textit{Фундаментальной системой решений} однородной системы $Ax = 0$ называется базис пространства ее решений. Матрица, образованная столбцами фундаментальной системы решений, называется \textit{фундаментальной матрицей системы} и обозначается через $\Phi$.
    \end{definition}
    
    \begin{definition}
        Совокупность всех решений совместной системы называется \textit{общим решением}.
    \end{definition}
    
    \begin{proposition}
    	Пусть $Ax = b$ "--- совместная система, $x_0 \in F^n$ "--- решение системы, $V$ "--- пространство решений однородной системы $Ax = 0$. Тогда множество решений системы $Ax = b$ имеет вид $x_0 + V \hm{=} \{x_0 + v: v \in V\}$.
    \end{proposition}
    
    \begin{proof}
    	Пусть $U$ "--- множество решений системы $Ax = b$.
    	\begin{itemize}
    		\item Если $v \hm{\in} V$, то $A(x_0 + v) = Ax_0 + Av = b$, откуда $x_0 + v \in U$
    		\item Если $u \in U$, то $A(u - x_0) = 0$, откуда $u - x_0 \in V$
    	\end{itemize}
    
    	Таким образом, $U = x_0 + V$.
    \end{proof}