\subsection{Классификация линий второго порядка. Приведение уравнения второго порядка с двумя переменными к каноническому виду в прямоугольной системе координат. Центр кривой второго порядка.}
    
    \begin{theorem}
    	Любое уравнение кривой второго порядка в некоторой прямоугольной декартовой системой координат в $P_2$ имеет один из девяти канонических видов:
    	\begin{itemize}
    		\item Кривые эллиптического типа:
    		\begin{enumerate}
    			\item $\frac{x^2}{a^2} + \frac{y^2}{b^2} = 1$, $a \ge b > 0$, "--- \textit{эллипс}
    			\item $\frac{x^2}{a^2} + \frac{y^2}{b^2} = 0$, $a \ge b > 0$, "--- \textit{точка}
    			\item $\frac{x^2}{a^2} + \frac{y^2}{b^2} = -1$, $a \ge b > 0$, "--- \textit{мнимый эллипс}
    		\end{enumerate}
    	
    		\item Кривые гиперболического типа:
    		\begin{enumerate}
    			\item $\frac{x^2}{a^2} - \frac{y^2}{b^2} = 1$, $a, b > 0$, "--- \textit{гипербола}
    			\item $\frac{x^2}{a^2} - \frac{y^2}{b^2} = 0$, $a, b > 0$, "--- \textit{пара пересекающихся прямых}
    		\end{enumerate}
    	
    		\item Кривые параболического типа:
    		\begin{enumerate}
    			\item $y^2 = 2px$, $p > 0$, "--- \textit{парабола}
    			\item $\frac{y^2}{a^2} = 1$, $a > 0$, "--- \textit{пара параллельных прямых}
    			\item $\frac{y^2}{a^2} = 0$, $a > 0$, "--- \textit{пара совпадающих прямых}
    			\item $\frac{y^2}{a^2} = -1$, $a > 0$, "--- \textit{пара мнимых параллельных прямых}
    		\end{enumerate}
    	\end{itemize}
    \end{theorem}
    
    \begin{proof}
    	Пусть в исходной прямоугольной декартовой системе координат в $P_2$ кривая второго порядка задается уравнением $Ax^2 + 2Bxy + Cy^2 \hm{+} 2Dx + 2Ey + F = 0$. Процесс перехода в искомую систему координат происходит в три этапа:
    	\begin{enumerate}
    		\item Если $B \ne 0$, избавимся от монома $2Bxy$. Для этого произведем поворот системы координат на угол $\alpha$ против часовой стелки. Матрица перехода $S$ при таком преобразовании имеет следующий вид:
    		\[S = \begin{pmatrix}\cos{\alpha}&-\sin{\alpha}\\\sin{\alpha}&\cos{\alpha}\end{pmatrix}
    		\]
    		
    		Тогда, по свойству замены координат:
    		\[\left\{\begin{aligned}
    			&x = x'\cos{\alpha}-y'\sin{\alpha}\\
    			&y = x'\sin{\alpha}+y'\cos{\alpha}
    		\end{aligned}\right.\]
    		
    		Определим значение $\alpha$, при котором коэффициент при $x'y'$ обращается в $0$:
    		\[-2A\sin{\alpha}\cos{\alpha} + 2B(\cos^2{\alpha} - \sin^2{\alpha}) + 2C\sin{\alpha}\cos{\alpha} = 0 \ra 2B\cos{2\alpha} = (A - C)\sin{2\alpha}\]
    		
    		Если $A = C$, то выберем $\alpha = \frac{\pi}{4}$, иначе --- такой $\alpha$, что $\tg{2\alpha} = \frac{2B}{A - C}$. В новой системе координат получим выражение вида $A'x'^2 + C'y'^2 + 2D'x' +2E'y' + F' = 0$.
    		
    		\item Если $A' \ne 0$, избавимся от монома $2D'x'$. Для этого произведем следующий сдвиг системы координат:
    		\[\left\{\begin{aligned}
    		&x' = x'' + \frac{D'}{A'}\\
    		&y' = y''
    		\end{aligned}\right.\]
    	
    		После этого получим выражение $A''x''^2+C''y''^2+2E''y'' \hm{+} F'' = 0$.
    		
    		\item Если $C'' \ne 0$, избавимся от монома $2E''y''$, аналогично пункту $(2)$.
    	\end{enumerate}
    	
    	Опустим штрихи в записи уравнения в полученной системе координат. После того, как произведены операции выше, могут быть получены три различных результата:
    	\begin{enumerate}
    		\item Если $AC > 0$, то ни один из мономов $x^2$, $y^2$ не сократился, и полученное уравнение имеет вид $Ax^2 + Cy^2 + F = 0$. Если $A, C < 0$, домножим уравнение на $-1$. Перенесем $F$ в другую часть и, если $F \ne 0$, разделим уравнение на $|F|$. После данных операций получим уравнение следующего вида:
    		\[\frac{x^2}{a^2} + \frac{y^2}{b^2} = \epsilon,~a, b > 0,~\epsilon \in \{-1, 0, 1\}\]
    		
    		Если $a < b$, то поменяем координаты местами. Получено уравнение кривой эллиптического типа.
    		
    		\item Если $AC < 0$, то ни один из мономов $x^2$, $y^2$ не сократился, и полученное уравнение имеет вид $Ax^2 + Cy^2 + F = 0$. Аналогичными описанным в предыдущем пункте преобразованиями, получим уравнение следующего вида:
    		\[\frac{x^2}{a^2} - \frac{y^2}{b^2} = \epsilon,~a, b > 0,~\epsilon \in \{0, 1\}\]
    		
    		Получено уравнение кривой гиперболического типа.
    		
    		\item Если $AC = 0$, то одно из чисел $A, C$ осталось ненулевым, поскольку многочлен в уравнении должен иметь степень $2$. Заменой системы координат можно добиться того, чтобы это было число $C$. Тогда полученное уравнение имеет вид $Cy^2+2Dx+F = 0$. Если $D \ne 0$, то сдвиг системы координат позволяет избавиться от $F$ и получить уравнение следующего вида:
    		\[y^2 = 2px,~p>0\]
    		
    		Если же $D = 0$, то уравнение можно привести к следующему виду:
    		\[\frac{y^2}{a^2} \epsilon,~a > 0,~\epsilon \in \{-1, 0, 1\}\]
    		
    		Получено уравнение кривой параболического типа.\qedhere
    	\end{enumerate}
    \end{proof}
    
    \begin{definition}
    	\textit{Центром многочлена} $P(x, y)$ в декартовой системе координат $(O, e)$ в $P_2$ называется такая точка $A \in P_2$, $A \leftrightarrow_{(O, e)} \alpha$, что для любых чисел $x, y \in \mathbb{R}$ выполнено равенство $P(\alpha_1 - x, \alpha_2 - y) = P(\alpha_1 + x, \alpha_2 + y)$.
    \end{definition}
    
    \begin{proposition}
    	Пусть $A \in P_2$, в декартовой системе координат $(O, e)$ в $P_2$ выполнено $A \leftrightarrow_{(O, e)} \alpha$, и пусть $P(x, y) \hm{=} Ax^2 + 2Bxy + Cy^2 + 2Dx + 2Ey + F$. Тогда:
    	\[A \text{ "--- центр многочлена } P(x, y) \Leftrightarrow \left\{\begin{aligned}
    	A\alpha + B\beta + D = 0\\
    	B\alpha + C\beta + E = 0
    	\end{aligned}\right.\]
        upd: у математиков закончились буквы, $A \neq A$.
    \end{proposition}
    
    \begin{proof}
    	Доказательство производится непосредственной проверкой.
    \end{proof}