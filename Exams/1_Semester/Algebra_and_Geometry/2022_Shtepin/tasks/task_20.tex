\subsection{Ранг системы векторов, его связь с размерностью линейной оболочки. Ранг матрицы. Теорема о ранге матрицы. Ранг произведения матриц. Теорема о базисном миноре. Нахождение ранга с помощью элементарных преобразований. Теорема Кронекера-Капелли. Невырожденные и обратимые матрицы. Нахождение обратной матрицы при помощи элементарных преобразований.}

    \begin{definition}
    	Пусть $V$ "--- конечнопорожденное линейное пространство, $X \subset V$. \textit{Рангом} системы $X$ называется наибольший размер линейно независимой подсистемы в $X$. Обозначение "--- $\rk{X}$.
    \end{definition}
    
    \begin{proposition}
    	Пусть $V$ "--- конечнопорожденное линейное пространство, $X \subset V$. Тогда $\rk{X} = \dim{\langle X\rangle}$.
    \end{proposition}
    
    \begin{proof}
    	Пусть $k := \rk{X}$ и $(\overline{v_1}, \dots, \overline{v_k})$ "--- линейно независимая система в $X$. Тогда для любого $\overline{v} \in X$ система $(\overline{v_1}, \dots, \overline{v_k}, \overline{v})$ линейно зависима, откуда $X \subset \langle\overline{v_1}, \dots, \overline{v_k}\rangle$. Но тогда $\langle X \rangle \subset \langle\overline{v_1}, \dots, \overline{v_k}\rangle \subset \langle X \rangle$, откуда $\langle X \rangle = \langle\overline{v_1}, \dots, \overline{v_k}\rangle$. Значит, $(\overline{v_1}, \dots, \overline{v_k})$ "--- базис в $\langle X\rangle$, поэтому $\dim{\langle X\rangle} = k \hm{=} \rk{X}$.
    \end{proof}
    
    \begin{definition}
    	Пусть $A \in M_{n \times k}(F)$.
    	\begin{itemize}
    		\item \textit{Строчным рангом} матрицы $A$ называется ранг $\rk_{r}{A}$ системы ее строк
    		\item \textit{Столбцовым рангом} матрицы $A$ называется ранг $\rk_{c}{A}$ системы ее столбцов
    	\end{itemize}
    \end{definition}
    
    \begin{proposition}
    	Для любых матриц $A \in M_{n \times k}(F)$ и $B \in M_{k \times m}(F)$ выполнены неравенства $\rk_c{AB} \le \rk_cA$ и $\rk_r{AB} \le \rk_rB$.
    \end{proposition}
    
    \begin{proof}
    	Докажем первое неравенство, поскольку второе неравенство доказывается аналогично. Пусть $U$ "--- линейная оболочка столбцов матрицы $A$, $V$ "--- линейная оболочка столбцов матрицы $AB$. Уже было доказано, что столбцы матрицы $AB$ являются линейными комбинациями столбцов матрицы $A$, поэтому $V \le U$. Следовательно, $\rk_r(AB) = \dim{V} \le \dim{U} \hm{=} \rk_rA$.
    \end{proof}
    
    \begin{theorem}[о ранге матрицы]
    	Для любой матрицы $A \in M_{n \times k}(F)$ выполнено следующее равенство:
    	\[\rk_rA = \rk_cA\]
    \end{theorem}
    
    \begin{proof}
    	Пусть $r := rk_cA$, тогда столбцы матрицы $A$ выражаются через некоторые $r$ столбцов. Составим из этих $r$ столбцов матрицу $B$, тогда каждый столбец матрицы $A$ имеет вид $B\gamma$ для некоторого $\gamma \in F^r$. Следовательно, $A$ можно представить в виде $B(\gamma_1|\dots|\gamma_k)$. По уже доказанному, $\rk_rA \le \rk_r(\gamma_1|\dots|\gamma_k) \le r$, поскольку в матрице $(\gamma_1|\dots|\gamma_k)$ ровно $r$ строк. Аналогично показывается, что $\rk_cA \le \rk_rA$. Таким образом, $\rk_rA = \rk_cA$.
    \end{proof}
    
    \begin{definition}
    	\textit{Рангом матрицы} $A \in M_{n \times k}(F)$ называется ее строчный или столбцовый ранг. Обозначение "--- $\rk{A}$.
    \end{definition}
    
    \begin{theorem}[о ранге произведения матриц]
        \[\begin{cases}
            \rk(AB) \leq \rk(A) \\
            \rk(AB) \leq \rk(B)
        \end{cases}\]
    \end{theorem}
    
    \begin{proof}
        Пусть $C = AB$. $i$--ый столбец матрицы $C$ является линейной комбинацией столбцов матрицы $A$ с коэффициентами из $i$--ого столбца матрицы $B$, а $j$--ая строка матрицы $C$ является линейной комбинацией строк матрицы $B$ с коэффициентами из $j$--ой строки матрицы $A$. Таким образом, система столбцов матрицы $C$ линейно выражается через систему столбцов матрицы $A$, и ранг системы столбцов $C$ не превышает ранга системы столбцов $A$ и $\rk(C) \leq \rk(A)$. Аналогично, система строк матрицы $C$ линейно выражается через систему строк матрицы $B$, и ранг системы строк $C$ не превышает ранга системы строк $B$ и $\rk(C) \leq \rk(B)$. 
    \end{proof}
    
    \begin{theorem}[о базисном миноре]
    	Пусть $A \in M_{n \times k}(F)$, $\rk{A} = r$. Тогда в $A$ найдется подматрица размера $r \times r$ ранга $r$. Более того, если выбрать линейно независимую систему из $r$ строк матрицы $A$ и линейно независимую систему из $r$ столбцов матрицы $A$, то искомая матрица будет расположена на их пересечении.
    \end{theorem}
    
    \begin{proof}
    	Докажем сразу вторую часть утверждения. Без ограничения общности можно считать, что подматрица $M$ на пересечении $r$ линейно независимых строк и столбцов расположена в левом верхнем углу матрицы $A$. Пусть $R \in M_{r \times k}$ "--- подматрица из первых $r$ строк $A$, $C \in M_{n \times r}$ "--- подматрица из первых $r$ столбцов $A$.
    	
    	Столбцы матрицы $A$ выражаются через столбцы матрицы $C$, поэтому $A = CB$ для некоторой $B \in M_{r \times n}(F)$. Но тогда столбцы матрицы $R$ выражаются через столбцы матрицы $M$ с теми же коэффициентами, то есть $R = MB$. Кроме того, строки матрицы $A$ выражаются через строки матрицы $R$, то есть $A = SR$ для некоторой $S \in M_{n \times r}(F)$. Таким образом, $A = SMB$, тогда $r = \rk{A} \le \rk{M} \le r$, откуда $\rk{M} = r$.
    \end{proof}
    
    \begin{proposition}
    	Пусть $A \in M_{n \times k}(F)$, и $D \in M_{n}(F)$ "--- обратимая матрица. Тогда столбцы матрицы $A$ с некоторыми номерами линейно зависимы $\lra$ столбцы матрицы $DA$ с теми же номерами линейно зависимы.
    \end{proposition}
    
    \begin{proof}
    	Пусть $\gamma \in F^k$, тогда:
    	\begin{gather*}
    		A\gamma = 0 \Rightarrow DA\gamma = 0\\
    		DA\gamma = 0 \Rightarrow D^{-1}DA\gamma = 0 \Rightarrow A\gamma = 0
    	\end{gather*}
    	
    	Значит, столбцы с одинаковыми номерами в $A$ и $DA$ образуют или не образуют линейно зависимую систему одновременно.
    \end{proof}
    
    \begin{corollary}
    	При элементарных преобразованиях строк матрицы $A \in M_{n \times k}(F)$ не меняется ее ранг и линейная зависимость столбцов.
    \end{corollary}
    
    \begin{theorem}[Кронекера-Капелли]
    	Система $Ax = b$ совместна $\Leftrightarrow$ $\rk{A} = \rk{(A|b)}$.
    \end{theorem}
    
    \begin{proof}
    	Приведем расширенную матрицу системы $(A|b)$ к упрощенному виду $(A'|b')$. Поскольку перестановки столбцов не происходит, то матрица $A'$ "--- это упрощенный вид матрицы $A$. Тогда система совместна $\Leftrightarrow$ в $(A'|b')$ нет ступеньки, начинающейся в столбце $b'$, $\Leftrightarrow$ у $A'$ и $(A'|b')$ одно и то же число ступенек $\Leftrightarrow$ $\rk{A} = \rk{(A|b)}$.
    \end{proof}
    
    \begin{definition}
    	Матрица $A \in M_n(F)$ называется \textit{обратимой}, если существует матрица $A^{-1} \in M_n(F)$ такая, что $AA^{-1} = A^{-1}A = E$.
    \end{definition}
    
    \begin{definition}
    	Матрица $A \in M_{n}(F)$ называется \textit{невырожденной}, если $\rk{A} = n$.
    \end{definition}
    
    \begin{theorem}
    	Пусть $A \in M_{n}(F)$. Тогда следующие условия эквивалентны:
    	\begin{enumerate}
    		\item Матрица $A$ невырожденна
    		\item Матрица $A$ элементарными преобразованиями строк приводится к $E$
    		\item Матрица $A$ является произведением элементарных матриц
    		\item Матрица $A$ обратима
    		\item Матрица $A$ обратима слева, то есть существует матрица $B \in M_n(F)$ такая, что $BA = E$, или справа
    	\end{enumerate}
    \end{theorem}
    
    \begin{proof}~
    	\begin{itemize}
    		\item\imp{1}{2} Приведем $A$ к упрощенному виду $A'$. Так как $\rk{A'} = \rk{A} = n$, то $A' = E$.
    		
    		\item\imp{2}{3} Пусть последовательности преобразований, приводящих $A$ к $E$, соответствует последовательность элементарных матриц $M_1, \dots, M_{k} \in M_n(F)$, тогда:
    		\[M_k\dots M_1A = E \ra A = M_1^{-1}\dots M_k^{-1}\]
    		
    		\item\imp{3}{4} Если $A = M_1^{-1}\dots M_k^{-1}$, то $A$ обратима, причем $A^{-1} = M_k\dots M_1$.
    		
    		\item\imp{4}{5} Если $A$ обратима, то, в частности, $A$ обратима слева или справа.
    		
    		\item\imp{5}{1} Пусть без ограничения общности $A$ обратима слева, тогда существует матрица $B \in M_n(F)$ такая, что $BA = E$. Тогда $n = \rk{E} = \rk{BA} \hm{\le} \rk{A}$, откуда $\rk{A} = n$.\qedhere
    	\end{itemize}
    \end{proof}
    
    \begin{corollary}
     	Пусть $A$ "--- невырожденная матрица, и матрица $(A|E)$ приводится к упрощенному виду $(E|C)$. Тогда матрица $C$ является обратной к $A$.
    \end{corollary}
    
    \begin{proof}
    	Пусть последовательности преобразований, приводящих $(A|E)$ к $(E|C)$, соответствует последовательность элементарных матриц $M_1, \dots, M_{k} \in M_n(F)$, то есть $M_k\dots M_1(A|E) = (E|C)$. Тогда:
    	\[M_k\dots M_1(A|E) \hm{=} (M_k\dots M_1A|M_k\dots M_1E) = (M_k\dots M_1A|M_k\dots M_1)\]
    	
    	Следовательно, $M_k\dots M_1 = C$ и $CA = E$.
    \end{proof}
    