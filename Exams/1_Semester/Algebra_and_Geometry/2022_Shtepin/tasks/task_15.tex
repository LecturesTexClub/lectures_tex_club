\subsection{Понятия группы, кольца и поля, примеры. Поле комплексных чисел. Характеристика поля, простое подполе. Группа перестановок, знак подстановки. Изоморфизм групп, теорема Кэли. Порядок элемента. Циклические группы, теорема об изоморфизме циклических групп, подгруппы циклических групп. Теорема Лагранжа о порядке подгруппы, её следствия.}

    \begin{definition}
    	\textit{Группой} называется множество $G$ с определенной на нем бинарной операцией \textit{умножения} $\cdot: G \times G \rightarrow G$, удовлетворяющей следующим условиям:
    	\begin{itemize}
    		\item $\forall a, b, c \in G : (ab)c = a(bc)$ (ассоциативность)
    		\item $\exists e \in G: \forall a \in G: ae = ea = a$ (существование нейтрального элемента)
    		\item $\forall a \in G: \exists a^{-1} \in G: aa^{-1} = a^{-1}a = e$ (существование обратного элемента)
    	\end{itemize}
    \end{definition}
    
    \begin{definition}
    	Группа называется $(G, \cdot)$ \textit{абелевой}, если умножение в ней коммутативно, то есть для любых $a, b \in G$ выполнено $ab = ba$.
    \end{definition}
    
    \begin{definition}
    	\textit{Кольцом} называется множество $R$ с определенными на нем бинарными операциями \textit{сложения} $+ : R \times R \to R$ и \textit{умножения} $\cdot: R \times R \rightarrow R$, удовлетворяющими следующим условиям:
    	\begin{itemize}
    		\item $(R, +)$ "--- абелева группа, нейтральный элемент в которой обозначается через $0$
    		\item $\forall a, b, c \in R: (ab)c = a(bc)$ (ассоциативность умножения)
    		\item $\forall a, b, c \in R: a(b + c) = ab + ac$ и $(a + b)c = ac + bc$ (дистрибутивность умножения относительно сложения)
    	\end{itemize}
    \end{definition}
    
    \begin{definition}
    	Пусть $(R, +, \cdot)$ "--- кольцо. Элемент $a \in R$ называется \textit{обратимым}, если существует $a^{-1} \in R$ такой, что $aa^{-1} \hm{=} a^{-1}a = 1$. \textit{Группой обратимых элементов} кольца $(R, +, \cdot)$ называется множество $R^*$ его обратимых элементов.
    \end{definition}
    
    \begin{definition}
    	\textit{Полем} называется такое коммутативное кольцо $(F, +, \cdot)$, для которого выполнено равенство $F^* = F\backslash\{0\}$.
    \end{definition}
    
    \begin{example}
    	Рассмотрим несколько примеров:
    	\begin{itemize}
    		\item $(\mathbb{Q}, +, \cdot)$, $(\mathbb{R}, +, \cdot)$, $(\mathbb{C}, +, \cdot)$ являются полями;
    		\item $(\mathbb{Z}, +, \cdot)$, $(\mathbb{Q}, +, \cdot)$, $(\mathbb{R}, +, \cdot)$, $(\mathbb{C}, +, \cdot)$ являются коммутативными кольцами;
                \item $(\mathbb{Z}, +)$, $(\mathbb{Q}, +)$, $(\mathbb{R}, +)$, $(\mathbb{C}, +)$ являются абелевыми группами.
    	\end{itemize}
    \end{example}
    
    \begin{definition}
    	\textit{Подполем} поля $(F, +, \cdot)$ называется такое его непустое подмножество $S \subset F$, что выполнены следующие условия:
    	\begin{itemize}
    		\item $(S, +, \cdot)$ "--- подкольцо в $(F, +, \cdot)$
    		\item $\forall a \in S\backslash\{0\}: a^{-1} \in S$
    	\end{itemize}
    \end{definition}
    
    \begin{note}
    	Имеет место эквивалентное определение подполя, согласно которому подполем поля $(F, +, \cdot)$ называется такое его непустое подмножество $S \subset F$, что $(S, +, \cdot)$ тоже является полем.
    \end{note}
    
    \begin{definition}
    	Пусть $F$ "--- поле. Его \textit{характеристикой} называется наименьшее число $k \in \mathbb{N}$ такое, что в поле $F$ выполнено равенство $k = 0$. Если такого $k$ не существует, то характеристика поля считается равной $0$. Обозначение "--- $\cha{F}$.
    \end{definition}
    
    \begin{proposition}
    	Пусть $F$ "--- поле. Тогда если $\cha{F} > 0$, то $\cha{F}$ "--- простое число.
    \end{proposition}
    
    \begin{proof}
    	Пусть $\cha{F} = n$. Если $n = 1$, то элементы $0$ и $1$ в $F$ совпадают, откуда $F^* = F$, что невозможно. Пусть теперь $n$ "--- составное число, то есть $n = ab$ для некоторых $a, b \in \mathbb{N}$ таких, что $a, b > 1$. Тогда в поле $F$ числа $a, b$ отличны от нуля, но $ab = 0$. Умножая обе части равенства на $a^{-1}$, получим, что $b = 0$, --- противоречие. Значит, возможен только случай простого числа $n$.
    \end{proof}
    
    \begin{definition}
    	Поле называется \textit{простым}, если оно не имеет подполей, отличных от него самого.
    \end{definition}
    
    \begin{theorem}[о простом подполе]
    	Пусть $F$ "--- поле. Тогда:
    	\begin{enumerate}
    		\item Если $\cha{F} = p > 0$, то в $F$ существует подполе, изоморфное $\mathbb{Z}_p$
    		\item Если $\cha{F} = 0$, то в $F$ существует подполе, изоморфное $\mathbb{Q}$
    	\end{enumerate}
    \end{theorem}
    
    \begin{proof}~
    	\begin{enumerate}
    		\item Пусть $\cha{F} = p$. Определим $K$ как множество всех целых чисел в $F$, и зададим отображение $\phi: \mathbb{Z}_p \rightarrow K$ как $\phi(\overline{a}) := a$ для каждого $\overline{a} \in \Z_p$. Покажем, что отображение определено корректно. Пусть $\overline{a} = \overline{a'}$ для некоторых $a, a' \in \Z$, тогда $a' = a + kp$ для некоторого $k \in \mathbb{Z}$, откуда в поле $F$ выполнены равенства $a' = a + kp = a$. Ясно, что определенное таким образом отображение $\phi$ сохраняет операции сложения и умножения.
    		
    		Сюръективность отображения $\phi$ очевидна, проверим его инъективность. Пусть для некоторых $\overline{a}, \overline{b} \in Z_p$ выполнено $\phi(\overline{a}) = \phi(\overline{b})$. Без ограничения общности можно считать, что $a, b \in \{0,\dots, p-1\}$ и $a \ge b$, тогда $\phi\big(\overline{a - b}\big) = \phi(\overline{a}) - \phi\big(\overline{b}\big) = 0$. Но это возможно только в том случае, когда $p \mid (a - b)$, откуда $a = b$.
    		
    		Из доказанного также следует, что $K$ "--- подполе в $F$. Например, замкнутость относительно взятия обратного элемента по умножению можно показать, используя свойства отображения $\phi$. Пусть $a \in K \bs \{0\}$, тогда обратным к нему является элемент $\phi(\overline{a}^{-1})$:
    		\[\phi(\overline{a}^{-1}) a = \phi(\overline{a}^{-1})\phi(\overline{a}) = \phi(\overline{1}) = 1\]
    		
    		Проверка остальных свойств подполя позволяет убедиться, что $K$ является полем, тогда отображение $\phi$ является изоморфизмом полей.
    	
    		\item Пусть $\cha{F} = 0$. Определим $K$ как множество всех выражений вида $\frac ab = ab^{-1}$, где $a, b \in F$ "--- целые числа в поле $F$, $b \ne 0$, и зададим $\phi: \mathbb{Q} \rightarrow K$ как $\phi(\frac{a}{b}) := \frac ab$ для каждого $\frac ab \in \Q$. Покажем, что отображение определено корректно. Пусть $\frac{a}{b} = \frac{a'}{b'}$ для некоторых $a, a', b, b' \in \Z$, $b, b' \ne 0$, тогда $a'b = ab'$, откуда в поле $F$ выполнены равенства $ab^{-1} = (aa')(a'b)^{-1} = (aa')(ab')^{-1} = a'b'^{-1}$. Ясно, что определенное таким образом отображение $\phi$ сохраняет операции сложения и умножения.
    		
    		Сюръективность отображения $\phi$ очевидна, проверим его инъективность. Пусть для некоторых $\frac ab, \frac cd \in \Q$ выполнено $\phi(\frac ab) = \phi(\frac cd)$, тогда $\phi(\frac{ad - bc}{bd}) = \phi(\frac ab) - \phi(\frac cd) = 0$. Но это возможно только в том случае, когда ${ad - bc} = 0$, откуда $\frac ab = \frac cd$.
    		
    		Из доказанного также следует, что $K$ "--- подполе в $F$. Например, замкнутость относительно взятия обратного элемента по умножению можно показать, используя свойства отображения $\phi$. Пусть $\frac ab \in K \bs \{0\}$, тогда $a \ne 0$, и обратным к элементу $\frac ab$ является элемент $\phi(\frac ba)$:
    		\[\frac ab\hspace{3pt}\phi\left(\frac{b}{a}\right) = \phi\left(\frac{a}{b}\right)\phi\left(\frac{b}{a}\right) = \phi(1) = 1\]
    		
    		Проверка остальных свойств подполя позволяет убедиться, что $K$ является полем, тогда отображение $\phi$ является изоморфизмом полей.\qedhere
    	\end{enumerate}
    \end{proof}
    
    \begin{definition}
    	\textit{Группой перестановок} $S_n$ называется следующее множество:
    	\[S_n := \{\sigma: \{1,\dots, n\} \rightarrow \{1,\dots, n\}: \sigma \text{ "--- биекция}\}\]
    	
    	Данное множество является группой с операцией композиции $\circ$. Элементы группы $S_n$ называются \textit{перестановками}.
    \end{definition}
    
    \begin{definition}
    	\textit{Беспорядком}, или \textit{инверсией}, в перестановке $\sigma \in S_n$ называется пара индексов $(i, j)$, $i, j \in \nset{n}$ такая, что $i < j$, но $\sigma(i) > \sigma(j)$. Числа беспорядков в $\sigma$ обозначается через $N(\sigma)$. \textit{Знаком} перестановки $\sigma \in S_n$ называется величина $(-1)^{N(\sigma)}$. Обозначения "--- $\sgn{\sigma}, (-1)^{\sigma}$.
    \end{definition}
    
    \begin{definition}
    	Перестановка $\sigma \in S_n$ называется:
    	\begin{itemize}
    		\item \textit{Четной}, если $\sgn{\sigma} = 1$
    		\item \textit{Нечетной}, если $\sgn{\sigma} = -1$
    	\end{itemize}
    \end{definition}
    
    \begin{definition}
    	\textit{Гомоморфизмом групп} $G$ и $H$ называется отображение $\phi : G \rightarrow H$ такое, что для любых $a,b \in G$ выполнено равенство $\phi(ab) = \phi(a)\phi(b)$. \textit{Изоморфизмом групп} $G$ и $H$ называется биективный гомоморфизм $\phi : G \rightarrow H$. Пространства $G$ и $H$ называются \textit{изоморфными}, если между ними существует изоморфизм. Обозначение "--- $G \cong H$.
    \end{definition}
    
    \begin{theorem}[Кэли]
    	Пусть $G$ "--- конечная группа, $|G| = n$. Тогда существует подгруппа $H \le S_n$ такая, что $H \cong G$, то есть группа $G$ вкладывается в группу $S_n$.
    \end{theorem}
    
    \begin{proof}
    	Рассмотрим группу $S(G)$ перестановок множества $G$, тогда $S(G) \cong S_n$, поскольку имеет место биекция между $G$ и $\{1, \dots, n\}$. Найдем требуемую подгруппу в $S(G)$. Для каждого элемента $a \in G$ определим перестановку $\sigma_a \in S(G)$ такую, что для любого $b \in G$ выполнено $\sigma_a(b) := ab$. Положим $H := \{\sigma_a \in S(G): a \in G\}$. Проверим, что $H \le S(G)$:
    	\begin{itemize}
    		\item $H \ne \emptyset$, поскольку $\sigma_e = \id \in H$
    		\item $\forall a, b \in G: \sigma_a \circ \sigma_b = \sigma_{ab} \in H$
    		\item $\forall a \in G: (\sigma_a)^{-1} = \sigma_{a^{-1}} \in H$
    	\end{itemize}
    	
    	Определим отображение $\phi: G \rightarrow H$ для каждого $a \in G$ как $\phi(a) := \sigma_a$. Очевидно, это гомоморфизм, причем сюръективный. Он также инъективен, поскольку для различных $a, b \in G$ выполнено $\sigma_a(e) \ne \sigma_b(e)$. Таким образом, $G \cong H \le S(G) \cong S_n$.
    \end{proof}
    
    \begin{definition}
    	\textit{Порядком} элемента $a$ называется наименьшее $n \in \mathbb{N}$ такое, что $a^n = e$. Если такого $n$ не существует, то порядок считается равным $\infty$. Обозначение "--- $\ord{a}$.
    \end{definition}

    \begin{proposition}
    	Пусть $G$ "--- группа, $a \in G$. Тогда $\ord{a} = |\langle a\rangle|$.
    \end{proposition}
    
    \begin{proof}
    	Если $\ord{a} = n \in \mathbb{N}$, то $\langle a\rangle = \{a^k: k \in \mathbb{Z}\} = \{e, a, \dots, a^{n-1}\}$, поэтому $|\langle a\rangle| \le n$. Кроме того, все элементы различны $e, a, \dots, a^{n-1}$. Действительно, если для некоторых $r, s \in \{1, \dots, n-1\}$, $r < s$ выполнено $a^r = a^s$, то $a^{s - r} = e$, откуда $s - r = 0$ в силу минимальности порядка $n$. Значит, $|\langle a\rangle| = n$. Если же $\ord{a} = \infty$, то для любых $\forall r, s \in \mathbb{Z}$, $r < s$, выполнено $a^r \ne a^s$ из аналогичных соображений, тогда $|\langle a\rangle| = \infty$.
    \end{proof}
    
    \begin{definition}
    	Пусть $G$ "--- группа, $X \subset G$. \textit{Подгруппой, порожденной множеством} $X$, называется следующая подгруппа:
    	\[\langle X\rangle := \bigcap_{H \le G, X \subset H}H\]
    \end{definition}
    
    \begin{note}
    	$\langle X\rangle$ "--- наименьшая по включению подгруппа в $G$, содержащая множество $X$.
    \end{note}
    
    \begin{definition}
    	Группа $G$ называется \textit{циклической}, если существует элемент $\exists a \in G$ такой, что $\langle a\rangle = G$.
    \end{definition}
    
    \begin{example}
    	Рассмотрим несколько примеров циклических групп:
    	\begin{itemize}
    		\item $\mathbb{Z} = \langle 1 \rangle$
    		\item $\mathbb{Z}_n = \langle \overline{1} \rangle$
    	\end{itemize}
    \end{example}
    
    \begin{theorem}
    	Любые две циклических группы одного порядка изоморфны.
    \end{theorem}
    
    \begin{proof}
    	Пусть $G$ "--- циклическая группа, $a \in G$, $G = \langle a\rangle$.
    	
    	\begin{itemize}
    		\item Пусть $|G| = \infty$. Докажем, что тогда $G \cong \mathbb{Z}$. Рассмотрим отображение $\phi: \mathbb{Z} \rightarrow G$, для каждого $k \in \Z$ имеющее вид $\phi(k) := a^k$. Очевидно, это гомоморфизм, причем сюръективный. Докажем его инъективность. Пусть для некоторых $k, l \in \Z$ выполнено равенство $a^k = a^l$, тогда $a^{k - l} = e$, что возможно только при $k = l$. Таким образом, получен изоморфизм между $\Z$ и $G$.
    		
    		\item Пусть $|G| = n \in \mathbb{N}$. Докажем, что тогда $G \cong \mathbb{Z}_n$. Рассмотрим отображение $\phi: \mathbb{Z}_n \rightarrow G$, для каждого $\overline{k} \in \Z_n$ имеющее вид $\phi(\overline{k}) := a^k$. Отображение $\phi$ определено корректно, поскольку если $a^k = a^l$ для некоторых $k, l \in \Z$, то $a^{k - l} = e$, откуда $n \mid (k - l)$ и $\overline{k} = \overline{l}$. Очевидно тогда, что это гомоморфизм, причем инъективный в силу уже доказанного и сюръективный.\qedhere
    	\end{itemize}
    \end{proof}
    
    \begin{theorem}[Лагранжа]
    	Пусть $G$ "--- конечная группа, $H \le G$. Тогда выполнены следующие равенства:
    	\[|G| = |H||G / H| \hm= |H||H \bs G|\]
    \end{theorem}
    
    \begin{proof}
    	Если смежные классы в $G$ пересекаются хотя бы по одному элементу, то они совпадают. Тогда, поскольку для любого $a \in G$ выполнено $a \hm\in aH$, вся группа $G$ разбивается на непересекающиеся смежные классы порядка $|H|$, откуда и следует требуемое равенство.
    \end{proof}
    
    \begin{corollary}
    	Пусть $G$ "--- конечная группа, $a \in G$. Тогда:
    	\begin{enumerate}
    		\item $\ord{a} \Mid |G|$
    		\item $a^{|G|} = e$
    	\end{enumerate}
    \end{corollary}
    
    \begin{proof}~
    	\begin{enumerate}
    		\item По теореме Лагранжа, $\ord{a} = |\langle a\rangle| \Mid |G|$
    		\item Пусть $\ord{a} = k$, тогда $k \Mid |G|$ в силу пункта $(1)$, откуда $a^{|G|} = e$\qedhere
    	\end{enumerate}
    \end{proof}
    
    \begin{corollary}[малая теорема Ферма]
    	Пусть $p$ "--- простое число, $a \in \Z$, $p \nmid a$. Тогда $a^{p-1} \equiv_p 1$.
    \end{corollary}
    
    \begin{proof}
    	Рассмотрим группу $(\mathbb{Z}_p\backslash\{\overline{0}\}, \cdot)$, $|\mathbb{Z}_p\backslash\{\overline{0}\}| = p - 1$, и применим пункт $(2)$ следствия выше. Получим, что $\overline{a}^{p - 1} = \overline{1}$.
    \end{proof}