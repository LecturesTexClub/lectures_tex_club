\subsection{Прямая на плоскости, различные способы задания, их эквивалентность. Формула для расстояния от точки до прямой в прямоугольной системе координат. Условия пересечения и параллельности двух прямых. Пучок прямых.}
    
    \begin{definition}
    	\textit{Направляющим вектором} прямой $l \subset P_3$ называется вектор $\overline{a} \in V_3$, $\overline{a} \ne \overline 0$, представителем которого является направленный отрезок, лежащий в $l$.
    \end{definition}
    
    \begin{definition}
    	Пусть $l \subset P_2$ "--- прямая с направляющим вектором $\overline{a} \in V_2$, $M \in l$, и в декартовой системе координат $(O, e)$ в $P_2$ выполнены соотношения $\overline{a} \leftrightarrow_{e} (\alpha_1, \alpha_2)^T$, $M \leftrightarrow_{(O, e)} (x_0, y_0)^T$, $\overline{r_0} := \overline{OM}$.
    	\begin{itemize}
    		\item \textit{Векторно-параметрическим уравнением прямой} называется следующее семейство уравнений:
    		\[\overline{r} = \overline{r_0} + t\overline{a},~t \in \mathbb{R}\]
    		\item \textit{Параметрическим уравнением прямой} называется следующее семейство систем:
    		\[\left\{
    		\begin{aligned}
    			x = x_0 + t\alpha_1\\
    			y = y_0 + t\alpha_2
    		\end{aligned}
    		\right.,~t \in \R
    		\]
    		\item \textit{Каноническим уравнением прямой} называется следующее уравнение:
    		\[\frac{x - x_0}{\alpha_1} = \frac{y - y_0}{\alpha_2}\]
    	\end{itemize}
    \end{definition}
    
    \begin{note}
    	Множество точек $X \in P_2$ таких, что $X \leftrightarrow_{(O, e)} (x, y)^T$, $\overline{r} := \overline{OX}$, являющихся решениями любого из уравнений прямой выше, совпадает с прямой $l$. Действительно, $X \in l \lra MX \parallel l \lra \overline{MX} \parallel \overline{a}$.
    \end{note}
    
    \begin{note}
    	В случае канонического уравнения прямой, если без ограничения общности $\alpha_1 = 0$, то тогда $\alpha_2 \ne 0$, и следует считать, что исходное уравнение эквивалентно условию $x = x_0$. Отметим также, что каноническое уравнение прямой эквивалентно следующему такому уравнению:
    	\[\alpha_2 x - \alpha_1 y + (\alpha_1 y_0 - \alpha_2 x_0) = 0\]
    \end{note}
    
    \begin{definition}
    	Пусть $A, B, C \in \R$, $A^2+B^2 \ne 0$. \textit{Общим уравнением прямой} называется следующее уравнение:
    	\[Ax+By+C = 0\]
    \end{definition}
    
    \begin{theorem}
        Пусть $l_{1}: A_{1}x + B_{1}y + C_{1} = 0$, $l_{2}: A_{2}x + B_{2}y + C_{2} = 0$. 
        \begin{enumerate}
            \item $l_1 \times l_2 \lra \begin{vmatrix}
                A_1 & B_1 \\
                A_2 & B_2
            \end{vmatrix} \neq 0;$
            \item $l_1 \parallel l_2 \lra \begin{vmatrix}
                A_1 & B_1 \\
                A_2 & B_2
            \end{vmatrix} = 0;$
            \item $l_1 = l_2 \lra$ их уравнения пропорциональны.
        \end{enumerate}
    \end{theorem}
    
    \begin{proof}
        1. Следует из п.2;\\
        2. Нормальные векторы прямых $l_1$ и $l_2$ имеют координаты $(A_1,B_1)$ и $(A_2, B_2)$ соответственно, а условие параллельности прямых запишется как $\begin{cases}
            A_1 = t A_2\\
            B_1 = t B_2
        \end{cases}$, откуда и следует искомое.\\
        3. Пусть $l_1 = l_2$. Тогда $\begin{cases}
            A_2 = \lambda A_1\\
            B_2 = \lambda B_1
        \end{cases}$. Запишем уравнения прямых $l_1: A_{1}x + B_{1}y + C_1 = 0$, $l_2: \lambda(A_{1}x + B_{1}y) + C_2 = 0$. В частности, для общего решения $(x_{0}, y_{0})$ имеем: $A_{1}x_{0} + B_{1}y_{0} + C_1 = 0$, $\lambda(A_{1}x_{0} + B_{1}y_{0}) + C_2 = 0$. Домножим первое равенство на $(-1)$ и прибавим ко второму, получим $C_2 - \lambda C_1 = 0$, $C_2 = \lambda C_1$.
        Заметим, что первые два пункта являются следствием теоремы Крамера.
    \end{proof}
    
    \begin{definition}
    	\textit{Вектором нормали} прямой $l \subset P_3$ называется вектор $\overline{n} \in V_3$, $\overline{n} \ne \overline 0$, представителем которого является направленный отрезок, ортогональный прямой $l$.
    \end{definition}
    
    \begin{definition}
    	Пусть $l \subset P_2$ "--- прямая с вектором нормали $\overline{n} \in V_2$, и пусть $M \in l$, $\overline{r_0} := \overline{OM}$. \textit{Нормальным уравнением прямой} называется следующее уравнение:
    	\[(\overline{r} - \overline{r_0}, \overline{n}) = 0\]
    \end{definition}
    
    \begin{note}
    	Множество точек $X \in P_2$, $\overline{r} := \overline{OX}$, являющихся решениями нормального уравнения прямой, совпадает с прямой $l$. Кроме того, это уравнение можно переписать в следующем виде при $\gamma := (\overline{r_0}, \overline{n})$:
    	\[(\overline{r}, \overline{n}) = \gamma\]
    \end{note}
    
    \begin{note}
    	Уравнения различного типа, задающие прямую, эквивалентны: из каждого из них можно получить любое другое.
    \end{note}
    
    \begin{definition}
    	\textit{Пучком прямых} называется либо множество всех прямых в $P_2$, проходящих через фиксированную точку $P \in P_2$, либо множество всех прямых, параллельных фиксированной прямой $l \subset P_2$.
    \end{definition}
    
    \begin{note}
    	Любые две прямые в $P_2$ лежат ровно в одном пучке.
    \end{note}
    
    \begin{theorem}
    	Пусть в декартовой системе координат $(O, e)$ в $P_2$ различные прямые $l_1, l_2$ заданы уравнениями $A_1x\hm{+}B_1y+C_1=0$, $A_2x+B_2y+C_2=0$. Тогда прямая $l \subset P_2$ лежит в одном пучке с прямыми $l_1$ и $l_2$ $\lra$ прямая $l$ задается уравнением следующего вида при некоторых $\alpha_1, \alpha_2 \in \R$:
    	\[\alpha_1(A_1x+B_1y+C_1) + \alpha_2(A_2x+B_2y+C_2) = 0\]
    \end{theorem}
    
    \begin{proof}~
    	\begin{itemize}
    		\item[$\la$] Возможны два случая:
    		\begin{enumerate}
    			\item Если $l_1 \cap l_2 = \{P\}$, $P \in P_2$, то координаты точки $P$ удовлетворяют требуемому уравнению, то есть $P \in l$.
    			\item Если $l_1 \parallel l_2$, то из требуемого уравнения направляющий вектор прямой $l$ параллелен направляющим векторам $l_1$ и $l_2$. В этом случае уравнение задает прямую не при всех $\alpha_1, \alpha_2 \in \R$, но если задает, то лежащую в данном пучке.
    		\end{enumerate}
    		
    		\item[$\ra$] Возможны два случая:
    		\begin{enumerate}
    			\item Если $l \cap l_1 \cap l_2 = \{P\}$, $P \in P_2$, то выберем на $l$ точку $Q \ne P$, $Q \leftrightarrow_{(O, e)} (x_0, y_0)^T$. Тогда $Q$ удовлетворяет уравнению с коэффициентами $\alpha_1 := A_2x_0+B_2y_0+C_2$, $\alpha_2 := -(A_1x_0+B_1y_0+C_1)$. Хотя бы один из коэффициентов ненулевой, поскольку $Q$ лежит не более, чем на одной из прямых $l_1$, $l_2$. Значит, такое уравнение задает $l$, так как ему удовлетворяют две различных точки этой прямой.
    			
    			\item Если $l \parallel l_1 \parallel l_2$, то аналогичным образом выберем любую точку $Q \in l$ и соответствующие коэффициенты, тогда полученное уравнение задает $l$ при условии, что оно задает прямую. Но оно всегда задает прямую, поскольку множество его решений непусто и не содержит хотя бы одну из прямых $l_1, l_2$.\qedhere
    		\end{enumerate}
    	\end{itemize}
    \end{proof}
    
    \begin{proposition}
    	Пусть в прямоугольной декартовой системе координат $(O, e)$ в $P_2$ прямая $l$ задана  уравнением $Ax\hm{+}By+C=0$, $M \in P_2$, $M \leftrightarrow_{(O, e)} (x_0, y_0)^T$. Тогда расстояние $\rho$ от точки $M$ до прямой $l$ равно следующей величине:
    	\[\rho = \frac{|Ax_0 + By_0 + C|}{\sqrt{A^2 + B^2}}\]
    \end{proposition}
    
    \begin{proof}
    	Пусть $\overline{n} \in V_2$, $\overline n \leftrightarrow_{e} (A, B)^T$ "--- вектор нормали прямой $l$, $\overline{r_0} := \overline{OM}$, и пусть $X \in l$, $\overline{r} := \overline{OX}$. Тогда:
    	\[\rho = |\pr_{\overline{n}}(\overline{r_0} - \overline{r})|
    	=
    	\left|\frac{(\overline{r_0} - \overline{r}, \overline{n})}{|\overline{n}|^2}\overline{n}\right|
    	=
    	\frac{|(\overline{r_0} - \overline{r}, \overline{n})|}{|\overline{n}|}
    	=
    	\frac{|Ax_0 + By_0 + C|}{\sqrt{A^2 + B^2}}\qedhere\]
    \end{proof}