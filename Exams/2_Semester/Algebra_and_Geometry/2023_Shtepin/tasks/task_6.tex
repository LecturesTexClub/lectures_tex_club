\section{6. Линейная независимость собственных векторов, имеющих попарно различные собственные значения. Алгебраическая и геометрическая кратности собственного значения. Условия диагонализируемости линейного оператора.}

\begin{definition}
    Пусть $\lambda$ -- собственное значение оператора $\phi: V \to V$. Собственным подпространством 
    оператора $\phi$, отвечающим $\lambda$ называется подпространство 
    $V_{\lambda} = \ker (\phi - \lambda \epsilon) \leq V$.
\end{definition}

\begin{definition}
    Подпространства $U_1, U_2, \ldots, U_n$ называются линейно независимыми, если из равенства 
    $x_1 + x_2 + \dots + x_n = \bar{0}$, где $x_i \in U_i$, следует, что $x_1 = x_2 = \dots = x_n = 0$.
\end{definition}

\begin{theorem}[О линейной независимости собственных подпространств, отвечающих попарно различным собственным значениям]
    \label{o_lnz}
    Пусть $\phi: V \to V$, $\lambda_1, \lambda_2, \dots, \lambda_n$ -- различные собственные значения. 
    Тогда $V_{\lambda_1}, \ldots, V_{\lambda_n}$ линейно независимы.
\end{theorem}

\begin{proof}
    От противного. Пусть $\exists x_1 \in V_{\lambda_1}, x_2 \in V_{\lambda_2}, \dots, 
    x_n \in V_{\lambda_n}$, не все равные нулю. 
    Назовем эти наборы опровергающими, а его мощностью количество ненулевых векторов. 
    Из всех таких наборов векторов выберем набор наименьшей мощности. Пусть 
    указанный нами набор без ограничения общности -- искомый, перенумеруем множества и $x_i$ так,
    чтобы ненулевыми были первые $j$ векторов. Тогда $x_1 + x_2 + \dots + x_j = 0$, все
    $x_i \neq 0$, иначе есть набор меньшей мощности. Применим к сумме оператор $\phi$ и получим $\lambda_1 x_1 + \lambda_2 x_2 + \ldots + \lambda_j x_j$. Умножим изначальную сумму на $-\lambda_1$
    и сложим с получившейся. Тогда мы получили набор меньшей мощности, противоречие.
\end{proof}

\begin{definition}
    Линейный оператор $\phi: V \to V$ над полем $F$ называется диагонализируемым, если в $V$ 
    существует базис $e$ такой, что $A_{\phi}$ -- диагональная матрица.
\end{definition}

\begin{theorem}[критерий диагонализируемости линейного оператора]
    \label{theorem4.1}
    Пусть $\phi: V \to V$, $V$ над $F$. Пусть $\lambda_1, \lambda_2, \dots, \lambda_k$ -- все 
    попарно различные собственные значения, тогда следующие условия эквивалентны:
    \begin{enumerate}
        \item $\phi$ -- диагонализируем.
        \item В $V$ существует базис, состоящий из собственных векторов оператора $\phi$.
        \item $V = V_{\lambda_1} \oplus V_{\lambda_2} \oplus \dots \oplus V_{\lambda_k}$.
    \end{enumerate}
\end{theorem}

\begin{proof}~
    \begin{enumerate}
        \item $1 \Rightarrow 2$ \\
        Так как $\phi$ диагонализируем, то существует базис, в котором матрица оператора выглядит 
        следующим образом:
        \begin{equation*}
        \left(
            \begin{array}{cccc}
            \lambda_{1} & 0 & \ldots & 0\\
            0 & \lambda_{2} & \ldots & 0\\
            \vdots & \vdots & \ddots & \vdots\\
            0 & 0 & \ldots & \lambda_n
            \end{array}
        \right)
        \end{equation*}
        Значит, $\phi(e_i) = \lambda_i e_i$ для любого $i$, откуда $e_1, \dots, e_n$ -- собственные 
        векторы для $\phi$. Значит, $e$ -- базис из собственных векторов.
        \item $2 \Rightarrow 3$ \\
        Пусть $e$ -- базис из собственных векторов оператора $\phi$. Перегруппируем базисные векторы 
        по собственным значениям: 
        $$\underbrace{(e_{11}, \dots, e_{1s_1})}_{\lambda_1} \underbrace{(e_{21}, \dots, e_{2s_2})}_{\lambda_2} 
        \dots \underbrace{(e_{k1}, \dots, e_{ks_k})}_{\lambda_k}$$
        $\langle e_{11}, \dots, e_{1s_1} \rangle \subseteq V_{\lambda_1}$, 
        $\dots, \langle e_{k1}, \dots, e_{ks_k} \rangle \subseteq V_{\lambda_k}$. 
        Откуда $V = V_{\lambda_1} + \dots + V_{\lambda_k}$, но так как собственные подпространства 
        линейно независимы (по \ref{o_lnz}), то по теореме о характеризации прямой суммы 
        $V = V_{\lambda_1} \oplus V_{\lambda_2} \oplus \dots \oplus V_{\lambda_k}$.
        \item $3 \Rightarrow 1$ \\
        Известно, что $V = V_{\lambda_1} \oplus V_{\lambda_2} \oplus \dots \oplus V_{\lambda_k}$. 
        Выберем в каждом $V_{\lambda_i}$ базис: $e_{i_1}, \dots, e_{is_i}$. Тогда, объединяя базисы 
        собственных подпространств, получим базис всего пространства $V$. 
        При этом по диагонали будут стоять сначала $s_1$ значений $\lambda_1$, 
        затем $s_2$ значений $\lambda_2$ и так далее. Остальные значения -- нули. 
        Значит, $\phi$ -- диагонализируем.
        \begin{equation*}
        \left(
            \begin{array}{ccccc}
            \lambda_{1} & 0 & \ldots & 0 & 0\\
            \vdots & \vdots & \ddots & \vdots & \vdots\\
            0 & \ldots & \lambda_{1} & \ldots & 0\\
            \vdots & \vdots & \ddots & \vdots & \vdots\\
            0 & 0 & \ldots & \ldots & \lambda_n
            \end{array}
        \right)
        \end{equation*}
    \end{enumerate}
\end{proof}

\begin{definition}
    Пусть $\phi: V \to V$, $\lambda \in F$ -- его собственное значение, $\chi_{\phi}(\lambda) = 0$.
    Кратность корня $\lambda$ как корня характеристического многочлена называется алгебраической 
    кратностью собственного значения $\lambda$. Обозначение: $alg(\lambda) \geq 1$.
\end{definition}

\begin{definition}
    Размерность собственного подпространства $V_{\lambda}$ называется геометрической кратностью 
    собственного значения $\lambda$. Обозначение: $geom(\lambda) = \dim V_{\lambda} \geq 1$.
\end{definition}

\begin{proposition}
    \label{pr4.1}
    Пусть $\phi: V \to V$ и $U$ -- инвариантное подпространство относительно $\phi$. 
    Пусть $\psi = \phi \mid_{U}$. Тогда $\chi_{\phi} \vdots \chi_{\psi}$.
\end{proposition}

\begin{proof}
    Пусть $e$ -- базис в $V$, согласованный с инвариантным подпространством $U$: 
    $e = (\underbrace{e_{1}, \dots, e_{k}}_{U}, e_{k + 1}, \dots, e_n)$
    \[A_{\phi} = \left(\begin{array}{@{}c|c@{}}
		B & C\\
		\hline
		0 & D
    \end{array}\right)\]
    \[\chi_{\phi}(\lambda) = \det \left(\begin{array}{@{}c|c@{}}
		B - \lambda E & C\\
		\hline
		0 & D - \lambda E
    \end{array}\right) = |B - \lambda E| |D - \lambda E| = \chi_{\psi} \chi_D.\]
\end{proof}

\begin{corollary}
    Для любого собственного значения $\lambda$: $geom(\lambda) \leq alg(\lambda)$.
\end{corollary}

\begin{proof}
    $U = V_{\lambda}$ -- инвариантно относительно оператора $\phi$.
    \begin{equation*}
    \psi =
        \left(
            \begin{array}{ccc}
            \lambda & \dots & 0 \\
            \vdots & \ddots & \vdots \\
            0 & \dots & \lambda \\
            \end{array}
        \right)
    \end{equation*}
    $\chi_{\psi} = (\lambda - t)^k$, где $k = \dim V_{\lambda} = geom(\lambda)$. 
    По утверждению \ref{pr4.1} $\chi_{\phi} \vdots \chi_{\psi}$, откуда следует, что 
    $\chi_{\phi} \vdots (\lambda - t)^k$. Значит, $alg(\lambda) \geq geom(\lambda)$.
\end{proof}

\begin{theorem}[критерий диагонализируемости в терминах алгебраической и геометрической кратностей линейного оператора]
    Пусть $\phi: V \to V$, $\dim V = n$. $\phi$ -- диагонализируем тогда и только тогда, когда 
    \begin{enumerate}
        \item $\chi_{\phi}(t)$ разлагается на линейные множители над $F$. (Далее: $\phi$ линейно факторизуем над полем $F$)
        \item Для любого собственного значения $\lambda$ оператора $\phi$ выполнено $alg(\lambda) = geom(\lambda).$
    \end{enumerate}
\end{theorem}

\begin{proof}~
    \begin{enumerate}
        \item $(\Rightarrow)$ Пусть $\phi$ -- диагонализируемый, тогда существует базис, в котором матрица оператора $\phi$ 
        имеет диагональный вид и по теореме \ref{theorem4.1} 
        $V = V_{\lambda_1} \oplus V_{\lambda_2} \oplus \dots \oplus V_{\lambda_k}$. 
        Тогда по свойству прямой суммы: 
        $$\sum_{i=1}^k geom(\lambda_i) = \sum_{i=1}^k \dim V_{\lambda_i} = 
        \dim V = n = \deg \chi \geq \sum_{i=1}^k alg(\lambda_i)$$
        С одной стороны, выполнено неравенство выше, но, с другой стороны, по предыдущему следствию 
        $geom(\lambda) \leq alg(\lambda)$, откуда верно, что $\forall i: alg(\lambda_i) = geom(\lambda_i)$.
        \item $(\Leftarrow)$ Пусть $\chi$ -- линейно факторизуем над $F$ и $alg(\lambda_i) = geom(\lambda_i)$. 
        Докажем диагонализируемость оператора $\phi$.
        $$\dim(V_{\lambda_1} \oplus V_{\lambda_2} \oplus \dots \oplus V_{\lambda_k}) = 
        \sum_{i=1}^k \dim V_{\lambda_i} = \sum_{i=1}^k geom(\lambda_i) = \sum_{i=1}^k alg(\lambda_i) = n$$
        Последнее равенство следует из линейной факторизуемости $\chi$. 
        Отсюда получаем, что $V = V_{\lambda_1} \oplus V_{\lambda_2} \oplus \dots \oplus V_{\lambda_k}$ -- 
        эквивалентное определение диагонализируемости.
    \end{enumerate}
\end{proof}
