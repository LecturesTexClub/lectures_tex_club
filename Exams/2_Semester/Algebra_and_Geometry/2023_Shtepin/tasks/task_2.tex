\section{2. Неприводимость многочленов. Основная теорема арифметики для многочленов.}

\begin{definition}
    Многочлены $A$ и $B$ называются \textit{ассоциироваными} если $B \vert A$ и $A \vert B$, то есть когда верны 
    представления $A = Q_1 B$, $B = Q_2 A$. При этом:
    \begin{eqnarray*}
        \deg A = \deg Q_1 + \deg B \geq \deg B, \\
        \deg B = \deg Q_2 + \deg A \geq \deg A,
    \end{eqnarray*}
    откуда $\deg A = \deg B$, $\deg Q_1 = \deg Q_2 = 0$.
\end{definition}

\begin{definition}
    Многочлен $P \in F[x]$ степени больше нуля называется \textit{неприводимым} над полем $F$, если из $P = AB$ 
    следует $\deg A = 0$ или $\deg B = 0$.

    Иначе говоря многочлен называется неприводимым над полем F, если его нельзя разложить в 
    произведение двух многочленов более низких степеней из этого же кольца $F[x]$.
\end{definition}

\begin{note}
    Важно над каким полем многочлен является неприводимым, например многочлен $x^2 + 1$ является 
    приводимым над полем комплексных чисел $\mathbb{C}$, но неприводимым над полем действительных 
    чисел $\mathbb{R}$.
\end{note}

\begin{proposition}
	Пусть $F$ "--- поле, $P, Q, R \in F[x]$, многочлен $P$ неприводим и выполнено $P\mid QR$. Тогда $P\mid Q$ или $P\mid R$.
\end{proposition}

\begin{proof}
	Предположим, что $P\nmid Q$. Тогда, в силу неприводимости многочлена $P$, выполнено равенство $\gd(P, Q) = 1$, поэтому существуют многочлены $K, L \in F[x]$ такие, что $KP + LQ = 1$. Умножая обе части равенства на $R$, получим, что $KPR + LQR = R$, откуда $P \mid KPR + LQR = R$.
\end{proof}

\begin{theorem} [основная теорема арифметики для многочлена]
    Пусть $F$ -- поле, $A \in F[x], A \neq 0$. Тогда верны следующие утверждения:
    \begin{enumerate}
        \item Существует разложение $A$ на неприводимые: $$A = \alpha P_1P_2 \dots P_n,$$ 
        где $\alpha \in F^*$, $P_i$ неприводимый над $F$ многочлен.
        \item Пусть A представляется в виде неприводимых многочленов двумя различными способами: 
        $$A = \alpha \cdot P_1P_2 \dots P_n = \beta \cdot Q_1Q_2 \dots Q_m,$$ 
        где $\beta\in F^*$, $Q_j$ -- неприводимый над $F$ многочлен. 

        Тогда $n = m$ и существует перестановка $\sigma\in S_n$ такая, что многочлены ассоциированы: 
        \begin{gather*}
            P_i \thicksim Q_j \, (1 \leq i \leq n), \text{ где } j = \sigma(i).
        \end{gather*}
    \end{enumerate}
\end{theorem}

\begin{proof}~
    \begin{enumerate}
        \item Докажем существование разложения на неприводимые множители индукцией по $\deg A$.
        Если $deg A = 0$, то $A = \alpha$, $\alpha \in F^*$.
        Пусть $A$ неприводим над $F$, $\deg A \geq 1$. Будем считать, что в таком случае разложение получено ($A = P$).
        Пусть теперь $A$ приводим над $F$, тогда его можно представить в виде $A = B \cdot C$, $B, C \in F[x]$, $\deg B < \deg A$, $\deg C < \deg A$. Тогда  к $B$ и $C$ применимо предположение индукции и, перемножая их, получим разложение для $A$.
        \item (Индукция по n)
        Пусть $A = \alpha P_1P_2 \dots P_n = \beta Q_1Q_2...Q_m$. Тогда произведение $Q_1Q_2 \dots Q_m \vdots P_n$, следовательно, $\exists j: Q_j \vdots P_n$ и $\exists \gamma \in F^*: Q_j = \gamma P_n$ ($P_n$ и $Q_j$ неприводимы). Теперь можно подставить выражение для $Q_j$ в представление для $A$ справа и сокрaтить $P_n$ в обеих частях (корректно, так как кольцо многочленов является областью целостности). По предположению индукции число множителей слева и справа после сокращения совпадает и существует биекция $\sigma$ между множествами $\{ 1, \dots , n - 1\} \longrightarrow \{ 1, \dots , j - 1, j + 1, \dots , n \}$, так что $P_i \thicksim Q_l$, где $l = \sigma(i)$. Доопределим биекцию $\sigma: \sigma(n) = j, \sigma \in S_n$. Теперь $\sigma$ удовлетворяет всем условиям, поэтому доказано для $n$.
    \end{enumerate}
\end{proof}
