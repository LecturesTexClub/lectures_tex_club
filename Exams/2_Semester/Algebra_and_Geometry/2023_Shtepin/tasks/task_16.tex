\section{16. Индексы инерции квадратичной формы в действительном линейном пространстве. Закон инерции. Метод Якоби приведения квадратичной формы к диагональному виду.}

\begin{definition}
    Квадратичная функция $q(x)$ называется \textit{положительно определенной (отрицательно опеределенной)}, 
    если для всех $x \neq 0$ верно $q(x) > 0$ ($q(x) < 0$).
\end{definition}

\begin{definition}
    Квадратичная функция $q(x)$ называется \textit{положительно полуопределенной (отрицательно полуоопределенной)} 
    если для всех $x \in V$ верно $q(x) \geq 0$ ($q(x) \leq 0$).
\end{definition}

\begin{agreement}
    До конца раздела будем считать что $V$ -- поле над пространством действительных чисел.
\end{agreement}

\begin{definition}
    Пусть $e$ -- канонический базис. Представим $q(x)$ как: 
    $$q(x) = \xi_1^2 + \, \dots \,+ \xi_p^2 - \xi^2_{p+1} - \, \dots \,- \xi^2_{p+q}.$$ 
    Числа $p$ и $q$ называются \textit{индексами инерции} относительно канонического базиса $e$.
\end{definition}

\begin{theorem}[Закон инерции]
    Пусть $q \in Q(V)$, $e$ -- канонический базис в $V$, $p$ и $q$ -- положительный и отрицательный 
    индексы инерции относительно базиса $e$. Тогда верно следующее: 
    \begin{enumerate}
        \item $p = \max\{\dim U \, \vert \, U \leq V : q \vert_{U} \text{ -- положительно определена}\}$,
        \item $q = \max\{\dim U \, \vert \, U \leq V : q \vert_{U} \text{ -- отрицательно определена}\}$,
        \item Индексы $p$ и $q$ не зависят от выбора базиса в $V$.
    \end{enumerate}
\end{theorem}

\begin{proof}~
    \begin{enumerate}
        \item Пусть $e = (e_1, e_2, \, \dots \,e_n)$. Рассмотрим следующие подпространства V:
        \begin{align*}
            U_0 = \langle e_1, e_2, \, \dots \,e_p \rangle && W_0 = \langle e_{p+1}, e_{p+2}, \, \dots \,e_n\rangle. 
        \end{align*}
        Их размерности равны $\dim U_0 = p$ и $\dim W_0 = n-p$ соответственно.

        Пусть $m = \max \{ \dim U \vert U \leq V: q\vert_U \text{ -- положительно определена}\}$.

        По построению $U_0$ верно что $q \vert_{U_0}$ положительно определена, а значит $m \geq p$. Пусть $m > p$.
        Тогда по построению $m$ существует $U_1  \leq V$ такое, что $q \vert_{U_1}$ положительно определена
        и $\dim U_1 = m$.
        При этом по формуле Грассмана: 
        $$\dim (U_1 \cap W_0) = \dim U_1 + \dim W_0  - \dim (U_1 + W_0) = m + n - p - \dim (U_1 + W_0) \geq 
        m + n - p - n > 0.$$
        Тогда $\exists z \in U_1 \cap W_0$. Однако по построению этих подпространств получим:
        \begin{align*}
            z \in U_1 \, & \Rightarrow q(z) > 0, \\
            z \in W_0 & \Rightarrow q(z) \leq 0. \\
        \end{align*}
        Таким образом предположение $m > p$ приводит к противоречию из-за невозможности существования 
        нетривиального пересечения $U_1$ и $W_0$. Это значит, что $m = p$. 

        \item Доказательство аналогично первому пункту.
        \item Истинность утверждения вытекает из первых двух пунктов, так как размерность подпространств 
        не зависит от выбора базисов в них.
    \end{enumerate}
\end{proof}

\begin{definition}
    Пусть квадратичная билинейная форма $q$ представляется как: \begin{gather*}
        q \leftrightarrow \begin{pmatrix}
        a_{11} & a_{12} & \dots  & a_{1n}   \\
        a_{21} & a_{22} & \dots  & a_{2n}   \\
        \vdots & \vdots & \ddots & \vdots   \\
        a_{n1} & a_{n2} & \dots  & a_{nn}
        \end{pmatrix}
    \end{gather*}
    \textit{Главным минором} $\Delta_i$ называется определитель левой верхней подматрицы размера $i \times i$:
    \begin{gather*}
        \Delta_i = \begin{pmatrix}
            a_{11} & a_{12} & \dots  & a_{1i}   \\
            a_{21} & a_{22} & \dots  & a_{2i}   \\
            \vdots & \vdots & \ddots & \vdots   \\
            a_{i1} & a_{i2} & \dots  & a_{ii}
        \end{pmatrix}
    \end{gather*} 
\end{definition}

\begin{theorem}[Як\'{о}би]
    Пусть $q(x)$ -- квадратичная функция в линейном пространстве над $\R$, $A$ -- её матрица относительно 
    некоторого базиса $e$ в $V$ и пусть $\forall i \: = 1, \dots n$ верно $\Delta_i \neq 0$. Тогда 
    существует базис $e'$ в $V$ такой что в нем $q(x)$ принимает вид: 
    \begin{gather*}
        q(x) = \frac{\Delta_0}{\Delta_1} \xi_1^2 + \frac{\Delta_1}{\Delta_2} \xi_2^2 + \, \dots \,+ 
        \frac{\Delta_{n-1}}{\Delta_n} \xi_n^2, \: \text{где } \, \Delta_0 = 1. 
    \end{gather*}
    Более того, $e'$ можно выбрать так, что матрица перехода $S = S_{e \to e'}$ 
    является верхнетреугольной. 
\end{theorem}

\begin{proof}
    Индукция по $n$ -- размерности пространства $V$:
    \begin{enumerate}
        \item База $n = 1$:  
        
        В пространстве размерности $1$ форма принимает вид $q(x) = a_{11} x_1^2$. 

        Тогда можно осуществить переход $e_1 \to e'_1 = \frac{1}{a_{11}} e_1$. Для нового базисного вектора: 
        $$q(e'_1) = f(\frac{e_1}{a_{11}}, \frac{e_1}{a_{11}}) = \frac{1}{a_{11}^2} a_{11} = \frac{1}{a_{11}}.$$
        Тогда в новом базисе $q(x) = a_{11} \xi_1 = \frac{1}{\Delta_1} \xi_1^2$, что и требовалось.

        \item Пусть теорема справедлива для любого $V$ для которого верно $\dim V < n$. 
        
        Рассмотрим пространство $V$ размерности $n$, 
        и его подпространство $U = \langle e_1, e_2, \, \dots \,e_{n-1} \rangle$. 

        По предположению индукции существует базис $e' = \langle e'_1, e'_2, \dots e'_{n-1} \rangle$ в $U$
        такой что $q$ имеет вид:
        $$q(x)\vert_U = \frac{\Delta_0}{\Delta_1} \xi_1^2 + \frac{\Delta_1}{\Delta_2} \xi_2^2 + \dots + 
        \frac{\Delta_{n-2}}{\Delta_{n-1}} \xi_{n-1}^2,$$ и матрица перехода от него к нашему базису 
        имеет верхнетреугольный вид:
        \begin{gather*}
            S_{e \to e'} = \begin{pmatrix}
            S_{11} & S_{12} & \dots  & S_{1, n-1} \\
            0      & S_{22} & \dots  & S_{2, n-1} \\
            \vdots & \vdots & \ddots & \vdots     \\
            0      & 0      & \dots  & S_{n-1, n-1}
            \end{pmatrix}
        \end{gather*}
        При этом форма $q(x)\vert_{U}$ невырождена, так как $\Delta_{n-1} \neq 0$. 

        Тогда по теореме о невырожденном подпространстве $V = U \oplus U^{\perp}$, 
        где ортогональное дополнение $U^{\perp}$ используется в смысле $f$ ассоциированного с $q$, 
        $\dim U^{\perp} = 1$. 
        
        Заметим, что в $U^{\perp}$ есть ненулевой вектор $e$, для которого верно $f(e_n, e) \neq 0$. 
        
        В противном случае для любого вектора $e \in U^{\perp}$ верно $f(e_n, e) = 0$, что значит, 
        что все вектора $e \in U^{\perp}$ перпендикулярны $e_n$. При этом $e \perp U = \langle e_1, \dots e_n\rangle$, 
        откуда $e \in \ker f$. 
        
        Это противоречит тому, что $\dim (\ker f) = \dim V - \rk f = 0$, а значит необходимый нам вектор существует.
        
        Положим $f(e, e_n) = c \neq 0$, 
        тогда $f(e_n, \frac{e}{c}) = 1$. Пусть $e'_n = \frac{e}{c} \in U^{\perp}$, откуда $f(e_n, e'_n)  = 1.$

        Покажем, что $e' = \langle e'_1, \dots e'_{n-1}, e'_n\rangle$ -- искомый базис. 

        Рассмотрим матрицу перехода $S = S_{e \to e'}$. Заметим, что $S_{ni} = 0$ для всех $i < n$ 
        в силу того, что $e'_i \in U$, а значит при переходе к новому базису вектор $e_n$ не повлияет 
        на него. Таким образом матрица $S_{e \to e'}$ диагональна.

        Осталось показать, что в новом базисе форма $q$ имеет необходимый вид. Благодаря предположению 
        индукции мы имеем:
        $$q(x)\vert_U = \frac{\Delta_0}{\Delta_1} \xi_1^2 + \frac{\Delta_1}{\Delta_2} \xi_2^2 + \dots + 
        \frac{\Delta_{n-2}}{\Delta_{n-1}} \xi_{n-1}^2.$$ 

        Таким образом необходимо только показать, что коэффициент при $\xi_n$ равен 
        $\frac{\Delta_{n-1}}{\Delta_n}$. Заметим, что этот коэффициент равен $q(e'_n)$.

        Вектор $e'_n$ выражается через коэффициенты матрицы перехода и векторы начального базиса: 
        $$e'_n = S_{1n}e_1 + \dots S_{nn}e_n.$$ Тогда:
        \begin{equation*}
            \begin{cases}
                f(e_1, e'_n) = 0,         \\
                f(e_2, e'_n) = 0,         \\
                \dots                     \\
                f(e_{n-1}, e'_n) = 0,     \\
                f(e_n, e'_n) = 1.
            \end{cases}
        \end{equation*}

        Первые $n-1$ значений равны $0$ в силу того, что $e'_n \in U^{\perp}$, $e_i \in U$.

        Тогда $q(e'_n)$ можно выразить следующим образом: \begin{gather*}
            q(e'_n) = f(e'_n, e'_n) = f(S_{1n} e_1 + \, \dots \, + S_{n-1, n} e_{n-1} + S_{nn}e_n, e_n') = \\
            = S_{1n} \cdot f(e_1, e'_1) + \, \dots \, + S_{nn} \cdot f(e_n, e'_n) = S_{nn}. 
        \end{gather*}  
        
        Выразим $S_{nn}$ из системы выше:

        \begin{equation*}
            \begin{cases}
                f(e_1, e'_n) = f(e_1, S_{1n}e_1 + \dots S_{nn}e_n) = 0,         \\
                f(e_2, e'_n) = f(e_2, S_{1n}e_1 + \dots S_{nn}e_n) = 0,         \\
                \dots                                                           \\
                f(e_{n-1}, e'_n) = f(e_{n-1}, S_{1n}e_1 + \dots S_{nn}e_n) = 0, \\
                f(e_n, e'_n) = f(e_1, S_{1n}e_1 + \dots S_{nn}e_n) = 1.
            \end{cases}
        \end{equation*}

        В силу невырожденности $q$ матрица перехода невырождена, а значит и система уравнений невырождена,
        так как её матрица в точности является матрицей оператора $q$ в базисе $e$.
        
        Тогда по теореме Крамера для неё существует единственное решение и $S_{nn} = \frac{\Delta_{n-1}}{\Delta_n}$.

        Таким образом мы получили диагональную матрицу $S_{e \to e'}$ и необходимое нам представление 
        $q$ в базисе $e'$ для пространства размерности $V$, что завершает доказательство по индукции.
    \end{enumerate} 
\end{proof}
