\begin{enumerate}
    \item Кольцо многочленов над полем. Наибольший общий делитель. Алгоритм Евклида. Линейное выражение НОД.
    \item Неприводимость многочленов. Основная теорема арифметики для многочленов.
    \item Корни многочленов. Теорема Безу. Формальная производная. Кратные корни.
    \item Лемма Даламбера. Основная теорема алгебры (схема доказательства).
    \item Инвариантные подпространства. Собственные векторы и собственные значения. Характеристический многочлен и его свойства. Инвариантность следа и определителя матрицы оператора.
    \item Линейная независимость собственных векторов, имеющих попарно различные собственные значения. Алгебраическая и геометрическая кратности собственного значения. Условия диагонализируемости линейного оператора.
    \item Приведение матрицы преобразования к треугольному виду. Теорема Гамильтона-Кэли (случай, когда характеристический многочлен линейного оператора раскладывается на линейные множители).
    \item Корневое подпространство линейного оператора. Свойства корневых подпространств. Разложение пространства в прямую сумму корневых подпространств (случай, когда характеристический многочлен линейного оператора раскладывается на линейные множители).
    \item Циклические подпространства. Теорема о нильпотентном операторе. Жорданова нормальная форма и жорданов базис линейного оператора. (Теорема существования жорданова базиса).
    \item Жорданова диаграмма. Метод ее построения без поиска жорданова базиса. Теорема о единственности жордановой нормальной формы с точностью до перестановки клеток.
    \item Аннулирующий и минимальный многочлен линейного оператора. Связь минимального многочлена с жордановой нормальной формой.
    \item Норма в линейном пространстве. Норма линейного оператора. Вычисление многочлена и аналитической функции от линейного оператора.
    \item Линейные рекурренты. Общий вид линейной рекурренты над произвольным полем (случай, когда характеристический многочлен раскладывается на линейные множители).
    \item Билинейные функции. Координатная запись билинейной функции. Матрица билинейной функции и ее изменение при замене базиса. Ортогональное дополнение к подпространству относительно симметричной (кососимметричной) билинейной функции и его свойства.
    \item Симметричные билинейные и квадратичные функции, связь между ними. Поляризационное тождество. Метод Лагранжа приведения квадратичной формы к каноническому виду.
    \item Индексы инерции квадратичной формы в действительном линейном пространстве. Закон инерции. Метод Якоби приведения квадратичной формы к диагональному виду.
    \item Положительно определенные квадратичные функции. Критерий Сильвестра. Кососимметрический билинейные функции, приведение их к каноническому виду.
    \item Полуторалинейные формы в комплексном линейном пространстве. Эрмитовы полуторалинейные и квадратичные формы, связь мужду ними. Приведение их к каноническому виду. Закон инерции для эрмитовых квадратичных форм. Критерий Сильвестра.
    \item Евклидово и эрмитово пространство. Выражение скалярного произведения в координатах. Матрица Грама системы векторов и ее свойства. Неравенства Коши-Буняковского и треугольника.
    \item Ортонормированные базисы и ортогональные (унитарные) матрицы. Существование ортонормированного базиса в пространстве со скалярным произведением. Изоморфизм евклидовых и эрмитовых пространств. Канонический изоморфизм евклидова пространства и сопряженного к нему.
    \item Ортогональное дополнение к подпространству. Задача об ортогональной проекции и ортогональной составляющей. Процедура ортогонализации Грама-Шмидта. Объем параллелепипеда.
    \item Преобразование, сопряженное данному. Существование и единственность такого преобразования, его свойства. Теорема Фредгольма.
    \item Самосопряженное линейное преобразование. Свойства самосопряженных преобразований. Основная теорема о самосопряженных операторах (существование ортонормированного базиса из собственных векторов).
    \item Ортогональные преобразования и их свойства. Канонический вид ортогонального преобразования. Инвариантные подпространства малых размерностей для линейного оператора в действительном линейном пространстве.
    \item Полярное разложение линейного преобразования в евклидовом пространстве. Единственность полярного разложения для невырожденного оператора.
    \item Приведение квадратичной формы в пространстве со скалярным произведением к главным осям. Одновременное приведение пары квадратичных форм к диагональному виду.
    \item Унитарные преобразования, их свойства. Канонический вид унитарного преобразования.
    \item Тензоры типа $(p, q)$. Тензорное произведение тензоров. Координатная запись тензора, изменение координат при замене базиса, тензорный базис.
    \item Алгебраические операции над тензорами (перестановка индексов, свертка). Симметричные и кососимметричные тензоры. Операторы симметрирования и альтернирования и их свойства.
\end{enumerate}